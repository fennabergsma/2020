% !TEX root = thesis.tex

\chapter{Case attraction}


\ex.
\ag. Aer antuurta demo zaimo sprah.\\
he replied\scsub{dat} who.\tsc{dat} {to him} spoke\scsub{nom}\\
`He replied to the one who spoke to him.' Pittner 199
\bg. Der bewîst in d.es er suochte.\\
he showed\scsub{gen} him what.\tsc{gen} he {looked for}\scsub{acc}\\
`He showed him what he was looking for.' Pittner 199
\z.


\section{Headless relative clauses}

\subsection{Proper attraction}

\ex. Old Saxon
\ag. bôtta, them thar blinde uuârun\\
 comfort.\tsc{3.sg.prät}\scsub{dat} that.\tsc{pl.dat} there blind were\scsub{nom}\\
 `He comforted those that were blind' \hfill (Old Saxon, Heliand 2358, \citealt[761]{behaghel1923})
\bg. them mannun, the hêr minniston sindun, [thero] nu undar thesaru menegi [standad]\\
 the men, that here lowest are.\tsc{3pl}\scsub{gen} that.\tsc{pl.gen} now under these crowd stand.\tsc{3.pl.präs}\scsub{nom}\\
 `To the men who are the least of those standing here ’mid the many,' \hfill (Old Saxon, Heliand 4411, \citealt[761]{behaghel1923})
\bg. bethiu he thes uuiht ne bisprak,
thes sie imu thurh inuuidnîð ôgean uueldun.\\
 therefore he that.\tsc{m.sg.gen} something.\tsc{sg.acc} not {speak about}.\tsc{3.sg.prät}
 that.\tsc{m.sg.gen} they ? through hostility show want.\tsc{3.pl.prät}\\
 `therefore he reproached them no whit, for that which they would do unto him in their hatred and anger.' \hfill (Old Saxon, Helian 4923, \citealt[761]{behaghel1923})
\bg. huat, thu mahtis man uuesan, giungaro fan Galilea, thes the thar genouuer stêd\\
 what.\tsc{acc} \tsc{2.sg} may.\tsc{2.sg.prät.conj} man.\tsc{nom.sg} be.\tsc{inf} disciple from Galilee.\tsc{dat.pl} that.\tsc{m.sg.gen} that.\tsc{nom.sg.m} there there stood.\tsc{3.sg.prät} \\
 `what, you might have been a man, disciple from Galilee, of him who stood there' \hfill (Old Saxon, Heliand 4957-4958, \citealt[761]{behaghel1923})

\ex. Old High German
\ag. ih bibringu fona Juda dhen mina berga chisetzit\\
 \\
 `' \hfill (Old High German, Isidor 34,3, \citealt[761]{behaghel1923})
\bg. aer antwurta demo za imo sprah\\
 \\
 `' \hfill (Old High German, Monsee Fragments 7,24, \citealt[761]{behaghel1923})
\bg. gaat uz diu halt za dem iz forchaufent\\
 \\
 `' \hfill (Old High German, Monsee Fragments 20,14, \citealt[761]{behaghel1923})
\bg. thisiu fon thiu, iru wan ist, siu alla iru libnara santa (ex eo, quod)\\
 \\
 `' \hfill (Old High German, Tatian 118,1, \citealt[761]{behaghel1923})
\bg. er spráh zi then es rúahtun\\
 he spoke to those.\tsc{dat}\\
 `' \hfill (Old High German, Otfrid I,23,35, \citealt[761]{behaghel1923})
\bg. thaz iru thiu sin guati nirzigi, thes siu bati\\
 \\
 `' \hfill (Old High German, Otfrid II,8,24, \citealt[761]{behaghel1923})
\bg. bistu furira Abrahame, ouh then man hiar nu zalta.\\
 \\
 `' \hfill (Old High German, Otfrid III,18,33, \citealt[761]{behaghel1923})
\bg. thia laz ih themo iz lisit thar\\
 \\
 `' \hfill (Old High German, Otfrid I,19,25, \citealt[761]{behaghel1923})
\bg. suachit thes nan sentit\\
 \\
 `' \hfill (Old High German, Otfrid III,16,21, \citealt[761]{behaghel1923})
\bg. noh so neduohti in gnuoge des si habetin\\
 \\
 `' \hfill (Old High German, Notker I,63,29, \citealt[761]{behaghel1923})
\bg. tannoh pito ih tes noh fore ist (id quod)\\
 \\
 `' \hfill (Old High German, Notker 193,19, \citealt[761]{behaghel1923})


\ex. Middle High German
\ag. der bewiset in des er suochte\\
 he directed\scsub{gen} him {that}.\tsc{gen} he sought\scsub{acc}\\
 `He directed him to what he sought.' \hfill (Middle High German, Iwein 988, \citealt[761]{behaghel1923}), trans. Hartmann von Aue-Portal


\ex. ?
\ag. diu habe niemer niht entuot, des der seele schaden si\\
 \\
 `' \hfill (?, Warn. 2490, \citealt[761]{behaghel1923})
\bg. des himels unt der erde und swes din kint dar inne begriffen hat\\
 \\
 `' \hfill (?, R. v. Zw. 21,8, \citealt[761]{behaghel1923})
\bg. die bevogtet werden sollen mit dem nechsten vattermagen oder dem dazu erkoren wird\\
 \\
 `' \hfill (?, Weist. 1,65, \citealt[761]{behaghel1923})
\bg. diese Handlung der Merope gefalle, wem da will\\
 \\
 `' \hfill (?, Gotthold Ephraim Lessing 11,203, \citealt[761]{behaghel1923})
\bg. dem Wanderer zu bieten Schutz und Rast und wen's auch sei zu wärmen und zu laben\\
 \\
 `' \hfill (?, Redwitz, Amaranth (Bl), \citealt[761]{behaghel1923})

\exg. Magst gefallen wem du wilt\\
 \\
 `' \hfill Adversus - Komm' oh Tod




\section{Headless relative clauses}

\subsection{Proper attraction}

\ex. \tsc{gen} instead of \tsc{dat}?

\ex. \tsc{gen} instead of \tsc{acc}
\ag. alles des ich ie gesach\\
 all.\tsc{gen} that.\tsc{gen} I ever saw\\
 `all that I ever saw.' \hfill (Middle High German, Nibelungenlied 1698,1, \citealt[756]{behaghel1923}, glosses and translation by \citealt[199]{pittner1995})
\bg. Do sagete er Parziwale danc prises des erwarp sin hant\\
 there said he Parzival thanks\scsub{gen} prize.\tsc{gen} that.\tsc{gen} acquired\scsub{acc}\\
 `he thanked Parzival for the prize that he acquired' \hfill (Middle High German, Parzival 3:1209, glosses and translation by \citealt[174]{helgander1971} after \citealt[198]{pittner1995})

\ex. \tsc{gen} instead of \tsc{nom}
\ag. daz er [...] alles des verplac des im ze schaden mohte komen \\
 that he all that.\tsc{gen} abandoned\scsub{gen} that.\tsc{gen} him to damage might come\scsub{nom}\\
 `that he abandoned all that might cause damage to him' \hfill (Middle High German, Iwein 5338, \citealt[756]{behaghel1923}, glosses and translation by \citealt[198]{pittner1995})
\bg.  sie gedâht' ouch maniger leide, der ir dâ héimé geschach\\
 she thought\scsub{gen} also some sufferings.\tsc{gen} that.\tsc{gen} her at home happened\scsub{nom}\\
 `she thought about some misfortunes that happened to her at home.' \hfill (Middle High German, Nibelungenlied 1391,14, \citealt[756]{behaghel1923}, glosses and translation by \citealt[198]{pittner1995})

\ex. \tsc{dat} instead of \tsc{acc}?

\ex. \tsc{dat} instead of \tsc{nom}




\ex. \tsc{acc} instead of \tsc{nom}
\ag. unde ne wolden níet besên den mort den dô was geschên\\
 and not wanted not see the murder.\tsc{acc} that.\tsc{acc} there had happened\\
 `and they didn't want to see the murder that had happened.' \hfill (Middle High German, Nibelungenlied 1391,14, \citealt[756]{behaghel1923}, glosses and translation by \citealt[198]{pittner1995})









\phantom{x}
