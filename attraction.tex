% !TEX root = thesis.tex

\chapter{Case attraction}


\ex.
\ag. Aer antuurta demo zaimo sprah.\\
he replied\scsub{dat} who.\tsc{dat} {to him} spoke\scsub{nom}\\
`He replied to the one who spoke to him.' Pittner 199
\bg. Der bewîst in d.es er suochte.\\
he showed\scsub{gen} him what.\tsc{gen} he {looked for}\scsub{acc}\\
`He showed him what he was looking for.' Pittner 199
\z.


\section{Headless relative clauses}

\subsection{Proper attraction}


\ex. Old Saxon
\ag. botta them thar blinde warun\\
 \\
 `' \hfill (Old Saxon, Heliand 2358, \citealt[761]{behaghel1923})
\bg. minniston sindun thero nu...standad\\
 \\
 `' \hfill (Old Saxon, Heliand 4411, \citealt[761]{behaghel1923})
\bg. he thes wiht ne bisprak, thes sie imu thurk inwidnid ogean weldun\\
 \\
 `' \hfill (Old Saxon, Helian 4923, \citealt[761]{behaghel1923})
\bg. thu mahtis man wesan, giungaro fan Galilea, thes the thar genower sted\\
 \\
 `' \hfill (Old Saxon, Heliand 4957, \citealt[761]{behaghel1923})

\ex. Old High German
\ag. ih bibringu fona Juda dhen mina berga chisetzit\\
 \\
 `' \hfill (Old High German, Isidor 34,3, \citealt[761]{behaghel1923})
\bg. aer antwurta demo za imo sprah\\
 \\
 `' \hfill (Old High German, Monsee Fragments 7,24, \citealt[761]{behaghel1923})
\bg. gaat uz diu halt za dem iz forchaufent\\
 \\
 `' \hfill (Old High German, Monsee Fragments 20,14, \citealt[761]{behaghel1923})
\bg. thisiu fon thiu, iru wan ist, siu alla iru libnara santa (ex eo, quod)\\
 \\
 `' \hfill (Old High German, Tatian 118,1, \citealt[761]{behaghel1923})
\bg. er sprah zi then es ruahtun\\
 \\
 `' \hfill (Old High German, Otfrid I,23,35, \citealt[761]{behaghel1923})
\bg. thaz iru thiu sin guati nirzigi, thes siu bati\\
 \\
 `' \hfill (Old High German, Otfrid II,8,24, \citealt[761]{behaghel1923})
\bg. bistu furira Abrahame, ouh then man hiar nu zalta.\\
 \\
 `' \hfill (Old High German, Otfrid III,18,33, \citealt[761]{behaghel1923})
\bg. thia laz ih themo iz lisit thar\\
 \\
 `' \hfill (Old High German, Otfrid I,19,25, \citealt[761]{behaghel1923})
\bg. suachit thes nan sentit\\
 \\
 `' \hfill (Old High German, Otfrid III,16,21, \citealt[761]{behaghel1923})
\bg. noh so neduohti in gnuoge des si habetin\\
 \\
 `' \hfill (Old High German, Notker I,63,29, \citealt[761]{behaghel1923})
\bg. tannoh pito ih tes noh fore ist (id quod)\\
 \\
 `' \hfill (Old High German, Notker 193,19, \citealt[761]{behaghel1923})


\ex. Middle High German
\ag. der bewiset in des er suochte\\
 he directed\scsub{gen} him {what}.\tsc{gen} he sought\scsub{acc}\\
 `He directed him to what he sought.' \hfill (Middle High German, Iwein 988, \citealt[761]{behaghel1923}), trans. Hartmann von Aue-Portal


\ex. ?
\ag. diu habe niemer niht entuot, des der seele schaden si\\
 \\
 `' \hfill (?, Warn. 2490, \citealt[761]{behaghel1923})
\bg. des himels unt der erde und swes din kint dar inne begriffen hat\\
 \\
 `' \hfill (?, R. v. Zw. 21,8, \citealt[761]{behaghel1923})
\bg. die bevogtet werden sollen mit dem nechsten vattermagen oder dem dazu erkoren wird\\
 \\
 `' \hfill (?, Weist. 1,65, \citealt[761]{behaghel1923})
\bg. diese Handlung der Merope gefalle, wem da will\\
 \\
 `' \hfill (?, Less. 11,203, \citealt[761]{behaghel1923})
\bg. dem Wanderer zu bieten Schutz und Rast und wen's auch sei zu wärmen und zu laben\\
 \\
 `' \hfill (?, Redwitz, Amaranth (Bl), \citealt[761]{behaghel1923})

\phantom{x}
