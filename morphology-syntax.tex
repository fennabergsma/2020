% !TEX root = thesis.tex

\chapter{Connecting morphology and syntax}


\section{Background: relative clause theory}
Standard raising, probably Cinque's double-headed structures

\section{Analysis}

\subsection{Old High German}
In \ac{ohg}, proper attraction in headless relatives can be derived from headed relatives. The relative pronoun is the determiner from the main clause. Under a double-headed Cinque-analysis, it is the internal DP that is deleted.




\ex. \ac{acc} instead of \ac{nom}
\ag. unde ne wolden níet besên den mort den dô was geschên\\
 and not wanted not see the murder.\ac{acc} that.\ac{acc} there had happened\\
 `and they didn't want to see the murder that had happened.' \flushfill{MHG, \ac{nib} 1391,14, \pgcitealt{behaghel1923}{756}, after \pgcitealt{pittner1995}{198}}



\subsection{Modern German}
In German, inverse attraction in headed relatives can be shown to be very different from inverse attraction in headless relatives. I am not set on an analysis yet. Under a double-headed Cinque-analysis, it is the external DP that is deleted. Grafting is also still an option.


\subsection{Gothic}
In Gothic, ?
