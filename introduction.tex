% !TEX root = thesis.tex

\chapter{Introduction}

This dissertation is about case competition, a situation in which two cases are assigned but only one of them surfaces. One of the constructions in which case competition appears is relative clauses that lack a head, i.e. headless relatives.

I show that one aspect about case competition in headless relatives holds for all languages (under discussion here at least). That is, there is a fixed order which decides which case wins the competition. Another aspect of case competition in headless relatives differs per language. That is, whether the competition takes place to begin with. I connect this variable to the morphology of the language in question.

This phenomenon has been described as some special property of a few special languages. Therefore, language-specific rules have been postulated to account for the data. My goal is to show that this phenomenon can be captured with `normal' syntactic processes, like ellipsis, c-command. The account makes predictions about how a language behaves based on the shape of its relative pronouns. And we see that the phenomenon is actually more wide-spread than what has been assumed.
%%this is too difficult

In this introduction I first introduce what I mean exactly with case competition in headless relatives. Then I introduce the topics I discuss in this dissertation.


\section{Introducing the title}

First, case marks the grammatical role of the noun phrases. Case also appears on relative pronoun. Case on head can differ from case on relative pronoun. What happens if there is no noun? Two cases come together on the relative pronoun. What holds for all languages: there is a fixed order of who wins the competition. Specific from language to language: when does the competition take place?

Languages can use case to mark the grammatical role of a noun phrase in a clause. Consider the two Modern German sentences in \ref{ex:germancase}. The case marking of the noun phrases is reflected on the determiner in the noun phrase.
In \ref{ex:germancase1}, \tit{der} in \tit{der Lehrer} `the teacher' is assigned nominative case, because it is the subject in the clause. \tit{Den} in \tit{den Schüler} `the pupil' is assigned accusative case, because it is an object of \tit{mag} `likes'.
In \ref{ex:germancase2}, the roles are reversed: \tit{der} in \tit{der Schüler} `the pupil' is assigned nominative case, because it is the subject in the clause. \tit{Den} in \tit{den Lehrer} `the teacher' is assigned accusative case, because it is the object of \tit{mag} `likes'.
The grammatical roles of the noun phrases in \ref{ex:germancase} can also be derived from the positioning in the clause. The subjects precede the predicate \tit{mag} `likes' and the objects follow it. As it is not relevant for the discussion here, I do not discuss the positioning of noun phrases in the clause into further detail.

\ex.\label{ex:germancase}
\ag. Der Lehrer mag den Schüler.\\
 the.\tsc{nom} teacher likes the.\tsc{acc} student\\
 `The teacher likes the pupil.'\label{ex:germancase1}
\bg. Der Schüler mag den Lehrer.\\
 the.\tsc{nom} student likes the.\tsc{acc}\\
 `the pupil likes the teacher.'\label{ex:germancase2}

Not only full noun phrases, but also other elements can be marked for case, such relative pronouns. Modern German marks relative pronouns, just like full noun phrases, for the grammatical role they have in the clause. Consider the two sentences in \ref{ex:germanrelatives}. These two sentences both consist of a main clause that is modified by a relative clause, which is placed between brackets.
In \ref{ex:germanrelative1}, the relative clause \tit{der nach draußen guckt} `that looks outside' modifies \tit{den Schüler} `the pupil'. \tit{Den Schüler} `the pupil` is called the head (noun) or the antecedent of the relative clause. \tit{Den} in \tit{den Schüler} `the pupil` is assigned accusative case, because it is the object of \tit{mag} `likes' in the main clause. The relative pronoun \tit{der} `that.\tsc{nom}' is assigned nominative case, because it is the subject of in the relative clause.

In \ref{ex:germanrelative2}, the relative clause \tit{den er beim Verstecktspiel sucht} `that he is searching for playing hide-and-seek' modifies \tit{den Schüler} `the pupil'. \tit{Den} in \tit{den Schüler} `the pupil` is again marked as accusative, because it is the object of \tit{mag} `likes' in the main clause. The relative pronoun \tit{den} `that.\tsc{acc}' is assigned accusative case, because it is the object of \tit{sucht} `searches' in the relative clause.

\ex.\label{ex:germanrelatives}
\ag. Der Lehrer mag den Schüler, [der nach draußen guckt].\\
 the.\tsc{nom} teacher likes the.\tsc{acc} student that.\tsc{nom} to outside looks\\
 `The teacher likes the pupil that is looking outside.'\label{ex:germanrelative1}
 \bg. Der Lehrer mag den Schüler, [den er beim Verstecktspiel sucht].\\
 the.\tsc{nom} teacher likes the.\tsc{acc} student that.\tsc{acc} he {at the} {hide-and-seek game} searches\\
 `The teacher likes the pupil that he is searching for playing hide-and-seek.'\label{ex:germanrelative2}

Compare the two sentences in \ref{ex:germanrelatives}. In both sentences the head is marked accusative because it is the object in the main clause. The case of the relative pronoun in \ref{ex:germanrelative2} is also accusative, because of it is the object in the relative clause. The case of the relative pronoun in \ref{ex:germanrelative1} differs from the case of the head, it is nominative.


The focus of this dissertation lies on the headless relative, i.e. a relative clause that does not have a head. As the name suggests, this type of relative clause lacks a head.\footnote{
This `missing noun' has been interpreted in two different ways. Some researchers argue that the noun is truly missing, it is absent, cf. \citealt{vanriemsdijk2006}. Others claim that there is actually a head, but it is phonologically zero, \citealt{himmelreich2017}. At this point in the discussion this distinction is not relevant. I return to the issue in Chapter \ref{ch:connecting}.
}
Consider the Gothic example of a headless relative in \ref{ex:gothicaccacc}. I placed subscripts between the square brackets on the glosses of verbs. They indicate which case the verbs assign to their object.
In \ref{ex:gothicaccacc}, the relative clause \tit{þan -ei arma} `who I pity' is placed between square brackets. There is no head that this relative clause modifies, it is a headless relative. This is different from the examples from German I gave above, which each had a head.
The relative pronoun \tit{þan(a)} `who.\ac{acc}' is assigned accusative case.\footnote{
The relative pronoun without the complementizer \tit{-ei} is \tit{þana}. Therefore, I refer to the relative pronoun as \tit{þan(a)}.
}

\exg. gaarma [þan -ei arma]\\
 pity\scsub{[acc]} who.\ac{acc} -\tsc{comp} pity\scsub{[acc]}\\
 `I will pity him whom I pity' \flushfill{Gothic, \ac{rom} 9:15, after \pgcitealt{harbert1978}{339}}\label{ex:gothicaccacc}

Where does this accusative case assignment come from? Logically speaking, there are two candidates: the predicate in the main clause \tit{gaarma} `pity' and the predicate in the relative clause \tit{arma} `pity'. Did the predicate in the relative clause \tit{arma} `pity' assign accusative case? In the headed relative clauses in \ref{ex:germanrelatives}, the relative pronoun received its case from the predicate in the relative clause. The crucial difference with that type of relative clause is that there is a head for the main clause to assign its case to. Did the predicate in the main clause \tit{gaarma} `pity' assign accusative case? I will argue that both of them did. Actually, \ref{ex:gothicaccacc} is the first example I gave of case competition in a headless relative. It is an uninteresting one, because the two competing cases are identical.

In the remainder of this section I show evidence for the claim that relative pronouns in headless relatives take the case of the predicate in the relative clause and the predicate in the main clause. This evidence comes from headless relatives in which the predicate in the relative clause takes a different case from the predicate in the main clause.

Consider the example in \ref{ex:gothicaccdat}. The relative clause is \tit{ana þamm -ei lag} `on what he lay'.
- here the subscript are on the preposition, because the preposition selects for a case
- on selects for dative, which is in the relative clause
- picking up selects for accusative which is in the relative clause
- we see this dative on the relative pronoun, and this can only come from the relative clause predicate

\exg. ushafjands [ana þamm -ei lag]\\
 {picking up}\scsub{[acc]} on\scsub{[dat]} what.\ac{dat} -\tsc{comp} lay\\
 `picking up that on which he lay' \flushfill{Gothic, \ac{luke} 5:25, after \pgcitealt{harbert1978}{343}}\label{ex:gothicaccdat}

-realtive: say selects for accusative
-main: do selects dative
- we see the dative again, and this can only come from the main clause

\exg. hva nu wileiþ ei taujau [þamm -ei qiþiþ þiudan Iudaie]?\\
 what now want that do\scsub{[dat]} who.\ac{dat} -\tsc{comp} say\scsub{[acc]} king {of Jews}\\
 `what now do you wish that I do to him whom you call King of the Jews?' \flushfill{Gothic, \ac{mark} 15:12, after \pgcitealt{harbert1978}{339}}





\section{The content of this dissertation}


1=case competition, there is a heirarchy. cite people
first, what holds for all languages is that there is a single order: highest in the hierarchy wins
in the first part, I..

secondly I adress an aspect that differs across languages
that is, not all languages are like gothic
I connect this crosslinguistic variation to morphology.. so i reduce it to differences in the lexicon

finally, i show how all of this can be derived in derivations
