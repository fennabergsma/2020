% !TEX root = thesis.tex

\chapter{Introduction}

--general intro--

Languages can use case to mark the grammatical role of a noun phrase in a clause. Consider the two Modern German sentences in \ref{ex:germancase}. In \ref{ex:germancase1}, \tit{der Lehrer} `the teacher' is marked nominative, and it is the subject. \tit{Den Schüler} `the student' is marked accusative, and it is an object. In \ref{ex:germancase2}, the roles are reversed: \tit{der Schüler} `the student' is marked nominative and it is the subject, and \tit{den Lehrer} `the teacher' is marked accusative and it is the object. Notice also that the subject precedes the predicate \tit{mag} `likes' and the object follows it.

\ex.\label{ex:germancase}
\ag. Der Lehrer mag den Schüler.\\
 the.\tsc{m.nom} teacher likes the.\tsc{m.acc} student\\
 `The teacher likes the student.'\label{ex:germancase1}
\bg. Der Schüler mag den Lehrer.\\
 the.\tsc{m.nom} student likes the.\tsc{m.acc}\\
 `The student likes the teacher.'\label{ex:germancase2}

Not only full noun phrases, but also other elements can be marked for case, such relative pronouns. Modern German marks relative pronouns, just like full noun phrases, for the grammatical role they have in the clause. Consider the two sentences in \ref{ex:germanrelatives}. In \ref{ex:germanrelative1}, the relative pronoun \tit{der} `that.\tsc{m.nom}' introduces a clause that modifies \tit{den Schüler} `the student'. \tit{Der} `that.\tsc{m.nom}' is marked masculine and nominative. The relative pronoun is marked masculine, because it agrees in gender with its antecedent \tit{den Schüler} `the student'. It is marked nominative, because of its grammatical role: it is the subject in the relative clause.
In \ref{ex:germanrelative2}, the relative pronoun \tit{den} `that.\tsc{m.acc}' is marked masculine and accusative. Again, the relative pronoun is marked masculine, because it agrees in gender with its antecedent \tit{den Schüler} `the student'. It is marked accusative, because of its grammatical role: it is the object in the relative clause.

\ex.\label{ex:germanrelatives}
\ag. Der Lehrer mag den Schüler, der nach draußen guckt.\\
 the.\tsc{m.nom} teacher likes the.\tsc{m.acc} student that.\tsc{m.nom} to outside looks\\
 `The teacher likes the student that is looking outside.'\label{ex:germanrelative1}
 \bg. Der Lehrer mag den Schüler, den er beim Verstecktspiel sucht.\\
 the.\tsc{m.nom} teacher likes the.\tsc{m.acc} student that.\tsc{m.acc} he {at the} {hide-and-seek game} seeks\\
 `The teacher likes the student that he is looking for playing hide-and-seek.'\label{ex:germanrelative2}

%%work out after this
If there is no head, we have a headless relative
in these, there is only one place for the case.

--give example of a matching one--

then the cases can differ
then, there is a case conflict

there we have case competition in headless relatives
this thesis discusses different aspects of this phenomenon

first, what holds for all languages is that there is a single order: highest in the hierarchy wins
in the first part, I..

secondly I adress an aspect that differs across languages
that is, not all languages are like gothic
I connect this crosslinguistic variation to morphology.. so i reduce it to differences in the lexicon

finally, i show how all of this can be derived in derivations
