% !TEX root = thesis.tex

\chapter{Introduction}

This dissertation is about case competition, a situation in which two cases are assigned but only one of them surfaces. One of the constructions in which case competition appears is relative clauses that lack a head, i.e. headless relatives.

% I show that one aspect about case competition in headless relatives holds for all languages (under discussion here at least). That is, there is a fixed order which decides which case wins the competition. I let this follow from what we observe in morphology. Another aspect of case competition in headless relatives differs per language. That is, whether the competition takes place to begin with. I connect this variable to the morphology of the language in question.
%
% Case competition in headless relatives has been described as some special property of a few special languages. Therefore, language-specific rules have been postulated to account for the data. My goal is to show that this phenomenon can be captured with `normal' syntactic processes, like ellipsis, c-command. The account makes predictions about how a language behaves based on the shape of its relative pronouns. And we see that case competition in headless relatives is actually more wide-spread than what has been assumed.
% %%this is too difficult

In this introduction I first introduce what I mean exactly with case competition in headless relatives. Then I introduce the topics I discuss in this dissertation.


\section{Decomposing the title}

% First, case marks the grammatical role of the noun phrases. Case also appears on relative pronoun. Case on head can differ from case on relative pronoun. What happens if there is no noun? Two cases come together on the relative pronoun. What holds for all languages: there is a fixed order of who wins the competition. Specific from language to language: when does the competition take place?

Languages can use case to mark the grammatical role of a noun phrase in a clause \citep[cf.][]{moravcsik2009}. Consider the two Modern German sentences in \ref{ex:german-case}. What can descriptively be called the subject of the predicate \tit{mögen} `to like' is marked as nominative. What can be described as the object of \tit{mögen} `to like' is marked as accusative. The case marking of the noun phrases is reflected on the determiner in the noun phrase.
In \ref{ex:german-case-1}, \tit{der} in \tit{der Lehrer} `the teacher' appears in nominative case, because it is the descriptive subject in the clause. \tit{Den} in \tit{den Schüler} `the pupil' appears in accusative case, because it is a descriptive object of \tit{mögen} `to like'.
In \ref{ex:german-case-2}, the grammatical roles are reversed: \tit{der} in \tit{der Schüler} `the pupil' appears in nominative case, because it is the descriptive subject in the clause. \tit{Den} in \tit{den Lehrer} `the teacher' appears in accusative case, because it is the descriptive object of \tit{mögen} `to like'.

\ex.\label{ex:german-case}
\ag. Der Lehrer mag den Schüler.\\
 the.\ac{nom} teacher likes the.\ac{acc} student\\
 `The teacher likes the pupil.'\label{ex:german-case-1}
\bg. Der Schüler mag den Lehrer.\\
 the.\ac{nom} student likes the.\ac{acc} teacher\\
 `The pupil likes the teacher.'\label{ex:german-case-2}

Not only full noun phrases, but also other elements can be marked for case, such as relative pronouns. Modern German marks relative pronouns, just like full noun phrases, for the grammatical role they have in the clause. Consider the two sentences in \ref{ex:german-relatives}. These two sentences both contain a main clause that is modified by a relative clause.
In \ref{ex:german-relative-1}, the relative clause \tit{der nach draußen guckt} `that looks outside' modifies \tit{den Schüler} `the pupil'. \tit{Schüler} `pupil` is called the head (noun) or the antecedent of the relative clause. \tit{Den} in \tit{den Schüler} `the pupil` appears in accusative case, because it is the descriptive object of \tit{mögen} `to like' in the main clause. The relative pronoun \tit{der} `\ac{rel}.\ac{nom}.\ac{sg}.\ac{m}' appears in nominative case, because it is the descriptive subject of \tit{mögen} `to like' in the relative clause.

In \ref{ex:german-relative-2}, the relative clause \tit{den er beim Verstecktspiel sucht} `that he is searching for playing hide-and-seek' modifies \tit{den Schüler} `the pupil'. \tit{Den} in \tit{den Schüler} `the pupil` appears again in accusative, because it is the descriptive object of \tit{mögen} `to like' in the main clause. The relative pronoun \tit{den} `\ac{rel}.\ac{acc}.\ac{sg}.\ac{m}' appears in accusative case, because it is the descriptive object of \tit{suchen} `to search' in the relative clause.

\ex.\label{ex:german-relatives}
\ag. Der Lehrer mag den Schüler, der nach draußen guckt.\\
 the.\ac{nom} teacher likes the.\ac{acc} student \ac{rel}.\ac{nom}.\ac{sg}.\ac{m} to outside looks\\
 `The teacher likes the pupil that is looking outside.'\label{ex:german-relative-1}
 \bg. Der Lehrer mag den Schüler, den er beim Versteckspiel sucht.\\
 the.\ac{nom} teacher likes the.\ac{acc} student \ac{rel}.\ac{acc}.\ac{sg}.\ac{m} he {at the} {hide-and-seek game} searches\\
 `The teacher likes the pupil that he is searching for playing hide-and-seek.'\label{ex:german-relative-2}

Compare the two sentences in \ref{ex:german-relatives}. In both sentences the head is marked as accusative because it is the descriptive object in the main clause. The case of the relative pronoun in \ref{ex:german-relative-2} is also accusative, because it is the descriptive object in the relative clause. The case of the relative pronoun in \ref{ex:german-relative-1} is nominative, because it is the descriptive subject in the relative clause. So, the case of the relative pronoun in \ref{ex:german-relative-1} differs from the case of the head.

The focus of this dissertation lies on headless relatives. As the name suggests, this type of relative clause lacks a head.\footnote{
This `missing noun' has been interpreted in two different ways. Some researchers argue that the noun is truly missing, it is absent, cf. \citealt{citko2005,vanriemsdijk2006}. Others claim that there is actually a head, but it is phonologically zero, \citealt{bresnan1978,groos1981,grosu2003}. At this point in the discussion this distinction is not relevant. I return to the issue in Chapter \ref{ch:relativization}.
}
Even though Modern German also has case competition in headless relatives, I turn to Gothic now. The patterns among the two languages differ slightly, and the first part of the dissertation can be illustrated best with Gothic.

I give an example of a headless relative in Gothic in \ref{ex:gothic-acc-acc}.
There is no head that this relative clause modifies, because it is a headless relative. This is different from the examples from German I gave above, which each had a head.
The predicate \tit{arman} `to pity' takes accusative objects, as indicated by the subscript on the gloss of the verb. The predicate \tit{gaarman} `to pity' also takes accusative objects, indicated again by the subscript.
The relative pronoun \tit{þan(a)} `\ac{rel}.\ac{acc}.\ac{sg}.\ac{m}' appears in accusative case.\footnote{
The relative pronoun without the complementizer \tit{-ei} is \tit{þana}. Therefore, I refer to the relative pronoun as \tit{þan(a)}.
}

\exg. gaarma þan -ei arma\\
 pity.\ac{pres}.1\ac{sg}\scsub{[acc]} \ac{rel}.\ac{acc}.\ac{sg}.\ac{m} -\ac{comp} pity.\ac{pres}.1\ac{sg}\scsub{[acc]}\\
 `I pity him whom I pity' \flushfill{Gothic, \ac{rom} 9:15, adapted from \pgcitealt{harbert1978}{339}}\label{ex:gothic-acc-acc}

Where does this accusative case come from? Logically speaking, there are two possible sources: the predicate in the main clause \tit{gaarman} `to pity', the predicate in the relative clause \tit{arman} `to pity'. From now on, I use the terms internal and external case to refer to these two possible case sources. Now there are three logical possibilities for the source of the accusative case on \tit{þan(a)} `\ac{rel}.\ac{acc}.\ac{sg}.\ac{m}' in \ref{ex:gothic-acc-acc}: the internal case, the external case, or both.

Internal case refers to the case associated with the relative pronoun internal to the relative clause. More precisely, it is the case, which is associated with the grammatical role that the relative pronoun has internal to the relative clause. In \ref{ex:gothic-acc-acc}, the relative pronoun is the descriptive object of \tit{arman} `to pity'. The predicate \tit{arman} `to pity' takes accusative objects. So, the internal case is accusative.

External case refers to the case associated with the missing head in the main clause, which is external to the relative clause. Concretely, it is the case which is associated with the grammatical role that the missing head has external to the relative clause. In \ref{ex:gothic-acc-acc}, the missing head is the descriptive object of \tit{gaarman} `to pity'. The predicate \tit{gaarman} `to pity' takes accusative objects. In \ref{ex:gothic-acc-acc}, the external case is accusative.

Now I return to the question where \tit{þan(a)} `\ac{rel}.\ac{acc}.\ac{sg}.\ac{m}' in \ref{ex:gothic-acc-acc} got its case from. In the remainder of this section I show evidence for the claim that the relative pronoun is sensitive to both the internal and the external case.
This is easy to imagine for the internal case: the internal case reflects the grammatical role of the relative clause. It is a bit more complicated for the external case. The external case is associated with the grammatical role of the missing head in the main clause. The idea is going to be that the external case cannot be reflected on a non-existing head. Indirectly, it appears on the relative pronoun.\footnote{
Later on I will argue that this indirect process is actually a deletion operation.
}
This means that the internal and external case come together on the relative pronoun. In other words, there is case competition going on in headless relatives. \ref{ex:gothic-acc-acc} is indeed the first example I gave of case competition in a headless relative. It is an uninteresting one, because the two competing cases are identical.

Consider the example in \ref{ex:gothic-acc-nom}, in which the internal case is accusative and the external case is nominative.
The internal case is accusative. The predicate \tit{frijon} `to love' takes accusative objects, as indicated by the subscript on the predicate.
The external case is accusative. The predicate \tit{wisan} `to be' takes nominative subjects, indicated by the subscript on the predicate.
The relative pronoun \tit{þan(a)} `\ac{rel}.\ac{acc}.\ac{sg}.\ac{m}' appears in accusative. This accusative can only come from the predicate \tit{frijon} `to love', which is the internal case here. The relative pronoun is marked in bold, just as the relative clause, showing that the relative pronoun patterns with the relative clause.

\exg. \tbf{þan} \tbf{-ei} \tbf{frijos} siuks ist\\
 \ac{rel}.\ac{acc}.\ac{sg}.\ac{m} -\ac{comp} love.\ac{pres}.2\ac{sg}.\scsub{[acc]} sick be.\ac{pres}.3\ac{sg}\scsub{[nom]}\\
 `the one whom you love is sick' \flushfill{Gothic, \ac{john} 11:3, adapted from \pgcitealt{harbert1978}{342}}\label{ex:gothic-acc-nom}

The conclusion that follows is that the relative pronoun can take the internal case. At this point it remains unclear what happened to the external nominative case.

Now consider the example in \ref{ex:gothic-nom-acc}, in which the internal case is nominative and the external case is accusative.
The internal case is nominative. The predicate \tit{wisan} `to be' takes nominative subjects, as indicated by the subscript on the predicate.
The external case is accusative. The predicate \tit{ussiggwan} `to read' takes accusative objects, as indicated by the subscript on the predicate.
The relative pronoun \tit{þo} `\ac{rel}.\ac{acc}.\ac{sg}.\ac{n}' appears in the accusative case. This accusative can only come from the predicate \tit{ussiggwan} `to read', which is the external case here. The relative pronoun is not marked in bold, just like as the main clause, showing that the relative pronoun patterns with the main clause.

\exg. jah þo \tbf{-ei} \tbf{ist} \tbf{us} \tbf{Laudeikaion} jus ussiggwaid\\
 and \ac{rel}.\ac{acc}.\ac{sg}.\ac{n} -\ac{comp} be.\ac{pres}.3\ac{sg}\scsub{[nom]} from Laodicea you.\ac{pl} read.?.\scsub{[acc]}\\
 `and you read the one which is from Laodicea' \flushfill{Gothic, \ac{col} 4:16, adapted from \pgcitealt{harbert1978}{357}}\label{ex:gothic-nom-acc}

The conclusion that follows is that the relative pronoun can take the external case. At this point it remains unclear what happened to the internal nominative case.

The examples in \ref{ex:gothic-acc-nom} and \ref{ex:gothic-nom-acc} have shown that the relative pronoun in headless relatives can take either the internal or the external case. In the examples, the predicates take nominative and accusative, and in both cases, the relative pronoun appeared in accusative case. In other words, there was a competition between nominative and accusative, and accusative won.

In the next section, I discuss the content of this dissertation. Before that, I comment on two notational conventions I use throughout this dissertation. First, I place subscripts on the glosses of the predicates. They indicate what the internal or external case is. The subscript on the predicate in the relative clause indicates the internal case. The subscript on the predicate in the main clause indicates the external case. This subscript can mean different things.
For \tit{frijon} `to love' in \ref{ex:gothic-acc-nom} the subscript indicates which case the complement of the verb appears in. The subscript on \tit{wisan} `to be' in \ref{ex:gothic-acc-nom} refers to the case the descriptive subject appears in. A subscript can also refer to the case of the indirect object of a predicate, a possibility that arises in the next chapter.
In other words, the subscript can refer several elements: a subject, direct object or indirect object of a predicate. There is no overarching theoretical notion that the subscript makes reference to. The subscript simply indicates which case is required within the (main or relative) clause.

Second, I write the relative clause in bold. When the relative pronoun takes the internal case, I mark it in bold as well, as shown in \ref{ex:gothic-acc-nom}. When the relative pronoun takes the external case, I leave it black, indicating it patterns with the main clause. An example of that is \ref{ex:gothic-nom-acc}.


\section{The content of this dissertation}

In the previous section I introduced the notion of case competition, and I illustrated how it appears in headless relatives. This dissertation discusses two question regarding this phenomenon.
The first one is which case is going to win the case competition, i.e. which case surfaces. I discuss this in Part \ref{part:case-facts}.
The second question is whether both competitors are able to compete in the competition, i.e. whether one of the cases is surfacing or both are ungrammatical. I discuss this in Part \ref{part:variation}.
For both I will show that morphology is leading. What we observe in syntax is a reflex of the morphology.


\section{The scope of this dissertation}

\subsection{Case attraction}\label{sec:attraction}

Case attraction in headed relatives seems related, but I will not account for it.

\exg. unde ne wolden níet besên den mort den dô was geschên\\
 and not wanted not see the murder.\ac{acc} that.\ac{acc} there had happened\\
 `and they didn't want to see the murder that had happened.' \flushfill{MHG, \ac{nib} 1391,14, \pgcitealt{behaghel1923}{756}, after \pgcitealt{pittner1995}{198}}

 \exg. Den schilt den er {vür bôt} der wart schiere zeslagen\\
 the.\tsc{acc} shield.\tsc{acc} which.\tsc{acc} he held\scsub{acc}, that.\tsc{nom} was quickly shattered\scsub{nom}\\
 `The shield he held was quickly shattered' \label{ex:iaheaded}\hfill Iwein 6722f., Lenerz 1984: 116)

OHG has case attraction in headed relatives, Gothic does not, but both show case competition in headless relatives. So, there does not seem to be a one-to-one connection there. I leave it for further research.

\subsection{Syncretism}

For a long time it has been noted that syncretism seems to resolve case conflicts. --references--

A language like Polish, that normally doesn't allow for any case mismatches, even allows for it. In this dissertation I do not offer a detailed account for what a derivation looks like.

\exg. Jan unika kogokolwiek wczoraj obraził.\\
Jan avoid.\tsc{3sg}\scsub{[gen]} \tsc{rel}.\tsc{acc/gen}.\ac{sg}.\ac{m} yesterday offend.\tsc{3sg}.\tsc{past}\scsub{[acc]}.\\
`Jan avoided whoever he offended yesterday.'

I won't talk about the details.

\subsection{The genitive}\label{sec:genitive}

In Gothic headless relatives, there is data available of the genitive in case competition with the accusative. The genitive wins in this competition.
I give an example in which the internal case is accusative and the external case is genitive in \ref{ex:gothic-gen-acc}.
The relative clause is marked in bold, the relative pronoun is not.
The internal case is accusative. The predicate \tit{gasehvun} `saw' takes accusative objects.
The external case is genitive. The noun \tit{waiht} `thing' combines with a genitive.
The relative pronoun \tit{þiz(e)} `what.\ac{gen}' appears in the external case: the genitive.

\exg. ni waiht þiz \tbf{-ei} \tbf{gasehvun}\\
 not thing\scsub{[gen]} what.\ac{gen} -\ac{comp} saw\scsub{[acc]}\\
 `not any of (that) which they saw' \flushfill{Gothic, \ac{luke} 9:36, adapted from \pgcitealt{harbert1978}{340}}\label{ex:gothic-gen-acc}

If the internal case is genitive and the external case is accusative, the genitive wins as well. Crucially, there are no attested examples in Gothic of genitives in case competition with nominatives or datives.

The same holds for the two other main languages discussed in this thesis: Modern German and Old High German.
In Modern German, case competitions have been reported between all possible case combinations, so also between genitives and nominatives, between genitives and accusatives, and between genitives and datives \citep[cf.][]{vogel2001}. The genitive wins over the nominative and the accusative. In a competition between the genitive and the dative neither of them gives a grammatical result.
Old High German might show some examples of case competition between genitives and accusatives and genitives and nominative. In these cases, the genitive always wins. No examples of datives against genitives are attested \citep{behaghel1923}.
In sum, the genitive does not appear in all possible case competition combinations in all three languages, and is therefore excluded.

What do I predict for the genitive? Starke: S-acc --- S-dat --- gen --- B-acc --- B-dat

hierarchies for each language individually. Gothic syncretisms: acc-dat, acc-nom, nom-gen(!). Modern German: nom-acc-dat-gen? Old High German: ?

then the predictions would be..

The genitive differs from the other cases in a particular way. That is, nominative, accusative and dative are dependents of the verb (or prepositions). Genitives can be dependents of verbs, or they can be dependents of nouns, as possessors or partitives. Consider the example in \ref{ex:gothic-gen-acc}. The genitive relative pronoun \tit{þiz(e)} `what.\tsc{gen}' is a dependent of the noun \tit{waiht} `thing'. Most of the examples in headless relatives contain genitives that depend on nouns and not those that depend on verbs. The (genitive) possessor is also placed far away from the other three cases in \posscitet{keenan1977} relativization hiearchy.

more: in middle high german only the genitive shows case attraction in headed relatives. again, it is different from the others.

I leave it for future research..
