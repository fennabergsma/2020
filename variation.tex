% !TEX root = thesis.tex

\chapter{The variation}


\section{The different patterns}

In Gothic, the more complex case wins.
In \ac{ohg}, the more complex case wins, only if it is external.
In \ac{mg}, the more complex case wins, only if it is internal.
In Italian, case mismatch is not allowed.


\begin{table}[H]
	\center
	\caption {Variation}
		\begin{tabular}{ccc}
		\toprule
		 					& \ac{int}>\ac{ext}		& \ac{ext}>\ac{int}	\\
								\cmidrule{2-3}
		\ac{mg} 	& ✔			 							&	*									\\
		\ac{ohg}	& *										&	✔									\\
		Gothic		&	✔										&	✔									\\
		Italian		&	*										& *									\\
		\bottomrule
		\end{tabular}
\end{table}



\subsection{Both: Gothic}


% !TEX root = ../thesis.tex

\begin{tabular}{c|c|c|c}
  \toprule
    \diagbox[linecolor=white]{\ac{int}}{\ac{ext}}
        & [\ac{nom}]
        & [\ac{acc}]
        & [\ac{dat}]
        \\ \cmidrule{1-4}
    [\ac{nom}]
        & 
        & \diagbox[linecolor=white]{*\ac{nom}}{\colorbox{LG}{\ac{acc}}}
        & \diagbox[linecolor=white]{*\ac{nom}}{\colorbox{LG}{\ac{dat}}}
        \\ \cmidrule{1-4}
    [\ac{acc}]
        & \diagbox[linecolor=white]{\colorbox{LG}{\ac{acc}}}{*\ac{nom}}
        &
        & \diagbox[linecolor=white]{*\ac{acc}}{\colorbox{LG}{\ac{dat}}}
        \\ \cmidrule{1-4}
    [\ac{dat}]
        & \diagbox[linecolor=white]{\colorbox{LG}{\ac{dat}}}{*\ac{nom}}
        & \diagbox[linecolor=white]{\colorbox{LG}{\ac{dat}}}{*\ac{acc}}
        &
        \\
  \bottomrule
\end{tabular}



\subsection{Only from external: Old High German}


\ex. \ac{int}:\ac{nom}, \ac{ext}:\ac{acc}
\a. \ac{nom} not attested
\bg. ih bibringu fona Juda [dhen mina berga chisetzit]\\
 I educate\scsub{[acc]} about Juda who.\ac{acc} my mountains {through pull}\scsub{[nom]}\\
 `I educate the one who wanders through my mountains about Judas' \flushfill{\ac{ohg}, \ac{isid} 34:3, \pgcitealt{behaghel1923}{761}}

\ex. \ac{int}:\ac{nom}, \ac{ext}:\ac{dat}
\a. \ac{nom} not attested
\bg. aer antuurta [demo zaimo sprah]\\
 he replied\scsub{[dat]} who.\ac{dat} {to him} spoke\scsub{[nom]}\\
 `he replied to the one who spoke to him' \flushfill{\ac{ohg}, \ac{mons} 7:24, \pgcitealt{behaghel1923}{761}, after \pgcitealt{pittner1995}{199}}

\ex. \ac{int}:\ac{acc}, \ac{ext}:\ac{nom}
\a. \ac{acc} not attested
\b. \ac{nom} not attested

\ex. \ac{int}:\ac{acc}, \ac{ext}:\ac{dat}
\a. \ac{acc} not attested
\bg. istû furira Abrâhame, ouh [thên man hiar nû zalta]?\\
 {are you} superior\scsub{[dat]} {to Abraham} also who.\ac{dat} one here now named\scsub{[acc]}\\
 `are you superior to Abraham to those which they just mentioned?' \flushfill{\ac{ohg}, \ac{otfrid} III 18:33, \pgcitealt{behaghel1923}{761}}

\ex. \ac{int}:\ac{dat}, \ac{ext}:\ac{nom}
\a. \ac{dat} not attested
\b. \ac{nom} not attested

\ex. \ac{int}:\ac{dat}, \ac{ext}:\ac{acc}
\a. \ac{dat} not attested
\b. \ac{acc} not attested



Don't know:

\ex. \ac{ohg}
\ag. gaat uz diu halt za dem iz forchaufent\\
 \\
 `' \flushfill{\ac{ohg}, Monsee Fragments 20,14, \citealt[761]{behaghel1923}}
% \bg. thisiu fon thiu, iru wan ist, siu alla iru libnara santa (ex eo, quod)\\
%  \\
%  `hæc autem ex eo quod deest illi, totum victum suum quem habuit misit.' \flushfill{\ac{ohg}, Tatian 118,1, \citealt[761]{behaghel1923}}
% \bg. thaz iru thiu sin guati nirzigi, thes siu bati\\
%  \\
%  `' \flushfill{\ac{ohg}, Otfrid II,8,24, \citealt[761]{behaghel1923}}
\bg. thia laz ih themo iz lisit thar\\
 \\
 `' \flushfill{\ac{ohg}, Otfrid I,19,25, \citealt[761]{behaghel1923}}
% \bg. noh so neduohti in gnuoge des si habetin\\
%  \\
%  `' \flushfill{\ac{ohg}, Notker I,63,29, \citealt[761]{behaghel1923}}
% \bg. tannoh pito ih tes noh fore ist (id quod)\\
%  \\
%  `' \flushfill{\ac{ohg}, Notker 193,19, \citealt[761]{behaghel1923}}

So, to sum up:



\begin{table}[H]
  \center
  \caption {Case attraction in headless relatives in \ac{ohg}}
    \begin{tabular}{c|c|c|c}
			\toprule
				\diagbox[linecolor=white]{\ac{int}}{\ac{ext}}
						& [\ac{nom}]
						& [\ac{acc}]
						& [\ac{dat}]
						\\ \cmidrule{1-4}
				[\ac{nom}]
						& \colorbox{LG}{\ac{nom}}
						& \diagbox[linecolor=white]{*\ac{nom}}{\colorbox{DG}{\ac{acc}}}
						& \diagbox[linecolor=white]{*\ac{nom}}{\colorbox{DG}{\ac{dat}}}
						\\ \cmidrule{1-4}
				[\ac{acc}]
						& \diagbox[linecolor=white]{*\ac{acc}}{*\ac{nom}}
						&	\colorbox{LG}{\ac{acc}}
						&	\diagbox[linecolor=white]{*\ac{acc}}{\colorbox{DG}{\ac{dat}}}
						\\ \cmidrule{1-4}
				[\ac{dat}]
						& \diagbox[linecolor=white]{*\ac{dat}}{*\ac{nom}}
						&	\diagbox[linecolor=white]{*\ac{dat}}{*\ac{acc}}
						& \colorbox{LG}{\ac{dat}}
						\\
			\bottomrule
    \end{tabular}
\end{table}




\subsection{Only from internal: Modern German}

\ex. \ac{int}:\ac{nom}, \ac{ext}:\ac{acc}
\ag. *Ich {lade ein}, [wer mir sympathisch ist].\\
 I invite\scsub{[acc]} who.\ac{nom} me nice is\scsub{[nom]}\\
 `I invite who I like.' \flushfill{\pgcitealt{vogel2001}{344}}
\bg. *Ich {lade ein}, [wen mir sympathisch ist].\\
 I invite\scsub{[acc]} who.\ac{acc} me nice is\scsub{[nom]}\\
 `I invite who I like.' \flushfill{\pgcitealt{vogel2001}{344}}

\ex. \ac{int}:\ac{nom}, \ac{ext}:\ac{dat}
\ag. *Ich vertraue, [wer Hitchcock mag].\\
 I trust\scsub{[dat]} who.\ac{nom} Hitchcock likes\scsub{[nom]}\\
 `I trust who likes Hitchcock.' \flushfill{\pgcitealt{vogel2001}{345}}
\bg. *Ich vertraue, [wem Hitchcock mag].\\
 I trust\scsub{[dat]} who.\ac{dat} Hitchcock likes\scsub{[nom]}\\
 `I trust who likes Hitchcock.' \flushfill{\pgcitealt{vogel2001}{345}}

\ex. \ac{int}:\ac{acc}, \ac{ext}:\ac{nom}
\ag. Uns besucht [wen Maria mag].\\
 Us visits\scsub{[nom]} who.\ac{acc} Maria.\ac{nom} likes\scsub{[acc]}\\
 `Who visits us likes Maria likes.' \flushfill{\pgcitealt{vogel2001}{343}}
\bg. *Uns besucht [wer Maria mag].\\
 Us visits\scsub{[nom]} who.\ac{nom} Maria.\ac{nom} likes\scsub{[acc]}\\
 `Who visits us likes Maria likes.' \flushfill{\pgcitealt{vogel2001}{343}}

 \ex. \ac{int}:\ac{acc}, \ac{ext}:\ac{dat}
\ag. *Ich vertraue [wem auch Maria mag]. \\
 I trust\scsub{[dat]} who.\ac{dat} also Maria likes\scsub{[acc]}.\\
 `I trust whoever Maria also likes.' \flushfill{\pgcitealt{vogel2001}{345}}
\bg. *Ich vertraue [wen auch Maria mag]. \\
 I trust\scsub{[dat]} who.\ac{acc} also Maria likes\scsub{[acc]}.\\
 `I trust whoever Maria also likes.' \flushfill{\pgcitealt{vogel2001}{345}}

\ex. \ac{int}:\ac{dat}, \ac{ext}:\ac{nom}
\ag. Uns besucht [wem Maria vertraut].\\
 us visits\scsub{[nom]} who.\ac{dat} Maria trusts\scsub{[dat]}\\
 `Who visits us, Maria trusts.' \flushfill{\pgcitealt{vogel2001}{343}}
\bg. *Uns besucht [wer Maria vertraut].\\
 us visits\scsub{[nom]} who.\ac{nom} Maria trusts\scsub{[dat]}\\
 `Who visits us, Maria trusts.' \flushfill{\pgcitealt{vogel2001}{343}}

\ex. \ac{int}:\ac{dat}, \ac{ext}:\ac{acc}
\ag. Ich {lade ein} [wem auch Maria vertraut]. \\
 I invite\scsub{[acc]} who.\ac{dat} also Maria trusts\scsub{[dat]}.\\
 `I invite whoever Maria also trusts.' \flushfill{\pgcitealt{vogel2001}{344}}
\bg. *Ich {lade ein} [wen auch Maria vertraut]. \\
 I invite\scsub{[acc]} who.\ac{acc} also Maria trusts\scsub{[dat]}.\\
 `I invite whoever Maria also trusts.' \flushfill{\pgcitealt{vogel2001}{344}}


 \begin{table}[H]
   \center
   \caption {Case attraction in headless relatives in \ac{mg}}
     \begin{tabular}{c|c|c|c}
			 \toprule
				 \diagbox[linecolor=white]{\ac{int}}{\ac{ext}}
						 & [\ac{nom}]
						 & [\ac{acc}]
						 & [\ac{dat}]
						 \\ \cmidrule{1-4}
				 [\ac{nom}]
						 & \colorbox{LG}{\ac{nom}}
						 & \diagbox[linecolor=white]{*\ac{nom}}{*\ac{acc}}
						 & \diagbox[linecolor=white]{*\ac{nom}}{*\ac{dat}}
						 \\ \cmidrule{1-4}
				 [\ac{acc}]
						 & \diagbox[linecolor=white]{\colorbox{DG}{\ac{acc}}}{*\ac{nom}}
						 &	\colorbox{LG}{\ac{acc}}
						 &	\diagbox[linecolor=white]{*\ac{acc}}{*\ac{dat}}
						 \\ \cmidrule{1-4}
				 [\ac{dat}]
						 & \diagbox[linecolor=white]{\colorbox{DG}{\ac{dat}}}{*\ac{nom}}
						 &	\diagbox[linecolor=white]{\colorbox{DG}{\ac{dat}}}{*\ac{acc}}
						 & \colorbox{LG}{\ac{dat}}
						 \\
			 \bottomrule
     \end{tabular}
 \end{table}

\subsection{None: Italian}




\section{Shape of relative pronoun}

\begin{table}[H]
	\center
	\caption {Shape of relative pronoun per language}
		\begin{tabular}{ccc}
		\toprule
							& rel pron in headless rel	& rel prons in light-headed rel		\\
		\midrule
		Gothic		& \tsc{a} +\tsc{c}					&	\tsc{a} + \tsc{a} + \tsc{c}			\\
		\ac{ohg}	&	\tsc{a} 									&	\tsc{a} + \tsc{a} 							\\
		\ac{mg}		& \tsc{b} 									&	\tsc{a} + \tsc{a} 							\\
		Italian		& \tsc{b} 									&	\tsc{a} + \tsc{b} 							\\
		\bottomrule
	\end{tabular}
\end{table}

\subsection{Gothic}

\subsubsection{Headless relatives}

\tsc{d} + \tsc{comp}

\begin{table}[H]
	\center
	\caption {Relative pronouns in headless relatives in Gothic}
		\begin{tabular}{cccc}
		\toprule
							& \ac{n}.\ac{sg} 	& \ac{m}.\ac{sg}	& \ac{f}.\ac{sg}  \\
		 						\cmidrule{2-4}
    \ac{nom} 	& þ-at-ei 	 			& s-a-ei 					& s-ō-ei					\\
    \ac{acc}	& þ-at-ei    			& þ-an-ei  				& þ-ō-ei  				\\
    \ac{dat} 	& þ-amm-ei 				& þ-amm-ei				& þ-izái-ei 			\\
		\bottomrule
    					& \ac{n}.\ac{pl}	& \ac{m}.\ac{pl}	& \ac{f}.\ac{pl}	\\
						    \cmidrule{2-4}
    \ac{nom} 	& þ-ō-ei					&	þ-ái-ei					&	þ-ōz-ei					\\
    \ac{acc} 	& þ-ō-ei 					&	þ-anz-ei				&	þ-ōz-ei					\\
    \ac{dat} 	& þ-áim-ei				&	þ-áim-ei 				&	þ-áim-ei 				\\
    \bottomrule
		\end{tabular}
\end{table}

\subsubsection{Light-headed relatives}

\tsc{d}, \tsc{d} + \tsc{comp}



\subsection{Old High German}

\subsubsection{Headless relatives}

\tsc{d}

\begin{table}[H]\label{tbl:paradigmohg}
	\center
	\caption {Relative pronouns in headless relatives in \ac{ohg}}
		\begin{tabular}{cccc}
		\toprule
							& \ac{n}.\ac{sg}	& \ac{m}.\ac{sg}  & \ac{f}.\ac{sg}	\\
								\cmidrule{2-4}
		\ac{nom}	& d-aȥ          	& d-ër       			& d-iu						\\
		\ac{acc}	& d-aȥ   					& d-ën						& d-ea/-ia/(-ie)	\\
		\ac{dat}	& d-ëmu/-ëmo	    & d-ëmu/-ëmo			& d-ëru/-ëro			\\
		\bottomrule
	    				& \ac{n}.\ac{pl}	& \ac{m}.\ac{pl}  	& \ac{f}.\ac{pl}	\\
	    					\cmidrule{2-4}
    \ac{nom} 	& d-iu/-ei   			&  d-ē/-ea/-ia/-ie	& d-eo/-io        \\
    \ac{acc} 	& d-iu/-ei   			&  d-ē/-ea/-ia/-ie	& d-eo/-io        \\
    \ac{dat} 	& d-ēm/-ēn   			&  d-ēm/-ēn       	& d-ēm/-ēn        \\
    \bottomrule
		\end{tabular}
\end{table}

\subsubsection{Light-headed relatives}

\tsc{d}, \tsc{d}

Wouldn't we now not expect that Modern German patterns with Old High German wrt attraction in headed constructions. Yes, we would. And yes, this is exactly what we see. Paper by Bader on case attraction.




\subsection{Modern German}

\subsubsection{Headless relatives}

\tsc{wh}

\begin{table}[H]
	\center
	\caption {Relative pronouns in headless relatives in \ac{mg}}
		\begin{tabular}{ccc}
		\toprule
							& \ac{inan}	& \ac{an}	\\
								\cmidrule{2-3}
    \ac{nom} 	& w-as    	& w-er   	\\
    \ac{acc} 	& w-as    	& w-en  	\\
    \ac{dat} 	& -  				& w-em   	\\
		\bottomrule
		\end{tabular}
\end{table}

\subsubsection{Light-headed relatives}

Pattern in light-headed relatives: \tsc{d}, \tsc{d}




\subsection{Italian}

\subsubsection{Headless relatives}

\tsc{wh}: \tit{che}

\subsubsection{Light-headed relatives}

\tsc{d}, \tsc{wh}: \tit{quello}, \tit{che}


\section{Bringing this together}

\begin{table}[H]
	\center
	\caption {Variation and relative pronoun shape}
		\begin{tabular}{ccccc}
		\toprule
							&	rel pron in headless rel	& rel prons in light-headed rel		& \ac{int}>\ac{ext}		& \ac{ext}>\ac{int}	\\
		\midrule
		Gothic 		& \tsc{a} +\tsc{c}					&	\tsc{a} + \tsc{a} + \tsc{c}			& ✔										&	✔									\\
		\ac{ohg}	& \tsc{a} 									&	\tsc{a} + \tsc{a} 							& *										&	✔									\\
		\ac{mg}		&	\tsc{b} 									&	\tsc{a} + \tsc{a} 							& ✔										&	*									\\
		Italian		& \tsc{b} 									&	\tsc{a} + \tsc{b} 							& *										&	*									\\
		\bottomrule
	\end{tabular}
\end{table}

And how can we now derive this?
