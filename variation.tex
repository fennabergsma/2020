% !TEX root = thesis.tex

\chapter{The variation}

\section{\ac{ohg} and \ac{mg} have the same winner}

\subsection{\ac{mg}}

let me show that the claim I made for Gothic holds for \ac{mg} as well: \tsc{dat} wins over \tsc{acc} wins over \tsc{nom}.

Examples in which the internal case is dative, the external case is accusative and the relative pronoun appears in accusative case is ungrammatical.

\exg. *Ich {lade ein} wen \tbf{auch} \tbf{Maria} \tbf{vertraut}. \\
 I invite\scsub{[acc]} \tsc{rel}.\ac{acc}.\tsc{an} also Maria trusts\scsub{[dat]}.\\
 `I invite whoever Maria also trusts.' \flushfill{\pgcitealt{vogel2001}{344}}\label{ex:mg-acc-dat-u}

Examples in which the internal case is dative, the external case is accusative and the relative pronoun appears in accusative case is ungrammatical.

\exg. *Uns besucht wer \tbf{Maria} \tbf{vertraut}.\\
 us visits\scsub{[nom]} \tsc{rel}.\ac{nom}.\tsc{an} Maria trusts\scsub{[dat]}\\
 `Who visits us, Maria trusts.' \flushfill{\pgcitealt{vogel2001}{343}}

Examples in which the internal case is accusative, the external case is nominative and the relative pronoun appears in nominative case is ungrammatical.

\exg. *Uns besucht wer \tbf{Maria} \tbf{mag}.\\
 Us visits\scsub{[nom]} \tsc{rel}.\ac{nom}.\tsc{an} Maria.\ac{nom} likes\scsub{[acc]}\\
 `Who visits us likes Maria likes.' \flushfill{\pgcitealt{vogel2001}{343}}\label{ex:mg-nom-acc-u}

Examples in which the internal case is accusative, the external case is dative and the relative pronoun appears in accusative case is ungrammatical.

\exg. *Ich vertraue \tbf{wen} \tbf{auch} \tbf{Maria} \tbf{mag}. \\
 I trust\scsub{[dat]} \tsc{rel}.\ac{acc}.\tsc{an} also Maria likes\scsub{[acc]}.\\
 `I trust whoever Maria also likes.' \flushfill{\pgcitealt{vogel2001}{345}}

Examples in which the internal case is nominative, the external case is dative and the relative pronoun appears in nominative case is ungrammatical.

\exg. *Ich vertraue, \tbf{wer} \tbf{Hitchcock} \tbf{mag}.\\
 I trust\scsub{[dat]} \tsc{rel}.\ac{nom}.\tsc{an} Hitchcock likes\scsub{[nom]}\\
 `I trust who likes Hitchcock.' \flushfill{\pgcitealt{vogel2001}{345}}

Examples in which the internal case is nominative, the external case is accusative and the relative pronoun appears in nominative case is ungrammatical.

\exg. *Ich {lade ein}, \tbf{wer} \tbf{mir} \tbf{sympathisch} \tbf{ist}.\\
 I invite\scsub{[acc]} \tsc{rel}.\ac{nom}.\tsc{an} me nice is\scsub{[nom]}\\
 `I invite who I like.' \flushfill{\pgcitealt{vogel2001}{344}}




\subsection{\ac{ohg}}

same holds for \ac{ohg} but you just can't give the ungrammatical examples. but they are not attested.



\section{The different patterns}

In Gothic, the more complex case wins.
In \ac{ohg}, the more complex case wins, only if it is external.
In \ac{mg}, the more complex case wins, only if it is internal.
In Italian, case mismatch is not allowed.


\begin{table}[H]
	\center
	\caption {Variation}
		\begin{tabular}{ccc}
		\toprule
		 					& \ac{int}>\ac{ext}		& \ac{ext}>\ac{int}	\\
								\cmidrule{2-3}
		\ac{mg} 	& ✔			 							&	*									\\
		\ac{ohg}	& *										&	✔									\\
		Gothic		&	✔										&	✔									\\
		\bottomrule
		\end{tabular}
\end{table}



\subsection{Both: Gothic}


% !TEX root = ../thesis.tex

\begin{tabular}{c|c|c|c}
  \toprule
      \textsubscript{\ac{int}} \textsuperscript{\ac{ext}}
        & [\ac{nom}]
        & [\ac{acc}]
        & [\ac{dat}]
        \\ \cmidrule{1-4}
    [\ac{nom}]
        & \ac{nom}
        & \ac{acc}
        & \ac{dat}
        \\ \cmidrule{1-4}
    [\ac{acc}]
        & \ac{acc}
        & \ac{acc}
        & \ac{dat}
        \\ \cmidrule{1-4}
    [\ac{dat}]
        & \ac{dat}
        & (\ac{dat})
        & \ac{dat}
        \\
  \bottomrule
\end{tabular}




\subsection{Only from internal: Modern German}

\subsubsection{external case = ungrammatical}

\ac{int}:\ac{nom}, \ac{ext}:\ac{acc}

\exg. *Ich {lade ein}, wen \tbf{mir} \tbf{sympathisch} \tbf{ist}.\\
 I invite\scsub{[acc]} \tsc{rel}.\ac{acc}.\tsc{an} me nice is\scsub{[nom]}\\
 `I invite who I like.' \flushfill{\pgcitealt{vogel2001}{344}}

\ac{int}:\ac{nom}, \ac{ext}:\ac{dat}

\exg. *Ich vertraue, wem \tbf{Hitchcock} \tbf{mag}.\\
 I trust\scsub{[dat]} \tsc{rel}.\ac{dat}.\tsc{an} Hitchcock likes\scsub{[nom]}\\
 `I trust who likes Hitchcock.' \flushfill{\pgcitealt{vogel2001}{345}}

\ac{int}:\ac{acc}, \ac{ext}:\ac{dat}

\exg. *Ich vertraue wem \tbf{auch} \tbf{Maria} \tbf{mag}. \\
 I trust\scsub{[dat]} \tsc{rel}.\ac{dat}.\tsc{an} also Maria likes\scsub{[acc]}.\\
 `I trust whoever Maria also likes.' \flushfill{\pgcitealt{vogel2001}{345}}



\subsubsection{internal case = grammatical}

Consider the example in \ref{ex:mg-acc-dat}. In this example, the internal case is dative and the external case is accusative.
The internal case is dative. The predicate \tit{vertraut} `trusts' takes dative objects.
The external case is accusative. The predicate \tit{lade ein} `invite' takes accusative objects.
The relative pronoun \tit{wem} `\tsc{rel}.\ac{dat}.\tsc{an}' appears in the internal case: the dative. The relative pronoun is marked in bold, just like as the relative clause, showing that the relative pronoun patterns with the relative clause.

\exg. Ich {lade ein} \tbf{wem} \tbf{auch} \tbf{Maria} \tbf{vertraut}. \\
 I invite\scsub{[acc]} \tsc{rel}.\ac{dat}.\tsc{an} also Maria trusts\scsub{[dat]}.\\
 `I invite whoever Maria also trusts.' \flushfill{\pgcitealt{vogel2001}{344}}\label{ex:mg-acc-dat}

Consider the example in \ref{ex:mg-nom-dat}. In this example, the internal case is dative and the external case is nominative.
The internal case is dative. The predicate \tit{vertraut} `trusts' takes dative objects.
The external case is nominative. The predicate \tit{besucht} `visits' takes nominative subjects.
The relative pronoun \tit{wem} `\tsc{rel}.\ac{dat}.\tsc{an}' appears in the internal case: the dative. The relative pronoun is marked in bold, just like as the relative clause, showing that the relative pronoun patterns with the relative clause.

\exg. Uns besucht \tbf{wem} \tbf{Maria} \tbf{vertraut}.\\
 us visits\scsub{[nom]} \tsc{rel}.\ac{dat}.\tsc{an} Maria trusts\scsub{[dat]}\\
 `Who visits us, Maria trusts.' \flushfill{\pgcitealt{vogel2001}{343}}\label{ex:mg-nom-dat}

 Consider the example in \ref{ex:mg-nom-acc}. In this example, the internal case is accusative and the external case is nominative.
 The internal case is accusative. The predicate \tit{mag} `likes' takes accusative objects.
 The external case is nominative. The predicate \tit{besucht} `visits' takes nominative subjects.
 The relative pronoun \tit{wen} `\tsc{rel}.\ac{acc}.\tsc{an}' appears in the internal case: the accusative. The relative pronoun is marked in bold, just like as the relative clause, showing that the relative pronoun patterns with the relative clause.

 \exg. Uns besucht \tbf{wen} \tbf{Maria} \tbf{mag}.\\
  Us visits\scsub{[nom]} \tsc{rel}.\ac{acc}.\tsc{an} Maria.\ac{nom} likes\scsub{[acc]}\\
  `Who visits us likes Maria likes.' \flushfill{\pgcitealt{vogel2001}{343}}\label{ex:mg-nom-acc}


 \begin{table}[H]
   \center
   \caption {Summary of \ac{mg} matching headless relative data}
 		\begin{tabular}{c|c|c|c}
 		  \toprule
 			\textsubscript{\ac{int}} \textsuperscript{\ac{ext}}
 		        & [\ac{nom}]
 		        & [\ac{acc}]
 		        & [\ac{dat}]
 		        \\ \cmidrule{1-4}
 		    [\ac{nom}]
 		        &
 		        &
 		        &
 		        \\ \cmidrule{1-4}
 		    [\ac{acc}]
 		        & \ac{acc}
 		        &
 		        &
 		        \\ \cmidrule{1-4}
 		    [\ac{dat}]
 		        & \ac{dat}
 		        & \ac{dat}
 		        &
 		        \\
 		  \bottomrule
 		\end{tabular}
 \end{table}


 \subsection{Only from external: Old High German}


  \subsubsection{external is grammatical}


Consider the example in \ref{ex:ohg-dat-nom}. In this example, the internal case is accusative and the external case is dative.
The internal case is nominative. The predicate \tit{zalta} `named' takes accusative objects.
The external case is dative. The predicate \tit{furira} `superior' takes dative objects.
The relative pronoun \tit{thên} `\ac{rel}.\ac{dat}.\ac{pl}' appears in the external case: the dative. The relative pronoun is not marked in bold, just like as the main clause, showing that the relative pronoun patterns with the main clause.
Examples in which the internal case is accusative, the external case is dative and the relative pronoun appears in accusative case are unattested.

\exg. istû furira Abrâhame, ouh thên \tbf{man} \tbf{hiar} \tbf{nû} \tbf{zalta}?\\
 {are you} superior\scsub{[dat]} {to Abraham} also \tsc{rel}.\ac{dat}.\tsc{pl} one here now named\scsub{[acc]}\\
 `are you superior to Abraham to those which they just mentioned?' \flushfill{\ac{ohg}, \ac{otfrid} III 18:33, \pgcitealt{behaghel1923}{761}}\label{ex:ohg-dat-acc}

Consider the example in \ref{ex:ohg-dat-nom}. In this example, the internal case is nominative and the external case is dative.
The internal case is nominative. The predicate \tit{sprah} `spoke' takes nominative subjects.
The external case is dative. The predicate \tit{antuurta} `replied' takes dative objects.
The relative pronoun \tit{demo} `\ac{rel}.\ac{dat}.\ac{m}.\ac{sg}' appears in the external case: the dative. The relative pronoun is not marked in bold, just like as the main clause, showing that the relative pronoun patterns with the main clause.
Examples in which the internal case is nominative, the external case is dative and the relative pronoun appears in nominative case are unattested.

\exg. aer antuurta demo \tbf{zaimo} \tbf{sprah}\\
 he replied\scsub{[dat]} \tsc{rel}.\ac{dat}.\tsc{m}.\tsc{sg} {to him} spoke\scsub{[nom]}\\
 `he replied to the one who spoke to him' \flushfill{\ac{ohg}, \ac{mons} 7:24, \pgcitealt{behaghel1923}{761}, after \pgcitealt{pittner1995}{199}}\label{ex:ohg-dat-nom}

Consider the example in \ref{ex:ohg-acc-nom}. In this example, the internal case is nominative and the external case is accusative.
The internal case is nominative. The predicate \tit{chisetzit} `pull through' takes nominative subjects.
The external case is accusative. The predicate \tit{bibringu} `educate' takes accusative objects.
The relative pronoun \tit{dhen} `\ac{rel}.\ac{acc}.\ac{m}.\ac{sg}' appears in the external case: the accusative. The relative pronoun is not marked in bold, just like as the main clause, showing that the relative pronoun patterns with the main clause.
Examples in which the internal case is nominative, the external case is accusative and the relative pronoun appears in nominative case are unattested.

\exg. ih bibringu fona Juda dhen \tbf{mina} \tbf{berga} \tbf{chisetzit}\\
 I educate\scsub{[acc]} about Juda \tsc{rel}.\ac{acc}.\tsc{m}.\tsc{sg} my mountains {through pull}\scsub{[nom]}\\
 `I educate the one who wanders through my mountains about Judas' \flushfill{\ac{ohg}, \ac{isid} 34:3, \pgcitealt{behaghel1923}{761}}\label{ex:ohg-acc-nom}

  \exg. gaat uz diu halt za dem iz forchaufent\\
   \\
   `' \flushfill{\ac{ohg}, Monsee Fragments 20,14, \citealt[761]{behaghel1923}}
  % \bg. thisiu fon thiu, iru wan ist, siu alla iru libnara santa (ex eo, quod)\\
  %  \\
  %  `hæc autem ex eo quod deest illi, totum victum suum quem habuit misit.' \flushfill{\ac{ohg}, Tatian 118,1, \citealt[761]{behaghel1923}}
  % \bg. thaz iru thiu sin guati nirzigi, thes siu bati\\
  %  \\
  %  `' \flushfill{\ac{ohg}, Otfrid II,8,24, \citealt[761]{behaghel1923}}


  \exg. thia laz ih themo iz lisit thar\\
   \\
   `' \flushfill{\ac{ohg}, Otfrid I,19,25, \citealt[761]{behaghel1923}}
  % \bg. noh so neduohti in gnuoge des si habetin\\
  %  \\
  %  `' \flushfill{\ac{ohg}, Notker I,63,29, \citealt[761]{behaghel1923}}
  % \bg. tannoh pito ih tes noh fore ist (id quod)\\
  %  \\
  %  `' \flushfill{\ac{ohg}, Notker 193,19, \citealt[761]{behaghel1923}}





  \subsubsection{internal is ungrammatical}

  \ex. \ac{int}:\ac{acc}, \ac{ext}:\ac{nom}
  \a. \ac{acc} not attested
  \b. \ac{nom} not attested

 \ex. \ac{int}:\ac{dat}, \ac{ext}:\ac{nom}
 \a. \ac{dat} not attested
 \b. \ac{nom} not attested

 \ex. \ac{int}:\ac{dat}, \ac{ext}:\ac{acc}
 \a. \ac{dat} not attested
 \b. \ac{acc} not attested


 Don't know:

 So, to sum up:





 \begin{table}[H]
   \center
   \caption {Summary of \ac{ohg} matching headless relative data}
 		\begin{tabular}{c|c|c|c}
 		  \toprule
 			\textsubscript{\ac{int}} \textsuperscript{\ac{ext}}
 		        & [\ac{nom}]
 		        & [\ac{acc}]
 		        & [\ac{dat}]
 		        \\ \cmidrule{1-4}
 		    [\ac{nom}]
 		        &
 		        & \ac{acc}
 		        & \ac{dat}
 		        \\ \cmidrule{1-4}
 		    [\ac{acc}]
 		        &
 		        &
 		        & \ac{dat}
 		        \\ \cmidrule{1-4}
 		    [\ac{dat}]
 		        &
 		        &
 		        &
 		        \\
 		  \bottomrule
 		\end{tabular}
 \end{table}




\section{Shape of relative pronoun}

\begin{table}[H]
	\center
	\caption {Shape of relative pronoun per language}
		\begin{tabular}{ccc}
		\toprule
							& rel pron in headless rel	& rel prons in light-headed rel		\\
		\midrule
		Gothic		& \tsc{a} +\tsc{c}					&	\tsc{a} + \tsc{a} + \tsc{c}			\\
		\ac{ohg}	&	\tsc{a} 									&	\tsc{a} + \tsc{a} 							\\
		\ac{mg}		& \tsc{b} 									&	\tsc{a} + \tsc{a} 							\\
		\bottomrule
	\end{tabular}
\end{table}

\subsection{Gothic}

\subsubsection{Headless relatives}

\tsc{d} + \tsc{comp}

\begin{table}[H]
	\center
	\caption {Relative pronouns in headless relatives in Gothic}
		\begin{tabular}{cccc}
		\toprule
							& \ac{n}.\ac{sg} 	& \ac{m}.\ac{sg}	& \ac{f}.\ac{sg}  \\
		 						\cmidrule{2-4}
    \ac{nom} 	& þ-at-ei 	 			& s-a-ei 					& s-ō-ei					\\
    \ac{acc}	& þ-at-ei    			& þ-an-ei  				& þ-ō-ei  				\\
    \ac{dat} 	& þ-amm-ei 				& þ-amm-ei				& þ-izái-ei 			\\
		\bottomrule
    					& \ac{n}.\ac{pl}	& \ac{m}.\ac{pl}	& \ac{f}.\ac{pl}	\\
						    \cmidrule{2-4}
    \ac{nom} 	& þ-ō-ei					&	þ-ái-ei					&	þ-ōz-ei					\\
    \ac{acc} 	& þ-ō-ei 					&	þ-anz-ei				&	þ-ōz-ei					\\
    \ac{dat} 	& þ-áim-ei				&	þ-áim-ei 				&	þ-áim-ei 				\\
    \bottomrule
		\end{tabular}
\end{table}

\subsubsection{Light-headed relatives}

\tsc{d}, \tsc{d} + \tsc{comp}



\subsection{Old High German}

\subsubsection{Headless relatives}

\tsc{d}

\begin{table}[H]\label{tbl:paradigmohg}
	\center
	\caption {Relative pronouns in headless relatives in \ac{ohg}}
		\begin{tabular}{cccc}
		\toprule
							& \ac{n}.\ac{sg}	& \ac{m}.\ac{sg}  & \ac{f}.\ac{sg}	\\
								\cmidrule{2-4}
		\ac{nom}	& d-aȥ          	& d-ër       			& d-iu						\\
		\ac{acc}	& d-aȥ   					& d-ën						& d-ea/-ia/(-ie)	\\
		\ac{dat}	& d-ëmu/-ëmo	    & d-ëmu/-ëmo			& d-ëru/-ëro			\\
		\bottomrule
	    				& \ac{n}.\ac{pl}	& \ac{m}.\ac{pl}  	& \ac{f}.\ac{pl}	\\
	    					\cmidrule{2-4}
    \ac{nom} 	& d-iu/-ei   			&  d-ē/-ea/-ia/-ie	& d-eo/-io        \\
    \ac{acc} 	& d-iu/-ei   			&  d-ē/-ea/-ia/-ie	& d-eo/-io        \\
    \ac{dat} 	& d-ēm/-ēn   			&  d-ēm/-ēn       	& d-ēm/-ēn        \\
    \bottomrule
		\end{tabular}
\end{table}

\subsubsection{Light-headed relatives}

\tsc{d}, \tsc{d}

Wouldn't we now not expect that Modern German patterns with Old High German wrt attraction in headed constructions. Yes, we would. And yes, this is exactly what we see. Paper by Bader on case attraction.




\subsection{Modern German}

\subsubsection{Headless relatives}

\tsc{wh}

\begin{table}[H]
	\center
	\caption {Relative pronouns in headless relatives in \ac{mg}}
		\begin{tabular}{ccc}
		\toprule
							& \ac{inan}	& \ac{an}	\\
								\cmidrule{2-3}
    \ac{nom} 	& w-as    	& w-er   	\\
    \ac{acc} 	& w-as    	& w-en  	\\
    \ac{dat} 	& -  				& w-em   	\\
		\bottomrule
		\end{tabular}
\end{table}

\subsubsection{Light-headed relatives}

Pattern in light-headed relatives: \tsc{d}, \tsc{d}





\section{Bringing this together}

\begin{table}[H]
	\center
	\caption {Variation and relative pronoun shape}
		\begin{tabular}{ccccc}
		\toprule
							&	rel pron in headless rel	& rel prons in light-headed rel		& \ac{int}>\ac{ext}		& \ac{ext}>\ac{int}	\\
		\midrule
		Gothic 		& \tsc{a} +\tsc{c}					&	\tsc{a} + \tsc{a} + \tsc{c}			& ✔										&	✔									\\
		\ac{ohg}	& \tsc{a} 									&	\tsc{a} + \tsc{a} 							& *										&	✔									\\
		\ac{mg}		&	\tsc{b} 									&	\tsc{a} + \tsc{a} 							& ✔										&	*									\\
		\bottomrule
	\end{tabular}
\end{table}

And how can we now derive this?





\section{Two points: all or nothing}

\subsection{No matches work}

Italian doesnt allow any of them, because it has \tit{d, wh} as light headed relative?


\subsection{All allow for matching ones (and syncretic ones! whuut)}

First, I discuss the matching headless relatives, in which the internal and external case match.

Consider the example in \ref{ex:gothicaccaccrep}, repeated from the introduction. In this example, the internal case and the external case are accusative.
The relative clause, including the relative pronoun, is marked in gray.
The internal case is accusative. The predicate \tit{arma} `pity' takes accusative objects.
The external case is accusative as well. Here the predicate \tit{gaarma} `pity' takes accusative objects.
The relative pronoun \tit{þan(a)} `who.\ac{acc}' appears in the accusative.

\exg. gaarma \tcol{DG}{þan} \tcol{DG}{-ei} \tcol{DG}{arma}\\
 pity\scsub{[acc]} who.\ac{acc} -\ac{comp} pity\scsub{[acc]}\\
 `I will pity (him) whom I pity' \flushfill{Gothic, \ac{rom} 9:15, after \pgcitealt{harbert1978}{339}}\label{ex:gothicaccaccrep}

Consider the example in \ref{ex:gothicnomnom}, in which the internal case and the external case are nominative.
The relative clause, including the relative pronoun, is marked in gray.
The internal case is nominative. The predicate \tit{matjai} `eats' takes nominative subjects.
The external case is nominative as well. Here the predicate \tit{gadauþnai} `die' takes nominative subjects.
The relative pronoun \tit{sa} `who.\ac{nom}' appears in the nominative.

\exg. ei \tcol{DG}{sa} \tcol{DG}{-ei} \tcol{DG}{þis} \tcol{DG}{matjai}, ni gadauþnai\\
 that who.\ac{nom} -\ac{comp} {of this} eats\scsub{[nom]} not die\scsub{[nom]}\\
 `that (he) who eats of this may not die' \flushfill{Gothic, \ac{john} 6:50, after \pgcitealt{harbert1978}{337}}\label{ex:gothicnomnom}

% or is this one better?
% \exg. saei sokeib saiwala sema ganasjan, fraqisteib izai\\
%  who seeks soul his save loses it\\
%  `Who seeks to save his soul loses it.'\flushfill{Luk 17:33}

Consider the examples in \ref{ex:gothicdatdat}, in which the internal case and the external case are dative.
The relative clauses, including the relative pronoun, is marked in gray.
The internal case is dative. The predicates \tit{gabaur} `tribute', \tit{mota} `custom', \tit{agis} `fear' and \tit{sweriþa} `honour' takes dative objects.
The external case is dative as well. The same predicates as in the relative clause take dative objects.
The relative pronouns \tit{þamm(a)} `who.\ac{dat}' appear in the dative.

\ex.\label{ex:gothicdatdat}
\ag. \tcol{DG}{þamm} \tcol{DG}{-ei} \tcol{DG}{gabaur} gabaur\\
 who.\ac{dat} -\ac{comp} tribute\scsub{[dat]} tribute\scsub{[dat]}\\
 `tribute to (him) whom tribute is due'
\bg. \tcol{DG}{þamm} \tcol{DG}{-ei} \tcol{DG}{mota} mota\\
 \tcol{DG}{who.\ac{dat}} \tcol{DG}{-\ac{comp}} \tcol{DG}{custom\scsub{[dat]}} custom\scsub{[dat]}\\
 `custom to (him) whom custom is due'
\bg. \tcol{DG}{þamm} \tcol{DG}{-ei} \tcol{DG}{agis} agis\\
 \tcol{DG}{who.\ac{dat}} \tcol{DG}{-\ac{comp}} \tcol{DG}{fear\scsub{[dat]}} fear\scsub{[dat]}\\
 `fear (him) whom fear is due'
\bg. \tcol{DG}{þamm} \tcol{DG}{-ei} \tcol{DG}{sweriþa} sweriþa\\
 \tcol{DG}{who.\ac{dat}} \tcol{DG}{-\ac{comp}} \tcol{DG}{honour\scsub{[dat]}} honour\scsub{[dat]}\\
 `honour (him) whom honour is due' \flushfill{Gothic, \ac{rom} 13:7, after \pgcitealt{harbert1978}{339}}

 So far only the diagonal line is filled. These are the matching examples, the examples in which the internal case matches the external case. The relative pronoun appears in the case which is the internal and external case. The nominative is given in \ref{ex:gothicnomnom}, the accusative in \ref{ex:gothicaccaccrep}, and the dative in \ref{ex:gothicdatdat}.

 \begin{table}[H]
   \center
   \caption {Summary of Gothic matching headless relative data}
     \begin{tabular}{c|c|c|c}
       \toprule
         \diagbox[linecolor=white]{\ac{int}}{\ac{ext}}
             & [\ac{nom}]
             & [\ac{acc}]
             & [\ac{dat}]
             \\ \cmidrule{1-4}
         [\ac{nom}]
             & \ac{nom}
             & \diagbox[linecolor=white]{\phantom{nom}}{\phantom{nom}}
             & \diagbox[linecolor=white]{\phantom{nom}}{\phantom{nom}}
             \\ \cmidrule{1-4}
         [\ac{acc}]
             & \diagbox[linecolor=white]{\phantom{nom}}{\phantom{nom}}
             & \ac{acc}
             & \diagbox[linecolor=white]{\phantom{nom}}{\phantom{nom}}
             \\ \cmidrule{1-4}
         [\ac{dat}]
             & \diagbox[linecolor=white]{\phantom{nom}}{\phantom{nom}}
             & \diagbox[linecolor=white]{\phantom{nom}}{\phantom{nom}}
             & \ac{dat}
             \\
       \bottomrule
     \end{tabular}
     \label{tbl:summarygothicmatch}
 \end{table}
