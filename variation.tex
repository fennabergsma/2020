% !TEX root = thesis.tex

\chapter{The variation}

\section{\ac{ohg} and \ac{mg} have the same winner}

\subsection{\ac{mg}}

let me show that the claim I made for Gothic holds for \ac{mg} as well: \tsc{dat} wins over \tsc{acc} wins over \tsc{nom}.

Examples in which the internal case is dative, the external case is accusative and the relative pronoun appears in accusative case is ungrammatical.

\exg. *Ich {lade ein} wen \tbf{auch} \tbf{Maria} \tbf{vertraut}. \\
 I invite\scsub{[acc]} \tsc{rel}.\ac{acc}.\tsc{an} also Maria trusts\scsub{[dat]}.\\
 `I invite whoever Maria also trusts.' \flushfill{\pgcitealt{vogel2001}{344}}\label{ex:mg-acc-dat-u}

Examples in which the internal case is dative, the external case is accusative and the relative pronoun appears in accusative case is ungrammatical.

\exg. *Uns besucht wer \tbf{Maria} \tbf{vertraut}.\\
 us visits\scsub{[nom]} \tsc{rel}.\ac{nom}.\tsc{an} Maria trusts\scsub{[dat]}\\
 `Who visits us, Maria trusts.' \flushfill{\pgcitealt{vogel2001}{343}}

Examples in which the internal case is accusative, the external case is nominative and the relative pronoun appears in nominative case is ungrammatical.

\exg. *Uns besucht wer \tbf{Maria} \tbf{mag}.\\
 Us visits\scsub{[nom]} \tsc{rel}.\ac{nom}.\tsc{an} Maria.\ac{nom} likes\scsub{[acc]}\\
 `Who visits us likes Maria likes.' \flushfill{\pgcitealt{vogel2001}{343}}\label{ex:mg-nom-acc-u}

Examples in which the internal case is accusative, the external case is dative and the relative pronoun appears in accusative case is ungrammatical.

\exg. *Ich vertraue \tbf{wen} \tbf{auch} \tbf{Maria} \tbf{mag}. \\
 I trust\scsub{[dat]} \tsc{rel}.\ac{acc}.\tsc{an} also Maria likes\scsub{[acc]}.\\
 `I trust whoever Maria also likes.' \flushfill{\pgcitealt{vogel2001}{345}}

Examples in which the internal case is nominative, the external case is dative and the relative pronoun appears in nominative case is ungrammatical.

\exg. *Ich vertraue, \tbf{wer} \tbf{Hitchcock} \tbf{mag}.\\
 I trust\scsub{[dat]} \tsc{rel}.\ac{nom}.\tsc{an} Hitchcock likes\scsub{[nom]}\\
 `I trust who likes Hitchcock.' \flushfill{\pgcitealt{vogel2001}{345}}

Examples in which the internal case is nominative, the external case is accusative and the relative pronoun appears in nominative case is ungrammatical.

\exg. *Ich {lade ein}, \tbf{wer} \tbf{mir} \tbf{sympathisch} \tbf{ist}.\\
 I invite\scsub{[acc]} \tsc{rel}.\ac{nom}.\tsc{an} me nice is\scsub{[nom]}\\
 `I invite who I like.' \flushfill{\pgcitealt{vogel2001}{344}}




\subsection{\ac{ohg}}

same holds for \ac{ohg} but you just can't give the ungrammatical examples. but they are not attested.



\section{The different patterns}

In Gothic, the more complex case wins.
In \ac{ohg}, the more complex case wins, only if it is external.
In \ac{mg}, the more complex case wins, only if it is internal.
In Italian, case mismatch is not allowed.


\begin{table}[H]
	\center
	\caption {Variation}
		\begin{tabular}{ccc}
		\toprule
		 					& \ac{int}>\ac{ext}		& \ac{ext}>\ac{int}	\\
								\cmidrule{2-3}
		\ac{mg} 	& ✔			 							&	*									\\
		\ac{ohg}	& *										&	✔									\\
		Gothic		&	✔										&	✔									\\
		\bottomrule
		\end{tabular}
\end{table}



\subsection{Both: Gothic}


% !TEX root = ../thesis.tex

\begin{tabular}{c|c|c|c}
  \toprule
      \textsubscript{\ac{int}} \textsuperscript{\ac{ext}}
        & [\ac{nom}]
        & [\ac{acc}]
        & [\ac{dat}]
        \\ \cmidrule{1-4}
    [\ac{nom}]
        & \ac{nom}
        & \ac{acc}
        & \ac{dat}
        \\ \cmidrule{1-4}
    [\ac{acc}]
        & \ac{acc}
        & \ac{acc}
        & \ac{dat}
        \\ \cmidrule{1-4}
    [\ac{dat}]
        & \ac{dat}
        & (\ac{dat})
        & \ac{dat}
        \\
  \bottomrule
\end{tabular}




\subsection{Only from internal: Modern German}

\subsubsection{external case = ungrammatical}

\ac{int}:\ac{nom}, \ac{ext}:\ac{acc}

\exg. *Ich {lade ein}, wen \tbf{mir} \tbf{sympathisch} \tbf{ist}.\\
 I invite\scsub{[acc]} \tsc{rel}.\ac{acc}.\tsc{an} me nice is\scsub{[nom]}\\
 `I invite who I like.' \flushfill{\pgcitealt{vogel2001}{344}}

\ac{int}:\ac{nom}, \ac{ext}:\ac{dat}

\exg. *Ich vertraue, wem \tbf{Hitchcock} \tbf{mag}.\\
 I trust\scsub{[dat]} \tsc{rel}.\ac{dat}.\tsc{an} Hitchcock likes\scsub{[nom]}\\
 `I trust who likes Hitchcock.' \flushfill{\pgcitealt{vogel2001}{345}}

\ac{int}:\ac{acc}, \ac{ext}:\ac{dat}

\exg. *Ich vertraue wem \tbf{auch} \tbf{Maria} \tbf{mag}. \\
 I trust\scsub{[dat]} \tsc{rel}.\ac{dat}.\tsc{an} also Maria likes\scsub{[acc]}.\\
 `I trust whoever Maria also likes.' \flushfill{\pgcitealt{vogel2001}{345}}



\subsubsection{internal case = grammatical}

Consider the example in \ref{ex:mg-acc-dat}. In this example, the internal case is dative and the external case is accusative.
The internal case is dative. The predicate \tit{vertraut} `trusts' takes dative objects.
The external case is accusative. The predicate \tit{lade ein} `invite' takes accusative objects.
The relative pronoun \tit{wem} `\tsc{rel}.\ac{dat}.\tsc{an}' appears in the internal case: the dative. The relative pronoun is marked in bold, just like as the relative clause, showing that the relative pronoun patterns with the relative clause.

\exg. Ich {lade ein} \tbf{wem} \tbf{auch} \tbf{Maria} \tbf{vertraut}. \\
 I invite\scsub{[acc]} \tsc{rel}.\ac{dat}.\tsc{an} also Maria trusts\scsub{[dat]}.\\
 `I invite whoever Maria also trusts.' \flushfill{\pgcitealt{vogel2001}{344}}\label{ex:mg-acc-dat}

Consider the example in \ref{ex:mg-nom-dat}. In this example, the internal case is dative and the external case is nominative.
The internal case is dative. The predicate \tit{vertraut} `trusts' takes dative objects.
The external case is nominative. The predicate \tit{besucht} `visits' takes nominative subjects.
The relative pronoun \tit{wem} `\tsc{rel}.\ac{dat}.\tsc{an}' appears in the internal case: the dative. The relative pronoun is marked in bold, just like as the relative clause, showing that the relative pronoun patterns with the relative clause.

\exg. Uns besucht \tbf{wem} \tbf{Maria} \tbf{vertraut}.\\
 us visits\scsub{[nom]} \tsc{rel}.\ac{dat}.\tsc{an} Maria trusts\scsub{[dat]}\\
 `Who visits us, Maria trusts.' \flushfill{\pgcitealt{vogel2001}{343}}\label{ex:mg-nom-dat}

 Consider the example in \ref{ex:mg-nom-acc}. In this example, the internal case is accusative and the external case is nominative.
 The internal case is accusative. The predicate \tit{mag} `likes' takes accusative objects.
 The external case is nominative. The predicate \tit{besucht} `visits' takes nominative subjects.
 The relative pronoun \tit{wen} `\tsc{rel}.\ac{acc}.\tsc{an}' appears in the internal case: the accusative. The relative pronoun is marked in bold, just like as the relative clause, showing that the relative pronoun patterns with the relative clause.

 \exg. Uns besucht \tbf{wen} \tbf{Maria} \tbf{mag}.\\
  Us visits\scsub{[nom]} \tsc{rel}.\ac{acc}.\tsc{an} Maria.\ac{nom} likes\scsub{[acc]}\\
  `Who visits us likes Maria likes.' \flushfill{\pgcitealt{vogel2001}{343}}\label{ex:mg-nom-acc}


 \begin{table}[H]
   \center
   \caption {Summary of \ac{mg} matching headless relative data}
 		\begin{tabular}{c|c|c|c}
 		  \toprule
 			\textsubscript{\ac{int}} \textsuperscript{\ac{ext}}
 		        & [\ac{nom}]
 		        & [\ac{acc}]
 		        & [\ac{dat}]
 		        \\ \cmidrule{1-4}
 		    [\ac{nom}]
 		        &
 		        &
 		        &
 		        \\ \cmidrule{1-4}
 		    [\ac{acc}]
 		        & \ac{acc}
 		        &
 		        &
 		        \\ \cmidrule{1-4}
 		    [\ac{dat}]
 		        & \ac{dat}
 		        & \ac{dat}
 		        &
 		        \\
 		  \bottomrule
 		\end{tabular}
 \end{table}


 \subsection{Only from external: Old High German}


  \subsubsection{external is grammatical}


Consider the example in \ref{ex:ohg-dat-acc}. In this example, the internal case is accusative and the external case is dative.
The internal case is nominative. The predicate \tit{zalta} `named' takes accusative objects.
The external case is dative. The predicate \tit{furira} `older' takes dative objects.
The relative pronoun \tit{thên} `\ac{rel}.\ac{dat}.\ac{pl}' appears in the external case: the dative. The relative pronoun is not marked in bold, just like as the main clause, showing that the relative pronoun patterns with the main clause.
Examples in which the internal case is accusative, the external case is dative and the relative pronoun appears in accusative case are unattested.

\exg. bistû furira Abrâhame, ouh thên \tbf{man} \tbf{hiar} \tbf{nû} \tbf{zalta}?\\
 {are you} older\scsub{[dat]} {Abraham} and \tsc{rel}.\ac{dat}.\tsc{pl} one here just named\scsub{[acc]}\\
 `are you really older than Abraham and (those) who have been mentioned here?' \flushfill{\ac{ohg}, \ac{otfrid} III 18:33, \pgcitealt{behaghel1923}{761}, after Delbrück}\label{ex:ohg-dat-acc}

%% zoek voor deze!! Übersetzung in Delbrück, S. 373
%also take his spelling


Consider the example in \ref{ex:ohg-dat-nom}. In this example, the internal case is nominative and the external case is dative.
The internal case is nominative. The predicate \tit{sprah} `spoke' takes nominative subjects.
The external case is dative. The predicate \tit{antuurta} `replied' takes dative objects.
The relative pronoun \tit{demo} `\ac{rel}.\ac{dat}.\ac{m}.\ac{sg}' appears in the external case: the dative. The relative pronoun is not marked in bold, just like as the main clause, showing that the relative pronoun patterns with the main clause.
Examples in which the internal case is nominative, the external case is dative and the relative pronoun appears in nominative case are unattested.

\exg. aer antuurta demo \tbf{zaimo} \tbf{sprah}\\
 he replied\scsub{[dat]} \tsc{rel}.\ac{dat}.\tsc{m}.\tsc{sg} {to him} spoke\scsub{[nom]}\\
 `he replied to the one who spoke to him' \flushfill{\ac{ohg}, \ac{mons} 7:24, \pgcitealt{behaghel1923}{761}, after \pgcitealt{pittner1995}{199}}\label{ex:ohg-dat-nom}

 %check for pittner's spelling


Consider the example in \ref{ex:ohg-acc-nom}. In this example, the internal case is nominative and the external case is accusative.
The internal case is nominative. The predicate \tit{chisetzit} `pull through' takes nominative subjects.
The external case is accusative. The predicate \tit{bibringu} `bring to light' takes accusative objects.
The relative pronoun \tit{dhen} `\ac{rel}.\ac{acc}.\ac{m}.\ac{sg}' appears in the external case: the accusative. The relative pronoun is not marked in bold, just like as the main clause, showing that the relative pronoun patterns with the main clause.
Examples in which the internal case is nominative, the external case is accusative and the relative pronoun appears in nominative case are unattested.

\exg. ih bibringu fona (iacobes samin endi) Iuda dhen \tbf{mina} \tbf{berga} \tbf{chisetzit}\\
 I {bring to light}\scsub{[acc]} about (Jacob together end?) Judas \tsc{rel}.\ac{acc}.\tsc{m}.\tsc{sg} my mountains {through pull}\scsub{[nom]}\\
 `I bright to light the one who wanders through my mountains about Jacob and end of Judas' \flushfill{\ac{ohg}, \ac{isid} 34:3, \pgcitealt{behaghel1923}{761}}\label{ex:ohg-acc-nom}

%%check for hans eggers spelling
%(Der althochdeutsche Isidor. Nach der Pariser Handschrift und den Monseer Fragmenten neu herausgegeben von Hans Eggers. Tübingen: Niemeyer 1964, S. 61 und 63)









  \subsubsection{internal is ungrammatical}

  \ex. \ac{int}:\ac{acc}, \ac{ext}:\ac{nom}
  \a. \ac{acc} not attested
  \b. \ac{nom} not attested

 \ex. \ac{int}:\ac{dat}, \ac{ext}:\ac{nom}
 \a. \ac{dat} not attested
 \b. \ac{nom} not attested

 \ex. \ac{int}:\ac{dat}, \ac{ext}:\ac{acc}
 \a. \ac{dat} not attested
 \b. \ac{acc} not attested




 \begin{table}[H]
   \center
   \caption {Summary of \ac{ohg} matching headless relative data}
 		\begin{tabular}{c|c|c|c}
 		  \toprule
 			\textsubscript{\ac{int}} \textsuperscript{\ac{ext}}
 		        & [\ac{nom}]
 		        & [\ac{acc}]
 		        & [\ac{dat}]
 		        \\ \cmidrule{1-4}
 		    [\ac{nom}]
 		        &
 		        & \ac{acc}
 		        & \ac{dat}
 		        \\ \cmidrule{1-4}
 		    [\ac{acc}]
 		        &
 		        &
 		        & \ac{dat}
 		        \\ \cmidrule{1-4}
 		    [\ac{dat}]
 		        &
 		        &
 		        &
 		        \\
 		  \bottomrule
 		\end{tabular}
 \end{table}
