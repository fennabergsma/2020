% !TEX root = thesis.tex

\chapter{The variation}

I showed that headless relatives crosslinguistically make reference to the case scale: if two cases are in competition, it is always the same that wins.
However, there is a second aspect to headless relatives, which differs across languages. That is, sometimes the case competition does not even take place, or rather, there is no winner to the competition. I show that whether is not the case competition takes place depends on the type of relative pronoun that is used. I distinguish two types of relative pronouns: \tsc{wh}-pronouns and \tsc{d}-pronouns.

When the relative pronoun is a \tsc{wh}-pronoun, only the internal case can win. When the relative pronoun is a \tsc{d}-pronoun, both the internal and the external case can win.

%then have a particular order with the variation

Every speaker of a language needs to learn what the pattern for its language is. Headless relatives are infrequent, is what can be said about at least German. Even though not everybody likes the construction to begin with (they prefer (light-)headed relatives), people seem to have the clear intuition that \tsc{int}>\tsc{ext} is much better than the other way around. It seems implausible that learners of German learn this pattern from the few examples they got (there are just too few to make a generalization). Still, the intuition exist. And it is very particular: more complex case wins over less complex case, but only if the internal case is more complex than the external case. This already sounds hard to learn from the input as a generalization.

People have also been describing it like this: formulation from Cinque in his book.

If it does not come from the input, where does it come from? I claim that it comes from other properties of German (or another language).

A similar avenue was pursued by \citealt{himmelreich2017}. She specific languages for having different types of agree (up, down) and different types of probes (active, non-active). Doing that, she successfully derived free relatives and parasitic gaps in different languages.
The crucial difference with I'm doing is that I'm not relying on an arbitrary value I assigned to a language (say null head is active probe, probing only happens upwards). Instead, I look for patterns within the languages themselves, and let the facts of the headless relatives follow from those. To be more specific, I decompose the relative pronouns, and I determine what parts of syntactic structure they correspond to. Having independently done that, I return to the headless relatives, and I derive the facts.



\section{Only internal}

In \ac{mg}, only internal case can win.

Internal is grammatical.

Consider the example in \ref{ex:mg-acc-dat}. In this example, the internal case is dative and the external case is accusative.
The internal case is dative. The predicate \tit{vertrauen} `to trust' takes dative objects.
The external case is accusative. The predicate \tit{einladen} `to invite' takes accusative objects.
The relative pronoun \tit{wem} `\tsc{rel}.\ac{dat}.\tsc{an}' appears in the internal case: the dative. The relative pronoun is marked in bold, just like as the relative clause, showing that the relative pronoun patterns with the relative clause.

\exg. Ich {lade ein} \tbf{wem} \tbf{auch} \tbf{Maria} \tbf{vertraut}. \\
I.\ac{nom} invite.1\tsc{sg}\scsub{[acc]} \tsc{rel}.\ac{dat}.\tsc{an} also Maria.\ac{nom} trust.3\tsc{sg}\scsub{[dat]}.\\
`I invite whoever Maria also trusts.' \flushfill{adapted from \pgcitealt{vogel2001}{344}}\label{ex:mg-acc-dat}

Consider the example in \ref{ex:mg-nom-dat}. In this example, the internal case is dative and the external case is nominative.
The internal case is dative. The predicate \tit{vertrauen} `to trust' takes dative objects.
The external case is nominative. The predicate \tit{besuchen} `to visit' takes nominative subjects.
The relative pronoun \tit{wem} `\tsc{rel}.\ac{dat}.\tsc{an}' appears in the internal case: the dative. The relative pronoun is marked in bold, just like as the relative clause, showing that the relative pronoun patterns with the relative clause.

\exg. Uns besucht \tbf{wem} \tbf{Maria} \tbf{vertraut}.\\
we.\ac{acc} visit.3\ac{sg}\scsub{[nom]} \tsc{rel}.\ac{dat}.\tsc{an} Maria.\ac{nom} trust.3\ac{sg}\scsub{[dat]}\\
`Who visits us, Maria trusts.' \flushfill{adapted from \pgcitealt{vogel2001}{343}}\label{ex:mg-nom-dat}

Consider the example in \ref{ex:mg-nom-acc}. In this example, the internal case is accusative and the external case is nominative.
The internal case is accusative. The predicate \tit{mögen} `to like' takes accusative objects.
The external case is nominative. The predicate \tit{besuchen} `to visit' takes nominative subjects.
The relative pronoun \tit{wen} `\tsc{rel}.\ac{acc}.\tsc{an}' appears in the internal case: the accusative. The relative pronoun is marked in bold, just like as the relative clause, showing that the relative pronoun patterns with the relative clause.

\exg. Uns besucht \tbf{wen} \tbf{Maria} \tbf{mag}.\\
 we.\ac{acc} visit.3\ac{sg}\scsub{[nom]} \tsc{rel}.\ac{acc}.\tsc{an} Maria.\ac{nom} like.3\ac{sg}\scsub{[acc]}\\
 `Who visits us, Maria likes.' \flushfill{adapted from \pgcitealt{vogel2001}{343}}\label{ex:mg-nom-acc}

External is ungrammatical.

Consider the example in \ref{ex:mg-dat-acc}. In this example, the internal case is accusative and the external case is dative.
The internal case is accusative. The predicate \tit{mögen} `to like' takes accusative objects.
The external case is dative. The predicate \tit{vertrauen} `to trust' takes dative objects.
The relative pronoun \tit{wem} `\tsc{rel}.\ac{dat}.\tsc{an}' appears in the external case: the dative. The relative pronoun is not marked in bold, just like as the main clause, showing that the relative pronoun patterns with the main clause.
This is ungrammatical in German: only the internal case can win.

\exg. *Ich vertraue wem \tbf{auch} \tbf{Maria} \tbf{mag}. \\
I.\ac{nom} trust.1\ac{sg}\scsub{[dat]} \tsc{rel}.\ac{dat}.\tsc{an} also Maria.\ac{nom} like.3\ac{sg}\scsub{[acc]}.\\
`I trust whoever Maria also likes.' \flushfill{adapted from \pgcitealt{vogel2001}{345}}\label{ex:mg-dat-acc}

Consider the example in \ref{ex:mg-dat-nom}. In this example, the internal case is nominative and the external case is dative.
The internal case is nominative. The predicate \tit{mögen} `to like' takes nominative subjects.
The external case is dative. The predicate \tit{vertrauen} `to trust' takes dative objects.
The relative pronoun \tit{wem} `\tsc{rel}.\ac{dat}.\tsc{an}' appears in the external case: the dative. The relative pronoun is not marked in bold, just like as the main clause, showing that the relative pronoun patterns with the main clause.
This is ungrammatical in German: only the internal case can win.

\exg. *Ich vertraue, wem \tbf{Hitchcock} \tbf{mag}.\\
I.\ac{nom} trust.1\ac{sg}\scsub{[dat]} \tsc{rel}.\ac{dat}.\tsc{an} Hitchcock.\ac{acc} like.3\ac{sg}\scsub{[nom]}\\
`I trust who likes Hitchcock.' \flushfill{adapted from \pgcitealt{vogel2001}{345}}\label{ex:mg-dat-nom}

Consider the example in \ref{ex:mg-acc-nom}. In this example, the internal case is nominative and the external case is accusative.
The internal case is nominative. The predicate \tit{sein} `to be' takes nominative subjects.
The external case is accusative. The predicate \tit{einladen} `to invite' takes accusative objects.
The relative pronoun \tit{wen} `\tsc{rel}.\ac{acc}.\tsc{an}' appears in the external case: the accusative. The relative pronoun is not marked in bold, just like as the main clause, showing that the relative pronoun patterns with the main clause.
This is ungrammatical in German: only the internal case can win.

\exg. *Ich {lade ein}, wen \tbf{mir} \tbf{sympathisch} \tbf{ist}.\\
I.\ac{nom} invite.1\ac{sg}\scsub{[acc]} \tsc{rel}.\ac{acc}.\tsc{an} I.\ac{dat} nice be.3\ac{sg}\scsub{[nom]}\\
`I invite who I like.' \flushfill{adapted from \pgcitealt{vogel2001}{344}}\label{ex:mg-acc-nom}

To summarize

\begin{table}[H]
  \center
  \caption {Summary of \ac{mg} matching headless relative data}
		\begin{tabular}{c|c|c|c}
		  \toprule
			\textsubscript{\ac{int}} \textsuperscript{\ac{ext}}
		        & [\ac{nom}]
		        & [\ac{acc}]
		        & [\ac{dat}]
		        \\ \cmidrule{1-4}
		    [\ac{nom}]
		        &
		        &
		        &
		        \\ \cmidrule{1-4}
		    [\ac{acc}]
		        & \ac{acc}
		        &
		        &
		        \\ \cmidrule{1-4}
		    [\ac{dat}]
		        & \ac{dat}
		        & \ac{dat}
		        &
		        \\
		  \bottomrule
		\end{tabular}
\end{table}







\section{Generalizations}

\begin{itemize}
  \item when a language's light-headed relatives are \tsc{d} - \tsc{wh}, non-matching is never allowed
  \item with \tsc{wh} morphology, the internal case can win
  \item with \tsc{d} morphology (or is it when the relative pronoun can take a complement), both the external and the internal can win
\end{itemize}


\section{Two \tsc{d}-examples}

\subsection{\tsc{d} in OHG}

works via attraction

\subsection{\tsc{d} in Gothic}

does not work via attraction, but complementizer(?)








\section{\tsc{wh} in German}

show how this is different from d-relatives (the impossiblitiy for non-matching for \tsc{d} \tsc{w} light headed relative languages has to follow from this)








\chapter*{old variation}

\section{\ac{ohg} and \ac{mg} have the same winner}


\subsection{\ac{ohg}}

same holds for \ac{ohg} but you just can't give the ungrammatical examples. but they are not attested.



\section{The different patterns}

In Gothic, the more complex case wins.
In \ac{ohg}, the more complex case wins, only if it is external.
In \ac{mg}, the more complex case wins, only if it is internal.
In Italian, case mismatch is not allowed.


\begin{table}[H]
	\center
	\caption {Variation}
		\begin{tabular}{ccc}
		\toprule
		 					& \ac{int}>\ac{ext}		& \ac{ext}>\ac{int}	\\
								\cmidrule{2-3}
		\ac{mg} 	& ✔			 							&	*									\\
		\ac{ohg}	& *										&	✔									\\
		Gothic		&	✔										&	✔									\\
		\bottomrule
		\end{tabular}
\end{table}



\subsection{Both: Gothic}


% !TEX root = ../thesis.tex

\begin{tabular}{c|c|c|c}
  \toprule
      \textsubscript{\ac{int}} \textsuperscript{\ac{ext}}
        & [\ac{nom}]
        & [\ac{acc}]
        & [\ac{dat}]
        \\ \cmidrule{1-4}
    [\ac{nom}]
        & \ac{nom}
        & \ac{acc}
        & \ac{dat}
        \\ \cmidrule{1-4}
    [\ac{acc}]
        & \ac{acc}
        & \ac{acc}
        & \ac{dat}
        \\ \cmidrule{1-4}
    [\ac{dat}]
        & \ac{dat}
        & (\ac{dat})
        & \ac{dat}
        \\
  \bottomrule
\end{tabular}







 \subsection{Only from external: Old High German}


  \subsubsection{external is grammatical}


Consider the example in \ref{ex:ohg-dat-acc}. In this example, the internal case is accusative and the external case is dative.
The internal case is nominative. The predicate \tit{zalta} `named' takes accusative objects.
The external case is dative. The predicate \tit{furira} `older' takes dative objects.
The relative pronoun \tit{thên} `\ac{rel}.\ac{dat}.\ac{pl}' appears in the external case: the dative. The relative pronoun is not marked in bold, just like as the main clause, showing that the relative pronoun patterns with the main clause.
Examples in which the internal case is accusative, the external case is dative and the relative pronoun appears in accusative case are unattested.

\exg. bistû furira Abrâhame, ouh thên \tbf{man} \tbf{hiar} \tbf{nû} \tbf{zalta}?\\
 {are you} older\scsub{[dat]} {Abraham} and \tsc{rel}.\ac{dat}.\tsc{pl} one here just named\scsub{[acc]}\\
 `are you really older than Abraham and (those) who have been mentioned here?' \flushfill{\ac{ohg}, \ac{otfrid} III 18:33, \pgcitealt{behaghel1923}{761}, after Delbrück}\label{ex:ohg-dat-acc}

\exg. Bis	tú	nu	{zi wáre}	furira	Ábrahame	?	ouh	thén	man	hiar	nu	zálta\\
sein	du	nun,jetzt	wahrlich	erhabener,wertvoller	Abraham	?	auch,und	der,die,das	man	hier	nun,jetzt	erzählen,verkünden\\

%% zoek voor deze!! Übersetzung in Delbrück, S. 373
%also take his spelling


Consider the example in \ref{ex:ohg-dat-nom}. In this example, the internal case is nominative and the external case is dative.
The internal case is nominative. The predicate \tit{sprah} `spoke' takes nominative subjects.
The external case is dative. The predicate \tit{antuurta} `replied' takes dative objects.
The relative pronoun \tit{demo} `\ac{rel}.\ac{dat}.\ac{m}.\ac{sg}' appears in the external case: the dative. The relative pronoun is not marked in bold, just like as the main clause, showing that the relative pronoun patterns with the main clause.
Examples in which the internal case is nominative, the external case is dative and the relative pronoun appears in nominative case are unattested.

\exg. aer antuurta demo \tbf{zaimo} \tbf{sprah}\\
 he replied\scsub{[dat]} \tsc{rel}.\ac{dat}.\tsc{m}.\tsc{sg} {to him} spoke\scsub{[nom]}\\
 `he replied to the one who spoke to him' \flushfill{\ac{ohg}, \ac{mons} 7:24, \pgcitealt{behaghel1923}{761}, after \pgcitealt{pittner1995}{199}}\label{ex:ohg-dat-nom}

 %check for pittner's spelling


Consider the example in \ref{ex:ohg-acc-nom}. In this example, the internal case is nominative and the external case is accusative.
The internal case is nominative. The predicate \tit{chisetzit} `pull through' takes nominative subjects.
The external case is accusative. The predicate \tit{bibringu} `bring to light' takes accusative objects.
The relative pronoun \tit{dhen} `\ac{rel}.\ac{acc}.\ac{m}.\ac{sg}' appears in the external case: the accusative. The relative pronoun is not marked in bold, just like as the main clause, showing that the relative pronoun patterns with the main clause.
Examples in which the internal case is nominative, the external case is accusative and the relative pronoun appears in nominative case are unattested.

\exg. ih bibringu fona (iacobes samin endi) Iuda dhen \tbf{mina} \tbf{berga} \tbf{chisetzit}\\
 I {bring to light}\scsub{[acc]} about (Jacob together end?) Judas \tsc{rel}.\ac{acc}.\tsc{m}.\tsc{sg} my mountains {through pull}\scsub{[nom]}\\
 `I bright to light the one who wanders through my mountains about Jacob and end of Judas' \flushfill{\ac{ohg}, \ac{isid} 34:3, \pgcitealt{behaghel1923}{761}}\label{ex:ohg-acc-nom}







%%check for hans eggers spelling
%(Der althochdeutsche Isidor. Nach der Pariser Handschrift und den Monseer Fragmenten neu herausgegeben von Hans Eggers. Tübingen: Niemeyer 1964, S. 61 und 63)









  \subsubsection{internal is ungrammatical}

  \ex. \ac{int}:\ac{acc}, \ac{ext}:\ac{nom}
  \a. \ac{acc} not attested
  \b. \ac{nom} not attested

 \ex. \ac{int}:\ac{dat}, \ac{ext}:\ac{nom}
 \a. \ac{dat} not attested
 \b. \ac{nom} not attested

 \ex. \ac{int}:\ac{dat}, \ac{ext}:\ac{acc}
 \a. \ac{dat} not attested
 \b. \ac{acc} not attested




 \begin{table}[H]
   \center
   \caption {Summary of \ac{ohg} matching headless relative data}
 		\begin{tabular}{c|c|c|c}
 		  \toprule
 			\textsubscript{\ac{int}} \textsuperscript{\ac{ext}}
 		        & [\ac{nom}]
 		        & [\ac{acc}]
 		        & [\ac{dat}]
 		        \\ \cmidrule{1-4}
 		    [\ac{nom}]
 		        &
 		        & \ac{acc}
 		        & \ac{dat}
 		        \\ \cmidrule{1-4}
 		    [\ac{acc}]
 		        &
 		        &
 		        & \ac{dat}
 		        \\ \cmidrule{1-4}
 		    [\ac{dat}]
 		        &
 		        &
 		        &
 		        \\
 		  \bottomrule
 		\end{tabular}
 \end{table}


 % The nativeness of the headless relative constructions under
 % consideration, and in particular those represented by group (I), can be
 % established for the moment by citing instances in which they occur independently
 % of the Greek, e.g., in which the Gothic modified relative clause
 % reflects a Greek participial construction or a noun, or instances in which
 % 18
 % they occur in opposition to the Greek. Evidence of the first type is
 % found in Mk 10:32, Luk 3:13, Rom 14:19, II Cor 8:11, Col 3:2, and elsewhere.

Old English, Old Icelandic both show: \tsc{d}-pronoun plus invariant relativizer + always take case from the main clause

Polish and Italian have \tsc{wh}-pronouns but do not allow for conflicts
