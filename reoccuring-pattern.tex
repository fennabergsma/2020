% !TEX root = thesis.tex

\chapter{A reoccurring pattern}

First I introduce the pattern that forms the focus of the first part of the dissertation. I show that headless relatives in Gothic adhere to the case strength scale: \ac{nom} < \ac{acc} < \ac{dat}.

Then I show two phenomena that follow the same ordering of \ac{nom}, \ac{acc} and \ac{dat}. The two phenomena are accessibility hierarchies. The first one is about agreement, the second one about relativization.

In the last section of this chapter I discuss how \ac{nom}, \ac{acc} and \ac{dat} pattern in morphology.


\section{Case competition in Gothic headless relatives}

In this section I show the behavior of Gothic headless relatives in detail. I systematically go through all case combinations, except for the genitive, to which I return in Section \ref{sec:genitive}. This leaves the nominative, accusative and dative. First, I discuss the matching headless relatives, in which the internal and external case match.

Consider the example in \ref{ex:gothicaccaccrep}, repeated from the introduction. In this example, the internal case and the external case are accusative.
The relative clause, including the relative pronoun, is marked in gray.
The internal case is accusative. The predicate \tit{arma} `pity' takes accusative objects.
The external case is accusative as well. Here the predicate \tit{gaarma} `pity' takes accusative objects.
The relative pronoun \tit{þan(a)} `who.\ac{acc}' appears in the accusative.

\exg. gaarma \tcol{DG}{þan} \tcol{DG}{-ei} \tcol{DG}{arma}\\
 pity\scsub{[acc]} who.\ac{acc} -\ac{comp} pity\scsub{[acc]}\\
 `I will pity (him) whom I pity' \flushfill{Gothic, \ac{rom} 9:15, after \pgcitealt{harbert1978}{339}}\label{ex:gothicaccaccrep}

 Consider the example in \ref{ex:gothicnomnom}, in which the internal case and the external case are nominative.
 The relative clause, including the relative pronoun, is marked in gray.
 The internal case is nominative. The predicate \tit{matjai} `eats' takes nominative subjects.
 The external case is nominative as well. Here the predicate \tit{gadauþnai} `die' takes nominative subjects.
 The relative pronoun \tit{sa} `who.\ac{nom}' appears in the nominative.

\exg. ei \tcol{DG}{sa} \tcol{DG}{-ei} \tcol{DG}{þis} \tcol{DG}{matjai}, ni gadauþnai\\
 that who.\ac{nom} -\ac{comp} {of this} eats\scsub{[nom]} not die\scsub{[nom]}\\
 `that (he) who eats of this may not die' \flushfill{Gothic, \ac{john} 6:50, after \pgcitealt{harbert1978}{337}}\label{ex:gothicnomnom}

 Consider the examples in \ref{ex:gothicdatdat}, in which the internal case and the external case are dative.
 The relative clauses, including the relative pronoun, is marked in gray.
 The internal case is dative. The predicates \tit{gabaur} `tribute', \tit{mota} `custom', \tit{agis} `fear' and \tit{sweriþa} `honour' takes dative objects.
 The external case is dative as well. The same predicates as in the relative clause take dative objects.
 The relative pronouns \tit{þamm(a)} `who.\ac{dat}' appear in the dative.

\ex.\label{ex:gothicdatdat}
\ag. \tcol{DG}{þamm} \tcol{DG}{-ei} \tcol{DG}{gabaur} gabaur\\
 who.\ac{dat} -\ac{comp} tribute\scsub{[dat]} tribute\scsub{[dat]}\\
 `tribute to (him) whom tribute is due'
\bg. \tcol{DG}{þamm} \tcol{DG}{-ei} \tcol{DG}{mota} mota\\
 \tcol{DG}{who.\ac{dat}} \tcol{DG}{-\ac{comp}} \tcol{DG}{custom\scsub{[dat]}} custom\scsub{[dat]}\\
 `custom to (him) whom custom is due'
\bg. \tcol{DG}{þamm} \tcol{DG}{-ei} \tcol{DG}{agis} agis\\
 \tcol{DG}{who.\ac{dat}} \tcol{DG}{-\ac{comp}} \tcol{DG}{fear\scsub{[dat]}} fear\scsub{[dat]}\\
 `fear (him) whom fear is due'
\bg. \tcol{DG}{þamm} \tcol{DG}{-ei} \tcol{DG}{sweriþa} sweriþa\\
 \tcol{DG}{who.\ac{dat}} \tcol{DG}{-\ac{comp}} \tcol{DG}{honour\scsub{[dat]}} honour\scsub{[dat]}\\
 `honour (him) whom honour is due' \flushfill{Gothic, \ac{rom} 13:7, after \pgcitealt{harbert1978}{339}}

A summary of data so far is given in Table \ref{tbl:summarygothicmatch}. The left column shows the internal case between square brackets. The upper row shows the external case between square brackets. The other cells indicate the case of the relative pronoun.
So far only the diagonal line is filled. These are the matching examples, the examples in which the internal case matches the external case. The relative pronoun appears in the internal and external case, and it is marked in dark gray. The nominative is given in \ref{ex:gothicnomnom}, the accusative in \ref{ex:gothicaccaccrep}, and the dative in \ref{ex:gothicdatdat}.

\begin{table}[H]
  \center
  \caption {Summary of Gothic matching headless relative data}
    \begin{tabular}{c|c|c|c}
      \toprule
        \diagbox[linecolor=white]{\ac{int}}{\ac{ext}}
            & [\ac{nom}]
            & [\ac{acc}]
            & [\ac{dat}]
            \\ \cmidrule{1-4}
        [\ac{nom}]
            & \colorbox{LG}{\ac{nom}}
            & \diagbox[linecolor=white]{\phantom{nom}}{\phantom{nom}}
            & \diagbox[linecolor=white]{\phantom{nom}}{\phantom{nom}}
            \\ \cmidrule{1-4}
        [\ac{acc}]
            & \diagbox[linecolor=white]{\phantom{nom}}{\phantom{nom}}
            & \colorbox{LG}{\ac{acc}}
            & \diagbox[linecolor=white]{\phantom{nom}}{\phantom{nom}}
            \\ \cmidrule{1-4}
        [\ac{dat}]
            & \diagbox[linecolor=white]{\phantom{nom}}{\phantom{nom}}
            & \diagbox[linecolor=white]{\phantom{nom}}{\phantom{nom}}
            & \colorbox{LG}{\ac{dat}}
            \\
      \bottomrule
    \end{tabular}
    \label{tbl:summarygothicmatch}
\end{table}

In what follows, I discuss the non-matching headless relatives, in which the internal and external case differ.

Consider the example in \ref{ex:gothicnomacc}, in which the internal case is nominative and the external case is accusative.
The relative clause, excluding the relative pronoun, is marked in gray.
The internal case is nominative. The predicate \tit{ist us Laudeikaion} `is from Laodicea' takes nominative subjects.
The external case is accusative. The predicate \tit{ussiggwaid} `read' takes accusative objects.
The relative pronoun \tit{þo} `what.\ac{acc}' appears in the external case: the accusative.
Examples, in which the relative pronoun appears in nominative case, the internal case is nominative and the external case is accusative, are unattested.

\exg. jah þo \tcol{DG}{-ei} \tcol{DG}{ist} \tcol{DG}{us} \tcol{DG}{Laudeikaion} jus ussiggwaid\\
 and what.\ac{acc} -\ac{comp} is\scsub{[nom]} from Laodicea you read\scsub{[acc]}\\
 `and read that which is from Laodicea' \flushfill{Gothic, \ac{col} 4:16, after \pgcitealt{harbert1978}{357}}\label{ex:gothicnomacc}

Consider the example in \ref{ex:gothicnomdat}, in which the internal case is nominative and the external case is dative.
The relative clause, excluding the relative pronoun, is marked in gray.
The internal case is nominative. The predicate \tit{sind fraþjaiþ} `are above' takes a nominative subject.
The external case is dative. The predicate \tit{fraþjaiþ} `think on' takes dative indirect objects.
The relative pronoun \tit{þaim} `what.\ac{dat}' appears in the external case: the dative.
Examples, in which the relative pronoun appears in nominative case, the internal case is nominative and the external case is dative, are unattested.

\exg. þaim \tcol{DG}{-ei} \tcol{DG}{iupa} \tcol{DG}{sind} fraþjaiþ \\
 what.\ac{dat} -\ac{comp} above are\scsub{[nom]} {think on}\scsub{[dat]}\\
 `set your mind on those which are above' \flushfill{Gothic, \ac{col} 3:2, after \pgcitealt{harbert1978}{339}}\label{ex:gothicnomdat}

Consider the example in \ref{ex:gothicaccnom}, in which the internal case is accusative and the external case is nominative.
The relative clause, including the relative pronoun, is marked in gray.
The internal case is accusative. The predicate \tit{frijos} `love' takes accusative objects.
The external case is nominative. The predicate \tit{siuks ist} `is sick' takes nominative subjects.
The relative pronoun \tit{þan} `who.\ac{acc}' appears in the internal case: the accusative.
Examples, in which the relative pronoun appears in nominative case, the internal case is accusative and the external case is nominative, are unattested.

\exg. \tcol{DG}{þan} \tcol{DG}{-ei} \tcol{DG}{frijos} siuks ist\\
 who.\ac{acc} -\ac{comp} love\scsub{[acc]} sick is\scsub{[nom]}\\
 `the one whom you love is sick' \flushfill{Gothic, \ac{john} 11:3, after \pgcitealt{harbert1978}{342}}\label{ex:gothicaccnom}

Consider the example in \ref{ex:gothicaccdatrep}, repeated from the introduction. In this example, the internal case is accusative and the external case is dative.
The relative clause, excluding the relative pronoun, is marked in gray.
The internal case is accusative. The predicate \tit{qiþiþ} `say' takes accusative objects.
The external case is dative. The predicate \tit{taujau} `do' takes dative indirect objects.
The relative pronoun \tit{þamm} `who.\ac{dat}' appears in the external case: the dative.
Examples, in which the relative pronoun appears in accusative case, the internal case is accusative and the external case is dative, are unattested.

\exg. hva nu wileiþ ei taujau þamm \tcol{DG}{-ei} \tcol{DG}{qiþiþ} \tcol{DG}{þiudan} \tcol{DG}{Iudaie}?\\
 what now want that do\scsub{[dat]} who.\ac{dat} -\ac{comp} say\scsub{[acc]} king {of Jews}\\
 `what now do you wish that I do to (him) whom you call King of the Jews?' \flushfill{Gothic, \ac{mark} 15:12, after \pgcitealt{harbert1978}{339}}\label{ex:gothicdataccrep}

Consider the example in \ref{ex:gothicdatnom}, in which the internal case is dative and the external case is nominative.
The relative clause, including the relative pronoun, is marked in gray.
The internal case is dative. The predicate \tit{fraletada} `is forgiven' takes dative objects.
The external case is nominative. The predicate \tit{frijod} `loves' takes nominative subjects.
The relative pronoun \tit{þamm(a)} `who.\ac{dat}' appears in the internal case: the dative.
Examples, in which the relative pronoun appears in nominative case, the internal case is dative and the external case is nominative, are unattested.

\exg. iþ þamm -ei leitil fraletada leitil frijod\\
 but \tcol{DG}{who.\ac{dat}} \tcol{DG}{-\ac{comp}} \tcol{DG}{little} \tcol{DG}{{is forgiven}\scsub{[dat]}} little loves\scsub{[nom]}\\
 `but the one whom little is forgiven loves little' \flushfill{Gothic, \ac{luke} 7:47, after \pgcitealt{harbert1978}{342}}\label{ex:gothicdatnom}

Consider the example in \ref{ex:gothicdataccrep}, repeated from the introduction. In this example, the internal case is dative and the external case is accusative.
The relative clause, including the relative pronoun, is marked in gray.
The internal case is dative. The preposition \tit{ana} `on' takes dative complements.
The external case is accusative. The predicate \tit{ushafjands} `picking up' takes accusative objects.
The relative pronoun \tit{þamm(a)} `who.\ac{dat}' appears in the internal case: the dative.
Examples, in which the relative pronoun appears in accusative case, the internal case is dative and the external case is accusative, are unattested.

\exg. ushafjands ana þamm -ei lag\\
 {picking up}\scsub{[acc]} \tcol{DG}{on\scsub{[dat]}} \tcol{DG}{what.\ac{dat}} \tcol{DG}{-\ac{comp}} \tcol{DG}{lay}\\
 `picking up (that) on which he lay' \flushfill{Gothic, \ac{luke} 5:25, after \pgcitealt{harbert1978}{343}}\label{ex:gothicaccdatrep}

A summary of the Gothic data as a whole is given in Table \ref{tbl:summarygothic}. The left column shows the internal case, the upper row shows the external case. The diagonal is filled with matching examples, marked dark gray.
The remaining six cells show instances where the internal and external case differ. Within the cells, two cases are given. The case in the lower left corner stands for the relative pronoun in the internal case. The case in the upper right corner stands for the relative pronoun in the external case. The grammatical examples are marked in light gray. The unattested examples are marked with an asterix and are unmarked.\footnote{
Throughout this dissertation * stands for 'not found in natural language'. For extinct languages this means that there are no attested examples. For modern languages it means that the examples are ungrammatical.
}

\begin{table}[H]
  \center
  \caption {Summary of Gothic headless relative data}
    % !TEX root = ../thesis.tex

\begin{tabular}{c|c|c|c}
  \toprule
    \diagbox[linecolor=white]{\ac{int}}{\ac{ext}}
        & [\ac{nom}]
        & [\ac{acc}]
        & [\ac{dat}]
        \\ \cmidrule{1-4}
    [\ac{nom}]
        & 
        & \diagbox[linecolor=white]{*\ac{nom}}{\colorbox{LG}{\ac{acc}}}
        & \diagbox[linecolor=white]{*\ac{nom}}{\colorbox{LG}{\ac{dat}}}
        \\ \cmidrule{1-4}
    [\ac{acc}]
        & \diagbox[linecolor=white]{\colorbox{LG}{\ac{acc}}}{*\ac{nom}}
        &
        & \diagbox[linecolor=white]{*\ac{acc}}{\colorbox{LG}{\ac{dat}}}
        \\ \cmidrule{1-4}
    [\ac{dat}]
        & \diagbox[linecolor=white]{\colorbox{LG}{\ac{dat}}}{*\ac{nom}}
        & \diagbox[linecolor=white]{\colorbox{LG}{\ac{dat}}}{*\ac{acc}}
        &
        \\
  \bottomrule
\end{tabular}

    \label{tbl:summarygothic}
\end{table}

The three instances in the lower left corner correspond to the examples \ref{ex:gothicaccnom}, \ref{ex:gothicdatnom} and \ref{ex:gothicdataccrep}. In the attested examples, the relative pronoun appears in the internal case.
The three instances in the upper right corner correspond to the examples in \ref{ex:gothicnomacc}, \ref{ex:gothicnomdat} and \ref{ex:gothicaccdatrep}. In the attested examples, the relative pronoun appears in the external case.

This can be reformulated as follows. In a competition, dative wins over accusative and nominative. This can be seen in the lowest row and the most right column. Additionally, accusative wins over nominative. In sum, the situation can be summarized as in \ref{ex:competition1by1}.

\ex.\label{ex:competition1by1}
\a. \tsc{acc} wins over \tsc{nom}
\b. \tsc{dat} wins over \tsc{nom}
\b. \tsc{dat} wins over \tsc{acc}

Formulated in a scale of `case strength' \citealt{harbert1978,pittner1995,vogel2001,grosu2003,caha2019}:

\ex. \tsc{nom} < \tsc{acc} < \tsc{dat}\label{ex:casestrength}

In the next few sections I show that this scale of case strength does not only appear in headless relatives. The next section shows two other morphosyntactic phenomena that follow the same scale. Section \ref{sec:casemorphology} shows how the case scale is reflected in morphophonology.


\section{Parallels in accessibility hierarchies}

In this section I discuss two additional phenomena in morphosyntax that reflect the \tsc{nom} < \tsc{acc} < \tsc{dat} ordering.

\subsection{Agreement}

% The hierarchy is to be read as follows: if a language allows the verb to agree with an argument marked by a case X, it also allows the verb to agree with all arguments to the left of X.
%
% Moravcsik (1974) presented a set of implicational universals regarding (NP-predicate) agreement. The universals are formulated in terms of grammatical functions (subject, object, etc.), and include implicational hierarchy schematized in (12). The hierarchy ranges over languages, not sentences, and conflates a set of statements such as the following (see Moravcsik 1978 for revisions):
% • If in a language the verb agrees with anything, it agrees with some or all subjects
% • If the verb agrees with anything other than subjects, it agrees with some or all direct objects

A language outside the circle does not show any agreement. A language in the white circle only shows agreement with the subject. A language within the light gray circle shows agreement with both subjects and direct objects. Notice it is not possible for a language to show agreement with the direct object and not show it with a subject. A language within the dark gray circle shown agreement with subjects, direct objects and indirect objects. Here it is impossible for a language to show agree

\begin{figure}[H]
  \centering
  \begin{tikzpicture}
    \draw (0,1) circle (2.25);
    \draw [fill opacity=0.4, fill=LG] (0,0.5) circle (1.75);
    \draw [fill opacity=0.4, fill=DG] (0,0) circle (1.25);

    \node[] at (0,2.75) {subject};
    \node[] at (0,1.5) {direct object};
    \node[align=center] at (0,0) {indirect\\ object};

    \node[] at (2.5,3) {\footnotesize{● Chinese}};
    \node[] at (2.25,2) {\footnotesize{● German}};
    \node[] at (2,1) {\footnotesize{● Hungarian}};
    \node[] at (1.375,0) {\footnotesize{● Basque}};
  \end{tikzpicture}
  \caption{\posscitealt{moravcsik1978} schema}
  \label{fig:subdoio}
\end{figure}


\citealt{moravcsik1978} with subject, object and indirect object.

Mandarin Chinese is an example of a language that does not show any agreement on the predicate. An example is given in \ref{ex:chineseagr}. The predicate \tit{gěi} `give' does not agree with the subject \tit{nǐ} `you', with the direct object \tit{shū} `book' or with the indirect object \tit{wǒ} `me'.

\exg. Nǐ bǎ shū gěi wǒ -le.\\
 you ba book give me -\ac{asp}\\
 `You gave me the book.' \flushfill{Mandarin Chinese, Zheng Shen p.c.}\label{ex:chineseagr}

German is an example of a language that shows agreement with the subject of the clause. An example is given in \ref{ex:germanagr}. The predicate \tit{gibst} `give' contains the morpheme \tit{-st}. This morpheme is the agreement morpheme for second person singular subjects. The predicate \tit{gibst} `give' agrees in person and number with the subject \tit{du} `you'. There is no agreement with the direct object \tit{das Buch} `the book' or the indirect object \tit{mir} `me'.

\exg. Du gib -st mir das Buch.\\
 you give -\tbf{2\ac{sg}} me the book\\
 `You give me the book.' \flushfill{German}\label{ex:germanagr}

Hungarian is an example of a language that shows agreement with the subject and the direct object of a clause. An example is given in \ref{ex:hungarianagr}. The predicate \tit{adom} `give' contains the morpheme \tit{-om}. This is a portmonteau morpheme for a first person singular subject and a third person object agreement. The predicate \tit{adom} `give' agrees with the subject \tit{én} `I' and the direct object \tit{a könyvet} `the book'. There is no agreement with the indirect object \tit{neked} `you'.

\exg. (Én) neked ad -om a könyv -et\\
 I you.\tsc{dat}.\tsc{sg} give -\tbf{\tsc{1sg}.\tsc{sbj}>3.\tsc{obj}} the book -\tsc{acc}\\
 `I give you the book.' \flushfill{Hungarian, András Bárány p.c.}\label{ex:hungarianagr}

Basque is an example of language that shows agreement with the subject, the direct object and the indirect object. Basque is an ergative-absolutive language, so in transitive clauses subjects are marked as ergative and objects are marked as absolutive. An example from the Bizkaian dialect is given in \ref{ex:basqueagr}. The stem of the auxiliary \tit{aus} combines with the morphemes \tit{d-}, \tit{-ta} and \tit{-zu}. The morpheme \tit{d-} is the agreement morpheme for third person singular as direct objects, which is here \tit{liburua} `the book'. The morpheme \tit{-ta} is the agreement morpheme for first person singular indirect objects, which is here \tit{niri} `me'. The morpheme \tit{-zu} is the agreement morpheme for second person singular ergative subjects, which is here \tit{zuk} `you'.

\exg. Zu -k ni -ri liburu -a emon d -aus -ta -zu.\\
 you -\ac{erg} me -\ac{dat} book -\ac{def}.\ac{abs} given \tbf{\ac{abs}.3\ac{sg}} -\ac{aux} \tbf{-\ac{dat}.1\ac{sg}} \tbf{-\ac{erg}.2\ac{sg}}\\
 `You gave me the book.' \flushfill{Bizkaian Basque, \pgcitealt{arregi2004}{45}}\label{ex:basqueagr}

\posscitealt{gilligan1987} typological study confirms the picture.

 \begin{table}[H]
   \center
   \caption {Agreement accessibility}
     \begin{tabular}[t]{ccccc}
       \toprule
             \multicolumn{3}{c}{agreement with}
             &
           & \\
       \cmidrule{1-3}
             & direct
             & indirect
             & number
           & \\
             subject
             & object
             & object
             & of languages
           & example \\
       \cmidrule{1-3} \cmidrule{4-4} \cmidrule{5-5}
             *
             & *
             & *
             & 23
           & Mandarin Chinese \\
             ✔
             & *
             & *
             & 31
           & German \\
             ✔
             & ✔
             & *
             & 25
           & Hungarian \\
             ✔
             & ✔
             & ✔
             & 23
           & Basque \\
             ✔
             & *
             & ✔
             & (1)
           & - \\
             {*}
             & ✔
             & ✔
             & 0
           & - \\
             {*}
             & x
             & *
             & 0
           & - \\
             {*}
             & *
             & ✔
             & 0
           & - \\
       \bottomrule
     \end{tabular}
 \end{table}

\citealt{bobaljik2006} makes it default/dependent/dative

\begin{figure}[H]
  \centering
  \begin{tikzpicture}
    \draw (0,1) circle (2.25);
    \draw [fill opacity=0.4, fill=LG] (0,0.5) circle (1.75);
    \draw [fill opacity=0.4, fill=DG] (0,0) circle (1.25);

    \node[] at (0,2.75) {default case};
    \node[] at (0,1.5) {dependent case};
    \node[] at (0,0) {dative};

    \node[] at (2.5,3) {\footnotesize{● Chinese}};
    \node[] at (2.25,2) {\footnotesize{● German}};
    \node[] at (2,1) {\footnotesize{● Hungarian}};
    \node[] at (1.375,0) {\footnotesize{● Basque}};
  \end{tikzpicture}
  \caption{\posscitealt{bobaljik2006} schema}
  \label{fig:defdepdat}
\end{figure}

For this dissertation, this translates to this:

\begin{figure}[H]
  \centering
  \begin{tikzpicture}
    \draw (0,1) circle (2.25);
    \draw [fill opacity=0.4, fill=LG] (0,0.5) circle (1.75);
    \draw [fill opacity=0.4, fill=DG] (0,0) circle (1.25);

    \node[] at (0,2.75) {\tsc{nom}};
    \node[] at (0,1.75) {\tsc{acc}};
    \node[] at (0,0) {\tsc{dat}};

    \node[] at (2.5,3) {\footnotesize{● Chinese}};
    \node[] at (2.25,2) {\footnotesize{● German}};
    \node[] at (2,1) {\footnotesize{● Hungarian}};
    \node[] at (1.375,0) {\footnotesize{● Basque}};
  \end{tikzpicture}
  \label{fig:nomaccdat}
  \caption{Schema for this dissertation}
\end{figure}



\subsection{Relativization}

Keenan Comrie sub/obj/ind obj


\begin{figure}[H]
  \centering
  \begin{tikzpicture}
    \draw (0,1) circle (2.25);
    \draw [fill opacity=0.4, fill=LG] (0,0.5) circle (1.75);
    \draw [fill opacity=0.4, fill=DG] (0,0) circle (1.25);

    \node[] at (0,2.75) {subject};
    \node[] at (0,1.5) {direct object};
    \node[align=center] at (0,0) {indirect\\ object};

    \node[] at (2.25,2) {\footnotesize{● Malagasy}};
    \node[] at (2,1) {\footnotesize{● Malay}};
    \node[] at (1.375,0) {\footnotesize{● Basque}};
  \end{tikzpicture}
  \caption{Schema for relativization}
  \label{fig:relativization}
\end{figure}

Malagasy is an example of a language that allows subjects to be relativized, but not direct and indirect objects. \ref{ex:malagasy-decl} is an example of a declarative sentence in Malagasy. It is a transitive that contains the subject \tit{ny mpianatra} `the student' and the direct object \tit{ny vehivavy} `the woman'.

\exg. Nahita ny vehivavy ny mpianatra.\\
 saw the woman the student\\
 `The student saw the woman.' \flushfill{Malagasy, \pgcitealt{keenan1977}{70}}\label{ex:malagasy-decl}

In \ref{ex:malagasy-sub}, the subject from the declarative sentence is relativized. The subject \tit{ny mpianatra} `the student' appears in the first position of the clause. It is followed by the invariable relativizer \tit{izay} `that'. After that, the rest of the relative clause follows, in this case \tit{nahita ny vehivavy} `saw the woman'.

\exg. ny mpianatra izay nahita ny vehivavy\\
 the student that saw the woman\\
 `the student that saw the woman' \flushfill{Malagasy, \pgcitealt{keenan1977}{70}}\label{ex:malagasy-sub}

The object of \ref{ex:malagasy-decl} cannot be relativized in the same way, as shown in \ref{ex:malagasy-no-do}. Here the object \tit{ny vehivavy} `the woman' appears in the first position of the clause. It is again followed by the relativizer \tit{izay} `that' and the rest of the relative clause, which is here \tit{nahita ny mpianatra} `saw the student'. This example is ungrammatical.

\exg. *ny vehivavy izay nahita ny mpianatra\\
 the woman that saw the student\\
 `the woman that the student saw' \flushfill{Malagasy, \pgcitealt{keenan1977}{70}}\label{ex:malagasy-no-do}

% Objects can be passivized and then relativized (again relativization of a subject).
%
% \ex.
% \ag. Nohitan' ny mpianatra ny vehivavy.\\
%  seen.\ac{pass} the student the woman\\
%  `The woman was seen by the student.'\label{ex:}
% \bg. ny vehivavy izay nohitan' ny mpianatra\\
%  the woman that seen.\ac{pass} the student\\
%  `the woman that was seen by the student' \flushfill{Malagasy, \pgcitealt{keenan1977}{70}}

Malay is an example of a language that has a relativization strategy for subjects and direct objects, but not for indirect objects. \ref{ex:malay-do} shows an example in which the object is relativized. The object here is \tit{ayam} `chicken'. It is followed by the relativizer \tit{yang} `that'. After that, the rest of the relative clause \tit{Aminah sedang memakan} `Aminah is eating' follows. The same strategy works to relativize subjects.

\exg. Ali bunoh ayam yang Aminah sedang memakan.\\
 Ali kill chicken that Aminah \ac{prog} eat\\
 `Ali killed the chicken that Aminah is eating.' \flushfill{Malay, \pgcitealt{keenan1977}{71}}\label{ex:malay-do}

Indirect objects cannot be relativized using the same strategy. \ref{ex:malay-decl} is an example of a ditransitive sentence in Malay. The indirect object \tit{kapada perempuan itu} `to the woman' cannot be relativized using \tit{yang}.

\exg. Ali beri {ubi kentang} itu kapada perempuan itu.\\
 Ali give potato the to woman the\\
 `Ali gave the potato to the woman.'\label{ex:malay-decl} \flushfill{Malay, \pgcitealt{keenan1977}{71}}\label{ex:malay-no-io2}

This is illustrated by the examples in \ref{ex:malay-no-io}. In \ref{ex:malay-no-io1}, the direct object \tit{perempuan kapada} `to the woman' appears in the first position of the clause. It is followed by the relativizer \tit{yang} `that' and the rest of the relative clause \tit{Ali beri ubi kentang itu} `Ali gave the potato to'. This example in ungrammatical. The example in \ref{ex:malay-no-io2} differs from \ref{ex:malay-no-io} in that the preposition \tit{kapada} `to' has been stranded in the relative clause. This example is ungrammatical as well, indicating this was not the reason for the ungrammaticality.

\ex.\label{ex:malay-no-io}
\ag. *perempuan kapada yang Ali beri {ubi kentang} itu\\
 woman to who Ali give potato that\\ \label{ex:malay-no-io1}
\bg. *perempuan yang Ali beri {ubi kentang} itu kapada\\
 woman that Ali give potato the to\\ \flushfill{Malay, \pgcitealt{keenan1977}{71}}\label{ex:malay-no-io2}

Basque is an example of a language that has a relativization strategy for subjects, direct objects and indirect objects. \ref{ex:basque-decl} is an example of a declarative ditransitive sentence in Basque. The sentence contains the subject \tit{gizonak} `the man', the direct object \tit{liburua} `the book' and the indirect object \tit{emakumeari} `the woman'.

\exg. Gizon-a-k emakume-a-ri liburu-a eman dio.\\
 man-\ac{def}-\ac{erg} woman-\ac{def}-\ac{dat} book-\ac{def}.\ac{abs} give has\\
 `The man has given the book to the woman.' \flushfill{Basque, \pgcitealt{keenan1977}{72}}\label{ex:basque-decl}

A relative clause in Basque appears in the prenominal position and it is marked by the invariable marker \tit{-n}.\footnote{
additionally, the relatized positions do not appear in verbal agreement anymore, but this not visible in the example, because they are all phonologicall zero(?).
}
\ref{ex:malagasy-sub} shows examples or relativeiations.


%%%%%


\ex.\label{ex:basque-rel}
\bg. emakume-a-ri liburu-a eman dio-n gizon-a\\
 woman-\ac{def}-\ac{dat} book-\ac{def}.\ac{abs} give has-\ac{rel} man-\ac{def}\\
 `the man who has given the book to the woman'\label{ex:basque-sub}
\bg. gizon-a-k emakume-a-ri eman dio-n liburu-a\\
 man-\ac{def}-\ac{erg} woman-\ac{def}-\ac{dat} give has-\ac{rel} book-\ac{def}\\
 `the book that the man has given to the woman'\label{ex:basque-do}
\bg. gizon-a-k liburu-a eman dio-n emakume-a\\
 man-\ac{def}-\ac{erg} book-\ac{def}.\ac{abs} give has-\ac{rel} woman-\ac{def}\\
 `the woman that the man has given the book to' \flushfill{Basque, \pgcitealt{keenan1977}{72}}\label{ex:basque-io}




 \begin{table}[H]
   \center
   \caption {Relativization accessibility}
     \begin{tabular}{cccc}
       \toprule
             \multicolumn{3}{c}{relativization of}
           & \\
       \cmidrule{1-3}
             & direct
             & indirect
           & \\
             subject
             & object
             & object
           & example \\
       \cmidrule{1-3} \cmidrule{4-4}
             ✔
             & *
             & *
           & Malagasy \\
             ✔
             & ✔
             & *
           & Malay \\
             ✔
             & ✔
             & ✔
           & Basque \\
       \bottomrule
     \end{tabular}
 \end{table}



Caha nom/acc/dat




\section{Case in morphology}\label{sec:casemorphology}




\subsection{Syncretism patterns}

Icelandic: \pgcitealt{einarsson1949}{68}
Teribe: ?
Lavukaleve: Yvonne?
Khinalugh: Beata etc.


\begin{table}[H]
  \center
  \caption {Syncretism patterns}
    \begin{tabular}{cccccccc}
      \toprule
          \multicolumn{3}{c}{pattern}
            & \ac{nom}
            & \ac{acc}
            & \ac{dat}
            & translation
            & language \\
      \cmidrule(lr){1-3} \cmidrule(lr){4-6} \cmidrule(lr){7-7} \cmidrule(lr){8-8}
          A & A & A
            & \cellcolor{LG}\tbf{inu}
            & \cellcolor{LG}\tbf{inu}
            & \cellcolor{LG}\tbf{inu}
            & 2\ac{pl}
            & Lavukaleve \\
          A & B & B
            & ta
            & \cellcolor{LG}\tbf{bor}
            & \cellcolor{LG}\tbf{bor}
            & 1\ac{pl}
            & Teribe \\
          A & A & B
            & \cellcolor{LG}\tbf{það}
            & \cellcolor{LG}\tbf{það}
            & því
            & 3\ac{pl}.\ac{n}
            & Icelandic \\
          A & B & C
            & zɨ
            & jä
            & as(ɨr
            & 1\ac{sg}
            & Khinalugh \\
          A & B & A
            & \cellcolor{LG}
            &
            & \cellcolor{LG}
            &
            & not attested \\
      \bottomrule
    \end{tabular}
\end{table}







\ex. \ac{nom} < \ac{acc} < \ac{dat}


\subsection{Morphological containment}

\pgcitealt{nikolaeva1999}{16}

\begin{table}[H]
  \center
	\caption {Case containment in Khanty}
		\begin{tabular}{clll}
		\toprule
              & \ac{1}\ac{sg}
              & \ac{3}\ac{sg}
              & \ac{1}\ac{pl}                           \\
		          \cmidrule{2-4}
    \ac{nom}  & ma
              & luw
              & muŋ                                     \\
    \ac{acc}  & ma\tbf{:-ne:m}
              & luw\tbf{-e:l}
              & muŋ\tbf{-e:w}                           \\
    \ac{dat}  & ma\tbf{:-ne:m}-\tcol{DG}{\tbf{na}}
              & luw\tbf{-e:l}-\tcol{DG}{\tbf{na}}
              & muŋ\tbf{-e:w}-\tcol{DG}{\tbf{na}}  \\
		\bottomrule
		\end{tabular}
\end{table}


\pgcitealt{boretzky1994}{31-46}

\begin{table}[H]
  \center
	\caption {Case containment in Kalderaš Romani}
		\begin{tabular}{cllll}
		\toprule
              & `brother'
              & `brothers'
              & `girl'
              & `girls'                                   \\
		\cmidrule{2-5}
    \ac{nom}  & phral
              & phral-(á)
              & rakl-í
              & rakl-já                                   \\
    \ac{acc}  & phral-\tbf{és}
              & phral-\tbf{én}
              & rakl-\tbf{já}
              & rakl-já-\tbf{n}                           \\
    \ac{dat}  & phral-\tbf{és}-\tcol{DG}{\tbf{kə}}
              & phral-\tbf{én}-\tcol{DG}{\tbf{gə}}
              & rakl-\tbf{já}-\tcol{DG}{\tbf{kə}}
              & rakl-já-\tbf{n}-\tcol{DG}{\tbf{gə}}  \\
		\bottomrule
		\end{tabular}
\end{table}

\pgcitealt{gippert1987}{23-24}

\begin{table}[H]
  \center
	\caption {Case containment in West Tocharian}
		\begin{tabular}{cll}
		\toprule
              & `horses'
              & `men'                                  \\
		\cmidrule{2-3}
    \ac{nom}  & yakwi
              & eṅkwi                                  \\
    \ac{acc}  & yakwe-\tbf{ṃ}
              & eṅkwe-\tbf{ṃ}                          \\
    \ac{dat}  & yäkwe-\tbf{ṃ}-\tcol{DG}{\tbf{ts}}
              & eṅkwe-\tbf{ṃ}-\tcol{DG}{\tbf{ts}} \\
		\bottomrule
		\end{tabular}
\end{table}

\ex. \ac{nom} < \ac{acc} < \ac{dat}

\phantom{nom}




\section{A side note on the genitive}\label{sec:genitive}

\begin{itemize}
  \item possessive
  \item accessibility hierarchy
  \item not available
\end{itemize}
