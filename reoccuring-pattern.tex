% !TEX root = thesis.tex

\chapter{A reoccurring pattern}

First I introduce the pattern that forms the focus of the first part of the dissertation. I show that headless relatives in Gothic adhere to the case strength scale: \ac{nom} < \ac{acc} < \ac{dat}.

Then I show two phenomena that follow the same ordering of \ac{nom}, \ac{acc} and \ac{dat}. The two phenomena are accessibility hierarchies. The first one is about agreement, the second one about relativization.

In the last section of this chapter I discuss how \ac{nom}, \ac{acc} and \ac{dat} pattern in morphology.


\section{Case competition in Gothic headless relatives}

In this section I show the behavior of Gothic headless relatives in detail. I systematically go through all case combinations, except for the genitive, which I discuss in Section \ref{sec:genitive}. This leaves the nominative, accusative and dative.




I repeat the headless relative from the introduction in which both predicates assign accusative case in \ref{ex:gothicaccaccrep}. The predicate in the relative clause \tit{arma} `pity' assigns accusative case. The predicate in the main clause \tit{gaarma} `pity' also assigns accusative case. The relative pronoun \tit{þan(a)} `who.\ac{acc}' appears in accusative case.

\exg. gaarma þan -ei arma\\
 pity\scsub{[acc]} \tcol{DG}{who.\ac{acc}} \tcol{DG}{-\ac{comp}} \tcol{DG}{pity\scsub{[acc]}}\\
 `I will pity (him) whom I pity' \flushfill{Gothic, \ac{rom} 9:15, after \pgcitealt{harbert1978}{339}}\label{ex:gothicaccaccrep}





 Before I discuss evidence for this claim, let me lay out three some terminology and notational conventions I use throughout this dissertation. I use them to make examples and tables as easy to understand as possible, and to allow to avoid long and complicated phrasing.

 First, discussing headless relatives, I use the terms internal and external case.
 Internal case refers to the case associated with the relative pronoun internal to the relative clause. More precisely, it is the case, which is associated with the grammatical role that the relative pronoun has internal to the relative clause. In \ref{ex:gothicaccacc}, the relative pronoun is the object of \tit{arma} `pity'. The predicate \tit{arma} `pity' takes accusative objects. So, the internal case is accusative.
 External case refers to the case associated with the missing head in the main clause, which is external to the relative clause. Concretely, it is the case which is associated with the grammatical role that the missing head has external to the relative clause. In \ref{ex:gothicaccacc}, the missing head is the object of \tit{gaarma} `pity'. The predicate \tit{gaarma} `pity' takes accusative objects. In \ref{ex:gothicaccacc}, the external case is accusative.

 Second, I place subscripts on the glosses of the predicates. They indicate what the internal or external case is. The subscript on the predicate in the relative clause indicates the internal case. The subscript on the predicate in the main clause indicates the external case.
 For \ref{ex:gothicaccacc} the subscript indicates which case the complement of the verb appears in. Another possibility is that the subscript is placed on a preposition and refers to the case the preposition combines with. A last possibility is that the subscript is [\tsc{nom}] and refers to the case in which the subject appears in.

 Third and last, for ease of exposition, I write the relative clause in gray, as in \ref{ex:gothicaccacc}.





% From this section on I use the terms internal case and external case. Internal case refers to the case assigned internal to the relative clause. In \ref{ex:gothicaccaccrep}, the internal case is assigned by \tit{arma} `pity'. The external case refers to the case assign external to the relative clause, so the main clause. In \ref{ex:gothicaccaccrep}, the external case is assigned by \tit{gaarma} `pity'.
%
% I placed subscripts between the square brackets on the glosses of verbs. They indicate which case the predicate requires from the relative pronoun. For \ref{ex:gothicaccaccrep} this simply means which case the verb assigns to its object. Another possibility is that the subscript is placed on a preposition and refer to the case the preposition assigns. A last possibility is that the subscript is [\tsc{nom}] and refers to the case in which the subject appears in. An example of that is given in \ref{ex:gothicnomnom}.

In this example the predicate \tit{matjai} `eats' combines with a nominative subject in the relative clause. In other words, the internal case is nominative. The predicate \tit{gadauþnai} `die' in the main clause also combines with a nominative subject. In other words, the external case is nominative. The relative pronoun \tit{sa} `who.\tsc{nom}' appears in nominative case.

\exg. ei sa -ei þis matjai, ni gadauþnai\\
 that \tcol{DG}{who.\ac{nom}} \tcol{DG}{-\ac{comp}} \tcol{DG}{{of this}} \tcol{DG}{eats\scsub{[nom]}} not die\scsub{[nom]}\\
 `that (he) who eats of this may not die' \flushfill{Gothic, \ac{john} 6:50, after \pgcitealt{harbert1978}{337}}\label{ex:gothicnomnom}

In the examples below, the internal case and external case are dative.

\ex.
\ag. þamm -ei gabaur gabaur\\
 \tcol{DG}{who.\ac{dat}} \tcol{DG}{-\ac{comp}} \tcol{DG}{tribute\scsub{[dat]}} tribute\scsub{[dat]}\\
 `tribute to (him) whom tribute is due'
\bg. þamm -ei mota mota\\
 \tcol{DG}{who.\ac{dat}} \tcol{DG}{-\ac{comp}} \tcol{DG}{custom\scsub{[dat]}} custom\scsub{[dat]}\\
 `custom to (him) whom custom is due'
\bg. þamm -ei agis agis\\
 \tcol{DG}{who.\ac{dat}} \tcol{DG}{-\ac{comp}} \tcol{DG}{fear\scsub{[dat]}} fear\scsub{[dat]}\\
 `fear to (him) whom fear is due'
\bg. þamm -ei sweriþa sweriþa\\
 \tcol{DG}{who.\ac{dat}} \tcol{DG}{-\ac{comp}} \tcol{DG}{honour\scsub{[dat]}} honour\scsub{[dat]}\\
 `honour to (him) whom honour is due' \flushfill{Gothic, \ac{rom} 13:7, after \pgcitealt{harbert1978}{339}}\label{ex:gothicdatdat}

Schematically, this looks like:

\begin{table}[H]
  \center
  \caption {Case attraction in headless relatives - only matching}
    \begin{tabular}{c|c|c|c}
      \toprule
        \diagbox[linecolor=white]{\ac{int}}{\ac{ext}}
            & [\ac{nom}]
            & [\ac{acc}]
            & [\ac{dat}]
            \\ \cmidrule{1-4}
        [\ac{nom}]
            & \colorbox{LG}{\ac{nom}}
            & \diagbox[linecolor=white]{\phantom{nom}}{\phantom{nom}}
            & \diagbox[linecolor=white]{\phantom{nom}}{\phantom{nom}}
            \\ \cmidrule{1-4}
        [\ac{acc}]
            & \diagbox[linecolor=white]{\phantom{nom}}{\phantom{nom}}
            & \colorbox{LG}{\ac{acc}}
            & \diagbox[linecolor=white]{\phantom{nom}}{\phantom{nom}}
            \\ \cmidrule{1-4}
        [\ac{dat}]
            & \diagbox[linecolor=white]{\phantom{nom}}{\phantom{nom}}
            & \diagbox[linecolor=white]{\phantom{nom}}{\phantom{nom}}
            & \colorbox{LG}{\ac{dat}}
            \\
      \bottomrule
    \end{tabular}
\end{table}

In what follows I discuss the pattern that occurs when the internal and external case differs.

First, consider a situation in which the internal case is nominative and the external case is accusative. An attested example of that is given in  \ref{ex:gothicnomacc}. Internal to the relative clause, the predicate \tit{ist us Laudeikaion} `is from Laodicea' requires a subject in nominative case. External to the relative clause, \tit{ussiggwaid} `read' requires its object to be in accusative. In the example, the relative pronoun \tit{þo} `what.\tsc{acc}' appears in accusative case. There are no examples of headless relatives with nominative as internal case and accusative as external case and a relative pronoun in nominative.

\exg. jah þo -ei ist us Laudeikaion jus ussiggwaid\\
 and what.\ac{acc} \tcol{DG}{-\ac{comp}} \tcol{DG}{is\scsub{[nom]}} \tcol{DG}{from} \tcol{DG}{Laodicea} you read\scsub{[acc]}\\
 `and read that which is from Laodicea' \flushfill{Gothic, \ac{col} 4:16, after \pgcitealt{harbert1978}{357}}\label{ex:gothicnomacc}

% Consider a situation in which the internal case is nominative and the external case is dative. An attested example of that is given in  \ref{ex:gothicnomdat}. Internal to the relative clause, the predicate \tit{iupa sind} `are above' requires a subject in nominative case. External to the relative clause, \tit{fraþjaiþ} `think on' requires its object to be in dative. In the example, the relative pronoun \tit{þaim} `what.\tsc{dat}' appears in dative case. There are no examples of headless relatives with nominative as internal case and dative as external case and a relative pronoun in nominative.

Consider the example in \ref{ex:gothicnomdat}. The relative clause is marked in gray.
The internal case is nominative. The predicate \tit{sind fraþjaiþ} `are above' takes a nominative subject, as indicated by the subscript on the predicate.
The external case is dative. The predicate \tit{fraþjaiþ} `think on' takes dative objects, indicated by the subscript on the verb.
The relative pronoun \tit{þaim} appears in dative. This dative can only come from the predicate \tit{fraþjaiþ} `think on'.
There are no attested examples of headless relatives with the nominative as internal case and the dative as external case in which the relative pronoun appears in nominative case.

\exg. þaim -ei iupa sind fraþjaiþ \\
 what.\ac{dat} \tcol{DG}{-\ac{comp}} \tcol{DG}{above} \tcol{DG}{are\scsub{[nom]}} {think on}\scsub{[dat]}\\
 `set your mind on those which are above' \flushfill{Gothic, \ac{col} 3:2, after \pgcitealt{harbert1978}{339}}\label{ex:gothicnomdat}

Consider a situation in which the internal case is accusative and the external case is nominative. An attested example of that is given in  \ref{ex:gothicaccnom}. Internal to the relative clause, \tit{frijos} `love' requires its object to be in accusative. External to the relative clause, the predicate \tit{siuks ist} `is sick' requires a subject in nominative case. In the example, the relative pronoun \tit{þan(a)} `who.\tsc{acc}' appears in accusative case. There are no examples of headless relatives with accusative as internal case and nominative as external case and a relative pronoun in nominative.

\exg. þan -ei frijos siuks ist\\
 \tcol{DG}{who.\ac{acc}} \tcol{DG}{-\ac{comp}} \tcol{DG}{love\scsub{[acc]}} sick is\scsub{[nom]}\\
 `the one whom you love is sick' \flushfill{Gothic, \ac{john} 11:3, after \pgcitealt{harbert1978}{342}}\label{ex:gothicaccnom}

Consider a situation in which the internal case is accusative and the external case is dative. An attested example of that is given in  \ref{ex:gothicaccdatrep}, repeated from the introduction. Internal to the relative clause, \tit{qiþiþ} `say' requires its object to be in accusative. External to the relative clause, \tit{taujau} `do' requires its direct object to be in dative case. In the example, the relative pronoun \tit{þamm(a)} `who.\tsc{dat}' appears in dative case. There are no examples of headless relatives with accusative as internal case and dative as external case and a relative pronoun in accusative.

\exg. hva nu wileiþ ei taujau þamm -ei qiþiþ þiudan Iudaie?\\
 what now want that do\scsub{[dat]} who.\ac{dat} \tcol{DG}{-\ac{comp}} \tcol{DG}{say\scsub{[acc]}} \tcol{DG}{king} \tcol{DG}{{of Jews}}\\
 `what now do you wish that I do to (him) whom you call King of the Jews?' \flushfill{Gothic, \ac{mark} 15:12, after \pgcitealt{harbert1978}{339}}\label{ex:gothicdataccrep}

Consider a situation in which the internal case is dative and the external case is nominative. An attested example of that is given in  \ref{ex:gothicdatnom}. Internal to the relative clause, \tit{fraletada} `is forgiven' requires its object to be in dative. External to the relative clause, the predicate \tit{frijod} `loves' requires a subject in nominative case. In the example, the relative pronoun \tit{þamm(a)} `who.\tsc{dat}' appears in dative case. There are no examples of headless relatives with dative as internal case and nominative as external case and a relative pronoun in nominative.

\exg. iþ þamm -ei leitil fraletada leitil frijod\\
 but \tcol{DG}{who.\ac{dat}} \tcol{DG}{-\ac{comp}} \tcol{DG}{little} \tcol{DG}{{is forgiven}\scsub{[dat]}} little loves\scsub{[nom]}\\
 `but the one whom little is forgiven loves little' \flushfill{Gothic, \ac{luke} 7:47, after \pgcitealt{harbert1978}{342}}\label{ex:gothicdatnom}

Consider a situation in which the internal case is dative and the external case is accusative. An attested example of that is given in  \ref{ex:gothicdataccrep}, repeated from the introduction. Internal to the relative clause, the preposition \tit{ana} `on' requires its complement to be in dative. External to the relative clause, \tit{ushafjands} `picking up' requires its object to be in accusative case. In the example, the relative pronoun \tit{þamm(a)} `who.\tsc{dat}' appears in dative case. There are no examples of headless relatives with dative as internal case and accusative as external case and a relative pronoun in accusative.

\exg. ushafjands ana þamm -ei lag\\
 {picking up}\scsub{[acc]} \tcol{DG}{on\scsub{[dat]}} \tcol{DG}{what.\ac{dat}} \tcol{DG}{-\ac{comp}} \tcol{DG}{lay}\\
 `picking up (that) on which he lay' \flushfill{Gothic, \ac{luke} 5:25, after \pgcitealt{harbert1978}{343}}\label{ex:gothicaccdatrep}

A summary of the data is given in Table \ref{tbl:summarygothic}. The left column gives the internal case between square brackets. The upper row indicates the external case between square brackets.
The other cells show the case in of the relative pronoun. In the diagonal there is only a single case. These are the headless relatives in which the internal case is identical to the external case. The relative pronoun is grammatical and marked dark gray. they correspond to the examples x, y and z.
Six cells show internal and external case differ. The lower left corner shows the internal case pronoun. The upper right corner shows the external case pronoun. The grammatical ones are marked in light gray. The unattested examples are marked with an asterix, and are not marked.\footnote{
Throughout this dissertation * stands for 'not found in natural language'. For extinct languages this means that there are no attested examples. For modern languages it means that the examples are ungrammatical.
}
The pattern we see is that outer guys are grammatical.


\begin{table}[H]
  \center
  \caption {Summary of Gothic headless relative data}
    % !TEX root = ../thesis.tex

\begin{tabular}{c|c|c|c}
  \toprule
    \diagbox[linecolor=white]{\ac{int}}{\ac{ext}}
        & [\ac{nom}]
        & [\ac{acc}]
        & [\ac{dat}]
        \\ \cmidrule{1-4}
    [\ac{nom}]
        & 
        & \diagbox[linecolor=white]{*\ac{nom}}{\colorbox{LG}{\ac{acc}}}
        & \diagbox[linecolor=white]{*\ac{nom}}{\colorbox{LG}{\ac{dat}}}
        \\ \cmidrule{1-4}
    [\ac{acc}]
        & \diagbox[linecolor=white]{\colorbox{LG}{\ac{acc}}}{*\ac{nom}}
        &
        & \diagbox[linecolor=white]{*\ac{acc}}{\colorbox{LG}{\ac{dat}}}
        \\ \cmidrule{1-4}
    [\ac{dat}]
        & \diagbox[linecolor=white]{\colorbox{LG}{\ac{dat}}}{*\ac{nom}}
        & \diagbox[linecolor=white]{\colorbox{LG}{\ac{dat}}}{*\ac{acc}}
        &
        \\
  \bottomrule
\end{tabular}

    \label{tbl:summarygothic}
\end{table}


In sum, the situation can be summarized as in \ref{ex:competition1by1}. In a competition, accusative and dative win over nominative. Additionally, dative wins over accusative.

\ex.\label{ex:competition1by1}
\a. \tsc{acc} wins over \tsc{nom}
\b. \tsc{dat} wins over \tsc{nom}
\b. \tsc{dat} wins over \tsc{acc}

Formulated in a scale of `case strength':

\ex. \tsc{nom} -- \tsc{acc} -- \tsc{dat}\label{ex:casestrength}




\section{Parallels in accessibility hierarchies}



\subsection{Agreement}

Moravcsik, Gilligan sub, obj, ind obj

Bobaljik default/dependent/dative

Mandarin Chinese does not show any agreement on the verb. In German, the verb agrees with the subject. In Huallaga Quechua, the verb agrees with the subject and the object. In Basque, the verb agrees with the subject, the object and the indirect object.


\ex.
\ag. Nǐ bǎ shū gěi wǒ-le.\\
 you ba book give me-\ac{asp}\\
 `You gave me the book.' \flushfill{Mandarin Chinese}
\bg. Du gib -st mir das Buch.\\
 you give -\tbf{2\ac{sg}} me the book\\
 `You give me the book.' \flushfill{German}
\bg. tayta-yki qam-ta qu -maran\\
 father-your you-\ac{acc} give -\tbf{3\ac{sg}→1\ac{sg}}.\ac{pst}\\
 `Your father gave you to me.' \flushfill{Huallaga Quechua, \pgcitealt{weber1983}{21}}
\bg. Zu-k ni-ri liburu-a emon d -austa -zu.\\
 you-\ac{erg} me-\ac{dat} book-\ac{def}.\ac{acc} given \tbf{\ac{acc}.3\ac{sg}} \tbf{-\ac{dat}.1\ac{sg}} \tbf{-\ac{erg}.2\ac{sg}}\\
 `You gave me the book.' \flushfill{Basque, \pgcitealt{arregi2004}{45}}

\begin{table}[H]
  \center
  \caption {Agreement accessibility}
    \begin{tabular}[t]{ccccc}
      \toprule
            \multicolumn{3}{c}{agreement with}
            &
          & \\
      \cmidrule{1-3}
            & direct
            & indirect
            & number
          & \\
            subject
            & object
            & object
            & of languages
          & example \\
      \cmidrule{1-3} \cmidrule{4-4} \cmidrule{5-5}
            *
            & *
            & *
            & 23
          & Mandarin Chinese \\
            ✔
            & *
            & *
            & 31
          & German \\
            ✔
            & ✔
            & *
            & 25
          & Huallaga Quechua \\
            ✔
            & ✔
            & ✔
            & 23
          & Basque \\
            ✔
            & *
            & ✔
            & (1)
          & - \\
            {*}
            & ✔
            & ✔
            & 0
          & - \\
            {*}
            & x
            & *
            & 0
          & - \\
            {*}
            & *
            & ✔
            & 0
          & - \\
      \bottomrule
    \end{tabular}
\end{table}



\subsection{Relativization}

Keenan Comrie sub/obj/ind obj

Caha nom/acc/dat

In Malagasy, only subjects can be relativized.

\ex.
\ag. Nahita ny vehivavy ny mpianatra.\\
 saw the woman the student\\
 `The student saw the woman.'
\bg. ny mpianatra izay nahita ny vehivavy\\
 the student that saw the woman\\
 `the student that saw the woman'
\bg. *ny vehivavy izay nahita ny mpianatra\\
 the woman that saw the student\\
 `the woman that the student saw' \flushfill{Malagasy, \pgcitealt{keenan1977}{70}}

Objects can be passivized and then relativized (again relativization of a subject).

\ex.
\ag. Nohitan' ny mpianatra ny vehivavy.\\
 seen.\ac{pass} the student the woman\\
 `The woman was seen by the student.'
\bg. ny vehivavy izay nohitan' ny mpianatra\\
 the woman that seen.\ac{pass} the student\\
 `the woman that was seen by the student' \flushfill{Malagasy, \pgcitealt{keenan1977}{70}}

In Malay, subjects and objects can be relativized using \tit{yang}. Below I only give an example of a relativized object.

\exg. Ali bunoh ayam yang Aminah sedang memakan.\\
 Ali kill chicken that Aminah \ac{prog} eat\\
 `Ali killed the chicken that Aminah is eating.' \flushfill{Malay, \pgcitealt{keenan1977}{71}}

Indirect objects cannot be relativized in the same way.

\ex.
\ag. Ali beri {ubi kentang} itu kapada perempuan itu.\\
 Ali give potato the to woman the\\
 `Ali gave the potato to the woman.'
\bg. *perempuan yang Ali beri {ubi kentang} itu kapada\\
 woman that Ali give potato the to\\
\bg. *perempuan kapada yang Ali beri {ubi kentang} itu\\
 woman to who Ali give potato that\\ \flushfill{Malay, \pgcitealt{keenan1977}{71}}

A different construction is made.

\exg. perempuan yang menerima {ubi kentang} itu daripada Ali\\
 woman that received potato the from Ali\\
 `the woman that received the potato from Ali'\flushfill{Malay, \pgcitealt{keenan1977}{71}}

In Basque, subjects, objects and indirect objects can be relativized with the same strategy.

\ex.
\ag. Gizon-a-k emakume-a-ri liburu-a eman dio.\\
 man-\ac{def}-\ac{erg} woman-\ac{def}-\ac{dat} book-\ac{def}.\ac{acc} give has\\
 `The man has given the book to the woman.'
\bg. emakume-a-ri liburu-a eman dio-n gizon-a\\
 woman-\ac{def}-\ac{dat} book-\ac{def}.\ac{acc} give has-\ac{rel} man-\ac{def}\\
 `the man who has given the book to the woman'
\bg. gizon-a-k emakume-a-ri eman dio-n liburu-a\\
 man-\ac{def}-\ac{erg} woman-\ac{def}-\ac{dat} give has-\ac{rel} book-\ac{def}\\
 `the book that the man has given to the woman'
\bg. gizon-a-k liburu-a eman dio-n emakume-a\\
 man-\ac{def}-\ac{erg} book-\ac{def}.\ac{acc} give has-\ac{rel} woman-\ac{def}\\
 `the woman that the man has given the book to' \flushfill{Basque, \pgcitealt{keenan1977}{72}}




 \begin{table}[H]
   \center
   \caption {Relativization accessibility}
     \begin{tabular}{cccc}
       \toprule
             \multicolumn{3}{c}{relativization of}
           & \\
       \cmidrule{1-3}
             & direct
             & indirect
           & \\
             subject
             & object
             & object
           & example \\
       \cmidrule{1-3} \cmidrule{4-4}
             ✔
             & *
             & *
           & Malagasy \\
             ✔
             & ✔
             & *
           & Malay \\
             ✔
             & ✔
             & ✔
           & Basque \\
       \bottomrule
     \end{tabular}
 \end{table}






\section{Case in morphology}




\subsection{Syncretism patterns}

Icelandic: \pgcitealt{einarsson1949}{68}
Teribe: ?
Lavukaleve: Yvonne?
Khinalugh: Beata etc.


\begin{table}[H]
  \center
  \caption {Syncretism patterns}
    \begin{tabular}{cccccccc}
      \toprule
          \multicolumn{3}{c}{pattern}
            & \ac{nom}
            & \ac{acc}
            & \ac{dat}
            & translation
            & language \\
      \cmidrule(lr){1-3} \cmidrule(lr){4-6} \cmidrule(lr){7-7} \cmidrule(lr){8-8}
          A & A & A
            & \cellcolor{LG}\tbf{inu}
            & \cellcolor{LG}\tbf{inu}
            & \cellcolor{LG}\tbf{inu}
            & 2\ac{pl}
            & Lavukaleve \\
          A & B & B
            & ta
            & \cellcolor{LG}\tbf{bor}
            & \cellcolor{LG}\tbf{bor}
            & 1\ac{pl}
            & Teribe \\
          A & A & B
            & \cellcolor{LG}\tbf{það}
            & \cellcolor{LG}\tbf{það}
            & því
            & 3\ac{pl}.\ac{n}
            & Icelandic \\
          A & B & C
            & zɨ
            & jä
            & as(ɨr
            & 1\ac{sg}
            & Khinalugh \\
          A & B & A
            & \cellcolor{LG}
            &
            & \cellcolor{LG}
            &
            & not attested \\
      \bottomrule
    \end{tabular}
\end{table}







\ex. \ac{nom} < \ac{acc} < \ac{dat}


\subsection{Morphological containment}

\pgcitealt{nikolaeva1999}{16}

\begin{table}[H]
  \center
	\caption {Case containment in Khanty}
		\begin{tabular}{clll}
		\toprule
              & \ac{1}\ac{sg}
              & \ac{3}\ac{sg}
              & \ac{1}\ac{pl}                           \\
		          \cmidrule{2-4}
    \ac{nom}  & ma
              & luw
              & muŋ                                     \\
    \ac{acc}  & ma\tbf{:-ne:m}
              & luw\tbf{-e:l}
              & muŋ\tbf{-e:w}                           \\
    \ac{dat}  & ma\tbf{:-ne:m}-\tcol{DG}{\tbf{na}}
              & luw\tbf{-e:l}-\tcol{DG}{\tbf{na}}
              & muŋ\tbf{-e:w}-\tcol{DG}{\tbf{na}}  \\
		\bottomrule
		\end{tabular}
\end{table}


\pgcitealt{boretzky1994}{31-46}

\begin{table}[H]
  \center
	\caption {Case containment in Kalderaš Romani}
		\begin{tabular}{cllll}
		\toprule
              & `brother'
              & `brothers'
              & `girl'
              & `girls'                                   \\
		\cmidrule{2-5}
    \ac{nom}  & phral
              & phral-(á)
              & rakl-í
              & rakl-já                                   \\
    \ac{acc}  & phral-\tbf{és}
              & phral-\tbf{én}
              & rakl-\tbf{já}
              & rakl-já-\tbf{n}                           \\
    \ac{dat}  & phral-\tbf{és}-\tcol{DG}{\tbf{kə}}
              & phral-\tbf{én}-\tcol{DG}{\tbf{gə}}
              & rakl-\tbf{já}-\tcol{DG}{\tbf{kə}}
              & rakl-já-\tbf{n}-\tcol{DG}{\tbf{gə}}  \\
		\bottomrule
		\end{tabular}
\end{table}

\pgcitealt{gippert1987}{23-24}

\begin{table}[H]
  \center
	\caption {Case containment in West Tocharian}
		\begin{tabular}{cll}
		\toprule
              & `horses'
              & `men'                                  \\
		\cmidrule{2-3}
    \ac{nom}  & yakwi
              & eṅkwi                                  \\
    \ac{acc}  & yakwe-\tbf{ṃ}
              & eṅkwe-\tbf{ṃ}                          \\
    \ac{dat}  & yäkwe-\tbf{ṃ}-\tcol{DG}{\tbf{ts}}
              & eṅkwe-\tbf{ṃ}-\tcol{DG}{\tbf{ts}} \\
		\bottomrule
		\end{tabular}
\end{table}

\ex. \ac{nom} < \ac{acc} < \ac{dat}

\phantom{nom}




\section{A side note on the genitive}\label{sec:genitive}

\begin{itemize}
  \item possessive
  \item accessibility hierarchy
  \item not available
\end{itemize}
