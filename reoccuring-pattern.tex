% !TEX root = thesis.tex

\chapter{A reoccurring pattern}

Start with the main focus of this dissertation
A phenomenon goes parallel with the ordering


\section{Case competition in Gothic headless relatives}

In the introduction I already showed accusative vs. dative, and dative always won. Here I additionaly add nomniative. I will show that ordering of strenght is nom-acc-dat.

First some terminilogy.
Intern
Extern

\begin{table}[H]
  \center
  \caption {Case attraction in headless relatives - empty}
    \begin{tabular}{c|c|c|c}
      \toprule
        \diagbox[linecolor=white]{\ac{int}}{\ac{ext}}
            & [\ac{nom}]
            & [\ac{acc}]
            & [\ac{dat}]
            \\ \cmidrule{1-4}
        [\ac{nom}]
            & \diagbox[linecolor=white]{\phantom{nom}}{\phantom{nom}}
            & \diagbox[linecolor=white]{\phantom{nom}}{\phantom{nom}}
            & \diagbox[linecolor=white]{\phantom{nom}}{\phantom{nom}}
            \\ \cmidrule{1-4}
        [\ac{acc}]
            & \diagbox[linecolor=white]{\phantom{nom}}{\phantom{nom}}
            & \diagbox[linecolor=white]{\phantom{nom}}{\phantom{nom}}
            & \diagbox[linecolor=white]{\phantom{nom}}{\phantom{nom}}
            \\ \cmidrule{1-4}
        [\ac{dat}]
            & \diagbox[linecolor=white]{\phantom{nom}}{\phantom{nom}}
            & \diagbox[linecolor=white]{\phantom{nom}}{\phantom{nom}}
            & \diagbox[linecolor=white]{\phantom{nom}}{\phantom{nom}}
            \\
      \bottomrule
    \end{tabular}
\end{table}




\subsection{matching}


\begin{table}[H]
  \center
  \caption {Case attraction in headless relatives - only matching}
    \begin{tabular}{c|c|c|c}
      \toprule
        \diagbox[linecolor=white]{\ac{int}}{\ac{ext}}
            & [\ac{nom}]
            & [\ac{acc}]
            & [\ac{dat}]
            \\ \cmidrule{1-4}
        [\ac{nom}]
            & \colorbox{LG}{\ac{nom}}
            & \diagbox[linecolor=white]{\phantom{nom}}{\phantom{nom}}
            & \diagbox[linecolor=white]{\phantom{nom}}{\phantom{nom}}
            \\ \cmidrule{1-4}
        [\ac{acc}]
            & \diagbox[linecolor=white]{\phantom{nom}}{\phantom{nom}}
            & \colorbox{LG}{\ac{acc}}
            & \diagbox[linecolor=white]{\phantom{nom}}{\phantom{nom}}
            \\ \cmidrule{1-4}
        [\ac{dat}]
            & \diagbox[linecolor=white]{\phantom{nom}}{\phantom{nom}}
            & \diagbox[linecolor=white]{\phantom{nom}}{\phantom{nom}}
            & \colorbox{LG}{\ac{dat}}
            \\
      \bottomrule
    \end{tabular}
\end{table}





\subsection{non-matching}

\ex. \ac{int}:\ac{nom}, \ac{ext}:\ac{acc}
\ag. jah [þo -ei ist us Laudeikaion] jus ussiggwaid\\
 and what.\ac{acc} -\tsc{comp} is\scsub{[nom]} from Laodicea you read\scsub{[acc]}\\
 `and read that which is from Laodicea' \flushfill{Gothic, \ac{col} 4:16, after \pgcitealt{harbert1978}{357}}

\ex. \ac{int}:\ac{nom}, \ac{ext}:\ac{dat}
\ag. [þaim -ei iupa sind] fraþjaiþ\\
 what.\ac{dat} -\tsc{comp} above are\scsub{[nom]} {think on}\scsub{[dat]}\\
 `set your mind on those which are above' \flushfill{Gothic, \ac{col} 3:2, after \pgcitealt{harbert1978}{339}}

\ex. \ac{int}:\ac{acc}, \ac{ext}:\ac{nom}
\ag. [þan -ei frijos] siuks ist\\
 who.\ac{acc} -\tsc{comp} love\scsub{[acc]} sick is\scsub{[nom]}\\
 `the one whom you love is sick' \flushfill{Gothic, \ac{john} 11:3, after \pgcitealt{harbert1978}{342}}

\ex. \ac{int}:\ac{acc}, \ac{ext}:\ac{dat}
\ag. hva nu wileiþ ei taujau [þamm -ei qiþiþ þiudan Iudaie]?\\
 what now want that do\scsub{[dat]} who.\ac{dat} -\tsc{comp} say\scsub{[acc]} king {of Jews}\\
 `what now do you wish that I do to him whom you call King of the Jews?' \flushfill{Gothic, \ac{mark} 15:12, after \pgcitealt{harbert1978}{339}}

\ex. \ac{int}:\ac{dat}, \ac{ext}:\ac{nom}
\ag. iþ [þamm -ei leitil fraletada] leitil frijod\\
 but who.\ac{dat} -\tsc{comp} little {is forgiven\scsub{[dat]}} little loves\scsub{[nom]}\\
 `but the one whom little is forgiven loves little' \flushfill{Gothic, \ac{luke} 7:47, after \pgcitealt{harbert1978}{342}}

\ex. \ac{int}:\ac{dat}, \ac{ext}:\ac{acc}
\ag. ushafjands [ana þamm -ei lag]\\
 {picking up}\scsub{[acc]} on\scsub{[dat]} what.\ac{dat} -\tsc{comp} lay\\
 `picking up that on which he lay' \flushfill{Gothic, \ac{luke} 5:25, after \pgcitealt{harbert1978}{343}}


\footnote{Throughout this dissertation * stands for 'not found in natural language'. For extinct languages this means that there are no attested examples. For modern languages it means that they examples are ungrammatical.}

% !TEX root = ../thesis.tex

\begin{tabular}{c|c|c|c}
  \toprule
    \diagbox[linecolor=white]{\ac{int}}{\ac{ext}}
        & [\ac{nom}]
        & [\ac{acc}]
        & [\ac{dat}]
        \\ \cmidrule{1-4}
    [\ac{nom}]
        & 
        & \diagbox[linecolor=white]{*\ac{nom}}{\colorbox{LG}{\ac{acc}}}
        & \diagbox[linecolor=white]{*\ac{nom}}{\colorbox{LG}{\ac{dat}}}
        \\ \cmidrule{1-4}
    [\ac{acc}]
        & \diagbox[linecolor=white]{\colorbox{LG}{\ac{acc}}}{*\ac{nom}}
        &
        & \diagbox[linecolor=white]{*\ac{acc}}{\colorbox{LG}{\ac{dat}}}
        \\ \cmidrule{1-4}
    [\ac{dat}]
        & \diagbox[linecolor=white]{\colorbox{LG}{\ac{dat}}}{*\ac{nom}}
        & \diagbox[linecolor=white]{\colorbox{LG}{\ac{dat}}}{*\ac{acc}}
        &
        \\
  \bottomrule
\end{tabular}


\ex. \ac{nom} - \ac{acc} - \ac{dat}

\phantom{nom}



\section{The accessibility hierarchy}

\subsection{Agreement}

\ex.
\ag. Nǐ bǎ shū gěi wǒ-le.\\
 you ba book give me-\tsc{asp}\\
 `You gave me the book.' \flushfill{Mandarin Chinese}
\bg. Du gib -st mi das Buch.\\
 \tsc{nom}.2\tsc{sg} give -\tbf{2\tsc{sg}} \tsc{acc}.3\tsc{sg}.f the book\\
 `You give me the book.' \flushfill{German}
\bg. tayta-yki qam-ta qu -maran\\
 father-2 you-\tsc{acc} give -\tbf{\tsc{3}->1}.\tsc{past}\\
 `Your father gave you to me.' \flushfill{Huallaga Quechua, \citealt{weber1983}{21}}
\bg. Zu-k ni-ri liburu-a emon d -austa -zu.\\
 2\tsc{sg}.\tsc{erg} 1\tsc{sg}.\tsc{dat} book.\tsc{acc} given \tbf{\tsc{acc}.3\tsc{sg}} \tbf{-\tsc{dat}.1\tsc{sg}} \tbf{-\tsc{erg}.2\tsc{sg}}\\
 `You gave me the book.' \flushfill{Basque, Arregi and Molina-Azaola}

\begin{table}[H]
  \center
  \caption {Agreement accessibility}
    \begin{tabular}{ccccc}
      \toprule
            \multicolumn{3}{c}{agreement with}
            &
          & \\
      \cmidrule{1-3}
            subject
            & direct object
            & indirect object
            & number of languages
          & example \\
      \cmidrule{1-3} \cmidrule{4-4} \cmidrule{5-5}
            -
            & -
            & -
            & 23
          & Mandarin Chinese \\
            x
            & -
            & -
            & 31
          & German \\
            x
            & x
            & -
            & 25
          & Huallaga Quechua \\
            x
            & x
            & x
            & 23
          & Basque \\
            x
            & -
            & x
            & (1)
          & - \\
            -
            & x
            & x
            & 0
          & - \\
            -
            & x
            & -
            & 0
          & - \\
            -
            & -
            & x
            & 0
          & - \\
      \bottomrule
    \end{tabular}
\end{table}



\subsection{Relativization}





\ex. \ac{nom} - \ac{acc} - \ac{dat}

gives examples of each of these languages




\section{Case in morphology}

\subsection{Morphological containment}

\pgcitealt{nikolaeva1999}{16}

\begin{table}[H]
  \center
	\caption {Transparent case containment in Khanty}
		\begin{tabular}{clll}
		\toprule
              & \ac{1}\ac{sg}
              & \ac{3}\ac{sg}
              & \ac{1}\ac{pl}                           \\
		          \cmidrule{2-4}
    \ac{nom}  & ma
              & luw
              & muŋ                                     \\
    \ac{acc}  & ma\tbf{:-ne:m}
              & luw\tbf{-e:l}
              & muŋ\tbf{-e:w}                           \\
    \ac{dat}  & ma\tbf{:-ne:m}-\textcolor{DG}{\tbf{na}}
              & luw\tbf{-e:l}-\textcolor{DG}{\tbf{na}}
              & muŋ\tbf{-e:w}-\textcolor{DG}{\tbf{na}}  \\
		\bottomrule
		\end{tabular}
\end{table}


\pgcitealt{boretzky1994}{31-46}

\begin{table}[H]
  \center
	\caption {Transparent case containment in Kalderaš Romani}
		\begin{tabular}{cllll}
		\toprule
              & `brother'
              & `brothers'
              & `girl'
              & `girls'                                   \\
		\cmidrule{2-5}
    \ac{nom}  & phral
              & phral-(á)
              & rakl-í
              & rakl-já                                   \\
    \ac{acc}  & phral-\tbf{és}
              & phral-\tbf{én}
              & rakl-\tbf{já}
              & rakl-já-\tbf{n}                           \\
    \ac{dat}  & phral-\tbf{és}-\textcolor{DG}{\tbf{kə}}
              & phral-\tbf{én}-\textcolor{DG}{\tbf{gə}}
              & rakl-\tbf{já}-\textcolor{DG}{\tbf{kə}}
              & rakl-já-\tbf{n}-\textcolor{DG}{\tbf{gə}}  \\
		\bottomrule
		\end{tabular}
\end{table}

\pgcitealt{gippert1987}{23-24}

\begin{table}[H]
  \center
	\caption {Transparent case containment in West Tocharian}
		\begin{tabular}{cll}
		\toprule
              & `horses'
              & `men'                                  \\
		\cmidrule{2-3}
    \ac{nom}  & yakwi
              & eṅkwi                                  \\
    \ac{acc}  & yakwe-\tbf{ṃ}
              & eṅkwe-\tbf{ṃ}                          \\
    \ac{dat}  & yäkwe-\tbf{ṃ}-\textcolor{DG}{\tbf{ts}}
              & eṅkwe-\tbf{ṃ}-\textcolor{DG}{\tbf{ts}} \\
		\bottomrule
		\end{tabular}
\end{table}

\ex. \ac{nom} < \ac{acc} < \ac{dat}

\phantom{nom}

\subsection{Suppletion patterns}

\ex. \ac{nom} < \ac{acc} < \ac{dat}

\phantom{nom}

\subsubsection{ABB}


cognates widespread in Indo-European - Icelandic\\
cognates across Slavic - Russian\\
cognates across Slavic - Serbian


\begin{table}[H]
  \center
	\caption {ABB patterns in suppletion}
		\begin{tabular}{cccccc}
		\toprule
              & Icelandic           & Russian             & \multicolumn{3}{c}{Serbian}                                           \\
		            \cmidrule(lr){2-2}    \cmidrule(lr){3-3}    \cmidrule(lr){4-6}
              & \ac{1}\ac{sg}       & \ac{1}\ac{pl}       & \ac{3}\ac{sg}.\ac{f}  & \ac{3}\ac{sg}.\ac{m}  & \ac{3}\ac{sg}.\ac{n}  \\
		            \cmidrule(lr){2-2}    \cmidrule(lr){3-3}    \cmidrule(lr){4-6}
    \ac{nom}  & ég                  & my                  &  ona                  & oni                   & on                    \\
    \ac{acc}  & \tbf{m}ig           & \tbf{n}as           & \tbf{nj}u             & \tbf{nji}h            & \tbf{nje}-ga          \\
    \ac{dat}  & \tbf{m}ér           & \tbf{n}am           & \tbf{nj}oj            & \tbf{nji}ma           & \tbf{nje}-mu          \\
    \bottomrule
		\end{tabular}
\end{table}





\subsubsection{AAB}


\begin{table}[H]
  \center
	\caption {AAB patterns in suppletion}
		\begin{tabular}{cccc}
		\toprule
              & Yurok                          & \multicolumn{2}{c}{Wardaman}               \\
		            \cmidrule(lr){2-2}               \cmidrule(lr){3-4}
              & \ac{3}\ac{sg}                  & \ac{3}\ac{sg}       & \ac{3}\ac{pl}        \\
		            \cmidrule(lr){2-2}               \cmidrule(lr){3-4}
    \ac{nom}  & \tbf{yoɂ}(o·t), \tbf{woɂ}(o·t)  & \tbf{narnaj}        & \tbf{narnaj}-bulu    \\
    \ac{acc}  & \tbf{yoɂ}o·t, \tbf{woɂ}o·t      & \tbf{narnaj}-(j)i   & \tbf{narnaj}-bulu-yi \\
    \ac{dat}  & weyaɂik                         & gunga               & wurrugu              \\
    \bottomrule
		\end{tabular}
\end{table}


\subsubsection{ABC}

\begin{table}[H]
  \center
	\caption {ABC patterns in suppletion}
		\begin{tabular}{ccc}
		\toprule
              & Khinalugh          \\
		            \cmidrule(lr){2-2}
              & \ac{1}\ac{sg}      \\
		            \cmidrule(lr){2-2}
    \ac{nom}  & zɨ                  \\
    \ac{acc}  & jä                  \\
    \ac{dat}  & as(ɨr)              \\
    \bottomrule
		\end{tabular}
\end{table}






\subsection{Syncretism patterns}

\subsubsection{ABB}

Van Baal, Don

\begin{table}[H]
  \center
	\caption {ABB patterns in suppletion}
		\begin{tabular}{cccccc}
		\toprule
              & Dutch           &                      & \multicolumn{3}{c}{x}                                           \\
		            \cmidrule(lr){2-2}    \cmidrule(lr){3-3}    \cmidrule(lr){4-6}
              & \ac{1}\ac{sg}   & \ac{2}\ac{sg}        & \ac{sg}.\ac{f}  & \ac{3}\ac{sg}.\ac{m}  & \ac{3}\ac{sg}.\ac{n}  \\
		            \cmidrule(lr){2-2}    \cmidrule(lr){3-3}    \cmidrule(lr){4-6}
    \ac{nom}  & ik              & jij                  &  ona                  & oni                   & on                    \\
    \ac{acc}  & \tbf{mij}       & \tbf{jou}            & \tbf{nj}u             & \tbf{nji}h            & \tbf{nje}-ga          \\
    \ac{dat}  & \tbf{mij}       & \tbf{jou}            & \tbf{nj}oj            & \tbf{nji}ma           & \tbf{nje}-mu          \\
    \bottomrule
		\end{tabular}
\end{table}


https://linguistlist.org/issues/13/13-1129.html


\subsubsection{AAB}

Russian: table pl,
stol-y
stol-y
stol-ov


Russian, building sg,
zdani-e
zdani-e
zdani-ju




endings of Latin nouns.9
n(n) n(mf) I(fm) V(fm) ffl(n) ffl(mf) nii(mf) mi(n) IV(mf) IV(n)
sg. -um -us -a -es 0,-s -is,-es -c -us -u(-u?)
Ac -um -um -am -em -em -em -e -um -u(-u?)
-I -~i -ai>-ae 4l -is -is -is -is -Qs -us
-6 -6 -ae -el -I -1 -I -1 -ul *u
Ab -6 -6 -a -S -e -c -e, -i -I -Q -u
pl. -a -I -ae -es -a -es -cs -ia -us -ua
Ac -a -6s -as -es -a  -es, -Is -ia -us -ua
-orum -omm -arum -erum -um -um -ium -ium -uum -uum
DAb -is -Ts -Is -ebus -ibus -ibus -ibus -ibus -ibus -ibus




german:
die die der
das das dem



\subsubsection{ABC}

all different endings


\ex. \ac{nom} < \ac{acc} < \ac{dat}




\section{A side note on the genitive}

\begin{itemize}
  \item possessive
  \item accessibility hierarchy
  \item not available
\end{itemize}
