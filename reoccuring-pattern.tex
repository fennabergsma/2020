% !TEX root = thesis.tex

\chapter{A reoccurring pattern}

Start with the main focus of this dissertation
A phenomenon goes parallel with the ordering


\section{Case competition in Gothic headless relatives}

In the introduction I already showed accusative vs. dative, and dative always won. Here I additionally add nominative. I will show that ordering of strength is nom-acc-dat.

First some terminology.
Intern
Extern

\begin{table}[H]
  \center
  \caption {Case attraction in headless relatives - empty}
    \begin{tabular}{c|c|c|c}
      \toprule
        \diagbox[linecolor=white]{\ac{int}}{\ac{ext}}
            & [\ac{nom}]
            & [\ac{acc}]
            & [\ac{dat}]
            \\ \cmidrule{1-4}
        [\ac{nom}]
            & \diagbox[linecolor=white]{\phantom{nom}}{\phantom{nom}}
            & \diagbox[linecolor=white]{\phantom{nom}}{\phantom{nom}}
            & \diagbox[linecolor=white]{\phantom{nom}}{\phantom{nom}}
            \\ \cmidrule{1-4}
        [\ac{acc}]
            & \diagbox[linecolor=white]{\phantom{nom}}{\phantom{nom}}
            & \diagbox[linecolor=white]{\phantom{nom}}{\phantom{nom}}
            & \diagbox[linecolor=white]{\phantom{nom}}{\phantom{nom}}
            \\ \cmidrule{1-4}
        [\ac{dat}]
            & \diagbox[linecolor=white]{\phantom{nom}}{\phantom{nom}}
            & \diagbox[linecolor=white]{\phantom{nom}}{\phantom{nom}}
            & \diagbox[linecolor=white]{\phantom{nom}}{\phantom{nom}}
            \\
      \bottomrule
    \end{tabular}
\end{table}


\subsection{matching}

\exg. ei [sa -ei þis matjai,] ni gadauþnai\\
 that who.\ac{acc} -\tsc{comp} of this eats\scsub{[nom]} not die\scsub{[nom]}\\
 `that (he) who eats of this may not die' \flushfill{Gothic, \ac{john} 6:50, after \pgcitealt{harbert1978}{337}}\label{ex:gothicnomnom}

\exg. gaarma [þan -ei arma]\\
 pity\scsub{[acc]} who.\ac{acc} -\tsc{comp} pity\scsub{[acc]}\\
 `I will pity (him) whom I pity' \flushfill{Gothic, \ac{rom} 9:15, after \pgcitealt{harbert1978}{339}}\label{ex:gothicaccacc}

\ex.
\ag. [þamm -ei gabaur] gabaur\\
 who.\ac{dat} -\tsc{comp} tribute\scsub{[dat]} tribute\scsub{[dat]}\\
 `tribute to (him) whom tribute is due'
\bg. [þamm -ei mota] mota\\
 who.\ac{dat} -\tsc{comp} custom\scsub{[dat]} custom\scsub{[dat]}\\
 `custom to (him) whom custom is due'
\bg. [þamm -ei agis] agis\\
 who.\ac{dat} -\tsc{comp} fear\scsub{[dat]} fear\scsub{[dat]}\\
 `fear to (him) whom fear is due'
\bg. [þamm -ei sweriþa] sweriþa\\
 who.\ac{dat} -\tsc{comp} honour\scsub{[dat]} honour\scsub{[dat]}\\
 `honour to (him) whom honour is due' \flushfill{Gothic, \ac{rom} 13:7, after \pgcitealt{harbert1978}{339}}\label{ex:gothicdatdat}


\begin{table}[H]
  \center
  \caption {Case attraction in headless relatives - only matching}
    \begin{tabular}{c|c|c|c}
      \toprule
        \diagbox[linecolor=white]{\ac{int}}{\ac{ext}}
            & [\ac{nom}]
            & [\ac{acc}]
            & [\ac{dat}]
            \\ \cmidrule{1-4}
        [\ac{nom}]
            & \colorbox{LG}{\ac{nom}}
            & \diagbox[linecolor=white]{\phantom{nom}}{\phantom{nom}}
            & \diagbox[linecolor=white]{\phantom{nom}}{\phantom{nom}}
            \\ \cmidrule{1-4}
        [\ac{acc}]
            & \diagbox[linecolor=white]{\phantom{nom}}{\phantom{nom}}
            & \colorbox{LG}{\ac{acc}}
            & \diagbox[linecolor=white]{\phantom{nom}}{\phantom{nom}}
            \\ \cmidrule{1-4}
        [\ac{dat}]
            & \diagbox[linecolor=white]{\phantom{nom}}{\phantom{nom}}
            & \diagbox[linecolor=white]{\phantom{nom}}{\phantom{nom}}
            & \colorbox{LG}{\ac{dat}}
            \\
      \bottomrule
    \end{tabular}
\end{table}





\subsection{non-matching}

\ex. \ac{int}:\ac{nom}, \ac{ext}:\ac{acc}
\ag. jah [þo -ei ist us Laudeikaion] jus ussiggwaid\\
 and what.\ac{acc} -\tsc{comp} is\scsub{[nom]} from Laodicea you read\scsub{[acc]}\\
 `and read that which is from Laodicea' \flushfill{Gothic, \ac{col} 4:16, after \pgcitealt{harbert1978}{357}}

\ex. \ac{int}:\ac{nom}, \ac{ext}:\ac{dat}
\ag. [þaim -ei iupa sind] fraþjaiþ\\
 what.\ac{dat} -\tsc{comp} above are\scsub{[nom]} {think on}\scsub{[dat]}\\
 `set your mind on those which are above' \flushfill{Gothic, \ac{col} 3:2, after \pgcitealt{harbert1978}{339}}

\ex. \ac{int}:\ac{acc}, \ac{ext}:\ac{nom}
\ag. [þan -ei frijos] siuks ist\\
 who.\ac{acc} -\tsc{comp} love\scsub{[acc]} sick is\scsub{[nom]}\\
 `the one whom you love is sick' \flushfill{Gothic, \ac{john} 11:3, after \pgcitealt{harbert1978}{342}}

\ex. \ac{int}:\ac{acc}, \ac{ext}:\ac{dat}
\ag. hva nu wileiþ ei taujau [þamm -ei qiþiþ þiudan Iudaie]?\\
 what now want that do\scsub{[dat]} who.\ac{dat} -\tsc{comp} say\scsub{[acc]} king {of Jews}\\
 `what now do you wish that I do to him whom you call King of the Jews?' \flushfill{Gothic, \ac{mark} 15:12, after \pgcitealt{harbert1978}{339}}

\ex. \ac{int}:\ac{dat}, \ac{ext}:\ac{nom}
\ag. iþ [þamm -ei leitil fraletada] leitil frijod\\
 but who.\ac{dat} -\tsc{comp} little {is forgiven\scsub{[dat]}} little loves\scsub{[nom]}\\
 `but the one whom little is forgiven loves little' \flushfill{Gothic, \ac{luke} 7:47, after \pgcitealt{harbert1978}{342}}

\ex. \ac{int}:\ac{dat}, \ac{ext}:\ac{acc}
\ag. ushafjands [ana þamm -ei lag]\\
 {picking up}\scsub{[acc]} on\scsub{[dat]} what.\ac{dat} -\tsc{comp} lay\\
 `picking up that on which he lay' \flushfill{Gothic, \ac{luke} 5:25, after \pgcitealt{harbert1978}{343}}


\footnote{Throughout this dissertation * stands for 'not found in natural language'. For extinct languages this means that there are no attested examples. For modern languages it means that they examples are ungrammatical.}

% !TEX root = thesis.tex

\begin{table}[H]
  \center
  \caption {Case attraction in headless relatives in Gothic}
    \begin{tabular}{c|c|c|c}
      \toprule
        \diagbox[linecolor=white]{\ac{int}}{\ac{ext}}
            & [\ac{nom}]
            & [\ac{acc}]
            & [\ac{dat}]
            \\ \cmidrule{1-4}
        [\ac{nom}]
            & \colorbox{LG}{\ac{nom}}
            & \diagbox[linecolor=white]{*\ac{nom}}{\colorbox{DG}{\ac{acc}}}
            & \diagbox[linecolor=white]{*\ac{nom}}{\colorbox{DG}{\ac{dat}}}
            \\ \cmidrule{1-4}
        [\ac{acc}]
            & \diagbox[linecolor=white]{\colorbox{DG}{\ac{acc}}}{*\ac{nom}}
            & \colorbox{LG}{\ac{acc}}
            & \diagbox[linecolor=white]{*\ac{acc}}{\colorbox{DG}{\ac{dat}}}
            \\ \cmidrule{1-4}
        [\ac{dat}]
            & \diagbox[linecolor=white]{\colorbox{DG}{\ac{dat}}}{*\ac{nom}}
            & \diagbox[linecolor=white]{\colorbox{DG}{\ac{dat}}}{*\ac{acc}}
            & \colorbox{LG}{\ac{dat}}
            \\
      \bottomrule
    \end{tabular}
\end{table}


\ex. \ac{nom} - \ac{acc} - \ac{dat}

\phantom{nom}



\section{The accessibility hierarchy}

\subsection{Agreement}

\ex.
\ag. Nǐ bǎ shū gěi wǒ-le.\\
 you ba book give me-\ac{asp}\\
 `You gave me the book.' \flushfill{Mandarin Chinese}
\bg. Du gib -st mir das Buch.\\
 you give -\tbf{2\tsc{sg}} me the book\\
 `You give me the book.' \flushfill{German}
\bg. tayta-yki qam-ta qu -maran\\
 father-your you-\ac{acc} give -\tbf{3\tsc{sg}→1\tsc{sg}}.\ac{pst}\\
 `Your father gave you to me.' \flushfill{Huallaga Quechua, \pgcitealt{weber1983}{21}}
\bg. Zu-k ni-ri liburu-a emon d -austa -zu.\\
 you-\ac{erg} me-\ac{dat} book-\ac{def}.\ac{acc} given \tbf{\ac{acc}.3\tsc{sg}} \tbf{-\ac{dat}.1\tsc{sg}} \tbf{-\ac{erg}.2\tsc{sg}}\\
 `You gave me the book.' \flushfill{Basque, \pgcitealt{arregi2004}{45}}

\begin{table}[H]
  \center
  \caption {Agreement accessibility}
    \begin{tabular}[t]{ccccc}
      \toprule
            \multicolumn{3}{c}{agreement with}
            &
          & \\
      \cmidrule{1-3}
            & direct
            & indirect
            & number
          & \\
            subject
            & object
            & object
            & of languages
          & example \\
      \cmidrule{1-3} \cmidrule{4-4} \cmidrule{5-5}
            *
            & *
            & *
            & 23
          & Mandarin Chinese \\
            ✔
            & *
            & *
            & 31
          & German \\
            ✔
            & ✔
            & *
            & 25
          & Huallaga Quechua \\
            ✔
            & ✔
            & ✔
            & 23
          & Basque \\
            ✔
            & *
            & ✔
            & (1)
          & - \\
            {*}
            & ✔
            & ✔
            & 0
          & - \\
            {*}
            & x
            & *
            & 0
          & - \\
            {*}
            & *
            & ✔
            & 0
          & - \\
      \bottomrule
    \end{tabular}
\end{table}



\subsection{Relativization}


\ex.
\ag. Nahita ny vehivavy ny mpianatra.\\
 saw the woman the student\\
 `The student saw the woman.'
\bg. ny mpianatra izay nahita ny vehivavy\\
 the student that saw the woman\\
 `the student that saw the woman'
\bg. *ny vehivavy izay nahita ny mpianatra\\
 the woman that saw the student\\
 `the woman that the student saw' \flushfill{Malagasy, \pgcitealt{keenan1977}{70}}

\ex.
\ag. Nohitan' ny mpianatra ny vehivavy.\\
 seen.\ac{pass} the student the woman\\
 `The woman was seen by the student.'
\bg. ny vehivavy izay nohitan' ny mpianatra\\
 the woman that seen.\ac{pass} the student\\
 `the woman that was seen by the student' \flushfill{Malagasy, \pgcitealt{keenan1977}{70}}





\exg. Ali bunoh ayam yang Aminah sedang memakan.\\
 Ali kill chicken that Aminah \ac{prog} eat\\
 `Ali killed the chicken that Aminah is eating.' \flushfill{Malay, \pgcitealt{keenan1977}{71}}

\ex.
\ag. Ali beri {ubi kentang} itu kapada perempuan itu.\\
 Ali give potato the to woman the\\
 `Ali gave the potato to the woman.'
\bg. *perempuan yang Ali beri {ubi kentang} itu kapada\\
 woman that Ali give potato the to\\
\bg. *perempuan kapada yang Ali beri {ubi kentang} itu\\
 woman to who Ali give potato that\\ \flushfill{Malay, \pgcitealt{keenan1977}{71}}

\exg. perempuan yang menerima {ubi kentang} itu daripada Ali\\
 woman that received potato the from Ali\\
 `the woman that received the potato from Ali'\flushfill{Malay, \pgcitealt{keenan1977}{71}}




\ex.
\ag. Gizon-a-k emakume-a-ri liburu-a eman dio.\\
 man-\ac{def}-\ac{erg} woman-\ac{def}-\ac{dat} book-\ac{def}.\ac{acc} give has\\
 `The man has given the book to the woman.'
\bg. emakume-a-ri liburu-a eman dio-n gizon-a\\
 woman-\ac{def}-\ac{dat} book-\ac{def}.\ac{acc} give has-\ac{rel} man-\ac{def}\\
 `the man who has given the book to the woman'
\bg. gizon-a-k emakume-a-ri eman dio-n liburu-a\\
 man-\ac{def}-\ac{erg} woman-\ac{def}-\tsc{dat} give has-\ac{rel} book-\ac{def}\\
 `the book that the man has given to the woman'
\bg. gizon-a-k liburu-a eman dio-n emakume-a\\
 man-\ac{def}-\ac{erg} book-\ac{def}.\ac{acc} give has-\ac{rel} woman-\ac{def}\\
 `the woman that the man has given the book to' \flushfill{Basque, \pgcitealt{keenan1977}{72}}




 \begin{table}[H]
   \center
   \caption {Relativization accessibility}
     \begin{tabular}{cccc}
       \toprule
             \multicolumn{3}{c}{relativization of}
           & \\
       \cmidrule{1-3}
             & direct
             & indirect
           & \\
             subject
             & object
             & object
           & example \\
       \cmidrule{1-3} \cmidrule{4-4}
             ✔
             & *
             & *
           & Malagasy \\
             ✔
             & ✔
             & *
           & Malay \\
             ✔
             & ✔
             & ✔
           & Basque \\
       \bottomrule
     \end{tabular}
 \end{table}






\section{Case in morphology}


\subsection{Suppletion patterns}

\begin{table}[H]
  \center
  \caption {Suppletion patterns}
    \begin{tabular}{cccccccc}
      \toprule
          \multicolumn{3}{c}{pattern}
            & \tsc{nom}
            & \tsc{acc}
            & \tsc{dat}
            & translation
            & language \\
      \cmidrule(lr){1-3} \cmidrule(lr){4-6} \cmidrule(lr){7-7} \cmidrule(lr){8-8}
          A & A & A
            & \cellcolor{LG}\tbf{þ}ú
            & \cellcolor{LG}\tbf{þ}ig
            & \cellcolor{LG}\tbf{þ}ér
            & 2\tsc{sg}
            & Icelandic \\
          A & B & B
            & my
            & \cellcolor{LG}\tbf{n}as
            & \cellcolor{LG}\tbf{n}am
            & 1\tsc{pl}
            & Russian \\
          A & A & B
            & \cellcolor{LG}\tbf{narnaj}
            & \cellcolor{LG}\tbf{narnaj}(j)i
            & gunga
            & 3\tsc{sg}
            & Wardaman \\
          A & B & C
            & zɨ
            & jä
            & as(ɨr)
            & 1\tsc{sg}
            & Khinalugh \\
          A & B & A
            & \cellcolor{LG}
            &
            & \cellcolor{LG}
            &
            & not attested \\
      \bottomrule
    \end{tabular}
\end{table}





\subsection{Syncretism patterns}

\begin{table}[H]
  \center
  \caption {Syncretism patterns}
    \begin{tabular}{cccccccc}
      \toprule
          \multicolumn{3}{c}{pattern}
            & \tsc{nom}
            & \tsc{acc}
            & \tsc{dat}
            & translation
            & language \\
      \cmidrule(lr){1-3} \cmidrule(lr){4-6} \cmidrule(lr){7-7} \cmidrule(lr){8-8}
          A & A & A
            & \cellcolor{LG}\tbf{inu}
            & \cellcolor{LG}\tbf{inu}
            & \cellcolor{LG}\tbf{inu}
            & 2\tsc{pl}
            & Lavukaleve \\
          A & B & B
            & ta
            & \cellcolor{LG}\tbf{bor}
            & \cellcolor{LG}\tbf{bor}
            & 1\tsc{pl}
            & Teribe \\
          A & A & B
            & \cellcolor{LG}\tbf{sie}
            & \cellcolor{LG}\tbf{sie}
            & ihr
            & 3\tsc{sg}.\tsc{f}
            & German \\
          A & B & C
            & zɨ
            & jä
            & as(ɨr
            & 1\tsc{sg}
            & Khinalugh \\
          A & B & A
            & \cellcolor{LG}
            &
            & \cellcolor{LG}
            &
            & not attested \\
      \bottomrule
    \end{tabular}
\end{table}







\ex. \ac{nom} < \ac{acc} < \ac{dat}


\subsection{Morphological containment}

\pgcitealt{nikolaeva1999}{16}

\begin{table}[H]
  \center
	\caption {Transparent case containment in Khanty}
		\begin{tabular}{clll}
		\toprule
              & \ac{1}\ac{sg}
              & \ac{3}\ac{sg}
              & \ac{1}\ac{pl}                           \\
		          \cmidrule{2-4}
    \ac{nom}  & ma
              & luw
              & muŋ                                     \\
    \ac{acc}  & ma\tbf{:-ne:m}
              & luw\tbf{-e:l}
              & muŋ\tbf{-e:w}                           \\
    \ac{dat}  & ma\tbf{:-ne:m}-\textcolor{DG}{\tbf{na}}
              & luw\tbf{-e:l}-\textcolor{DG}{\tbf{na}}
              & muŋ\tbf{-e:w}-\textcolor{DG}{\tbf{na}}  \\
		\bottomrule
		\end{tabular}
\end{table}


\pgcitealt{boretzky1994}{31-46}

\begin{table}[H]
  \center
	\caption {Transparent case containment in Kalderaš Romani}
		\begin{tabular}{cllll}
		\toprule
              & `brother'
              & `brothers'
              & `girl'
              & `girls'                                   \\
		\cmidrule{2-5}
    \ac{nom}  & phral
              & phral-(á)
              & rakl-í
              & rakl-já                                   \\
    \ac{acc}  & phral-\tbf{és}
              & phral-\tbf{én}
              & rakl-\tbf{já}
              & rakl-já-\tbf{n}                           \\
    \ac{dat}  & phral-\tbf{és}-\textcolor{DG}{\tbf{kə}}
              & phral-\tbf{én}-\textcolor{DG}{\tbf{gə}}
              & rakl-\tbf{já}-\textcolor{DG}{\tbf{kə}}
              & rakl-já-\tbf{n}-\textcolor{DG}{\tbf{gə}}  \\
		\bottomrule
		\end{tabular}
\end{table}

\pgcitealt{gippert1987}{23-24}

\begin{table}[H]
  \center
	\caption {Transparent case containment in West Tocharian}
		\begin{tabular}{cll}
		\toprule
              & `horses'
              & `men'                                  \\
		\cmidrule{2-3}
    \ac{nom}  & yakwi
              & eṅkwi                                  \\
    \ac{acc}  & yakwe-\tbf{ṃ}
              & eṅkwe-\tbf{ṃ}                          \\
    \ac{dat}  & yäkwe-\tbf{ṃ}-\textcolor{DG}{\tbf{ts}}
              & eṅkwe-\tbf{ṃ}-\textcolor{DG}{\tbf{ts}} \\
		\bottomrule
		\end{tabular}
\end{table}

\ex. \ac{nom} < \ac{acc} < \ac{dat}

\phantom{nom}




\section{A side note on the genitive}

\begin{itemize}
  \item possessive
  \item accessibility hierarchy
  \item not available
\end{itemize}
