% !TEX root = thesis.tex

\chapter{A reoccurring pattern}

First I introduce the pattern that forms the focus of the first part of the dissertation. I show that headless relatives in Gothic adhere to the case strength scale: \ac{nom} < \ac{acc} < \ac{dat}.

Then I show two phenomena that follow the same ordering of \ac{nom}, \ac{acc} and \ac{dat}. The two phenomena are accessibility hierarchies. The first one is about agreement, the second one about relativization.

In the last section of this chapter I discuss how \ac{nom}, \ac{acc} and \ac{dat} pattern in morphology.


\section{Case competition in Gothic headless relatives}

In this section I show the behavior of Gothic headless relatives in detail. I systematically go through all case combinations, except for the genitive, which I discuss in Section \ref{sec:genitive}. This leaves the nominative, accusative and dative.

I repeat the headless relative from the introduction in which both predicates assign accusative case in \ref{ex:gothicaccaccrep}. The predicate in the relative clause \tit{arma} `pity' assigns accusative case. The predicate in the main clause \tit{gaarma} `pity' also assigns accusative case. The relative pronoun \tit{þan(a)} `who.\ac{acc}' appears in accusative case.

\exg. gaarma [þan -ei arma]\\
 pity\scsub{[acc]} who.\ac{acc} -\ac{comp} pity\scsub{[acc]}\\
 `I will pity (him) whom I pity' \flushfill{Gothic, \ac{rom} 9:15, after \pgcitealt{harbert1978}{339}}\label{ex:gothicaccaccrep}

From this section on I use the terms internal case and external case. Internal case refers to the case assigned internal to the relative clause. In \ref{ex:gothicaccaccrep}, the internal case is assigned by \tit{arma} `pity'. The external case refers to the case assign external to the relative clause, so the main clause. In \ref{ex:gothicaccaccrep}, the external case is assigned by \tit{gaarma} `pity'.

I placed subscripts between the square brackets on the glosses of verbs. They indicate which case the predicate requires from the relative pronoun. For \ref{ex:gothicaccaccrep} this simply means which case the verb assigns to its object. Another possibility is that the subscript is placed on a preposition and refer to the case the preposition assigns. A last possibility is that the subscript is [\tsc{nom}] and refers to the case in which the subject appears in. An example of that is given in \ref{ex:gothicnomnom}.

In this example the predicate \tit{matjai} `eats' combines with a nominative subject in the relative clause. In other words, the internal case is nominative. The predicate \tit{gadauþnai} `die' in the main clause also combines with a nominative subject. In other words, the external case is nominative. The relative pronoun \tit{sa} `who.\tsc{nom}' appears in nominative case.

\exg.[ei [sa -ei þis matjai,] ni gadauþnai]\\
 that who.\ac{nom} -\ac{comp} {of this} eats\scsub{[nom]} not die\scsub{[nom]}\\
 `that (he) who eats of this may not die' \flushfill{Gothic, \ac{john} 6:50, after \pgcitealt{harbert1978}{337}}\label{ex:gothicnomnom}

In the examples below, the internal case and external case are dative.

\ex.
\ag. [þamm -ei gabaur] gabaur\\
 who.\ac{dat} -\ac{comp} tribute\scsub{[dat]} tribute\scsub{[dat]}\\
 `tribute to (him) whom tribute is due'
\bg. [þamm -ei mota] mota\\
 who.\ac{dat} -\ac{comp} custom\scsub{[dat]} custom\scsub{[dat]}\\
 `custom to (him) whom custom is due'
\bg. [þamm -ei agis] agis\\
 who.\ac{dat} -\ac{comp} fear\scsub{[dat]} fear\scsub{[dat]}\\
 `fear to (him) whom fear is due'
\bg. [þamm -ei sweriþa] sweriþa\\
 who.\ac{dat} -\ac{comp} honour\scsub{[dat]} honour\scsub{[dat]}\\
 `honour to (him) whom honour is due' \flushfill{Gothic, \ac{rom} 13:7, after \pgcitealt{harbert1978}{339}}\label{ex:gothicdatdat}

Schematically, this looks like:

\begin{table}[H]
  \center
  \caption {Case attraction in headless relatives - only matching}
    \begin{tabular}{c|c|c|c}
      \toprule
        \diagbox[linecolor=white]{\ac{int}}{\ac{ext}}
            & [\ac{nom}]
            & [\ac{acc}]
            & [\ac{dat}]
            \\ \cmidrule{1-4}
        [\ac{nom}]
            & \colorbox{LG}{\ac{nom}}
            & \diagbox[linecolor=white]{\phantom{nom}}{\phantom{nom}}
            & \diagbox[linecolor=white]{\phantom{nom}}{\phantom{nom}}
            \\ \cmidrule{1-4}
        [\ac{acc}]
            & \diagbox[linecolor=white]{\phantom{nom}}{\phantom{nom}}
            & \colorbox{LG}{\ac{acc}}
            & \diagbox[linecolor=white]{\phantom{nom}}{\phantom{nom}}
            \\ \cmidrule{1-4}
        [\ac{dat}]
            & \diagbox[linecolor=white]{\phantom{nom}}{\phantom{nom}}
            & \diagbox[linecolor=white]{\phantom{nom}}{\phantom{nom}}
            & \colorbox{LG}{\ac{dat}}
            \\
      \bottomrule
    \end{tabular}
\end{table}

In what follows I discuss the pattern that occurs when the internal and external case differs.

Internal case is nominative, external case is accusative.

\exg. jah [þo -ei ist us Laudeikaion] jus ussiggwaid\\
 and what.\ac{acc} -\ac{comp} is\scsub{[nom]} from Laodicea you read\scsub{[acc]}\\
 `and read that which is from Laodicea' \flushfill{Gothic, \ac{col} 4:16, after \pgcitealt{harbert1978}{357}}

Internal case is nominative, external case is dative.

\exg. [þaim -ei iupa sind] fraþjaiþ\\
 what.\ac{dat} -\ac{comp} above are\scsub{[nom]} {think on}\scsub{[dat]}\\
 `set your mind on those which are above' \flushfill{Gothic, \ac{col} 3:2, after \pgcitealt{harbert1978}{339}}

Internal case is accusative, external case is nominative.

\exg. [þan -ei frijos] siuks ist\\
 who.\ac{acc} -\ac{comp} love\scsub{[acc]} sick is\scsub{[nom]}\\
 `the one whom you love is sick' \flushfill{Gothic, \ac{john} 11:3, after \pgcitealt{harbert1978}{342}}

Internal case is accusative, external case is dative.

\exg. hva nu wileiþ ei taujau [þamm -ei qiþiþ þiudan Iudaie]?\\
 what now want that do\scsub{[dat]} who.\ac{dat} -\ac{comp} say\scsub{[acc]} king {of Jews}\\
 `what now do you wish that I do to him whom you call King of the Jews?' \flushfill{Gothic, \ac{mark} 15:12, after \pgcitealt{harbert1978}{339}}

Internal case is dative, external case is nominative.

\exg. iþ [þamm -ei leitil fraletada] leitil frijod\\
 but who.\ac{dat} -\ac{comp} little {is forgiven\scsub{[dat]}} little loves\scsub{[nom]}\\
 `but the one whom little is forgiven loves little' \flushfill{Gothic, \ac{luke} 7:47, after \pgcitealt{harbert1978}{342}}

Internal case is dative, external case is accusative.

\ex. ushafjands [ana þamm -ei lag]\\
 {picking up}\scsub{[acc]} on\scsub{[dat]} what.\ac{dat} -\ac{comp} lay\\
 `picking up that on which he lay' \flushfill{Gothic, \ac{luke} 5:25, after \pgcitealt{harbert1978}{343}}

\footnote{Throughout this dissertation * stands for 'not found in natural language'. For extinct languages this means that there are no attested examples. For modern languages it means that they examples are ungrammatical.}

% !TEX root = ../thesis.tex

\begin{tabular}{c|c|c|c}
  \toprule
    \diagbox[linecolor=white]{\ac{int}}{\ac{ext}}
        & [\ac{nom}]
        & [\ac{acc}]
        & [\ac{dat}]
        \\ \cmidrule{1-4}
    [\ac{nom}]
        & 
        & \diagbox[linecolor=white]{*\ac{nom}}{\colorbox{LG}{\ac{acc}}}
        & \diagbox[linecolor=white]{*\ac{nom}}{\colorbox{LG}{\ac{dat}}}
        \\ \cmidrule{1-4}
    [\ac{acc}]
        & \diagbox[linecolor=white]{\colorbox{LG}{\ac{acc}}}{*\ac{nom}}
        &
        & \diagbox[linecolor=white]{*\ac{acc}}{\colorbox{LG}{\ac{dat}}}
        \\ \cmidrule{1-4}
    [\ac{dat}]
        & \diagbox[linecolor=white]{\colorbox{LG}{\ac{dat}}}{*\ac{nom}}
        & \diagbox[linecolor=white]{\colorbox{LG}{\ac{dat}}}{*\ac{acc}}
        &
        \\
  \bottomrule
\end{tabular}


\ex.
\a. \tsc{acc} wins over \tsc{nom}
\b. \tsc{dat} wins over \tsc{nom}
\b. \tsc{dat} wins over \tsc{acc}


\section{Parallels in accessibility hierarchies}



\subsection{Agreement}

Moravcsik, Gilligan sub, obj, ind obj

Bobaljik default/dependent/dative

Mandarin Chinese does not show any agreement on the verb. In German, the verb agrees with the subject. In Huallaga Quechua, the verb agrees with the subject and the object. In Basque, the verb agrees with the subject, the object and the indirect object.


\ex.
\ag. Nǐ bǎ shū gěi wǒ-le.\\
 you ba book give me-\ac{asp}\\
 `You gave me the book.' \flushfill{Mandarin Chinese}
\bg. Du gib -st mir das Buch.\\
 you give -\tbf{2\ac{sg}} me the book\\
 `You give me the book.' \flushfill{German}
\bg. tayta-yki qam-ta qu -maran\\
 father-your you-\ac{acc} give -\tbf{3\ac{sg}→1\ac{sg}}.\ac{pst}\\
 `Your father gave you to me.' \flushfill{Huallaga Quechua, \pgcitealt{weber1983}{21}}
\bg. Zu-k ni-ri liburu-a emon d -austa -zu.\\
 you-\ac{erg} me-\ac{dat} book-\ac{def}.\ac{acc} given \tbf{\ac{acc}.3\ac{sg}} \tbf{-\ac{dat}.1\ac{sg}} \tbf{-\ac{erg}.2\ac{sg}}\\
 `You gave me the book.' \flushfill{Basque, \pgcitealt{arregi2004}{45}}

\begin{table}[H]
  \center
  \caption {Agreement accessibility}
    \begin{tabular}[t]{ccccc}
      \toprule
            \multicolumn{3}{c}{agreement with}
            &
          & \\
      \cmidrule{1-3}
            & direct
            & indirect
            & number
          & \\
            subject
            & object
            & object
            & of languages
          & example \\
      \cmidrule{1-3} \cmidrule{4-4} \cmidrule{5-5}
            *
            & *
            & *
            & 23
          & Mandarin Chinese \\
            ✔
            & *
            & *
            & 31
          & German \\
            ✔
            & ✔
            & *
            & 25
          & Huallaga Quechua \\
            ✔
            & ✔
            & ✔
            & 23
          & Basque \\
            ✔
            & *
            & ✔
            & (1)
          & - \\
            {*}
            & ✔
            & ✔
            & 0
          & - \\
            {*}
            & x
            & *
            & 0
          & - \\
            {*}
            & *
            & ✔
            & 0
          & - \\
      \bottomrule
    \end{tabular}
\end{table}



\subsection{Relativization}

Keenan Comrie sub/obj/ind obj

Caha nom/acc/dat

In Malagasy, only subjects can be relativized.

\ex.
\ag. Nahita ny vehivavy ny mpianatra.\\
 saw the woman the student\\
 `The student saw the woman.'
\bg. ny mpianatra izay nahita ny vehivavy\\
 the student that saw the woman\\
 `the student that saw the woman'
\bg. *ny vehivavy izay nahita ny mpianatra\\
 the woman that saw the student\\
 `the woman that the student saw' \flushfill{Malagasy, \pgcitealt{keenan1977}{70}}

Objects can be passivized and then relativized (again relativization of a subject).

\ex.
\ag. Nohitan' ny mpianatra ny vehivavy.\\
 seen.\ac{pass} the student the woman\\
 `The woman was seen by the student.'
\bg. ny vehivavy izay nohitan' ny mpianatra\\
 the woman that seen.\ac{pass} the student\\
 `the woman that was seen by the student' \flushfill{Malagasy, \pgcitealt{keenan1977}{70}}

In Malay, subjects and objects can be relativized using \tit{yang}. Below I only give an example of a relativized object.

\exg. Ali bunoh ayam yang Aminah sedang memakan.\\
 Ali kill chicken that Aminah \ac{prog} eat\\
 `Ali killed the chicken that Aminah is eating.' \flushfill{Malay, \pgcitealt{keenan1977}{71}}

Indirect objects cannot be relativized in the same way.

\ex.
\ag. Ali beri {ubi kentang} itu kapada perempuan itu.\\
 Ali give potato the to woman the\\
 `Ali gave the potato to the woman.'
\bg. *perempuan yang Ali beri {ubi kentang} itu kapada\\
 woman that Ali give potato the to\\
\bg. *perempuan kapada yang Ali beri {ubi kentang} itu\\
 woman to who Ali give potato that\\ \flushfill{Malay, \pgcitealt{keenan1977}{71}}

A different construction is made.

\exg. perempuan yang menerima {ubi kentang} itu daripada Ali\\
 woman that received potato the from Ali\\
 `the woman that received the potato from Ali'\flushfill{Malay, \pgcitealt{keenan1977}{71}}

In Basque, subjects, objects and indirect objects can be relativized with the same strategy.

\ex.
\ag. Gizon-a-k emakume-a-ri liburu-a eman dio.\\
 man-\ac{def}-\ac{erg} woman-\ac{def}-\ac{dat} book-\ac{def}.\ac{acc} give has\\
 `The man has given the book to the woman.'
\bg. emakume-a-ri liburu-a eman dio-n gizon-a\\
 woman-\ac{def}-\ac{dat} book-\ac{def}.\ac{acc} give has-\ac{rel} man-\ac{def}\\
 `the man who has given the book to the woman'
\bg. gizon-a-k emakume-a-ri eman dio-n liburu-a\\
 man-\ac{def}-\ac{erg} woman-\ac{def}-\ac{dat} give has-\ac{rel} book-\ac{def}\\
 `the book that the man has given to the woman'
\bg. gizon-a-k liburu-a eman dio-n emakume-a\\
 man-\ac{def}-\ac{erg} book-\ac{def}.\ac{acc} give has-\ac{rel} woman-\ac{def}\\
 `the woman that the man has given the book to' \flushfill{Basque, \pgcitealt{keenan1977}{72}}




 \begin{table}[H]
   \center
   \caption {Relativization accessibility}
     \begin{tabular}{cccc}
       \toprule
             \multicolumn{3}{c}{relativization of}
           & \\
       \cmidrule{1-3}
             & direct
             & indirect
           & \\
             subject
             & object
             & object
           & example \\
       \cmidrule{1-3} \cmidrule{4-4}
             ✔
             & *
             & *
           & Malagasy \\
             ✔
             & ✔
             & *
           & Malay \\
             ✔
             & ✔
             & ✔
           & Basque \\
       \bottomrule
     \end{tabular}
 \end{table}






\section{Behavior of the cases in morphology}


\subsection{Suppletion patterns}

\begin{table}[H]
  \center
  \caption {Suppletion patterns}
    \begin{tabular}{cccccccc}
      \toprule
          \multicolumn{3}{c}{pattern}
            & \ac{nom}
            & \ac{acc}
            & \ac{dat}
            & translation
            & language \\
      \cmidrule(lr){1-3} \cmidrule(lr){4-6} \cmidrule(lr){7-7} \cmidrule(lr){8-8}
          A & A & A
            & \cellcolor{LG}\tbf{þ}ú
            & \cellcolor{LG}\tbf{þ}ig
            & \cellcolor{LG}\tbf{þ}ér
            & 2\ac{sg}
            & Icelandic \\
          A & B & B
            & my
            & \cellcolor{LG}\tbf{n}as
            & \cellcolor{LG}\tbf{n}am
            & 1\ac{pl}
            & Russian \\
          A & A & B
            & \cellcolor{LG}\tbf{narnaj}
            & \cellcolor{LG}\tbf{narnaj}(j)i
            & gunga
            & 3\ac{sg}
            & Wardaman \\
          A & B & C
            & zɨ
            & jä
            & as(ɨr)
            & 1\ac{sg}
            & Khinalugh \\
          A & B & A
            & \cellcolor{LG}
            &
            & \cellcolor{LG}
            &
            & not attested \\
      \bottomrule
    \end{tabular}
\end{table}





\subsection{Syncretism patterns}

\begin{table}[H]
  \center
  \caption {Syncretism patterns}
    \begin{tabular}{cccccccc}
      \toprule
          \multicolumn{3}{c}{pattern}
            & \ac{nom}
            & \ac{acc}
            & \ac{dat}
            & translation
            & language \\
      \cmidrule(lr){1-3} \cmidrule(lr){4-6} \cmidrule(lr){7-7} \cmidrule(lr){8-8}
          A & A & A
            & \cellcolor{LG}\tbf{inu}
            & \cellcolor{LG}\tbf{inu}
            & \cellcolor{LG}\tbf{inu}
            & 2\ac{pl}
            & Lavukaleve \\
          A & B & B
            & ta
            & \cellcolor{LG}\tbf{bor}
            & \cellcolor{LG}\tbf{bor}
            & 1\ac{pl}
            & Teribe \\
          A & A & B
            & \cellcolor{LG}\tbf{sie}
            & \cellcolor{LG}\tbf{sie}
            & ihr
            & 3\ac{sg}.\ac{f}
            & German \\
          A & B & C
            & zɨ
            & jä
            & as(ɨr
            & 1\ac{sg}
            & Khinalugh \\
          A & B & A
            & \cellcolor{LG}
            &
            & \cellcolor{LG}
            &
            & not attested \\
      \bottomrule
    \end{tabular}
\end{table}







\ex. \ac{nom} < \ac{acc} < \ac{dat}


\subsection{Morphological containment}

\pgcitealt{nikolaeva1999}{16}

\begin{table}[H]
  \center
	\caption {Transparent case containment in Khanty}
		\begin{tabular}{clll}
		\toprule
              & \ac{1}\ac{sg}
              & \ac{3}\ac{sg}
              & \ac{1}\ac{pl}                           \\
		          \cmidrule{2-4}
    \ac{nom}  & ma
              & luw
              & muŋ                                     \\
    \ac{acc}  & ma\tbf{:-ne:m}
              & luw\tbf{-e:l}
              & muŋ\tbf{-e:w}                           \\
    \ac{dat}  & ma\tbf{:-ne:m}-\textcolor{DG}{\tbf{na}}
              & luw\tbf{-e:l}-\textcolor{DG}{\tbf{na}}
              & muŋ\tbf{-e:w}-\textcolor{DG}{\tbf{na}}  \\
		\bottomrule
		\end{tabular}
\end{table}


\pgcitealt{boretzky1994}{31-46}

\begin{table}[H]
  \center
	\caption {Transparent case containment in Kalderaš Romani}
		\begin{tabular}{cllll}
		\toprule
              & `brother'
              & `brothers'
              & `girl'
              & `girls'                                   \\
		\cmidrule{2-5}
    \ac{nom}  & phral
              & phral-(á)
              & rakl-í
              & rakl-já                                   \\
    \ac{acc}  & phral-\tbf{és}
              & phral-\tbf{én}
              & rakl-\tbf{já}
              & rakl-já-\tbf{n}                           \\
    \ac{dat}  & phral-\tbf{és}-\textcolor{DG}{\tbf{kə}}
              & phral-\tbf{én}-\textcolor{DG}{\tbf{gə}}
              & rakl-\tbf{já}-\textcolor{DG}{\tbf{kə}}
              & rakl-já-\tbf{n}-\textcolor{DG}{\tbf{gə}}  \\
		\bottomrule
		\end{tabular}
\end{table}

\pgcitealt{gippert1987}{23-24}

\begin{table}[H]
  \center
	\caption {Transparent case containment in West Tocharian}
		\begin{tabular}{cll}
		\toprule
              & `horses'
              & `men'                                  \\
		\cmidrule{2-3}
    \ac{nom}  & yakwi
              & eṅkwi                                  \\
    \ac{acc}  & yakwe-\tbf{ṃ}
              & eṅkwe-\tbf{ṃ}                          \\
    \ac{dat}  & yäkwe-\tbf{ṃ}-\textcolor{DG}{\tbf{ts}}
              & eṅkwe-\tbf{ṃ}-\textcolor{DG}{\tbf{ts}} \\
		\bottomrule
		\end{tabular}
\end{table}

\ex. \ac{nom} < \ac{acc} < \ac{dat}

\phantom{nom}




\section{A side note on the genitive}\label{sec:genitive}

\begin{itemize}
  \item possessive
  \item accessibility hierarchy
  \item not available
\end{itemize}
