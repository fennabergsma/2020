\documentclass[11pt,hidelinks]{memoir}

\usepackage{../fenna-files/packages}
\addbibresource{../fenna-files/references.bib}
\usepackage{../fenna-files/abbreviations}

\usepackage{fixltx2e}

\title{A Germanic typology of case competition in headless relatives}
\author{Fenna Bergsma}
\date{\today}

\begin{document}

%% Case competition in headless relatives
\begin{itemize}
  \item case competition: a situation in which two cases are assigned but only one of them surfaces
  \item headless relative: a relative clause constructions that lack a head
\end{itemize}

\exg. Uns besucht \tbf{wen} \tbf{Maria} \tbf{mag}.\\
 Us visits\scsub{[nom]} \tsc{rel}.\ac{acc}.\tsc{an} Maria.\ac{nom} likes\scsub{[acc]}\\
 `Who visits us likes Maria likes.' \flushfill{\pgcitealt{vogel2001}{343}}\label{ex:mg-nom-acc}
%%

%%The content of my dissertation

headless relatives in three Germanic languages:
\begin{itemize}
  \item Gothic
  \item Old High German
  \item Modern German
\end{itemize}

two aspects:
\begin{itemize}
  \item one is stable crosslinguistically
  \item one differs across languages
\end{itemize}
%%

%%The crosslinguistically stable one

\ex. \ac{nom} < \ac{acc} < \ac{dat}

this is not something special

in syntax
\begin{itemize}
  \item agreement
  \item relativization
\end{itemize}

in morphology
\begin{itemize}
  \item syncretism patterns
  \item formal containment
\end{itemize}
%%

%%A reflex of morphology in syntax

\begin{forest} boom
    [\ac{acc}P,name=accp
        [\ac{acc},name=acc]
        [\tsc{nomP},name=nomp
            [\ac{nom},name=nom]
            [NP,name=np
                [...,name=x,
                roof,baseline
                ]
            ]
        ]
    ]
\end{forest}






\begin{table}[H]
  \center
	\caption {\ac{acc}P deletes \ac{nom}P}
		\begin{tabular}[b]{cc}
      \begin{forest} boom
          [\ac{acc}P,name=accp
              [\ac{acc},name=acc]
              [\tsc{nomP},name=nomp,
              tikz={
              \node[draw,circle,
              xscale=0.75,yscale=0.95,
              fit=(nomp)(nom)(x)]{};
              }
                  [\ac{nom},name=nom]
                  [NP,name=np
                      [...,name=x,
                      roof,baseline
                      ]
                  ]
              ]
          ]
      \end{forest}
      &
      \begin{forest} boom
        [\textcolor{LG}{\tsc{nomP}},name=nomp,
        tikz={
        \node[draw,circle,
        xscale=0.75,yscale=0.95,
        fit=(nomp)(nom)(x)]{};
        }
            [\textcolor{LG}{\ac{nom}},name=nom,
            edge=LG]
            [\textcolor{LG}{NP},name=np,
            edge=LG
                [\textcolor{LG}{...},name=x,
                roof,baseline,edge=LG
                ]
            ]
        ]
      \end{forest}\\
  \end{tabular}
\end{table}

%%










Focus of this talk

\exg. \tbf{þan} \tbf{-ei} \tbf{frijos} siuks ist\\
\ac{rel}.\ac{acc}.\ac{m}.\ac{sg} -\ac{comp} love\scsub{[acc]} sick is\scsub{[nom]}\\
`the one whom you love is sick' \flushfill{Gothic, \ac{john} 11:3, adapted from \pgcitealt{harbert1978}{342}}\label{ex:gothic-acc-nom}

\exg. jah þo \tbf{-ei} \tbf{ist} \tbf{us} \tbf{Laudeikaion} jus ussiggwaid\\
and \ac{rel}.\ac{acc}.\ac{n}.\ac{sg} -\ac{comp} is\scsub{[nom]} from Laodicea you read\scsub{[acc]}\\
`and read that which is from Laodicea' \flushfill{Gothic, \ac{col} 4:16, adapted from \pgcitealt{harbert1978}{357}}\label{ex:gothic-nom-acc}




\exg. *Ich {lade ein}, wen \tbf{mir} \tbf{sympathisch} \tbf{ist}.\\
 I invite\scsub{[acc]} \tsc{rel}.\ac{acc}.\tsc{an} me nice is\scsub{[nom]}\\
 `I invite who I like.' \flushfill{\pgcitealt{vogel2001}{344}}

\exg. Uns besucht \tbf{wen} \tbf{Maria} \tbf{mag}.\\
 Us visits\scsub{[nom]} \tsc{rel}.\ac{acc}.\tsc{an} Maria.\ac{nom} likes\scsub{[acc]}\\
 `Who visits us likes Maria likes.' \flushfill{\pgcitealt{vogel2001}{343}}\label{ex:mg-nom-acc}

\exg. ih bibringu fona Juda dhen \tbf{mina} \tbf{berga} \tbf{chisetzit}\\
 I educate\scsub{[acc]} about Juda \tsc{rel}.\ac{acc}.\tsc{m}.\tsc{sg} my mountains {through pull}\scsub{[nom]}\\
 `I educate the one who wanders through my mountains about Judas' \flushfill{\ac{ohg}, \ac{isid} 34:3, \pgcitealt{behaghel1923}{761}}\label{ex:ohg-acc-nom}



\begin{table}[H]
	\center
	\caption {Variation}
		\begin{tabular}{ccc}
		\toprule
		 					      & \ac{int}>\ac{ext}		& \ac{ext}>\ac{int}	\\
								      \cmidrule{2-3}
		Modern German 	& ✔			 							&	*									\\
		Old High German	& *										&	✔									\\
		Gothic		      &	✔										&	✔									\\
		\bottomrule
		\end{tabular}
\end{table}



\end{document}
