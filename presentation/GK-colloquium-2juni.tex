\documentclass[12pt]{beamer}

\geometry{paperwidth=140mm,paperheight=105mm}

\usepackage{../fenna-files/packages-presentation}
\addbibresource{../fenna-files/references.bib}
\usepackage{../fenna-files/abbreviations}

\usepackage{framed} %boxes
\newcommand*{\mybox}[1]{\framebox{#1}} %box around word

\usepackage{booktabs}

% \usepackage{fixltx2e}

% Choose how your presentation looks.
%
% For more themes, color themes and font themes, see:
% http://deic.uab.es/~iblanes/beamer_gallery/index_by_theme.html
%
\mode<presentation>
{
\usetheme{Madrid}      % or try Darmstadt, Madrid, Warsaw, ...
\definecolor{goethe}{rgb}{0,0.37,0.66}
\usecolortheme[named=goethe]{structure}
	% \usefonttheme{serif}  % or try serif, structurebold, ...
\setbeamertemplate{navigation symbols}{}
\setbeamertemplate{caption}[numbered]
\resetcounteronoverlays{exx}
}

\renewcommand{\eachwordone}{\sffamily}
\renewcommand{\eachwordtwo}{\sffamily}
\renewcommand{\eachwordthree}{\sffamily}

%gets rid of bottom navigation bars
\setbeamertemplate{footline}{}

%gets rid of navigation symbols
%\setbeamertemplate{navigation symbols}{}


\addtobeamertemplate{navigation symbols}{}{%
\usebeamerfont{footline}%
\usebeamercolor[fg]{footline}%
\hspace{1em}%
\insertframenumber/\inserttotalframenumber
}


\title{Case competition in headless relatives: a Germanic typology}
\author{Fenna Bergsma}
\institute{\normalsize Goethe-Universität Frankfurt}
\date{GK colloquium\\ \today}


\begin{document}


\begin{frame}
\titlepage

\centering{
\includegraphics[width=0.4\textwidth]{dfg-logo}
\hspace{2cm}
\includegraphics[width=0.3\textwidth]{goethe-logo}}
\end{frame}

\begin{frame}{Case competition in headless relatives}
\begin{itemize}
  \item \tbf{case competition}: a situation in which two cases are assigned but only one of them surfaces \pause
  \item \tbf{headless relative}: a relative clause construction that lacks a head \pause
\end{itemize}

\vspace{1em}

\exg. Ich {lade ein} \mybox{\tbf{wem}} \tbf{auch} \tbf{Maria} \tbf{vertraut}. \\
 I invite\scsub{[acc]} \tsc{rel}.\ac{dat}.\tsc{an} also Maria trusts\scsub{[dat]}.\\
 `I invite whoever Maria also trusts.' \flushfill{\pgcitealt{vogel2001}{344}}\label{ex:mg-acc-dat-intro}

\end{frame}

%

\begin{frame}{The content of my dissertation}

\begin{itemize}
  \item headless relatives in three Germanic languages:
  \begin{itemize}
    \item Gothic
    \item Old High German
    \item Modern German
  \end{itemize}\pause
  \item two aspects:
  \begin{itemize}
    \item one is stable crosslinguistically
    \item one differs crosslinguitically
  \end{itemize}
\end{itemize}

\end{frame}
%%

\begin{frame}{The crosslinguistically stable one}

\ex. \ac{nom} < \ac{acc} < \ac{dat}\pause

\exg. Ich {lade ein} \mybox{\tbf{wem}} \tbf{auch} \tbf{Maria} \tbf{vertraut}. \\
 I invite\scsub{[acc]} \tsc{rel}.\ac{dat}.\tsc{an} also Maria trusts\scsub{[dat]}.\\
 `I invite whoever Maria also trusts.' \flushfill{\pgcitealt{vogel2001}{344}}\label{ex:mg-acc-dat-rep} \pause

this case scale is not unique to case attraction in headless relatives \pause

\begin{itemize}
  \item in syntax
  \begin{itemize}
    \item agreement
    \item relativization
  \end{itemize} \pause
  \item in morphology
  \begin{itemize}
    \item syncretism patterns
    \item formal containment
  \end{itemize}
\end{itemize}


\end{frame}
%%

\begin{frame}{A reflex of morphology in syntax}

\begin{table}[h]
  \center
		\begin{tabular}[b]{cc}
      \begin{forest} boom
      [\tsc{dat}P,name=datp
          [\tsc{dat},name=dat]
          [\ac{acc}P,name=accp,
          tikz={
          \node[draw,circle,
          xscale=0.75,yscale=0.95,
          color=white,
          fit=(accp)(acc)(x)]{};
          }
              [\ac{acc},name=acc],
              tikz={
              \node[draw,circle,
              xscale=0.75,yscale=0.95,
              color=white,
              fit=(nomp)(nom)(x)]{};
              }
              [\tsc{nomP},name=nomp
                  [\ac{nom},name=nom]
                  [NP,name=np
                      [...,name=x,
                      roof,baseline
                      ]
                  ]
              ]
          ]
      ]
      \end{forest}
      &\pause
      \begin{forest} boom
    [\tsc{acc}P,name=accp
        [\tsc{acc},name=acc],
        tikz={
        \node[draw,circle,
        xscale=0.75,yscale=0.95,
        color=white,
        fit=(accp)(acc)(x)]{};
        }
        [{\tsc{nomP}},name=nomp
            [{\ac{nom}},name=nom]
            [NP,name=np
                [...,name=x,
                roof,baseline
                ]
            ]
        ]
    ]
      \end{forest}\\
  \end{tabular}
\end{table}


\end{frame}


\begin{frame}{A reflex of morphology in syntax}

  \begin{table}[H]
    \center
  	% \caption {\ac{dat}P deletes \ac{acc}P}
  		\begin{tabular}[b]{c c}
        \begin{forest} boom
          [\tsc{datP}
              [\ac{dat}]
                [\ac{acc}P,name=accp,
                tikz={
                \node[draw,circle,
                xscale=0.775,yscale=0.975,
                fit=(accp)(acc)(nom)(x)]{};
                }
                  [\ac{acc},name=acc]
                  [\tsc{nomP},name=nomp
                      [\ac{nom},name=nom]
                      [NP,name=np
                          [...,name=x, roof ,baseline]
                      ]
                  ]
              ]
          ]
        \end{forest}
        &
        \begin{forest} boom
          [\textcolor{LG}{\tsc{accP}},name=accp,
          tikz={
          \node[draw,circle,
          xscale=0.775,yscale=0.975,
          fit=(accp)(acc)(nom)(x)]{};
          }
              [\textcolor{LG}{\ac{acc}},name=acc,edge=LG]
              [\textcolor{LG}{\tsc{nomP}},name=nomp,edge=LG
                  [\textcolor{LG}{\ac{nom}},name=nom,edge=LG]
                  [\textcolor{LG}{NP},name=np,edge=LG
                      [\textcolor{LG}{...},name=x,
                      roof, baseline, edge=LG
                      ]
                  ]
              ]
          ]
        \end{forest} \\
    \end{tabular}
  \end{table}

\end{frame}

%%

\begin{frame}{The crosslinguistically differing one}

\begin{itemize}
  \item \tbf{internal case} refers to the case associated with the relative pronoun internal to the relative clause
  \item \tbf{external case} refers to the case associated with the missing head in the main clause, which is external to the relative clause
\end{itemize}

\pause

\begin{table}[h]
	\center
		\begin{tabular}{ccc}
		\toprule
		 					      & \ac{int}>\ac{ext}		& \ac{ext}>\ac{int}	\\
								      \cmidrule{2-3}
		Modern German 	& yes			 						&	no								\\
		Old High German	& no									&	yes								\\
		Gothic		      &	yes									&	yes								\\
		\bottomrule
		\end{tabular}
\end{table}

\end{frame}

%%

\begin{frame}{Gothic: both directions}

\exg. hva nu wileiþ ei taujau \mybox{þamm} \tbf{-ei} \tbf{qiþiþ} \tbf{þiudan} \tbf{Iudaie}?\\
 what now want that do\scsub{[dat]} \ac{rel}.\ac{dat}.\ac{m}.\ac{sg} -\ac{comp} say\scsub{[acc]} king {of Jews}\\
 `what now do you wish that I do to (him) whom you call King of the Jews?' \flushfill{Gothic, \ac{mark} 15:12, adapted from \pgcitealt{harbert1978}{339}}\label{ex:gothic-dat-acc-rep}

\exg. ushafjands \tbf{ana} \mybox{\tbf{þamm}} \tbf{-ei} \tbf{lag}\\
{picking up}\scsub{[acc]} on\scsub{[dat]} \ac{rel}.\ac{dat}.\ac{n}.\ac{sg} -\ac{comp} lay\\
`picking up (that) on which he lay' \flushfill{Gothic, \ac{luke} 5:25, adapted from \pgcitealt{harbert1978}{343}}\label{ex:gothic-acc-dat-rep}

\end{frame}

%%

\begin{frame}{Modern German: only internal}

\exg. *Ich vertraue \mybox{wem} \tbf{auch} \tbf{Maria} \tbf{mag}. \\
 I trust\scsub{[dat]} \tsc{rel}.\ac{dat}.\tsc{an} also Maria likes\scsub{[acc]}.\\
 `I trust whoever Maria also likes.' \flushfill{\pgcitealt{vogel2001}{345}}

\exg. Ich {lade ein} \mybox{\tbf{wem}} \tbf{auch} \tbf{Maria} \tbf{vertraut}. \\
 I invite\scsub{[acc]} \tsc{rel}.\ac{dat}.\tsc{an} also Maria trusts\scsub{[dat]}.\\
 `I invite whoever Maria also trusts.' \flushfill{\pgcitealt{vogel2001}{344}}\label{ex:mg-acc-dat}

\end{frame}

%%

\begin{frame}{Old High German: only external}

\exg. bistû furira Abrâhame, ouh thên \tbf{man} \tbf{hiar} \tbf{nû} \tbf{zalta}?\\
 {are you} older\scsub{[dat]} {to Abraham} also \tsc{rel}.\ac{dat}.\tsc{pl} one here now named\scsub{[acc]}\\
 `are you really older than Abraham and those who have been mentioned here?' \flushfill{\ac{ohg}, \ac{otfrid} III 18:33, \pgcitealt{behaghel1923}{761}}\label{ex:ohg-dat-acc}

\end{frame}

%%


\begin{frame}{Again a reflex of morphology in syntax?}

  \begin{table}[h]
  	\center
  		\begin{tabular}{cccc}
  		\toprule
                      & \multicolumn{2}{l}{headless relatives}   & relative pronoun    \\
  		 					      & \ac{int}>\ac{ext}		& \ac{ext}>\ac{int}	&                      \\
  								      \cmidrule{2-3}   & \cmidrule{4-4}
  		Modern German 	& yes			 						&	no								& \tsc{wh}             \\
  		Old High German	& no									&	yes								& \tsc{d}              \\
  		Gothic		      &	yes									&	yes								& \tsc{d} + \tsc{comp} \\
  		\bottomrule
  		\end{tabular}
  \end{table}

how?

\end{frame}

%%


\begin{frame}{Old High German}

\ex. \ac{acc} instead of \ac{nom}
\ag. unde ne wolden níet besên den mort den dô was geschên\\
 and not wanted not see the murder.\ac{acc} that.\ac{acc} there had happened\\
 `and they didn't want to see the murder that had happened.' \flushfill{MHG, \ac{nib} 1391,14, \pgcitealt{behaghel1923}{756}, after \pgcitealt{pittner1995}{198}}

ellipsis under identity

\end{frame}

%

\begin{frame}

hoi


  	\begin{forest} boom
  	[PP
  			[P
  					[about, roof]
  			]
  			[DP
  					[D
  							[some-\tsc{gen}, roof]
  					]
  					[YP
  							[CP
  									[DP
  											[NP
  													[sufferings-\tsc{gen}, roof]
  											]
  											[DP
  													[DP
  															[which-\tsc{gen}, roof]
  													]
  													[t$_{NP}$ ]
  											]
  									]
  									[IP
  											[happened to her at home, roof]
  									]
  							]
  							[Y
  									[\sout{sufferings}, roof]
  							]
  					]
  			]
  	]
  	\end{forest}

\end{frame}




\end{document}
