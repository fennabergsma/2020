% !TEX root = thesis.tex

\chapter{Aside: languages without case competition}\label{ch:without-case-competition}

In the previous chapter, I discussed languages that show case competition in their headless relatives. These languages form the center of this dissertation, since the topic of the dissertation is case competition in headless relatives. However, there are also languages that do not show any case competition. In this chapter I take a small sidestep to discuss these languages, and gives a typology of headless relatives. Readers who are not interested in this detour can proceed directly to Part \ref{part:deriving}.

In languages without case competition, the internal and external case do not compete to surface on the relative pronoun. It is irrelevant how the two cases relate to each other on the case scale. Instead, it is fixed per language whether the relative pronoun appears in the internal or in the external case. Logically, there are two possible languages without case competition: one that lets the relative pronoun appear in the internal case and one that lets the relative pronoun appear in the external case.

Table \ref{tbl:no-case-competition-int} shows the pattern of a language in which the relative pronoun always appears in the internal case.

\begin{table}[H]
  \center
  \caption{Pattern of the always-internal type of language}
  \begin{tabular}{c|c|c|c}
    \toprule
   \textsubscript{\ac{int}} \textsuperscript{\ac{ext}}
          & [\ac{nom}]
          & [\ac{acc}]
          & [\ac{dat}]
          \\ \cmidrule{1-4}
      [\ac{nom}]
          & \ac{nom}
          & \ac{nom}
          & \ac{nom}
          \\ \cmidrule{1-4}
      [\ac{acc}]
          & \ac{acc}
          & \ac{acc}
          & \ac{acc}
          \\ \cmidrule{1-4}
      [\ac{dat}]
          & \ac{dat}
          & \ac{dat}
          & \ac{dat}
          \\
    \bottomrule
  \end{tabular}
  \label{tbl:no-case-competition-int}
\end{table}

In the second row, the internal case is nominative and the external case is nominative, accusative or dative. The relative pronoun appears in the nominative. It is irrelevant here that the nominative is less complex than the accusative and the dative, because no case competition is taking place. The third row shows that the relative pronoun always appears in the accusative when the internal case is the accusative, and the fourth row shows the same for the dative. I call this pattern the always-internal type. To my knowledge, this type is not attested in any natural language.

Table \ref{tbl:no-case-competition-ext} shows the pattern of a language in which the relative pronoun always appears in the external case.

\begin{table}[H]
  \center
  \caption{Pattern of the always-external type of language}
  \begin{tabular}{c|c|c|c}
    \toprule
   \textsubscript{\ac{int}} \textsuperscript{\ac{ext}}
          & [\ac{nom}]
          & [\ac{acc}]
          & [\ac{dat}]
          \\ \cmidrule{1-4}
      [\ac{nom}]
          & \ac{nom}
          & \ac{acc}
          & \ac{dat}
          \\ \cmidrule{1-4}
      [\ac{acc}]
          & \ac{nom}
          & \ac{acc}
          & \ac{dat}
          \\ \cmidrule{1-4}
      [\ac{dat}]
          & \ac{nom}
          & \ac{acc}
          & \ac{dat}
          \\
    \bottomrule
  \end{tabular}
  \label{tbl:no-case-competition-ext}
\end{table}

In the second column, the external case is nominative and the internal case is nominative, accusative or dative. The relative pronoun appears in the nominative. It is irrelevant here that the nominative is less complex than the accusative and the dative, because no case competition is taking place. The third column shows that the relative pronoun always appears in the accusative when the external case is the accusative, and the fourth column shows the same for the dative. I call this pattern the always-external type.

Section \ref{sec:always-ext} discusses two languages that let their relative pronouns in headless relatives always surface in the external case: Old English and Modern Greek. In Section \ref{sec:typology} I extend the typology from Section \ref{sec:summary-3-patterns} by adding the languages without case competition. As I briefly mentioned, I do not know of any language, whether extinct or alive, that lets the relative pronoun always surface in the internal case.

\section{Always external case}\label{sec:always-ext}

In this section I discuss two languages in which the relative pronoun always appears in the external case. I show that these languages do not show any case competition. In other words, these languages are of the always-external type shown in Table \ref{tbl:no-case-competition-ext} and not of the external-only type I discussed in Section \ref{sec:pattern-iii} or of the unrestricted type of Section \ref{sec:pattern-i}.

Two languages that shows this pattern are Old English and Modern Greek. In this section I discuss the Old English data with examples from \citet{harbert1983}. The Modern Greek data I discuss is taken from \citet{daskalaki2011}. For all examples holds that I made the glosses more detailed, and I added and modified translations.

I start with Old English. I give an example in which the external case is more complex than the internal case and the relative pronoun appears in the most complex external case.

Consider the example in \ref{ex:oe-dat-nom}.
The internal case is nominative, as the predicate \tit{gegyltan} `sin' takes nominative subjects.
The external case is dative, as the predicate \tit{for-gifan} `forgive' takes dative objects.
The relative pronoun \tit{ðam} `\ac{rp}.\ac{pl}.\ac{dat}' appears in the external case: the dative. The relative pronoun is not marked in bold, unlike the relative clause, showing that the relative pronoun patterns with the main clause.

\exg. ðæt is, ðæt man for-gife, ðam \tbf{ðe} \tbf{wið} \tbf{hine} \tbf{gegylte}\\
 that is that one forgive.\ac{subj}.\ac{sg}\scsub{[dat]} \ac{rp}.\ac{pl}.\ac{dat} \ac{comp} against 3\ac{sg}.\ac{m}.\ac{acc} sin.3\ac{sg}\scsub{[nom]}\\
 `that is, that one₂ forgive him₁, who sins against him₂' \flushfill{Old English, adapted from \pgcitealt{harbert1983}{549}} \label{ex:oe-dat-nom}

This example is compatible with three patterns. First, Old English could be a case competition language of the external-only type that only allows the external case to surface. I repeat Table \ref{tbl:case-competition-only-ext-repeated} from Section \ref{sec:pattern-iii} as Table \ref{tbl:oe-poss1}, and I mark the cell that corresponds to example \ref{ex:oe-dat-nom} in gray.

 \begin{table}[H]
   \center
   \caption{Old English headless relatives possibility 1}
   \begin{tabular}{c|c|c|c}
     \toprule
     \textsubscript{\ac{int}} \textsuperscript{\ac{ext}}
            & [\ac{nom}]
            & [\ac{acc}]
            & [\ac{dat}]
            \\ \cmidrule{1-4}
        [\ac{nom}]
            & \ac{nom}
            & \ac{acc}
            & \cellcolor{LG}\ac{dat}
            \\ \cmidrule{1-4}
        [\ac{acc}]
            & *
            & \ac{acc}
            & \ac{dat}
            \\ \cmidrule{1-4}
        [\ac{dat}]
            & *
            & *
            & \ac{dat}
            \\
      \bottomrule
   \end{tabular}
     \label{tbl:oe-poss1}
 \end{table}

Second, Old English could be a case competition language of the unrestricted type that allows the internal case and the external case to surface. I repeat Table \ref{tbl:case-competition-int-ext-repeated} from Section \ref{sec:pattern-i} as Table \ref{tbl:oe-poss2}, and I mark the cell that corresponds to example \ref{ex:oe-dat-nom} in gray.

  \begin{table}[H]
    \center
    \caption{Old English headless relatives possibility 2}
    \begin{tabular}{c|c|c|c}
      \toprule
      \textsubscript{\ac{int}} \textsuperscript{\ac{ext}}
             & [\ac{nom}]
             & [\ac{acc}]
             & [\ac{dat}]
             \\ \cmidrule{1-4}
         [\ac{nom}]
             & \ac{nom}
             & \ac{acc}
             & \cellcolor{LG}\ac{dat}
             \\ \cmidrule{1-4}
         [\ac{acc}]
             & \ac{acc}
             & \ac{acc}
             & \ac{dat}
             \\ \cmidrule{1-4}
         [\ac{dat}]
             & \ac{dat}
             & \ac{dat}
             & \ac{dat}
             \\
       \bottomrule
    \end{tabular}
      \label{tbl:oe-poss2}
  \end{table}

Third, Old English could be a language without case competition of the always-external type that lets the relative pronoun appear in the external case. I repeat Table \ref{tbl:no-case-competition-ext} from the introduction of the chapter as Table \ref{tbl:oe-poss3}, and I mark the cell that corresponds to example \ref{ex:oe-dat-nom} in gray.

 \begin{table}[H]
   \center
   \caption{Old English headless relatives possibility 3}
   \begin{tabular}{c|c|c|c}
     \toprule
    \textsubscript{\ac{int}} \textsuperscript{\ac{ext}}
           & [\ac{nom}]
           & [\ac{acc}]
           & [\ac{dat}]
           \\ \cmidrule{1-4}
       [\ac{nom}]
           & \ac{nom}
           & \ac{acc}
           & \cellcolor{LG}\ac{dat}
           \\ \cmidrule{1-4}
       [\ac{acc}]
           & \ac{nom}
           & \ac{acc}
           & \ac{dat}
           \\ \cmidrule{1-4}
       [\ac{dat}]
           & \ac{nom}
           & \ac{acc}
           & \ac{dat}
           \\
     \bottomrule
   \end{tabular}
   \label{tbl:oe-poss3}
 \end{table}

What sets Table \ref{tbl:oe-poss1}, Table \ref{tbl:oe-poss2} and Table \ref{tbl:oe-poss3} apart is the bottom-left corner of the table. These are situations in which the internal case is more complex than the external case.
In Table \ref{tbl:oe-poss1} the winning case is not allowed to surface, and there is no grammatical headless relative possible. If this is the pattern that Old English shows, then it would be a language of the external-only type, which I claimed in Section \ref{sec:pattern-iii} did not exist.
In Table \ref{tbl:oe-poss2} and in Table \ref{tbl:oe-poss3} there is a relative pronoun that can surface, but the case of the relative pronouns differs. In Table \ref{tbl:oe-poss2}, the relative pronoun surfaces in the most complex case that wins the case competition: the internal case. In Table \ref{tbl:oe-poss3}, there is no case competition taking place, and the relative pronoun surfaces in the external case.

In the example that follows I show that Old English is of the type in Table \ref{tbl:oe-poss3}. I give an example in which the internal case is more complex than the external one. The relative pronoun surfaces in the less complex external case. Old English is namely a language without case competition that always lets the relative pronoun surface in the external case.

Consider the example in \ref{ex:oe-acc-dat}.
The internal case is dative, as the preposition \tit{onuppan} `upon' takes dative objects.
The external case is accusative, as the predicate \tit{tōbrȳsan} `pulverize' takes accusative objects.
The relative pronoun \tit{ðone} `\ac{rp}.\ac{sg}.\ac{m}.\ac{acc}' appears in the external case: the accusative.
The relative pronoun appears in the external case, which is the least complex case of the two. The example is grammatical, because Old English does not show case competition, so the case scale is irrelevant. As long as the relative pronoun appears in the external case, the headless relative is grammatical.

\exg. he tobryst ðone \tbf{ðe} \tbf{he} \tbf{onuppan} \tbf{fylð}\\
 it pulverizes\scsub{[acc]} \ac{rp}.\ac{sg}.\ac{m}.\ac{acc} \ac{comp} it upon\scsub{[dat]} falls\\
`It pulverizes him whom it falls upon.' \flushfill{Old English, adapted from \pgcitealt{harbert1983}{550}} \label{ex:oe-acc-dat}

This example shows that Old English is neither an instance of the pattern in Section \ref{sec:pattern-iii}, in which only the external case is allowed to surface, nor is it an instance of the pattern in Section \ref{sec:pattern-i}, in which the internal case and external case are allowed to surface.
Instead, as illustrated by Table \ref{tbl:no-case-competition-old-english}, the language does not have any case competition. The relative pronoun appears in the external case: the external case can be the most complex case, illustrated by the example in \ref{ex:oe-dat-nom}, marked here in light gray, or it can be the least complex case, illustrated by the example in \ref{ex:oe-acc-dat}, marked here in dark gray.

\begin{table}[H]
  \center
  \caption{Summary of Old English headless relatives}
  \begin{tabular}{c|c|c|c}
    \toprule
   \textsubscript{\ac{int}} \textsuperscript{\ac{ext}}
          & [\ac{nom}]
          & [\ac{acc}]
          & [\ac{dat}]
          \\ \cmidrule{1-4}
      [\ac{nom}]
          & \ac{nom}
          & \ac{acc}
          & \cellcolor{LG}\ac{dat}
          \\ \cmidrule{1-4}
      [\ac{acc}]
          & \ac{nom}
          & \ac{acc}
          & \ac{dat}
          \\ \cmidrule{1-4}
      [\ac{dat}]
          & \ac{nom}
          & \cellcolor{DG}\ac{acc}
          & \ac{dat}
          \\
    \bottomrule
  \end{tabular}
  \label{tbl:no-case-competition-old-english}
\end{table}

I do not discuss more examples from Old English than I did until now. This does not change anything about the point I am making here: the only kind of system that is compatible with the examples given is the one in which the relative pronoun always appears in the external case.

The same pattern appears in Modern Greek. The only difference is that Modern Greek has the genitive, and not the dative. I start again with an example in which the external case is more complex than the internal case and the relative pronoun appears in the most complex external case.

Consider the example in \ref{ex:greek-acc-nom}.
The internal case is nominative, as the predicate \tit{voíθisó} `help' takes nominative subjects.
The external case is accusative, as the predicate \tit{efχarístisó} `thank' takes accusative objects.
The relative pronoun \tit{ópjus} `\ac{rp}.\ac{pl}.\ac{m}.\ac{acc}' appears in the external case: the accusative. The relative pronoun is not marked in bold, unlike the relative clause, showing that the relative pronoun patterns with the main clause.

\exg. Efχarístisa ópjus \tbf{me} \tbf{voíϑisan}.\\
thank.\ac{pst}.3\ac{pl}\scsub{[acc]} \ac{rp}.\ac{pl}.\ac{m}.\ac{acc} \ac{cl}.1\ac{sg}.\ac{acc} help.\ac{pst}.3\ac{pl}\scsub{[nom]}\\
`I thanked whoever helped me.' \flushfill{Modern Greek, adapted from \pgcitealt{daskalaki2011}{80}}\label{ex:greek-acc-nom}

This example is compatible with three patterns. First, Modern Greek could be a case competition language of the external-only type that only allows the external case to surface. I repeat Table \ref{tbl:case-competition-only-ext-repeated} from Section \ref{sec:pattern-iii} as Table \ref{tbl:greek-poss1}, and I mark the cell that corresponds to example \ref{ex:greek-acc-nom} in gray.

 \begin{table}[H]
   \center
   \caption{Modern Greek headless relatives possibility 1}
   \begin{tabular}{c|c|c|c}
     \toprule
     \textsubscript{\ac{int}} \textsuperscript{\ac{ext}}
            & [\ac{nom}]
            & [\ac{acc}]
            & [\ac{gen}]
            \\ \cmidrule{1-4}
        [\ac{nom}]
            & \ac{nom}
            & \cellcolor{LG}\ac{acc}
            & \ac{gen}
            \\ \cmidrule{1-4}
        [\ac{acc}]
            & *
            & \ac{acc}
            & \ac{gen}
            \\ \cmidrule{1-4}
        [\ac{gen}]
            & *
            & *
            & \ac{gen}
            \\
      \bottomrule
   \end{tabular}
     \label{tbl:greek-poss1}
 \end{table}

Second, Modern Greek could be a case competition language of the unrestricted type that allows the internal case and external case to surface. I repeat Table \ref{tbl:case-competition-int-ext-repeated} from Section \ref{sec:pattern-i} as Table \ref{tbl:greek-poss2}, and I mark the cell that corresponds to example \ref{ex:greek-acc-nom} in gray.

  \begin{table}[H]
    \center
    \caption{Modern Greek headless relatives possibility 2}
    \begin{tabular}{c|c|c|c}
      \toprule
      \textsubscript{\ac{int}} \textsuperscript{\ac{ext}}
             & [\ac{nom}]
             & [\ac{acc}]
             & [\ac{gen}]
             \\ \cmidrule{1-4}
         [\ac{nom}]
             & \ac{nom}
             & \cellcolor{LG}\ac{acc}
             & \ac{gen}
             \\ \cmidrule{1-4}
         [\ac{acc}]
             & \ac{acc}
             & \ac{acc}
             & \ac{gen}
             \\ \cmidrule{1-4}
         [\ac{gen}]
             & \ac{gen}
             & \ac{gen}
             & \ac{gen}
             \\
       \bottomrule
    \end{tabular}
      \label{tbl:greek-poss2}
  \end{table}

Third, Modern Greek could be a language without case competition of the always-external type that lets the relative pronoun appear in the external case. I repeat Table \ref{tbl:no-case-competition-ext} from the introduction of the chapter as Table \ref{tbl:greek-poss3}, and I mark the cell that corresponds to example \ref{ex:greek-acc-nom} in gray.

 \begin{table}[H]
   \center
   \caption{Modern Greek headless relatives possibility 3}
   \begin{tabular}{c|c|c|c}
     \toprule
    \textsubscript{\ac{int}} \textsuperscript{\ac{ext}}
           & [\ac{nom}]
           & [\ac{acc}]
           & [\ac{gen}]
           \\ \cmidrule{1-4}
       [\ac{nom}]
           & \ac{nom}
           & \cellcolor{LG}\ac{acc}
           & \ac{gen}
           \\ \cmidrule{1-4}
       [\ac{acc}]
           & \ac{nom}
           & \ac{acc}
           & \ac{gen}
           \\ \cmidrule{1-4}
       [\ac{gen}]
           & \ac{nom}
           & \ac{acc}
           & \ac{gen}
           \\
     \bottomrule
   \end{tabular}
   \label{tbl:greek-poss3}
 \end{table}

What sets Table \ref{tbl:greek-poss1}, Table \ref{tbl:greek-poss2} and Table \ref{tbl:greek-poss3} apart is the bottom-left corner of the table. These are cases in which the internal case is more complex than the external case.
In Table \ref{tbl:greek-poss1} the winning case is not allowed to surface, and there is no grammatical headless relative possible. If this is the pattern that Modern Greek shows, then it would be a language of the external-only type, which I claimed in Section \ref{sec:pattern-iii} did not exist.
In Table \ref{tbl:greek-poss2} and in Table \ref{tbl:greek-poss3} there is a relative pronoun that can surface, but the case of the relative pronouns differs. In Table \ref{tbl:greek-poss2}, the relative pronoun surfaces in the most complex case that wins the case competition: the internal case. In Table \ref{tbl:greek-poss3}, there is no case competition taking place, and the relative pronoun surfaces in the external case.

In the example that follows I show that Modern Greek is of the type in Table \ref{tbl:greek-poss3}. I give an example in which the internal case is more complex than the external one. The relative pronoun surfaces in the less complex external case. Modern Greek is namely a language without case competition that always lets the relative pronoun surface in the external case.

Consider the example in \ref{ex:greek-nom-acc}.
The internal case is accusative, as the predicate \tit{irθó} `invite' takes accusative objects.
The external case is nominative, as the predicate \tit{kálesó} `come' takes nominative subjects.
The relative pronoun \tit{ópji} `\ac{rp}.\ac{pl}.\ac{m}.\ac{nom}' appears in the external case: the nominative.
The relative pronoun appears in the external case, which is the least complex case of the two. The example is grammatical, because Modern Greek does not show case competition, so the case scale is irrelevant. As long as the relative pronoun appears in the external case, the headless relative is grammatical.

\exg. Irθan ópji \tbf{káleses}.\\
come.\ac{pst}.3\ac{pl}\scsub{[nom]} \ac{rp}.\ac{pl}.\ac{m}.\ac{nom} invite.\ac{pst}.2\ac{sg}\scsub{[acc]}\\
`Whoever you invited came.'\flushfill{Modern Greek, adapted from \pgcitealt{daskalaki2011}{80}}\label{ex:greek-nom-acc}

The example in \ref{ex:greek-nom-acc-u} is identical to \ref{ex:greek-nom-acc}, except for that the relative pronoun appears in the internal more complex case.
The relative pronoun is marked in bold, just as the relative clause, showing that the relative pronoun patterns with the relative clause.
This example is ungrammatical: the relative pronoun does not appear in the external case. The fact that the internal case is more complex is irrelevant.

\exg. *Irθan \tbf{ópjus} \tbf{káleses}.\\
come.\ac{pst}.3\ac{pl}\scsub{[nom]} \ac{rp}.\ac{pl}.\ac{m}.\ac{acc} invite.\ac{pst}.2\ac{sg}\scsub{[acc]}\\
`Whoever you invited came.'\flushfill{Modern Greek, adapted from \pgcitealt{daskalaki2011}{79}}\label{ex:greek-nom-acc-u}

This example shows that Modern Greek is neither an instance of the pattern in Section \ref{sec:pattern-iii}, in which only the external case is allowed to surface, nor is it an instance of the pattern in Section \ref{sec:pattern-i}, in which the internal case and external case are allowed to surface.
Instead, as illustrated by Table \ref{tbl:no-case-competition-greek}, the language does not have any case competition. The relative pronoun appears in the external case: the external case can be the most complex case, illustrated by the example in \ref{ex:greek-acc-nom}, marked here in light gray, or it can be the least complex case, illustrated by the example in \ref{ex:greek-nom-acc}, marked here in dark gray.

\begin{table}[H]
  \center
  \caption{Summary of Modern Greek headless relatives}
  \begin{tabular}{c|c|c|c}
    \toprule
   \textsubscript{\ac{int}} \textsuperscript{\ac{ext}}
          & [\ac{nom}]
          & [\ac{acc}]
          & [\ac{gen}]
          \\ \cmidrule{1-4}
      [\ac{nom}]
          & \ac{nom}
          & \cellcolor{LG}\ac{acc}
          & \ac{gen}
          \\ \cmidrule{1-4}
      [\ac{acc}]
          & \cellcolor{DG}\ac{nom}
          & \ac{acc}
          & \ac{gen}
          \\ \cmidrule{1-4}
      [\ac{gen}]
          & \ac{nom}
          & \ac{acc}
          & \ac{gen}
          \\
    \bottomrule
  \end{tabular}
  \label{tbl:no-case-competition-greek}
\end{table}

There is something more to be said about the situation in Modern Greek. When the internal case is genitive instead of accusative, a clitic is added to the sentence to make it grammatical.

Consider the example in \ref{ex:greek-nom-gen}.
The internal case is genitive, as the predicate \tit{eðósó} `give' takes genitive objects.
The external case is accusative, as the predicate \tit{efχarístisó} `thank' takes nominative subjects.
The relative pronoun \tit{ópjon} `\ac{rp}.\ac{pl}.\ac{m}.\ac{nom}' appears in the external case: the nominative.
The relative pronoun appears in the external case, which is the least complex case of the two. The example is grammatical, because Modern Greek does not show case competition, so the case scale is irrelevant. As long as the relative pronoun appears in the external case, the headless relative is grammatical. In addition, the relative clause obligatorily contains the genitive clitic \tit{tus} `\ac{cl}.3\ac{pl}.\ac{gen}'.\footnote{
In Modern German, it is possible to insert a light head to resolve a situation with a more complex external case. However, then the relative pronoun has to change as well (from a \tsc{wh}-pronoun into a \tsc{d}-pronoun). I assume this is a different construction, and the Modern Greek one with the clitic inserted is not.
}

\exg. Me efχarístisan ópji \tbf{tus} \tbf{íχa} \tbf{ðósi} \tbf{leftá}.\\
 \ac{cl}.1\ac{sg}.\ac{acc} thank.\ac{pst}.3\ac{pl}\scsub{[nom]} \ac{rp}.\ac{pl}.\ac{m}.\ac{nom} \ac{cl}.3\ac{pl}.\ac{gen} have.\ac{pst}.1\ac{sg} give.\ac{ptcp}\scsub{[gen]} money\\
 `Whoever I had given money to, thanked me.'\flushfill{Modern Greek, adapted from \pgcitealt{daskalaki2011}{80}}\label{ex:greek-nom-gen}

This once again confirms the picture of Modern Greek always letting the relative pronoun surface in the external case. The internal case is taken care of by the clitic, which is independent of the relative clause construction.

I do not discuss more examples from Modern Greek than I did until now. This does not change anything about the point I am making here: the only kind of system that is compatible with the examples given is the one in the relative pronoun always appears in the external case. For more examples that illustrate this pattern, I refer the reader to \pgcitet{daskalaki2011}{79-80} and \pgcitet{spyropoulos2011}{31-34}.\footnote{
When the relative clause is dislocated, both the internal and the external case can be used. In \ref{ex:greek-dislocated-acc-nom}, the internal case is accusative, and the external case is nominative. Normally the relative pronoun should appear in the external case, so the nominative. However, the accusative is also grammatical here.

\ex.
\ag. ópjos/ ópjon epiléksume θa pári to vravío\\
\ac{rp}.\ac{sg}.\ac{m}.\ac{nom}/ \ac{rp}.\ac{sg}.\ac{m}.\ac{acc} choose.1\tsc{pl}\scsub{[acc]} \tsc{fut} take.3\tsc{sg}\scsub{[nom]} the price.\tsc{acc}\\
`Whoever we may choose, he will get the price.'\label{ex:greek-dislocated-acc-nom}
\bg. ópjos/ ópjon me aɣapá ton aɣapó\\
\ac{rp}.\ac{sg}.\ac{m}.\ac{nom}/ \ac{rp}.\ac{sg}.\ac{m}.\ac{acc} \ac{cl}.1\ac{sg}.\ac{acc} love.3\tsc{sg}\scsub{[nom]}
\ac{cl}.3\ac{sg}.\ac{m}.\ac{acc} love.1\tsc{sg}\scsub{[acc]}\\
`Whoever loves me, I love him.'\label{ex:greek-dislocated-nom-acc}

\citet{spyropoulos2011} argues that in these left-dislocated structures, there is a silent \tit{pro} or a clitic (\tit{ton} in \ref{ex:greek-dislocated-nom-acc}) that satisfies the external case. This allows the relative pronoun to take the internal case. This makes this construction more of a correlative.
}\footnote{
Some accusatives in Modern Greek always require a clitic \citep[see][]{spyropoulos2011}. I assume this is because these are different types of accusatives \citep[see][]{starke2017}.
}

In sum, Old English and Modern Greek are languages without case competition in their headless relatives. The relative pronoun always appears in the external case.

\section{A typology of headless relatives}\label{sec:typology}

This section provides a typological overview of headless relatives. First, I describe the difference between the patterns of languages with and without case competition. Second, I add a parameter to the diagram I showed in Section \ref{sec:summary-3-patterns} to include languages without case competition. Third, I give an overview of all logically possible patterns, I show how the diagram generates the attested ones, and I discuss the non-attested patterns.

In Section \ref{sec:pattern-i} to \ref{sec:pattern-iv}, I discussed four different patterns. These four patterns are all based on a single table, shown in Table \ref{tbl:case-competition-int-ext-typology} (repeated from Section \ref{sec:pattern-i}).

\begin{table}[H]
  \center
  \caption{Relative pronoun follows case competition (repeated)}
  \begin{tabular}{c|c|c|c}
    \toprule
    \textsubscript{\ac{int}} \textsuperscript{\ac{ext}}
           & [\ac{nom}]
           & [\ac{acc}]
           & [\ac{dat}]
           \\ \cmidrule{1-4}
       [\ac{nom}]
           & \ac{nom}
           & \ac{acc}
           & \ac{dat}
           \\ \cmidrule{1-4}
       [\ac{acc}]
           & \ac{acc}
           & \ac{acc}
           & \ac{dat}
           \\ \cmidrule{1-4}
       [\ac{dat}]
           & \ac{dat}
           & \ac{dat}
           & \ac{dat}
           \\
     \bottomrule
  \end{tabular}
    \label{tbl:case-competition-int-ext-typology}
\end{table}

The cases in the cells are the ones that win the case competition.
The variation between the four patterns lies in whether all cells in the table are grammatical, or whether some of them are not. In none of the four patterns in Section \ref{sec:pattern-i} to \ref{sec:pattern-iv}, the cells are filled by a case different from what is given in Table \ref{tbl:case-competition-int-ext-typology}.

In this chapter I introduced two different ways of filling out the table. The first one is the one in which the relative pronoun appears in the internal case, as in Table \ref{tbl:no-case-competition-int-typology}, repeated from Table \ref{tbl:no-case-competition-int}.

\begin{table}[H]
  \center
  \caption{Relative pronoun in internal case}
  \begin{tabular}{c|c|c|c}
    \toprule
   \textsubscript{\ac{int}} \textsuperscript{\ac{ext}}
          & [\ac{nom}]
          & [\ac{acc}]
          & [\ac{dat}]
          \\ \cmidrule{1-4}
      [\ac{nom}]
          & \ac{nom}
          & \ac{nom}
          & \ac{nom}
          \\ \cmidrule{1-4}
      [\ac{acc}]
          & \ac{acc}
          & \ac{acc}
          & \ac{acc}
          \\ \cmidrule{1-4}
      [\ac{dat}]
          & \ac{dat}
          & \ac{dat}
          & \ac{dat}
          \\
    \bottomrule
  \end{tabular}
  \label{tbl:no-case-competition-int-typology}
\end{table}

The second one is the one in which the relative pronoun appears in the external case, as in Table \ref{tbl:no-case-competition-ext-typology} (repeated from Table \ref{tbl:no-case-competition-ext}).

\begin{table}[H]
  \center
  \caption{Relative pronoun in external case}
  \begin{tabular}{c|c|c|c}
    \toprule
   \textsubscript{\ac{int}} \textsuperscript{\ac{ext}}
          & [\ac{nom}]
          & [\ac{acc}]
          & [\ac{dat}]
          \\ \cmidrule{1-4}
      [\ac{nom}]
          & \ac{nom}
          & \ac{acc}
          & \ac{dat}
          \\ \cmidrule{1-4}
      [\ac{acc}]
          & \ac{nom}
          & \ac{acc}
          & \ac{dat}
          \\ \cmidrule{1-4}
      [\ac{dat}]
          & \ac{nom}
          & \ac{acc}
          & \ac{dat}
          \\
    \bottomrule
  \end{tabular}
  \label{tbl:no-case-competition-ext-typology}
\end{table}

I showed in section \ref{sec:always-ext} that only the always-external pattern is attested.
I incorporate the parameter that models this pattern into the diagram from Section \ref{sec:summary-3-patterns} in Figure \ref{fig:typology}.

\begin{figure}[H]
  \centering
  \begin{adjustbox}{max width=\textwidth}
  \begin{tabular}[b]{c}
      \toprule
		    \begin{tikzpicture}[node distance=1.5cm]
    \node (question1) [question]
    {\small{cases considered}};
        \node (outcome1) [outcome, below of=question1, xshift=-2cm, yshift=-0.5cm]
        {\small{always external}};
            \node (example1) [example, below of=outcome1, yshift=0.5cm]
            {\small{e.g. Old English, Modern Greek (5)\\\phantom{x}}};
    \node (question2) [question, below of=question1, xshift=2.5cm, yshift=-1cm]
      {\small{allow \tsc{int}}}; %ϕ+\tsc{k} portmanteau?
          \node (outcome2) [outcome, below of=question2, xshift=-2cm, yshift=-0.5cm]
          {\small{matching}};
              \node (example2) [example, below of=outcome2, yshift=0.5cm]
              {\small{e.g. Polish (9)\\\phantom{x}\\\phantom{x}}};
          \node (question3) [question, below of=question2, xshift=2.5cm, yshift=-1cm]
          {\small{allow \tsc{ext}}}; %\tsc{lh}-\ac{rp} syncretism?
              \node (outcome3) [outcome, below of=question3, xshift=-2cm, yshift=-0.5cm]
              {\small{internal-only}};
                  \node (example3) [example, below of=outcome3, yshift=0.5cm]
                  {\small{e.g. Modern German (3)\\\phantom{x}}};
              \node (outcome4) [outcome, below of=question3, xshift=2cm, yshift=-0.5cm]
              {\small{unrestricted}};
                  \node (example4) [example, below of=outcome4, yshift=0.5cm]
                  {\small{e.g. Gothic, Old High German, Classical Greek (2)}};

    \draw [arrow] (question1) -- node[anchor=north] {\small{\ac{int} + \ac{ext}}} (question2);
    \draw [arrow] (question1) -- node[anchor=east] {\small{\ac{ext}}} (outcome1);
    \draw [arrow] (question2) -- node[anchor=east] {\small{no}} (outcome2);
    \draw [arrow] (question2) -- node[anchor=west] {\small{yes}} (question3);
    \draw [arrow] (question3) -- node[anchor=east] {\small{no}} (outcome3);
    \draw [arrow] (question3) -- node[anchor=west] {\small{yes}} (outcome4);
    \end{tikzpicture}\\
    \bottomrule
  \end{tabular}
  \end{adjustbox}
    \caption{Three descriptive parameters generate four language types}
    \label{fig:typology}
\end{figure}

I added one parameter. This parameter, \tit{cases considered}, concerns which cases are considered to surface on the relative pronoun. This parameter distinguishes the always-external type of language from the matching, the internal-only and the unrestricted type of languages.
If the internal and the external case are considered, the language has case competition, and the pattern shown in Table \ref{tbl:case-competition-int-ext-typology} is generated. The two parameters that follow (\tit{allow \tsc{int}} and \tit{allow \tsc{ext}}) come into play, as described in Section \ref{sec:summary-3-patterns}.

If only the external case is considered, the always-external type of language is generated, illustrated by the pattern in Table \ref{tbl:no-case-competition-ext-typology}.
I left out the option for languages to only consider the internal case, because there does not seem to be a language that uses that strategy.
In this dissertation I do not offer an explanation for why this type of example should be absent. Future research should determine whether this pattern is actually attested, or whether this option should be excluded and how.

In Table \ref{tbl:possible-headless-relatives}, I give all logically possible patterns for headless relatives.

\begin{table}[H]
  \center
  \caption{Logically possible patterns for headless relatives}
    \begin{tabular}{ccc|ccc}
    \toprule
      &   \multicolumn{2}{c}{[\ac{int}]>[\ac{ext}]} & \multicolumn{2}{|c}{[\ac{ext}]>[\ac{int}]} &                  \\
          \cmidrule(lr){2-3}                      \cmidrule(lr){4-5}
      &   \ac{int}            & \ac{ext}          & \ac{int}          & \ac{ext}            & language              \\
          \cmidrule(lr){2-2}  \cmidrule(lr){3-3}  \cmidrule(lr){4-4}  \cmidrule(lr){5-5}    \cmidrule(lr){6-6}
    1 &   ✔                   & *                 & ✔               & *                     & n.a.                  \\
    2 &   ✔                   & *                 & *               & ✔                     & e.g. Old High German  \\
    3 &   ✔                   & *                 & *               & *                     & e.g. Modern German    \\
    4 &   {*}                 & ✔                 & ✔               & *                     & n.a.                  \\
    5 &   {*}                 & ✔                 & *               & ✔                     & e.g. Old English      \\
    6 &   {*}                 & ✔                 & *               & *                     & n.a.                  \\
    7 &   {*}                 & *                 & ✔               & *                     & n.a.                  \\
    8 &   {*}                 & *                 & *               & ✔                     & n.a.                  \\
    9 &   {*}                 & *                 & *               & *                     & e.g. Polish           \\
    \bottomrule
  \end{tabular}
    \label{tbl:possible-headless-relatives}
\end{table}

The top row sketches two different situations: \tit{[\ac{int}]>[\ac{ext}]} is the one in which the internal case is the most complex, and \tit{[\ac{ext}]>[\ac{int}]} is the one in which the external case is the most complex. The second row refers to the case the relative pronoun appears in, which can be either the internal case (\ac{int}) or the external case (\ac{ext}). The checkmark indicates that a relative pronoun surfaces in that particular situation in that particular case. The asterix indicates that the relative pronoun does not surface in that particular situation in that particular case.

When the internal case and the external case differ (which holds for both options the top row indicates), the relative pronoun cannot appear in both the internal and external case at the same time. This excluded the possibility of having a checkmark at both \ac{int} and \ac{ext} in the same situation. This leaves the possibility to have a checkmark at \ac{int}, at \ac{ext} or at none of them. This gives 3 × 3 = 9 logically possible options, which are listen in Table \ref{tbl:possible-headless-relatives}.
In what follows I show how Figure \ref{fig:typology} generates of all logically possible patterns only the attested patterns.

I start with the left-most pattern in Figure \ref{fig:typology}, which is number 5 in Table \ref{tbl:possible-headless-relatives}. In this pattern, there is no case competition, and the relative pronoun surfaces in the external case. This pattern is exemplified by Old English and Modern Greek.
The second pattern in Figure \ref{fig:typology} is number 9 in Table \ref{tbl:possible-headless-relatives}. In this pattern, there is case competition, and the relative pronoun is only allowed to surface in the case when there is a tie, i.e. when the internal and external case match. This pattern is exemplified by Polish.
The third pattern in Figure \ref{fig:typology} is number 3 in Table \ref{tbl:possible-headless-relatives}. In this pattern, there is case competition, and the relative pronoun is only allowed to surface in the internal case when it wins the case competition. This pattern is exemplified by Modern German.
The fourth and last pattern in Figure \ref{fig:typology} is number 2 in Table \ref{tbl:possible-headless-relatives}. In this pattern, there is case competition, and the relative pronoun is allowed to surface in the internal case and the external case when either of them wins the case competition. This pattern is exemplified by Old High German, Gothic and Classical Greek.

This leaves five patterns that are logically possible but not attested in languages: pattern numbers 1, 4, 6, 7 and 8 in Table \ref{tbl:possible-headless-relatives}. These patterns cannot be generated by the diagram in Figure \ref{fig:typology}. That means that they are not a result of any of the possible parameter settings in the diagram. I start with discussing patterns 4, 5 and 6, and then I turn to the patterns with number 1 and 8.

In the pattern number 4, the relative pronoun surfaces in the external case when the internal case is the most complex, and the relative pronoun surfaces in the internal case when the external case is the most complex. In other words, the relative pronoun  appears in the losing case in the case competition.
Pattern number 6 and 7 are both subsets of pattern number 4 in the sense that they allow part of what number 4 allows.
In the pattern number 6, the relative pronoun surfaces in the external case when the internal case is the most complex, and there is no grammatical option when the external case is the most complex.
Patterns number 7 is the opposite of pattern number 6: there is no grammatical option when the external case is the most complex, and the relative pronoun surfaces in the internal case when the external case is the most complex.
The absence of these three patterns across languages provides further evidence for the case scale in Chapter \ref{ch:recurring}.

In the pattern number 1, there is no case competition, and the relative pronoun surfaces in the internal case. As I mentioned earlier, I am not aware of a language that exemplified this pattern and future research should tell whether this option is attested or whether it should be excluded.
In the pattern number 8, the relative pronoun is only allowed to surface in the external case when it wins the case competition. This pattern is excluded as a result of the relative ordering of \tit{allow \tsc{int}} and \tit{allow \tsc{ext}} in the diagram in Figure \ref{fig:typology}. The next part of this dissertation discusses the linguistic counterpart of this ordering.


\section{Summary and discussion}

In Chapter \ref{ch:typology} I discussed different types of languages with case competition. In this chapter I showed languages without case competition. Logically, there are two languages possible without case competition: (1) languages of the always-internal type, in which the relative pronoun always appears in the internal case, and (2) languages of the always-external type, in which the relative pronoun always appears in the external case. To my knowledge, languages of the always-external type are attested, but languages of the always-internal type are not. I do not have an explanantion for why this is the case.

I also do not offer an analysis of the always-external type of language. It seems surprising that a language always takes the case of the main clause, even though a relative pronoun is often claimed to be part of the relative clause. I can see two options for how relative pronouns can take the external case. First, the relative pronoun was actually never in the relative clause, but it was always part of the main clause. Second, the relative pronoun was first part of the relative clause but it has moved to the main clause.

The first option might be what is going on in Old English. It is possible to analyze the relative pronoun as the (light) head of the relative clause. The complementizer \tit{ðe} should then license the internal case. Note that it is not the case that all languages with an overt complementizer behave like this. Gothic also has a complementizer (\tit{ei}), and it is an unrestricted type of language. The difference between the two languages should then be that the Old English complementizer is able to spell out case features, while the Gothic complementizer is not.
At first sight, there does not seem to be any support for this type of analysis for Modern Greek. Possibly, the second option I suggested can be considered for this language. Future research should shed more light on the analysis for these languages.
