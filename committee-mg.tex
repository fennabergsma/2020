% !TEX root = thesis.tex


\section{Deriving the internal-only type}\label{sec:deriving-only-internal}

Internal-only languages can be summarizes as in Table \ref{tbl:overview-rel-light-mg}.

\begin{table}[htbp]
  \center
  \caption{The surface pronoun with differing cases in Modern German}
\begin{tabular}{cccc}
  \toprule
                & \tsc{k}\scsub{int} > \tsc{k}\scsub{ext} & \tsc{k}\scsub{ext} > \tsc{k}\scsub{int} &   \\
                \cmidrule{2-3}
internal-only   & relative pronoun\scsub{int} & *  & Modern German    \\
\bottomrule
\end{tabular}
\label{tbl:overview-rel-light-mg}
\end{table}

A language of the internal-only type (like Modern German) allows only the internal case to surface when it wins the case competition. This means that the relative pronoun with its internal case can be the surface pronoun. A language of this type does not allow the external case to surface when it wins the case competition. This means that the light head with its external case cannot be the surface pronoun. The goal of this section is to derive these properties from the way light heads and relative pronouns are spelled out in Modern German.

The section is structured as follows.
First, I discuss the relative pronoun. According to my assumptions in Section \ref{sec:basic-idea}, relative pronouns are part of the relative clause. I confirm this independently for Modern German with data from extraposition. I decompose the relative pronouns into three morphemes, and I show which features each of the morphemes corresponds to.
Then I discuss the light head. I argue that Modern German headless relatives are derived from a light-headed relative clause that does not surface in the language. I show that the light head corresponds to one of the morphemes of the relative pronoun.
Finally, I compare the constituents of the light head and the relative pronoun.
When the internal and the external case match, the relative pronoun can delete the light head, because it contains all its constituents.
When the internal case is more complex than the external case, the relative pronoun can still delete the light head, for the same reason: the relative pronoun contains all constituents of the light head.
This is no longer the case when the external case is more complex than the internal case. The light head does not contain all constituents of the relative pronoun, and the relative pronoun does not contain all constituents of the light head. As a result, there is no grammatical form to surface when the external case is more complex.

\subsection{The relative pronoun}\label{sec:mg-rel}

In this section I discuss the relative pronoun in Modern German headless relatives.
First, I show, independent from case facts, that the surface pronoun is the relative pronoun. The evidence comes from extraposition data.

The sentences in \ref{ex:mg-extrapose-cp} show that it is possible to extrapose a CP. In \ref{ex:mg-extrapose-cp-base}, the clausal object \tit{wie es dir geht} `how you are doing', marked here in bold, appears in its base position. It can be extraposed to the right edge of the clause, shown in \ref{ex:mg-extrapose-cp-moved}.

\ex.\label{ex:mg-extrapose-cp}
\ag. Mir ist \tbf{wie} \tbf{es} \tbf{dir} \tbf{geht} egal.\\
1\tsc{sg}.\tsc{dat} is how it 2\tsc{sg}.\tsc{dat} goes {the same}\\
`I don't care how you are doing.'\label{ex:mg-extrapose-cp-base}
\bg. Mir is egal \tbf{wie} \tbf{es} \tbf{dir} \tbf{geht}.\\
1\tsc{sg}.\tsc{dat} is {the same} how it 2\tsc{sg}.\tsc{dat} goes\\
`I don't care how you are doing.' \label{ex:mg-extrapose-cp-moved}\flushfill{Modern German}

\ref{ex:mg-extrapose-dp} illustrates that it is impossible to extrapose a DP. The clausal object of \ref{ex:mg-extrapose-cp} is replaced by the simplex noun phrase \tit{die Sache} `that matter'.
In \ref{ex:mg-extrapose-dp-base} the object, marked in bold, appears in its base position. In \ref{ex:mg-extrapose-dp-moved} it is extraposed, and the sentence is no longer grammatical.

\ex.\label{ex:mg-extrapose-dp}
\ag. Mir ist \tbf{die} \tbf{Sache} egal.\\
1\tsc{sg}.\tsc{dat} is that matter {the same}\\
`I don't care about that matter.'\label{ex:mg-extrapose-dp-base}
\bg. *Mir ist egal \tbf{die} \tbf{Sache}.\\
1\tsc{sg}.\tsc{dat} is {the same} that matter\\
`I don't care about that matter.' \label{ex:mg-extrapose-dp-moved}\flushfill{Modern German}

The same asymmetry between CPs and DPs can be observed with relative clauses. A relative clause is a CP, and the head of a relative clause is a DP. The sentences in \ref{ex:extra-headed} contain the relative clause \tit{was er gekocht hat} `what he has stolen'. This is marked in bold in the examples. The (light) head of the relative clause is \tit{das}.\footnote{
Not all speakers of Modern German accept the combination of \tit{das} as a light head and \tit{was} as a relative pronoun and prefer \tit{das} as a relative pronoun instead. I use the combination of \tit{das} and \tit{was} to have a more minimal pair with the headless relatives (that uses the relative pronoun \tit{was}).
\label{ftn:das-was}
}
In \ref{ex:extra-headed-base}, the relative clause and its head appear in base position. In \ref{ex:extra-headed-only-clause}, the relative clause is extraposed. This is grammatical, because it is possible to extrapose CPs in Modern German. In \ref{ex:extra-headed-head-clause}, the relative clause and the head are extraposed. This is ungrammatical, because it is possible to extrapose DPs.

\ex.\label{ex:extra-headed}
\ag. Jan hat das, \tbf{was} \tbf{er} \tbf{gekocht} \tbf{hat}, aufgegessen.\\
Jan has that what he cooked has eaten\\
`Jan has eaten what he cooked.'\label{ex:extra-headed-base}
\bg. Jan hat das aufgegessen, \tbf{was} \tbf{er} \tbf{gekocht} \tbf{hat}.\\
Jan has that eaten what he cooked has\\
`Jan has eaten what he cooked.'\label{ex:extra-headed-only-clause}
\cg. *Jan hat aufgegessen, das, \tbf{was} \tbf{er} \tbf{gekocht} \tbf{hat}.\\
Jan has eaten that what he cooked has\\
`Jan has eaten what he cooked.'\label{ex:extra-headed-head-clause} \flushfill{Modern German}

The same can be observed in relative clauses without a head. \ref{ex:extra-headless} is the same sentence as in \ref{ex:extra-headed} only without the overt head. The relative clause is marked in bold again.
In \ref{ex:extra-headless-base}, the relative clause appears in base position. In \ref{ex:extra-headless-clause}, the relative clause is extraposed. This is grammatical, because it is possible to extrapose CPs in Modern German. In \ref{ex:extra-headless-no-rel}, the relative clause is extraposed without the relative pronouns. This is ungrammatical, because the relative pronoun is part of the CP.
This shows that the relative pronoun in headless relatives in Modern German are necessarily part of a CP, which is here a relative clause.

\ex.\label{ex:extra-headless}
\ag. Jan hat \tbf{was} \tbf{er} \tbf{gekocht} \tbf{hat} aufgegessen.\\
Jan has what he cooked has eaten\\
`Jan has eaten what he cooked.'\label{ex:extra-headless-base}
\bg. Jan hat aufgegessen \tbf{was} \tbf{er} \tbf{gekocht} \tbf{hat}.\\
Jan has eaten what he cooked has\\
`Jan has eaten what he cooked.'\label{ex:extra-headless-clause}
\bg. *Jan hat \tbf{was} aufgegessen \tbf{er} \tbf{gekocht} \tbf{hat}.\\
Jan has what eaten he cooked has\\
`Jan has eaten what he cooked.'\label{ex:extra-headless-no-rel}\flushfill{Modern German}

In conclusion, extraposition facts show that the surface pronoun in Modern German headless relatives is the relative pronoun.

Now I turn to the internal structure of the relative pronoun. In Section \ref{sec:basic-idea} I gave the structure in \ref{ex:mg-simplified-rel} as a simplified representation of the relative pronoun.

\ex.\label{ex:mg-simplified-rel}
\begin{forest} boom
  [\tsc{rel}P
      [\tsc{rel}]
      [\tsc{k}P
          [ϕP
              [\phantom{x}ϕ\phantom{x}, roof]
          ]
          [\tsc{k}P,
              [\tsc{k}]
          ]
      ]
  ]
\end{forest}

In what follows, I give the non-simplified representation. It is important to carefully establish the feature content of the relative pronoun. This constituents that it forms are namely determining whether the relative pronoun can delete the light head or not. Moreover, the features that I introduce for Modern German are present in the same way in the other two language types.

I discuss two relative pronouns: the animate nominative singular and in the animate accusative singular. These are the two forms that I compare the constituents of in Section \ref{sec:comparing-mg}. I show them in \ref{ex:mg-rels}.

\ex.\label{ex:mg-rels}
\ag. w-e-r\\
 `\tsc{rel}.\tsc{an}.\tsc{sg}.\tsc{nom}'\\
\bg. w-e-n\\
 `\tsc{rel}.\tsc{an}.\tsc{sg}.\tsc{acc}'\\

I decompose the relative pronouns in three morphemes: the \tit{w}, the \tit{e} and the final consonant. For each morpheme, I discuss which features they spell out, and I give their lexical entries. In the end, I derive the relative pronouns, given here in \ref{ex:mg-spellout-rel-nom-preview} and \ref{ex:mg-spellout-rel-acc-preview}.

\ex.\label{ex:mg-spellout-rel-nom-preview}
\scriptsize{
\begin{forest} boompje
  [\tsc{rel}P, s sep=12mm
      [\tsc{rel}P,
      tikz={
      \node[label=below:\tit{w},
      draw,circle,
      scale=0.9,
      fit to=tree]{};
      }
          [\tsc{rel}]
          [\tsc{wh}P
              [\tsc{wh}]
              [\tsc{dx}\scsub{2}]
          ]
      ]
      [\tsc{med}P, s sep=21mm
          [\tsc{med}P,
          tikz={
          \node[label=below:\tit{e},
          draw,circle,
          scale=0.85,
          fit to=tree]{};
          }
              [\tsc{dx}\scsub{2}]
              [\tsc{prox}P
                  [\tsc{dx}\scsub{1}]
                  [\tsc{ind}P
                      [\tsc{ind}]
                      [\tsc{an}]
                  ]
              ]
          ]
          [\tsc{nom}P,
          tikz={
          \node[label=below:\tit{r},
          draw,circle,
          scale=0.95,
          fit to=tree]{};
          }
              [\ac{f}1]
              [\tsc{ind}P
                  [\tsc{ind}]
                  [\tsc{an}P
                      [\tsc{an}]
                      [\tsc{cl}P
                          [\tsc{cl}]
                          [ΣP
                              [Σ]
                              [\tsc{ref}]
                          ]
                      ]
                  ]
              ]
          ]
      ]
  ]
\end{forest}
}

\ex.\label{ex:mg-spellout-rel-acc-preview}
\scriptsize{
\begin{forest} boompje
  [\tsc{rel}P, s sep=12mm
      [\tsc{rel}P,
      tikz={
      \node[label=below:\tit{w},
      draw,circle,
      scale=0.9,
      fit to=tree]{};
      }
          [\tsc{rel}]
          [\tsc{wh}P
              [\tsc{wh}]
              [\tsc{dx}\scsub{2}]
          ]
      ]
      [\tsc{med}P, s sep=23mm
          [\tsc{med}P,
          tikz={
          \node[label=below:\tit{e},
          draw,circle,
          scale=0.85,
          fit to=tree]{};
          }
              [\tsc{dx}\scsub{2}]
              [\tsc{prox}P
                  [\tsc{dx}\scsub{1}]
                  [\tsc{ind}P
                      [\tsc{ind}]
                      [\tsc{an}]
                  ]
              ]
          ]
          [\tsc{acc}P,
          tikz={
          \node[label=below:\tit{n},
          draw,circle,
          scale=0.95,
          fit to=tree]{};
          }
              [\ac{f}2]
              [\tsc{nom}P
                  [\ac{f}1]
                  [\tsc{ind}P
                      [\tsc{ind}]
                      [\tsc{an}P
                          [\tsc{an}]
                          [\tsc{cl}P
                              [\tsc{cl}]
                              [ΣP
                                  [Σ]
                                  [\tsc{ref}]
                              ]
                          ]
                      ]
                  ]
              ]
          ]
      ]
  ]
\end{forest}
}

I continue with the final consonants: \tit{r} and \tit{n}. They can be observed in several contexts besides relative pronouns. Table \ref{tbl:mg-dieser} gives an overview of the demonstrative \tit{dieser} `this' in Modern German in two numbers, three genders and three cases.\footnote{
The vowel preceding the final consonant is written as \tit{e}. I write it as \tit{ə}, because this is how it is pronounced. I make this distinction to emphasize that this differs from the vowel used in the relative pronouns.
}
Compare the final consonants in Table \ref{tbl:mg-paradigm-wh-rels} and Table \ref{tbl:mg-dieser}.

\begin{table}[htbp]
\center
\caption {Modern German demonstrative \tit{dieser} `this' \pgcitep{durrell2011}{Table 5.2}}
 \begin{tabular}{ccccc}
 \toprule
             & \ac{m}.\tsc{sg}    & \tsc{n}.\tsc{sg}  & \ac{f}.\tsc{sg}  & \tsc{pl} \\
   \cmidrule{2-5}
   \ac{nom}  & dies-ə-r           & dies-ə-s          & dies-ə            & dies-ə    \\
   \ac{acc}  & dies-ə-n           & dies-ə-s          & dies-ə            & dies-ə    \\
   \ac{dat}  & dies-ə-m           & dies-ə-m          & dies-ə-r          & dies-ə-n  \\
 \bottomrule
 \end{tabular}
 \label{tbl:mg-dieser}
\end{table}

\begin{table}[htbp]
\center
\caption {Modern German relative pronouns \pgcitep{durrell2011}{5.3.3} (repeated)}
\begin{tabular}{ccc}
\toprule
            & \ac{an}  & \tsc{inan}\\
  \cmidrule{2-3}
  \ac{nom}  & w-e-r    & w-a-s     \\
  \ac{acc}  & w-e-n    & w-a-s     \\
  \ac{dat}  & w-e-m    & (w-e-m)   \\
\bottomrule
\end{tabular}
\label{tbl:mg-paradigm-wh-rels}
\end{table}

Table \ref{tbl:mg-dieser} and \ref{tbl:mg-paradigm-wh-rels} show that the final consonants take different shapes depending on gender, number and case. I conclude from that that the consonant realizes features having to do with these three aspects.

Another context in which this consonant appears is in their use as a pronoun. More specifically, the final consonant corresponds to the weak pronoun in Modern German, which I illustrate in the following examples. I only give examples of the nominative and accusative masculine singular, because these are the forms used in the relative pronoun.

First, I show that the consonant is not a strong pronoun.
The example in \ref{ex:wk-pron-coord} illustrates this by showing that the weak pronoun cannot be coordinated.

\ex.\label{ex:wk-pron-coord}
\ag. Jan und er/ *r essen gerne Dampfnudeln.\\
 Jan und he.\tsc{str}/ he.\tsc{wk} eat {with pleasure} \tit{Dampfnudeln}\\
 `Jan and he like to eat \tit{Dampfnudeln}.'
\bg. Ich habe Jan und ihn/ n gesehen.\\
 I have Jan and him.\tsc{str}/ him.\tsc{wk} seen\\
 `I saw Jan and him.'

The example in \ref{ex:wk-pron-focus} illustrates the same point by showing that the weak pronoun cannot be focused.

\ex.\label{ex:wk-pron-focus}
\ag. Nur er/ *r isst gerne Saumagen.\\
 only he.\tsc{str}/ he.\tsc{wk} eats {with pleasure} \tit{Saumagen}\\
 `Only he likes \tit{Saumagen}'
\bg. Ich habe nur ihn/*n gesehen.\\
 I have only him.\tsc{str}/ him.\tsc{wk} seen\\
 `I saw only him.'

Second, I show that the consonant is not a clitic.
The example in \ref{ex:wk-pron-order} illustrates this by showing that the weak pronoun obligatorily follows dative objects.

\ex.\label{ex:wk-pron-order}
\ag. .. dass Jan Ursel ihn/ n empfohlen hat.\\
 {} that Jan Ursel him.\tsc{str}/ him.\tsc{wk} recommended has\\
 `that Jan recommended him to Ursel.'
\bg. *.. dass Jan ihn/ n Ursel emphfohlen hat.\\
 {} that Jan him.\tsc{str}/ him.\tsc{wk} Ursel recommended has\\
 `that Jan recommended him to Ursel.'

The example in \ref{ex:wk-pron-prep} illustrates the same point by showing that the weak pronoun can appear after prepositions (which clitics cannot).

\ex.\label{ex:wk-pron-prep}
\ag. Ich habe schon ein Geschenk für n gekauft.\\
 I have already a gift for him.\tsc{wk} bought\\
 `I already bought a gift for him.'
\bg. Ich habe gestern gegen n gespielt.\\
I have yesterday against him.\tsc{wk} played\\
`Yesterday I played against him.'
\bg. Ich habe ein schönen Brief an n geschrieben.\\
I have a nice letter to him.\tsc{wk} written\\
`I wrote a nice letter to him.'
\bg. Ich bin schnell auf n zu gelaufen.\\
I am fast on him.\tsc{wk} to walked\\
`I walked toward him fast.'

In sum, besides gender, number and case features, the final consonant of relative pronoun spell out pronominal features.

Since I discuss the animate nominative singular and in the animate accusative singular, I only introduce features that are realized by these morphemes.
For case, I adopt the features of \citet{caha2009}, already introduced in Chapter \ref{ch:decomposition}. The feature \ac{f}1 corresponds to the nominative, and the features \ac{f}1 and \ac{f}2 correspond to the accusative.

For number and gender, I adopt the features that are distinguished by \citet{harley2002} for pronouns. The feature \tsc{cl} corresponds to a gender feature, which is inanimate or neuter if it is not combined with any other features. Combining \tsc{cl} with the feature \tsc{an} gives the animate or masculine gender.\footnote{
If the features \tsc{cl} and \tsc{an} are combined with the feature \tsc{fem}, it becomes the feminine gender.
}
The feature \tsc{ind} corresponds to number, which is singular if it is not combined with any other features.

Regarding pronominal features, I assume the feature \tsc{ref} to be present. \citet{harley2002} claim that all pronouns contain this feature, because they are referential expressions. In addition, the feature Σ is present because it is a weak pronoun \citep{cardinaletti1994}.\footnote{
I assume that clitics lack the features \tsc{ref} (which corresponds to the LP in \pgcitealt{cardinaletti1994}{61}) and the feature Σ. Strong pronouns have, in addition to \tsc{ref} and Σ, another feature (C in terms of \pgcitealt{cardinaletti1994}{61}).
}

% \citet{baker2003} claims that it also present in nouns, and that this feature distinguishes nouns from other categories.

I give the lexical entries for \tit{r} and \tit{n} in \ref{ex:mg-entry-r} and \ref{ex:mg-entry-n}.
The \tit{r} is the nominative masculine singular, so it spells out the features \tsc{ref}, Σ, \tsc{cl}, \tsc{an}, \tsc{ind} and \ac{f}1. The \tit{n} is the accusative masculine singular, so it spells out the features that the \tit{r} spells out plus \ac{f}2.

\ex.\label{ex:mg-entries-rn}
\a.\begin{forest} boom
  [\tsc{nom}P
      [\ac{f}1]
      [\tsc{ind}P
          [\tsc{ind}]
          [\tsc{an}P
              [\tsc{an}]
              [\tsc{cl}P
                  [\tsc{cl}]
                  [ΣP
                      [Σ]
                      [\tsc{ref}]
                  ]
              ]
          ]
      ]
  ]
  {\draw (.east) node[right]{⇔ \tit{r}}; }
\end{forest}
\label{ex:mg-entry-r}
\b. \begin{forest} boom
  [\tsc{acc}P
      [\ac{f}2]
      [\tsc{nom}P
          [\ac{f}1]
          [\tsc{ind}P
              [\tsc{ind}]
              [\tsc{an}P
                  [\tsc{an}]
                  [\tsc{cl}P
                      [\tsc{cl}]
                      [ΣP
                          [Σ]
                          [\tsc{ref}]
                      ]
                  ]
              ]
          ]
      ]
  ]
  {\draw (.east) node[right]{⇔ \tit{n}}; }
\end{forest}
\label{ex:mg-entry-n}

Note that the ordering of the features here is not random. I motivate this later on in this section.

This leaves the \tit{e} in the relative pronoun. This morpheme is present in elements such as demonstratives and (\tsc{wh}-)relative pronouns. It spells out gender and number features and features regarding deixis. I start with the deixis features. In relative pronouns it does not express spatial deixis, but discourse deixis: it establishes a relation with an antecedent.

I assume that the \tsc{wh}-relative pronoun combines with the medial or the distal (when distinguishing between proximal, medial and distal). English has morphological evidence for this claim. Demonstratives in English can combine with either the proximal (\tit{this}) or this medial/distal (\tit{that}). \tsc{wh}-pronouns combine with the medial/distal (\tit{what}) and are ungrammatical when combined with the proximal (*\tit{whis}).

The use of the medial in \tsc{wh}-pronouns can be understood conceptually if one connects spatial deixis to discourse deixis \citep[cf.][]{colasanti2019}. The proximal is spatially near the speaker, and it refers to knowledge that the speaker possesses. The medial is spatially near the hearer, and it refers to knowledge that the hearer possesses. The distal is spatially away from the speaker and the hearer, and refers to knowledge that neither of them possess. In \tsc{wh}-pronouns, the speaker is not aware of the knowledge, so the use of the proximal is excluded. Since I do not have explicit evidence for the presence of the distal, I assume that it is the medial that combines with the \tsc{wh}-pronoun.

I adopt the features for deixis distinguished by \citet{lander2018}. The feature \tsc{dx}\scsub{1} corresponds to the proximal, the features \tsc{dx}\scsub{1} and \tsc{dx}\scsub{2} correspond to the medial, and the features \tsc{dx}\scsub{1}, \tsc{dx}\scsub{2} and \tsc{dx}\scsub{3} correspond to the distal.
The difference between the proximal, the medial and the distal cannot be observed in Modern German, because it is syncretic all of them \pgcitep{lander2018}{387}, see Table \ref{tbl:mg-paradigm-dem}.

What can be distinguished in Modern German is the differences of the vowel depending on number and gender.

\begin{table}[H]
\center
\caption {Modern German demonstratives \pgcitep{durrell2011}{5.4.1}}
 \begin{tabular}{ccc}
 \toprule
            & \ac{m} & \tsc{n} \\
   \cmidrule{2-3}
   \ac{sg}  & d-e-r  & d-a-s   \\
   \ac{pl}  & d-ie   & d-ie    \\
 \bottomrule
 \end{tabular}
 \label{tbl:mg-paradigm-wh-rels}
\end{table}

So, in sum:

\ex. %to be revised, add the \tsc{ind}
\begin{forest} boom
  [\tsc{dist}P
      [\tsc{dx}\scsub{3}]
      [\tsc{med}P
          [\tsc{dx}\scsub{2}]
          [\tsc{prox}P
              [\tsc{dx}\scsub{1}]
              [\tsc{ind}P
                  [\tsc{ind}]
                  [\tsc{an}]
              ]
          ]
      ]
  ]
  {\draw (.east) node[right]{⇔ \tit{e}}; }
  \label{ex:mg-entry-e}
\end{forest}


This leaves the morpheme \tit{w} of the relative pronoun. Compare Table \ref{tbl:mg-paradigm-wh-rels-rep} (repeated from Table \ref{tbl:mg-paradigm-wh-rels}) and Table \ref{tbl:mg-paradigm-dem}. The \tit{w} combines with the same endings as the \tit{d} does in demonstratives (or relative pronouns in headed relatives).\footnote{
Note here that the \tsc{wh}-relative pronouns, unlike the demonstratives, do not have a feminine form for the relative pronouns in Table \ref{tbl:mg-paradigm-wh-rels-rep}. Demonstratives also have plural forms (which are not given here), and \tsc{wh}-relative pronouns do not. As far as I know, this holds for all relative pronouns in languages of the internal-only type (cf. also for Finnish, even though it makes a lot of morphological distinctions) and of the matching type. Relative pronouns in languages of the unrestricted type do inflect for feminine, as well as always-external languages. In Chapter \ref{ch:discussion} I return to this observation in relation with the always-external languages.
}

\begin{table}[H]
\center
\caption {Modern German relative pronouns \pgcitep{durrell2011}{5.3.3}}
 \begin{tabular}{ccc}
 \toprule
             & \ac{an}  & \tsc{inan}\\
   \cmidrule{2-3}
   \ac{nom}  & w-er    & w-as     \\
   \ac{acc}  & w-en    & w-as     \\
   \ac{dat}  & w-em    & (w-em)   \\
 \bottomrule
 \end{tabular}
 \label{tbl:mg-paradigm-wh-rels-rep}
\end{table}

\begin{table}[H]
\center
\caption {Modern German demonstrative pronouns \pgcitep{durrell2011}{5.4.1}} %dieser source
 \begin{tabular}{cccc}
 \toprule
             & \ac{m}  & \tsc{n} & \ac{f} \\
   \cmidrule{2-4}
   \ac{nom}  & d-er   & d-as   & d-ie    \\
   \ac{acc}  & d-en   & d-as   & d-ie    \\
   \ac{dat}  & d-em   & d-em   & d-er    \\
 \bottomrule
 \end{tabular}
 \label{tbl:mg-paradigm-dem}
\end{table}

This identifies the \tit{d} and, more importantly for the discussion here, the \tit{w} as a separate morpheme. Three features that \tit{w} spells out are important for the discussion here.

The first feature I refer to as \tsc{wh}. This is a feature that \tsc{wh}-pronouns, such as \tsc{wh}-relative pronouns and interrogatives, share. The \tsc{wh}-element triggers the construction of a set of alternatives in the sense of \citet{rooth1985,rooth1992} \citep{hachem2015}. This contrasts with the \tsc{d} in Table \ref{tbl:mg-paradigm-dem}, which is responsible for establishing a definite reference.

The second relevant feature is \tsc{rel}, which establishes a relation. A language that overtly shows that \tsc{wh}-relative pronouns consist of two features is Hungarian. \ref{ex:hungarian-rel-wh} gives three examples of \tsc{wh}-pronouns that combine with the marker \tit{a} to become a \tsc{wh}-relative pronoun.

\exg. a-mi, a-ki, a-melyik \\
 \tsc{rel}-what \tsc{rel}-who \tsc{rel}-which\\
 \flushfill{Kenesei et al. 1998: 40}\label{ex:hungarian-rel-wh}

The third feature is \tsc{dx}\scsub{2}.. \tsc{wh}-element + `away from the speaker'

In sum, the \tit{w} spells out the features \tsc{wh} and \tsc{rel}, shown in \ref{ex:mg-entry-w}.

\ex. \begin{forest} boom
  [\tsc{rel}P
      [\tsc{rel}]
      [\tsc{wh}P
          [\tsc{wh}]
          [\tsc{dx}\scsub{2}]
      ]
  ]
  {\draw (.east) node[right]{⇔ \tit{w}}; }
\end{forest}\label{ex:mg-entry-w}

At this point, I gave lexical entries for each of the morphemes (in \ref{ex:mg-entry-r}, \ref{ex:mg-entry-n}, \ref{ex:mg-entry-e} and \ref{ex:mg-entry-w})
and I showed what the relative pronouns as a whole look like (in \ref{ex:mg-structure-rel-nom} and \ref{ex:mg-structure-rel-acc}).
What is still needed, is a theory for combining these morphemes into a relative pronoun. This theory should determine which morphemes should be combined with each other in which order. Ideally, the theory is not language-specific, but the same for all languages. In what follows I show how this is accomplished in Nanosyntax. Readers who are not interested in the precise mechanics can proceed directly to Section \ref{sec:light-mg}.

The way Nanosyntax combines different morphemes is not by glueing them together directly from the lexicon. Instead, features are merged one by one using two components that drive the derivation. These two components are (1) a functional sequence, in which the features that need to be merged and their order in which they are merged are specified, and (2) the spellout algorithm, which describes the spellout procedure. The lexical entries that are available within a language interact with the derivation in such a way that the morphemes get combined in the right way. Note that the functional sequence and the spellout algorithm are stable across languages. The only difference between languages lies in their lexical entries.

\ref{ex:fseq-wh-rel} shows the functional sequence for relative pronouns. It gives all features it contains and their hierarchical ordering.

\ex. \begin{forest} for tree={s sep=13mm, inner sep=0, l=0}
[(\tsc{acc}P)
   [(\ac{f}2)]
   [\tsc{nom}P
       [\ac{f}1]
       [\tsc{rel}P
           [\tsc{rel}]
           [\tsc{wh}P
               [\tsc{wh}]
               [\tsc{med}P
                   [\tsc{dx}\scsub{2}]
                   [\tsc{prox}P
                       [\tsc{dx}\scsub{1}]
                       [\tsc{ind}P
                           [\tsc{ind}]
                           [\tsc{an}P
                               [\tsc{an}]
                               [\tsc{cl}P
                                   [\tsc{cl}]
                                   [ΣP
                                        [Σ]
                                        [\tsc{ref}]
                                   ]
                               ]
                           ]
                       ]
                   ]
               ]
           ]
       ]
   ]
]
\end{forest}
\label{ex:fseq-wh-rel}

Starting from the bottom, these are pronominal features (\tsc{ref} and Σ) and features having to do with deixis (\tsc{dx}\scsub{1} and \tsc{dx}\scsub{2}), gender features (\tsc{cl} and \tsc{an}), number features (\tsc{ind}), operator features (\tsc{wh} and \tsc{rel}) and case features (\ac{f}1 and \ac{f}2). This order is independently supported by work in the literature. Both Picallo and Kramer argue that number is hierarchically higher than gender. Case is agreed to be higher than number (cf. Bittner and Hale).

\tsc{ref}, Σ, \tsc{deix}, \tsc{wh/rel}?

Features are merged one by one according to the functional sequence, starting from the bottom. Spellout is cyclic, as stated in \ref{ex:cyclic-phrasal-spellout}.

\ex. Cyclic phrasal spellout. Caha:declension\\
Spellout must successfully apply to the output of every Merge F operation. After successfull spellout, the derivation may terminate, or proceed to another round of Merge F.\label{ex:cyclic-phrasal-spellout}

After each instance of merge, the constructed phrase must be spelled out. Spellout happens according to the spellout algorithm, given in \ref{ex:spellout-algorithm}.

\ex. \tbf{Spellout Algorithm:}\label{ex:spellout-algorithm}
 \a. Merge F and spell out.\label{ex:spellout-algorithm-phrasal}
 \b. If \ref{ex:spellout-algorithm-phrasal} fails, move the Spec of the complement and spell out.\label{ex:spellout-algorithm-spec}
 \b. If \ref{ex:spellout-algorithm-spec} fails, move the complement of F and spell out.\label{ex:spellout-algorithm-comp}

I informally reformulate what is in \ref{ex:spellout-algorithm}. I start with the first line in \ref{ex:spellout-algorithm-phrasal}. This says that a feature F is merged, and the newly created phrase FP is attempted to spell out.
The next two lines, \ref{ex:spellout-algorithm-spec} and \ref{ex:spellout-algorithm-comp}, describe two types of rescue movements that take place when the spellout in \ref{ex:spellout-algorithm-phrasal} fails (i.e. when there is no match in the lexicon).
In the discussion about Modern German, only the first line leads to matching lexical entries. The second and third line do not lead to a match in the Modern German derivations I run. I introduce these two steps here anyway, because they will lead to successful matched in Polish.

I illustrate this by merging \tsc{ref} and Σ, creating a ΣP.

\ex.
\begin{forest} boom
  [ΣP
       [Σ]
       [\tsc{ref}]
  ]
\end{forest}

The syntactic structure is contained in the lexical tree in \ref{ex:mg-entry-r-rep}, repeated from \ref{ex:mg-entry-r}, which corresponds to the \tit{r}.

\ex.
\begin{forest} boom
  [\tsc{nom}P
      [\ac{f}1]
      [\tsc{ind}P
          [\tsc{ind}]
          [\tsc{an}P
              [\tsc{an}]
              [\tsc{cl}P
                  [\tsc{cl}]
                  [ΣP
                      [Σ]
                      [\tsc{ref}]
                  ]
              ]
          ]
      ]
  ]
  {\draw (.east) node[right]{⇔ \tit{r}}; }
\end{forest}
\label{ex:mg-spellout-r-rep}

Therefore, the ΣP is spelled out as \tit{r}. As usual, I mark this by circling the part of the structure that corresponds to the lexical entry, and placing the corresponding phonology under it.

\ex.
\begin{forest} boom
  [ΣP,
  tikz={
  \node[label=below:\tit{r},
  draw,circle,
  scale=0.85,
  fit to=tree]{};
  }
       [Σ]
       [\tsc{ref}]
  ]
\end{forest}
\label{ex:mg-spellout-e-refs}


%
% The next step in the derivation is to merge the following feature in the functional sequence in \ref{ex:fseq-wh-rel}. This is the feature \tsc{dx}, which creates the \tsc{dx}P.
%
% \ex.
% \begin{forest} boom
%   [\tsc{dx}P
%       [\tsc{dx}]
%       [\tsc{ref},
%       tikz={
%       \node[label=below:\tit{e},
%       draw,circle,
%       scale=0.85,
%       fit to=tree]{};
%       }
%           [\tsc{ref}2]
%           [\tsc{ref}1]
%       ]
%   ]
% \end{forest}
% \label{ex:mg-spellout-e-refs}
%
% The \tsc{dx}P is also contained in the lexical tree in \ref{ex:mg-entry-e-rep}. Therefore, the \tsc{dx}P spells out as \tit{e}, illustrated in \ref{ex:mg-spellout-e-dx}.
%
% \ex.
% \begin{forest} boom
%   [\tsc{dx}P,
%   tikz={
%   \node[label=below:\tit{e},
%   draw,circle,
%   scale=0.85,
%   fit to=tree]{};
%   }
%       [\tsc{dx}]
%       [\tsc{ref}
%           [\tsc{ref}2]
%           [\tsc{ref}1]
%       ]
%   ]
% \end{forest}
% \label{ex:mg-spellout-e-dx}
%
% Note here that the result of our derivation so far is not \tit{e}-\tit{e}. This is stated in \ref{ex:cyclic-override}.
%
% \ex. \tbf{Cyclic Override} \citep{starke2018}:\\
% Lexicalisation at a node XP overrides any previous match at a phrase contained in XP.
% \label{ex:cyclic-override}
%
% I reformulate this informally. If a lexical tree matches a syntactic structure, the lower matching items are replaced. This holds when the structure matches twice with the same lexical entry, and also when it matches with different lexical entries.
%
% The next feature in the functional sequence is \tsc{cl}, and a \tsc{cl}P is created.
%
% \ex.
% \begin{forest} boom
%   [\tsc{cl}{}
%       [\tsc{cl}]
%       [\tsc{dx}P,
%       tikz={
%       \node[label=below:\tit{e},
%       draw,circle,
%       scale=0.85,
%       fit to=tree]{};
%       }
%           [\tsc{dx}]
%           [\tsc{ref}
%               [\tsc{ref}2]
%               [\tsc{ref}1]
%           ]
%       ]
%   ]
% \end{forest}
% \label{ex:mg-spellout-e-class}

% Here a problem arises. The lexical entry for \tit{e} does not contain the feature \tsc{cl} to make it a \tsc{cl}P.
%
% \ex.
% \begin{forest} boom
%   [\tsc{dx}P
%       [\tsc{dx}P,name=tgt
%           [\tit{e}, roof]
%       ]
%       [\tsc{cl}P
%           [\tsc{cl}]
%           [\tsc{dx}P,name=src,
%           tikz={
%           \node[label=below:\tit{e},
%           draw,circle,
%           scale=0.85,
%           fit to=tree]{};
%           }
%               [\tsc{dx}]
%               [\tsc{ref}
%                   [\tsc{ref}2]
%                   [\tsc{ref}1]
%               ]
%           ]
%       ]
%   ]
%   \draw[->,dashed] (src) to[out=south west,in=east] (tgt);
% \end{forest}
% \label{ex:mg-spellout-e-r}

% The next feature that is merged feature \tsc{dx}\scsub{1} is
%
% There are two other lexical entry that contains the feature \tsc{cl}, namely the one in \ref{ex:mg-entry-r} which spells out as \tit{r} and the one in \ref{ex:mg-entry-r} which spells out as \tit{n}. I repeat the one in \ref{ex:mg-entry-r} here as \ref{ex:mg-entry-r-rep}, because it has the least amount of unused features. Following the Elsewhere Principle, which says that a lexical tree with the least amount of unused features wins, this is the lexical entry that wins.
%
% \ex.\begin{forest} boom
%   [\tsc{nom}P
%       [\ac{f}1]
%       [\tsc{ind}P
%           [\tsc{ind}]
%           [\tsc{an}P
%               [\tsc{an}]
%               [\tsc{cl}P
%                   [\tsc{cl}]
%               ]
%           ]
%       ]
%   ]
%   {\draw (.east) node[right]{⇔ \tit{r}}; }
% \end{forest}
% \label{ex:mg-entry-r-rep}
%
% As a result, the \tsc{cl}P spells out as \tit{r}.
%
% \ex.
% \begin{forest} boom
%   [\tsc{dx}P
%       [\tsc{dx}P,name=tgt
%           [\tit{e}, roof]
%       ]
%       [\tsc{cl}P,
%       tikz={
%       \node[label=below:\tit{r},
%       draw,circle,
%       scale=0.85,
%       fit to=tree]{};
%       }
%           [\tsc{cl}]
%       ]
%   ]
% \end{forest}
% \label{ex:mg-spellout-e-r}
%
% Note here that it is crucial that the lexical entry for \tit{r} in \ref{ex:mg-entry-r-rep} has a unary bottom. Therefore, it can be inserted as the result of movement. What follows is that the lexical entry follows the existing structure and is spelled out as a suffix.
%
% The spellout proceeds by merging one by one the features \tsc{an} and \tsc{ind} (see the functional sequence in \ref{ex:fseq-wh-rel}), and spelling them out using the spellout algorithm.

The next point of interest arises when the feature \tsc{dx}\scsub{1} is merged.
This feature cannot spell out together with all features merged so far (as the option in \ref{ex:spellout-algorithm-phrasal}).
There is no spec, so the second option is impossible.
Finally, it is impossible for \tsc{wh} to be spelled out as part of a suffix (as the option in \ref{ex:spellout-algorithm-spec}). This last option is impossible, because the lexical entry that contains the feature \tsc{wh} has a binary bottom. I repeat the lexical entry from \ref{ex:mg-entry-w} in \ref{ex:mg-entry-w-rep}.



The derivation turns to the last resort option, which is to build a complex left branch.

\ex. \tbf{Spec Formation} \citep{starke2018}:\\
If Merge F has failed to spell out, try to spawn a new derivation providing the feature F and merge that with the current derivation, projecting the feature F at the top node.\label{ex:specformation}

.. but the \tsc{wh} is on its own.. is \tsc{rel} merged right away?

The last problem is the case feature. What happens then is backtracking + elements are split up, merged onto both of them, case can be spelled out with suffix.




%
% The nominative masculine singular relative pronoun is built as follows.
% The \tsc{ref}P is spelled out as \tit{e}.
% The \tsc{ref}P is merged with \tsc{dx}\scsub{1}, and the whole phrase (\tsc{prox}P) is spelled out as \tit{e}.
% The \tsc{prox}P is merged with \tsc{dx}\scsub{2}, and the whole phrase (\tsc{med}P) is spelled out as \tit{e}.
%
% The \tsc{med}P is merged with \tsc{cl}. There is no lexical entry that matches the whole phrase. There is no specifier to move, so this movement is irrelevant. The complement of \tsc{cl}, the \tsc{med}P, is moved to the specifier of \tsc{cl}P, and the \tsc{cl}P is spelled out as \tit{r}.
% The \tsc{cl}P is merged with \tsc{an}. There is no lexical entry that matches the whole phrase. The specifier of \tsc{cl}P, the \tsc{med}P, is moved to the specifier of \tsc{an}P, and the \tsc{an}P is spelled out as \tit{r}.
% The \tsc{an}P is merged with \tsc{ind}. There is no lexical entry that matched the whole phrase. The specifier of \tsc{an}P, the \tsc{med}P, is moved to the specifier of \tsc{ind}P, and the \tsc{ind}P is spelled out as \tit{r}.
%
% The \tsc{ind}P is merged with \tit{wh}. There is no lexical entry that matches the whole phrase, there is no match after spec-to-spec movement, and there is no match after complement movement. Backtracking also does not lead to a matching lexical entry. A complex specifier is created.
%
% The \tsc{wh}P is merged with \tsc{rel}. There is no lexical entry that matches the whole phrase, there is no match after spec-to-spec movement, and there is no match after complement movement. The first step of backtracking is that the two branches, the \tsc{wh}P and the \tsc{ind}P are separated. The \tsc{rel} is merged with the \tsc{wh}P (that only contains the \tsc{wh}) in the left branch and with the \tsc{ind}P in the right branch. In the left branch, the whole phrase (\tsc{rel}P) is spelled out as \tit{w}. In the right branch, there is no spellout.
%
% The \tsc{rel}P is merged with \ac{f}1. There is no lexical entry that matches the whole phrase, there is no match after spec-to-spec movement, and there is no match after complement movement. The first step of backtracking is that the two branches, the \tsc{rel}P and the \tsc{ind}P are separated. The \ac{f}1 is merged with the \tsc{rel}P in the left branch and with the \tsc{ind}P in the right branch. In both branches, there is no lexical entry that matches the whole phrase. In the left branch, there is no specifier to move, so this movement is irrelevant. In the right branch, the specifier of \tsc{ind}P, the \tsc{med}P, is moved to the specifier of \tsc{nom}P, and the \tsc{nom}P is spelled out as \tit{r}.

The final result is given in \ref{ex:mg-spellout-rel-nom}.

\ex.\label{ex:mg-spellout-rel-nom}
\scriptsize{
\begin{forest} boompje
  [\tsc{rel}P, s sep=12mm
      [\tsc{rel}P,
      tikz={
      \node[label=below:\tit{w},
      draw,circle,
      scale=0.9,
      fit to=tree]{};
      }
          [\tsc{rel}]
          [\tsc{wh}P
              [\tsc{wh}]
              [\tsc{dx}\scsub{2}]
          ]
      ]
      [\tsc{med}P, s sep=21mm
          [\tsc{med}P,
          tikz={
          \node[label=below:\tit{e},
          draw,circle,
          scale=0.85,
          fit to=tree]{};
          }
              [\tsc{dx}\scsub{2}]
              [\tsc{prox}P
                  [\tsc{dx}\scsub{1}]
                  [\tsc{ind}P
                      [\tsc{ind}]
                      [\tsc{an}]
                  ]
              ]
          ]
          [\tsc{nom}P,
          tikz={
          \node[label=below:\tit{r},
          draw,circle,
          scale=0.95,
          fit to=tree]{};
          }
              [\ac{f}1]
              [\tsc{ind}P
                  [\tsc{ind}]
                  [\tsc{an}P
                      [\tsc{an}]
                      [\tsc{cl}P
                          [\tsc{cl}]
                          [ΣP
                              [Σ]
                              [\tsc{ref}]
                          ]
                      ]
                  ]
              ]
          ]
      ]
  ]
\end{forest}
}

\ex.\label{ex:mg-spellout-rel-acc}
\scriptsize{
\begin{forest} boompje
  [\tsc{rel}P, s sep=12mm
      [\tsc{rel}P,
      tikz={
      \node[label=below:\tit{w},
      draw,circle,
      scale=0.9,
      fit to=tree]{};
      }
          [\tsc{rel}]
          [\tsc{wh}P
              [\tsc{wh}]
              [\tsc{dx}\scsub{2}]
          ]
      ]
      [\tsc{med}P, s sep=23mm
          [\tsc{med}P,
          tikz={
          \node[label=below:\tit{e},
          draw,circle,
          scale=0.85,
          fit to=tree]{};
          }
              [\tsc{dx}\scsub{2}]
              [\tsc{prox}P
                  [\tsc{dx}\scsub{1}]
                  [\tsc{ind}P
                      [\tsc{ind}]
                      [\tsc{an}]
                  ]
              ]
          ]
          [\tsc{acc}P,
          tikz={
          \node[label=below:\tit{n},
          draw,circle,
          scale=0.95,
          fit to=tree]{};
          }
              [\ac{f}2]
              [\tsc{nom}P
                  [\ac{f}1]
                  [\tsc{ind}P
                      [\tsc{ind}]
                      [\tsc{an}P
                          [\tsc{an}]
                          [\tsc{cl}P
                              [\tsc{cl}]
                              [ΣP
                                  [Σ]
                                  [\tsc{ref}]
                              ]
                          ]
                      ]
                  ]
              ]
          ]
      ]
  ]
\end{forest}
}

% The accusative masculine singular relative pronoun is built as the nominative singular relative pronoun, except for that the feature \ac{f}2 is added to make it an accusative.
%
% The \tsc{nom}P is merged with \ac{f}2. There is no lexical entry that matches the whole phrase, there is no match after spec-to-spec movement, and there is no match after complement movement. The first step of backtracking is that the two branches, the \tsc{rel}P and the \tsc{nom}P are separated. The \ac{f}2 is merged with the \tsc{rel}P in the left branch and with the \tsc{nom}P in the right branch. In both branches, there is no lexical entry that matches the whole phrase. In the left branch, there is no specifier to move, so this movement is irrelevant. In the right branch, the specifier of \tsc{nom}P, the \tsc{med}P, is moved to the specifier of \tsc{acc}P, and the \tsc{acc}P is spelled out as \tit{n}.

To summarize, I showed independent evidence that the surface pronoun in Modern German is the relative pronoun. I decomposed the relative pronoun into the three morphemes \tit{w}, \tit{e} and the final consonant (\tit{r} and \tit{n}). I showed which features each of the morphemes spells out, and in which constituents the features are combined. It is these constituency that determine whether the relative pronoun can delete the light head or not.

\subsection{The (extra) light head}\label{sec:light-mg}

In Section \ref{sec:basic-idea}, I argued that headless relatives are derived from light-headed relatives. The relative pronoun can delete the light head when the relative contains all constituents of the light head. I suggested that this holds in Modern German, as long as the external case is not more complex than the internal case. In the previous section, I gave the internal structure of the relative pronoun, i.e. which constituents the relative pronoun consists of. In this section, I first need to identify the light head, as it does not surface in headless relatives. Then I show what its internal structure looks like: it is a constituent within the relative pronoun.

In this section, I consider two kinds of light-headed relatives as the source of the headless relative.
First, the light-headed relative is derived from an existing light-headed relative, and the deletion of the light head is optional. Second, the light-headed relative is derived from a light-headed relative that does not surfaces in Modern German, and the deletion of the light head is obligatory.
I consider the first option first, and I give two reasons against it. I take the light head from the existing light-headed relative as a point of departure, and I modify it in such a way that it is appropriate as a light head for a headless relative.

I give an example of a Modern German light-headed relative in \ref{ex:mg-den-wen}.\footnote{
Modern German also has another light-headed relative, in which the relative pronoun is the \tsc{d}-pronoun. I give an example in \ref{ex:mg-den-den}.

\exg. Jan umarmt den \tbf{den} \tbf{er} \tbf{mag}.\\
Jan hugs \tsc{d}.\tsc{m}.\tsc{sg}.\tsc{acc} \tsc{rel}.\tsc{m}.\tsc{sg}.\tsc{acc} he likes\\
`Jan hugs the man that he likes.'\label{ex:mg-den-den}

This relative pronoun generally appears in headed relatives, shown in \ref{ex:mg-den-headed}.

\exg. Jan umarmt den Mann \tbf{den} \tbf{er} \tbf{mag}.\\
Jan hugs \tsc{d}.\tsc{m}.\tsc{sg}.\tsc{acc} man \tsc{rel}.\tsc{m}.\tsc{sg}.\tsc{acc} he likes\\
`Jan hugs the man that he likes.'\label{ex:mg-den-headed}

I directly exclude the possibility that Modern German headless relatives are derived from these light-headed relatives, because they appear with the incorrect relative pronoun.
}

\exg. Jan umarmt den \tbf{wen} \tbf{er} \tbf{mag}.\\
Jan hugs \tsc{dem}.\tsc{m}.\tsc{sg}.\tsc{acc} \tsc{rel}.\tsc{an}.\tsc{acc} he likes\\
`Jan hugs the man that he likes.'\label{ex:mg-den-wen}

In \ref{ex:mg-den-wen}, the relative pronoun is the \tsc{wh}-pronoun \tit{wen} `\tsc{rel}.\tsc{an}.\tsc{acc}', and the light head is the \tsc{d}-pronoun \tit{den} `\tsc{dem}.\tsc{m}.\tsc{sg}.\tsc{acc}'. For easy reference, I call this light-headed relative the \tit{den}-\tit{wen} relative.

One hypothesis is that the demonstrative \tit{den} `\tsc{dem}.\tsc{m}.\tsc{sg}.\tsc{acc}' is deleted from the light-headed relative in \ref{ex:mg-den-wen} and that the headless relative in \ref{ex:mg-wen} remains.\footnote{
This is exactly what \citet{hanink2018} argues for. She claims that the feature content of the light head matches the feature content of the relative pronoun. Therefore, the light head is by default deleted. Only if the light head carries an extra focus feature it surfaces.
}
For easy reference, I call this headless relative the \tit{wen} relative.

\exg. Jan umarmt \tbf{wen} \tbf{er} \tbf{mag}.\\
Jan hugs \tsc{rel}.\tsc{an}.\tsc{acc} he likes\\
`Jan hugs who he likes.'\label{ex:mg-wen}

I give two arguments against this hypothesis. First, in headless relatives the morpheme \tit{auch immer} `ever' can appear, as shown in \ref{ex:mg-wh-for-headless-ever}.

\exg. Jan unarmt \tbf{wen} {\tbf{auch} \tbf{immer}} \tbf{er} \tbf{mag}.\\
Jan hugs \tsc{rel}.\tsc{an}.\tsc{acc} ever he likes\\
`Jan hugs whoever he likes.'\label{ex:mg-wh-for-headless-ever}

Light-headed relatives do not allow this morpheme to be inserted, illustrated in \ref{ex:mg-wh-for-headed-ever}.

\exg. *Jan unarmt den \tbf{wen} {\tbf{auch} \tbf{immer}} \tbf{er} \tbf{mag}.\\
Jan hugs \tsc{dem}.\tsc{m}.\tsc{sg}.\tsc{acc} \tsc{rel}.\tsc{an}.\tsc{acc} ever he likes\\
`Jan hugs him whoever he likes.'\label{ex:mg-wh-for-headed-ever}%source?

I assume that the headless relative is not derived from an ungrammatical structure.\footnote{
I am aware that such an analysis is common for sluicing.
}

The second argument against the \tit{den}-\tit{wen} relative being the source of the \tit{wen} relative comes from the interpretation differences between the two. Broadly speaking, the \tit{wen} relative has two interpretations (see \citealt{s̆imík2020} for a recent elaborate overview on the semantics of free relatives). The \tit{den}-\tit{wen} has only one of them. I show this schematically in Table \ref{tbl:mg-interpretations}.

\begin{table}[htbp]
  \center
  \caption{Intepretations of \tit{wen} and \tit{den}-\tit{wen} relatives}
\begin{tabular}{ccc}
  \toprule
                & \tit{wen} & \tit{den}-\tit{wen} \\
                \cmidrule{2-3}
definite-like   & ✔         & ✔                   \\
universal-like  & ✔         & *                   \\
\bottomrule
\end{tabular}
\label{tbl:mg-interpretations}
\end{table}

The first interpretation of the \tit{wen} relative is a definite-like one. This interpretation corresponds to a definite description: Jan hugs the person that he likes. The interpretation is available for the \tit{wen} relative and for the \tit{den}-\tit{wen} relative.
The second interpretation of the \tit{wen} relative is a universal-like one. This interpretation corresponds to a universal quantifier: Jan hugs everybody that he likes. This interpretation is available for the \tit{wen} relative, but not for the \tit{den}-\tit{wen} relative.

There are some indications that the universal-like interpretation of headless relatives is the main interpretation that should be accounted for.
First, informants have reported to me that headless relatives with case mismatches become more acceptable in the universal-like interpretation compared to the definite-like interpretation.
Second, \pgcitet{s̆imík2020}{4} notes that some languages do not easily allow for the definite-like interpretation of headless relatives with an \tit{ever}-morpheme. There is no language documented that does not allow for the universal-like interpretation, but does allow the definite-like interpretation.

In sum, there are two arguments against the \tit{den}-\tit{wen} relative being the source of the \tit{wen} relative. In what follows, I show how the presence of \tit{den} leads to having only the definite-like interpretation. I suggest that the problem lies in the feature content of the light head \tit{den}. I point out how the feature content should be modified such that it is a suitable light head.

The light head in the \tit{den}-\tit{wen} relative is a demonstrative. A demonstrative refers back to a linguistic or extra-linguistic antecedent. Consider the context which facilitates a definite-interpretation and the repeated \tit{den}-\tit{wen} relative in \ref{ex:mg-context-def}.

\ex.
\a. Context: Yesterday Jan met with two friends. He likes one of them. The other one he does not like so much.\label{ex:mg-context-def}
\bg. Jan umarmt den \tbf{wen} \tbf{er} \tbf{mag}.\\
Jan hugs \tsc{dem}.\tsc{m}.\tsc{sg}.\tsc{acc} \tsc{rel}.\tsc{an}.\tsc{acc} he likes\\
`Jan hugs the man that he likes.'

The demonstrative \tit{den} in the \tit{den}-\tit{wen} relative refers back to the friend of Jan that he likes.

Consider the context which facilitates a universal-interpretation and the repeated \tit{den}-\tit{wen} relative in \ref{ex:mg-context-univ}.

\ex.
\a. Jan has a general habit of hugging everybody that he likes.\label{ex:mg-context-univ}
\bg. \#Jan umarmt den \tbf{wen} \tbf{er} \tbf{mag}.\\
Jan hugs \tsc{dem}.\tsc{m}.\tsc{sg}.\tsc{acc} \tsc{rel}.\tsc{an}.\tsc{acc} he likes\\
`Jan hugs the man that he likes.'

In this case, there is no antecedent for the demonstrative \tit{den} to refer back to.

% In footnote \ref{ftn:das-was}, I already briefly noted that not all speakers of Modern German accept \tit{den}-\tit{wen} relatives. Some of my informants reported that the meaning of the light head \tit{den} and relative pronoun \tit{wen} are incompatible for them. They prefer the \tsc{d}-pronoun as relative pronoun instead (as in \ref{ex:mg-den-den}).
% That is why the relative pronoun \tit{den}, which also contains definiteness features, is a good match with a \tsc{d}-pronoun as a light head.

I zoom in on the internal structure of the demonstrative \tit{den} to investigate what it is about the demonstrative that forces the definite-like interpretation. The demonstrative consists of the three morphemes \tit{d}, \tit{e} and \tit{n}. Two of its morphemes are identical to the \tsc{wh}-relative pronoun: (1) \tit{n}, which spells out pronominal, number, gender and case features, and (2) the \tit{e} which spells out deictic features. One morphemes differs: the \tit{d}, which establishes a definite reference. The two morphemes that force the definite-interpretation are the \tit{d} and the \tit{e}. The \tit{e} establishes a reference, and the \tit{d} makes this reference a definite one.

I propose that the light head is the element that is left once the morphemes \tit{d} and \tit{e} are abandoned. This is the morpheme that is the final consonant of the relative pronoun.\footnote{
The two light heads I discuss resemble the strong and weak definite in \citet{schwarz2009}, at least morphologically (although my light head is always obligatorily deleted). \posscitet{schwarz2009} strong definite is anaphoric in nature, and the weak definite encodes uniqueness. I give an example of a strong definite in \ref{ex:mg-florian-strong}. The strong definite is \tit{dem} that precedes \tit{Freund} `friend'. It refers back to the linguistic antecedent \tit{einen Freund} `a friend'.

\exg. Hans hat heute einen Freund zum Essen mit nach Hause gebracht. Er hat uns vorher ein Foto von dem Freund gezeigt.\\
Hans has today a friend {to the} dinner with to home brought he has us beforehand a photo of the\scsub{strong} friend shown\\
`Hans brought a friend home for dinner today. He had shown us a photo of the friend beforehand.'\label{ex:mg-florian-strong}

Weak definites are used when situational uniqueness is involved. This uniqueness can be global or within a restricted domain. I give two examples in \ref{ex:mg-florian-weak}. In \ref{ex:mg-florian-weak-hund}, the dog is unique in this specific situation of the break-in. In \ref{ex:mg-florian-weak-mond}, the moon is unique for us people on the planet.

\ex.\label{ex:mg-florian-weak}
\ag. Der Einbrecher ist {zum Glück} vom Hund verjagt worden.\\
the burglar is luckily {by the\scsub{weak}} dog {chased away} been\\
`Luckily, the burglar was chased away by the dog.'\label{ex:mg-florian-weak-hund}
\bg. Armstrong flog als erster zum Mond.\\
Armstrong flew as {first one} {to the\scsub{weak}} moon\\
`Armstrong was the first one to fly to the moon.' \flushfill{Modern German, \pgcitealt{schwarz2009}{40}}\label{ex:mg-florian-weak-mond}

The meaning of \posscitet{schwarz2009} strong definite seems similar to the meaning of the light head in the \tit{den}-\tit{wen} relative.
I do not see right away how the light head in headless relatives could encode uniqueness. One possibility is that the feature content of his and my form differs slightly after all. Another possibility is that the fact that his form combines with a preposition and an overt nouns leads to a change in interpretation.
}
I give the light-headed relative from which the \tit{wen}-relative is derived in \ref{ex:mg-real-base}. The brackets around the light head indicate that it is obligatorily deleted.

\exg. Jan umarmt [n] \tbf{wen} \tbf{er} \tbf{mag}.\\
Jan hugs \tsc{lh}.\tsc{an}.\tsc{acc} \tsc{rel}.\tsc{an}.\tsc{acc} he likes\\
`Jan hugs who he likes.'\label{ex:mg-real-base}

In Section \ref{sec:basic-idea}, I gave the simplified structure of the light head, repeated here in \ref{ex:light-head-rep}.

\ex.
\begin{forest} boom
[\tsc{k}P
    [ϕP
        [\phantom{x}ϕ\phantom{x}, roof]
    ]
    [\tsc{k}P,
        [\tsc{k}]
    ]
]
\end{forest}
\label{ex:light-head-rep}

The idea was that the structures of the relative pronoun and the light heads match, but that the relative pronoun contains at least one feature more. I just argued that the light head has four feature less: \tsc{wh}, \tsc{rel}, \tsc{dx}\scsub{1} and \tsc{dx}\scsub{2}.

I discuss two light heads: the animate nominative singular and in the animate accusative singular. These are the two forms that I compare the constituents of in Section \ref{sec:comparing-mg}. I show them in \ref{ex:mg-lhs}.

\ex.\label{ex:mg-lhs}
\ag. r\\
 `\tsc{lh}.\tsc{an}.\tsc{sg}.\tsc{nom}'\\
\bg. n\\
 `\tsc{lh}.\tsc{an}.\tsc{sg}.\tsc{acc}'\\

The derivations of the light heads are simple ones. The features are merged one by one, and after each new phrase is created, it is spelled out as a whole.

I give the structures of the animate nominative singular light head in \ref{ex:fseq-wh-lh-nom}.

\ex. \begin{forest} boom
    [\tsc{nom}P,
    tikz={
    \node[label=below:\tit{r},
    draw,circle,
    scale=0.95,
    fit to=tree]{};
    }
        [\ac{f}1]
        [\tsc{ind}P
            [\tsc{ind}]
            [\tsc{an}P
                [\tsc{an}]
                [\tsc{cl}P
                    [\tsc{cl}]
                    [ΣP
                        [Σ]
                        [\tsc{ref}]
                    ]
                ]
            ]
        ]
    ]
\end{forest}
\label{ex:fseq-wh-lh-nom}

I give the structures of the animate accusative singular light head in \ref{ex:fseq-wh-lh-acc}.

\ex. \begin{forest} boom
[\tsc{acc}P,
tikz={
\node[label=below:\tit{n},
draw,circle,
scale=1,
fit to=tree]{};
}
    [\ac{f}2]
    [\tsc{nom}P
        [\ac{f}1]
        [\tsc{ind}P
            [\tsc{ind}]
            [\tsc{an}P
                [\tsc{an}]
                [\tsc{cl}P
                    [\tsc{cl}]
                    [ΣP
                        [Σ]
                        [\tsc{ref}]
                    ]
                ]
            ]
        ]
    ]
]
\end{forest}
\label{ex:fseq-wh-lh-acc}



% At first sight it seems like \citet{fuss2014} discuss a exception to this claim, namely headless relatives with \tsc{d}-pronouns. However, they claim that these headless relatives are actually light-headed relatives in which one of two syncretic elements is deleted by haplology.


\subsection{Comparing constituents}\label{sec:comparing-mg}

Consider the example in \ref{ex:mg-nom-nom-rep}, in which the internal nominative case competes against the external nominative case. The relative clause is marked in bold, and the extra light head and the relative pronoun are underlined.
The internal case is nominative, as the predicate \tit{mögen} `to like' takes nominative subjects. The relative pronoun \tit{wer} `\ac{rel}.\ac{an}.\ac{nom}' appears in the nominative case. This is the element that surfaces.
The external case is nominative as well, as the predicate \tit{besuchen} `to visit' also takes nominative subjects. The extra light head \tit{ər} `\ac{dem}.\ac{an}.\ac{nom}' appears in the nominative case. It is placed between square brackets because it does not surface.

\exg. Uns besucht [r], \tbf{wer} \tbf{Maria} \tbf{mag}.\\
 2\ac{pl}.\ac{acc} visit.\ac{pres}.3\ac{sg}\scsub{[nom]} \tsc{elh}.\ac{an}.\ac{nom} \ac{rel}.\ac{an}.\ac{nom} Maria.\ac{acc} like.\ac{pres}.3\ac{sg}\scsub{[nom]}\\
 `Who visits us likes Maria.' \flushfill{Modern German, adapted from \pgcitealt{vogel2001}{343}}\label{ex:mg-nom-nom-rep}

In Figure \ref{fig:mg-int=ext}, I give the syntactic structure of the extra light head at the top and the syntactic structure of the relative pronoun at the bottom.

\begin{figure}[htbp]
  \center
  \begin{tabular}[b]{c}
        \toprule
        \tsc{nom} extra light head \tit{r}\\
        \cmidrule{1-1}
      \scriptsize{
      \begin{forest} boompje
        [\tsc{nom}P,
        tikz={
        \node[label=below:{\tit{r}},
        draw,circle,
        scale=0.95,
        fit to=tree]{};
        \node[draw,circle,
        dashed,
        scale=1,
        fill=DG,fill opacity=0.2,
        fit to=tree]{};
        }
            [\ac{f}1]
            [\tsc{ind}P
                [\tsc{ind}]
                [\tsc{an}P
                    [\tsc{an}]
                    [\tsc{cl}P
                        [\tsc{cl}]
                        [ΣP
                            [Σ]
                            [\tsc{ref}]
                        ]
                    ]
                ]
            ]
        ]
      \end{forest}
      }
      \\
      \toprule
      \tsc{nom} relative pronoun \tit{w-e-r}
      \\
      \cmidrule{1-1}
      \scriptsize{
          \begin{forest} boompje
          [\tsc{rel}P
              [\tsc{rel}P
                  [\phantom{x}\tit{w}\phantom{x}, roof]
              ]
              [\tsc{nom}P, s sep=15mm
                  [\tsc{med}P,
                      [\phantom{x}\tit{e}\phantom{x}, roof]
                  ]
                  [\tsc{nom}P,
                  tikz={
                  \node[label=below:{\tit{r}},
                  draw,circle,
                  scale=0.95,
                  fit to=tree]{};
                  \node[draw,circle,
                  dashed,
                  scale=1,
                  fit to=tree]{};
                  }
                      [\ac{f}1]
                      [\tsc{ind}P
                          [\tsc{ind}]
                          [\tsc{an}P
                              [\tsc{an}]
                              [\tsc{cl}P
                                  [\tsc{cl}]
                                  [ΣP
                                      [Σ]
                                      [\tsc{ref}]
                                  ]
                              ]
                          ]
                      ]
                  ]
              ]
          ]
        \end{forest}
        }
        \\
      \bottomrule
  \end{tabular}
  \caption {Modern German \tsc{ext}\scsub{nom} vs. \tsc{int}\scsub{nom} → \tit{wer}}
  \label{fig:mg-int=ext}
\end{figure}

The relative pronoun consists of three morphemes: \tit{w}, \tit{e} and \tit{r}.
The extra light head consists of two morphemes: \tit{ə} and \tit{r}.
As usual, I circle the part of the structure that corresponds to a particular lexical entry, and I place the corresponding phonology under it.
I draw a dashed circle around each constituent that is a constituent in both the extra light head and the relative pronoun.
As each constituent of the extra light head is also a constituent within the relative pronoun, the extra light head can be absent. I illustrate this by marking the content of the dashed circles for the extra light head gray.

I explain this constituent by constituent.
I start with the right-most constituent of the extra light head that spells out as \tit{r} (\tsc{nom}P). This constituent is also a constituent in the relative pronoun.
I continue with the left-most constituent of the extra light head that spells out as \tit{ə} (\tsc{prox}P). This constituent is also a constituent in the relative pronoun, contained in \tsc{med}P.
Both constituent of the extra light head are also a constituent within the relative pronoun, and the extra light head can be absent.

Consider the example in \ref{ex:mg-nom-acc-rep}, in which the internal accusative case competes against the external nominative case. The relative clause is marked in bold, and the extra light head and the relative pronoun are underlined.
The internal case is accusative, as the predicate \tit{mögen} `to like' takes accusative objects. The relative pronoun \tit{wen} `\ac{rel}.\ac{an}.\ac{acc}' appears in the accusative case. This is the element that surfaces.
The external case is nominative, as the predicate \tit{besuchen} `to visit' takes nominative subjects. The extra light head \tit{ər} `\ac{dem}.\ac{an}.\ac{nom}' appears in the nominative case. It is placed between square brackets because it does not surface.

\exg. Uns besucht [r] \tbf{wen} \tbf{Maria} \tbf{mag}.\\
 we.\ac{acc} visit.3\ac{sg}\scsub{[nom]} \tsc{elh}.\ac{nom}.\tsc{an} \tsc{rel}.\ac{acc}.\tsc{an} Maria.\ac{nom} like.3\ac{sg}\scsub{[acc]}\\
 `Who visits us, Maria likes.' \flushfill{adapted from \pgcitealt{vogel2001}{343}}\label{ex:mg-nom-acc-rep}

In Figure \ref{fig:mg-int-wins}, I give the syntactic structure of the extra light head at the top and the syntactic structure of the relative pronoun at the bottom.

\begin{figure}[htbp]
  \center
  \begin{tabular}[b]{c}
      \toprule
      \tsc{nom} extra light head \tit{r}
      \\
      \cmidrule{1-1}
      \scriptsize{
      \begin{forest} boompje
        [\tsc{nom}P,
        tikz={
        \node[label=below:{\tit{r}},
        draw,circle,
        scale=0.95,
        fit to=tree]{};
        \node[draw,circle,
        dashed,
        scale=1,
        fill=DG,fill opacity=0.2,
        fit to=tree]{};
        }
            [\ac{f}1]
            [\tsc{ind}P
                [\tsc{ind}]
                [\tsc{an}P
                    [\tsc{an}]
                    [\tsc{cl}P
                        [\tsc{cl}]
                        [ΣP
                            [Σ]
                            [\tsc{ref}]
                        ]
                    ]
                ]
            ]
        ]
      \end{forest}
      }
      \\
      \toprule
      \tsc{acc} relative pronoun \tit{w-e-n}
      \\
      \cmidrule{1-1}
      \scriptsize{
          \begin{forest} boompje
            [\tsc{rel}P
                [\tsc{rel}P
                    [\phantom{x}\tit{w}\phantom{x}, roof]
                ]
                [\tsc{nom}P, s sep=15mm
                    [\tsc{med}P,
                        [\phantom{x}\tit{e}\phantom{x}, roof]
                    ]
                    [\tsc{acc}P,
                    tikz={
                    \node[label=below:{\tit{n}},
                    draw,circle,
                    scale=0.95,
                    fit to=tree]{};
                    }
                        [\ac{f}2]
                        [\tsc{nom}P, tikz={
                        \node[draw,circle,
                        dashed,
                        scale=0.9,
                        fit to=tree]{};
                        }
                            [\ac{f}1]
                            [\tsc{ind}P
                                [\tsc{ind}]
                                [\tsc{an}P
                                    [\tsc{an}]
                                    [\tsc{cl}P
                                        [\tsc{cl}]
                                        [ΣP
                                            [Σ]
                                            [\tsc{ref}]
                                        ]
                                    ]
                                ]
                            ]
                        ]
                    ]
                ]
            ]
        \end{forest}
        }
        \\
      \bottomrule
  \end{tabular}
   \caption {Modern German \tsc{ext}\scsub{nom} vs. \tsc{int}\scsub{acc} → \tit{wen}}
  \label{fig:mg-int-wins}
\end{figure}

The relative pronoun consists of three morphemes: \tit{w}, \tit{e} and \tit{n}.
The extra light head consists of one morpheme: \tit{r}.
Again, I circle the part of the structure that corresponds to a particular lexical entry, and I place the corresponding phonology under it.
I draw a dashed circle around each constituent that is a constituent in both the extra light head and the relative pronoun.
As each constituent of the extra light head is also a constituent within the relative pronoun, the extra light head can be absent. I illustrate this by marking the content of the dashed circles for the extra light head gray.

I explain this constituent by constituent.
I start with the right-most constituent of the extra light head that spells out as \tit{r} (\tsc{nom}P). This constituent is also a constituent in the relative pronoun, contained in \tsc{acc}P.
I continue with the left-most constituent of the extra light head that spells out as \tit{ə} (\tsc{prox}P). This constituent is also a constituent in the relative pronoun, contained in \tsc{med}P.
Both constituent of the extra light head are also a constituent within the relative pronoun, and the extra light head can be absent.

Consider the examples in \ref{ex:mg-acc-nom-rep}, in which the internal nominative case competes against the external accusative case. The relative clauses are marked in bold, and the extra light heads and the relative pronouns are underlined. It is not possible to make a grammatical headless relative in this situation.
The internal case is nominative, as the predicate \tit{sein} `to be' takes nominative subjects. The relative pronoun \tit{wer} `\ac{rel}.\ac{an}.\ac{nom}' appears in the nominative case.
The external case is accusative, as the predicate \tit{einladen} `to invite' takes accusative objects. The extra light head \tit{ən} `\ac{dem}.\ac{an}.\ac{acc}' appears in the accusative case.
\ref{ex:mg-acc-nom-rep-rel} is the variant of the sentence in which the extra light head is absent (indicated by the square brackets) and the relative pronoun surfaces, and it is ungrammatical.
\ref{ex:mg-acc-nom-rep-lh} is the variant of the sentence in which the relative pronoun is absent (indicated by the square brackets) and the extra light head surfaces, and it is ungrammatical too.

\ex.\label{ex:mg-acc-nom-rep}
\ag. *Ich {lade ein}, [n] \tbf{wer} \tbf{mir} \tbf{sympathisch} \tbf{ist}.\\
1\ac{sg}.\ac{nom} invite.\ac{pres}.1\ac{sg}\scsub{[acc]} \ac{rel}.\ac{an}.\ac{nom} 1\ac{sg}.\ac{dat} nice be.\ac{pres}.3\ac{sg}\scsub{[nom]}\\
`I invite who I like.' \flushfill{Modern German, adapted from \pgcitealt{vogel2001}{344}}\label{ex:mg-acc-nom-rep-rel}
\bg. *Ich {lade ein}, n [\tbf{wer}] \tbf{mir} \tbf{sympathisch} \tbf{ist}.\\
1\ac{sg}.\ac{nom} invite.\ac{pres}.1\ac{sg}\scsub{[acc]} \ac{rel}.\ac{an}.\ac{nom} 1\ac{sg}.\ac{dat} nice be.\ac{pres}.3\ac{sg}\scsub{[nom]}\\
`I invite who I like.' \flushfill{Modern German, adapted from \pgcitealt{vogel2001}{344}}\label{ex:mg-acc-nom-rep-lh}

In Figure \ref{fig:mg-ext-wins}, I give the syntactic structure of the extra light head at the top and the syntactic structure of the relative pronoun at the bottom.

\begin{figure}[htbp]
  \center
  \begin{tabular}[b]{c}
        \toprule
        \tsc{acc} extra light head \tit{n} \\
        \cmidrule{1-1}
      \scriptsize{
      \begin{forest} boompje
        [\tsc{acc}P,
        tikz={
        \node[label=below:{\tit{n}},
        draw,circle,
        scale=0.95,
        fit to=tree]{};
        }
            [\ac{f}2]
            [\tsc{nom}P, tikz={
            \node[draw,circle,
            dashed,
            scale=0.9,
            fit to=tree]{};
            }
                [\ac{f}1]
                [\tsc{ind}P
                    [\tsc{ind}]
                    [\tsc{an}P
                        [\tsc{an}]
                        [\tsc{cl}P
                            [\tsc{cl}]
                            [ΣP
                                [Σ]
                                [\tsc{ref}]
                            ]
                        ]
                    ]
                ]
            ]
        ]
      \end{forest}
      }
      \\
      \toprule
      \tsc{nom} relative pronoun \tit{w-e-r}
      \\
      \cmidrule{1-1}
      \scriptsize{
      \begin{forest} boompje
        [\tsc{rel}P
            [\tsc{rel}P
                [\phantom{x}\tit{w}\phantom{x}, roof]
            ]
            [\tsc{nom}P, s sep=15mm
                [\tsc{med}P,
                    [\phantom{x}\tit{e}\phantom{x}, roof]
                ]
                [\tsc{nom}P,
                tikz={
                \node[label=below:{\tit{r}},
                draw,circle,
                scale=0.95,
                fit to=tree]{};
                \node[draw,circle,
                dashed,
                scale=1,
                fit to=tree]{};
                }
                    [\ac{f}1]
                    [\tsc{ind}P
                        [\tsc{ind}]
                        [\tsc{an}P
                            [\tsc{an}]
                            [\tsc{cl}P
                                [\tsc{cl}]
                                [ΣP
                                    [Σ]
                                    [\tsc{ref}]
                                ]
                            ]
                        ]
                    ]
                ]
            ]
        ]
    \end{forest}
        }
      \\
      \bottomrule
  \end{tabular}
   \caption {Modern German \tsc{ext}\scsub{acc} vs. \tsc{int}\scsub{nom} ↛ \tit{wer}/\tit{n}}
  \label{fig:mg-ext-wins}
\end{figure}

The relative pronoun consists of three morphemes: \tit{w}, \tit{e} and \tit{r}.
The extra light head consists of two morphemes: \tit{ə} and \tit{n}.
Again, I circle the part of the structure that corresponds to a particular lexical entry, and I place the corresponding phonology under it.
I draw a dashed circle around each constituent that is a constituent in both the extra light head and the relative pronoun.
Neither of the elements contains all constituents that the other element contains. The relative pronoun does not contain all constituents that the extra light head contains, and the extra light head does not contain all constituents that the relative pronoun contains. As a result, none of the elements can be absent.\footnote{
Why do we not see this result surface? Very good question.
}

I explain this constituent by constituent.
I start by showing that the extra light head cannot be absent.
Consider the right-most constituent of the extra light head that spells out as \tit{n} (\tsc{acc}P). This constituent is not a constituent in the relative pronoun: the relative pronoun has a constituent \tsc{nom}P, but it does not contain \ac{f}2 to make it an \tsc{acc}P.
The extra light head has a constituent that is not a constituent in the relative pronoun, so the extra light head cannot be absent.

The relative pronoun can also not be absent.
Consider the middle constituent of the relative pronoun that spells out as \tit{e} (\tsc{med}P). This constituent is not a constituent in the extra light head: the extra light head has a constituent \tsc{med}P, but it does not contain \tsc{dx}\scsub{3} to make it an \tsc{med}P.
The same hold for the left-most constituent of the relative pronoun that spells out as \tit{w} (\tsc{rel}P). The extra light head lacks the features \tsc{wh} and \tsc{rel} that form the \tsc{rel}P.
The relative pronoun has constituents that are not constituents in the extra light head, so the relative pronoun cannot be absent.
In sum, neither of the elements contains all constituents that the other element contains, and none of the elements can be absent, so none of them is marked gray.
