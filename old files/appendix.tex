% !TEX root = thesis.tex

\appendix

\begin{table}[h]
	\begin{tabular}{|l|l|l|l|l|l|l|l|}
		\hline
		& gothic & greek & german & polish & romanian & icelandic & finnish													\\ \hline
		combine with NP            &        & yes   & no     &        &          &           & 			\\ \hline
		combine with D             &        & no    & no     &        &          &           & 			\\ \hline
		formal relation indefinite & no     & no    & yes    & yes    &          & no        & yes 	\\ \hline
		number distinctions        & yes    & yes   & no     & no     & yes      & yes       & 			\\ \hline
		visible d-element          & yes    & yes   & no     & no     &          & yes       & no 	\\ \hline
		visible wh-element         & no     & yes   & yes    & yes    &          & no        & yes 	\\ \hline
		visible complementizer     & yes   & no    & no     & no     & no       & yes       & no 	\\ \hline
	\end{tabular}
\end{table}

To keep in mind

\begin{itemize}
	\item Languages like Polish do not allow for case mismatches in free relatives, but they do allow for matching ones. What is the mechanism behind that? And syncretism!
	\item D-pronouns can inflect for number, wh-pronouns cannot. So there is number attraction.
\end{itemize}
