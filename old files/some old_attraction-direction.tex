% !TEX root = thesis.tex

\chapter{Direction of attraction}

--here is going to be an introduction to this section, something like this:

The case complexity story holds for all languages. The direction of the attraction does not. In this chapter I discuss three different patterns that appear in languages. I show that the direction of attraction correlates with the type of pronoun that is used as free relative pronoun. Inverse attraction = wh-pronoun, proper attraction = demonstrative, both attraction types = demonstrative + complementizer.--

I illustrated the case complexity story in the previous chapter mostly with Modern German. A relative pronoun can license the case requirement of a predicate if the case is contained in the case it expresses.

In this chapter I show there is an additional restriction on whether the relative pronoun can license a less complex case. This has to do with whether the case requirement comes from internal to the relative clause or from external to the relative clause. For Modern German it holds that the case requirement of the internal predicate needs to be more complex than the case requirement of the external predicate. In the literature this is referred to as \tit{inverse attraction} (lots of references). An example is given in \ref{ex:invattmg}. The internal predicate \tit{vertraut} `trusts' requires its object to be in dative, and the external predicate \tit{lade ein} `invite' requires its object to be in accusative. The internal dative case requirement is more complex than the external accusative one.
The free relative pronoun appears in the internal dative case as \tit{wem} `who.\tsc{dat}'.

\exg. Ich {lade ein}, wem Maria vertraut. \\
I invite\scsub{acc} who.\tsc{dat} also Maria trusts\scsub{dat}\\
`I invite whoever Maria also trusts.' \hfill \label{ex:invattmg}

In Modern German, examples become ungrammatical when the complexity is reversed, i.e. when the case requirement of the external predicate is more complex than the one of the interal predicate. This is called \tit{proper attraction} in the literature (lots of examples). An example of that is shown in \ref{ex:*attmg}. The external predicate \tit{vertraut} `trusts' requires its object to be in dative, and the internal predicate \tit{mag} `likes' requires its object to be in accusative. The external dative case requirement is more complex than the internal accusative one. The free relative pronoun appears in the internal dative case as \tit{wem} `who.\tsc{dat}'.\footnote{Also changing the relative pronoun to the accusative case required by the internal predicate does not render a grammatical result. This can be understood from Chapter X: only more complex cases can license less complex cases, and dative is more complex than accusative, so an accusative relative pronoun cannot license a dative case requirement.}

\exg. *Ich vertraue, wem Maria mag. \\
I trust\scsub{dat} who.\tsc{dat} also Maria likes\scsub{acc}\\
`I trust whoever Maria also likes.' \hfill Vogel \label{ex:*attmg}

Modern German can be schematically represented as follows in Table \ref{tbl:intextmg}. Mismatching headless relatives are grammatical if the internal case is more complex than the external case, and they are ungrammatical if they external case is more complex than the internal case. In other words, Moden German allows for inverse attraction but it does not allow for proper attraction.

\begin{table}[h]\label{tbl:intextmg}
	\center
	\caption {\tsc{int} vs. \tsc{ext} in Modern German}
	\begin{minipage}{\linewidth}
		\begin{tabularx}{\textwidth}{c *{2}{Y}}
		\toprule
		 							& \tsc{int>ext} 			& \tsc{ext>int}				\\
									& inverse attraction	& proper attraction		\\
		\midrule
		Modern German & ✔						  			& *										\\
		\bottomrule
		\end{tabularx}
	\end{minipage}
\end{table}

Old High German is the exact opposite of Modern German: what is grammatical in Old High German is not in Modern German and the other way around. Old High German allows for proper attraction, i.e. mismatching headless relatives are grammatical if the external case is more complex than the internal one. This is exemplified by the example in \ref{ex:attohg}. The external predicate \tit{antuurta} `replied' requires its object to be in dative, and the internal predicate \tit{sprah} `spoke' requires a nominative subject. This configuration is similar to the Modern German one in \ref{ex:*attmg}: the external dative case requirement is more complex than the internal nominative one. However, the example in Old High German is grammatical and it is not in Modern German.

\exg. Aer antuurta demo zaimo sprah.\\
he replied\scsub{dat} who.\tsc{dat} {to him} spoke\scsub{nom}\\
`He replied to the one who spoke to him.' \hfill Lenerz \label{ex:attohg}

Second, inverse attraction is not permitted in Old High German, i.e. mismatching headless relatives are not grammatical if the internal case is more complex than the external one. This claim can only be motivated by the lack of instances in historical texts, since there are no longer speakers of Old High German.

Table \ref{tbl:intextohg} shows schematically that Modern German and Old High German are each others mirror image.

\begin{table}[h]
	\center
	\caption {\tsc{int} vs. \tsc{ext} in Modern and Old High German}
	\begin{minipage}{\linewidth}
		\begin{tabularx}{\textwidth}{c *{2}{Y}}
		\toprule
		 								& \tsc{int>ext}				& \tsc{ext>int}				\\
										& inverse attraction	& proper attraction		\\
		\midrule
		Modern German 	& ✔			 							&	*										\\
		Old High German	& *										&	✔										\\
		\bottomrule
		\end{tabularx}
	\end{minipage}
\end{table}\label{tbl:intextohg}

Gothic allows for both types of attraction. \ref{ex:attgot} gives is a proper attraction example with the internal case being more complex than the external case. The external predicate \tit{taujau} `do' requires its object to be in dative, and the internal predicate \tit{qiþiþ} `say' requires its object to be in accusative. Gothic it like Old High German here and unlike Modern German.

\exg. hva nu wileiþ ei taujau þamm-ei qiþiþ þiudan Iudaie?\\
what now want that do\scsub{dat} \tsc{dat}-\tsc{comp} say\scsub{acc} king {of Jews}\\
`What now do you wish that I do to him whom you call King of the Jews?' \hfill \citep[339,434]{harbert1978}\label{ex:attgot}

In \ref{ex:invattgot}, I give an example of inverse attraction in which the external case is more complex than the internal case. Just like in Modern German, and unlike in Old High German, this type of mismatch is allowed in Gothic. The internal predicate \tit{lag} `lay' requires its object to be in dative, and the internal predicate \tit{ushafjands} `picking up' requires its object to be in accusative.

\exg. ushafjands ana þamm-ei lag\\
{picking up}\scsub{acc} on \tsc{dat}-\tsc{comp} lay\scsub{dat}\\
`picking up that on which he lay' \hfill \citep[339,434]{harbert1978}\label{ex:invattgot}

In Table \ref{tbl:intextgoth}, Gothic is added to the schematic representation.



The fourth logical option that is missing from this table, namely a language that does not allow mismatching cases at all. --am I going to discuss that here or not? the languages that are like this have the following profile: lightheaded relatives are = wh + d.--


Ok, so languages differ in which direction of attraction they allow for. Is this variation parametric per language or can we trace it back to other properties within the language? In this chapter I show that the property variation can be traced back to it the form of the relative pronoun in the headless relative. In order to show that I first need to discuss the underlying syntactic structure of a relative clause. I start by looking at headed relatives that exhibit proper attraction, and I that proper attraction in headless relatives can readily be derived from proper attraction in headed relatives. Then I show that inverse attraction in headless relatives differs strongly from inverse attraction in headed relatives, and the former cannot be derived from the latter (Section. I end up..


\section{Proper attraction}\label{sec:propatt}

\subsection{Proper attraction in headed relative clauses}\label{sec:attheaded}

Normal situation without attraction: head of the relative clause takes external case, relative pronoun takes internal case.

\exg. Ich sehe den Mann, der einen lustigen Hut trägt.\\
I see\scsub{acc} the.\tsc{acc} man, that.\tsc{nom} a.\tsc{acc} funny hat wears\scsub{nom}\\
`I see the man that is wearing a funny hat.'

The example in \ref{ex:aheaded} shows an example of attraction with a headed relative clause in Old High German. The predicate in main clause \tit{gedâht} `thought' combines with genitive objects. The predicate in the embedded clause \tit{geschach} `happened' combines with nominative subjects. As expected, the head of the relative clause takes the external case, the genitive\footnote{External case = the case required by the predicate that is external to the relative clause; internal case = the case required by the predicate that is internal to the relative clause}. Unexpectedly, the relative pronoun does not take the internal nominative case. Instead, it takes the external genitive case, just like head of the relative clause. The relative pronoun is attracted to the head of the relative clause and takes on its case.

\exg. sie gedâht' ouch maniger leide, der ir dâ héimé geschach.\\
she thought\scsub{gen} also some.\tsc{gen} sufferings.\tsc{gen} which.\tsc{gen} her at home happened\scsub{nom}\\
`She thought about some misfortunes that happened to her at home' \label{ex:aheaded}\hfill attraction

A few things should be noted here. First, proper (but also inverse) attraction is always optional (references). Proper attraction never occurs when the head of the relative clause and the relative pronoun are split (show Bianchi with interfering elements). Attracted elements are `low in agentivity' and `high in discourse prominence' (Scott Grimm, see what I can do with that later). Specific for proper attraction, a prototypical sentence is `main clause - head of relative clause - relative pronoun - embedded clause'. Cases that are involved are \tsc{acc}, \tsc{gen} and \tsc{dat}, and they override each other from left to right.

In what follows I will work towards the following structure.

\ex. tree of \ref{ex:aheaded}\\
	\begin{forest} boom
	[PP
			[P
					[about, roof]
			]
			[DP
					[D
							[some-\tsc{gen}, roof]
					]
					[YP
							[CP
									[DP
											[NP
													[sufferings-\tsc{gen}, roof]
											]
											[DP
													[DP
															[which-\tsc{gen}, roof]
													]
													[t$_{NP}$ ]
											]
									]
									[IP
											[happened to her at home, roof]
									]
							]
							[Y
									[\sout{sufferings}, roof]
							]
					]
			]
	]
	\end{forest}

Why this structure? First, \tit{sufferings} and \tit{which} cannot be split.\footnote{A question that remains open is whether \tit{some} and \tit{sufferings} can be split. --look into the Bianchi paper--} Second, it does not come as a surprise that attraction occurs: it is not only \tit{which} but also \tit{sufferings} that is situated within the relative clause.

--But more general why this double-headed Cinque structure? Maybe I will be able to get away with just the raising story, so then there is no need to go into more detail. I'll leave this for now and concentrate on the details--

In what comes next I go through the derivation of \ref{ex:aheaded}. I start from a double headed Cinque structure. The head of the relative clause \tit{suffering} is in both clauses.

\ex.
\begin{forest} boom
	[YP
			[CP
					[DP
							[DP
									[which, roof]
							]
							[NP
									[sufferings, roof]
							]
					]
					[IP
							[happened to her at home, roof]
					]
			]
			[Y
					[sufferings, roof]
			]
	]
\end{forest}

Standard raising: \tit{which} moves over \tit{sufferings}. --what triggers this movement? I am looking for an answer different from `there is a feature rel'-- From this position, however, \tit{sufferings} in the relative clause can delete \tit{sufferings} outside of the relative clause. \tit{Sufferings} is namely in the spec of the spec of Y.

\ex. \begin{forest} boom
	[YP
			[CP
					[DP
							[NP
									[sufferings, roof]
							]
							[DP
									[DP
											[which, roof]
									]
									[t$_{NP}$ ]
							]
					]
					[IP
							[happened to her at home, roof]
					]
			]
			[Y
					[\sout{sufferings}, roof]
			]
	]
	\end{forest}

Now \tit{some} comes into the picture. It merges outside the relative clause.

\ex. \begin{forest} boom
			[DP
					[D
							[some, roof]
					]
					[YP
							[CP
									[DP
											[NP
													[sufferings, roof]
											]
											[DP
													[DP
															[which, roof]
													]
													[t$_{NP}$ ]
											]
									]
									[IP
											[happened to her at home, roof]
									]
							]
							[Y
									[\sout{sufferings}, roof]
							]
					]
			]
	\end{forest}

Then case features need to be merged, because \tit{about} takes a genitive complement. The genitive is not only realized on \tit{some}, but also on \tit{sufferings} and \tit{which}. --actually, I do not see any genitive marking on \tit{sufferings}, even though it has been glossed like that. The point still is that genitive needs to land on elements both inside and outside the relative clause.--

\ex. \begin{forest} boom
			[DP
					[D
							[some-\tsc{gen}, roof]
					]
					[YP
							[CP
									[DP
											[NP
													[sufferings-\tsc{gen}, roof]
											]
											[DP
													[DP
															[which-\tsc{gen}, roof]
													]
													[t$_{NP}$ ]
											]
									]
									[IP
											[happened to her at home, roof]
									]
							]
							[Y
									[\sout{sufferings}, roof]
							]
					]
			]
	\end{forest}

Finally, about is merged.

\ex. \begin{forest} boom
	[PP
			[P
					[about, roof]
			]
			[DP
					[D
							[some-\tsc{gen}, roof]
					]
					[YP
							[CP
									[DP
											[NP
													[sufferings-\tsc{gen}, roof]
											]
											[DP
													[DP
															[which-\tsc{gen}, roof]
													]
													[t$_{NP}$ ]
											]
									]
									[IP
											[happened to her at home, roof]
									]
							]
							[Y
									[\sout{sufferings}, roof]
							]
					]
			]
	]
	\end{forest}

	In this section I showed what the structure is for headed relatives that show proper attraction.


\subsection{Proper attraction in headless relatives}\label{sec:attheadless}

In this section I discuss headless relatives with proper attraction. I show that the structure of these headless relatives is identical to the structure of their headed counterparts. What we regards as the relative pronoun is actually the demonstrative in the main clause.

In \ref{ex:aheadless} I give an example of proper attraction in a headless relative in Old High German. The internal predicate \tit{sprah} `spoke' requires a nominative subject. The external predicate \tit{antuurta} `replied' requires a dative object. The relative pronoun \tit{demo} appers in dative case.

\exg. Aer antuurta demo zaimo sprah.\\
he replied\scsub{dat} D-\tsc{m.sg.dat} {to him} spoke\scsub{nom}\\
`He replied to the one who spoke to him.'\label{ex:aheadless}

The structure I will work towards here is the following. This is basically the same syntactic configuration as the headed relative clause has. There are two differences: (1) the NP is not overt but an ∅ one, and (2) a d-pronoun is deleted under identity under c-command.

\ex. tree of \ref{ex:aheadless}\\
\begin{forest} boom
	[VP
			[V
					[replied, roof]
			]
			[DP
					[D
							[D-\tsc{dat}, roof]
					]
					[YP
							[CP
									[CP
											[NP
													[∅, roof]
											]
											[DP
													[DP
															[\sout{D-\tsc{dat}}, roof]
													]
													[t$_{∅}$ ]
											]
									]
									[IP
											[spoke to him, roof]
									]
							]
							[Y
									[\sout{∅}, roof]
							]
					]
			]
	]
	\end{forest}

I start again from a double-headed Cinque structure.

\ex.
\begin{forest} boom
	[YP
			[CP
					[CP
							[DP
									[dër, roof]
							]
							[NP
									[∅, roof]
							]
					]
					[IP
							[spoke to him, roof]
					]
			]
			[Y
					[∅, roof]
			]
	]
\end{forest}

The ∅ moves over the d-pronoun, coming into a position from which it can delete the external head (which is also ∅).

\ex.
\begin{forest} boom
	[YP
			[CP
					[CP
							[NP
									[∅, roof]
							]
							[DP
									[DP
											[dër, roof]
									]
									[t$_{NP}$ ]
							]
					]
					[IP
							[spoke to him, roof]
					]
			]
			[Y
					[\sout{∅}, roof]
			]
	]
\end{forest}

Now the external D comes into the picture. At this point the D does not have any phi and case features yet.

\ex.
\begin{forest} boom
[DP
		[D
				[D, roof]
		]
		[YP
				[CP
						[CP
								[NP
										[∅, roof]
								]
								[DP
										[DP
												[dër, roof]
										]
										[t$_{NP}$ ]
								]
						]
						[IP
								[spoke to him, roof]
						]
				]
				[Y
						[\sout{∅}, roof]
				]
		]
]
\end{forest}

Then phi and case features are merged. Case features do not only end up on the highest D, but also on the lower one.

\ex.
\begin{forest} boom
[DP
		[D
				[demo, roof]
		]
		[YP
				[CP
						[CP
								[NP
										[∅, roof]
								]
								[DP
										[DP
												[demo, roof]
										]
										[t$_{NP}$ ]
								]
						]
						[IP
								[spoke to him, roof]
						]
				]
				[Y
						[\sout{∅}, roof]
				]
		]
]
\end{forest}

Finally \tit{replied} can be merged. Then, optionally, the lowest \tit{demo} can be deleted. This is deletion under identity under c-command. It is a process that happens at PF.

\ex. tree of \ref{ex:aheadless}\\
\begin{forest} boom
	[VP
			[V
					[replied, roof]
			]
			[DP
					[D
							[D-\tsc{dat}, roof]
					]
					[YP
							[CP
									[CP
											[NP
													[∅, roof]
											]
											[DP
													[DP
															[\sout{D-\tsc{dat}}, roof]
													]
													[t$_{∅}$ ]
											]
									]
									[IP
											[spoke to him, roof]
									]
							]
							[Y
									[\sout{∅}, roof]
							]
					]
			]
	]
	\end{forest}

Do these two empty elements make sense? I would suggest yes. We independently see that the lower \tit{d}-element is optional in Old High German relative clauses. We can also independently observe that the head NP can be zero, in light headed relatives.

Deletion of the lower d-pronoun is optional. --show evidence for this: Old High German example with full NP as antecedent but no relative pronoun--

The NP is optionally ∅. --show evidence for this: Old High German example of light headed relative--



\section{Inverse attraction}\label{sec:invatt}

\subsection{Inverse attraction in headed relatives}\label{sec:invattheaded}

I repeat here the normal situation without attraction: head of the relative clause takes external case, relative pronoun takes internal case.

\exg. Ich sehe den Mann, der einen lustigen Hut trägt.\\
I see\scsub{acc} the.\tsc{acc} man, that.\tsc{nom} a.\tsc{acc} funny hat wears\scsub{nom}\\
`I see the man that is wearing a funny hat.'

The example in \ref{ex:iaheaded} shows an example of inverse attraction with a headed relative clause in Old High German. The predicate in the main clause \tit{zeslagen} `shattered' combines with nominative subjects. The predicate in the embedded clause \tit{held} `held' combines with accusative objects. Under inverse attraction, the head of the relative clause attracts the relative pronoun, and it is the relative pronoun that gives its case to the head of the relative clause. The relative pronoun takes the internal accusative case. The head of the relative pronoun does not take the external nominative case, but rather the internal accusative case.

\exg. Den schilt den er {vür bôt} der wart schiere zeslagen\\
the.\tsc{acc} shield.\tsc{acc} which.\tsc{acc} he held\scsub{acc}, that.\tsc{nom} was quickly shattered\scsub{nom}\\
`The shield he held was quickly shattered' \label{ex:iaheaded}\hfill inverse attraction

A few things should be noted here. First, inverse (but also proper) attraction is always optional (references). Inverse attraction never occurs when the head of the relative clause and the relative pronoun are split (show Bianchi with interfering elements). Attracted elements are `low in agentivity' and `high in discourse prominence' (Scott Grimm, see what I can do with that later). Furthermore, specific to inverse attraction is that the construction looks like it has been left-dislocated. Quite often there is a resumptive pronoun (except for nominatives, but pro-drop?). Inverse attraction mostly targets \tsc{acc} that overrides \tsc{nom}. A typical sentence with inverse attraction has the following shape: head NP - relative pronoun pronoun - embedded clause - resumptive pronoun - main clause.

\ex. tree of \ref{ex:iaheaded}\\
\begin{forest} boom
		[CP
				[DP
						[D
								[the-\tsc{acc}, roof]
						]
						[YP
								[CP
										[DP
												[NP
														[shield-\tsc{acc}, roof]
												]
												[DP
														[DP
																[which-\tsc{acc}, roof]
														]
														[\phantom{shield-\tsc{acc}}]
												]
										]
										[IP
												[he held, roof]
										]
								]
								[Y
										[\sout{shield}, roof]
								]
						]
				]
				[CP
						[that-\tsc{nom} was quickly shattered, roof]
				]
		]
\end{forest}
\z.

Several languages show that inverse attraction and extraposition are incompatible. Below I give an example of ? IA = inverse attraction, EP = extraposition. The examples show that inverse attraction and extraposition do not cooccur.

\ex.
\ag. doxtar ey ∅ [ke jon mišnose] inja æs\\
girl \tsc{art} \tsc{nom} \tsc{comp} John know.3 here be.3\\
`the girl that John knows is here' \hfill no IA, no EP
\bg. doxtar ey ra [ke jon mišnose] inja æs\\
girl \tsc{art} \tsc{acc} \tsc{comp} John know.3 here be.3\\
`the girl that John knows is here' \hfill IA, no EP
\bg. doxtar ey ∅ inja æs [ke jon mišnose]\\
girl \tsc{art} \tsc{nom} here be.3 \tsc{comp} John know.3\\
`the girl that John knows is here' \hfill no IA, EP
\bg. *doxtar ey ra inja æs [ke jon mišnose]\\
girl \tsc{art} \tsc{acc} here be.3 \tsc{comp} John know.3\\
`the girl that John knows is here' \hfill IA, EP

\subsection{Inverse attraction in headless relatives}

Now the examples from inverse attraction in Modern German headless relatives does not resemble inverse attraction in Old High German at all. First, in the headless relatives, there is no resumptive. Second, extraposition makes the situation worse, not better. Third, \tsc{nom/acc} combinations are actually harder to get than \tsc{acc/dat}.

\ex. \a. Ich lade ein wem du vertraust.
\b. ?Wem du vertraust, lade ich ein.
\b. ?Wen du getreten hast, hat einen kaputen Fuß.

The headless headless showing inverse attraction look structurally actually more like the headed structures with attraction. I will come back to now. The point here is: attraction in headless relatives can be derived from attraction in headed relatives, but inverse attraction in headless relatives canNOT be derived from inverse attraction in headed relatives.




\section{Deriving direction from the relative pronoun}

--I do not know yet how to do this--

I will connect demonstratives as relative pronoun to attraction and interrogatives as relative pronouns to inverse attraction.

\subsection{German}

The German free relative pronoun consists of two morphemes: a \tit{w-} and suffix that expresses phi and case features. In Standard German the free relative pronoun can be used in several different syntactic enviroments apart from the free relative pronoun use: as an interrogative, as an indefinite pronoun and as an exclamative marker.

--give examples of these contexts--

A combination of a \tit{d}- and the same phi and case feature suffix gives demonstratives and deteminers.

\subsubsection{\tit{W}- in German}

--here a discussion that ends with this tree:--

\ex. \begin{forest} boom
	[DP
 			[D]
 			[RelP
					[Rel]
 					[Wh]
			]
	]
	{\draw (.east) node[right]{$\Rightarrow$ \tit{w-}}; }
\end{forest}

\tit{W-} does not combine with all possible phi and case combinations.

\begin{table}[h]
	\center
	\caption {Paradigm for \tit{w}-elements in German}
	\begin{minipage}{0.57\linewidth}
		\begin{tabularx}{\textwidth}{c *{3}{Y}}
		\toprule
		 					& \multicolumn{2}{c}{\tsc{sg}} 	& \tsc{pl}			\\
							\cmidrule(lr){2-3} 							\cmidrule(l){4-4}
							& \tsc{n}	& \tsc{m/f} 					& \tsc{n/m/f} 	\\
		\midrule
		\tsc{nom} & w-as    & w-er   							&  							\\
		\tsc{acc} & w-as    & w-en   							& 							\\
		\tsc{dat} &     		& w-em   							&  							\\
		\tsc{gen} & 		   	& w-essen							& 							\\
		\bottomrule
		\end{tabularx}
	\end{minipage}
\end{table}

\subsubsection{\tit{D}- in German}

--here a discussion that ends with this tree:--

\ex. \begin{forest} boom
	[DP
			[D, roof]
	]
	{\draw (.east) node[right]{$\Rightarrow$ \tit{d-}}; }
\end{forest}

Actually not anaphor as a feature, but there needs to be something that does anaphoricity.. tbc

\tit{D}- combines with all phi and case combinations as a determiner.

\begin{table}[h]
	\center
	\caption {Paradigm for \tit{d}-elements in German}
	\begin{minipage}{0.71\linewidth}
		\begin{tabularx}{\textwidth}{c *{4}{Y}}
		\toprule
						& \multicolumn{3}{c}{\tsc{sg}} 	& \tsc{pl}		\\
						\cmidrule(lr){2-4} 							\cmidrule(l){5-5}
						& \tsc{n}	& \tsc{m}	& \tsc{f} 	& \tsc{n/m/f} \\
		\midrule
		\tsc{nom} & d-as    & d-er   	& d-ie   		& d-ie   		\\
		\tsc{acc} & d-as    & d-en   	& d-ie   		& d-ie   		\\
		\tsc{dat} & d-em    & d-em   	& d-er   		& d-en   		\\
		\tsc{gen} & d-es    & d-es   	& d-er   		& d-er			\\
		\bottomrule
		\end{tabularx}
	\end{minipage}
\end{table}



\citet[77]{hachem2015a} ``When \tit{der} is in a position where it refers anaphorically to a noun as an RP or DPR, it displays an additional morpheme in the genitive forms of all numbers and gender and in the dative plural.'' This should help me explain why \tit{w-} does not combine with plurals (and neuter genitive and dative?).

\begin{table}[h]
	\center
	\caption {More complex \tit{d}-elements in the paradigm}
	\begin{minipage}{0.71\linewidth}
		\begin{tabularx}{\textwidth}{c *{4}{Y}}
		\toprule
		 					& \multicolumn{3}{c}{\tsc{sg}} 	& \tsc{pl}		\\
							\cmidrule(lr){2-4} 							\cmidrule(l){5-5}
							& \tsc{n}	& \tsc{m}	& \tsc{f} 	& \tsc{n/m/f} 										\\
		\midrule
		\tsc{nom} & d-as    & d-er   	& d-ie   		& d-ie   													\\
		\tsc{acc} & d-as    & d-en   	& d-ie   		& d-ie   													\\
		\tsc{dat} & d-em    & d-em   	& d-er   		& \cellcolor{Gray}{d-en-\tbf{en}}	\\
		\tsc{gen} & \cellcolor{Gray}{d-es-\tbf{sen}}
							& \cellcolor{Gray}{d-es-\tbf{sen}}
							& \cellcolor{Gray}{d-er-\tbf{en}}
							& \cellcolor{Gray}{d-en-\tbf{en}}																	\\
		\bottomrule
		\end{tabularx}
	\end{minipage}
\end{table}




\subsection{Gothic}

\subsubsection{\tit{þ}- in Gothic}

\ex. \begin{forest} boom
 	[DP
			[D, roof]
	]
	{\draw (.east) node[right]{$\Rightarrow$ \tit{þ-}}; }
\end{forest}


\begin{table}[h]
	\center
	\caption {Paradigm for \tit{þ}-elements in German}
	\begin{minipage}{\linewidth}
		\begin{tabularx}{\textwidth}{c *{6}{Y}}
		\toprule
		 					& \multicolumn{3}{c}{\tsc{sg}} 	& \multicolumn{3}{c}{\tsc{pl}}	\\
							\cmidrule(lr){2-4} 							\cmidrule(l){5-7}
							& \tsc{n}	& \tsc{m}	& \tsc{f} 	& \tsc{n}	& \tsc{m}	& \tsc{f} 	\\
		\midrule
		\tsc{nom} & þ-ata	  & s-a   	& s-ō  			& þ-ō   	&	þ-ái		&	þ-ōs			\\
		\tsc{acc} & þ-ata   & þ-ana  	& þ-ō   		& þ-ō   	&	þ-ans		&	þ-ōs			\\
		\tsc{dat} & þ-amma  & þ-amma  & þ-izái  	& þ-áim   &	þ-áim		&	þ-áim			\\
		\tsc{gen} & þ-is    & þ-is    & þ-izōs  	& þ-ize 	&	þ-ize		&	þ-izō			\\
		\bottomrule
		\end{tabularx}
	\end{minipage}
\end{table}




\subsubsection{-\tit{ei} in Gothic}

\ex. \begin{forest} boom
	[RelP
			[Rel]
			[WhP
					[Wh]
			]
	]
	{\draw (.east) node[right]{$\Rightarrow$ \tit{-ei}}; }
\end{forest}


\begin{table}[h]
	\center
	\caption {Paradigm for \tit{þ-}+\tit{-ei}-elements in German}
	\begin{minipage}{\linewidth}
		\begin{tabularx}{\textwidth}{c *{6}{Y}}
		\toprule
		 					& \multicolumn{3}{c}{\tsc{sg}} 	& \multicolumn{3}{c}{\tsc{pl}}	\\
							\cmidrule(lr){2-4} 							\cmidrule(l){5-7}
							& \tsc{n}		& \tsc{m}		& \tsc{f} 	& \tsc{n}		& \tsc{m}		& \tsc{f} 	\\
		\midrule
		\tsc{nom} & þ-at-ei	  & s-a-ei 		& s-ō-ei 		& þ-ō-ei   	&	þ-ái-ei		&	þ-ōz-ei		\\
		\tsc{acc} & þ-at-ei   & þ-an-ei  	& þ-ō-ei   	& þ-ō-ei   	&	þ-anz-ei	&	þ-ōz-ei		\\
		\tsc{dat} & þ-amm-ei  & þ-amm-ei	& þ-izái-ei & þ-áim-ei	&	þ-áim-ei 	&	þ-áim-ei	\\
		\tsc{gen} & þ-iz-ei   & þ-iz-ei		& þ-izōz-ei & þ-izē-ei 	&	þ-izē-ei	&	þ-izō-ei	\\
		\bottomrule
		\end{tabularx}
	\end{minipage}
\end{table}



\subsubsection{\tit{ƕ}- in Gothic}

\ex. \begin{forest} boom
 	[WhP
			[Wh, roof]
	]
	{\draw (.east) node[right]{$\Rightarrow$ \tit{ƕ-}}; }
\end{forest}


\begin{table}[h]
	\center
	\caption {Paradigm for \tit{ƕ}-elements in Gothic}
	\begin{minipage}{0.71\linewidth}
		\begin{tabularx}{\textwidth}{c *{4}{Y}}
		\toprule
		 					& \multicolumn{3}{c}{\tsc{sg}} 	& \tsc{pl}		\\
							\cmidrule(lr){2-4} 							\cmidrule(l){5-5}
							& \tsc{n}	& \tsc{m}	& \tsc{f} 	& \tsc{n/m/f}	\\
		\midrule
		\tsc{nom} & ƕ-a	  	& ƕ-as   	& ƕ-ō  			& 						\\
		\tsc{acc} & ƕ-a   	& ƕ-ana  	& ƕ-ō   		& 						\\
		\tsc{dat} & ƕ-amma  & ƕ-amma  & ƕ-izái  	& 						\\
		\tsc{gen} & ƕ-is    & ƕ-is    & ƕ-izōs  	& 						\\
		\bottomrule
		\end{tabularx}
	\end{minipage}
\end{table}





\subsection{Old High German}


\subsubsection{\tit{d-} in Old High German}

\ex. \begin{forest} boom
	[DP
			[D, roof]
	]
	{\draw (.east) node[right]{$\Rightarrow$ \tit{d-}}; }
\end{forest}




\begin{table}[h]
	\center
	\caption {Paradigm for \tit{d}-elements in Old High German}
	\begin{minipage}{\linewidth}
		\begin{tabularx}{\textwidth}{c *{6}{Y}}
		\toprule
		 					& \multicolumn{3}{c}{\tsc{sg}} 		& \multicolumn{3}{c}{\tsc{pl}}	\\
							\cmidrule(lr){2-4} 								\cmidrule(l){5-7}
							& \tsc{n}	& \tsc{m}	& \tsc{f} 		& \tsc{n}	& \tsc{m}	& \tsc{f} 	\\
		\midrule
		\tsc{nom} & d-aʐ   	& d-ër   	& d-iu   			& d-iu 		& d-ē  		&	d-eo/-io	\\
		\tsc{acc} & d-aʐ    & d-ën   	& d-ea, d-ia 	& d-iu 		& d-ē  		&	d-eo/io		\\
		\tsc{dat} & d-ëmu/o	& d-ëmu/o & d-ëru/o  		& d-ēm/n 	& d-ēm/n	&	d-ēm/n		\\
		\tsc{gen} & d-ës    & d-ës   	& d-ëra/u/o   &	d-ëro		&	d-ëro		&	d-ëro			\\
		\bottomrule
		\end{tabularx}
	\end{minipage}
\end{table}


m.pl can also be dea dia die


\subsubsection{\tit{(h)w-} in Old High German}


\ex. \begin{forest} boom
			[WhP
					[Wh, roof]
			]
			{\draw (.east) node[right]{$\Rightarrow$ \tit{(h)w-}}; }
\end{forest}

\begin{table}[h]
	\center
	\caption {Paradigm for \tit{(h)w}-elements in Gothic}
	\begin{minipage}{0.71\linewidth}
		\begin{tabularx}{\textwidth}{c *{3}{Y}}
		\toprule
		 	& \multicolumn{2}{c}{\tsc{sg}} 							& \tsc{pl}				\\
				\cmidrule(lr){2-3} 												\cmidrule(l){4-4}
			& \tsc{n}		& \tsc{m/f} 										& \tsc{n/m/f}			\\
				\midrule
				\tsc{nom} & (h)w-aʐ	 			& (h)w-ër				& 								\\
				\tsc{acc} & (h)w-aʐ  			& (h)w-ën(an)		& 								\\
				\tsc{dat} & hw-ëmu/w-ëmo	& hw-ëmu/w-ëmo	&  								\\
				\tsc{gen} & (h)w-ës 			& (h)w-ës			 	& 								\\
			\bottomrule
		\end{tabularx}
	\end{minipage}
\end{table}




\section{Conclusion}

All languages

\begin{table}[h]\label{tbl:morph}
	\center
	\caption {\tsc{int} vs. \tsc{ext} in Modern and Old High German and Gothic}
	\begin{minipage}{\linewidth}
		\begin{tabularx}{\textwidth}{c *{3}{Y}}
		\toprule
		 								& free relative	& light headed relative &	\\
		\midrule
		Modern German 	& w-			 				&	d-, d-							&	inverse attraction	\\
		Old High German	& d-							&	d-, d-?							&	attraction					\\
		Gothic					&	þ- + -ei				&	þ-, þ-?							&	both								\\
		Greek						& o-pj-						&	?										&	attraction					\\
		Czech						& c-							&	t-, c-,							& none								\\
		Italian					& c-							& q-, c-,							& none								\\
		\bottomrule
		\end{tabularx}
	\end{minipage}
\end{table}
