% !TEX root = thesis.tex

\chapter{Direction of attraction}


  \section{Typology}
Old High German only has proper attraction. Modern German only has inverse attraction. Gothic has both proper and inverse attraction.


\begin{table}[h]\label{tbl:intextgoth}
	\center
	\caption {\tsc{int} vs. \tsc{ext} in Modern and Old High German and Gothic}
	\begin{minipage}{\linewidth}
		\begin{tabularx}{\textwidth}{c *{2}{Y}}
		\toprule
		 								& \tsc{int>ext}				& \tsc{ext>int}				\\
										& inverse attraction	& proper attraction		\\
		\midrule
		Modern German 	& ✔			 							&	*										\\
		Old High German	& *										&	✔										\\
		Gothic					&	✔										&	✔										\\
		\bottomrule
		\end{tabularx}
	\end{minipage}
\end{table}


    \subsection{Gothic}
    \subsection{Old High German}



\ex. \tsc{ext:gen}, \tsc{int:nom}
\ag. suachit thes nan sentit\\
 search.\tsc{3.sg}\scsub{gen?} D-\tsc{n.sg.gen} \tsc{3.sg.m.acc} sent\\
 `he searched for who sent him/ der sucht die Sachen dessen der ihn sendet' \hfill (Old High German, Otfrid III,16,21, \citealt[761]{behaghel1923})
\bg. diu habe niemer niht entuot, des der seele schaden si\\
 they have not never {not do} D-\tsc{gen.sg.n} the soul damage is\scsub{[nom]}\\
 `' \hfill (Middle High German, Warn. 2490, \citealt[761]{behaghel1923})

\ex. \tsc{ext:dat}, \tsc{int:nom}
\ag. Aer antuurta demo zaimo sprah.\\
he replied\scsub{[dat]} D-\tsc{m.sg.dat} {to him} spoke\scsub{[nom]}\\
`He replied to the one who spoke to him.' \hfill (Old High German, Monsee Fragments 7,24, \citealt[761]{behaghel1923})
\bg. gebe themo ni eigi\\
 give\scsub{[dat]} D--\tsc{m.sg.dat} not posses\scsub{[nom]}\\
 `give to whom that does not have' gebe dem der nicht hat \hfill (Old High German, Otfrid I,24,77, Schrodt,175)

\ex. \tsc{ext:acc}, \tsc{int:nom}
\ag. ih bibringu fona Juda dhen mina berga chisetzit\\
  I educate von Juda D-\tsc{m/n/f.pl.dat}/\tsc{m.sg.acc} my? mountain order/put\\
  `' \hfill (Old High German, Isidor 34,3, \citealt[761]{behaghel1923})

\ex. \tsc{ext:gen}, \tsc{int:acc}
\ag. der bewiset in des er suochte\\
 he directed\scsub{[gen]} him D-\tsc{n.sg.gen} he sought\scsub{[acc]}\\
 `He directed him to what he sought.' \hfill (Middle High German, Iwein 988, \citealt[761]{behaghel1923}), trans. Hartmann von Aue-Portal
\bg. giwisso ni birut ir thero ih irwellu zi mir\\
 bestimmt not belong\scsub{[gen]} \tsc{2.pl.nom} D-\tsc{gen.pl} I choose\scsub{[acc]} to me\\
 `you surely do not belong to those that I choose for myself //sicherlich gehört ihr nicht zu denen die ich mir erwähle' \hfill (Old High German, Otfrid III,22,20, Schrodt,p.175)

\ex. \tsc{ext:dat}, \tsc{int:acc}
\ag. istû furira Abrâhame, ouh thên man hiar nû zalta?\\
 {are you} {superior to}\scsub{[dat]} Abraham also D-\tsc{d.pl} one here now mentioned\scsub{[acc]}\\
 `are you bigger than Abraham and those people named just now?' \hfill (Old High German, Otfrid III,18,33, \citealt[761]{behaghel1923})

\ex. \tsc{ext:gen}, \tsc{int:dat}







Don't know:

\ex. Old High German
\ag. gaat uz diu halt za dem iz forchaufent\\
 \\
 `' \hfill (Old High German, Monsee Fragments 20,14, \citealt[761]{behaghel1923})
\bg. thisiu fon thiu, iru wan ist, siu alla iru libnara santa (ex eo, quod)\\
 \\
 `hæc autem ex eo quod deest illi, totum victum suum quem habuit misit.' \hfill (Old High German, Tatian 118,1, \citealt[761]{behaghel1923})
\bg. thaz iru thiu sin guati nirzigi, thes siu bati\\
 \\
 `' \hfill (Old High German, Otfrid II,8,24, \citealt[761]{behaghel1923})
\bg. thia laz ih themo iz lisit thar\\
 \\
 `' \hfill (Old High German, Otfrid I,19,25, \citealt[761]{behaghel1923})
\bg. noh so neduohti in gnuoge des si habetin\\
 \\
 `' \hfill (Old High German, Notker I,63,29, \citealt[761]{behaghel1923})
\bg. tannoh pito ih tes noh fore ist (id quod)\\
 \\
 `' \hfill (Old High German, Notker 193,19, \citealt[761]{behaghel1923})

So, to sum up:




  \begin{table}[h]
    \center
    \caption {Case attraction in headless relatives in Old High German}
      \begin{minipage}{\linewidth}
        \begin{tabularx}{\textwidth}{c|Y|Y|Y|Y}
          \toprule
            \diagbox[linecolor=white]{\tsc{ext}}{\tsc{int}}	  & \tsc{[nom]}                                                       & \tsc{[acc]} 	                                                  & \tsc{[dat]} 											                & \tsc{[gen]} 											                  \\ \cmidrule(lr){1-5}
            \tsc{[nom]} 		                                  & \colorbox{LG}{\tsc{nom}} 			                                    & \diagbox[linecolor=white]{*\tsc{nom}}{*\tsc{acc}} 	            & \diagbox[linecolor=white]{*\tsc{nom}}{*\tsc{dat}} & \diagbox[linecolor=white]{*\tsc{nom}}{*\tsc{gen}} 	\\ \cmidrule(lr){1-5}
            \tsc{[acc]} 		                                  & \diagbox[linecolor=white]{\colorbox{DG}{\tsc{acc}}}{*\tsc{nom}} 	&	\colorbox{LG}{\tsc{acc}} 		                                   	&	\diagbox[linecolor=white]{*\tsc{acc}}{*\tsc{dat}} &	\diagbox[linecolor=white]{*\tsc{acc}}{*\tsc{gen}}   \\ \cmidrule(lr){1-5}
            \tsc{[dat]} 		                                  & \diagbox[linecolor=white]{\colorbox{DG}{\tsc{dat}}}{*\tsc{nom}}	  &	\diagbox[linecolor=white]{\colorbox{DG}{\tsc{dat}}}{*\tsc{acc}} & \colorbox{LG}{\tsc{dat}} 			                    & \diagbox[linecolor=white]{*\tsc{dat}}{*\tsc{gen}}	  \\ \cmidrule(lr){1-5}
            \tsc{[gen]} 		                                  & \diagbox[linecolor=white]{\colorbox{DG}{\tsc{gen}}}{*\tsc{nom}} 	& \diagbox[linecolor=white]{\colorbox{DG}{\tsc{gen}}}{*\tsc{acc}} & \diagbox[linecolor=white]{\colorbox{DG}{\tsc{gen}}}{*\tsc{dat}} 	& \colorbox{LG}{\tsc{gen}}				                  \\
          \bottomrule
        \end{tabularx}
      \end{minipage}
  \end{table}






    \subsection{Modern German}



  \section{Background: relative clause theory}
Standard raising, probably Cinque's double-headed structures


  \section{Shape of relative pronoun}
Old High German has a d-pronoun. Modern German has a wh-pronoun. Gothic has a d-pronoun plus a caseless relativizer.


\begin{table}[h]\label{tbl:intextgoth}
	\center
	\caption {\tsc{int} vs. \tsc{ext} in Modern and Old High German and Gothic}
	\begin{minipage}{\linewidth}
		\begin{tabularx}{\textwidth}{c *{6}{Y}}
		\toprule
		\tsc{sg}	& \tsc{m}     & \tsc{n}     &\tsc{f}           \\
		\midrule
		\tsc{nom} & dër       	& daȥ					& diu              \\
		\tsc{acc}	& dën					&	daȥ		      & dea, dia (die)	 \\
		\tsc{dat}	&	dëmu, dëmo	&	dëmu, dëmo	& dëru, dëro			 \\
    \tsc{gen} & dës         & dës         & dëra, dëru, dëro \\
		\bottomrule
    \toprule
    \tsc{pl}	& \tsc{m}            & \tsc{n}      & \tsc{f}   \\
    \midrule
    \tsc{nom} &  dē, dea, dia, die & diu, dei     & deo, dio  \\
    \tsc{acc} &  dē, dea, dia, die & diu, dei     & deo, dio  \\
    \tsc{dat} &  dēm, dēn          & dēm, dēn     & dēm, dēn  \\
    \tsc{gen} &  dëro              & dëro         & dëro      \\
    \bottomrule
		\end{tabularx}
	\end{minipage}
\end{table}




  \section{Analysis}

    \subsection{Old High German}
In Old High German, proper attraction in headless relatives can be derived from headed relatives. The relative pronoun is the determiner from the main clause. Under a double-headed Cinque-analysis, it is the internal DP that is deleted.




\ex. \tsc{acc} instead of \tsc{nom}
\ag. unde ne wolden níet besên den mort den dô was geschên\\
 and not wanted not see the murder.\tsc{acc} that.\tsc{acc} there had happened\\
 `and they didn't want to see the murder that had happened.' \hfill (Middle High German, Nibelungenlied 1391,14, \citealt[756]{behaghel1923}, glosses and translation by \citealt[198]{pittner1995})



    \subsection{Modern German}
In German, inverse attraction in headed relatives can be shown to be very different from inverse attraction in headless relatives. I am not set on an analysis yet. Under a double-headed Cinque-analysis, it is the external DP that is deleted. Grafting is also still an option.

    \subsection{Gothic}
In Gothic, ?


\section{No attraction}
Italian has none. Italian uses its free relative pronoun also in light-headed relative pronouns.


  \section{Bigger picture}
Relative pronoun is a descriptive term. What we analyze as relative pronouns are sometimes wh-elements, sometimes determiners.

Case attraction is also a descriptive term. The constructions are underlyingly very different.
