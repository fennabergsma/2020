% !TEX root = thesis.tex

\chapter{Direction of attraction}


  \section{Typology}
Old High German only has proper attraction. Modern German only has inverse attraction. Gothic has both proper and inverse attraction.

\begin{table}[h]\label{tbl:intextgoth}
	\center
	\caption {\tsc{int} vs. \tsc{ext} in Modern and Old High German and Gothic}
	\begin{minipage}{\linewidth}
		\begin{tabularx}{\textwidth}{c *{2}{Y}}
		\toprule
		 								& \tsc{int>ext}				& \tsc{ext>int}				\\
										& inverse attraction	& proper attraction		\\
		\midrule
		Modern German 	& ✔			 							&	*										\\
		Old High German	& *										&	✔										\\
		Gothic					&	✔										&	✔										\\
		\bottomrule
		\end{tabularx}
	\end{minipage}
\end{table}


    \subsection{Gothic}
    \subsection{Old High German}
    \subsection{Modern German}



  \section{Background: relative clause theory}
Standard raising, probably Cinque's double-headed structures


  \section{Shape of relative pronoun}
Old High German has a d-pronoun. Modern German has a wh-pronoun. Gothic has a d-pronoun plus a caseless relativizer.

  \section{Analysis}

    \subsection{Old High German}
In Old High German, proper attraction in headless relatives can be derived from headed relatives. The relative pronoun is the determiner from the main clause. Under a double-headed Cinque-analysis, it is the internal DP that is deleted.

    \subsection{Modern German}
In German, inverse attraction in headed relatives can be shown to be very different from inverse attraction in headless relatives. I am not set on an analysis yet. Under a double-headed Cinque-analysis, it is the external DP that is deleted. Grafting is also still an option.

    \subsection{Gothic}
In Gothic, ?


\section{No attraction}
Italian has none. Italian uses its free relative pronoun also in light-headed relative pronouns.


  \section{Bigger picture}
Relative pronoun is a descriptive term. What we analyze as relative pronouns are sometimes wh-elements, sometimes determiners.

Case attraction is also a descriptive term. The constructions are underlyingly very different.
