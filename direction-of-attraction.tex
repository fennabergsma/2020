% !TEX root = thesis.tex

\chapter{Direction of attraction}


\section{The typology}
Old High German only has proper attraction. Modern German only has inverse attraction. Gothic has both proper and inverse attraction.


\begin{table}[h]
	\center
	\caption {\ac{int} vs. \ac{ext} in Modern and Old High German and Gothic}
		\begin{tabular}{ccc}
		\toprule
		 								& \ac{int}>\ac{ext}				& \ac{ext}>\ac{int}				\\
										& inverse attraction	& proper attraction		\\
										\cmidrule{2-3}
		Modern German 	& ✔			 							&	*										\\
		Old High German	& *										&	✔										\\
		Gothic					&	✔										&	✔										\\
		\bottomrule
		\end{tabular}
\end{table}



\subsection{Gothic}


\begin{table}[h]
  \center
  \caption {Case attraction in headless relatives in Gothic}
    \begin{tabular}{c|c|c|c}
			\toprule
				\diagbox[linecolor=white]{\ac{int}}{\ac{ext}}
						& [\ac{nom}]
						& [\ac{acc}]
						& [\ac{dat}]
						\\ \cmidrule{1-4}
				[\ac{nom}]
						& \colorbox{LG}{\ac{nom}}
						& \diagbox[linecolor=white]{?\ac{nom}}{\colorbox{DG}{\ac{acc}}}
						& \diagbox[linecolor=white]{?\ac{nom}}{\colorbox{DG}{\ac{dat}}}
						\\ \cmidrule{1-4}
				[\ac{acc}]
						& \diagbox[linecolor=white]{\colorbox{DG}{\ac{acc}}}{?\ac{nom}}
						&	\colorbox{LG}{\ac{acc}}
						&	\diagbox[linecolor=white]{?\ac{acc}}{\colorbox{DG}{\ac{dat}}}
						\\ \cmidrule{1-4}
				[\ac{dat}]
						& \diagbox[linecolor=white]{\colorbox{DG}{\ac{dat}}}{?\ac{nom}}
						&	\diagbox[linecolor=white]{\colorbox{DG}{\ac{dat}}}{?\ac{acc}}
						& \colorbox{LG}{\ac{dat}}
						\\
			\bottomrule
    \end{tabular}
\end{table}


\subsection{Old High German}


\ex. \ac{int}:\ac{nom}, \ac{ext}:\ac{acc}
\a. \ac{nom} not attested
\bg. ih bibringu fona Juda dhen mina berga chisetzit\\
 I educate\scsub{[acc]} about Juda who.\ac{acc} my mountains {through pull}\scsub{[nom]}\\
 `I educate the one who wanders through my mountains about Judas' \flushfill{(Old High German, Isidor 34:3, \citealt[761]{behaghel1923})}

\ex. \ac{int}:\ac{nom}, \ac{ext}:\ac{dat}
\a. \ac{nom} not attested
\bg. aer antuurta demo zaimo sprah\\
 he replied\scsub{[dat]} who.\ac{dat} {to him} spoke\scsub{[nom]}\\
 `he replied to the one who spoke to him' \flushfill{(Old High German, Monsee Fragments 7:24, \citealt[761]{behaghel1923}, gloss and translation by \citealt[199]{pittner1995})}
\bg. gebe themo ni eigi\\
 give\scsub{[dat]} who.\ac{dat} not posses\scsub{[nom]}\\
 `give to the one who does not have' \flushfill{(Old High German, Otfrid I 24:77)}\\
 `gebe dem der nicht hat' \flushfill{(translation by Schrodt, 175)}

\ex. \ac{int}:\ac{acc}, \ac{ext}:\ac{nom}
\a. \ac{acc} not attested
\b. \ac{nom} not attested

\ex. \ac{int}:\ac{acc}, \ac{ext}:\ac{dat}
\a. \ac{acc} not attested
\bg. istû furira Abrâhame, ouh thên man hiar nû zalta?\\
 {are you} superior\scsub{[dat]} {to Abraham} also who.\ac{dat} one here now named\scsub{[acc]}\\
 `are you superior to Abraham to those which they just mentioned?' \flushfill{(Old High German, Otfrid III 18:33, \citealt[761]{behaghel1923})}

\ex. \ac{int}:\ac{dat}, \ac{ext}:\ac{nom}
\a. \ac{dat} not attested
\b. \ac{nom} not attested

\ex. \ac{int}:\ac{dat}, \ac{ext}:\ac{acc}
\a. \ac{dat} not attested
\b. \ac{acc} not attested





Don't know:

\ex. Old High German
\ag. gaat uz diu halt za dem iz forchaufent\\
 \\
 `' \flushfill{(Old High German, Monsee Fragments 20,14, \citealt[761]{behaghel1923})}
\bg. thisiu fon thiu, iru wan ist, siu alla iru libnara santa (ex eo, quod)\\
 \\
 `hæc autem ex eo quod deest illi, totum victum suum quem habuit misit.' \flushfill{(Old High German, Tatian 118,1, \citealt[761]{behaghel1923})}
\bg. thaz iru thiu sin guati nirzigi, thes siu bati\\
 \\
 `' \flushfill{(Old High German, Otfrid II,8,24, \citealt[761]{behaghel1923})}
\bg. thia laz ih themo iz lisit thar\\
 \\
 `' \flushfill{(Old High German, Otfrid I,19,25, \citealt[761]{behaghel1923})}
\bg. noh so neduohti in gnuoge des si habetin\\
 \\
 `' \flushfill{(Old High German, Notker I,63,29, \citealt[761]{behaghel1923})}
\bg. tannoh pito ih tes noh fore ist (id quod)\\
 \\
 `' \flushfill{(Old High German, Notker 193,19, \citealt[761]{behaghel1923})}

So, to sum up:



\begin{table}[h]
  \center
  \caption {Case attraction in headless relatives in Old High German}
    \begin{tabular}{c|c|c|c}
			\toprule
				\diagbox[linecolor=white]{\ac{int}}{\ac{ext}}
						& [\ac{nom}]
						& [\ac{acc}]
						& [\ac{dat}]
						\\ \cmidrule{1-4}
				[\ac{nom}]
						& \colorbox{LG}{\ac{nom}}
						& \diagbox[linecolor=white]{?\ac{nom}}{\colorbox{DG}{\ac{acc}}}
						& \diagbox[linecolor=white]{?\ac{nom}}{\colorbox{DG}{\ac{dat}}}
						\\ \cmidrule{1-4}
				[\ac{acc}]
						& \diagbox[linecolor=white]{?\ac{acc}}{?\ac{nom}}
						&	\colorbox{LG}{\ac{acc}}
						&	\diagbox[linecolor=white]{?\ac{acc}}{\colorbox{DG}{\ac{dat}}}
						\\ \cmidrule{1-4}
				[\ac{dat}]
						& \diagbox[linecolor=white]{?\ac{dat}}{?\ac{nom}}
						&	\diagbox[linecolor=white]{?\ac{dat}}{?\ac{acc}}
						& \colorbox{LG}{\ac{dat}}
						\\
			\bottomrule
    \end{tabular}
\end{table}




\subsection{Modern German}

\ex. \ac{int}:\ac{nom}, \ac{ext}:\ac{acc}
\ag. *Ich {lade ein}, wer mir sympathisch ist.\\
 I invite\scsub{[acc]} who.\ac{nom} me nice is\scsub{[nom]}\\
 `I invite who I like.' \flushfill{\citep[344]{vogel2001}}
\bg. *Ich {lade ein}, wen mir sympathisch ist.\\
 I invite\scsub{[acc]} who.\ac{acc} me nice is\scsub{[nom]}\\
 `I invite who I like.' \flushfill{\citep[344]{vogel2001}}

\ex. \ac{int}:\ac{nom}, \ac{ext}:\ac{dat}
\ag. *Ich vertraue, wer Hitchcock mag.\\
 I trust\scsub{[dat]} who.\ac{nom} Hitchcock likes\scsub{[nom]}\\
 `I trust who likes Hitchcock.' \flushfill{\citep[345]{vogel2001}}
\bg. *Ich vertraue, wem Hitchcock mag.\\
 I trust\scsub{[dat]} who.\ac{dat} Hitchcock likes\scsub{[nom]}\\
 `I trust who likes Hitchcock.' \flushfill{\citep[345]{vogel2001}}

\ex. \ac{int}:\ac{acc}, \ac{ext}:\ac{nom}
\ag. Uns besucht wen Maria mag.\\
 Us visits\scsub{[nom]} who.\ac{acc} Maria.\ac{nom} likes\scsub{[acc]}\\
 `Who visits us likes Maria likes.' \flushfill{\citep[343]{vogel2001}}
\bg. *Uns besucht wer Maria mag.\\
 Us visits\scsub{[nom]} who.\ac{nom} Maria.\ac{nom} likes\scsub{[acc]}\\
 `Who visits us likes Maria likes.' \flushfill{\citep[343]{vogel2001}}

 \ex. \ac{int}:\ac{acc}, \ac{ext}:\ac{dat}
\ag. *Ich vertraue wem auch Maria mag. \\
 I trust\scsub{[dat]} who.\ac{dat} also Maria likes\scsub{[acc]}.\\
 `I trust whoever Maria also likes.' \flushfill{\citep[345]{vogel2001}}
\bg. *Ich vertraue wen auch Maria mag. \\
 I trust\scsub{[dat]} who.\ac{acc} also Maria likes\scsub{[acc]}.\\
 `I trust whoever Maria also likes.' \flushfill{\citep[345]{vogel2001}}

\ex. \ac{int}:\ac{dat}, \ac{ext}:\ac{nom}
\ag. Uns besucht wem Maria vertraut.\\
 us visits\scsub{[nom]} who.\ac{dat} Maria trusts\scsub{[dat]}\\
 `Who visits us, Maria trusts.' \flushfill{\citep[343]{vogel2001}}
\bg. *Uns besucht wer Maria vertraut.\\
 us visits\scsub{[nom]} who.\ac{nom} Maria trusts\scsub{[dat]}\\
 `Who visits us, Maria trusts.' \flushfill{\citep[343]{vogel2001}}

\ex. \ac{int}:\ac{dat}, \ac{ext}:\ac{acc}
\ag. Ich {lade ein} wem auch Maria vertraut. \\
 I invite\scsub{[acc]} who.\ac{dat} also Maria trusts\scsub{[dat]}.\\
 `I invite whoever Maria also trusts.' \flushfill{\citep[344]{vogel2001}}
\bg. *Ich {lade ein} wen auch Maria vertraut. \\
 I invite\scsub{[acc]} who.\ac{acc} also Maria trusts\scsub{[dat]}.\\
 `I invite whoever Maria also trusts.' \flushfill{\citep[344]{vogel2001}}


 \begin{table}[h]
   \center
   \caption {Case attraction in headless relatives in Modern German}
     \begin{tabular}{c|c|c|c}
			 \toprule
				 \diagbox[linecolor=white]{\ac{int}}{\ac{ext}}
						 & [\ac{nom}]
						 & [\ac{acc}]
						 & [\ac{dat}]
						 \\ \cmidrule{1-4}
				 [\ac{nom}]
						 & \colorbox{LG}{\ac{nom}}
						 & \diagbox[linecolor=white]{*\ac{nom}}{*\ac{acc}}
						 & \diagbox[linecolor=white]{*\ac{nom}}{*\ac{dat}}
						 \\ \cmidrule{1-4}
				 [\ac{acc}]
						 & \diagbox[linecolor=white]{\colorbox{DG}{\ac{acc}}}{*\ac{nom}}
						 &	\colorbox{LG}{\ac{acc}}
						 &	\diagbox[linecolor=white]{*\ac{acc}}{*\ac{dat}}
						 \\ \cmidrule{1-4}
				 [\ac{dat}]
						 & \diagbox[linecolor=white]{\colorbox{DG}{\ac{dat}}}{*\ac{nom}}
						 &	\diagbox[linecolor=white]{\colorbox{DG}{\ac{dat}}}{*\ac{acc}}
						 & \colorbox{LG}{\ac{dat}}
						 \\
			 \bottomrule
     \end{tabular}
 \end{table}






  \section{Background: relative clause theory}
Standard raising, probably Cinque's double-headed structures


  \section{Shape of relative pronoun}
Old High German has a d-pronoun. Modern German has a wh-pronoun. Gothic has a d-pronoun plus a caseless relativizer.


\subsection{Old High German}

\begin{table}[h]\label{tbl:paradigmohg}
	\center
	\caption {Old High German relative pronouns in headless relatives}
		\begin{tabular}{cccc}
		\toprule
							& \ac{n}.\ac{sg}	    & \ac{m}.\ac{sg}  & \ac{f}.\ac{sg}			\\
								\cmidrule{2-4}
		\ac{nom} & d-aȥ          	& d-ër       	& d-iu						\\
		\ac{acc}	& d-aȥ   					& d-ën				& d-ea/-ia/(-ie)	\\
		\ac{dat}	& d-ëmu/-ëmo	    & d-ëmu/-ëmo	& d-ëru/-ëro			\\
		\bottomrule
	    				& \ac{n}.\ac{pl} & \ac{m}.\ac{pl}       & \ac{f}.\ac{pl}      \\
	    					\cmidrule{2-4}
    \ac{nom} & d-iu/-ei   &  d-ē/-ea/-ia/-ie & d-eo/-io        \\
    \ac{acc} & d-iu/-ei   &  d-ē/-ea/-ia/-ie & d-eo/-io        \\
    \ac{dat} & d-ēm/-ēn   &  d-ēm/-ēn        & d-ēm/-ēn        \\
    \bottomrule
		\end{tabular}
\end{table}


\subsection{Gothic}

\begin{table}[h]
	\center
	\caption {Gothic relative pronouns in headless relatives}
		\begin{tabular}{cccc}
		\toprule
							& \ac{n}.\ac{sg} 	& \ac{m}.\ac{sg}	& \ac{f}.\ac{sg}   \\
		 						\cmidrule{2-4}
    \ac{nom} & þ-at-ei 	 	& s-a-ei 			& s-ō-ei			\\
    \ac{acc}	& þ-at-ei    	& þ-an-ei  		& þ-ō-ei  		\\
    \ac{dat} & þ-amm-ei 		& þ-amm-ei		& þ-izái-ei 	\\
		\bottomrule
    					& \ac{n}.\ac{pl}	& \ac{m}.\ac{pl}	& \ac{f}.\ac{pl}	\\
						    \cmidrule{2-4}
    \ac{nom} & þ-ō-ei			&	þ-ái-ei			&	þ-ōz-ei			\\
    \ac{acc} & þ-ō-ei 			&	þ-anz-ei		&	þ-ōz-ei			\\
    \ac{dat} & þ-áim-ei		&	þ-áim-ei 		&	þ-áim-ei 		\\
    \bottomrule
		\end{tabular}
\end{table}


\subsection{Modern German}

\begin{table}[h]
	\center
	\caption {Modern German relative pronouns in headless relatives}
		\begin{tabular}{ccc}
		\toprule
							& \ac{inan}	& \ac{an}	\\
								\cmidrule{2-3}
    \ac{nom} & w-as    		& w-er   		\\
    \ac{acc} & w-as    		& w-en   		\\
    \ac{dat} & -  					& w-em    	\\
		\bottomrule
		\end{tabular}
\end{table}

\section{Analysis}

\subsection{Old High German}
In Old High German, proper attraction in headless relatives can be derived from headed relatives. The relative pronoun is the determiner from the main clause. Under a double-headed Cinque-analysis, it is the internal DP that is deleted.




\ex. \ac{acc} instead of \ac{nom}
\ag. unde ne wolden níet besên den mort den dô was geschên\\
 and not wanted not see the murder.\ac{acc} that.\ac{acc} there had happened\\
 `and they didn't want to see the murder that had happened.' \flushfill{(Middle High German, Nibelungenlied 1391,14, \citealt[756]{behaghel1923}, glosses and translation by \citealt[198]{pittner1995})}



    \subsection{Modern German}
In German, inverse attraction in headed relatives can be shown to be very different from inverse attraction in headless relatives. I am not set on an analysis yet. Under a double-headed Cinque-analysis, it is the external DP that is deleted. Grafting is also still an option.


    \subsection{Gothic}
In Gothic, ?



\section{No attraction allowed}
Italian has none. Italian uses its free relative pronoun also in light-headed relative pronouns.


  \section{Bigger picture}
Relative pronoun is a descriptive term. What we analyze as relative pronouns are sometimes wh-elements, sometimes determiners.

Case attraction is also a descriptive term. The constructions are underlyingly very different.
