% !TEX root = thesis.tex

\chapter{Direction of attraction}


  \section{The typology}
Old High German only has proper attraction. Modern German only has inverse attraction. Gothic has both proper and inverse attraction.


\begin{table}[h]\label{tbl:intextgoth}
	\center
	\caption {\tsc{int} vs. \tsc{ext} in Modern and Old High German and Gothic}
	\begin{minipage}{0.6\linewidth}
		\begin{tabularx}{\textwidth}{c *{2}{Y}}
		\toprule
		 								& \tsc{int>ext}				& \tsc{ext>int}				\\
										& inverse attraction	& proper attraction		\\
		\midrule
		Modern German 	& ✔			 							&	*										\\
		Old High German	& *										&	✔										\\
		Gothic					&	✔										&	✔										\\
		\bottomrule
		\end{tabularx}
	\end{minipage}
\end{table}


\subsection{Gothic}

\begin{table}[h]
  \center
  \caption {Case attraction in headless relatives in Gothic}
    \begin{minipage}{\linewidth}
      \begin{tabularx}{\textwidth}{c|Y|Y|Y|Y}
        \toprule
          \diagbox[linecolor=white]{\tsc{int}}{\tsc{ext}}
              & \tsc{[nom]}
              & \tsc{[acc]}
              & \tsc{[dat]}
              & \tsc{[gen]}
              \\ \cmidrule(lr){1-5}
          \tsc{[nom]}
              & \colorbox{LG}{\tsc{nom}}
              & \diagbox[linecolor=white]{?\tsc{nom}}{\colorbox{DG}{\tsc{acc}}}
              & \diagbox[linecolor=white]{?\tsc{nom}}{\colorbox{DG}{\tsc{dat}}}
              & \diagbox[linecolor=white]{?\tsc{nom}}{?\tsc{gen}}
              \\ \cmidrule(lr){1-5}
          \tsc{[acc]}
              & \diagbox[linecolor=white]{\colorbox{DG}{\tsc{acc}}}{?\tsc{nom}}
              &	\colorbox{LG}{\tsc{acc}}
              &	\diagbox[linecolor=white]{?\tsc{acc}}{\colorbox{DG}{\tsc{dat}}}
              &	\diagbox[linecolor=white]{?\tsc{acc}}{\colorbox{DG}{\tsc{gen}}}
              \\ \cmidrule(lr){1-5}
          \tsc{[dat]}
              & \diagbox[linecolor=white]{\colorbox{DG}{\tsc{dat}}}{?\tsc{nom}}
              &	\diagbox[linecolor=white]{\colorbox{DG}{\tsc{dat}}}{?\tsc{acc}}
              & \colorbox{LG}{\tsc{dat}}
              & \diagbox[linecolor=white]{?\tsc{dat}}{?\tsc{gen}}
              \\ \cmidrule(lr){1-5}
          \tsc{[gen]}
              & \diagbox[linecolor=white]{?\tsc{gen}}{?\tsc{nom}}
              & \diagbox[linecolor=white]{\colorbox{DG}{\tsc{gen}}}{?\tsc{acc}}
              & \diagbox[linecolor=white]{?\tsc{gen}}{?\tsc{dat}}
              & \colorbox{LG}{\tsc{gen}}
              \\
        \bottomrule
      \end{tabularx}
    \end{minipage}
\end{table}



\subsection{Old High German}

\ex. \tsc{int:nom}, \tsc{ext:acc}
\a. \tsc{nom} not attested
\bg. ih bibringu fona Juda dhen mina berga chisetzit\\
 I educate von Juda D-\tsc{m/n/f.pl.dat}/\tsc{m.sg.acc} my? mountain order/put\\
 `' \hfill (Old High German, Isidor 34,3, \citealt[761]{behaghel1923})

\ex. \tsc{int:nom}, \tsc{ext:dat}
\a. \tsc{nom} not attested
\bg. aer antuurta demo zaimo sprah\\
he replied\scsub{[dat]} D-\tsc{m.sg.dat} {to him} spoke\scsub{[nom]}\\
`He replied to the one who spoke to him.' \hfill (Old High German, Monsee Fragments 7,24, \citealt[761]{behaghel1923})
\bg. gebe themo ni eigi\\
 give\scsub{[dat]} D--\tsc{m.sg.dat} not posses\scsub{[nom]}\\
 `give to whom that does not have' gebe dem der nicht hat \hfill (Old High German, Otfrid I,24,77, Schrodt,175)

\ex. \tsc{int:nom}, \tsc{ext:gen}
\a. \tsc{nom} not attested
\bg. suachit thes nan sentit\\
 search.\tsc{3.sg}\scsub{gen?} D-\tsc{n.sg.gen} \tsc{3.sg.m.acc} sent\\
 `he searched for who sent him/ der sucht die Sachen dessen der ihn sendet' \hfill (Old High German, Otfrid III,16,21, \citealt[761]{behaghel1923})
\bg. diu habe niemer niht entuot, des der seele schaden si\\
 they have not never {not do} D-\tsc{gen.sg.n} the soul damage is\scsub{[nom]}\\
 `' \hfill (Middle High German, Warn. 2490, \citealt[761]{behaghel1923})

\ex. \tsc{int:acc}, \tsc{ext:nom}
\a. \tsc{acc} not attested
\b. \tsc{nom} not attested

\ex. \tsc{int:acc}, \tsc{ext:dat}
\a. \tsc{acc} not attested
\bg. istû furira Abrâhame, ouh thên man hiar nû zalta?\\
 {are you} {superior to}\scsub{[dat]} Abraham also D-\tsc{d.pl} one here now mentioned\scsub{[acc]}\\
 `are you bigger than Abraham and those people named just now?' \hfill (Old High German, Otfrid III,18,33, \citealt[761]{behaghel1923})

\ex. \tsc{int:acc}, \tsc{ext:gen}
\a. \tsc{acc} not attested
\bg. der bewiset in des er suochte\\
 he directed\scsub{[gen]} him D-\tsc{n.sg.gen} he sought\scsub{[acc]}\\
 `He directed him to what he sought.' \hfill (Middle High German, Iwein 988, \citealt[761]{behaghel1923}), trans. Hartmann von Aue-Portal
\bg. giwisso ni birut ir thero ih irwellu zi mir\\
 bestimmt not belong\scsub{[gen]} \tsc{2.pl.nom} D-\tsc{gen.pl} I choose\scsub{[acc]} to me\\
 `you surely do not belong to those that I choose for myself //sicherlich gehört ihr nicht zu denen die ich mir erwähle' \hfill (Old High German, Otfrid III,22,20, Schrodt,p.175)

\ex. \tsc{int:dat}, \tsc{ext:nom}
\a. \tsc{dat} not attested
\b. \tsc{nom} not attested

\ex. \tsc{int:dat}, \tsc{ext:acc}
\a. \tsc{dat} not attested
\b. \tsc{acc} not attested

\ex. \tsc{int:dat}, \tsc{ext:gen}
\a. \tsc{dat} not attested
\b. \tsc{gen} not attested

\ex. \tsc{int:gen}, \tsc{ext:nom}
\a. \tsc{gen} not attested
\b. \tsc{nom} not attested

\ex. \tsc{int:gen}, \tsc{ext:acc}
\a. \tsc{gen} not attested
\b. \tsc{acc} not attested

\ex. \tsc{int:gen}, \tsc{ext:dat}
\a. \tsc{gen} not attested
\b. \tsc{dat} not attested


Don't know:

\ex. Old High German
\ag. gaat uz diu halt za dem iz forchaufent\\
 \\
 `' \hfill (Old High German, Monsee Fragments 20,14, \citealt[761]{behaghel1923})
\bg. thisiu fon thiu, iru wan ist, siu alla iru libnara santa (ex eo, quod)\\
 \\
 `hæc autem ex eo quod deest illi, totum victum suum quem habuit misit.' \hfill (Old High German, Tatian 118,1, \citealt[761]{behaghel1923})
\bg. thaz iru thiu sin guati nirzigi, thes siu bati\\
 \\
 `' \hfill (Old High German, Otfrid II,8,24, \citealt[761]{behaghel1923})
\bg. thia laz ih themo iz lisit thar\\
 \\
 `' \hfill (Old High German, Otfrid I,19,25, \citealt[761]{behaghel1923})
\bg. noh so neduohti in gnuoge des si habetin\\
 \\
 `' \hfill (Old High German, Notker I,63,29, \citealt[761]{behaghel1923})
\bg. tannoh pito ih tes noh fore ist (id quod)\\
 \\
 `' \hfill (Old High German, Notker 193,19, \citealt[761]{behaghel1923})

So, to sum up:


\begin{table}[h]
  \center
  \caption {Case attraction in headless relatives in Old High German}
    \begin{minipage}{\linewidth}
      \begin{tabularx}{\textwidth}{c|Y|Y|Y|Y}
        \toprule
          \diagbox[linecolor=white]{\tsc{int}}{\tsc{ext}}
              & \tsc{[nom]}
              & \tsc{[acc]}
              & \tsc{[dat]}
              & \tsc{[gen]}
              \\ \cmidrule(lr){1-5}
          \tsc{[nom]}
              & \colorbox{LG}{\tsc{nom}}
              & \diagbox[linecolor=white]{?\tsc{nom}}{\colorbox{DG}{\tsc{acc}}}
              & \diagbox[linecolor=white]{?\tsc{nom}}{\colorbox{DG}{\tsc{dat}}}
              & \diagbox[linecolor=white]{?\tsc{nom}}{\colorbox{DG}{\tsc{gen}}}
              \\ \cmidrule(lr){1-5}
          \tsc{[acc]}
              & \diagbox[linecolor=white]{?\tsc{acc}}{?\tsc{nom}}
              &	\colorbox{LG}{\tsc{acc}}
              &	\diagbox[linecolor=white]{?\tsc{acc}}{\colorbox{DG}{\tsc{dat}}}
              &	\diagbox[linecolor=white]{?\tsc{acc}}{\colorbox{DG}{\tsc{gen}}}
              \\ \cmidrule(lr){1-5}
          \tsc{[dat]}
              & \diagbox[linecolor=white]{?\tsc{dat}}{?\tsc{nom}}
              &	\diagbox[linecolor=white]{?\tsc{dat}}{?\tsc{acc}}
              & \colorbox{LG}{\tsc{dat}}
              & \diagbox[linecolor=white]{?\tsc{dat}}{?\tsc{gen}}
              \\ \cmidrule(lr){1-5}
          \tsc{[gen]}
              & \diagbox[linecolor=white]{?\tsc{gen}}{?\tsc{nom}}
              & \diagbox[linecolor=white]{?\tsc{gen}}{?\tsc{acc}}
              & \diagbox[linecolor=white]{?\tsc{gen}}{?\tsc{dat}}
              & \colorbox{LG}{\tsc{gen}}
              \\
        \bottomrule
      \end{tabularx}
    \end{minipage}
\end{table}



\subsection{Modern German}

\ex. \tsc{int:nom}, \tsc{ext:acc}
\ag. *Ich {lade ein}, w-er mir sympathisch ist.\\
 I invite\scsub{[acc]} W-\tsc{m/f.sg.nom} me nice is\scsub{[nom]}\\
 `I invite who I like.' \hfill \citep[344]{vogel2001}
\bg. *Ich {lade ein}, w-en mir sympathisch ist.\\
 I invite\scsub{[acc]} W-\tsc{m/f.sg.acc} me nice is\scsub{[nom]}\\
 `I invite who I like.' \hfill \citep[344]{vogel2001}

\ex. \tsc{int:nom}, \tsc{ext:dat}
\ag. *Ich vertraue, w-er Hitchcock mag.\\
 I trust\scsub{[dat]} W-\tsc{m/f.sg.nom} Hitchcock likes\scsub{[nom]}\\
 `I trust who likes Hitchcock.' \hfill \citep[345]{vogel2001}
\bg. *Ich vertraue, w-em Hitchcock mag.\\
 I trust\scsub{[dat]} W-\tsc{m/f.sg.dat} Hitchcock likes\scsub{[nom]}\\
 `I trust who likes Hitchcock.' \hfill \citep[345]{vogel2001}

\ex. \tsc{int:nom}, \tsc{ext:gen}
\ag. *Bodo entledigt sich, w-er immer andere Ansichten hat als er.\\
 Bodo rids\scsub{[gen]} self W-\tsc{m/f.sg.nom} ever other opinions has\scsub{[nom]} than he\\
 `Bodo gets rid of whoever has different opinions than he.' \hfill \citep[345]{vogel2001}
\bg. *Bodo entledigt sich, w-essen immer andere Ansichten hat als er.\\
 Bodo rids\scsub{[gen]} self  W-\tsc{m/f.sg.gen} ever other opinions has\scsub{[nom]} than he\\
 `Bodo gets rid of whoever has different opinions than he.' \hfill \citep[345]{vogel2001}

\ex. \tsc{int:acc}, \tsc{ext:nom}
\ag. Uns besucht w-en Maria mag.\\
 Us visits\scsub{[nom]} W-\tsc{m/f.sg.acc} Maria.\tsc{nom} likes\scsub{[acc]}\\
 `Who visits us likes Maria likes.' \hfill \citep[343]{vogel2001}
\bg. *Uns besucht w-er Maria mag.\\
 Us visits\scsub{[nom]} W-\tsc{m/f.sg.nom} Maria.\tsc{nom} likes\scsub{[acc]}\\
 `Who visits us likes Maria likes.' \hfill \citep[343]{vogel2001}

 \ex. \tsc{int:acc}, \tsc{ext:dat}
\ag. *Ich vertraue w-em auch Maria mag. \\
 I trust\scsub{[dat]} W-\tsc{m/f.sg.dat} also Maria likes\scsub{[acc]}.\\
 `I trust whoever Maria also likes.' \hfill \citep[345]{vogel2001}
\bg. *Ich vertraue w-en auch Maria mag. \\
 I trust\scsub{[dat]} W-\tsc{m/f.sg.acc} also Maria likes\scsub{[acc]}.\\
 `I trust whoever Maria also likes.' \hfill \citep[345]{vogel2001}

\ex. \tsc{int:acc}, \tsc{ext:gen}
\ag. *Bodo entledigt sich, w-en immer Henkel nicht mag.\\
 Bodo rids\scsub{[gen]} self W-\tsc{m/f.sg.acc} ever Henkel not likes\scsub{[acc]}\\
 `Bodo gets rid of whoever Henkel does not like.' \hfill \citep[344]{vogel2001}
\bg. *Bodo entledigt sich, w-essen immer Henkel nicht mag.\\
 Bodo rids\scsub{[gen]} self  W-\tsc{m/f.sg.gen} ever Henkel not likes\scsub{[acc]}\\
 `Bodo gets rid of whoever Henkel does not like.' \hfill \citep[344]{vogel2001}

\ex. \tsc{int:dat}, \tsc{ext:nom}
\ag. Uns besucht w-em Maria vertraut.\\
 us visits\scsub{[nom]} W-\tsc{m/f.sg.dat} Maria trusts\scsub{[dat]}\\
 `Who visits us, Maria trusts.' \hfill \citep[343]{vogel2001}
\bg. *Uns besucht w-er Maria vertraut.\\
 us visits\scsub{[nom]} W-\tsc{m/f.sg.dat} Maria trusts\scsub{[dat]}\\
 `Who visits us, Maria trusts.' \hfill \citep[343]{vogel2001}

\ex. \tsc{int:dat}, \tsc{ext:acc}
\ag. Ich {lade ein} w-em auch Maria vertraut. \\
 I invite\scsub{[acc]} W-\tsc{m/f.sg.dat} also Maria trusts\scsub{[dat]}.\\
 `I invite whoever Maria also trusts.' \hfill \citep[344]{vogel2001}
\bg. *Ich {lade ein} w-en auch Maria vertraut. \\
 I invite\scsub{[acc]} W-\tsc{m/f.sg.acc} also Maria trusts\scsub{[dat]}.\\
 `I invite whoever Maria also trusts.' \hfill \citep[344]{vogel2001}

\ex. \tsc{int:dat}, \tsc{ext:gen}
\ag. *Maria hilft, w-essen andere sich entledigen möchten.\\
 Maria helps\scsub{[dat]} W-\tsc{m/f.sg.gen} others self rid\scsub{[gen]} want\\
 `Maria helps whoever others want to get rid of.' \hfill \citep[344]{vogel2001}
\bg. *Maria hilft, w-em andere sich entledigen möchten.\\
 Maria helps\scsub{[dat]} W-\tsc{m/f.sg.dat} others self rid\scsub{[gen]} want\\
 `Maria helps whoever others want to get rid of.' \hfill \citep[344]{vogel2001}

\ex. \tsc{int:gen}, \tsc{ext:nom}
\ag. Uns besucht w-essen Maria sich erfreuen würde.\\
 us visits\scsub{[nom]} W-\tsc{m/f.sg.gen} Maria self {be happy}\scsub{[gen]} would\\
 `Who visits us, Maria would be happy about' \hfill \citep[343]{vogel2001}
\bg. *Uns besucht w-er Maria sich erfreuen würde.\\
 us visits\scsub{[nom]} W-\tsc{m/f.sg.nom} Maria self {be happy}\scsub{[gen]} would\\
 `Who visits us, Maria would be happy about' \hfill \citep[343]{vogel2001}

\ex. \tsc{int:gen}, \tsc{ext:acc}
\ag. Ich {lade ein}, w-essen sich auch Maria erfreuen würde.\\
 I invite\scsub{[acc]} W-\tsc{m/f.sg.gen} self also Maria {be happy}\scsub{[gen]} would.\\
 `I invite whoever also Maria would be happy to meet.' \hfill \citep[344]{vogel2001}
\bg. *Ich {lade ein}, w-en sich auch Maria erfreuen würde.\\
 I invite\scsub{[acc]} W-\tsc{m/f.sg.acc} self also Maria {be happy}\scsub{[gen]} would.\\
 `I invite whoever also Maria would be happy to meet.' \hfill \citep[344]{vogel2001}

\ex. \tsc{int:gen}, \tsc{ext:dat}
\ag. *Bodo entledigt sich, w-em immer Gerhard misstraut.\\
 Bodo rids\scsub{[gen]} self W-\tsc{m/f.sg.dat} ever Gerhard mistrusts\scsub{[dat]}\\
 `Bodo gets rid of whoever Gerhard mistrusts.' \hfill \citep[345]{vogel2001}
\bg. *Bodo entledigt sich, w-essen immer Gerhard misstraut.\\
 Bodo rids\scsub{[gen]} self W-\tsc{m/f.sg.gen} ever Gerhard mistrusts\scsub{[dat]}\\
 `Bodo gets rid of whoever Gerhard mistrusts.' \hfill \citep[345]{vogel2001}



\subsubsection{Summary of the data}

\begin{table}[h]
  \center
  \caption {Case attraction in headless relatives in Modern German}
    \begin{minipage}{\linewidth}
      \begin{tabularx}{\textwidth}{c|Y|Y|Y|Y}
        \toprule
          \diagbox[linecolor=white]{\tsc{int}}{\tsc{ext}}
              & \tsc{[nom]}
              & \tsc{[acc]}
              & \tsc{[dat]}
              & \tsc{[gen]}
              \\ \cmidrule(lr){1-5}
          \tsc{[nom]}
              & \colorbox{LG}{\tsc{nom}}
              & \diagbox[linecolor=white]{*\tsc{nom}}{*\tsc{acc}}
              & \diagbox[linecolor=white]{*\tsc{nom}}{*\tsc{dat}}
              & \diagbox[linecolor=white]{*\tsc{nom}}{*\tsc{gen}}
              \\ \cmidrule(lr){1-5}
          \tsc{[acc]}
              & \diagbox[linecolor=white]{\colorbox{DG}{\tsc{acc}}}{*\tsc{nom}}
              &	\colorbox{LG}{\tsc{acc}}
              &	\diagbox[linecolor=white]{*\tsc{acc}}{*\tsc{dat}}
              &	\diagbox[linecolor=white]{*\tsc{acc}}{*\tsc{gen}}
              \\ \cmidrule(lr){1-5}
          \tsc{[dat]}
              & \diagbox[linecolor=white]{\colorbox{DG}{\tsc{dat}}}{*\tsc{nom}}
              &	\diagbox[linecolor=white]{\colorbox{DG}{\tsc{dat}}}{*\tsc{acc}}
              & \colorbox{LG}{\tsc{dat}}
              & \diagbox[linecolor=white]{*\tsc{dat}}{*\tsc{gen}}
              \\ \cmidrule(lr){1-5}
          \tsc{[gen]}
              & \diagbox[linecolor=white]{\colorbox{DG}{\tsc{gen}}}{*\tsc{nom}}
              & \diagbox[linecolor=white]{\colorbox{DG}{\tsc{gen}}}{*\tsc{acc}}
              & \diagbox[linecolor=white]{\tsc{gen}}{*\tsc{dat}}
              & \colorbox{LG}{\tsc{gen}}
              \\
        \bottomrule
      \end{tabularx}
    \end{minipage}
\end{table}





  \section{Background: relative clause theory}
Standard raising, probably Cinque's double-headed structures


  \section{Shape of relative pronoun}
Old High German has a d-pronoun. Modern German has a wh-pronoun. Gothic has a d-pronoun plus a caseless relativizer.


\subsection{Old High German}

\begin{table}[h]\label{tbl:paradigmohg}
	\center
	\caption {Old High German relative pronouns in headless relatives}
	\begin{minipage}{0.85\linewidth}
		\begin{tabularx}{\textwidth}{c *{6}{Y}}
		\toprule
		\tsc{sg}	& \tsc{f}          & \tsc{m}           & \tsc{n}    \\
		\midrule
		\tsc{nom} & d-iu             & d-ër       	     & d-aȥ				\\
		\tsc{acc}	& d-ea/-ia/(-ie)   & d-ën				       & d-aȥ		    \\
		\tsc{dat}	& d-ëru/-ëro	     & d-ëmu/-ëmo	       & d-ëmu/-ëmo \\
    \tsc{gen} & d-ëra/-ëru/-ëro  & d-ës              & d-ës       \\
		\bottomrule
    \toprule
    \tsc{pl}	& \tsc{f}          & \tsc{m}           & \tsc{n}    \\
    \midrule
    \tsc{nom} & d-eo/-io         &  d-ē/-ea/-ia/-ie  & d-iu/-ei   \\
    \tsc{acc} & d-eo/-io         &  d-ē/-ea/-ia/-ie  & d-iu/-ei   \\
    \tsc{dat} & d-ēm/-ēn         &  d-ēm/-ēn         & d-ēm/-ēn   \\
    \tsc{gen} & d-ëro            &  d-ëro            & d-ëro      \\
    \bottomrule
		\end{tabularx}
	\end{minipage}
\end{table}


\subsection{Gothic}

\begin{table}[h]\label{tbl:paradigmgothic}
	\center
	\caption {Gothic relative pronouns in headless relatives}
	\begin{minipage}{0.7\linewidth}
		\begin{tabularx}{\textwidth}{c *{4}{Y}}
		\toprule
		\tsc{sg}	& \tsc{f}   & \tsc{m}   & \tsc{n}  \\
		\midrule
    \tsc{nom} & s-ō-ei 	  & s-a-ei 		& þ-at-ei	 \\
    \tsc{acc} & þ-ō-ei    & þ-an-ei  	& þ-at-ei  \\
    \tsc{dat} & þ-izái-ei & þ-amm-ei	& þ-amm-ei \\
    \tsc{gen} & þ-izōz-ei & þ-iz-ei		& þ-iz-ei  \\
		\bottomrule
    \toprule
    \tsc{pl}	& \tsc{f} 	& \tsc{m}		& \tsc{n}		\\
    \midrule
    \tsc{nom} &	þ-ōz-ei		&	þ-ái-ei		& þ-ō-ei   	\\
    \tsc{acc} &	þ-ōz-ei		&	þ-anz-ei	& þ-ō-ei   	\\
    \tsc{dat} &	þ-áim-ei	&	þ-áim-ei 	& þ-áim-ei	\\
    \tsc{gen} &	þ-izō-ei	&	þ-izē-ei	& þ-izē-ei 	\\
    \bottomrule
		\end{tabularx}
	\end{minipage}
\end{table}


\subsection{Modern German}

\begin{table}[h]\label{tbl:paradigmg}
	\center
	\caption {Modern German relative pronouns in headless relatives}
	\begin{minipage}{0.4\linewidth}
		\begin{tabularx}{\textwidth}{c *{3}{Y}}
		\toprule
		\tsc{sg}	& \tsc{f/m} & \tsc{n}  \\
		\midrule
    \tsc{nom}  & w-er   	& w-as     \\
    \tsc{acc}  & w-en   	& w-as     \\
    \tsc{dat}  & w-em   	&  		     \\
    \tsc{gen}  & w-essen	& 		     \\
		\bottomrule
		\end{tabularx}
	\end{minipage}
\end{table}

  \section{Analysis}

    \subsection{Old High German}
In Old High German, proper attraction in headless relatives can be derived from headed relatives. The relative pronoun is the determiner from the main clause. Under a double-headed Cinque-analysis, it is the internal DP that is deleted.




\ex. \tsc{acc} instead of \tsc{nom}
\ag. unde ne wolden níet besên den mort den dô was geschên\\
 and not wanted not see the murder.\tsc{acc} that.\tsc{acc} there had happened\\
 `and they didn't want to see the murder that had happened.' \hfill (Middle High German, Nibelungenlied 1391,14, \citealt[756]{behaghel1923}, glosses and translation by \citealt[198]{pittner1995})



    \subsection{Modern German}
In German, inverse attraction in headed relatives can be shown to be very different from inverse attraction in headless relatives. I am not set on an analysis yet. Under a double-headed Cinque-analysis, it is the external DP that is deleted. Grafting is also still an option.


    \subsection{Gothic}
In Gothic, ?



\section{No attraction allowed}
Italian has none. Italian uses its free relative pronoun also in light-headed relative pronouns.


  \section{Bigger picture}
Relative pronoun is a descriptive term. What we analyze as relative pronouns are sometimes wh-elements, sometimes determiners.

Case attraction is also a descriptive term. The constructions are underlyingly very different.
