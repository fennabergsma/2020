% !TEX root = thesis.tex

\chapter{Ellipsis account}

The intuition

\ex.
\a.
\begin{forest} boom
  [\tsc{datP}
      [\tsc{dat}]
      [\tsc{accP}
          [\tsc{acc}]
          [\tsc{nomP}
              [\tsc{nom}]
              [..,roof]
          ]
      ]
  ]
\end{forest}
\b.
\begin{forest} boom
  [\tsc{accP}
      [\tsc{acc}]
      [\tsc{nomP}
          [\tsc{nom}]
          [..,roof]
      ]
  ]
\end{forest}
\b.
\begin{forest} boom
  [\tsc{nomP}
      [\tsc{nom}]
      [..,roof]
  ]
\end{forest}

\section{Elipsis}

Elipsis targets phrases



\section{Phrasal spellout}

Single morphemes spell out phrases




\section{Concretely: Nanosyntax}

\subsection{Basics}

\subsection{Spellout}

\ex. \tbf{The Superset Principle} \citet{starke2009}:\\
A lexically stored tree matches a syntactic node iff the lexically stored tree contains the syntactic node.

\ex. \tbf{The Elsewhere Condition} (\citealt{kiparsky1973}, formulated as in \citealt{caha2020}):\\
When two entries can spell out a given node, the more specific entry wins. Under the Superset Principle governed insertion, the more specific entry is the one which has fewer unused features.

\ex. \tbf{Spellout Algorithm:}\\
Merge F and \label{ex:spellout}
 \a. Spell out FP.
 \b. If (a) fails, attempt movement of the spec of the complement of \tsc{f}, and retry (a).
 \b. If (b) fails, move the complement of \tsc{f}, and retry (a).

When a new match is found, it overrides previous spellouts.

\ex. \tbf{Cyclic Override} \citep{starke2018}:\\
Lexicalisation at a node XP overrides any previous match at a phrase contained in XP.

If the spellout procedure in \ref{ex:spellout} fails, backtracking takes place.

\ex. \tbf{Backtracking} \citep{starke2018}:\\
When spellout fails, go back to the previous cycle, and try the next option for that cycle.\label{ex:backtracking}

If backtracking also does not help, a specifier is constructed.

\ex. \tbf{Spec Formation} \citep{starke2018}:\\
If Merge F has failed to spell out (even after backtracking), try to spawn a new derivation providing the feature F and merge that with the current derivation, projecting the feature F at the top node.\label{ex:specformation}





\phantom{hi}
