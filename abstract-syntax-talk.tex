\documentclass[11pt,hidelinks]{paper}

\usepackage{fenna-files/packages}
\addbibresource{fenna-files/references.bib}
\usepackage{fenna-files/abbreviations}

\begin{document}

\tbf{A typology of case competition in headless relatives}

In case competition in headless relatives two aspects play a role. The first one is which case wins the case competition. It is a crosslinguistically stable fact that this is determined by the case scale in \ref{ex:case-scale-two-patterns-sum}. A case more to the right on the scale wins over a case more to the left on the scale.

\ex. \ac{nom} < \ac{acc} < \ac{dat}\label{ex:case-scale-two-patterns-sum}

This generates the pattern shown in Table \ref{tbl:case-competition-table}. The left column shows the internal case between square brackets. The top row shows the external case between square brackets. The other cells indicate the case of the relative pronoun. When the dative wins over the accusative, the relative pronoun appears in the dative case. When the dative wins over the nominative, the relative pronoun appears in the nominative case. When the accusative wins over the nominative, the relative pronoun appears in the accusative case.

\begin{table}[H]
  \center
  \caption{Relative pronoun follows case competition}
  \begin{tabular}{c|c|c|c}
    \toprule
    \textsubscript{\tsc{int}} \textsuperscript{\tsc{ext}}
           & [\ac{nom}]
           & [\ac{acc}]
           & [\ac{dat}]
           \\ \cmidrule{1-4}
       [\ac{nom}]
           & \ac{nom}
           & \cellcolor{DG}\ac{acc}
           & \cellcolor{DG}\ac{dat}
           \\ \cmidrule{1-4}
       [\ac{acc}]
           & \cellcolor{LG}\ac{acc}
           & \ac{acc}
           & \cellcolor{DG}\ac{dat}
           \\ \cmidrule{1-4}
       [\ac{dat}]
           & \cellcolor{LG}\ac{dat}
           & \cellcolor{LG}\ac{dat}
           & \ac{dat}
           \\
     \bottomrule
  \end{tabular}
    \label{tbl:case-competition-table-marking}
\end{table}

The second aspect that plays a role in headless relatives is whether the internal and the external case are allowed to surface when either of them wins the case competition. This differs across languages. There are four logical possibilities:

\ex.
\a. The non-matching type: the internal and the external case are allowed to surface when either of them wins the case competition\label{ex:int-ext}
\b. The internal only type: only the internal case is allowed to surface when it wins the case competition\label{ex:int-only}
\b. The external only type: only the external case is allowed to surface when it wins the case competition\label{ex:ext-only}
\b. The matching type: neither the internal case nor in the external case is allowed to surface when either of them wins the case competition\label{ex:matching}
\global\let\alph=\oldalph

As far as I am aware, not all of these logical possibilities are attested in natural languages. I discuss the types one by one, and I give example when they are attested. In my description, I refer to the differ gray-marking in Table \ref{tbl:case-competition-table-marking}. The cells marked in light gray are the ones in which the internal case wins the case competition, the cells marked in dark gray are the ones in which the external case wins the case competition, and the unmarked cells are the ones in which the internal and external case match.

Gothic, Old High German and Classical Greek are examples of the non-matching type, in which relative pronouns in the unmarked, light gray and dark gray cells are attested.
Modern German is an example of the internal-only type, in which relative pronouns in the unmarked and light gray cells are grammatical.
To my knowledge, the external-only type is not attested. This would be a language in which relative pronouns in the unmarked and the dark gray cells are grammatical.
Polish is an example of a language of the matching type, in which relative pronoun in only in the unmarked cells are grammatical.

If time permits, I either show a language that

\end{document}
