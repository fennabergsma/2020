% !TEX root = thesis.tex

\chapter{Languages with case competition}\label{ch:typology}

In Part \ref{part:case-facts} of this dissertation, I discussed a first aspect of case competition in headless relatives. There is a fixed scale that determines which case wins the case competition. This is the same case scale crosslinguistically. I repeat the case scale from Chapter \ref{ch:recurring} in \ref{ex:case-scale-two-patterns}.

\ex. \ac{nom} < \ac{acc} < \ac{dat}\label{ex:case-scale-two-patterns}

In Chapter \ref{ch:decomposition} within Part \ref{part:case-facts}, I argued that a cumulative case decomposition can derive the case scale. This does not only hold for case competition in headless relatives, but also for syncretism patterns and morphological case containment patterns. In a cumulative case composition, the scale in \ref{ex:case-scale-two-patterns} can be interpreted as follows: the accusative contains all features the nominative contains plus one more. Similarly, the dative contains all features the accusative contains plus one. Therefore, the dative can be considered more complex than the accusative, and the accusative more complex than the nominative. In line with that, I refer to cases more to the right on the case scale as being more complex cases than cases more to the left on the scale.

This part of the dissertation, Part \ref{part:variation}, focuses on a second aspect to headless relatives. This aspect is not stable crosslinguistically, but it differs across languages. Languages differ in whether they allow the internal case (the case from the relative clause) and the external case (the case from the main clause) to surface when either of them wins the case competition. Metaphorically speaking, even though a case wins the case competition, it is a second matter whether it is allowed to come forward as a winner. Four patterns are logically possible for languages: (1) the internal case and the external case are allowed to surface when either of them wins the case competition, (2) only the internal case is allowed to surface when it wins the case competition, and the external case is not, (3) only the external case is allowed to surface when it wins the case competition, and the internal case is not, (4) neither the internal case nor the external case is allowed to surface when either of them wins the competition.\footnote{
On the surface, the last pattern cannot be distinguished from a language that does not have case competition and does not allow for any case mismatches. I come back to this matter in \ref{sec:possible-patterns}, where I argue that there actually is case competition in play.
}
I show in this chapter that one of these logically possible patterns is not attested in any natural language.

In this dissertation I discuss languages of which headless relatives have been described in the literature. As I write about case competition, I only focus on languages that morphologically distinguish between case, specifically the nominative, the accusative and the dative. By no means do I claim that my language sample is representative for the languages of the world. However, they build on independently established facts, which are the case scale from Chapter \ref{ch:recurring} and the subset requirement of the external head, to be discussed in Chapter \ref{part:deriving}. Therefore, I predict that my generalizations hold for all natural languages.

The next section introduces the patterns that are logically possible with case competition. In Section \ref{sec:pattern-i} to Section \ref{sec:pattern-iv}, I discuss the patterns one by one, and I give examples when the pattern is attested. In Section \ref{sec:without-case-competition}, I make a sidestep to languages that do not show any case competition, and I give a typology of headless relatives.


\section{Four possible patterns}\label{sec:possible-patterns}

Case competition has two aspects. The first aspect is the topic of Part \ref{part:case-facts} of the dissertation. It concerns which case wins the case competition. This is decided by the same case scale for all languages. The second aspects is the topic of Part \ref{part:variation} of the dissertation. This one concerns whether the case that wins the case competition is actually allowed to surface. It namely differs per language whether it allows the internal or the external case to do so.

Metaphorically, the second aspect can be described as a language-specific approval committee. The committee learns (from the first aspect) which case wins the case competition. Then it can either approve this case or not approve it. This approval happens based on where the winning case comes from: from inside of the relative clause (internal) or from outside of the relative clause (external). It is determined per language whether it approves the internal case, the external case, both of them or none of them. The approval committee can only approve the winner of the competition or deny it, it cannot propose an alternative winner. In this metaphor, the approval of the committee means that a particular case is allowed to surface. When the case is not allowed to surface, the headless relative as a whole is ungrammatical.

Taking this all together, there are four patterns possible in languages. First, the internal case and the external case are allowed to surface. Second, only the internal case is allowed to surface, and the external case is not. Third, only the external case is allowed to surface, and the internal case is not. Fourth, neither the internal case nor the external case is allowed to surface when either of them wins the competition. In what follows, I introduce these four possible patterns.

The first possible pattern is that of a language that allows the internal case and the external case to surface when either of them wins the case competition. I call this the unrestricted type of language \citep[just as cf.][]{grosu1987,cinqueforthcoming}: the internal and external case do not need to match. The pattern might look familiar, because it is the one that Gothic has, which I discussed in Chapter \ref{ch:recurring}. Table \ref{tbl:case-competition-int-ext} (repeated from Table \ref{tbl:summary-gothic-simple}) illustrates what the pattern for such a language looks like.

The left column shows the internal case between square brackets. The top row shows the external case between square brackets. The other cells indicate the case of the relative pronoun.
The top-left to bottom-right diagonal corresponds to the examples in which the internal and external case match. The three cells in the bottom-left corner, marked in light gray, are the situations in which the internal case surfaces when it wins the competition. The three cells in the top-right corner, marked in dark gray, are the situations in which the external case surfaces when it wins the competition.
All these instances are grammatical.

\begin{table}[ht]
  \center
  \caption{Internal and external case allowed}
  \begin{tabular}{c|c|c|c}
    \toprule
    \textsubscript{\ac{int}} \textsuperscript{\ac{ext}}
           & [\ac{nom}]
           & [\ac{acc}]
           & [\ac{dat}]
           \\ \cmidrule{1-4}
       [\ac{nom}]
           & \ac{nom}
           & \cellcolor{DG}\ac{acc}
           & \cellcolor{DG}\ac{dat}
           \\ \cmidrule{1-4}
       [\ac{acc}]
           & \cellcolor{LG}\ac{acc}
           & \ac{acc}
           & \cellcolor{DG}\ac{dat}
           \\ \cmidrule{1-4}
       [\ac{dat}]
           & \cellcolor{LG}\ac{dat}
           & \cellcolor{LG}\ac{dat}
           & \ac{dat}
           \\
     \bottomrule
  \end{tabular}
    \label{tbl:case-competition-int-ext}
\end{table}

The second possible pattern is that of a language that allows the internal case to surface when it wins the case competition, but it does not allow the external case to do so. In this type of language, the internal case gets to surface when it is more complex than the external one. When the external case is more complex, it is not allowed to surface, and the headless relative construction is ungrammatical. I call this the internal-only type of language: the internal and external case do not need to match, but only the internal case is allowed to surface as a winner.

Table \ref{tbl:case-competition-only-int} illustrates what the pattern for such a language looks like. Compared to the unrestricted type, it has three cells in which there is no grammatical relative pronoun.
The top-left to bottom-right diagonal corresponds to the examples in which the internal and external case match.
The three cells in the bottom-left corner, marked in light gray, are the situations in which the internal case surfaces when it wins the competition.
Just as in the unrestricted type, these six instances are grammatical.
The three cells in the top-right corner, marked in dark gray, are the situations in which the external case surfaces when it wins the competition. These instances are not grammatical for this type of language. The reasoning behind that is that the language does not allow the external case to surface when it wins the case competition.


\begin{table}[ht]
  \center
  \caption{Only internal case allowed}
  \begin{tabular}{c|c|c|c}
    \toprule
    \textsubscript{\ac{int}} \textsuperscript{\ac{ext}}
           & [\ac{nom}]
           & [\ac{acc}]
           & [\ac{dat}]
           \\ \cmidrule{1-4}
       [\ac{nom}]
           & \ac{nom}
           & \cellcolor{DG}*
           & \cellcolor{DG}*
           \\ \cmidrule{1-4}
       [\ac{acc}]
           & \cellcolor{LG}\ac{acc}
           & \ac{acc}
           & \cellcolor{DG}*
           \\ \cmidrule{1-4}
       [\ac{dat}]
           & \cellcolor{LG}\ac{dat}
           & \cellcolor{LG}\ac{dat}
           & \ac{dat}
           \\
     \bottomrule
  \end{tabular}
    \label{tbl:case-competition-only-int}
\end{table}

The third possible pattern is that of a language that allows the external case to surface when it wins the case competition, but it does not allow the internal case to do so. In this type of language, only the external case gets to surface when it is more complex. When the internal case is more complex, it is not allowed to surface, and the headless relative construction is ungrammatical. I call this the external-only type of language: the internal and external case do not need to match, but only the external case is allowed to surface as a winner.

Table \ref{tbl:case-competition-only-ext} illustrates what the pattern for such a language looks like. Comparing this pattern to the second one, the ungrammatical cells are here the three on the other side of the diagonal.
The top-left to bottom-right diagonal corresponds to the examples in which the internal and external case match. Just as in the unrestricted type and the `unrestricted --- internal-only' type, these instances are grammatical.
The three cells in the bottom-left corner, marked in light gray, are the situations in which the internal case surfaces when it wins the competition. Unlike in the unrestricted type and the `unrestricted --- internal-only' type, these instances are not grammatical for this type of language. The reasoning behind that is that the language does not allow the internal case to surface when it wins the case competition.
The three cells in the top-right corner, marked in dark gray, are the situations in which the external case surfaces when it wins the competition. Just as in the unrestricted type but unlike in the `unrestricted --- internal-only' type, these instances are grammatical.

\begin{table}[ht]
  \center
  \caption{Only external case allowed}
  \begin{tabular}{c|c|c|c}
    \toprule
    \textsubscript{\ac{int}} \textsuperscript{\ac{ext}}
           & [\ac{nom}]
           & [\ac{acc}]
           & [\ac{dat}]
           \\ \cmidrule{1-4}
       [\ac{nom}]
           & \ac{nom}
           & \cellcolor{DG}\ac{acc}
           & \cellcolor{DG}\ac{dat}
           \\ \cmidrule{1-4}
       [\ac{acc}]
           & \cellcolor{LG}*
           & \ac{acc}
           & \cellcolor{DG}\ac{dat}
           \\ \cmidrule{1-4}
       [\ac{dat}]
           & \cellcolor{LG}*
           & \cellcolor{LG}*
           & \ac{dat}
           \\
     \bottomrule
  \end{tabular}
    \label{tbl:case-competition-only-ext}
\end{table}

The fourth possible pattern is that of a language that allows neither the internal case nor the external case to surface when either of them wins the competition. In other words, when the internal and the external case differ, there is no grammatical headless relative construction possible. Only when there is a tie, i.e. when the internal and external case match, there is a grammatical result. I call this the matching type of language: the internal and external case need to match.

Table \ref{tbl:case-competition-none} illustrates what the pattern for such a language looks like.
The top-left to bottom-right diagonal corresponds to the examples in which the internal and external case match. Just as in the other three pattern, these instances are grammatical.
The three cells in the bottom-left corner, marked in light gray, are the situations in which the internal case surfaces when it wins the competition. Just as the `unrestricted --- external-only' type, but unlike the unrestricted type and the `unrestricted --- internal-only' type, these instances are not grammatical for this type of language.
The three cells in the top-right corner, marked in dark gray, are the situations in which the external case surfaces when it wins the competition. Just as the `unrestricted --- internal-only' type, but unlike the unrestricted type and the `unrestricted --- external-only' pattern, these instances are not grammatical for this type of language. The reasoning behind the ungrammaticality of these six cells is that the language allows neither the internal case nor the external case to surface when either of them wins the competition.

On the surface, this pattern cannot be distinguished from a pattern that does not have case competition and does not allow for any case mismatches. I understand `a language with case competition' as a language that compares the internal and external case in its headless relatives. If the internal and external case are not compared in this type of language, it would be unclear why the diagonal is different from all the other cells. The source of ungrammaticality for the cells in Table \ref{tbl:case-competition-none} can only come from the comparing the internal and external case and concluding that the internal case and the external case differ. The grammaticality of the diagonal follows from the conclusion that the internal and the external case match. In Section \ref{sec:without-case-competition} I discuss languages in which the internal and external case are not compared to each other.

\begin{table}[ht]
  \center
  \caption{Only matching allowed}
  \begin{tabular}{c|c|c|c}
    \toprule
    \textsubscript{\ac{int}} \textsuperscript{\ac{ext}}
           & [\ac{nom}]
           & [\ac{acc}]
           & [\ac{dat}]
           \\ \cmidrule{1-4}
       [\ac{nom}]
           & \ac{nom}
           & \cellcolor{DG}*
           & \cellcolor{DG}*
           \\ \cmidrule{1-4}
       [\ac{acc}]
           & \cellcolor{LG}*
           & \ac{acc}
           & \cellcolor{DG}*
           \\ \cmidrule{1-4}
       [\ac{dat}]
           & \cellcolor{LG}*
           & \cellcolor{LG}*
           & \ac{dat}
           \\
     \bottomrule
  \end{tabular}
    \label{tbl:case-competition-none}
\end{table}

In this chapter I show that three of these four patterns I introduced are attested crosslinguistically. Section \ref{sec:pattern-i} shows that the unrestricted type, in which either the internal case or the external case can surface, is exemplified by Gothic (repeated from Chapter \ref{ch:recurring}) and by Old High German. The `unrestricted --- internal-only' type, in which only the internal case can surface, is illustrated by Modern German in Section \ref{sec:pattern-ii}. To my knowledge, there is no language in which only the external case can surface when it wins the case competition. This is discussed in \ref{sec:pattern-iii}. Section \ref{sec:pattern-iv} shows a language that only allows the case to surface when there is a tie, i.e. when the internal and external case match, namely Polish.

\section{Internal and external case allowed}\label{sec:pattern-i}

This section discusses the situation in which the internal case and the external case are allowed to surface when either of them wins the case competition. I repeat the pattern from Section \ref{sec:possible-patterns} in Table \ref{tbl:case-competition-int-ext-repeated}.

\begin{table}[ht]
  \center
  \caption{Internal and external case allowed (repeated)}
  \begin{tabular}{c|c|c|c}
    \toprule
    \textsubscript{\ac{int}} \textsuperscript{\ac{ext}}
           & [\ac{nom}]
           & [\ac{acc}]
           & [\ac{dat}]
           \\ \cmidrule{1-4}
       [\ac{nom}]
           & \ac{nom}
           & \ac{acc}
           & \ac{dat}
           \\ \cmidrule{1-4}
       [\ac{acc}]
           & \ac{acc}
           & \ac{acc}
           & \ac{dat}
           \\ \cmidrule{1-4}
       [\ac{dat}]
           & \ac{dat}
           & \ac{dat}
           & \ac{dat}
           \\
     \bottomrule
  \end{tabular}
    \label{tbl:case-competition-int-ext-repeated}
\end{table}

Two examples of languages that show this pattern are Gothic and Old High German. In this section, I repeat the summary of the findings from Gothic (from Chapter \ref{ch:recurring}), and I present the data for Old High German, which is the result of my own research.

In Chapter \ref{ch:recurring}, I discussed case competition in Gothic headless relatives, based on the work of \citet{harbert1978}. I repeat the results from Section \ref{sec:pattern-rels} in Table \ref{tbl:summary-gothic-repeated}.
In Gothic, the relative pronoun is allowed to surface in the internal case and the external case. The top-left to bottom-right diagonal corresponds to the examples in which the internal and external case match. The three cells in the bottom-left corner, marked in light gray, are the situations in which the internal case surfaces when it wins the competition. The three cells in the top-right corner, marked in dark gray, are the situations in which the external case surfaces when it wins the competition.
 All these instances are grammatical. The examples corresponding to the cells in Table \ref{tbl:summary-gothic-repeated} can be found in Section \ref{sec:pattern-rels}.

\begin{table}[ht]
  \center
  \caption{Summary of Gothic headless relatives (repeated)}
    % !TEX root = ../thesis.tex

\begin{tabular}{c|c|c|c}
  \toprule
      \textsubscript{\ac{int}} \textsuperscript{\ac{ext}}
        & [\ac{nom}]
        & [\ac{acc}]
        & [\ac{dat}]
        \\ \cmidrule{1-4}
    [\ac{nom}]
        & \ac{nom}
        & \ac{acc}
        & \ac{dat}
        \\ \cmidrule{1-4}
    [\ac{acc}]
        & \ac{acc}
        & \ac{acc}
        & \ac{dat}
        \\ \cmidrule{1-4}
    [\ac{dat}]
        & \ac{dat}
        & (\ac{dat})
        & \ac{dat}
        \\
  \bottomrule
\end{tabular}

    \label{tbl:summary-gothic-repeated}
\end{table}

Old High German is another instance of a language in which the relative pronoun is allowed to surface in the internal case and the external case. This conclusion follows from my own research of the texts `Der althochdeutsche Isidor', `The Monsee fragments', `Otfrid's Evangelienbuch' and `Tatian' in ANNIS \citep{krause2016}.\footnote{
Old High German is widely discussed in the literature because of its case attraction in headed relatives \citep[cf.][]{pittner1995}, a phenomenon that seems related to case competition in headless relatives (see Section \ref{sec:attraction} for why attraction is not further discussed in this dissertation).
A common observation is that case attraction in headed relatives in Old High German adheres to the case scale. The same is claimed for headless relatives.
What, to my knowledge, has not been studied systematically is whether Old High German headless relatives allow the internal case and the external case to surface when either of them wins the case competition. This is what I investigated in my work.
}
The examples follow the spelling and the detailed glosses in ANNIS. The translations are my own.

First I discuss examples in which the internal and the external case match, and then examples in which they differ. If the internal case and the external case are identical, so there is a tie, the relative pronoun simply surfaces in that case. I illustrate this for the nominative, the accusative and the dative.

Consider the example in \ref{ex:ohg-nom-nom}, in which the internal nominative case competes against the external nominative case.
The internal case is nominative, as the predicate \tit{senten} `to send' takes nominative subjects.
The external case is nominative as well, as the predicate \tit{queman} `to come' also takes nominative subjects.
The relative pronoun \tit{dher} `\tsc{rp}.\ac{sg}.\ac{m}.\ac{nom}' appears in the internal and external case: the nominative.

\exg. quham \tbf{dher} \tbf{chisendit} \tbf{scolda} \tbf{uuerdhan}\\
 come.\ac{pst}.3\ac{sg}\scsub{[nom]} \tsc{rp}.\ac{sg}.\ac{m}.\ac{nom} send.\ac{pst}.\ac{ptcp}\scsub{[nom]} should.\ac{pst}.3\ac{sg} become.\ac{inf}\\
 `the one, who should have been sent, came' \flushfill{Old High German, \ac{isid} 35:5}\label{ex:ohg-nom-nom}

Consider the example in \ref{ex:ohg-acc-acc}, in which the internal accusative case competes against the external accusative case.
The internal case is accusative, as the predicate \tit{quedan} `to speak' takes accusative objects.
The external case is accusative as well, as the predicate \tit{gihoren} `to listen to' also takes accusative objects.
The relative pronoun \tit{thiu} `\tsc{rp}.\ac{pl}.\ac{n}.\ac{acc}' appears in the internal and external case: the accusative.

\exg. gihortut ir \tbf{thiu} \tbf{ih} \tbf{íu} \tbf{quad}\\
 listen.\ac{pst}.2\ac{pl}\scsub{[acc]} 2\ac{pl}.\ac{nom} \tsc{rp}.\ac{pl}.\ac{n}.\ac{nom} 1\ac{sg}.\ac{nom} 2\ac{pl}.\ac{dat} speak.\ac{pst}.1\ac{sg}\scsub{[acc]}\\
 `you listened to those things, that I said to you' \flushfill{Old High German, \ac{tatian} 165:6}\label{ex:ohg-acc-acc}

Consider the example in \ref{ex:os-dat-dat}, in which the internal dative case competes against the external dative case.\footnote{
I could not find such an instance for this situation in any of the Old High German texts. This example comes from the `Heliand', an Old Saxon text written around the same time as the Old High German works I give examples from. Old Saxon is linguistically speaking the closest relative of Old High German.
}
The internal case is dative, as the predicate \tit{willian} `to wish' takes dative objects.
The external case is dative as well, as the predicate \tit{seggian} `to say' takes dative indirect objects.
The relative pronoun \tit{them} `\tsc{rp}.\ac{pl}.\ac{m}.\ac{dat}' appears in the internal and external case: the dative.

\exg. sagda \tbf{them} \tbf{siu} \tbf{uuelda}\\
 say.\ac{pst}.3\ac{sg}\scsub{[dat]} \tsc{rp}.\ac{pl}.\ac{m}.\ac{dat} 3\ac{sg}.\ac{f}.\ac{nom} wish.\ac{pst}.3\ac{sg}\scsub{[dat]}\\
 `she said to those, whom she wished for' \flushfill{Old Saxon, \ac{hel} 4:293}\label{ex:os-dat-dat}

These findings can be summarized as in Table \ref{tbl:summary-ohg-matching}. The top-left to bottom-right diagonal corresponds to the examples I have given so far in which the internal and external case match. The nominative marked in light gray corresponds to \ref{ex:ohg-nom-nom}, in which the internal nominative case competes against the external nominative case, and the relative pronoun surfaces in the nominative case. The accusative marked in dark gray corresponds to \ref{ex:ohg-acc-acc}, in which the internal accusative case competes against the external accusative case, and the relative pronoun surfaces in the accusative case. The unmarked dative corresponds to \ref{ex:os-dat-dat}, in which the internal dative case competes against the external dative case, and the relative pronoun surfaces in the dative case.

\begin{table}[ht]
  \center
  \caption{Old High German headless relatives (matching)}
  \begin{tabular}{c|c|c|c}
    \toprule
     \textsubscript{\ac{int}} \textsuperscript{\ac{ext}}
          & [\ac{nom}]
          & [\ac{acc}]
          & [\ac{dat}]
          \\ \cmidrule{1-4}
      [\ac{nom}]
          & \colorbox{LG}{\ac{nom}}
          &
          &
          \\ \cmidrule{1-4}
      [\ac{acc}]
          &
          & \colorbox{DG}{\ac{acc}}
          &
          \\ \cmidrule{1-4}
      [\ac{dat}]
          &
          &
          & (\ac{dat})
          \\
    \bottomrule
  \end{tabular}
    \label{tbl:summary-ohg-matching}
\end{table}

In Table \ref{tbl:summary-ohg-matching}, six cells remain empty. These are the cases in which the internal and the external case differ. In the remainder of this section, I discuss them one by one.

I start with the competition between the accusative and the nominative. Following the case scale, the relative pronoun appears in the accusative case and never in nominative. As Old High German allows the internal and external case to surface, the accusative surfaces when it is the internal case and when it is the external case.

Consider the example in \ref{ex:ohg-nom-acc}. In this example, the internal accusative case competes against the external nominative case.
The internal case is accusative, as the predicate \tit{zellen} `to tell' takes accusative objects.
The external case is nominative, as the predicate \tit{sin} `to be' takes nominative objects.
The relative pronoun \tit{then} `\tsc{rp}.\ac{sg}.\ac{m}.\ac{acc}' appears in the internal case: the accusative. The relative pronoun is marked in bold, just as the relative clause, showing that the relative pronoun patterns with the relative clause.
Examples in which the internal case is accusative, the external case is nominative and the relative pronoun appears in the nominative case are unattested.

\exg. Thíz ist \tbf{then} \tbf{sie} \tbf{zéllent}\\
\ac{dem}.\ac{sg}.\ac{n}.\ac{nom} be.\ac{pres}.3\ac{sg}\scsub{[nom]} \tsc{rp}.\ac{sg}.\ac{m}.\ac{acc} 3\ac{pl}.\ac{m}.\ac{nom} tell.\ac{pres}.3\ac{pl}\scsub{[acc]}\\
`this is the one whom they talk about' \flushfill{Old High German, \ac{otfrid} III 16:50}\label{ex:ohg-nom-acc}

Consider the example in \ref{ex:ohg-acc-nom}. In this example, the internal nominative case competes against the external accusative case.
The internal case is nominative, as the predicate \tit{gisizzen} `to possess' takes nominative subjects.
The external case is accusative, as the predicate \tit{bibringan} `to create' takes accusative objects.
The relative pronoun \tit{dhen} `\tsc{rp}.\ac{sg}.\ac{m}.\ac{acc}' appears in the external case: the accusative. The relative pronoun is not marked in bold, just as the main clause, showing that the relative pronoun patterns with the main clause.\footnote{
At the end of this section I discuss a counterexample to the case scale, in which the internal case is nominative, the external case is accusative, and the relative pronoun appears in the nominative case.
}

\exg. ih bibringu fona iacobes samin endi fona iuda dhen \tbf{mina} \tbf{berga} \tbf{chisitzit}\\
1\ac{sg}.\ac{nom} {create}.\ac{pres}.1\ac{sg}\scsub{[acc]} of Jakob.\ac{gen} seed.\ac{sg}.\ac{dat} and of Judah.\ac{dat} \tsc{rp}.\ac{sg}.\ac{m}.\ac{acc} my.\ac{acc}.\ac{m}.\ac{pl} mountain.\ac{acc}.\ac{pl} possess.\ac{pres}.3\ac{sg}\scsub{[nom]}\\
`I create of the seed of Jacob and of Judah the one, who possess my mountains' \flushfill{Old High German, \ac{isid} 34:3}\label{ex:ohg-acc-nom}

The two examples in which the nominative and the accusative compete are highlighted in Table \ref{tbl:summary-old-high-german-nom-acc}. The light gray marking corresponds to \ref{ex:ohg-nom-acc}, in which the internal accusative wins over the external nominative, and the relative pronoun surfaces in the accusative case. The dark gray marking corresponds to \ref{ex:ohg-acc-nom}, in which the external accusative wins over the internal nominative, and the relative pronoun surfaces in the accusative case.

\begin{table}[ht]
  \center
  \caption{Old High German headless relatives (\ac{nom} --- \ac{acc})}
  \begin{tabular}{c|c|c|c}
    \toprule
        \textsubscript{\ac{int}} \textsuperscript{\ac{ext}}
          & [\ac{nom}]
          & [\ac{acc}]
          & [\ac{dat}]
          \\ \cmidrule{1-4}
      [\ac{nom}]
          & \ac{nom}
          & \cellcolor{DG}\ac{acc}
          &
          \\ \cmidrule{1-4}
      [\ac{acc}]
          & \cellcolor{LG}\ac{acc}
          & \ac{acc}
          &
          \\ \cmidrule{1-4}
      [\ac{dat}]
          &
          &
          & (\ac{dat})
          \\
    \bottomrule
  \end{tabular}
    \label{tbl:summary-old-high-german-nom-acc}
\end{table}

I continue with the competition between the dative and the nominative. Following the case scale, the relative pronoun appears in the dative case and never in nominative. As Old High German allows the internal and the external case to surface, the dative surfaces when it is the internal case and when it is the external case.

Consider the example in \ref{ex:ohg-nom-dat}. In this example, the internal dative case competes against the external nominative case.
The internal case is dative, as the predicate \tit{forlazan} `to read' takes dative indirect objects.
The external case is nominative, as the predicate \tit{minnon} `to love' takes nominative subjects.
The relative pronoun \tit{themo} `\tsc{rp}.\ac{sg}.\ac{m}.\ac{dat}' appears in the internal case: the dative. The relative pronoun is marked in bold, just as the relative clause, showing that the relative pronoun patterns with the relative clause.
Examples in which the internal case is dative, the external case is nominative and the relative pronoun appears in the nominative case are unattested.

\exg. \tbf{themo} \tbf{min} \tbf{uuirdit} \tbf{forlazan}, min minnot\\
\tsc{rp}.\ac{sg}.\ac{m}.\ac{dat} less become.\ac{pres}.3\ac{sg} read.\ac{inf}\scsub{[dat]} less love.\ac{pres}.3\ac{sg}\scsub{[nom]}\\
`to whom less is read, loves less' \flushfill{Old High German, \ac{tatian} 138:13}\label{ex:ohg-nom-dat}

Consider the example in \ref{ex:ohg-dat-nom}. In this example, the internal nominative case competes against the external dative case.
The internal case is nominative, as the predicate \tit{sprehhan} `to speak' takes nominative subjects.
The external case is dative, as the predicate \tit{antwurten} `to reply' takes dative objects.
The relative pronoun \tit{demo} `\tsc{rp}.\ac{sg}.\ac{m}.\ac{dat}' appears in the external case: the dative. The relative pronoun is not marked in bold, just as the main clause, showing that the relative pronoun patterns with the main clause.
Examples in which the internal case is nominative, the external case is dative and the relative pronoun appears in the nominative case are unattested.

\exg. enti aer {ant uurta} demo \tbf{zaimo} \tbf{sprah}\\
and 3\ac{sg}.\ac{m}.\ac{nom} reply.\ac{pst}.3\ac{sg}\scsub{[dat]} \tsc{rp}.\ac{sg}.\ac{m}.\ac{dat} {to 3\ac{sg}.\ac{m}.\ac{dat}} speak.\ac{pst}.3\ac{sg}\scsub{[nom]}\\
`and he replied to the one who spoke to him' \flushfill{Old High German, \ac{mons} 7:24, adapted from \pgcitealt{pittner1995}{199}}\label{ex:ohg-dat-nom}

The two examples in which the nominative and the dative compete are highlighted in Table \ref{tbl:summary-old-high-german-nom-dat}. The light gray marking corresponds to \ref{ex:ohg-nom-dat}, in which the internal dative wins over the external nominative, and the relative pronoun surfaces in the dative case. The dark gray marking corresponds to \ref{ex:ohg-dat-nom}, in which the external dative wins over the internal nominative, and the relative pronoun surfaces in the dative case.

\begin{table}[ht]
  \center
  \caption{Old High German headless relatives (\ac{nom} --- \ac{dat})}
  \begin{tabular}{c|c|c|c}
    \toprule
        \textsubscript{\ac{int}} \textsuperscript{\ac{ext}}
          & [\ac{nom}]
          & [\ac{acc}]
          & [\ac{dat}]
          \\ \cmidrule{1-4}
      [\ac{nom}]
          & \ac{nom}
          & \ac{acc}
          & \cellcolor{DG}\ac{dat}
          \\ \cmidrule{1-4}
      [\ac{acc}]
          & \ac{acc}
          & \ac{acc}
          &
          \\ \cmidrule{1-4}
      [\ac{dat}]
          & \cellcolor{LG}\ac{dat}
          &
          & (\ac{dat})
          \\
    \bottomrule
  \end{tabular}
    \label{tbl:summary-old-high-german-nom-dat}
\end{table}

I end with the competition between the dative and the accusative. Following the case scale, the relative pronoun appears in the dative case and never in accusative. As Old High German allows the internal and the external case to surface, the dative surfaces when it is the internal case and when it is the external case.

Consider the example in \ref{ex:ohg-acc-dat}. In this example, the internal dative case competes against the external accusative case.
The internal case is dative, as the predicate \tit{zawen} `to tell' takes dative subjects.
The external case is accusative, as the predicate \tit{weizan} `to know' takes accusative objects.
The relative pronoun \tit{thémo} `\tsc{rp}.\ac{sg}.\ac{m}.\ac{dat}' appears in the external case: the dative. The relative pronoun is marked in bold, just as the relative clause, showing that the relative pronoun patterns with the relative clause.
Examples in which the internal case is accusative, the external case is dative and the relative pronoun appears in the accusative case are unattested.

\exg. weiz \tbf{thémo} \tbf{ouh} \tbf{baz} \tbf{záweta}\\
know.1\ac{sg}\scsub{[acc]} \tsc{rp}.\ac{sg}.\ac{m}.\ac{dat} also better manage.\ac{pst}.3\ac{sg}\scsub{[dat]}\\
`I know the one who also managed it better' \flushfill{Old High German, \ac{otfrid} V 5:5}\label{ex:ohg-acc-dat}

%sunta	ni	furliez	themo	ih	mit	rehtu	scolta (ext:dat, int:nom)

Consider the example in \ref{ex:ohg-dat-acc}. In this example, the internal accusative case competes against the external dative case.
The internal case is accusative, as the predicate \tit{zellen} `to tell' takes accusative objects.
The external case is dative, as the comparative of the adjective \tit{furiro} `great' takes dative objects.
The relative pronoun \tit{thên} `\tsc{rp}.\ac{pl}.\ac{m}.\ac{dat}' appears in the external case: the dative. The relative pronoun is not marked in bold, just as the main clause, showing that the relative pronoun patterns with the main clause.
Examples in which the internal case is accusative, the external case is dative and the relative pronoun appears in the accusative case are unattested.

\exg. bis -tú nu {zi wáre} furira Ábrahame? ouh thén \tbf{man} \tbf{hiar} \tbf{nu} \tbf{zálta}\\
be.\ac{pres}.2\ac{sg} -2\ac{sg}.\ac{nom} now truly {great}.\ac{cmpr}\scsub{[dat]} Abraham.\ac{dat} and \tsc{rp}.\ac{pl}.\ac{m}.\ac{dat} one.\ac{nom}.\ac{m}.\ac{sg} here now tell.\ac{pst}.3\ac{sg}\scsub{[acc]}\\
`are you now truly greater than Abraham? and than those, who one talked about here now' \flushfill{Old High German, \ac{otfrid} III 18:33}\label{ex:ohg-dat-acc}

%sagda them siu welda = she said to whom she wished

The two examples in which the accusative and the dative compete are highlighted in Table \ref{tbl:summary-old-high-german-acc-dat}. The light gray marking corresponds to \ref{ex:ohg-acc-dat}, in which the internal dative wins over the external accusative, and the relative pronoun surfaces in the dative case. The dark gray marking corresponds to \ref{ex:ohg-dat-acc}, in which the external dative wins over the internal accusative, and the relative pronoun surfaces in the dative case.

\begin{table}[ht]
  \center
  \caption{Old High German headless relatives (\ac{acc} --- \ac{dat})}
  \begin{tabular}{c|c|c|c}
    \toprule
        \textsubscript{\ac{int}} \textsuperscript{\ac{ext}}
          & [\ac{nom}]
          & [\ac{acc}]
          & [\ac{dat}]
          \\ \cmidrule{1-4}
      [\ac{nom}]
          & \ac{nom}
          & \ac{acc}
          & \ac{dat}
          \\ \cmidrule{1-4}
      [\ac{acc}]
          & \ac{acc}
          & \ac{acc}
          & \cellcolor{DG}\ac{dat}
          \\ \cmidrule{1-4}
      [\ac{dat}]
          & \ac{dat}
          & \cellcolor{LG}\ac{dat}
          & (\ac{dat})
          \\
    \bottomrule
  \end{tabular}
    \label{tbl:summary-old-high-german-acc-dat}
\end{table}

In my research I encountered a single counterexample to the pattern I just described.
Consider the example in \ref{ex:ohg-counterexample}. In this example, the internal nominative case competes against the external accusative case.
The internal case is nominative, as the predicate \tit{giheilen} `to save' takes nominative subjects.
The external case is accusative, as the predicate \tit{beran} `to bear' takes accusative objects.
Surprisingly, the relative pronoun \tit{thér} `\tsc{rp}.\ac{sg}.\ac{m}.\ac{nom}' appears in the internal case: the nominative, which is the less complex of the two cases. The relative pronoun is marked in bold, just as the relative clause, showing that the relative pronoun patterns with the relative clause.

\exg. Tház si uns béran scolti \tbf{thér} \tbf{unsih} \tbf{gihéilti}\\
 that 3\ac{sg}.\ac{f}.\ac{nom} 1\ac{pl}.\ac{dat} bear.\ac{inf}\scsub{[acc]} should.\ac{subj}.\ac{pst}.3\ac{sg} \tsc{rp}.\ac{sg}.\ac{m}.\ac{nom} 1\ac{pl}.\ac{acc} save.\ac{sbjv}.\ac{pst}.3\ac{sg}\scsub{[nom]}\\
 `that she should have beared for us the one, who had saved us' \flushfill{Old High German, \ac{otfrid} I 3:38}\label{ex:ohg-counterexample}

This example is unexpected, because the least complex case (the nominative) wins and not the most complex case (the accusative).
The only explanation for this I can see is a functional one. The \tit{thér} `\tsc{rp}.\ac{sg}.\ac{m}.\ac{nom}' in \ref{ex:ohg-counterexample} refers to Jesus. In the relative clause he is the subject of \tit{unsih gihéilti} `had saved us', hence the internal nominative case. In the main clause he is the object of \tit{tház si uns béran scolti} `that she should have beared', hence the external accusative case.
Letting the relative pronoun surface in the internal case could be interpreted as emphasizing the role of Jesus as a savior, rather than him being the object of being given birth to. In line with that reasoning, it is expected that certain grammatical facts more often deviate from regular patterns if Jesus is involved. I leave investigating this prediction for future research.
Of course, this does not answer the question of what happens to the accusative case required by the external predicate. It also does not explain why not another emphasizing strategy is used, for instance forming a light-headed relative, which would leave space for two cases.
I acknowledge this example as a counterexample to the pattern I describe, but I do not change my generalization, as this is a single occurrence.

Leaving the counterexample aside, I conclude that Gothic and Old High German are both instances of languages that allow the internal and the external case to surface. The relative pronoun surfaces in the case that wins the case competition.


\section{Only internal case allowed}\label{sec:pattern-ii}

%there is variation, see vogel 2011 on exceptions. sometimes modern german seems to be like polish, sometimes like itself and sometimes like gothic. (1) in case of polish: do these headless relatives maybe have a definite interprettaion, and the light head is not r but der? then it cannot be deleted! (2) in case of modern german: it is really wen - r, (3) there is a syncretism between acc and dat (object case, brandenburg) or between nom and acc (see also was already being a syncretism), so that's why this is all allowed. the nom-dat incompatibility corroborates this hypothesis

This section discusses the situation in which only the internal case is allowed to surface when it wins the case competition. When the internal case wins the case competition, the result is ungrammatical. I repeat the pattern from Section \ref{sec:possible-patterns} in Table \ref{tbl:case-competition-only-int-repeated}.

\begin{table}[ht]
  \center
  \caption{Only internal case allowed (repeated)}
  \begin{tabular}{c|c|c|c}
    \toprule
    \textsubscript{\ac{int}} \textsuperscript{\ac{ext}}
           & [\ac{nom}]
           & [\ac{acc}]
           & [\ac{dat}]
           \\ \cmidrule{1-4}
       [\ac{nom}]
           & \ac{nom}
           & *
           & *
           \\ \cmidrule{1-4}
       [\ac{acc}]
           & \ac{acc}
           & \ac{acc}
           & *
           \\ \cmidrule{1-4}
       [\ac{dat}]
           & \ac{dat}
           & \ac{dat}
           & \ac{dat}
           \\
     \bottomrule
  \end{tabular}
    \label{tbl:case-competition-only-int-repeated}
\end{table}

An example of a language that shows this pattern is Modern German. In this section I discuss the Modern German data, based on the research of \citet{vogel2001}. The examples and the judgements are \posscitet{vogel2001}. I made the glosses more detailed, and I added translations where they were absent.

First I discuss examples in which the internal and the external case match, and then examples in which they differ. If the internal case and the external case are identical, so there is a tie, the relative pronoun simply surfaces in that case. I illustrate this for the nominative, the accusative and the dative.

Consider the example in \ref{ex:mg-nom-nom}, in which the internal nominative case competes against the external nominative case.
The internal case is nominative, as the predicate \tit{mögen} `to like' takes nominative subjects.
The external case is nominative as well, as the predicate \tit{besuchen} `to visit' also takes nominative subjects.
The relative pronoun \tit{wer} `\tsc{rp}.\ac{an}.\ac{nom}' appears in the internal and external case: the nominative.

\exg. Uns besucht, \tbf{wer} \tbf{Maria} \tbf{mag}.\\
 2\ac{pl}.\ac{acc} visit.\ac{pres}.3\ac{sg}\scsub{[nom]} \tsc{rp}.\ac{an}.\ac{nom} Maria.\ac{acc} like.\ac{pres}.3\ac{sg}\scsub{[nom]}\\
 `Who visits us likes Maria.' \flushfill{Modern German, adapted from \pgcitealt{vogel2001}{343}}\label{ex:mg-nom-nom}

Consider the example in \ref{ex:mg-acc-acc}, in which the internal accusative case competes against the external accusative case.
The internal case is accusative, as the predicate \tit{mögen} `to like' takes accusative objects.
The external case is accusative as well, as the predicate \tit{einladen} `to invite' also takes accusative objects.
The relative pronoun \tit{wen} `\tsc{rp}.\ac{an}.\ac{acc}' appears in the internal and external case: the accusative.

\exg. Ich {lade ein}, \tbf{wen} \tbf{auch} \tbf{Maria} \tbf{mag}.\\
 1\ac{sg}.\ac{nom} invite.\ac{pres}.1\ac{sg}\scsub{[acc]} \tsc{rp}.\ac{an}.\ac{acc} Maria.\ac{nom} like.\ac{pres}.3\ac{sg}\scsub{[acc]}\\
 `I invite who Maria also likes.' \flushfill{Modern German, adapted from \pgcitealt{vogel2001}{344}}\label{ex:mg-acc-acc}

Consider the examples in \ref{ex:mg-dat-dat}, in which the internal dative case competes against the external dative case.
The internal case is dative, as the predicate \tit{vertrauen} `to please' takes dative objects.
The external case is dative as well, as the predicate \tit{folgen} `to follow' also takes dative objects.
The relative pronoun \tit{wem} `\tsc{rp}.\ac{an}.\ac{dat}' appears in the internal and external case: the dative.

\exg. Ich folge, \tbf{wem} \tbf{immer} \tbf{ich} \tbf{vertraue}.\\
 1\ac{sg}.\ac{nom} folge.\ac{pres}.1\ac{sg}\scsub{[dat]} \tsc{rp}.\ac{an}.\ac{dat} ever 1\ac{sg}.\ac{nom} vertraue.\ac{pres}.3\ac{sg}\scsub{[dat]}\\
 `I follow whoever I trust.' \flushfill{Modern German, adapted from \pgcitealt{vogel2001}{342}}\label{ex:mg-dat-dat}

These findings can be summarized as in Table \ref{tbl:summary-mg-matching}. The top-left to bottom-right diagonal corresponds to the examples I have given so far in which the internal and external case match. The nominative marked in light gray corresponds to \ref{ex:mg-nom-nom}, in which the internal nominative case competes against the external nominative case, and the relative pronoun surfaces in the nominative case. The accusative marked in dark gray corresponds to \ref{ex:mg-acc-acc}, in which the internal accusative case competes against the external accusative case, and the relative pronoun surfaces in the accusative case. The unmarked dative corresponds to \ref{ex:mg-dat-dat}, in which the internal dative case competes against the external dative case, and the relative pronoun surfaces in the dative case.

\begin{table}[ht]
 \center
 \caption{Modern German headless relatives (matching)}
 \begin{tabular}{c|c|c|c}
   \toprule
    \textsubscript{\ac{int}} \textsuperscript{\ac{ext}}
         & [\ac{nom}]
         & [\ac{acc}]
         & [\ac{dat}]
         \\ \cmidrule{1-4}
     [\ac{nom}]
         & \colorbox{LG}{\ac{nom}}
         &
         &
         \\ \cmidrule{1-4}
     [\ac{acc}]
         &
         & \colorbox{DG}{\ac{acc}}
         &
         \\ \cmidrule{1-4}
     [\ac{dat}]
         &
         &
         & \ac{dat}
         \\
   \bottomrule
 \end{tabular}
   \label{tbl:summary-mg-matching}
\end{table}

In Table \ref{tbl:summary-mg-matching}, six cells remain empty. These are the cases in which the internal and the external case differ. In the remainder of this section, I discuss them one by one.

I start with the competition between the accusative and the nominative. Following the case scale, the relative pronoun appears in the accusative case and never in nominative. Following the internal-only requirement, when the accusative case is the internal case, the sentence is grammatical. When the accusative is the external case, the sentence is ungrammatical.

I start with the situation in which the internal case wins the competition, and it is possible to have a grammatical Modern German headless relative.
Consider the example in \ref{ex:mg-nom-acc}. In this example, the internal accusative case competes against the external nominative case.
The internal case is accusative, as the predicate \tit{mögen} `to like' takes accusative objects.
The external case is nominative, as the predicate \tit{besuchen} `to visit' takes nominative subjects.
The relative pronoun \tit{wen} `\tsc{rp}.\ac{an}.\ac{acc}' appears in the internal case: the accusative. The relative pronoun is marked in bold, just as the relative clause, showing that the relative pronoun patterns with the relative clause.
The example is grammatical, because the example adheres to the case scale, and the most complex case (here the accusative) is the internal case.

\exg. Uns besucht, \tbf{wen} \tbf{Maria} \tbf{mag}.\\
 2\ac{pl}.\ac{acc} visit.\ac{pres}.3\ac{sg}\scsub{[nom]} \tsc{rp}.\ac{an}.\ac{acc} Maria.\ac{nom} like.\ac{pres}.3\ac{sg}\scsub{[acc]}\\
 `Who visits us, Maria likes.' \flushfill{Modern German, adapted from \pgcitealt{vogel2001}{343}}\label{ex:mg-nom-acc}

The example in \ref{ex:mg-nom-acc-u} is identical to \ref{ex:mg-nom-acc}, except for that the relative pronoun appears in the external less complex nominative case. This example is ungrammatical: although the internal case is more complex, the relative pronoun appears in the least complex case (the nominative) and not in the most complex case (the accusative).

\exg. *Uns besucht, wer \tbf{Maria} \tbf{mag}.\\
 2\ac{pl}.\ac{acc} visit.\ac{pres}.3\ac{sg}\scsub{[nom]} \tsc{rp}.\ac{an}.\ac{nom} Maria.\ac{nom} like.\ac{pres}.3\ac{sg}\scsub{[acc]}\\
 `Who visits us, Maria likes.' \flushfill{Modern German, adapted from \pgcitealt{vogel2001}{343}}\label{ex:mg-nom-acc-u}

Now I turn to the situation in which the external case wins the competition, and there is no grammatical outcome possible, whichever case the relative pronoun appears in.
Consider the example in \ref{ex:mg-acc-nom}. In this example, the internal nominative case competes against the external accusative case.
The internal case is nominative, as the predicate \tit{sein} `to be' takes nominative subjects.
The external case is accusative, as the predicate \tit{einladen} `to invite' takes accusative objects.
The relative pronoun \tit{wen} `\tsc{rp}.\ac{an}.\ac{acc}' appears in the external case: the accusative. The relative pronoun is not marked in bold, just as the main clause, showing that the relative pronoun patterns with the main clause.
The example adheres to the case scale, but the most complex case (here the accusative) is not the internal case. The example is ungrammatical, because only the internal can win the case competition in Modern German.

\exg. *Ich {lade ein}, wen \tbf{mir} \tbf{sympathisch} \tbf{ist}.\\
1\ac{sg}.\ac{nom} invite.\ac{pres}.1\ac{sg}\scsub{[acc]} \tsc{rp}.\ac{an}.\ac{acc} 1\ac{sg}.\ac{dat} nice be.\ac{pres}.3\ac{sg}\scsub{[nom]}\\
`I invite who I like.' \flushfill{Modern German, adapted from \pgcitealt{vogel2001}{344}}\label{ex:mg-acc-nom}

The example in \ref{ex:mg-acc-nom-u} is identical to \ref{ex:mg-acc-nom}, except for that the relative pronoun appears in the external less complex nominative case. This example is also ungrammatical: in addition to the most complex case not being the internal case, the relative pronoun also does not appear in the most complex case (the accusative) but in the least complex case (the nominative).\footnote{
Not every speaker or Modern German agrees with the ungrammaticality of \ref{ex:mg-acc-nom-u}. A sentence for which also has been claimed that speakers accept it is given in \ref{ex:mg-liebe-hasse}. This example was originally marked as ungrammatical by \pgcitet{groos1981}{206}.

\exg. Ich liebe \tbf{wer} \tbf{gutes} \tbf{tut}, und hasse, \tbf{wer} \tbf{mich} \tbf{verletzt}.\\
1\ac{sg}.\ac{nom} love.1\ac{sg}\scsub{[acc]} \tsc{rp}.\ac{an}.\ac{nom} good.\ac{nmlz} do.3\ac{sg}\scsub{[nom]}
and hate.1\ac{sg}\scsub{[acc]} \tsc{rp}.\ac{an}.\ac{nom} I\ac{sg}.\ac{acc} hurt.3\ac{sg}\scsub{[nom]}\\
`I love who does good and hate who hurts me.' \flushfill{Modern German, adapted from \pgcitealt{groos1981}{206}}\label{ex:mg-liebe-hasse}

The relative acceptability of \ref{ex:mg-acc-nom-u} and \ref{ex:mg-liebe-hasse} is unexpected because the relative pronoun appears in the least complex case (the nominative) and not in the more complex case (the accusative). However, the more complex case would also not be grammatical, because it is the external case, and Modern German only allows the relative pronoun to surface in the internal case. My hypothesis is that, because there is no way of making the headless relative grammatical, speakers try to make the construction work by somehow repairing it. I can think of two strategies for that: (1) they can take \tit{wer gutes tut} `who does good' and \tit{wer mich verletzt} `who hurts me' as clauses objects, which are not case-marked in German, or (2) they insert a morphologically silent object as the head of the relative clause.

Notice that this type of example is crucially different from the Old High German counterexample in \ref{ex:ohg-counterexample}. In the Old High German situation, there was a grammatical possibility which was not used, and in the Modern German situation, there is no grammatical way to make a headless relative.
}

\exg. *Ich {lade ein}, \tbf{wer} \tbf{mir} \tbf{sympathisch} \tbf{ist}.\\
1\ac{sg}.\ac{nom} invite.\ac{pres}.1\ac{sg}\scsub{[acc]} \tsc{rp}.\ac{an}.\ac{nom} 1\ac{sg}.\ac{dat} nice be.\ac{pres}.3\ac{sg}\scsub{[nom]}\\
`I invite who I like.' \flushfill{Modern German, adapted from \pgcitealt{vogel2001}{344}}\label{ex:mg-acc-nom-u}

The two examples in which the nominative and the accusative compete are highlighted in Table \ref{tbl:case-competition-mg-nom-acc}.
The light gray marking corresponds to \ref{ex:mg-nom-acc}, in which the internal accusative wins over the external nominative, and the relative pronoun surfaces in the accusative case (and not in the losing nominative case as in \ref{ex:mg-nom-acc-u}).
The dark gray marking corresponds to \ref{ex:mg-acc-nom}, in which the external accusative wins over the internal nominative, but the relative pronoun is not allowed to surface in the accusative case (or in the losing nominative case as in \ref{ex:mg-acc-nom-u}).

 \begin{table}[ht]
   \center
   \caption{Modern German headless relatives (\ac{nom} --- \ac{acc})}
   \begin{tabular}{c|c|c|c}
     \toprule
     \textsubscript{\ac{int}} \textsuperscript{\ac{ext}}
            & [\ac{nom}]
            & [\ac{acc}]
            & [\ac{dat}]
            \\ \cmidrule{1-4}
        [\ac{nom}]
            & \ac{nom}
            & \cellcolor{DG}*
            &
            \\ \cmidrule{1-4}
        [\ac{acc}]
            & \cellcolor{LG}\ac{acc}
            & \ac{acc}
            &
            \\ \cmidrule{1-4}
        [\ac{dat}]
            &
            &
            & \ac{dat}
            \\
      \bottomrule
   \end{tabular}
     \label{tbl:case-competition-mg-nom-acc}
 \end{table}

I continue with the competition between the dative and the nominative. Following the case scale, the relative pronoun appears in the dative case and never in nominative. Following the internal-only requirement, when the dative case is the internal case, the sentence is grammatical.

I start again with the situation in which the internal case wins the competition, and it is possible to have a grammatical Modern German headless relative.
Consider the example in \ref{ex:mg-nom-dat}. In this example, the internal dative case competes against the external nominative case.
The internal case is dative, as the predicate \tit{vertrauen} `to trust' takes dative objects.
The external case is nominative, as the predicate \tit{besuchen} `to visit' takes nominative subjects.
The relative pronoun \tit{wem} `\tsc{rp}.\ac{an}.\ac{dat}' appears in the internal case: the dative. The relative pronoun is marked in bold, just as the relative clause, showing that the relative pronoun patterns with the relative clause.
The example adheres to the case scale, and the most complex case (here the dative) is the internal case, so the example is grammatical.

\exg. Uns besucht, \tbf{wem} \tbf{Maria} \tbf{vertraut}.\\
2\ac{pl}.\ac{acc} visit.\ac{pres}.3\ac{sg}\scsub{[nom]} \tsc{rp}.\ac{an}.\ac{dat} Maria.\ac{nom} trust.\ac{pres}.3\ac{sg}\scsub{[dat]}\\
`Who visits us, Maria trusts.' \flushfill{Modern German, adapted from \pgcitealt{vogel2001}{343}}\label{ex:mg-nom-dat}

The example in \ref{ex:mg-nom-dat-u} is identical to \ref{ex:mg-nom-dat}, except for that the relative pronoun appears in the external less complex nominative case. This example is ungrammatical: although the internal case is more complex, the relative pronoun appears in the least complex case (the nominative) and not in the most complex case (the dative).

\exg. *Uns besucht, wer \tbf{Maria} \tbf{vertraut}.\\
2\ac{pl}.\ac{acc} visit.\ac{pres}.3\ac{sg}\scsub{[nom]} \tsc{rp}.\ac{an}.\ac{nom} Maria.\ac{nom} trust.\ac{pres}.3\ac{sg}\scsub{[dat]}\\
`Who visits us, Maria trusts.' \flushfill{Modern German, adapted from \pgcitealt{vogel2001}{343}}\label{ex:mg-nom-dat-u}

Now I turn again to the situation in which the external case wins the competition, and there is no grammatical outcome possible, whichever case the relative pronoun appears in.
Consider the example in \ref{ex:mg-dat-nom}. In this example, the internal nominative case competes against the external dative case.
The internal case is nominative, as the predicate \tit{mögen} `to like' takes nominative subjects.
The external case is dative, as the predicate \tit{vertrauen} `to trust' takes dative objects.
The relative pronoun \tit{wem} `\tsc{rp}.\ac{an}.\ac{dat}' appears in the external case: the dative. The relative pronoun is not marked in bold, just as the main clause, showing that the relative pronoun patterns with the main clause.
The example adheres to the case scale, but the most complex case (here the dative) is not the internal case. The example is ungrammatical, because only the internal can win the case competition in Modern German.

\exg. *Ich vertraue, wem \tbf{Hitchcock} \tbf{mag}.\\
1\ac{sg}.\ac{nom} trust.\ac{pres}.1\ac{sg}\scsub{[dat]} \tsc{rp}.\ac{an}.\ac{dat} Hitchcock.\ac{acc} like.\ac{pres}.3\ac{sg}\scsub{[nom]}\\
`I trust who likes Hitchcock.' \flushfill{Modern German, adapted from \pgcitealt{vogel2001}{345}}\label{ex:mg-dat-nom}

The example in \ref{ex:mg-dat-nom-u} is identical to \ref{ex:mg-dat-nom}, except for that the relative pronoun appears in the external less complex nominative case. This example is also ungrammatical: in addition to the most complex case not being the internal case, the relative pronoun also does not appear in the most complex case (the dative) but in the least complex case (the nominative).

\exg. *Ich vertraue, \tbf{wer} \tbf{Hitchcock} \tbf{mag}.\\
1\ac{sg}.\ac{nom} trust.\ac{pres}.1\ac{sg}\scsub{[dat]} \tsc{rp}.\ac{an}.\ac{nom} Hitchcock.\ac{acc} like.\ac{pres}.3\ac{sg}\scsub{[nom]}\\
`I trust who likes Hitchcock.' \flushfill{Modern German, adapted from \pgcitealt{vogel2001}{345}}\label{ex:mg-dat-nom-u}

The two examples in which the nominative and the dative compete are highlighted in Table \ref{tbl:case-competition-mg-nom-dat}.
The light gray marking corresponds to \ref{ex:mg-nom-dat}, in which the internal dative wins over the external nominative, and the relative pronoun surfaces in the dative case (and not in the losing nominative case as in \ref{ex:mg-nom-dat-u}).
The dark gray marking corresponds to \ref{ex:mg-dat-nom}, in which the external dative wins over the internal nominative, but the relative pronoun is not allowed to surface in the dative case (or in the losing nominative case as in \ref{ex:mg-dat-nom-u}).

\begin{table}[ht]
  \center
  \caption{Modern German headless relatives (\ac{nom} --- \ac{dat})}
  \begin{tabular}{c|c|c|c}
    \toprule
    \textsubscript{\ac{int}} \textsuperscript{\ac{ext}}
           & [\ac{nom}]
           & [\ac{acc}]
           & [\ac{dat}]
           \\ \cmidrule{1-4}
       [\ac{nom}]
           & \ac{nom}
           & *
           & \cellcolor{DG}*
           \\ \cmidrule{1-4}
       [\ac{acc}]
           & \ac{acc}
           & \ac{acc}
           &
           \\ \cmidrule{1-4}
       [\ac{dat}]
           & \cellcolor{LG}\ac{dat}
           &
           & \ac{dat}
           \\
     \bottomrule
  \end{tabular}
    \label{tbl:case-competition-mg-nom-dat}
\end{table}

I end with the competition between the dative and the accusative. Following the case scale, the relative pronoun appears in the dative case and never in accusative. Following the internal-only requirement, when the dative case is the internal case, the sentence is grammatical.

I start again with the situation in which the internal case wins the competition, and it is possible to have a grammatical Modern German headless relative.
Consider the example in \ref{ex:mg-acc-dat}. In this example, the internal dative case competes against the external accusative case.
The internal case is dative, as the predicate \tit{vertrauen} `to trust' takes dative objects.
The external case is accusative, as the predicate \tit{einladen} `to invite' takes accusative objects.
The relative pronoun \tit{wem} `\tsc{rp}.\ac{an}.\ac{dat}' appears in the internal case: the dative. The relative pronoun is marked in bold, just as the relative clause, showing that the relative pronoun patterns with the relative clause.
The example adheres to the case scale, and the most complex case (here the dative) is the internal case, so the example is grammatical.

\exg. Ich {lade ein}, \tbf{wem} \tbf{auch} \tbf{Maria} \tbf{vertraut}. \\
1\ac{sg}.\ac{nom} invite.\ac{pres}.1\ac{sg}\scsub{[acc]} \tsc{rp}.\ac{an}.\ac{dat} also Maria.\ac{nom} trust.\ac{pres}.3\ac{sg}\scsub{[dat]}\\
`I invite whoever Maria also trusts.' \flushfill{Modern German, adapted from \pgcitealt{vogel2001}{344}}\label{ex:mg-acc-dat}

The example in \ref{ex:mg-acc-dat-u} is identical to \ref{ex:mg-acc-dat}, except for that the relative pronoun appears in the external less complex accusative case. This example is ungrammatical: although the internal case is more complex, the relative pronoun appears in the least complex case (the accusative) and not in the most complex case (the dative).

\exg. *Ich {lade ein}, wen \tbf{auch} \tbf{Maria} \tbf{vertraut}. \\
1\ac{sg}.\ac{nom} invite.\ac{pres}.1\ac{sg}\scsub{[acc]} \tsc{rp}.\ac{an}.\ac{acc} also Maria.\ac{nom} trust.\ac{pres}.3\ac{sg}\scsub{[dat]}\\
`I invite whoever Maria also trusts.' \flushfill{Modern German, adapted from \pgcitealt{vogel2001}{344}}\label{ex:mg-acc-dat-u}

Now I turn again to the situation in which the external case wins the competition, and there is no grammatical outcome possible, whichever case the relative pronoun appears in.
Consider the example in \ref{ex:mg-dat-acc}. In this example, the internal accusative case competes against the external dative case.
The internal case is accusative, as the predicate \tit{mögen} `to like' takes accusative objects.
The external case is dative, as the predicate \tit{vertrauen} `to trust' takes dative objects.
The relative pronoun \tit{wem} `\tsc{rp}.\ac{an}.\ac{dat}' appears in the external case: the dative. The relative pronoun is not marked in bold, just as the main clause, showing that the relative pronoun patterns with the main clause.
The example adheres to the case scale, but the most complex case (here the dative) is not the internal case. The example is ungrammatical, because only the internal can win the case competition in Modern German.

\exg. *Ich vertraue, wem \tbf{auch} \tbf{Maria} \tbf{mag}. \\
1\ac{sg}.\ac{nom} trust.\ac{pres}.1\ac{sg}\scsub{[dat]} \tsc{rp}.\ac{an}.\ac{dat} also Maria.\ac{nom} like.\ac{pres}.3\ac{sg}\scsub{[acc]}\\
`I trust whoever Maria also likes.' \flushfill{Modern German, adapted from \pgcitealt{vogel2001}{345}}\label{ex:mg-dat-acc}

The example in \ref{ex:mg-dat-acc-u} is identical to \ref{ex:mg-dat-acc}, except for that the relative pronoun appears in the external less complex accusative case. This example is also ungrammatical: in addition to the most complex case not being the internal case, the relative pronoun also does not appear in the most complex case (the dative) but in the least complex case (the accusative).

\exg. *Ich vertraue, \tbf{wen} \tbf{auch} \tbf{Maria} \tbf{mag}. \\
1\ac{sg}.\ac{nom} trust.\ac{pres}.1\ac{sg}\scsub{[dat]} \tsc{rp}.\ac{an}.\ac{acc} also Maria.\ac{nom} like.\ac{pres}.3\ac{sg}\scsub{[acc]}\\
`I trust whoever Maria also likes.' \flushfill{Modern German, adapted from \pgcitealt{vogel2001}{345}}\label{ex:mg-dat-acc-u}

The two examples in which the nominative and the dative compete are highlighted in Table \ref{tbl:case-competition-mg-acc-dat}.
The light gray marking corresponds to \ref{ex:mg-acc-dat}, in which the internal dative wins over the external accusative, and the relative pronoun surfaces in the dative case (and not in the losing accusative case as in \ref{ex:mg-acc-dat-u}).
The dark gray marking corresponds to \ref{ex:mg-dat-acc}, in which the external dative wins over the internal nominative, but the relative pronoun is not allowed to surface in the dative case (or in the losing accusative case as in \ref{ex:mg-dat-acc-u}).

\begin{table}[ht]
  \center
  \caption{Modern German headless relatives (\ac{acc} --- \ac{dat})}
  \begin{tabular}{c|c|c|c}
    \toprule
    \textsubscript{\ac{int}} \textsuperscript{\ac{ext}}
           & [\ac{nom}]
           & [\ac{acc}]
           & [\ac{dat}]
           \\ \cmidrule{1-4}
       [\ac{nom}]
           & \ac{nom}
           & *
           & *
           \\ \cmidrule{1-4}
       [\ac{acc}]
           & \ac{acc}
           & \ac{acc}
           & \cellcolor{DG}*
           \\ \cmidrule{1-4}
       [\ac{dat}]
           & \ac{dat}
           & \cellcolor{LG}\ac{dat}
           & \ac{dat}
           \\
     \bottomrule
  \end{tabular}
    \label{tbl:case-competition-mg-acc-dat}
\end{table}

In sum, Modern German is an instance of a language that only allows the internal case to surface. The relative pronoun surfaces in the most complex case, but only when this more complex case is the internal case.


\section{Only external case allowed}\label{sec:pattern-iii}

This section discusses the situation in which only the external case is allowed to surface when it wins the case competition. When the internal case wins the case competition, the result is ungrammatical. I repeat the pattern from Section \ref{sec:possible-patterns} in Table \ref{tbl:case-competition-only-ext-repeated}.

\begin{table}[ht]
  \center
  \caption{Only external case allowed (repeated)}
  \begin{tabular}{c|c|c|c}
    \toprule
    \textsubscript{\ac{int}} \textsuperscript{\ac{ext}}
           & [\ac{nom}]
           & [\ac{acc}]
           & [\ac{dat}]
           \\ \cmidrule{1-4}
       [\ac{nom}]
           & \ac{nom}
           & \ac{acc}
           & \ac{dat}
           \\ \cmidrule{1-4}
       [\ac{acc}]
           & *
           & \ac{acc}
           & \ac{dat}
           \\ \cmidrule{1-4}
       [\ac{dat}]
           & *
           & *
           & \ac{dat}
           \\
     \bottomrule
  \end{tabular}
    \label{tbl:case-competition-only-ext-repeated}
\end{table}

To my knowledge, this pattern is not attested in any natural language, whether extinct or alive. Classical Greek has been mentioned in the literature both as a language of the third type (c.f. \citealt[120]{cinqueforthcoming}, who actually classifies Gothic also as such) and as a language of the first type \citep[cf.][41]{grosu1987}. I show that the correct description of Classical Greek is the latter, and that it patterns with Gothic and Old High German.\footnote{
It does seem to be the case that examples in which the external case wins over the internal case are more frequent in Classical Greek than examples in which the internal case wins over the external case (see \citealt{kakarikos2014} for numerous examples of the former type).
In this dissertation I do not address the question of why certain constructions and configurations are more frequent than others. My goal is to set up a system that generates the grammatical patterns and excludes the ungrammatical or unattested patterns.
}
I start with an example in which a more complex external case wins the case competition over a less complex internal case, and the relative pronoun surfaces in the external case.

Consider the example in \ref{ex:ag-dat-acc}. In this example, the internal accusative case competes against the external dative case.
The internal case is accusative, as the predicate \tit{tíktō} `to give birth to' takes accusative objects.
The external case is dative, as the predicate \tit{ékhō} `to provide' takes dative indirect objects.
The relative pronoun \tit{hō̃ͅ} `\tsc{rp}.\ac{sg}.\ac{m}.\ac{dat}' appears in the internal case: the dative. The relative pronoun is not marked in bold, unlike as the relative clause, showing that the relative pronoun patterns with the main clause.

\exg. pãn {tò tekòn} trophḕn ékhei hō̃ͅ \tbf{án} \tbf{tékēͅ}\\
any parent.\ac{sg}.\ac{nom} food.\ac{sg}.\ac{acc} provide.\ac{pres}.3\ac{sg}\scsub{[dat]} \tsc{rp}.\ac{sg}.\ac{m}.\ac{dat} \ac{mod} {gives birth}.\ac{aor}.3\ac{sg}\scsub{[acc]}\\
`any parent provides food to what he would have given birth to' \flushfill{Classical Greek, \ac{pl.men} 237e, adapted from \pgcitealt{kakarikos2014}{292}}\label{ex:ag-dat-acc}

This example is compatible with the picture of Classical Greek only allowing the external case to surface when it wins the competition. I repeat Table \ref{tbl:case-competition-only-ext-repeated} from the beginning of this section as Table \ref{tbl:case-competition-ag-poss1}, and I mark the cell that corresponds to the example in \ref{ex:ag-dat-acc} in gray.

\begin{table}[ht]
  \center
  \caption{Classical Greek headless relatives possibility 1}
  \begin{tabular}{c|c|c|c}
    \toprule
    \textsubscript{\ac{int}} \textsuperscript{\ac{ext}}
           & [\ac{nom}]
           & [\ac{acc}]
           & [\ac{dat}]
           \\ \cmidrule{1-4}
       [\ac{nom}]
           & \ac{nom}
           & \ac{acc}
           & \ac{dat}
           \\ \cmidrule{1-4}
       [\ac{acc}]
           & *
           & \ac{acc}
           & \cellcolor{LG}\ac{dat}
           \\ \cmidrule{1-4}
       [\ac{dat}]
           & *
           & *
           & \ac{dat}
           \\
     \bottomrule
  \end{tabular}
    \label{tbl:case-competition-ag-poss1}
\end{table}

However, the example in \ref{ex:ag-dat-acc} is not only compatible with the external-only type. Considering only the example I have given so far, it is still possible for Classical Classical Greek to be of the unrestricted type. I repeat Table \ref{tbl:case-competition-int-ext-repeated} from Section \ref{sec:pattern-i} as Table \ref{tbl:case-competition-ag-poss2}, and I mark the cell that corresponds to the example in \ref{ex:ag-dat-acc} in gray.

\begin{table}[ht]
  \center
  \caption{Classical Greek headless relatives possibility 2}
  \begin{tabular}{c|c|c|c}
    \toprule
    \textsubscript{\ac{int}} \textsuperscript{\ac{ext}}
           & [\ac{nom}]
           & [\ac{acc}]
           & [\ac{dat}]
           \\ \cmidrule{1-4}
       [\ac{nom}]
           & \ac{nom}
           & \ac{acc}
           & \ac{dat}
           \\ \cmidrule{1-4}
       [\ac{acc}]
           & \ac{acc}
           & \ac{acc}
           & \cellcolor{LG}\ac{dat}
           \\ \cmidrule{1-4}
       [\ac{dat}]
           & \ac{dat}
           & \ac{dat}
           & \ac{dat}
           \\
     \bottomrule
  \end{tabular}
    \label{tbl:case-competition-ag-poss2}
\end{table}

What sets Table \ref{tbl:case-competition-ag-poss1} and Table \ref{tbl:case-competition-ag-poss2} apart is the bottom-left corner of the table. These are cases in which the internal case wins the case competition.
In Table \ref{tbl:case-competition-ag-poss1} these examples are not allowed to surface, and in Table \ref{tbl:case-competition-ag-poss2} they are.
In what follows, I give an example in which a more complex internal case wins over a less complex external case. This indicates that Classical Greek cannot be of the type shown in Table \ref{tbl:case-competition-ag-poss1}, but is has to be of the type shown in Table \ref{tbl:case-competition-ag-poss2}. In other words, it is not of the type that only allows the external case to surface when it wins the case competition.

Consider the example in \ref{ex:ag-nom-acc}. In this example, the internal accusative case competes against the external nominative case.
The internal case is accusative, as the predicate \tit{philéō} `to love' takes accusative objects.
The external case is nominative, as the predicate \tit{apothnḗiskō} `to die' takes nominative subjects.
The relative pronoun \tit{hòn} `\tsc{rp}.\ac{sg}.\ac{m}.\ac{acc}' appears in the internal case: the accusative. The relative pronoun is marked in bold, just as the relative clause, showing that the relative pronoun patterns with the relative clause.\footnote{
The sentence in \ref{ex:ag-nom-acc} can also be analyzed as a headed relative, in which the relative clause modifies the phonologically empty subject of \tit{apothnḗiskō} `to die'. Then, however, more needs to be said about how it is possible for a relative clause to modify a phonologically empty element.
}

\exg. \tbf{hòn} \tbf{hoi} \tbf{theoì} \tbf{philoũsin} apothnḗͅskei néos\\
\tsc{rp}.\ac{sg}.\ac{m}.\ac{acc} the god.\ac{pl} love.3\ac{pl}\scsub{[acc]} die.3\ac{sg}\scsub{[nom]} young\\
`He, whom the gods love, dies young.' \flushfill{Classical Greek, \ac{men.dd}, 125}\label{ex:ag-nom-acc}

This example shows that Classical Greek is not an instance of the third possible pattern, in which only the external case is allowed to surface. Instead, as illustrated by Table \ref{tbl:case-competition-classical-greek}, the language allows the internal case (marked light gray) and the external case (marked dark gray) to surface when either of them wins the case competition.

\begin{table}[ht]
  \center
  \caption{Summary of Classical Greek headless relatives}
  \begin{tabular}{c|c|c|c}
    \toprule
    \textsubscript{\ac{int}} \textsuperscript{\ac{ext}}
           & [\ac{nom}]
           & [\ac{acc}]
           & [\ac{dat}]
           \\ \cmidrule{1-4}
       [\ac{nom}]
           & \ac{nom}
           & \ac{acc}
           & \ac{dat}
           \\ \cmidrule{1-4}
       [\ac{acc}]
           & \cellcolor{LG}\ac{acc}
           & \ac{acc}
           & \cellcolor{DG}\ac{dat}
           \\ \cmidrule{1-4}
       [\ac{dat}]
           & \ac{dat}
           & \ac{dat}
           & \ac{dat}
           \\
     \bottomrule
  \end{tabular}
    \label{tbl:case-competition-classical-greek}
\end{table}

I do not discuss more examples from Classical Greek than I did until now. This does not change anything about the point I am making here: the only kind of system that is compatible with the examples given is the one in which the internal and the external case are allowed to surface when either of them wins the case competition. For more examples in which the external case wins, I refer the reader to \pgcitet{kakarikos2014}{292-294}. An example in which the external dative wins over the internal nominative can be found in \citet{noussia2015}. I am not aware of an example in which the internal dative wins over the external accusative.

To sum up, to my knowledge, there is no language in which only the external case is allowed to surface when it wins the case competition, and the internal case is not. Classical Greek patterns with Gothic and Old High German in that is allows the internal and the external case to surface.



\section{Only matching allowed}\label{sec:pattern-iv}

This section discusses the situation in which the case is neither the internal case nor the external case allowed to surface when either of them wins the competition. In other words, when the internal and the external case differ, there is no grammatical headless relative construction possible. Only when there is a tie, i.e. when the internal and external case match, there is a grammatical result. I repeat the pattern from Section \ref{sec:possible-patterns} in Table \ref{tbl:case-competition-none-repeated}.

\begin{table}[ht]
  \center
  \caption{The matching type (repeated)}
  \begin{tabular}{c|c|c|c}
    \toprule
    \textsubscript{\ac{int}} \textsuperscript{\ac{ext}}
           & [\ac{nom}]
           & [\ac{acc}]
           & [\ac{dat}]
           \\ \cmidrule{1-4}
       [\ac{nom}]
           & \ac{nom}
           & *
           & *
           \\ \cmidrule{1-4}
       [\ac{acc}]
           & *
           & \ac{acc}
           & *
           \\ \cmidrule{1-4}
       [\ac{dat}]
           & *
           & *
           & \ac{dat}
           \\
     \bottomrule
  \end{tabular}
    \label{tbl:case-competition-none-repeated}
\end{table}

An example of a language that shows this pattern is Polish. In this section I discuss the Polish data, based on the research of \citet{citko2013} after \citet{himmelreich2017}. I only go through the case competition between accusative and dative, as only this data is discussed. This does not change anything about the point I am making here: the only kind of system that is compatible with the examples given is the one in which the case is allowed to surface in neither the internal case nor in the external case, when either of them wins the case competition. I made the glosses more detailed, and I added translations where they were absent.

First I discuss examples in which the internal and the external case match, and then examples in which they differ. If the internal case and the external case are identical, so there is a tie, the relative pronoun simply surfaces in that case. I illustrate this for the nominative, the accusative and the dative.

Consider the example in \ref{ex:polish-acc-acc}, in which the internal accusative case competes against the external accusative case.
The internal case and external case are accusative, as the predicate \tit{lubić} `to like' in both clauses takes accusative objects.
The relative pronoun \tit{kogo} `\tsc{rp}.\tsc{acc}.\tsc{an}' appears in the internal and external case: the accusative.

\exg. Jan lubi kogo -\tbf{kolkwiek} \tbf{Maria} \tbf{lubi}.\\
 Jan like.\tsc{3sg}\scsub{[acc]} \tsc{rp}.\tsc{acc}.\tsc{an} ever Maria like.\tsc{3sg}\scsub{[acc]}\\
 `Jan likes whoever Maria likes.' \flushfill{Polish, adapted from \citealt{citko2013} after \pgcitealt{himmelreich2017}{17}}\label{ex:polish-acc-acc}

Consider the example in \ref{ex:polish-dat-dat}, in which the internal dative case competes against the external dative case.
The internal case is dative, as the predicate \tit{ufać} `to trust' takes dative objects.
The external case is dative as well, as the predicate \tit{pomagać} `to help' also takes dative objects.
The relative pronoun \tit{them} `\tsc{rp}.\ac{pl}.\ac{an}.\ac{dat}' appears in the internal and external case: the dative.

\exg. Jan pomaga komu -\tbf{kolkwiek} \tbf{ufa}.\\
 Jan help.\tsc{3sg}\scsub{[dat]} \tsc{rp}.\tsc{dat}.\tsc{an} ever trust.\tsc{3sg}\scsub{[dat]}\\
 `Jan helps whomever he trusts.' \flushfill{Polish, adapted from \citealt{citko2013} after \pgcitealt{himmelreich2017}{17}}\label{ex:polish-dat-dat}

These findings can be summarized as in Table \ref{tbl:summary-polish-matching}. The top-left to bottom-right diagonal corresponds to the examples I have given so far in which the internal and external case match. The accusative marked in light gray corresponds to \ref{ex:polish-acc-acc}, in which the internal accusative case competes against the external accusative case, and the relative pronoun surfaces in the accusative case. The dative marked in dark gray corresponds to \ref{ex:polish-dat-dat}, in which the internal dative case competes against the external dative case, and the relative pronoun surfaces in the dative case.

\begin{table}[ht]
 \center
 \caption{Polish headless relatives (matching)}
 \begin{tabular}{c|c|c}
   \toprule
   \textsubscript{\ac{int}} \textsuperscript{\ac{ext}}
          & [\ac{acc}]
          & [\ac{dat}]
          \\ \cmidrule{1-3}
      [\ac{acc}]
          & \cellcolor{LG}\ac{acc}
          &
          \\ \cmidrule{1-3}
      [\ac{dat}]
          &
          & \cellcolor{DG}\ac{dat}
          \\
    \bottomrule
 \end{tabular}
   \label{tbl:summary-polish-matching}
\end{table}

In Table \ref{tbl:summary-polish-matching}, two cells remain empty. These are the cases in which the internal and the external case differ. In the remainder of this section, I discuss them one by one.

I give examples from the case competition between accusative and dative. According to the case scale, the dative would win over the accusative. However, as the case is neither allowed to surface in the internal case nor in the external case, all examples are ungrammatical.

I start with the situation in which the internal case wins the competition, and there is no grammatical outcome possible, whichever case the relative pronoun appears in.
Consider the example in \ref{ex:mg-acc-dat}. In this example, the internal dative case competes against the external accusative case.
The internal case is dative, as the predicate \tit{dokuczać} `to tease' takes dative objects.
The external case is accusative, as the predicate \tit{lubić} `to like' takes accusative objects.
The relative pronoun \tit{komu} `\tsc{rp}.\ac{an}.\ac{dat}' appears in the internal case: the dative. The relative pronoun is marked in bold, just as the relative clause, showing that the relative pronoun patterns with the relative clause.
The example adheres to the case scale, but the internal case is not allowed to surface when it wins the case competition. Therefore, the example is ungrammatical.

\exg. *Jan lubi \tbf{komu} \tbf{-kolkwiek} \tbf{dokucza}.\\
Jan like.\tsc{3sg}\scsub{[acc]} \tsc{rp}.\tsc{dat}.\tsc{an} ever tease.\tsc{3sg}\scsub{[dat]}\\
`Jan likes whoever he teases.' \flushfill{Polish, adapted from \citealt{citko2013} after \pgcitealt{himmelreich2017}{17}}\label{ex:polish-acc-dat}

The example in \ref{ex:polish-acc-dat-u} is identical to \ref{ex:polish-acc-dat}, except for that the relative pronoun appears in the external less complex accusative case. This example is also ungrammatical: the external case is less complex, and the external case is not allowed to surface when it wins the case competition.

\exg. *Jan lubi kogo \tbf{-kolkwiek} \tbf{dokucza}.\\
Jan like.\tsc{3sg}\scsub{[acc]} \tsc{rp}.\tsc{acc}.\tsc{an} ever tease.\tsc{3sg}\scsub{[dat]}\\
`Jan likes whoever he teases.' \flushfill{Polish, adapted from \citealt{citko2013} after \pgcitealt{himmelreich2017}{17}}\label{ex:polish-acc-dat-u}

Now I turn to the situation in which the external case wins the competition, and there is no grammatical outcome possible, whichever case the relative pronoun appears in.
Consider the example in \ref{ex:polish-dat-acc}. In this example, the internal accusative case competes against the external dative case.
The internal case is accusative, as the predicate \tit{wpuścić} `to let' takes accusative objects.
The external case is dative, as the predicate \tit{ufać} `to trust' takes dative objects.
The relative pronoun \tit{komu} `\tsc{rp}.\ac{an}.\ac{dat}' appears in the external case: the dative. The relative pronoun is not marked in bold, just as the main clause, showing that the relative pronoun patterns with the main clause.
The example adheres to the case scale, but the external case is (as the internal case) not allowed to surface when it wins the case competition. Therefore, the example is ungrammatical.

\exg. *Jan ufa komu \tbf{-kolkwiek} \tbf{wpuścil} \tbf{do} \tbf{domu}.\\
Jan trust.\tsc{3sg}\scsub{[dat]} \tsc{rp}.\tsc{dat}.\tsc{an} ever let.\tsc{3sg}\scsub{[acc]} to home\\
`Jan trusts whoever he let into the house.' \flushfill{Polish, adapted from \citealt{citko2013} after \pgcitealt{himmelreich2017}{17}}\label{ex:polish-dat-acc}

The example in \ref{ex:polish-dat-acc-u} is identical to \ref{ex:polish-dat-acc}, except for that the relative pronoun appears in the internal less complex accusative case. This example is also ungrammatical: the internal case is less complex, and the internal case is not allowed to surface when it wins the case competition.

\exg. *Jan ufa \tbf{kogo} \tbf{-kolkwiek} \tbf{wpuścil} \tbf{do} \tbf{domu}.\\
Jan trust.\tsc{3sg}\scsub{[dat]} \tsc{rp}.\tsc{acc}.\tsc{an} ever let.\tsc{3sg}\scsub{[acc]} to home\\
`Jan trusts whoever he let into the house.' \flushfill{Polish, adapted from \citealt{citko2013} after \pgcitealt{himmelreich2017}{17}}\label{ex:polish-dat-acc-u}

The two examples in which the accusative and the dative compete are highlighted in Table \ref{tbl:summary-polish-acc-dat}.
The light gray marking corresponds to \ref{ex:polish-acc-dat}, in which the internal dative wins over the external accusative, but the relative pronoun is not allowed to surface in the dative case (or in the losing accusative case as in \ref{ex:polish-acc-dat-u}).
The dark gray marking corresponds to \ref{ex:polish-dat-acc}, in which the external dative wins over the internal accusative, but the relative pronoun is not allowed to surface in the dative case (or in the losing accusative case as in \ref{ex:polish-dat-acc-u}).

\begin{table}[ht]
  \center
  \caption{Polish headless relatives (\ac{acc} --- \ac{dat})}
  \begin{tabular}{c|c|c}
    \toprule
    \textsubscript{\ac{int}} \textsuperscript{\ac{ext}}
           & [\ac{acc}]
           & [\ac{dat}]
           \\ \cmidrule{1-3}
       [\ac{acc}]
           & \ac{acc}
           & \cellcolor{DG}{*}
           \\ \cmidrule{1-3}
       [\ac{dat}]
           & \cellcolor{LG}{*}
           & \ac{dat}
           \\
     \bottomrule
  \end{tabular}
    \label{tbl:summary-polish-acc-dat}
\end{table}

In sum, Polish is an instance of a language that only allows for matching cases. When the internal and the external case differ in Polish, there is no way to form a grammatical headless relative construction.

\section{Summary}\label{sec:summary-3-patterns}

In case competition in headless relatives two aspects play a role. The first one is which case wins the case competition. It is a crosslinguistically stable fact that this is determined by the case scale in \ref{ex:case-scale-two-patterns-sum}, repeated from Chapter \ref{ch:recurring}. A case more to the right on the scale wins over a case more to the left on the scale.

\ex. \ac{nom} < \ac{acc} < \ac{dat}\label{ex:case-scale-two-patterns-sum}

This generates the pattern shown in Table \ref{tbl:case-competition-table}. The left column shows the internal case between square brackets. The top row shows the external case between square brackets. The other cells indicate the case of the relative pronoun. When the dative wins over the accusative, the relative pronoun appears in the dative case. When the dative wins over the nominative, the relative pronoun appears in the nominative case. When the accusative wins over the nominative, the relative pronoun appears in the accusative case.

\begin{table}[ht]
  \center
  \caption{Relative pronoun follows case competition}
  \begin{tabular}{c|c|c|c}
    \toprule
    \textsubscript{\ac{int}} \textsuperscript{\ac{ext}}
           & [\ac{nom}]
           & [\ac{acc}]
           & [\ac{dat}]
           \\ \cmidrule{1-4}
       [\ac{nom}]
           & \ac{nom}
           & \ac{acc}
           & \ac{dat}
           \\ \cmidrule{1-4}
       [\ac{acc}]
           & \ac{acc}
           & \ac{acc}
           & \ac{dat}
           \\ \cmidrule{1-4}
       [\ac{dat}]
           & \ac{dat}
           & \ac{dat}
           & \ac{dat}
           \\
     \bottomrule
  \end{tabular}
    \label{tbl:case-competition-table}
\end{table}

The second aspect is whether the internal and the external case are allowed to surface when either of them wins the case competition. This differs across languages. There are four logical possibilities, listed in \ref{ex:logical-possibilities}.

\ex. Logically possibile language types\label{ex:logical-possibilities}
\let\oldalph=\alph\let\alph=\roman
\a. The unrestricted type: the internal and the external case are allowed to surface when either of them wins the case competition\label{ex:int-ext}
\b. The internal-only type: only the internal case is allowed to surface when it wins the case competition\label{ex:int-only}
\b. The external-only type: only the external case is allowed to surface when it wins the case competition\label{ex:ext-only}
\b. The matching type: neither the internal case nor in the external case is allowed to surface when either of them wins the case competition\label{ex:matching}
\global\let\alph=\oldalph

As far as I am aware, not all of these logical possibilities are attested in natural languages. I discuss the types one by one, and I give example when they are attested. In my description, I refer to the differ gray-marking in Table \ref{tbl:case-competition-table-marking}. The cells marked in light gray are the ones in which the internal case wins the case competition, the cells marked in dark gray are the ones in which the external case wins the case competition, and the unmarked cells are the ones in which the internal and external case match.

Gothic, Old High German and Classical Greek are examples of the unrestricted type in \ref{ex:int-ext}. In these languages, relative pronouns in the unmarked, light gray and dark gray cells are attested.
Modern German is an example of the `unrestricted --- internal-only' type in \ref{ex:int-only}. In this language, relative pronouns in the unmarked and light gray cells are grammatical.
To my knowledge, the `unrestricted --- external-only' type in \ref{ex:ext-only} is not attested. This would be a language in which relative pronouns in the unmarked and the dark gray cells are grammatical.
Polish is an example of a language of the matching type in \ref{ex:matching}. In this language, relative pronoun in only in the unmarked cells are grammatical.

\begin{table}[ht]
  \center
  \caption{Relative pronoun follows case competition}
  \begin{tabular}{c|c|c|c}
    \toprule
    \textsubscript{\ac{int}} \textsuperscript{\ac{ext}}
           & [\ac{nom}]
           & [\ac{acc}]
           & [\ac{dat}]
           \\ \cmidrule{1-4}
       [\ac{nom}]
           & \ac{nom}
           & \cellcolor{DG}\ac{acc}
           & \cellcolor{DG}\ac{dat}
           \\ \cmidrule{1-4}
       [\ac{acc}]
           & \cellcolor{LG}\ac{acc}
           & \ac{acc}
           & \cellcolor{DG}\ac{dat}
           \\ \cmidrule{1-4}
       [\ac{dat}]
           & \cellcolor{LG}\ac{dat}
           & \cellcolor{LG}\ac{dat}
           & \ac{dat}
           \\
     \bottomrule
  \end{tabular}
    \label{tbl:case-competition-table-marking}
\end{table}

Figure \ref{fig:attested-headless-relatives-case-competition} shows a diagram that generates the three attested patterns and not the unattested one. The diamonds stand for parameters that distinguish different types of languages. The texts along the arrows to the rectangles (and to a diamond) indicate how the different types of languages behave with respect to the parameters. The rectangles describe the form that the relative pronoun appears in. Below the rectangle I give examples of languages that are of this particular type.

The first parameter is whether or not a language allows for a mismatch between the internal and external case.
If a language does not allow for a mismatch, the matching type of language \ref{ex:matching} is generated.
If a language allows for a mismatch between the internal and external case, the second parameter comes into play. This one is concerned with the case the relative pronoun is allowed to surface when it wins the case competition. Here I give two options: (1) it is allowed to surface in only the internal case or (2) it is allowed to surface in the internal and the external case.\footnote{
I do not introduce the option of allowing the relative pronoun to surface only in the external case. The reason for this is that this pattern is not attested crosslinguistically. If a language like this appears, this option could in principle be added. However, I predict that it will not appear. In Chapter \ref{part:deriving}, I show how it follows from general properties of relative clauses that this type of language is excluded.
}
If a language allows the internal case to surface when it wins the case competition, the `unrestricted --- internal-only' type is generated.
If a language allows the internal and the external case to surface, the unrestricted type is generated.\footnote{
The matching type could also have been generated with the second parameter. The text along the arrow would have been \tit{none}. I choose to not do this, because in Chapter \ref{part:deriving} I propose separate mechanisms for each of the parameters in Figure \ref{fig:attested-headless-relatives-case-competition}. The first one distinguishes matching languages from unrestricted (i.e. unrestricted and internal-only) languages, and the second one distinguishes unrestricted from internal-only languages.
}

\begin{figure}[ht]
  \centering
  \begin{adjustbox}{max width=0.9\textwidth}
  \begin{tabular}[b]{c}
    \toprule
  \begin{tikzpicture}[node distance=1.5cm]
    \node (question2) [question]
    {allow \tsc{int}?};
        \node (outcome2) [outcome, below of=question2, xshift=-1.5cm]
        {matching};
            \node (example2) [example, below of=outcome2, yshift=0.25cm]
            {e.g. Polish\\\phantom{x}};
        \node (question3) [question, below of=question2, xshift=2cm, yshift=-0.5cm]
        {allow \tsc{ext}?};
            \node (outcome3) [outcome, below of=question3, xshift=-1.5cm]
            {internal-only};
                \node (example3) [example, below of=outcome3, yshift=0.25cm]
                {e.g. Modern German\\\phantom{x}};
            \node (outcome4) [outcome, below of=question3, xshift=1.5cm]
            {unrestricted};
                \node (example4) [example, below of=outcome4, yshift=0.25cm]
                {e.g. Gothic, Old High German, Classical Greek};

  \draw [arrow] (question2) -- node[anchor=east] {no} (outcome2);
  \draw [arrow] (question2) -- node[anchor=west] {yes} (question3);
  \draw [arrow] (question3) -- node[anchor=east] {no} (outcome3);
  \draw [arrow] (question3) -- node[anchor=west] {yes} (outcome4);
  \end{tikzpicture}\\
  \bottomrule
\end{tabular}
\end{adjustbox}
    \caption{Attested patterns in headless relatives with case competition}
    \label{fig:attested-headless-relatives-case-competition}
\end{figure}

The main focus of Chapter \ref{part:deriving} is the linguistic counterpart of the second parameter. I show with general properties of relative clauses how the difference between the unrestricted and the `unrestricted --- internal-only' type can be modeled, and how the exclusion of the `unrestricted --- external-only' type follows from these particular properties. I also introduce a linguistic counterpart for the first parameter, which distinguishes matching from unrestricted languages.
