% !TEX root = thesis.tex

\chapter{The basic idea}\label{ch:the-basic-idea}

In Chapter \ref{ch:typology} I introduced two descriptive parameters that generate the attested languages, as shown in Figure \ref{fig:two-parameters}.
The first parameter concerns whether the external case is allowed to surface when it wins the case competition (allow \tsc{ext}?). This parameter distinguishes between unrestricted languages (e.g. Old High German) on the one hand and internal-only languages (e.g. Modern German) and matching languages (e.g. Polish) on the other hand.
The second parameter concerns whether the internal case is allowed to surface when it wins the case competition (allow \tsc{int?}). This parameter distinguishes between internal-only languages (e.g. as Modern German) on the one hand and unrestricted languages (e.g. Old High German) on the other hand.

\begin{figure}[htbp]
  \centering
    \footnotesize{
    \begin{tikzpicture}[node distance=1.5cm]
      \node (question2) [question]
      {allow \tsc{int}?};
          \node (outcome2) [outcome, below of=question2, xshift=-1.5cm]
          {matching};
              \node (example2) [example, below of=outcome2, yshift=0.25cm]
              {\scriptsize{e.g. Polish\\\phantom{x}}};
          \node (question3) [question, below of=question2, xshift=2cm, yshift=-0.5cm]
          {allow \tsc{ext}?};
              \node (outcome3) [outcome, below of=question3, xshift=-1.5cm]
              {internal-only};
                  \node (example3) [example, below of=outcome3, yshift=0.25cm]
                  {\scriptsize{e.g. Modern German\\\phantom{x}}};
              \node (outcome4) [outcome, below of=question3, xshift=1.5cm]
              {un-restricted};
                  \node (example4) [example, below of=outcome4, yshift=0.25cm]
                  {\scriptsize{e.g. Gothic, Old High German, Classical Greek}};

    \draw [arrow] (question2) -- node[anchor=east] {no} (outcome2);
    \draw [arrow] (question2) -- node[anchor=west] {yes} (question3);
    \draw [arrow] (question3) -- node[anchor=east] {no} (outcome3);
    \draw [arrow] (question3) -- node[anchor=west] {yes} (outcome4);
    \end{tikzpicture}
    }
    \caption{Two descriptive parameters generate three language types}
    \label{fig:two-parameters}
\end{figure}

``A natural question at this point is whether this typology needs to be fully stipulative, or is to some extent derivable from independent properties of individual languages'' \citet{grosu1994}{147}

In this chapter I show how the typology can be derived from the morphology of the languages.

This chapter is structured as follows.


This chapter gives the basic idea behind my proposal. First, I introduce the the underlying assumptions that I am making. Second, I briefly go through the three options that arise as a consequence of these assumptions. Throughout the rest of the chapter I motivate the proposal, and I illustrate it with examples.

\section{Underlying assumptions}

I start with my assumption that headless relatives are derived from light-headed relatives.\footnote{
The same is argued for headless relatives with \tsc{d}-pronouns in Modern German by \citealt{fuss2014,hanink2018} and for Polish by \citealt{citko2004}.
A difference with Modern German and Polish is that one of the elements can only be absent when the cases match. In Chapter \ref{ch:discussion} I return to the point why Modern German does not have unrestricted headless relatives that look like Old High German, although it still has syncretic light heads and relative pronouns.

Several others claim that headless relatives have a head, but that it is phonologically empty, cf. \citealt{bresnan1978,groos1981,himmelreich2017}.
}
The light head bears the external case, and the relative pronoun bears the internal case, as illustrated in \ref{ex:light+rel}.

\ex. light head\scsub{ext} [\tsc{rp}\scsub{int} ... ]\label{ex:light+rel}

In a headless relative, either the light head or the relative pronoun is absent.

To see what a light-headed relative looks like, consider the Old High German light-headed relative in \ref{ex:ohg-light-headed}. The relative clause, including the relative pronoun, is marked in bold.
\tit{Thér} `\tsc{lh}.\tsc{sg}.\tsc{m}.\tsc{nom}' is the light head of the relative clause. This is the element that appears in the external case, the case that reflects the grammatical role in the main clause.
\tit{Then} `\tsc{rp}.\tsc{sg}.\tsc{m}.\tsc{acc}' is the relative pronoun in the relative clause. This is the element that appears in the internal case, the case that reflects the grammatical role within the relative clause.

\exg. eno nist thiz thér \tbf{then} \tbf{ir} \tbf{suochet} \tbf{zi} \tbf{arslahanne}?\\
 now {not be.3\ac{sg}} \tsc{dem}.\tsc{sg}.\tsc{n}.\tsc{nom} \tsc{lh}.\tsc{sg}.\tsc{m}.\tsc{nom}
 \tsc{rp}.\tsc{sg}.\tsc{m}.\tsc{acc} 2\ac{pl}.\tsc{nom} seek.2\tsc{pl} to kill.\tsc{inf}.\ac{sg}.\tsc{dat}\\
 `Isn't this now the one, who you seek to kill?'\label{ex:ohg-light-headed}

The difference between a light-headed relative and a headless relative is that in a headless relative either the light head or the relative pronoun does not surface.
The surfacing element is the one that bears the winning case, and the absent element is the one that bears the losing case. This means that what I have so far been glossing as the relative pronoun and calling the relative pronoun is actually sometimes the light head (when the relative pronoun is deleted) and sometimes the relative pronoun (when the light head is deleted). To reflect that, I call the surfacing element from now on the surface pronoun.

This brings me to my second assumption, which concerns the circumstances under which the light head or the relative pronoun can be deleted. A light head or a relative pronoun can be deleted when its content can be recovered. The content can be recovered when there is an antecedent which contains the deleted element. %this containment can either be structural, or phonological, so
For light heads and relative pronouns this means that one of them can be absent when it forms (or is syncretic with) a constituent within the other element.\footnote{In Section \ref{sec:three-types} I show constituent containment is also a necessary requirement in other types of deletion operations.}
In other words, it depends on the comparison between the light head and the relative pronoun which one of them is absent. Specifically, it depends on the comparison of the constituents that the two elements consist of. Note that it is also possible that neither of the elements form a constituent within the other one. The consequence is then that neither of them is deleted, which describes the situation in which there is no grammatical headless relative.

In order to be able to compare the light head and the relative pronoun, I zoom in on their syntactic structures. In Chapter \ref{ch:deriving-onlyinternal} to \ref{ch:discussing-unrestricted} I give arguments to support the structures I am assuming here. Figure \ref{fig:rel-lh-intonly} gives a simplified representation of them.

\begin{figure}[htbp]
  \center
  \begin{tabular}[b]{ccc}
      \toprule
      light head & & relative pronoun \\
      \cmidrule(lr){1-1} \cmidrule(lr){3-3}
      \begin{forest} boom
      [\tsc{k}P,
          [\tsc{k}]
          [ϕP, baseline]
      ]
      \end{forest}
      & \phantom{x} &
    \begin{forest} boom
      [\tsc{rel}P
          [\tsc{rel}]
          [\tsc{k}P
              [\tsc{k}]
              [ϕP, baseline]
          ]
      ]
    \end{forest}\\
      \bottomrule
  \end{tabular}
   \caption {\tsc{lh} and \tsc{rp}}
  \label{fig:rel-lh-intonly}
\end{figure}

I assume that the light head and the relative pronoun partly contain the same syntactic features. The features they have in common are case features (\tsc{k}) and what I here simplify as phi-features (ϕ). The light head and the relative pronoun differ from each other in that the relative pronoun has at least one feature in addition, which I call here \tsc{rel}.

This system excludes the external-only type. An external-only type would be one in which the relative pronoun can be deleted, but the light head cannot be deleted. In my proposal, an element can be deleted if forms a constituent within the other element. Relative pronouns always contain one more feature than light heads: \tsc{rel}. From that it follows that the light head does not contain all features that the relative pronoun contains. Therefore, it is impossible for a relative pronoun to form a constituent within the light head.%too strong, effect of syncretism
%this syncretism means that the other way around also always works

My last assumption makes crucial use of Nanosyntax. I assume that the only way in which languages differ is in their lexicon: different languages contain different lexical entries. In Nanosyntax, lexical entries have the form of constitutions, as those are the only objects that can be spelled out. This means that languages differ in how they organize their features into constituents. This has consequences for the deletion of light heads and relative pronouns, since they need to be contained in the other the be deleted. For light heads and relative pronouns the crucial difference is whether phi-features are case features are each spelled out by their own lexical entry or whether they form a portmanteau.

Summing up this section, I make four assumptions. My first assumption is that headless relative clauses are derived from light-headed relatives. Light-headed relatives contain a light head and relative pronoun. In a headless relative either the light head or the relative pronoun is deleted.
My second assumption is that the necessary requirement for deletion is that the deleted element (either the light head or relative pronoun) forms a constituent within the other element. My third assumption is that light heads and relative pronouns contain the same features, and that the relative pronoun contains one additional feature: \tsc{rel}. My last assumption is that the only way in which languages differ is in their lexicon, which contains different constituents with features that correspond to phonology.


\section{The three language types}\label{sec:three-types}

In this I show how different types of constituency leads to different patterns in differences in whether or not the light head and the relative pronoun is deleted, and therefore to different language types. In Chapters \ref{ch:deriving-onlyinternal} to \ref{ch:discussing-unrestricted}, I motivate the constituency I propose for each of the languages, and I go through the derivations in detail.

For each of the three languages types, I first give the constituency within light heads and relative pronouns. Then I compare the constituents of the two elements when they appear in different cases, which is (i) when the cases on the light head and the relative pronoun match, (ii) when the relative pronoun bears the more complex case, and (iii) when the light head bears the more complex case. I show for each of the language types that the correct results are derived in the different situations.

\subsection{The internal-only type}

I start with the internal-only type of language. In this type of language, the constituency is as shown in Figure \ref{fig:rel-lh-intonly-1}.

\begin{figure}[htbp]
  \center
  \begin{tabular}[b]{ccc}
      \toprule
      light head & & relative pronoun \\
      \cmidrule(lr){1-1} \cmidrule(lr){3-3}
      \begin{forest} boom
      [\tsc{k}P,
          [\tsc{k}]
          [ϕP
              [\phantom{xxx}, roof, baseline]
          ]
      ]
      \end{forest}
      & \phantom{x} &
    \begin{forest} boom
      [\tsc{rel}P
          [\tsc{rel}P
              [\phantom{xxx}, roof, baseline]
          ]
          [\tsc{k}P
              [\tsc{k}]
              [ϕP
                  [\phantom{xxx}, roof, baseline]
              ]
          ]
      ]
    \end{forest}\\
      \bottomrule
  \end{tabular}
   \caption {\tsc{lh} and \tsc{rp} in the internal-only type}
  \label{fig:rel-lh-intonly-1}
\end{figure}

The \tsc{k}P is spelled out a whole, including the ϕP. The \tsc{rel}P has its own spellout and is merged as a prefix to the \tsc{k}P. Chapter \ref{ch:deriving-onlyinternal} motivates this analysis.

In Figure \ref{fig:nom-nom-intonly}, I give an example in which the relative pronoun and the light head bear the same case.

\begin{figure}[htbp]
  \center
  \begin{tabular}[b]{ccc}
      \toprule
      light head & & relative pronoun \\
      \cmidrule(lr){1-1} \cmidrule(lr){3-3}
      \begin{forest} boom
        [\tsc{nom}P,
        tikz={
        \node[draw,circle,
        dashed,
        scale=0.85,
        fill=DG,fill opacity=0.2,
        fit to=tree]{};
        }
            [\tsc{f}1]
            [ϕP
                [\phantom{xxx}, roof, baseline]
            ]
        ]
      \end{forest}
      & \phantom{x} &
      \begin{forest} boom
        [\tsc{rel}P
            [\tsc{rel}P
                [\phantom{xxx}, roof, baseline]
            ]
            [\tsc{nom}P,
            tikz={
            \node[draw,circle,
            dashed,
            scale=0.85,
            fit to=tree]{};
            }
                [\tsc{f}1]
                [ϕP
                    [\phantom{xxx}, roof, baseline]
                ]
            ]
        ]
      \end{forest}\\
      \bottomrule
  \end{tabular}
   \caption {\tsc{ext}\scsub{nom} vs. \tsc{int}\scsub{nom} in the internal-only type}
  \label{fig:nom-nom-intonly}
\end{figure}

I draw a dashed circle around each constituent that is a constituent in both the light head and the relative pronoun.
The light head (the \tsc{nom}P) forms a constituent within the relative pronoun (the \tsc{rel}P), so the light head can be deleted. I illustrate this by marking the content of the dashed circles for the light head gray.

In Figure \ref{fig:nom-acc-intonly}, I give an example in which the relative pronoun bears a more complex case than the light head.

\begin{figure}[htbp]
  \center
  \begin{tabular}[b]{ccc}
      \toprule
      light head & & relative pronoun \\
      \cmidrule(lr){1-1} \cmidrule(lr){3-3}
      \begin{forest} boom
        [\tsc{nom}P,
        tikz={
        \node[draw,circle,
        dashed,
        scale=0.85,
        fill=DG,fill opacity=0.2,
        fit to=tree]{};
        }
            [\tsc{f}1]
            [ϕP
                [\phantom{xxx}, roof, baseline]
            ]
        ]
      \end{forest}
      & \phantom{x} &
      \begin{forest} boom
        [\tsc{rel}P
            [\tsc{rel}P
                [\phantom{xxx}, roof, baseline]
            ]
            [\tsc{acc}P
                [\tsc{f}2]
                [\tsc{nom}P,
                tikz={
                \node[draw,circle,
                dashed,
                scale=0.85,
                fit to=tree]{};
                }
                    [\tsc{f}1]
                    [ϕP
                        [\phantom{xxx}, roof, baseline]
                    ]
                ]
            ]
        ]
      \end{forest}\\
      \bottomrule
  \end{tabular}
   \caption {\tsc{ext}\scsub{nom} vs. \tsc{int}\scsub{acc} in the internal-only type}
  \label{fig:nom-acc-intonly}
\end{figure}

I draw a dashed circle around each constituent that is a constituent in both the light head and the relative pronoun.
The light head (the \tsc{nom}P) still forms a constituent within the relative pronoun (the \tsc{rel}P), so the light head can be deleted. I illustrate this by marking the content of the dashed circles for the light head gray.

In Figure \ref{fig:acc-nom-intonly}, I give an example in which the light head bears a more complex case than the relative pronoun.

\begin{figure}[htbp]
  \center
  \begin{tabular}[b]{ccc}
      \toprule
      light head & & relative pronoun \\
      \cmidrule(lr){1-1} \cmidrule(lr){3-3}
      \begin{forest} boom
        [\tsc{acc}P
            [\tsc{f}2]
            [\tsc{nom}P,
            tikz={
            \node[draw,circle,
            dashed,
            scale=0.85,
            fit to=tree]{};
            }
                [\tsc{f}1]
                [ϕP
                    [\phantom{xxx}, roof, baseline]
                ]
            ]
        ]
      \end{forest}
      & \phantom{x} &
      \begin{forest} boom
        [\tsc{rel}P
            [\tsc{rel}P
                [\phantom{xxx}, roof, baseline]
            ]
            [\tsc{nom}P,
            tikz={
            \node[draw,circle,
            dashed,
            scale=0.85,
            fit to=tree]{};
            }
                [\tsc{f}1]
                [ϕP
                    [\phantom{xxx}, roof, baseline]
                ]
            ]
        ]
      \end{forest}\\
      \bottomrule
  \end{tabular}
   \caption {\tsc{ext}\scsub{acc} vs. \tsc{int}\scsub{nom} in the internal-only type}
  \label{fig:acc-nom-intonly}
\end{figure}

I draw a dashed circle around each constituent that is a constituent in both the light head and the relative pronoun.
Different from the examples in Figure \ref{fig:nom-nom-intonly} and \ref{fig:acc-nom-intonly}, the light head does not form a constituent within the relative pronoun.
The \tsc{nom}P of the light head forms a constituent within the relative pronoun, but the relative pronoun does not contain the feature \tsc{f}2 that forms an \tsc{acc}P.
The \tsc{nom}P of the relative pronoun forms a constituent within the relative pronoun, but the light head does not contain the feature \tsc{rel} that forms a \tsc{rel}P.
As a result, none of the elements can be absent. I illustrate this by leaving the content of both dashed circles unfilled.

These comparisons correctly the derive the observed patterns in internal-only languages. An overview of the patterns is shown in Table \ref{tbl:overview-rel-light-mg}.

\begin{table}[htbp]
  \center
  \caption{Grammaticality in the internal-only type}
\begin{tabular}{cc}
  \toprule
                                        & surface pronoun         \\
  \cmidrule(lr){2-2}
\tsc{k}\scsub{int} = \tsc{k}\scsub{ext} & \tsc{rp}\scsub{int/ext} \\
\tsc{k}\scsub{int} > \tsc{k}\scsub{ext} & \tsc{rp}\scsub{int}     \\
\tsc{k}\scsub{int} < \tsc{k}\scsub{ext} & *                       \\
\bottomrule
\end{tabular}
\label{tbl:overview-rel-light-mg}
\end{table}

Headless relatives in internal-only languages are grammatical when the internal and the external case match and when the internal case is more complex than the external case. In these situations, the light head forms a constituent within the relative pronoun, and the light head is deleted. Headless relatives are ungrammatical when the external case is more complex than the internal case, because then the light head no longer forms a constituent within the relative pronoun.


\subsection{The matching type}

I continue with the matching type of language. In this type of language, the constituency is as shown in Figure \ref{fig:rel-lh-matching}.

\begin{figure}[htbp]
  \center
  \begin{tabular}[b]{ccc}
      \toprule
      light head & & relative pronoun \\
      \cmidrule(lr){1-1} \cmidrule(lr){3-3}
      \begin{forest} boom
      [\tsc{k}P
          [ϕP
              [\phantom{xxx}, roof]
          ]
          [\tsc{k}P
              [\tsc{k}, baseline]
          ]
      ]
      \end{forest}
      & \phantom{x} &
    \begin{forest} boom
      [\tsc{rel}P
          [\tsc{rel}P
              [\phantom{xxx}, roof, baseline]
          ]
          [\tsc{k}P
              [ϕP
                  [\phantom{xxx}, roof]
              ]
              [\tsc{k}P
                  [\tsc{k}, baseline]
              ]
          ]
      ]
    \end{forest}\\
      \bottomrule
  \end{tabular}
   \caption {\tsc{lh} and \tsc{rp} in the matching type}
  \label{fig:rel-lh-matching}
\end{figure}

Like in the internal-only type of language, the \tsc{rel}P has its own spellout and is merged as a prefix. The difference between the two language types lies in how the ϕP and the \tsc{k}P are spelled out. In the matching type of language, the ϕP and the \tsc{k}P both correspond to a morpheme, which means that they both form separate constituents. As a result, the ϕP is moved over the \tsc{k}P (allowing the \tsc{k}P to form a constituent on its own, as only constituents can be spelled out). This crucially differs from the internal-only type of language, in which \tsc{k}P and ϕP are spelled out a by a single morpheme and no movement is taking place. In this section I show how this difference leads to different deletion possibilities and, therefore, to different grammaticality patterns. In Chapter \ref{ch:deriving-matching}, I motivate this analysis I put forward for the matching type of language.

In Figure \ref{fig:nom-nom-matching}, I give an example in which the light head and the relative pronoun bear the same case.

\begin{figure}[htbp]
  \center
  \begin{tabular}[b]{ccc}
    \toprule
    light head & & relative pronoun \\
    \cmidrule(lr){1-1} \cmidrule(lr){3-3}
    \begin{forest} boom
      [\tsc{nom}P,
      tikz={
      \node[draw,circle,
      dashed,
      fill=DG,fill opacity=0.2,
      scale=0.8,
      fit to=tree]{};
      }
          [ϕP
              [\phantom{xxx}, roof]
          ]
          [\tsc{nom}P
              [\tsc{f}1, baseline]
          ]
      ]
    \end{forest}
    & \phantom{x} &
    \begin{forest} boom
      [\tsc{rel}P
          [\tsc{rel}P
              [\phantom{xxx}, roof, baseline]
          ]
          [\tsc{nom}P,
          tikz={
          \node[draw,circle,
          dashed,
          scale=0.8,
          fit to=tree]{};
          }
              [ϕP
                  [\phantom{xxx}, roof]
              ]
              [\tsc{nom}P
                  [\tsc{f}1, baseline]
              ]
          ]
      ]
    \end{forest}\\
    \bottomrule
  \end{tabular}
  \caption {\tsc{ext}\scsub{nom} vs. \tsc{int}\scsub{nom} in the matching type}
 \label{fig:nom-nom-matching}
\end{figure}

I draw a dashed circle around each constituent that is a constituent in both the light head and the relative pronoun.
In this instance it is no problem that the ϕP has moved over the \tsc{nom}P.
The light head (the \tsc{nom}P) still forms a constituent within the relative pronoun (the \tsc{rel}P), so the light head can be deleted. I illustrate this by marking the content of the dashed circles for the light head gray.

In Figure \ref{fig:nom-acc-matching}, I give an example in which the relative pronoun bears a more complex case than the light head.

\begin{figure}[htbp]
  \center
  \begin{tabular}[b]{ccc}
    \toprule
    light head & & relative pronoun \\
    \cmidrule(lr){1-1} \cmidrule(lr){3-3}
    \begin{forest} boom
      [\tsc{nom}P, s sep=15mm
          [ϕP,
          tikz={
          \node[draw,circle,
          dashed,
          scale=0.85,
          fit to=tree]{};
          }
              [\phantom{xxx}, roof]
          ]
          [\tsc{nom}P,
          tikz={
          \node[draw,circle,
          dashed,
          scale=0.85,
          fit to=tree]{};
          }
              [\tsc{f}1, baseline]
          ]
      ]
    \end{forest}
    & \phantom{x} &
    \begin{forest} boom
      [\tsc{rel}P
          [\tsc{rel}P
              [\phantom{xxx}, roof, baseline]
          ]
          [\tsc{acc}P
              [ϕP,
              tikz={
              \node[draw,circle,
              dashed,
              scale=0.85,
              fit to=tree]{};
              }
                  [\phantom{xxx}, roof]
              ]
              [\tsc{acc}P
                  [\tsc{f}2]
                  [\tsc{nom}P,
                  tikz={
                  \node[draw,circle,
                  dashed,
                  scale=0.85,
                  fit to=tree]{};
                  }
                      [\tsc{f}1, baseline]
                  ]
              ]
          ]
      ]
    \end{forest}\\
    \bottomrule
  \end{tabular}
  \caption {\tsc{ext}\scsub{nom} vs. \tsc{int}\scsub{acc} in the matching type}
 \label{fig:nom-acc-matching}
\end{figure}

I draw a dashed circle around each constituent that is a constituent in both the light head and the relative pronoun.
The light head (the \tsc{nom}P) no longer forms a constituent within the relative pronoun (the \tsc{rel}P). Therefore, the relative pronoun cannot delete the light head, which I illustrate by leaving the content of both dashed circles unfilled.
It shows that in this instance it is a problem the ϕP has moved over the \tsc{nom}P or \tsc{acc}P.

Something else the example shows is the necessity to formulate the proposal in terms of constituent containment instead of feature containment. To illustrate the difference, I show the example from the internal-only type in which the relative pronoun could delete the light head in Figure \ref{fig:nom-acc-intonly-rep}, repeated from \ref{fig:nom-acc-intonly}.

\begin{figure}[htbp]
  \center
  \begin{tabular}[b]{ccc}
      \toprule
      light head & & relative pronoun \\
      \cmidrule(lr){1-1} \cmidrule(lr){3-3}
      \begin{forest} boom
        [\tsc{nom}P,
        tikz={
        \node[draw,circle,
        dashed,
        scale=0.85,
        fill=DG,fill opacity=0.2,
        fit to=tree]{};
        }
            [\tsc{f}1]
            [ϕP
                [\phantom{xxx}, roof, baseline]
            ]
        ]
      \end{forest}
      & \phantom{x} &
      \begin{forest} boom
        [\tsc{rel}P
            [\tsc{rel}P
                [\phantom{xxx}, roof, baseline]
            ]
            [\tsc{acc}P
                [\tsc{f}2]
                [\tsc{nom}P,
                tikz={
                \node[draw,circle,
                dashed,
                scale=0.85,
                fit to=tree]{};
                }
                    [\tsc{f}1]
                    [ϕP
                        [\phantom{xxx}, roof, baseline]
                    ]
                ]
            ]
        ]
      \end{forest}\\
      \bottomrule
  \end{tabular}
   \caption {\tsc{ext}\scsub{nom} vs. \tsc{int}\scsub{acc} in the internal-only type (repeated)}
  \label{fig:nom-acc-intonly-rep}
\end{figure}

In Figure \ref{fig:nom-acc-intonly-rep}, two different types of containment hold: feature containment and constituent containment.
I start with feature containment. Each feature of the light head (i.e. features contained in ϕP and \tsc{f}1) is also a feature within the relative pronoun. Therefore, the relative pronoun contains the light head.
Constituent containment works as follows. The \tsc{nom}P forms a constituent within the \tsc{rel}P. Therefore, the relative pronoun contains contains the light head.

Consider Figure \ref{fig:nom-acc-matching} again. Here feature containment holds, but constituent containment does not.
The light head and the relative pronoun contain exactly the same features as in \ref{fig:nom-acc-intonly-rep}, so also here each feature of the light head (i.e. features contained in ϕP and \tsc{f}1) is also a feature within the relative pronoun
However, the features are structured differently, in such a way that the light head does no longer form a single constituent within the relative pronoun.

In sum, constituent containment is a stronger requirement than feature containment. Only this stronger requirement is able to distinguish the internal-only from the matching type. Therefore, this account crucially relies on constituent containment being the containment requirement that needs to be fulfilled.

Constituent containment is not only the requirement for deletion in headless relatives. It is also what seems to be crucial in NP ellipsis in general. \citet{cinqueforthcoming} argues that nominal modifiers can only be absent if they form a constituent with the NP. If they do not, they can also not be interpreted and ellipsis is ungrammatical.

In \ref{ex:dutch-houses}, I give an example of a conjunction with two noun phrases from Dutch. The first conjunct consists of a demonstrative, an adjective and a noun, and the second one only of a demonstrative.

\exg. deze witte huizen en die\\
 these white houses and those\\
 `these white houses and those white houses' \flushfill{Dutch}\label{ex:dutch-houses}

In Figure \ref{fig:dutch-houses}, I schematically show the first and second conjunct of \ref{ex:dutch-houses}.

 \begin{figure}[htbp]
   \center
   \begin{tabular}[b]{ccc}
       \toprule
       first conjunct & & second conjunct \\
       \cmidrule(lr){1-1} \cmidrule(lr){3-3}
       \begin{forest} boom
         [WP
             [DemP]
             [YP,
             tikz={
             \node[draw,circle,
             dashed,
             scale=0.85,
             fit to=tree]{};
             }
                 [AP]
                 [NP]
             ]
         ]
       \end{forest}
       & \phantom{x} &
       \begin{forest} boom
         [WP
             [DemP]
             [YP,
             tikz={
             \node[draw,circle,
             dashed,
             fill=DG,fill opacity=0.2,
             scale=0.85,
             fit to=tree]{};
             }
                 [AP]
                 [NP]
             ]
         ]
       \end{forest}\\
       \bottomrule
   \end{tabular}
    \caption {Nominal ellipsis in Dutch}
   \label{fig:dutch-houses}
 \end{figure}

The YP in the second conjunct is the constituent that is deleted. I drew a dashed circle around it, and I marked the content gray. This YP contains the adjective and the noun. The interpretation of the YP in the second conjunct can be recovered, because the YP in the first conjunct served as the antecedent. What is crucial here is that the deleted material forms a single constituent, and that is why it can be recovered.

The situation is different in Kipsigis, a Nilotic Kalenjin language spoken in Kenya. In \ref{ex:kipsigis-houses}, I give an example of a conjunction of two noun phrases in Kipsigis. The first conjunct consists of a noun, a demonstrative and an adjective, and the second one only of a demonstrative \citep{cinqueforthcoming}.

\exg. kaarii-chuun leel-ach ak chu\\
houses-those white-\tsc{pl} and these\\
`those white houses and these houses'\\
not: `those white houses and these white houses'\label{ex:kipsigis-houses} \flushfill{Kipsigis, \pgcitealt{cinqueforthcoming}{24}}

The order between the noun, the demonstrative and the adjective indicates that the NP must have moved (probably cyclically via YP) to the specifier of WP. I show this in \ref{ex:np-dem-adj-kipsigis}.

\ex.\label{ex:np-dem-adj-kipsigis}
\begin{forest} boom
  [WP
      [NP
          [kaarii,name=tgt2]
      ]
      [WP
          [DemP
              [chuun]
          ]
          [YP
              [\sout{NP},
               tikz={
               \node[
               draw,circle,
               scale=0.8,
               fit to=tree]{};
               }
                  [kaarii, name=tgt1]
              ]
              [YP
                  [AP
                      [leel]
                  ]
                  [\sout{NP},
                   tikz={
                   \node[
                   draw,circle,
                   scale=0.8,
                   fit to=tree]{};
                   }
                      [kaarii, name=src]
                  ]
              ]
          ]
      ]
  ]
  \draw[->,dashed] (src) to[out=south west,in=south] (tgt1);
  \draw[->,dashed] (tgt1) to[out=south west,in=south] (tgt2);
\end{forest}

In Figure \ref{fig:kipsigis-houses}, I schematically show the first and second conjunct of \ref{ex:kipsigis-houses}.

\begin{figure}[htbp]
  \center
  \begin{tabular}[b]{ccc}
      \toprule
      first conjunct & & second conjunct \\
      \cmidrule(lr){1-1} \cmidrule(lr){3-3}
      \begin{forest} boom
        [WP
            [NP,
            tikz={
            \node[draw,circle,
            dashed,
            scale=0.85,
            fit to=tree]{};
            }
            ]
            [WP
                [DemP]
                [YP
                    [AP,
                    tikz={
                    \node[draw,circle,
                    dashed,
                    scale=0.85,
                    fit to=tree]{};
                    }
                    ]
                ]
            ]
        ]
      \end{forest}
      & \phantom{x} &
      \begin{forest} boom
        [WP
            [NP,
            tikz={
            \node[draw,circle,
            dashed,
            scale=0.85,
            fit to=tree]{};
            }
            ]
            [WP
                [DemP]
                [YP
                    [AP,
                    tikz={
                    \node[draw,circle,
                    dashed,
                    scale=0.85,
                    fit to=tree]{};
                    }
                    ]
                ]
            ]
        ]
      \end{forest}\\
      \bottomrule
  \end{tabular}
   \caption {Nominal ellipsis in Kipsigis}
   \label{fig:kipsigis-houses}
\end{figure}

Different from in the Dutch example, the adjective and the noun that are deleted in the second conjunct of \ref{ex:kipsigis-houses} do not form a constituent. I draw a dashed circle about the deleted elements and their antecedents in Figure \ref{fig:kipsigis-houses}. Since the adjective and the noun do not form a single constituent together, they cannot be interpreted in the second conjunct of \ref{ex:kipsigis-houses}. Instead, only the noun can be recovered.

This observation regarding NP ellipsis provides independent evidence for my assumption that constituent containment is the crucial requirement for deletion of the light head or the relative pronoun in headless relatives.

I do not give an example in which the light head bears a more complex case than the relative pronoun. The reasoning here is the same as for the internal-only type: both the light head and the relative pronoun contain a feature that the other element does not contain. Since the weaker requirement of feature containment is not met, the stronger requirement of constituent containment cannot be met either.

The comparisons between the constituents within the light heads and the relative pronouns correctly derive the patterns observed in the matching type of language. An overview of the patterns is shown in Table \ref{tbl:overview-rel-light-pol}.

\begin{table}[htbp]
  \center
  \caption{Grammaticality in the matching type}
\begin{tabular}{cc}
  \toprule
                                        & surface pronoun             \\
  \cmidrule(lr){2-2}
\tsc{k}\scsub{int} = \tsc{k}\scsub{ext} & \tsc{rp}\scsub{int} \\
\tsc{k}\scsub{int} > \tsc{k}\scsub{ext} & *                           \\
\tsc{k}\scsub{int} < \tsc{k}\scsub{ext} & *                           \\
\bottomrule
\end{tabular}
\label{tbl:overview-rel-light-pol}
\end{table}

In matching languages, headless relatives are only grammatical when the internal and the external case match. When one of them is more complex than the other one, there is no longer a grammatical outcome possible. This follows from the fact that in matching languages ϕP and \tsc{k}P both both correspond to a morpheme, which means that they both form separate constituents. As a result, the light head only forms a constituent within the relative pronoun when the internal and external case match. When the internal and external case differ, neither of the two forms is contained in the other one.




\subsection{The unrestricted type}

I end with the unrestricted type of language. In this type of language, the constituency is the same as in the internal-only type of language as shown in Figure \ref{fig:rel-lh-intonly-2}.

\begin{figure}[htbp]
  \center
  \begin{tabular}[b]{ccc}
      \toprule
      light head & & relative pronoun \\
      \cmidrule(lr){1-1} \cmidrule(lr){3-3}
      \begin{forest} boom
      [\tsc{k}P,
          [\tsc{k}]
          [ϕP
              [\phantom{xxx}, roof, baseline]
          ]
      ]
      \end{forest}
      & \phantom{x} &
    \begin{forest} boom
      [\tsc{rel}P
          [\tsc{rel}P
              [\phantom{xxx}, roof, baseline]
          ]
          [\tsc{k}P
              [\tsc{k}]
              [ϕP
                  [\phantom{xxx}, roof, baseline]
              ]
          ]
      ]
    \end{forest}\\
      \bottomrule
  \end{tabular}
   \caption {\tsc{lh} and \tsc{rp} in the unrestricted type}
  \label{fig:rel-lh-intonly-2}
\end{figure}

The \tsc{k}P is spelled out a whole, including the ϕP. The \tsc{rel}P has its own spellout and is merged as a prefix to the \tsc{k}P. Chapter \ref{ch:deriving-unrestricted} motivates this analysis.

so for the first two, see up there

The difference between the internal-only type and the unrestricted type lies in when the external case is more complex than the internal case. In the internal-only type this is ungrammatical, and in the unrestricted type, it is grammatical. This means that the light head can delete the relative pronoun when its case is more complex. This is because of a syncretism.

\footnote{
Another option is that the relative pronoun does not actually form a constituent within the light head. Instead, the relative features form a separate constituent which is not deleted. In this chapter I only discuss the situation in which the relative pronoun as a whole forms a constituent within the light head, and the relative pronoun is deleted as a whole. I return to the deletion of separate constituents in Chapter \ref{ch:discussing-unrestricted}.

give a tree here that illustrates it

}

(2) cyclicity



I suggest that there is a syncretism between the phi-features and the phi-features plus the relative features. That is, there is a lexical entry for the \tsc{rel}P which contains the feature \tsc{rel} and feature ϕ, but not a more specific one that spells out ϕ on its own. In \ref{ex:entry-alpha}, I give the lexical entry, which spells out as /α/.

\ex.\label{ex:entry-alpha}
\begin{forest} boom
  [\tsc{rel}P
      [\tsc{rel}P
          [\phantom{xxx}, roof, baseline]
      ]
      [\tsc{k}P
          [\tsc{k}]
          [ϕP
              [\phantom{xxx}, roof, baseline]
          ]
      ]
  ]
  {\draw (.east) node[right]{⇔ \tit{α}}; }
\end{forest}






so there is a stage in the derivation in which the light head is nominative and so is the relative pronoun

\begin{figure}[htbp]
  \center
  \begin{tabular}[b]{ccc}
      \toprule
      light head & & relative pronoun \\
      \cmidrule(lr){1-1} \cmidrule(lr){3-3}
      \begin{forest} boom
            [\tsc{nom}P,
            tikz={
            \node[label=below:\tit{α},
            draw,circle,
            scale=0.8,
            fit to=tree]{};
            \node[draw,circle,
            dashed,
            scale=0.85,
            fit to=tree]{};
            }
                [\tsc{f}1]
                [ϕP
                    [\phantom{xxx}, roof, baseline]
                ]
            ]
        ]
      \end{forest}
      & \phantom{x} &
      \begin{forest} boom
        [\tsc{rel}P,
        tikz={
        \node[label=below:\tit{α},
        draw,circle,
        scale=0.85,
        fill=DG,fill opacity=0.2,
        fit to=tree]{};
        \node[draw,circle,
        dashed,
        scale=0.9,
        fit to=tree]{};
        }
            [\tsc{rel}P
                [\phantom{xxx}, roof, baseline]
            ]
            [\tsc{nom}P
                [\tsc{f}1]
                [ϕP
                    [\phantom{xxx}, roof, baseline]
                ]
            ]
        ]
      \end{forest}\\
      \bottomrule
  \end{tabular}
   \caption {\tsc{ext}\scsub{nom} vs. \tsc{int}\scsub{nom} in the unrestricted type}
  \label{fig:acc-nom-intonly}
\end{figure}



% The ϕP in the light head corresponds to \tit{α}, illustrated by the circle around the ϕP and the \tit{α} under it. The \tsc{rel}P in the relative pronoun corresponds to \tit{α} too, illustrated in the same way.
% The light head (the \tsc{nom}P) consists of the feature \tsc{f}1 and a constituent that corresponds to \tit{α}. This constituent, a \tsc{nom}P that consists of the feature \tsc{f}1 and a constituent that corresponds to \tit{α} is contained in the relative pronoun (the \tsc{acc}P).
% I illustrate this by drawing a dashed circle around each constituent that is a constituent in both the light head and the relative pronoun and by marking the content of the dashed circles for the light head gray.
%
% The ϕP in the light head corresponds to \tit{α}, illustrated by the circle around the ϕP and the \tit{α} under it. The \tsc{rel}P in the relative pronoun corresponds to \tit{α} too, illustrated in the same way.
% The relative pronoun (the \tsc{nom}P) consists of the feature \tsc{f}1 and a constituent that corresponds to \tit{α}. This constituent, a \tsc{nom}P that consists of the feature \tsc{f}1 and a constituent that corresponds to \tit{α} is contained in the relative pronoun (the \tsc{acc}P).
% I illustrate this by drawing a dashed circle around each constituent that is a constituent in both the light head and the relative pronoun and by marking the content of the dashed circles for the relative pronoun gray.
% This means that a light head can delete a relative pronoun when there is a syncretic form between the two, even though the light head lacks the feature \tsc{rel} that the relative pronoun contains.








%%%%%%%

The fact that syncretism licenses deletion is not specific to the syncretism between \tsc{rel}P and ϕP. Syncretism between different cases has the same effect. I illustrate this with an example from Modern German.

Consider the example in \ref{ex:mg-syn}, in which the internal nominative case competes against the external accusative case. The relative clause is marked in bold.
The internal case is nominative, as the predicate \tit{gefällen} `to please' takes nominative subjects.
The external case is accusative, as the predicate \tit{erzählen} `to tell' takes accusative objects.
The relative pronoun \tit{was} `\ac{rel}.\ac{inan}.\tsc{nom/acc}' is syncretic between the nominative and the accusative.

\exg. Ich erzähle \tbf{was} \tbf{immer} \tbf{mir} \tbf{gefällt}.\\
 1\ac{sg}.\ac{nom} tell.\ac{pres}.1\ac{sg}\scsub{[acc]} \tsc{rp}.\ac{inan}.\tsc{nom/acc} ever 1\tsc{sg}.\tsc{dat} pleases.\ac{pres}.3\ac{sg}\scsub{[nom]}\\
 `I tell whatever pleases me.' \flushfill{Modern German, adapted from \pgcitealt{vogel2001}{344}}\label{ex:mg-syn}

There is a syncretism between the nominative and the accusative. That is, there is a lexical entry for the \tsc{acc}P which contains the feature \tsc{f}2 and the \tsc{nom}P, but not a more specific one that spells out the \tsc{nom}P on its own. In \ref{ex:nom-acc-syn}, I give the lexical entry, which spells out as /β/.

\ex.\label{ex:nom-acc-syn}
\begin{forest} boom
  [\tsc{acc}P
      [\tsc{f}2]
      [\tsc{nom}P
          [\tsc{f}1]
          [ϕP
              [\phantom{xxx}, roof]
          ]
      ]
  ]
  {\draw (.east) node[right]{⇔ \tit{β}}; }
\end{forest}

In Figure \ref{fig:acc-nom-syn}, I give an example in which the light head bears a more complex case than the relative pronoun and there is a syncretism between the nominative and the accusative case.

\begin{figure}[htbp]
  \center
  \begin{tabular}[b]{ccc}
      \toprule
      light head & & relative pronoun \\
      \cmidrule(lr){1-1} \cmidrule(lr){3-3}
      \begin{forest} boom
        [\tsc{acc}P,
        tikz={
        \node[label=below:\tit{β},
        draw,circle,
        scale=0.9,
        fill=DG,fill opacity=0.2,
        fit to=tree]{};
        \node[draw,circle,
        dashed,
        scale=0.95,
        fit to=tree]{};
        }
            [\tsc{f}2]
            [\tsc{nom}P
                [\tsc{f}1]
                [ϕP
                    [\phantom{xxx}, roof, baseline]
                ]
            ]
        ]
      \end{forest}
      & \phantom{x} &
      \begin{forest} boom
        [\tsc{rel}P
            [\tsc{rel}P
                [\phantom{xxx}, roof]
            ]
            [\tsc{nom}P,
            tikz={
            \node[draw,circle,
            dashed,
            scale=0.9,
            fit to=tree]{};
            \node[label=below:\tit{β},
            draw,circle,
            scale=0.85,
            fit to=tree]{};
            }
                [\tsc{f}1]
                [ϕP
                    [\phantom{xxx}, roof, baseline]
                ]
            ]
        ]
      \end{forest}\\
      \bottomrule
  \end{tabular}
   \caption {\tsc{ext}\scsub{acc} vs. \tsc{int}\scsub{nom} with case syncretism}
  \label{fig:acc-nom-syn}
\end{figure}

The \tsc{acc}P in the light head corresponds to \tit{β}, illustrated by the circle around the \tsc{acc}P and the \tit{β} under it. The \tsc{nom}P in the relative pronoun corresponds to \tit{β} too, illustrated in the same way.
The light head (the \tsc{acc}P) forms a constituent that corresponds to \tit{β}. A constituent that corresponds to \tit{α} is contained in the relative pronoun (namely the \tsc{nom}P).
I illustrate this by drawing a dashed circle around each constituent that is a constituent in both the light head and the relative pronoun and by marking the content of the dashed circles for the light head gray.
This means that a less complex case can delete a more complex case when there is a syncretic form between the two.





The comparisons between the constituents within the light heads and the relative pronouns correctly derive the patterns observed in the unrestricted type of language.
In unrestricted languages, all types of headless relatives are grammatical: when the internal and the external case match, when the internal case is more complex and when the external case is more complex. I summarize this in \ref{tbl:overview-rel-light-ohg}.

\begin{table}[htbp]
  \center
  \caption{Grammaticality in the matching type}
\begin{tabular}{cc}
  \toprule
                                        & surface pronoun             \\
  \cmidrule(lr){2-2}
\tsc{k}\scsub{int} = \tsc{k}\scsub{ext} & \tsc{rp}\scsub{int/ext}     \\
\tsc{k}\scsub{int} > \tsc{k}\scsub{ext} & \tsc{rp}\scsub{int}         \\
\tsc{k}\scsub{int} < \tsc{k}\scsub{ext} & \tsc{lh}\scsub{ext}         \\
\bottomrule
\end{tabular}
\label{tbl:overview-rel-light-ohg}
\end{table}







\section{Summary}

In summing up this chapter, I return to the metaphor with the committee that I introduced in Chapter \ref{ch:typology}. I wrote that first case competition takes place, in which a more complex case wins over a less complex case. This case competition can now be reformulated into a more general mechanism, namely constituent comparison. A more complex case corresponds to a constituent that contains the constituent of a less complex case.

Subsequently, I noted that there is a committee that can either approve the winning case or not approve it. In Chapter \ref{ch:typology} I wrote that the approval happens based on where the winning case comes from: from inside of the relative clause (internal) or from outside of the relative clause (external). I argued in this chapter that headless relatives are derived from light-headed relatives. The light head bears that external case and the relative pronoun bears the internal case. The `approval' of an internal or external case relies on the same mechanism as case competition, namely constituent comparison.
If the light head forms (or is syncretic with) a constituent within the relative pronoun, the relative pronoun can delete the light head. The light head with its external case is absent, and the relative pronoun with its internal case surfaces. This is what corresponds to the the internal case `being allowed to surface'.
If the relative pronoun is syncretic with a constituent within the light head, the light head can delete the relative pronoun. The relative pronoun with its internal case is absent, and the light head with its external case surfaces. This is what corresponds to the the external case `being allowed to surface'.

In other words, the grammaticality of a headless relative depends on constituent comparison. The constituents that are compared are those of the light head and the relative pronoun, which both bear their own case. Case is special in that it can differ from sentence to sentence within a language. Therefore, the grammaticality of a sentence can differ within a language depending on the internal and external case. The part of the light head and relative pronoun that does not involve case features is stable within a language. Therefore, whether the internal or external case is `allowed to surface' does not differ within a language.

In this dissertation I describe different language types in case competition in headless relatives. In my account, the different language types are a result of a comparison of the light head and the relative pronoun in the language.
The larger syntactic context in which this takes place should be kept stable. The operation that deletes the light head or the relative pronoun is the same for all language types. In this work, I do not specify on which larger syntactic structure and which deletion operation should be used. In Chapter \ref{ch:larger-syntax} I discuss existing proposals on these topics and to what extend they are compatible with my account.

To conclude, in this chapter I introduced the assumptions that headless relatives are derived from light-headed relatives and that relative pronouns contain at least one more feature than light heads. A headless relative is grammatical when either the light head or the relative pronoun forms a constituent within the other element. This set of assumptions derives that only the most complex case can surface and that there is no language of the external-only type.

%at first sight this seems very much related to what Hanink proposes for Modern German. Something is non-pronounced if it contains the features. A crucial difference here is that she formulates it in terms of context sensitive rules, but she does not motivate where these rules come from. I do not have language-specific rules.
