% !TEX root = thesis.tex

\chapter{The source of variation}\label{ch:the-basic-idea}

In Chapter \ref{ch:typology}, I introduced two descriptive parameters that describe the differences between the attested languages. I repeat the overview in Figure \ref{fig:two-parameters}.

\begin{figure}[htbp]
  \centering
  \begin{tabular}[b]{c}
    \toprule
    \begin{tikzpicture}[node distance=1.5cm]
      \node (question2) [question]
      {allow \tsc{int}};
          \node (outcome2) [outcome, below of=question2, xshift=-2cm, yshift=-0.5cm]
          {matching};
              \node (example2) [example, below of=outcome2]
              {e.g. Polish\\\phantom{x}\\\phantom{x}};
          \node (question3) [question, below of=question2, xshift=2.5cm, yshift=-1cm]
          {allow \tsc{ext}};
              \node (outcome3) [outcome, below of=question3, xshift=-2cm, yshift=-0.5cm]
              {internal-only};
                  \node (example3) [example, below of=outcome3]
                  {e.g. Modern German\\\phantom{x}};
              \node (outcome4) [outcome, below of=question3, xshift=2cm, yshift=-0.5cm]
              {unrestricted};
                  \node (example4) [example, below of=outcome4]
                  {e.g. Gothic, Old High German, Classical Greek};

    \draw [arrow] (question2) -- node[anchor=east] {no} (outcome2);
    \draw [arrow] (question2) -- node[anchor=west] {yes} (question3);
    \draw [arrow] (question3) -- node[anchor=east] {no} (outcome3);
    \draw [arrow] (question3) -- node[anchor=west] {yes} (outcome4);
    \end{tikzpicture}\\
    \bottomrule
  \end{tabular}
    \caption{Two descriptive parameters generate three language types (repeated)}
    \label{fig:two-parameters}
\end{figure}

The first parameter, \tit{allow \tsc{int}}, is whether the internal case is allowed to surface when it wins the case competition. This parameter distinguishes the matching type of language from the internal-only and the unrestricted type of languages.
The second parameter, \tit{allow \tsc{ext}}, is whether the external case is allowed to surface when it wins the case competition. This parameter distinguishes the internal-only type of language from the unrestricted type of language.

When the parameters are formulated like this, they describe the different language types, but they are specific to the headless relative construction. Ideally, differences between languages can be derived from independent properties of the language.\footnote{
Exactly this point was raised by in \citet[][147]{grosu1994}:
``A natural question at this point is whether this typology needs to be fully stipulative, or is to some extent derivable from independent properties of individual languages.''
He investigated the correlation between morphological richness and the willingness for a language to show headless relatives. He found a certain tendency, but no absolute rule.
}
I argue that the independent property that causes the derivation is the different lexical entries that are present in different languages. These different lexical entries are the links between lexical trees, phonological representations and conceptual representations, which are part of the language's lexicon.
I call the lexical entries independent properties, because I motivate them by investigating the morphology of the language.

The goal of Part \ref{part:deriving} of this dissertation is to show how different lexical entries lead to differences in language types and to illustrate in detail how this works for the three different language types discussed in Chapter \ref{ch:typology}. The goal of the current chapter is to give the basic idea behind my proposal. In the following three chapters, I work the proposal for the three different language types out in detail, and I give evidence for the lexical entries I propose.

This chapter is structured as follows.
First, I discuss the basic assumptions that I am making, which are the same for each of the discussed language types. Then I introduce the source of the crosslinguistic variation: the lexical entries that are present in the different language types. I show how differences in lexical entries ultimately lead to different language types.


\section{Underlying assumptions}\label{sec:assumptions}

This section lays out the underlying assumptions that I make in my proposal. First, it introduces three assumptions that hold for each of the language types. Then, it discusses how lexical entries lead to differences in languages.

I start with my assumption that headless relatives are derived from relative clauses headed by a light head.\footnote{
The same is argued for headless relatives with \tsc{d}-pronouns in Modern German by \citet{fuss2014,hanink2018} and for Polish by \citet{citko2004}.
Several others claim that headless relatives have a head, but that it is phonologically empty \citep[cf.][]{bresnan1978,groos1981,himmelreich2017}.
}
The light head bears the external case, and the relative pronoun bears the internal case, as illustrated in \ref{ex:light+rel}.

\ex. light head\scsub{ext} [relative pronoun\scsub{int} ... ]\label{ex:light+rel}

In a headless relative, either the light head or the relative pronoun is deleted.

To see what a light-headed relative looks like, consider the Old High German light-headed relative in \ref{ex:ohg-light-headed}. The relative clause, including the relative pronoun, is marked in bold.
\tit{Thér} `\tsc{lh}.\tsc{sg}.\tsc{m}.\tsc{nom}' is the light head of the relative clause. This is the element that appears in the external case, the case that reflects the grammatical role in the main clause.
\tit{Then} `\tsc{rp}.\tsc{sg}.\tsc{m}.\tsc{acc}' is the relative pronoun in the relative clause. This is the element that appears in the internal case, the case that reflects the grammatical role within the relative clause.

\exg. eno nist thiz thér \tbf{then} \tbf{ir} \tbf{suochet} \tbf{zi}
 \tbf{arslahanne}?\\
 now {not be.3\ac{sg}}\scsub{[nom]} \tsc{dem}.\tsc{sg}.\tsc{n}.\tsc{nom} \tsc{lh}.\tsc{sg}.\tsc{m}.\tsc{nom}
 \tsc{rp}.\tsc{sg}.\tsc{m}.\tsc{acc} 2\ac{pl}.\tsc{nom} seek.2\tsc{pl}\scsub{[acc]} to kill.\tsc{inf}.\ac{sg}.\tsc{dat}\\
 `Isn't this now the one, who you seek to kill?' \flushfill{Old High German, \ac{tatian} 349:20}\label{ex:ohg-light-headed}

The difference between a light-headed relative and a headless relative is that in a headless relative either the light head or the relative pronoun does not surface.
The surfacing element is the one that bears the winning case, and the absent element is the one that bears the losing case. This means that what I have so far been glossing as the relative pronoun and calling the relative pronoun is actually sometimes the light head (when the relative pronoun is deleted) and sometimes the relative pronoun (when the light head is deleted). To reflect that, I call the surfacing element from now on the surface element.

This brings me to my second assumption, which concerns the circumstances under which the light head or the relative pronoun can be deleted. A light head or a relative pronoun can be deleted when their content can be recovered. The content can be recovered when there is an antecedent which contains the deleted element. More specifically, the deleted element needs to be contained as a whole within the antecedent.\footnote{
In Section \ref{sec:basic-matching} I show that `containment as a whole' is also a necessary requirement in other types of deletion operations.
}\footnote{
Throughout this chapter I elaborate further on the exact requirements for containment. There are two types of containment possible. The first type is structural containment: an element can be absent if it is structurally contained in the other element. I elaborate on this in Section \ref{sec:basic-matching}. The second type is formal containment: an element can be absent if it is formally contained in the other element. I elaborate on this in Section \ref{sec:basic-unrestricted}.
}
For light heads and relative pronouns this means that one of them can be absent when it is contained in the other element. That is, the light head can be absent when it is contained in the relative pronoun, or the relative pronoun can be absent when it is contained in the light head.
In other words, it depends on the comparison between the light head and the relative pronoun themselves which one of them is absent.
Note that it is also possible that neither of the elements is contained in the other one. The consequence is then that neither of them is deleted, which describes the situation in which there is no grammatical headless relative.

I continue with my third assumption.
In order to be able to compare the light head and the relative pronoun, I zoom in on their internal syntax. In Chapter \ref{ch:deriving-onlyinternal} to \ref{ch:deriving-unrestricted} I give arguments to support the structures I am assuming here. I assume that all languages have two possible light heads.
Figure \ref{fig:rel-lh-intonly-1} gives a simplified representation of the first possible light head and the relative pronoun.

\begin{figure}[htbp]
  \center
  \begin{tabular}[b]{ccc}
      \toprule
      light head 1 & & relative pronoun \\
      \cmidrule(lr){1-1} \cmidrule(lr){3-3}
      \begin{forest} boom
      [\tsc{k}P,
          [\tsc{k}]
          [ϕP, baseline]
      ]
      \end{forest}
      & \phantom{x} &
    \begin{forest} boom
      [\tsc{rel}P
          [\tsc{rel}]
          [\tsc{k}P
              [\tsc{k}]
              [ϕP, baseline]
          ]
      ]
    \end{forest}\\
      \bottomrule
  \end{tabular}
   \caption {\tsc{lh}-1 and \tsc{rp}}
  \label{fig:rel-lh-intonly-1}
\end{figure}

I assume that the first possible light head and the relative pronoun partly contain the same syntactic features. The features they have in common are case features (\tsc{k}) and what I here simplify as phi features (ϕ). The light head and the relative pronoun differ from each other in that the relative pronoun has at least one feature more, which I call here \tsc{rel}.

Figure \ref{fig:rel-lh-intonly-2} gives a simplified representation of the second possible light head and the relative pronoun.

\begin{figure}[htbp]
  \center
  \begin{tabular}[b]{ccc}
      \toprule
      light head 2 & & relative pronoun \\
      \cmidrule(lr){1-1} \cmidrule(lr){3-3}
      \begin{forest} boom
        [XP
            [X]
            [\tsc{rel}P
                [\tsc{rel}]
                [\tsc{k}P
                    [\tsc{k}]
                    [ϕP, baseline]
                ]
            ]
        ]
      \end{forest}
      & \phantom{x} &
    \begin{forest} boom
      [\tsc{rel}P
          [\tsc{rel}]
          [\tsc{k}P
              [\tsc{k}]
              [ϕP, baseline]
          ]
      ]
    \end{forest}\\
      \bottomrule
  \end{tabular}
   \caption {\tsc{lh}-2 and \tsc{rp}}
  \label{fig:rel-lh-intonly-2}
\end{figure}

I assume that the second possible light head and the relative pronoun also partly contain the same syntactic features. The features they have in common are case features (\tsc{k}), phi features (ϕ) and the feature \tsc{rel}. The light head and the relative pronoun differ from each other in that the light head has at least one feature more, which I call here X. In Chapter \ref{ch:deriving-unrestricted} I discuss in detail what X refers to.

The three assumptions I just introduced hold for all language types I discuss. In all language types, headless relatives are derived from light-headed relatives. For all language types, the deletion operation requires containment. And in all language types, there are two possible light heads: the first possible light head contains at least one feature less than the relative pronoun, and the second possible light head contains at least one feature more than the relative pronoun.
The difference between languages does not come from modifying these assumptions in any way, but from how different languages package their features into constituents. Before I explain how differences in internal syntax lead to different language types, I show how different lexical entries lead to differences in internal syntax.

In Chapter \ref{ch:decomposition} I discussed the third person singular feminine pronoun in German. I repeat the lexical entry I gave for it in \ref{ex:german-sie-lexicon-rep}.

\ex.
\begin{forest} boom
  [\ac{acc}P
      [\tsc{k}2]
      [\ac{nom}P
          [\tsc{k}1]
          [3\ac{sg}.\tsc{k}P
              [\phantom{xxx}, roof]
          ]
      ]
  ]
  {\draw (.east) node[right]{⇔ \tit{sie}}; }
\end{forest}
\label{ex:german-sie-lexicon-rep}

The lexical entry corresponds to the pronominal features, \tsc{k}1 and \tsc{k}2 and the phonological form \tit{sie}.

Consider the syntactic structure of the accusative pronoun in German in \ref{ex:german-sie-spellout-rep}.

\ex. \begin{forest} boom
[\ac{acc}P,
tikz={
\node[label=below:\tit{sie},
draw,circle,
scale=0.825,
fit to=tree]{};
}
    [\tsc{k}2]
    [\ac{nom}P
        [\tsc{k}1]
        [3\ac{sg}.\tsc{k}P
            [\phantom{xxx}, roof]
        ]
    ]
]
\end{forest}
\label{ex:german-sie-spellout-rep}

This syntactic structure is contained in the lexical tree in \ref{ex:german-sie-lexicon-rep}, so is spelled out as \tit{sie}.
This means that the accusative pronoun in German is spelled out by a single lexical entry.

The situation is different for the third person singular pronoun in Khanty, which I also showed in Chapter \ref{ch:decomposition}. In Khanty, there is not a single lexical entry that spells out all features that the German lexical entry in \ref{ex:german-sie-lexicon-rep} spells out. Instead, the same features are realized by two separate lexical entries, shown in \ref{ex:khanty-lexicon-rep}.

\ex.\label{ex:khanty-lexicon-rep}
\a.
\begin{forest} boom
  [\ac{nom}P
      [\tsc{k}1]
      [3\ac{sg}P
          [\phantom{xxx}, roof]
      ]
  ]
  {\draw (.east) node[right]{⇔ \tit{luw}}; }
\end{forest}\label{ex:khanty-luw-lexicon-rep}
\b. \begin{forest} boom
  [\ac{acc}P
      [\tsc{k}2]
  ]
  {\draw (.east) node[right]{⇔ \tit{e:l}}; }
\end{forest}\label{ex:khanty-el-lexicon-rep}

The lexical entry in \ref{ex:khanty-luw-lexicon-rep} corresponds to the pronominal features and the feature \tsc{k}1 and the phonological form \tit{luw}. The lexical entry in \ref{ex:khanty-el-lexicon-rep} corresponds to the feature \tsc{k}2 and the phonological form \tit{e:l}.

Consider the syntactic structure of the accusative pronoun in Khanty in \ref{ex:khanty-luw-el-spellout-rep}.

\ex. \begin{forest} boom
[\ac{acc}P, s sep=15mm
    [\ac{nom}P,
    tikz={
    \node[label={below:\tit{luw}},
    draw,circle,
    scale=0.775,
    fit to=tree]{};
    }
        [\tsc{k}1]
        [3\ac{sg}P
            [\phantom{xxx}, roof]
        ]
    ]
    [\ac{acc}P,
    tikz={
    \node[label={below:\tit{e:l}},
    draw,circle,
    scale=0.775,
    fit to=tree]{};
    }
     [\tsc{k}2]
    ]
]
\end{forest}
\label{ex:khanty-luw-el-spellout-rep}

The only available lexical entry in Khanty that contains the \tsc{acc}P is \ref{ex:khanty-el-lexicon-rep}. Nanosyntax only allows constituents to be spelled out, which means that in order to spell out the \tsc{acc}P, the \tsc{nom}P needs to be moved out of the way first.\footnote{
The movement operation is part of the spellout algorithm in Nanosyntax, which is the same for all languages. I elaborate on this spellout algorithm in Chapters \ref{ch:deriving-onlyinternal} and \ref{ch:deriving-matching}.
}
Now compare the syntactic structures of the German accusative pronoun in \ref{ex:german-sie-spellout-rep} and the Khanty one in \ref{ex:khanty-luw-el-spellout-rep}. The feature content is the same (except for the feminine feature, which does not play a role here), but the internal syntax looks different.
This change in internal syntax is a direct consequence of the lexical entries that I gave for the different languages.

Exactly this type of difference is what leads to the different language types in headless relatives. Languages contain different lexical entries that spell out the features of the light heads and the relative pronouns.
The different lexical entries lead to differences in the internal syntax of the light heads and the relative pronouns.
Differences in the internal syntax of the light heads and the relative pronouns lead to differences in whether or not one of them is contained in the other.
Whether or not one of them is contained in the other determines whether or not the light head or relative pronoun can be recovered and, therefore, deleted.
Whether or not the light head or relative pronoun can be deleted determines whether or not there is a single surface element and, with that, a grammatical headless relative.
I summarize this chain in \ref{ex:summary-differences-languages}.

\ex.\label{ex:summary-differences-languages} lexical entries → internal syntax → containment → deletion → surface element

The different language types appear by going through the chain in \ref{ex:summary-differences-languages} in the three different situations: (i) when the internal and external case match, (ii) when the internal case is the more complex case, and (iii) when the external case is the more complex case. An overview of these situation and what (if any) is the surface element is shown in Table \ref{tbl:overview-situations}.

\begin{table}[htbp]
  \center
  \caption{Different language types in different situations}
  \begin{adjustbox}{max width=\textwidth}
  \begin{tabular}{ccc}
    \toprule
  language type &   situation                               & surface element         \\
  \cmidrule(lr){1-1}  \cmidrule(lr){2-2} \cmidrule(lr){3-3}
  unrestricted  &   \tsc{k}\scsub{int} = \tsc{k}\scsub{ext} & \tsc{rp}\scsub{int}\tsc{lh}\scsub{ext} \\
                &   \tsc{k}\scsub{int} > \tsc{k}\scsub{ext} & \tsc{rp}\scsub{int}     \\
                &   \tsc{k}\scsub{int} < \tsc{k}\scsub{ext} & \tsc{lh}\scsub{ext}     \\
                \cmidrule(lr){2-2} \cmidrule(lr){3-3}
  internal-only &   \tsc{k}\scsub{int} = \tsc{k}\scsub{ext} & \tsc{rp}\scsub{int/ext} \\
                &   \tsc{k}\scsub{int} > \tsc{k}\scsub{ext} & \tsc{rp}\scsub{int}     \\
                &   \tsc{k}\scsub{int} < \tsc{k}\scsub{ext} & *                       \\
                \cmidrule(lr){2-2} \cmidrule(lr){3-3}
  matching      &   \tsc{k}\scsub{int} = \tsc{k}\scsub{ext} & \tsc{rp}\scsub{int/ext} \\
                &   \tsc{k}\scsub{int} > \tsc{k}\scsub{ext} & *                       \\
                &   \tsc{k}\scsub{int} < \tsc{k}\scsub{ext} & *                       \\
  \bottomrule
  \end{tabular}
  \end{adjustbox}
  \label{tbl:overview-situations}
  \end{table}

In the unrestricted type of language, the lexical entries are such that there is a grammatical headless relative when the cases match, when the internal case is more complex and when the external case is more complex. When the cases match, the surface element can be either the relative pronoun that bears the internal case or the light head that bears the external case.\footnote{
In Section \ref{sec:basic-unrestricted} I show why the surface element can be both the relative pronoun or the light head.
} When the internal case is more complex, the surface is element is the relative pronoun that bears the internal case. When the external case is more complex, the surface element is the light head that bears the external case.

In the internal-only type of language, the lexical entries are such that there is a grammatical headless relative when the cases match and when the internal case is more complex but not when external case is more complex.\footnote{
In Section \ref{sec:basic-internal} I show why the surface pronoun can only be the relative pronoun.
} When the cases match, the surface element is the relative pronoun that bears the internal (and external) case. When the internal case is more complex, the surface is element is the relative pronoun that bears the internal case.

In the matching type of language, the lexical entries are such that there is a grammatical headless relative when the cases match but not when the internal case is more complex or when the external case is more complex. When the cases match, the surface element is the relative pronoun that bears the internal (and external) case.\footnote{
In Section \ref{sec:basic-matching} I show why the surface pronoun can only be the relative pronoun.
}

In sum, I assume that headless relative clauses are derived from light-headed relatives. Light-headed relatives contain a light head and a relative pronoun. In a headless relative either the light head or the relative pronoun is deleted.
The necessary requirement for deletion is that the deleted element (either the light head or relative pronoun) is structurally or formally contained in the other element.
All languages have two possible light heads, which partly overlap in feature content with the relative pronoun.
The difference between language types arises from languages having different lexical entries that spell out the features of the light heads and the relative pronouns.


\section{The three language types}\label{sec:three-types}

In Chapter \ref{ch:typology} I discussed three different language types. In this section I broadly sketch the kind of lexical entries these language types have that ultimately lead to them being of these types.
For each language type I start with describing the kind of lexical entries they have, and I show the internal syntax that the light head and the relative pronoun have because of that.\footnote{
In this chapter I do not motivate the lexical entries I propose. In chapters \ref{ch:deriving-onlyinternal} to \ref{ch:deriving-unrestricted} I take a concrete example for each language type and I show evidence for the lexical entries I am proposing.}
For each language type, I compare the internal syntax of the light head and the relative pronoun in the three different situations: (i) when the cases of the light head and the relative pronoun match, (ii) when the relative pronoun bears the more complex case, and (iii) when the light head bears the more complex case.
I show that the internal syntax I assume for the light heads and the relative pronouns leads to the different patterns observed in the given language types.


\subsection{The internal-only type}\label{sec:basic-internal}

I start with the internal-only type of language. In Chapter \ref{ch:typology} I showed that Modern German is a language of the internal-only type. Chapter \ref{ch:deriving-onlyinternal} motivates the analysis I propose in this section for Modern German.

In this type of language, grammatical headless relatives can only be derived from light-headed relatives headed by the first possible light head.
Light-headed relatives headed by the second possible light head cannot be the source of headless relatives. For Modern German, I provide evidence for this claim based on interpretation in Chapter \ref{ch:deriving-onlyinternal}. In Chapter \ref{ch:deriving-unrestricted} I give an argument that come from phonology. I already briefly introduce to the phonology argument in Section \ref{sec:basic-unrestricted}.
In this section, I only discuss the first possible light head, and I leave the second possible light head aside.

I suggest that the light head and the relative pronoun in this type of language have the internal syntax as shown in Figure \ref{fig:rel-lh-intonly-3}.

\begin{figure}[htbp]
  \center
  \begin{tabular}[b]{ccc}
      \toprule
      light head & & relative pronoun \\
      \cmidrule(lr){1-1} \cmidrule(lr){3-3}
      \begin{forest} boom
      [\tsc{k}P,
      tikz={
      \node[draw,circle,
      scale=0.85,
      fit to=tree]{};
      }
          [\tsc{k}]
          [ϕP
              [\phantom{xxx}, roof, baseline]
          ]
      ]
      \end{forest}
      & \phantom{x} &
    \begin{forest} boom
      [\tsc{rel}P, s sep=15mm
          [\tsc{rel}P,
          tikz={
          \node[draw,circle,
          scale=0.85,
          fit to=tree]{};
          }
              [\phantom{xxx}, roof, baseline]
          ]
          [\tsc{k}P,
          tikz={
          \node[draw,circle,
          scale=0.85,
          fit to=tree]{};
          }
              [\tsc{k}]
              [ϕP
                  [\phantom{xxx}, roof, baseline]
              ]
          ]
      ]
    \end{forest}\\
      \bottomrule
  \end{tabular}
   \caption {\tsc{lh} and \tsc{rp} in the internal-only type}
  \label{fig:rel-lh-intonly-3}
\end{figure}

This is a consequence of the following lexical entries.
The light head is spelled out by a single lexical entry, indicated by the circle around the \tsc{k}P. This lexical entry is a portmanteau of a phi and case features.
The relative pronoun is spelled out by two lexical entries, indicated by the circles around the \tsc{k}P and the \tsc{rel}P. The phi and case features of the relative pronoun are spelled out by the same portmanteau as the light head is. The \tsc{rel}P is spelled out by a separate lexical entry.
In Chapter \ref{ch:deriving-onlyinternal} I work out this proposal for Modern German, and I give evidence for the lexical entries I suggest here.

In Figure \ref{fig:nom-nom-intonly}, I give an example in which the relative pronoun and the light head bear the same case.

\begin{figure}[htbp]
  \center
  \begin{tabular}[b]{ccc}
      \toprule
      light head & & relative pronoun \\
      \cmidrule(lr){1-1} \cmidrule(lr){3-3}
      \begin{forest} boom
        [\tsc{nom}P,
        tikz={
        \node[draw,circle,
        dashed,
        scale=0.85,
        fill=DG,fill opacity=0.2,
        fit to=tree]{};
        }
            [\tsc{k}1]
            [ϕP
                [\phantom{xxx}, roof, baseline]
            ]
        ]
      \end{forest}
      & \phantom{x} &
      \begin{forest} boom
        [\tsc{rel}P
            [\tsc{rel}P
                [\phantom{xxx}, roof, baseline]
            ]
            [\tsc{nom}P,
            tikz={
            \node[draw,circle,
            dashed,
            scale=0.85,
            fit to=tree]{};
            }
                [\tsc{k}1]
                [ϕP
                    [\phantom{xxx}, roof, baseline]
                ]
            ]
        ]
      \end{forest}\\
      \bottomrule
  \end{tabular}
   \caption {\tsc{ext}\scsub{nom} vs. \tsc{int}\scsub{nom} in the internal-only type}
  \label{fig:nom-nom-intonly}
\end{figure}

I draw a dashed circle around the \tsc{nom}P, as it is the biggest possible element that is contained in both the light head and the relative pronoun.
The light head (the \tsc{nom}P) is contained in the relative pronoun (the \tsc{rel}P), so the light head can be deleted. I illustrate this by marking the content of the dashed circles for the light head gray.
As the light head is deleted, the headless relative surfaces with the relative pronoun that bears the internal case.

In Figure \ref{fig:nom-acc-intonly}, I give an example in which the relative pronoun bears a more complex case than the light head.

\begin{figure}[htbp]
  \center
  \begin{tabular}[b]{ccc}
      \toprule
      light head & & relative pronoun \\
      \cmidrule(lr){1-1} \cmidrule(lr){3-3}
      \begin{forest} boom
        [\tsc{nom}P,
        tikz={
        \node[draw,circle,
        dashed,
        scale=0.85,
        fill=DG,fill opacity=0.2,
        fit to=tree]{};
        }
            [\tsc{k}1]
            [ϕP
                [\phantom{xxx}, roof, baseline]
            ]
        ]
      \end{forest}
      & \phantom{x} &
      \begin{forest} boom
        [\tsc{rel}P
            [\tsc{rel}P
                [\phantom{xxx}, roof, baseline]
            ]
            [\tsc{acc}P
                [\tsc{k}2]
                [\tsc{nom}P,
                tikz={
                \node[draw,circle,
                dashed,
                scale=0.85,
                fit to=tree]{};
                }
                    [\tsc{k}1]
                    [ϕP
                        [\phantom{xxx}, roof, baseline]
                    ]
                ]
            ]
        ]
      \end{forest}\\
      \bottomrule
  \end{tabular}
   \caption {\tsc{ext}\scsub{nom} vs. \tsc{int}\scsub{acc} in the internal-only type}
  \label{fig:nom-acc-intonly}
\end{figure}

I draw a dashed circle around the \tsc{nom}P, as it is the biggest possible element that is contained in both the light head and the relative pronoun.
The light head (the \tsc{nom}P) still is contained in the relative pronoun (the \tsc{rel}P), so the light head can be deleted. I illustrate this by marking the content of the dashed circles for the light head gray.
As the light head is deleted, the headless relative surfaces with the relative pronoun that bears the internal case.

In Figure \ref{fig:acc-nom-intonly}, I give an example in which the light head bears a more complex case than the relative pronoun.

\begin{figure}[htbp]
  \center
  \begin{tabular}[b]{ccc}
      \toprule
      light head & & relative pronoun \\
      \cmidrule(lr){1-1} \cmidrule(lr){3-3}
      \begin{forest} boom
        [\tsc{acc}P
            [\tsc{k}2]
            [\tsc{nom}P,
            tikz={
            \node[draw,circle,
            dashed,
            scale=0.85,
            fit to=tree]{};
            }
                [\tsc{k}1]
                [ϕP
                    [\phantom{xxx}, roof, baseline]
                ]
            ]
        ]
      \end{forest}
      & \phantom{x} &
      \begin{forest} boom
        [\tsc{rel}P
            [\tsc{rel}P
                [\phantom{xxx}, roof, baseline]
            ]
            [\tsc{nom}P,
            tikz={
            \node[draw,circle,
            dashed,
            scale=0.85,
            fit to=tree]{};
            }
                [\tsc{k}1]
                [ϕP
                    [\phantom{xxx}, roof, baseline]
                ]
            ]
        ]
      \end{forest}\\
      \bottomrule
  \end{tabular}
   \caption {\tsc{ext}\scsub{acc} vs. \tsc{int}\scsub{nom} in the internal-only type}
  \label{fig:acc-nom-intonly}
\end{figure}

I draw a dashed circle around the \tsc{nom}P, as it is the biggest possible element that is contained in both the light head and the relative pronoun.
Different from the examples in Figures \ref{fig:nom-nom-intonly} and \ref{fig:acc-nom-intonly}, the light head is not contained in the relative pronoun.
The \tsc{nom}P of the light head is contained in the relative pronoun, but the relative pronoun does not contain the feature \tsc{k}2 that forms an \tsc{acc}P.
The \tsc{nom}P of the relative pronoun is contained in the relative pronoun, but the light head does not contain the feature \tsc{rel} that forms a \tsc{rel}P.
As a result, none of the elements can be absent. I illustrate this by leaving the content of both dashed circles unfilled.
As none of the items is deleted, there is no grammatical headless relative possible.

The comparisons between the light head and the relative pronoun in different cases correctly derive the observed patterns in the internal-only type of language. An overview of the patterns is shown in Table \ref{tbl:overview-rel-light-mg}.

\begin{table}[htbp]
  \center
  \caption{Grammaticality in the internal-only type}
  \begin{adjustbox}{max width=\textwidth}
  \begin{tabular}{cccccc}
    \toprule
  situation           & \multicolumn{2}{c}{lexical entries}       & containment         & deleted             & surfacing           \\
  \cmidrule(lr){1-1}    \cmidrule(lr){2-3}                          \cmidrule(lr){4-4}    \cmidrule(lr){5-5}    \cmidrule(lr){6-6}
                      & \tsc{lh}            & \tsc{rp}            &                     &                     &                     \\
                        \cmidrule(lr){2-2}    \cmidrule(lr){3-3}
  \tsc{k}\scsub{int} = \tsc{k}\scsub{ext}               &
  [\tsc{k}\scsub{1}[ϕ]]                                 &
  [\tsc{rel}], [\tsc{k}\scsub{1}[ϕ]]                    &
  structure & \tsc{lh} & \tsc{rp}\scsub{int}            \\
  \tsc{k}\scsub{int} > \tsc{k}\scsub{ext}               &
  [\tsc{k}\scsub{1}[ϕ]]                                 &
  [\tsc{rel}], [\tsc{k}\scsub{2}[\tsc{k}\scsub{1}[ϕ]]]  &
  structure & \tsc{lh} & \tsc{rp}\scsub{int}            \\
  \tsc{k}\scsub{int} < \tsc{k}\scsub{ext}               &
  [\tsc{k}\scsub{2}[\tsc{k}\scsub{1}[ϕ]]]               &
  [\tsc{rel}], [\tsc{k}\scsub{1}[ϕ]]                    &
  no & none & *                                         \\
  \bottomrule
  \end{tabular}
  \end{adjustbox}
  \label{tbl:overview-rel-light-mg}
  \end{table}

Languages of the internal-only type have a lexical entry that spells out phi and case features and a lexical entry that spells out the feature \tsc{rel}.
Headless relatives in this type of language are grammatical when the internal and the external case match and when the internal case is more complex than the external case. In these situations, the light head is contained in the relative pronoun, the light head is deleted, and the relative pronoun is the surface element. Headless relatives are ungrammatical when the external case is more complex than the internal case, because then the light head no longer is contained in the relative pronoun, and none of the elements is deleted.


\subsection{The matching type}\label{sec:basic-matching}

I continue with the matching type of language. In Chapter \ref{ch:typology} I showed that Polish is a language of the matching type. Chapter \ref{ch:deriving-matching} motivates the analysis I propose in this section for Polish.

In this type of language, grammatical headless relatives can only be derived from light-headed relatives headed by the first possible light head.
Light-headed relatives headed by the second possible light head cannot be the source of headless relatives. For Polish, I provide evidence for this claim based on interpretation in Chapter \ref{ch:deriving-matching}. In Chapter \ref{ch:deriving-unrestricted} I give an argument that come from phonology. I already briefly introduce to the phonology argument in Section \ref{sec:basic-unrestricted}.
In this section, I only discuss the first possible light head, and I leave the second possible light head aside.

I suggest that the light head and the relative pronoun in this type of language have the internal syntax as shown in Figure \ref{fig:rel-lh-matching}.

\begin{figure}[htbp]
  \center
  \begin{tabular}[b]{ccc}
      \toprule
      light head & & relative pronoun \\
      \cmidrule(lr){1-1} \cmidrule(lr){3-3}
      \begin{forest} boom
      [\tsc{k}P, s sep = 20 mm
          [ϕP,
          tikz={
          \node[draw,circle,
          scale=0.85,
          fit to=tree]{};
          }
              [\phantom{xxx}, roof]
          ]
          [\tsc{k}P,
          tikz={
          \node[draw,circle,
          scale=0.85,
          fit to=tree]{};
          }
              [\tsc{k}, baseline]
          ]
      ]
      \end{forest}
      & \phantom{x} &
    \begin{forest} boom
      [\tsc{rel}P, s sep = 15 mm
          [\tsc{rel}P,
          tikz={
          \node[draw,circle,
          scale=0.85,
          fit to=tree]{};
          }
              [\phantom{xxx}, roof, baseline]
          ]
          [\tsc{k}P, s sep = 20 mm
              [ϕP,
              tikz={
              \node[draw,circle,
              scale=0.85,
              fit to=tree]{};
              }
                  [\phantom{xxx}, roof]
              ]
              [\tsc{k}P,
              tikz={
              \node[draw,circle,
              scale=0.85,
              fit to=tree]{};
              }
                  [\tsc{k}, baseline]
              ]
          ]
      ]
    \end{forest}\\
      \bottomrule
  \end{tabular}
   \caption {\tsc{lh} and \tsc{rp} in the matching type}
  \label{fig:rel-lh-matching}
\end{figure}

This is a consequence of the following lexical entries.
The light head is spelled out by two lexical entries: one that spells out the ϕP and one that spells out the \tsc{k}P which does not contain the ϕP. I indicate this by circling the ϕP and the \tsc{k}P. Notice that the ϕP has moved over the \tsc{k}P, which is a direct consequence of the available lexical entries.
Remember that Nanosyntax only allows constituents to be spelled out. \tsc{k}P can only be spelled out if the ϕP is moved out of the way.
This is the crucial difference between the internal-only type of language and the matching type of language: the former has a single lexical entry that spells out both phi and case features and the latter has two separate ones. Exactly this ultimately leads to two different language types.
The relative pronoun in the matching type of language is spelled out by three lexical entries: the ϕP and the \tsc{k}P that are also part of the light head, and in addition the \tsc{rel}P. I indicate this by circling the \tsc{rel}P, the ϕP and the \tsc{k}P.
In Chapter \ref{ch:deriving-matching} I work out this proposal for Polish, and I give evidence for the lexical entries I suggest here.

In Figure \ref{fig:nom-nom-matching}, I give an example in which the light head and the relative pronoun bear the same case.

\begin{figure}[htbp]
  \center
  \begin{tabular}[b]{ccc}
    \toprule
    light head & & relative pronoun \\
    \cmidrule(lr){1-1} \cmidrule(lr){3-3}
    \begin{forest} boom
      [\tsc{nom}P,
      tikz={
      \node[draw,circle,
      dashed,
      fill=DG,fill opacity=0.2,
      scale=0.85,
      fit to=tree]{};
      }
          [ϕP
              [\phantom{xxx}, roof]
          ]
          [\tsc{nom}P
              [\tsc{k}1, baseline]
          ]
      ]
    \end{forest}
    & \phantom{x} &
    \begin{forest} boom
      [\tsc{rel}P
          [\tsc{rel}P
              [\phantom{xxx}, roof, baseline]
          ]
          [\tsc{nom}P,
          tikz={
          \node[draw,circle,
          dashed,
          scale=0.85,
          fit to=tree]{};
          }
              [ϕP
                  [\phantom{xxx}, roof]
              ]
              [\tsc{nom}P
                  [\tsc{k}1, baseline]
              ]
          ]
      ]
    \end{forest}\\
    \bottomrule
  \end{tabular}
  \caption {\tsc{ext}\scsub{nom} vs. \tsc{int}\scsub{nom} in the matching type}
 \label{fig:nom-nom-matching}
\end{figure}

I draw a dashed circle around the \tsc{nom}P, as it is the biggest possible element that is contained in both the light head and the relative pronoun.
In this instance it is no problem that the ϕP has moved over the \tsc{nom}P.
The light head (the \tsc{nom}P) still is contained in the relative pronoun (the \tsc{rel}P), so the light head can be deleted. I illustrate this by marking the content of the dashed circles for the light head gray.
As the light head is deleted, the headless relative surfaces with the relative pronoun that bears the internal case.

In Figure \ref{fig:nom-acc-matching}, I give an example in which the relative pronoun bears a more complex case than the light head.

\begin{figure}[htbp]
  \center
  \begin{tabular}[b]{ccc}
    \toprule
    light head & & relative pronoun \\
    \cmidrule(lr){1-1} \cmidrule(lr){3-3}
    \begin{forest} boom
      [\tsc{nom}P, s sep=15mm
          [ϕP,
          tikz={
          \node[draw,circle,
          dashed,
          scale=0.85,
          fit to=tree]{};
          }
              [\phantom{xxx}, roof]
          ]
          [\tsc{nom}P,
          tikz={
          \node[draw,circle,
          dashed,
          scale=0.85,
          fit to=tree]{};
          }
              [\tsc{k}1, baseline]
          ]
      ]
    \end{forest}
    & \phantom{x} &
    \begin{forest} boom
      [\tsc{rel}P
          [\tsc{rel}P
              [\phantom{xxx}, roof, baseline]
          ]
          [\tsc{acc}P
              [ϕP,
              tikz={
              \node[draw,circle,
              dashed,
              scale=0.85,
              fit to=tree]{};
              }
                  [\phantom{xxx}, roof]
              ]
              [\tsc{acc}P
                  [\tsc{k}2]
                  [\tsc{nom}P,
                  tikz={
                  \node[draw,circle,
                  dashed,
                  scale=0.85,
                  fit to=tree]{};
                  }
                      [\tsc{k}1, baseline]
                  ]
              ]
          ]
      ]
    \end{forest}\\
    \bottomrule
  \end{tabular}
  \caption {\tsc{ext}\scsub{nom} vs. \tsc{int}\scsub{acc} in the matching type}
 \label{fig:nom-acc-matching}
\end{figure}

I draw a dashed circle around the ϕP and the \tsc{nom}P, as they are the biggest possible elements that are contained in both the light head and the relative pronoun.
The light head (the \tsc{nom}P) no longer is contained in the relative pronoun (the \tsc{rel}P). Therefore, the light head cannot be deleted, which I illustrate by leaving the content of both dashed circles unfilled.
As none of the items is deleted, there is no grammatical headless relative possible.
Figure \ref{fig:nom-acc-matching} shows that in this instance it is a problem the ϕP has moved over the \tsc{nom}P or \tsc{acc}P.

Something else the example shows is the necessity to formulate the proposal in terms of structural containment instead of feature containment. To illustrate the difference, I repeat the example from the internal-only type in which the relative pronoun could delete the light head in Figure \ref{fig:nom-acc-intonly-rep} from Figure \ref{fig:nom-acc-intonly}.

\begin{figure}[htbp]
  \center
  \begin{tabular}[b]{ccc}
      \toprule
      light head & & relative pronoun \\
      \cmidrule(lr){1-1} \cmidrule(lr){3-3}
      \begin{forest} boom
        [\tsc{nom}P,
        tikz={
        \node[draw,circle,
        dashed,
        scale=0.85,
        fill=DG,fill opacity=0.2,
        fit to=tree]{};
        }
            [\tsc{k}1]
            [ϕP
                [\phantom{xxx}, roof, baseline]
            ]
        ]
      \end{forest}
      & \phantom{x} &
      \begin{forest} boom
        [\tsc{rel}P
            [\tsc{rel}P
                [\phantom{xxx}, roof, baseline]
            ]
            [\tsc{acc}P
                [\tsc{k}2]
                [\tsc{nom}P,
                tikz={
                \node[draw,circle,
                dashed,
                scale=0.85,
                fit to=tree]{};
                }
                    [\tsc{k}1]
                    [ϕP
                        [\phantom{xxx}, roof, baseline]
                    ]
                ]
            ]
        ]
      \end{forest}\\
      \bottomrule
  \end{tabular}
   \caption {\tsc{ext}\scsub{nom} vs. \tsc{int}\scsub{acc} in the internal-only type (repeated)}
  \label{fig:nom-acc-intonly-rep}
\end{figure}

In Figure \ref{fig:nom-acc-intonly-rep}, two different types of containment hold: feature containment and structural containment.
With feature containment, each feature of the light head (i.e. features contained in ϕP and \tsc{k}1) is also a feature within the relative pronoun. Therefore, the relative pronoun contains the light head.
With structural containment, the \tsc{nom}P is structurally contained in the \tsc{rel}P. Therefore, the relative pronoun contains contains the light head.

Consider Figure \ref{fig:nom-acc-matching} again. Here feature containment holds, but structural containment does not.
The light head and the relative pronoun contain exactly the same features for the light head and the relative pronoun as in Figure \ref{fig:nom-acc-intonly-rep}, so also here each feature of the light head (i.e. features contained in ϕP and \tsc{k}1) is also a feature within the relative pronoun.
However, the features form a different syntactic structure, in such a way that the light head no longer forms a single constituent within the relative pronoun.

In sum, structural containment is a stronger requirement than feature containment. Only this stronger requirement is able to distinguish the internal-only type of language from the matching type of language. Therefore, this account crucially relies on structural containment being the containment requirement that needs to be fulfilled.

Structural containment is not an ad hoc requirement for deletion of a light head or relative pronoun. It is also what seems to be crucial in NP ellipsis in general. \citet{cinqueforthcoming} argues that nominal modifiers can only be absent if they form a constituent with the NP. If they do not, they cannot be deleted while still being interpreted, meaning that ellipsis is ungrammatical. In what follows, I present his argument.

In \ref{ex:dutch-houses}, I give an example of a conjunction with two noun phrases from Dutch. The first conjunct consists of a demonstrative, an adjective and a noun, and the second one of only a demonstrative.

\exg. deze witte huizen en die\\
 these white houses and those\\
 `these white houses and those white houses' \flushfill{Dutch}\label{ex:dutch-houses}

In Figure \ref{fig:dutch-houses}, I schematically show the first and second conjunct of \ref{ex:dutch-houses}.

 \begin{figure}[htbp]
   \center
   \begin{tabular}[b]{ccc}
       \toprule
       first conjunct & & second conjunct \\
       \cmidrule(lr){1-1} \cmidrule(lr){3-3}
       \begin{forest} boom
         [WP, s sep = 15 mm
             [DemP]
             [YP,
             tikz={
             \node[draw,circle,
             dashed,
             scale=0.9,
             fit to=tree]{};
             }
                 [AP]
                 [NP]
             ]
         ]
       \end{forest}
       & \phantom{x} &
       \begin{forest} boom
         [WP, s sep = 15 mm
             [DemP]
             [YP,
             tikz={
             \node[draw,circle,
             dashed,
             fill=DG,fill opacity=0.2,
             scale=0.9,
             fit to=tree]{};
             }
                 [AP]
                 [NP]
             ]
         ]
       \end{forest}\\
       \bottomrule
   \end{tabular}
    \caption {Nominal ellipsis in Dutch}
   \label{fig:dutch-houses}
 \end{figure}

The YP in the second conjunct is the constituent that is deleted. I draw a dashed circle around it, and I mark the content gray. This YP contains the adjective and the noun. The interpretation of the YP in the second conjunct can be recovered, because the YP in the first conjunct serves as the antecedent. What is crucial here is that the deleted material forms a single constituent, and that is why it can be recovered.

The situation is different in Kipsigis, a Nilotic Kalenjin language spoken in Kenya. In \ref{ex:kipsigis-houses}, I give an example of a conjunction of two noun phrases in Kipsigis. The first conjunct consists of a noun, a demonstrative and an adjective, and the second one only of a demonstrative.

\exg. kaarii-chuun leel-ach ak chu\\
houses-those white-\tsc{pl} and these\\
`those white houses and these houses'\\
not: `those white houses and these white houses'\label{ex:kipsigis-houses} \flushfill{Kipsigis, \pgcitealt{cinqueforthcoming}{24}}

The order of the noun, the demonstrative and the adjective indicates that the NP must have moved (probably cyclically via YP) to the specifier of WP. I show this in \ref{ex:np-dem-adj-kipsigis}.

\ex.\label{ex:np-dem-adj-kipsigis}
\begin{forest} boom
  [WP
      [NP
          [\tit{kaarii} `houses',name=tgt2]
      ]
      [WP
          [DemP
              [\tit{chuun} `those']
          ]
          [YP
              [\sout{NP}
                  [\sout{kaarii}, name=tgt1]
              ]
              [YP
                  [AP
                      [\tit{leel} `white']
                  ]
                  [\sout{NP}
                      [\sout{kaarii}, name=src]
                  ]
              ]
          ]
      ]
  ]
  \draw[->,dashed] (src) to[out=south west,in=south] (tgt1);
  \draw[->,dashed] (tgt1) to[out=south west,in=south] (tgt2);
\end{forest}

In Figure \ref{fig:kipsigis-houses}, I schematically show the first and second conjunct of \ref{ex:kipsigis-houses}.

\begin{figure}[htbp]
  \center
  \begin{tabular}[b]{ccc}
      \toprule
      first conjunct & & second conjunct \\
      \cmidrule(lr){1-1} \cmidrule(lr){3-3}
      \begin{forest} boom
        [WP
            [NP,
            tikz={
            \node[draw,circle,
            dashed,
            scale=0.85,
            fit to=tree]{};
            }
            ]
            [WP
                [DemP]
                [YP
                    [AP,
                    tikz={
                    \node[draw,circle,
                    dashed,
                    scale=0.85,
                    fit to=tree]{};
                    }
                    ]
                ]
            ]
        ]
      \end{forest}
      & \phantom{x} &
      \begin{forest} boom
        [WP
            [NP,
            tikz={
            \node[draw,circle,
            dashed,
            scale=0.85,
            fit to=tree]{};
            }
            ]
            [WP
                [DemP]
                [YP
                    [AP,
                    tikz={
                    \node[draw,circle,
                    dashed,
                    scale=0.85,
                    fit to=tree]{};
                    }
                    ]
                ]
            ]
        ]
      \end{forest}\\
      \bottomrule
  \end{tabular}
   \caption {Nominal ellipsis in Kipsigis}
   \label{fig:kipsigis-houses}
\end{figure}

Different from the Dutch example, the adjective and the noun that are deleted in the second conjunct of \ref{ex:kipsigis-houses} do not form a constituent. I draw a dashed circle around the deleted elements and their antecedents in Figure \ref{fig:kipsigis-houses}. Since the adjective and the noun in Figure \ref{fig:kipsigis-houses} do not form a single constituent together, they cannot be interpreted in the second conjunct of \ref{ex:kipsigis-houses}. Instead, only the noun can be recovered.

These data show that structural containment is not only the crucial requirement for deletion of the light head and the relative pronoun in headless relatives. It is also the crucial requirement in NP ellipsis.

Coming back to the matching type of language, I do not give an example in which the light head bears a more complex case than the relative pronoun. The reasoning here is the same as for the internal-only type: both the light head and the relative pronoun contain a feature that the other element does not contain (\tsc{k}2 or \tsc{rel}). Since the weaker requirement of feature containment is not met, the stronger requirement of structural containment cannot be met either. As none of the elements contains the other one, none of them is deleted, and there is no grammatical headless relative possible.

The comparisons between the light head and the relative pronoun in different cases correctly derive the observed patterns in the matching type of language. An overview of the patterns is shown in Table \ref{tbl:overview-rel-light-pol}.

\begin{table}[htbp]
  \center
  \caption{Grammaticality in the matching type}
  \begin{adjustbox}{max width=\textwidth}
  \begin{tabular}{cccccc}
    \toprule
    situation           & \multicolumn{2}{c}{lexical entries}       & containment         & deleted             & surfacing           \\
    \cmidrule(lr){1-1}    \cmidrule(lr){2-3}                          \cmidrule(lr){4-4}    \cmidrule(lr){5-5}    \cmidrule(lr){6-6}
                        & \tsc{lh}            & \tsc{rp}            &                     &                     &                     \\
                          \cmidrule(lr){2-2}    \cmidrule(lr){3-3}
  \tsc{k}\scsub{int} = \tsc{k}\scsub{ext}               &
  [\tsc{k}\scsub{1}], [ϕ]                               &
  [\tsc{rel}], [\tsc{k}\scsub{1}], [ϕ]                  &
  structure & \tsc{lh} & \tsc{rp}\scsub{int}            \\
  \tsc{k}\scsub{int} > \tsc{k}\scsub{ext}               &
  [\tsc{k}\scsub{1}], [ϕ]                               &
  [\tsc{rel}], [\tsc{k}\scsub{2}[\tsc{k}\scsub{1}]], [ϕ]&
  no & none & *                                         \\
  \tsc{k}\scsub{int} < \tsc{k}\scsub{ext}               &
  [\tsc{k}\scsub{2}[\tsc{k}\scsub{1}]], [ϕ]             &
  [\tsc{rel}], [\tsc{k}\scsub{1}], [ϕ]                  &
  no & none & *                                         \\
  \bottomrule
  \end{tabular}
  \end{adjustbox}
\label{tbl:overview-rel-light-pol}
\end{table}

Languages of the matching type have a lexical entry that spells out phi features, a lexical entry that spells out case features and a lexical entry that spells out the feature \tsc{rel}.
Headless relatives in this type of language are only grammatical when the internal and the external case match. In this situation, the light head is structurally contained in the relative pronoun, the light head is deleted, and the relative pronoun is the surface element.
When one of the cases is more complex than the other one, there is no longer a grammatical outcome possible. This follows from the fact that in the matching type of language ϕP and \tsc{k}P are both spelled out by their own lexical entry, which means that they both form separate constituents. As a result, the light head no longer is structurally contained in the relative pronoun, and none of the elements is deleted.

\subsection{The unrestricted type}\label{sec:basic-unrestricted}

I end with the unrestricted type of language. In Chapter \ref{ch:typology} I showed that Old High German is a language of the unrestricted type. Chapter \ref{ch:deriving-unrestricted} motivates the analysis I propose in this section for Old High German.

In this type of language, grammatical headless relatives can be derived from both light-headed relatives headed by the first possible light head and from light-headed relatives headed by the second possible light head.

I suggest that the first possible light head and the relative pronoun in this type of language have the internal syntax as shown in Figure \ref{fig:rel-lh-unres-1}.

\begin{figure}[htbp]
  \center
  \begin{tabular}[b]{ccc}
      \toprule
      light head 1 & & relative pronoun \\
      \cmidrule(lr){1-1} \cmidrule(lr){3-3}
      \begin{forest} boom
      [\tsc{k}P,
      tikz={
      \node[draw,circle,
      scale=0.85,
      fit to=tree]{};
      }
          [\tsc{k}]
          [ϕP
              [\phantom{xxx}, roof, baseline]
          ]
      ]
      \end{forest}
      & \phantom{x} &
    \begin{forest} boom
      [\tsc{rel}P, s sep = 20 mm
          [\tsc{rel}P,
          tikz={
          \node[draw,circle,
          scale=0.85,
          fit to=tree]{};
          }
              [\phantom{xxx}, roof, baseline]
          ]
          [\tsc{k}P,
          tikz={
          \node[draw,circle,
          scale=0.85,
          fit to=tree]{};
          }
              [\tsc{k}]
              [ϕP
                  [\phantom{xxx}, roof, baseline]
              ]
          ]
      ]
    \end{forest}\\
      \bottomrule
  \end{tabular}
   \caption {\tsc{lh}-1 and \tsc{rp} in the unrestricted type}
  \label{fig:rel-lh-unres-1}
\end{figure}

This is a consequence of the following lexical entries, which are exactly the same as they are in the internal-only type of language.
The light head is spelled out by a single lexical entry, indicated by the circle around the \tsc{k}P. This lexical entry is a portmanteau of a phi and case features.
The relative pronoun is spelled out by two lexical entries, indicated by the circles around the \tsc{k}P and the \tsc{rel}P. The phi and case features of the relative pronoun are spelled out by the same portmanteau as the light head is. The \tsc{rel}P is spelled out by a separate lexical entry.
In Chapter \ref{ch:deriving-unrestricted} I work out this proposal for Old High German, and I give evidence for the lexical entries I suggest here.

Because the internal syntax of the light head and the relative pronoun is the same as in the internal-only type of language, the outcomes of the comparison between them in different cases are also the same as in the internal-only type of language. This means that when the internal case and the external case match or when the internal case is more complex than the external case, the light head is structurally contained in the relative pronoun, and the light head is deleted, as shown in Figure \ref{fig:nom-nom-intonly} and Figure \ref{fig:nom-acc-intonly}. This is the pattern that is observed in the unrestricted type of language.

Crucially, the unrestricted type of language differs from the internal-only type of language when the external case is more complex than the internal case. The structures given in Figure \ref{fig:rel-lh-unres-1} cannot lead to a grammatical headless relative, which I have shown in Figure \ref{fig:acc-nom-intonly}. Before I introduce the second possible light head, I investigate whether it is possible to let a more complex external case surface while still keeping the light head but changing something else: a different kind of containment.

I zoom in on the situation in which the external case is more complex.
At first sight, it is unexpected that the light head bearing the external case surfaces to begin with. Recall that the feature content of the light head is that of the relative pronoun minus the feature \tsc{rel}.
So far, I proposed that the light head can be deleted when all of its features are structurally contained in the relative pronoun. This is impossible the other way around: all features of the relative pronoun can never be structurally contained in the light head, because the relative pronoun contains the feature \tsc{rel} that the light head does not. It seems that there is one case that (crosslinguistically) defies this rule: syncretism \citep[cf.]{groos1981,dyta1984,zaenen1984,pullum1986,ingria1990,dalrymple2000,sag2003}.
In what follows I show a situation similar to the missing \tsc{rel} feature: a syncretism between nominative and accusative case in Modern German.
The phenomenon can be understood if we assume that there is a third type of containment: formal containment.

Consider the example in \ref{ex:mg-syn}, in which the internal nominative case competes against the external accusative case. The relative clause is marked in bold.
The internal case is nominative, as the experiencer predicate \tit{gefallen} `to please' takes nominative subjects.
The external case is accusative, as the predicate \tit{erzählen} `to tell' takes accusative objects.
The relative pronoun \tit{was} `\ac{rp}.\ac{inan}.\tsc{nom/acc}' is syncretic between the nominative and the accusative.

\exg. Ich erzähle \tbf{was} \tbf{immer} \tbf{mir} \tbf{gefällt}.\\
 1\ac{sg}.\ac{nom} tell.\ac{pres}.1\ac{sg}\scsub{[acc]} \tsc{rp}.\ac{inan}.\tsc{nom/acc} ever 1\tsc{sg}.\tsc{dat} pleases.\ac{pres}.3\ac{sg}\scsub{[nom]}\\
 `I tell whatever pleases me.' \flushfill{Modern German, adapted from \pgcitealt{vogel2001}{344}}\label{ex:mg-syn}

Remember from Chapter \ref{ch:typology} that Modern German is an internal-only type of language. This means that it allows the internal case to surface when it wins the case competition, but it does not allow the external case to do so. Solely looking at the cases in the example, it is expected that the example is ungrammatical: the internal nominative case cannot win over the external accusative case, and the external case cannot surface because it is not allowed to. However, the example in \ref{ex:mg-syn} is grammatical, because there is a syncretism between the nominative and the accusative in the inanimate gender.

This leads me to distinguish a third type of containment: formal containment. This type of containment holds when an element is formally (i.e. with its phonological form) contained in the other element.
Technically, it works as follows. The fact that there is a syncretism between the nominative and the accusative means that there is a lexical entry for the \tsc{acc}P which contains the feature \tsc{k}2 and the \tsc{nom}P, but not a more specific one that spells out only the \tsc{nom}P. In \ref{ex:nom-acc-syn}, I give such a lexical entry, which spells out as \tit{s}.

\ex.\label{ex:nom-acc-syn}
\begin{forest} boom
  [\tsc{acc}P
      [\tsc{k}2]
      [\tsc{nom}P
          [\tsc{k}1]
          [ϕP
              [\phantom{xxx}, roof]
          ]
      ]
  ]
  {\draw (.east) node[right]{⇔ \tit{s}}; }
\end{forest}

In Figure \ref{fig:acc-nom-syn}, I give the example in which the light head bears a more complex case than the relative pronoun and there is a syncretism between the nominative and the accusative case.

\begin{figure}[htbp]
  \center
  \begin{tabular}[b]{ccc}
      \toprule
      light head & & relative pronoun \\
      \cmidrule(lr){1-1} \cmidrule(lr){3-3}
      \begin{forest} boom
        [\tsc{acc}P,
        tikz={
        \node[label=below:\tit{s},
        draw,circle,
        scale=0.9,
        fit to=tree]{};
        \node[draw,circle,
        dotted,very thick,
        fill=DG,fill opacity=0.2,
        scale=0.95,
        fit to=tree]{};
        }
            [\tsc{k}2]
            [\tsc{nom}P
                [\tsc{k}1]
                [ϕP
                    [\phantom{xxx}, roof, baseline]
                ]
            ]
        ]
      \end{forest}
      & \phantom{x} &
      \begin{forest} boom
        [\tsc{rel}P
            [\tsc{rel}P
                [\phantom{xxx}, roof]
            ]
            [\tsc{nom}P,
            tikz={
            \node[draw,circle,
            dotted,very thick,
            scale=0.9,
            fit to=tree]{};
            \node[label=below:\tit{s},
            draw,circle,
            scale=0.85,
            fit to=tree]{};
            }
                [\tsc{k}1]
                [ϕP
                    [\phantom{xxx}, roof, baseline]
                ]
            ]
        ]
      \end{forest}\\
      \bottomrule
  \end{tabular}
   \caption {\tsc{ext}\scsub{acc} vs. \tsc{int}\scsub{nom} with case syncretism in the internal-only type}
  \label{fig:acc-nom-syn}
\end{figure}

The \tsc{acc}P in the light head corresponds to \tit{s}, illustrated by the circle around the \tsc{acc}P and the \tit{s} below it. The \tsc{nom}P in the relative pronoun corresponds to \tit{s} too, illustrated in the same way.
I draw a dotted circle around the biggest possible element that is formally contained in both the light head and the relative pronoun.
The light head (the \tsc{acc}P realized by \tit{s}) is formally contained in the relative pronoun (the \tsc{nom}P realized by \tit{s}), so the light head is deleted.
I illustrate this by marking the content of the dotted circle for the light head gray.
As the light head is deleted, the headless relative surfaces with the relative pronoun that bears the internal case.

Note here that a deletion based on formal containment happens at the same point in the derivation as a deletion based on structural containment. Remember that spellout in Nanosyntax takes place after each instance of merge. That means that both the structural and the formal information is available at the same point. If there is structural containment, deletion can take place based on structure, and if there is formal containment, deletion can take place on form.

In sum, a more complex case can be deleted when it is syncretic with the less complex case, even though the more complex case contains a case feature more. If that is the case, then a relative pronoun can also be deleted when it is syncretic with the light head, even though the relative pronoun contains at least one feature more. Consider such a situation in \ref{fig:rel-lh-unres-mono}.\footnote{
Note here that the two cases need to match in this situation as well. This can be achieved by making reference to an intermediate step in the derivation, which I explain later on in this section.
}\footnote{
Another option to get a relative pronoun deleted is to let the relative features form a separate constituent which is not deleted.

\ex.\label{ex:gothic-nutshell}
\begin{forest} boom
  [\tsc{k}P, s sep = 15 mm
      [\tsc{k}P,
      tikz={
      \node[draw,circle,
      scale=0.85,
      fit to=tree]{};
      \node[draw,circle,
      dotted,very thick,
      fill=DG,fill opacity=0.2,
      scale=0.9,
      fit to=tree]{};
      }
          [\tsc{k}]
          [ϕP
              [\phantom{xxx}, roof]
          ]
      ]
      [\tsc{rel},
      tikz={
      \node[draw,circle,
      scale=0.85,
      fit to=tree]{};
      }
      ]
  ]
\end{forest}

This is in a nutshell what I assume the analysis for Gothic to be.
In this chapter and in Chapter \ref{ch:deriving-unrestricted} (in which I work out the proposal for Old High German) I only discuss the situation in which the relative pronoun as a whole is formally contained in the light head, and the relative pronoun is deleted.}

\begin{figure}[htbp]
  \center
  \begin{adjustbox}{max width=\textwidth}
  \begin{tabular}[b]{ccc}
      \toprule
      light head & & relative pronoun \\
      \cmidrule(lr){1-1} \cmidrule(lr){3-3}
      \begin{forest} boom
      [\tsc{k}P,
      tikz={
      \node[label=below:\tit{P},
      draw,circle,
      scale=0.85,
      fit to=tree]{};
      \node[draw,circle,
      dotted,very thick,
      scale=0.9,
      fit to=tree]{};
      }
          [\tsc{k}]
          [ϕP
              [\phantom{xxx}, roof, baseline]
          ]
      ]
      \end{forest}
      & \phantom{x} &
    \begin{forest} boom
      [\tsc{rel}P,
      tikz={
      \node[label=below:\tit{P},
      draw,circle,
      scale=0.85,
      fit to=tree]{};
      \node[draw,circle,
      dotted,very thick,
      fill=DG,fill opacity=0.2,
      scale=0.9,
      fit to=tree]{};
      }
          [\tsc{rel}]
          [\tsc{k}P
              [\tsc{k}]
              [ϕP
                  [\phantom{xxx}, roof, baseline]
              ]
          ]
      ]
    \end{forest}\\
      \bottomrule
  \end{tabular}
  \end{adjustbox}
   \caption {Syncretism between \tsc{lh} and \tsc{rp}}
  \label{fig:rel-lh-unres-mono}
\end{figure}

The light head corresponds to \tit{P}, illustrated by the circle around the \tsc{k}P and the \tit{P} below it. The relative pronoun corresponds to \tit{P} too, illustrated by the circle around the \tsc{rel}P and the \tit{P}.
I draw a dotted circle around the biggest possible element that is formally contained in both the light head and the relative pronoun.

The relative pronoun (the \tsc{rel}P realized by \tit{P}) is formally contained in the light head (the \tsc{k}P realized by \tit{P}), so the relative pronoun can be deleted.\footnote{
The same holds the other way around: the \tsc{k}P realized by \tit{P} is formally contained in the \tsc{rel}P that is realized by \tit{P}. Therefore, the light head can be deleted too.
Moreover, there is also structural containment: the \tsc{k}P is structurally contained in the \tsc{rel}P, so the light head can be deleted.
Since I am discussing how it is possible for the relative pronoun to be deleted, I leave this point aside for now.
}
Although in this situation the relative pronoun can be deleted, this does not describe the situation in Old High German, the language I discuss in Chapter \ref{ch:deriving-unrestricted}. I leave it open for future research to find out whether a language like the one described in Figure \ref{fig:rel-lh-unres-mono} exists or not.

In Old High German, the second possible light head that I introduced in Section \ref{sec:assumptions} generates a grammatical headless relative.
Now consider the second possible light head and the relative pronoun in Figure \ref{fig:rel-lh-unres-2}.

\begin{figure}[htbp]
  \center
  \begin{adjustbox}{max width=\textwidth}
  \begin{tabular}[b]{ccc}
      \toprule
      light head 2 & & relative pronoun \\
      \cmidrule(lr){1-1} \cmidrule(lr){3-3}
      \begin{forest} boom
      [XP, s sep = 20 mm
          [XP,
          tikz={
          \node[
          draw,circle,
          scale=0.85,
          fit to=tree]{};
          }
              [X]
              [\tsc{rel}P
                  [\phantom{xxx}, roof, baseline]
              ]
          ]
          [\tsc{k}P,
          tikz={
          \node[draw,circle,
          scale=0.85,
          fit to=tree]{};
          }
              [\tsc{k}]
              [ϕP
                  [\phantom{xxx}, roof, baseline]
              ]
          ]
      ]
      \end{forest}
      & \phantom{x} &
    \begin{forest} boom
      [\tsc{rel}P, s sep = 20 mm
          [\tsc{rel}P,
          tikz={
          \node[
          draw,circle,
          scale=0.85,
          fit to=tree]{};
          }
              [\phantom{xxx}, roof, baseline]
          ]
          [\tsc{k}P,
          tikz={
          \node[draw,circle,
          scale=0.85,
          fit to=tree]{};
          }
              [\tsc{k}]
              [ϕP
                  [\phantom{xxx}, roof, baseline]
              ]
          ]
      ]
    \end{forest}\\
      \bottomrule
  \end{tabular}
  \end{adjustbox}
   \caption {\tsc{lh}-2 and \tsc{rp} in the unrestricted type}
  \label{fig:rel-lh-unres-2}
\end{figure}

As discussed, I propose that this light head does not only consist of phi and case features, but it also contains a feature I here refer to as X. In Chapter \ref{ch:deriving-unrestricted} I motivate this claim and I discuss what X refers to.

The internal syntax of the light head and the relative pronoun is the consequence of the following lexical entries.
The light head is spelled out by two lexical entries.
The feature X and \tsc{rel} are spelled out by their own lexical entry, indicated by the circle around the XP. The rest of the light head is spelled out by the portmanteau of phi and case features.
The relative pronoun is the same as the one I introduced in Figure \ref{fig:rel-lh-unres-1}. It is spelled out by two lexical entries, indicated by the circles around the \tsc{k}P and the \tsc{rel}P. The phi and case features of the relative pronoun are spelled out by the same portmanteau as the light head is. The \tsc{rel}P is spelled out by a separate lexical entry.

It is crucial for the analysis that the XP in the light head and the \tsc{rel}P (that contains the XP) in the relative pronoun have the same spellout. This means that they need to be spelled out by the same lexical entry. I give it in \ref{ex:entry-x}.

\ex.\label{ex:entry-x}
\begin{forest} boom
  [XP
      [X]
      [\tsc{rel}P
          [\phantom{xxx}, roof]
      ]
  ]
  {\draw (.east) node[right]{⇔ \tit{α}}; }
\end{forest}

In Chapter \ref{ch:deriving-unrestricted} I work out this proposal for Old High German, and I give evidence for the lexical entries I suggest here.\footnote{
In Chapter \ref{ch:deriving-onlyinternal} and in Chapter \ref{ch:deriving-matching}, I show that Modern German and Polish also have the second possible light head in their language. In these chapters, I argue that light-headed relatives headed by that head cannot be the source of headless relatives in these languages based on their interpretation.

In Chapter \ref{ch:deriving-unrestricted}, I explain that there is another reason. The crucial reason for Old High German allowing headless relatives being derived from light-headed relatives headed by the second possible light head is that Old High German has a syncretism between the relative pronoun and the second possible light head. Modern German and Polish do not have that syncretism, so the light-headed relative headed by this second possible light head cannot be the source of the headless relative.}

I now return to the problem at hand, being that in the unrestricted type of language a relative pronoun can be deleted.
In Figure \ref{fig:nom-nom-unres}, I give an example in which this can happen. It contains the second possible light head and the relative pronoun, which both bear the same case.

\begin{figure}[htbp]
  \center
  \begin{adjustbox}{max width=\textwidth}
  \begin{tabular}[b]{ccc}
      \toprule
      light head 2 & & relative pronoun \\
      \cmidrule(lr){1-1} \cmidrule(lr){3-3}
      \begin{forest} boom
        [XP, s sep=20mm,
        tikz={
        \node[draw,
        constituent-deletion,yshift=-0.4cm,
        dotted,very thick,
        scale=1.25,
        fit to=tree]{};
        }
            [XP,
            tikz={
            \node[label=below:\tit{α},
            draw,circle,
            scale=0.85,
            fit to=tree]{};
            }
                [X]
                [\tsc{rel}P
                    [\phantom{xxx}, roof, baseline]
                ]
            ]
            [\tsc{nom}P,
            tikz={
            \node[label=below:\tit{β},
            draw,circle,
            scale=0.85,
            fit to=tree]{};
            }
                [\tsc{k}1]
                [ϕP
                    [\phantom{xxx}, roof, baseline]
                ]
            ]
        ]
      \end{forest}
      & \phantom{x} &
      \begin{forest} boom
        [\tsc{rel}P, s sep=20mm,
        tikz={
        \node[draw,
        constituent-deletion,yshift=-0.4cm,
        dotted,very thick,
        fill=DG,fill opacity=0.2,
        scale=1.3,
        fit to=tree]{};
        }
            [\tsc{rel}P,
            tikz={
            \node[label=below:\tit{α},
            draw,circle,
            scale=0.85,
            fit to=tree]{};
            }
                [\phantom{xxx}, roof, baseline]
            ]
            [\tsc{nom}P,
            tikz={
            \node[label=below:\tit{β},
            draw,circle,
            scale=0.85,
            fit to=tree]{};
            }
                [\tsc{k}1]
                [ϕP
                    [\phantom{xxx}, roof, baseline]
                ]
            ]
        ]
      \end{forest}\\
      \bottomrule
  \end{tabular}
  \end{adjustbox}
   \caption {\tsc{ext}\scsub{nom} vs. \tsc{int}\scsub{nom} in the unrestricted type}
  \label{fig:nom-nom-unres}
\end{figure}

The light head corresponds to \tit{αβ}, illustrated by the circle around the XP and the \tit{α} below it and the circle around the \tsc{nom}P and the \tit{β} below it. The relative pronoun corresponds to \tit{αβ} too, illustrated by the circle around the \tsc{rel}P and the \tit{α} below it and the circle around the \tsc{nom}P and the \tit{β} below it.
I draw a dotted circle around the biggest possible element that is formally contained in both the light head and the relative pronoun.
The relative pronoun (the \tsc{rel}P realized by \tit{αβ}) is formally contained in the light head (the XP realized by \tit{αβ}), so the relative pronoun is deleted.
I illustrate this by marking the content of the dotted circle for the relative pronoun gray.

The same holds the other way around: the light head (the XP realized by \tit{αβ}) is formally contained in the relative pronoun (the \tsc{rel}P realized by \tit{αβ}). Therefore, either the light head or the relative pronoun can be deleted. I delete the relative pronoun here, since I am discussing how it is possible for the relative pronoun to be deleted even though it contains one feature less than the light head.\footnote{A possible way to distinguish which of the two elements is deleted is by investigating extraposition possibilities. If the language under investigation can only extrapose CPs not not DPs (just as Modern German, cf. \citealt{vanriemsdijk2006}), it is expected that it is grammatical to extrapose the relative clause that contains the relative pronoun but not the relative clause that contains the light head.}

Finally arriving at the situation in which the external case is more complex than the internal case, I show that the analysis of Figure \ref{fig:nom-nom-unres} cannot simply be extended to this situation.
In Figure \ref{fig:acc-nom-unres} I give an example of the second possible light head and the relative pronoun, in which the light head bears the more complex case.

\begin{figure}[htbp]
  \center
  \begin{adjustbox}{max width=\textwidth}
  \begin{tabular}[b]{ccc}
      \toprule
      light head 2 & & relative pronoun \\
      \cmidrule(lr){1-1} \cmidrule(lr){3-3}
      \begin{forest} boom
        [XP, s sep=20mm
            [XP,
            tikz={
            \node[label=below:\tit{α},
            draw,circle,
            scale=0.85,
            fit to=tree]{};
            \node[draw,circle,
            dotted,very thick,
            scale=0.9,
            fit to=tree]{};
            }
                [X]
                [\tsc{rel}P
                    [\phantom{xxx}, roof, baseline]
                ]
            ]
            [\tsc{acc}P,
            tikz={
            \node[label=below:\tit{ɣ},
            draw,circle,
            scale=0.85,
            fit to=tree]{};
            }
                [\tsc{k}2]
                [\tsc{nom}P
                    [\tsc{k}1]
                    [ϕP
                        [\phantom{xxx}, roof, baseline]
                    ]
                ]
            ]
        ]
      \end{forest}
      & \phantom{x} &
      \begin{forest} boom
        [\tsc{rel}P, s sep=20mm
            [\tsc{rel}P,
            tikz={
            \node[label=below:\tit{α},
            draw,circle,
            scale=0.85,
            fit to=tree]{};
            \node[draw,circle,
            dotted,very thick,
            scale=0.9,
            fit to=tree]{};
            }
                [\phantom{xxx}, roof, baseline]
            ]
            [\tsc{nom}P,
            tikz={
            \node[label=below:\tit{β},
            draw,circle,
            scale=0.85,
            fit to=tree]{};
            }
                [\tsc{k}1]
                [ϕP
                    [\phantom{xxx}, roof, baseline]
                ]
            ]
        ]
      \end{forest}\\
      \bottomrule
  \end{tabular}
  \end{adjustbox}
   \caption {\tsc{ext}\scsub{acc} vs. \tsc{int}\scsub{nom} in the unrestricted type}
  \label{fig:acc-nom-unres}
\end{figure}

The light head corresponds to \tit{αɣ}, illustrated by the circle around the XP and the \tit{α} below it and the circle around the \tsc{acc}P and the \tit{ɣ} below it. The relative pronoun corresponds to \tit{αβ}, illustrated by the circle around the \tsc{rel}P and the \tit{α} below it and the circle around the \tsc{nom}P and the \tit{β} below it.
I draw a dotted circle around the biggest possible element that is formally contained in both the light head and the relative pronoun.
The relative pronoun is no longer formally contained in the light head: \tit{αɣ} does not contain \tit{αβ}. Therefore, the relative pronoun cannot be deleted, which I illustrate by leaving the content of both dotted circles unfilled.
As none of the items is deleted, it is expected that there is no grammatical headless relative possible.\footnote{
I do not consider the option to combine structural and formal containment (i.e. the \tsc{acc}P structurally contains the \tsc{nom}P and \tit{α} formally contains \tit{α}) because I assume that the antecedent needs to contain the deleted element as a whole.
}

However, this is not what is observed in the unrestricted type of language.
For this type of language I need to make an assumption explicit that concerns the larger syntactic structure of headless relatives. I assume that the relative clause is built first, which includes the relative pronoun that bears its case.
At a later stage in the derivation, the light head is built. The last features of the light head that are merged are the case features. Remember that in Nanosyntax, features are merged step by step, and spellout takes place after each instance of merge. This means that there is a stage in the derivation in which the light head bears the nominative case (as in Figure \ref{fig:nom-nom-unres}). At that point, the relative pronoun is deleted. The light head remains as the surface element. Subsequently the feature \tsc{k}2 is merged to the light head to make it a \tsc{acc}P.\footnote{
Thanks to Pavel Caha for suggesting this possibility.
}

This type of derivation is not possible in the situation in which the internal case is more complex than the external case. In that situation, there is no stage in the derivation in which the case of the relative pronoun and the case of the light head match. The relative pronoun is built before the light head, and even at the end of the derivation the light head does not have the more complex case that the relative pronoun has.
In Chapter \ref{ch:deriving-unrestricted} I discuss these derivations in more detail.

Crucially, this deletion option is only successful for languages of the unrestricted type but not for languages of the internal-only or the matching type. In Chapter \ref{ch:deriving-unrestricted} I show why this deletion option only works in the unrestricted type of language and not in the other two types, by giving an argument that concerns phonology.

The comparisons between the first possible light head and the relative pronoun correctly derive the observed patterns for the situation in which cases match and for the situation in which internal case is more complex than the external case. An overview of the patterns is shown in Table \ref{tbl:overview-rel-light-ohg-lh1}.

\begin{table}[htbp]
  \center
  \caption{Grammaticality in the unrestricted type with \tsc{lh}-1}
  \begin{adjustbox}{max width=\textwidth}
  \begin{tabular}{cccccc}
    \toprule
    situation           & \multicolumn{2}{c}{lexical entries}       & containment         & deleted             & surfacing           \\
    \cmidrule(lr){1-1}    \cmidrule(lr){2-3}                          \cmidrule(lr){4-4}    \cmidrule(lr){5-5}    \cmidrule(lr){6-6}
                        & \tsc{lh}-1          & \tsc{rp}            &                     &                     &                     \\
                          \cmidrule(lr){2-2}    \cmidrule(lr){3-3}
  \tsc{k}\scsub{int} = \tsc{k}\scsub{ext}               &
  [\tsc{k}\scsub{1}[ϕ]]                                 &
  [\tsc{rel}], [\tsc{k}\scsub{1}[ϕ]]                    &
  structure & \tsc{lh} & \tsc{rp}\scsub{int}            \\
  \tsc{k}\scsub{int} > \tsc{k}\scsub{ext}               &
  [\tsc{k}\scsub{1}[ϕ]]                                 &
  [\tsc{rel}], [\tsc{k}\scsub{2}[\tsc{k}\scsub{1}[ϕ]]]  &
  structure & \tsc{lh} & \tsc{rp}\scsub{int}            \\
  \tsc{k}\scsub{int} < \tsc{k}\scsub{ext}               &
  [\tsc{rel}], [\tsc{k}\scsub{1}[ϕ]]                    &
  [\tsc{k}\scsub{2}[\tsc{k}\scsub{1}[ϕ]]]               &
  no & none & *                                         \\
  \bottomrule
  \end{tabular}
  \end{adjustbox}
\label{tbl:overview-rel-light-ohg-lh1}
\end{table}

Focusing on the first possible light head, languages of the unrestricted type have a lexical entry that spells out phi and case features and a lexical entry that spells out the feature \tsc{rel}.
Headless relatives in this language are grammatical in all situations: when the internal and the external case match, when the internal case is more complex and when the external case is more complex.
The first possible light head only derives the correct result for the first two situations and not for the last one.
In the first two situations, the light head is structurally contained in the relative pronoun, the light head is deleted, and the relative pronoun is the surface element.
In the last situation, the light head no longer is structurally contained in the relative pronoun, and none of the elements is deleted.

The comparisons between the second possible light head and the relative pronoun correctly derive the observed patterns for the situation in which cases match and for the situation in which external case is more complex than the internal case. An overview of the patterns is shown in Table \ref{tbl:overview-rel-light-ohg-lh2}.

\begin{table}[htbp]
  \center
  \caption{Grammaticality in the unrestricted type with \tsc{lh}-2}
  \begin{adjustbox}{max width=\textwidth}
  \begin{tabular}{cccccc}
    \toprule
    situation           & \multicolumn{2}{c}{lexical entries}       & containment         & deleted             & surfacing           \\
    \cmidrule(lr){1-1}    \cmidrule(lr){2-3}                          \cmidrule(lr){4-4}    \cmidrule(lr){5-5}    \cmidrule(lr){6-6}
                        & \tsc{lh}-2           & \tsc{rp}            &                     &                     &                     \\
                          \cmidrule(lr){2-2}    \cmidrule(lr){3-3}
  \tsc{k}\scsub{int} = \tsc{k}\scsub{ext}               &
  \tit{α}, \tit{β}                                      &
  \tit{α}, \tit{β}                                      &
  form & \tsc{rp} & \tsc{lh}\scsub{ext}                 \\
  \tsc{k}\scsub{int} > \tsc{k}\scsub{ext}               &
  \tit{α}, \tit{β}                                      &
  \tit{α}, \tit{ɣ}                                      &
  no & none & *                                         \\
  \tsc{k}\scsub{int} < \tsc{k}\scsub{ext}               &
  \tit{α}, \tit{β}                                      &
  \tit{α}, \tit{β}                                      &
  form & \tsc{rp} & \tsc{lh}\scsub{ext}                 \\
  \bottomrule
  \end{tabular}
  \end{adjustbox}
\label{tbl:overview-rel-light-ohg-lh2}
\end{table}

Focusing on the second possible light head, languages of the unrestricted type have a lexical entry that spells out phi and case features and a lexical entry that spells out the features X and \tsc{rel} and crucially not a lexical entry that provides a different spellout for only the feature \tsc{rel}.
Headless relatives in this language are grammatical in all situations: when the internal and the external case match, when the internal case is more complex and when the external case is more complex.
The second possible light head only derives the correct result for the first and the last situation but not for the second one.
In the first and last situation, the relative pronoun is (at some point of the derivation) formally contained in the light head, the relative pronoun is deleted, and the light head is the surface element.\footnote{
This means that in the first situation the headless relative can be derived from a light-headed relative with the first possible light head or with the second possible light head. In Section \ref{sec:coming-back} I return to this matter.
}
In the second situation, the relative pronoun is at no point in the derivation formally contained in the light head, and none of the elements is deleted.


\section{Summary}

In summing up this chapter, I return to the metaphor with the committee that I introduced in Chapter \ref{ch:typology}. I wrote that first case competition takes place, in which a more complex case wins over a less complex case. This case competition can now be reformulated into a more general mechanism, namely containment. A more complex case contains a less complex case.

Subsequently, I noted that there is a committee that can either approve the winning case or not approve it. In Chapter \ref{ch:typology} I wrote that the approval happens based on where the winning case comes from: from inside of the relative clause (internal) or from outside of the relative clause (external). I argued in this chapter that headless relatives are derived from light-headed relatives. The light head bears that external case and the relative pronoun bears the internal case. The `approval' of an internal or external case relies on the same mechanism as case competition, namely containment.
If the light head is (structurally) contained in the relative pronoun, the light head can be deleted. Then the light head with its external case is deleted, and the relative pronoun with its internal case surfaces. This is what corresponds to the internal case `being allowed to surface'.
If the relative pronoun is (formally) contained in the light head, the relative pronoun can be deleted. Then the relative pronoun with its internal case is deleted, and the light head with its external case surfaces. This is what corresponds to the external case `being allowed to surface'.

In other words, the grammaticality of a headless relative depends on containment. What is being compared is the internal syntax of the light head and the relative pronoun, which both bear their own case. Case is special in that it can differ from sentence to sentence within a language. Therefore, the grammaticality of a sentence can differ within a language depending on the internal and external case. The part of the light head and relative pronoun that does not involve case features is stable within a language. Therefore, whether the internal or external case is `allowed to surface' does not differ within a language.

The source of variation between languages is the different lexical entries that languages have. The parameters introduced in Chapter \ref{ch:typology} and repeated in the introduction of the chapter can be reformulated as in Figure \ref{fig:lexical-entries}.

\begin{figure}[htbp]
  \centering
  \begin{tabular}[b]{c}
    \toprule
    \begin{tikzpicture}[node distance=1.5cm]
      \node (question2) [question]
      {ϕ+\tsc{k} portmanteau};
          \node (outcome2) [outcome, below of=question2, xshift=-2cm, yshift=-0.5cm]
          {matching};
              \node (example2) [example, below of=outcome2]
              {e.g. Polish\\\phantom{x}\\\phantom{x}};
          \node (question3) [question, below of=question2, xshift=2.5cm, yshift=-1cm]
          {\tsc{lh}-\tsc{rp} syncretism};
              \node (outcome3) [outcome, below of=question3, xshift=-2cm, yshift=-0.5cm]
              {internal-only};
                  \node (example3) [example, below of=outcome3]
                  {e.g. Modern German\\\phantom{x}};
              \node (outcome4) [outcome, below of=question3, xshift=2cm, yshift=-0.5cm]
              {unrestricted};
                  \node (example4) [example, below of=outcome4]
                  {e.g. Gothic, Old High German, Classical Greek};

    \draw [arrow] (question2) -- node[anchor=east] {no} (outcome2);
    \draw [arrow] (question2) -- node[anchor=west] {yes} (question3);
    \draw [arrow] (question3) -- node[anchor=east] {no} (outcome3);
    \draw [arrow] (question3) -- node[anchor=west] {yes} (outcome4);
    \end{tikzpicture}\\
    \bottomrule
  \end{tabular}
    \caption{Different lexical entries generate three language types}
    \label{fig:lexical-entries}
\end{figure}

The first parameter distinguishes the matching type of language from the internal-only and the unrestricted type of languages. The internal-only and unrestricted type of languages have a portmanteau that spells out these two features. The matching type of language does not have that, but it has two separate lexical entries for the phi and case features.
The second parameter distinguishes the internal-only type of language from the unrestricted type of language. The unrestricted type of language has a light head that is syncretic with the relative pronoun. The internal-only type of language does not have such a syncretism.

This system excludes the external-only type. An external-only type would be a language type in which the relative pronoun can be deleted, but the light head cannot be deleted.
In my proposal, an element can be deleted if it is structurally or formally contained in the other element.
First consider only structural containment, leaving formal containment aside for now.
Every language has two possible light heads. The first possible light head contains one feature less than the relative pronoun, and the second possible light head contains one feature more than the relative pronoun.
Since the first possible light head contains one feature less than the relative pronoun, it can never structurally contain the relative pronoun. Therefore, the relative pronoun can never be deleted.

Now consider also formal containment.
Remember that an external-only type of language is a language in which the relative pronoun can be deleted, but the light head cannot be deleted.
In Figure \ref{fig:rel-lh-unres-mono}, I showed a situation in which the light head is syncretic with the relative pronoun, which I repeat here in Figure \ref{fig:rel-lh-unres-mono-rep}.

\begin{figure}[htbp]
  \center
  \begin{adjustbox}{max width=\textwidth}
  \begin{tabular}[b]{ccc}
      \toprule
      light head & & relative pronoun \\
      \cmidrule(lr){1-1} \cmidrule(lr){3-3}
      \begin{forest} boom
      [\tsc{k}P,
      tikz={
      \node[label=below:\tit{P},
      draw,circle,
      scale=0.85,
      fit to=tree]{};
      \node[draw,circle,
      dotted,very thick,
      scale=0.9,
      fit to=tree]{};
      }
          [\tsc{k}]
          [ϕP
              [\phantom{xxx}, roof, baseline]
          ]
      ]
      \end{forest}
      & \phantom{x} &
    \begin{forest} boom
      [\tsc{rel}P,
      tikz={
      \node[label=below:\tit{P},
      draw,circle,
      scale=0.85,
      fit to=tree]{};
      \node[draw,circle,
      dotted,very thick,
      fill=DG,fill opacity=0.2,
      scale=0.9,
      fit to=tree]{};
      }
          [\tsc{rel}]
          [\tsc{k}P
              [\tsc{k}]
              [ϕP
                  [\phantom{xxx}, roof, baseline]
              ]
          ]
      ]
    \end{forest}\\
      \bottomrule
  \end{tabular}
  \end{adjustbox}
   \caption {Syncretism between \tsc{lh} and \tsc{rp} (repeated)}
  \label{fig:rel-lh-unres-mono-rep}
\end{figure}

In Figure \ref{fig:rel-lh-unres-mono-rep}, the relative pronoun is formally contained in the light head, and the relative pronoun can be deleted. Note here that the internal and external case need to be identical too. Only then the two forms are fully syncretic, and deletion can take place.
As I explained at the of Section \ref{sec:basic-unrestricted}, this is a situation that appears when the internal and external cases match, but also when the external case is more complex. In a derivation with a more complex external case, there is always a stage in which the internal and external case match, since the external case features are the last features to be merged with the light head.
When the internal case is more complex, the light head cannot be deleted by formal containment. There is no stage in the derivation in which the internal and external case match and the light head and the relative pronoun are fully syncretic.
However, consider Figure \ref{fig:rel-lh-unres-mono-rep} again.
Although the light head cannot be deleted by formal containment, it can be deleted by structural containment. The light head is still formally contained in the relative pronoun.\footnote{This reasoning holds for monomorphemic light heads and relative pronouns.}

In this dissertation I describe different language types in case competition in headless relatives. In my account, the different language types are a result of a comparison of the light head and the relative pronoun in the language.
The larger syntactic context in which this takes place should be kept stable across languages. The operation that deletes the light head or the relative pronoun is the same for all language types. Therefore, the larger syntactic structure and the deletion operation do not play a central role in the account.
At the end of Chapter \ref{ch:deriving-unrestricted}, the larger syntactic structure of headless relatives enters the discussion when I account for how an external case can win the case competition. There I show where in the larger syntax the (different) light heads are situated and that deletion takes place under c-command.
The deletion operation is optional. However, I show in Chapters \ref{ch:deriving-onlyinternal}, \ref{ch:deriving-matching} and \ref{ch:deriving-unrestricted} that one of the possible light heads cannot be the head of a relative clause for independent reasons, so the source structure is ruled out for independent reasons.

To conclude, in this chapter I introduced the assumptions that headless relatives are derived from light-headed relatives and that relative pronouns contain at least one more feature than light heads. A headless relative is grammatical when either the light head or the relative pronoun is structurally or formally contained in the other element. This set of assumptions derives that only the most complex case can surface and that there is no language of the external-only type.

%at first sight this seems very much related to what Hanink proposes for Modern German. Something is non-pronounced if it contains the features. A crucial difference here is that she formulates it in terms of context sensitive rules, but she does not motivate where these rules come from. I do not have language-specific rules.
