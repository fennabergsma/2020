% !TEX root = thesis.tex

\chapter{Deriving the unrestricted type}\label{ch:deriving-unrestricted}

In Chapter \ref{ch:the-basic-idea}, I suggested that languages of the unrestricted type have two possible light heads, which are part of the derivation under different circumstances. The first possible light head can part of the derivation used when the internal and external case match, and it appears when the internal case is more complex than the external one. The second possible light head can be part of the derivation when the internal and the external case too, and it appears when the external case is more complex than the internal one.

In the first possible light head, the light head corresponds to the phi- and case-feature part of the relative pronoun. The phi- and case-features are spelled out by a portmanteau morpheme, just as they are in the internal-only type of language. This means that the features of the relative pronoun and the light head are spelled out in such a way that they form the constituents shown in Figure \ref{fig:rel-lh-unres-simple}.

\begin{figure}[htbp]
  \center
  \begin{tabular}[b]{ccc}
      \toprule
      light head 1 & & relative pronoun \\
      \cmidrule(lr){1-1} \cmidrule(lr){3-3}
      \begin{forest} boom
      [\tsc{k}P,
      tikz={
      \node[draw,circle,
      scale=0.85,
      fit to=tree]{};
      }
          [\tsc{k}]
          [ϕP
              [\phantom{xxx}, roof, baseline]
          ]
      ]
      \end{forest}
      & \phantom{x} &
    \begin{forest} boom
      [\tsc{rel}P, s sep = 20 mm
          [\tsc{rel}P,
          tikz={
          \node[draw,circle,
          scale=0.85,
          fit to=tree]{};
          }
              [\phantom{xxx}, roof, baseline]
          ]
          [\tsc{k}P,
          tikz={
          \node[draw,circle,
          scale=0.85,
          fit to=tree]{};
          }
              [\tsc{k}]
              [ϕP
                  [\phantom{xxx}, roof, baseline]
              ]
          ]
      ]
    \end{forest}\\
      \bottomrule
  \end{tabular}
   \caption {\tsc{lh}-1 and \tsc{rp} in the unrestricted type}
  \label{fig:rel-lh-unres-simple-1}
\end{figure}

These lexical entries lead to a grammaticality pattern as shown in Table \ref{tbl:overview-unres-1}.

\begin{table}[htbp]
  \center
  \caption{Grammaticality in the unrestricted type (part 1)}
  \begin{adjustbox}{max width=\textwidth}
  \begin{tabular}{cccccc}
    \toprule
    situation           & \multicolumn{2}{c}{lexical entries}       & containment         & deleted             & surfacing           \\
    \cmidrule(lr){1-1}    \cmidrule(lr){2-3}                          \cmidrule(lr){4-4}    \cmidrule(lr){5-5}    \cmidrule(lr){6-6}
                        & \tsc{lh}            & \tsc{rp}            &                     &                     &                     \\
                          \cmidrule(lr){2-2}    \cmidrule(lr){3-3}
  \tsc{k}\scsub{int} = \tsc{k}\scsub{ext}               &
  [\tsc{k}\scsub{1}[ϕ]]                                 &
  [\tsc{rel}], [\tsc{k}\scsub{1}[ϕ]]                    &
  structure & \tsc{lh} & \tsc{rp}\scsub{int}            \\
  \tsc{k}\scsub{int} > \tsc{k}\scsub{ext}               &
  [\tsc{k}\scsub{1}[ϕ]]                                 &
  [\tsc{rel}], [\tsc{k}\scsub{2}[\tsc{k}\scsub{1}[ϕ]]]  &
  structure & \tsc{lh} & \tsc{rp}\scsub{int}            \\
  \bottomrule
  \end{tabular}
  \end{adjustbox}
\label{tbl:overview-unres-1}
\end{table}

First consider the situation in which the internal and the external case match. The situation here is identical to the one in the internal-only type of language. The light head consists of a phi- and case-feature portmanteau. The relative pronoun consists of the same morpheme plus an additional morpheme that spells out the feature \tsc{rel}. These lexical entries create such syntactic structures that the light head structurally forms a constituent within the relative pronoun. Therefore, the light head can be deleted, and the relative pronoun that bears the internal case surfaces.

Consider now the situation in the internal case wins the case competition. Here situation is identical to the one in the internal-only type of language too. The light head consists of a phi- and case-feature portmanteau. The relative pronoun consists of a phi- and case-feature portmanteau that contains at least one more case feature than the light head (\tsc{k}\scsub{2} in Figure \ref{tbl:overview-unres-1}) plus an additional morpheme that spells out the feature \tsc{rel}. These lexical entries create such syntactic structures that the light head structurally forms a constituent within the relative pronoun. Therefore, the light head can be deleted, and the relative pronoun that bears the internal case surfaces.

In Chapter \ref{ch:typology}, I showed that Old High German is a language of the unrestricted type. In this chapter, I show that Old High German has light heads and relative pronouns of type of structure described in Figure \ref{fig:rel-lh-unres-simple-1}. I give a compact version of the structures in Figure \ref{fig:rel-lh-ohg-1}.

\begin{figure}[htbp]
  \center
  \begin{tabular}[b]{ccc}
      \toprule
      light head & & relative pronoun \\
      \cmidrule(lr){1-1} \cmidrule(lr){3-3}
      \begin{forest} boom
        [\tsc{k}P,
        tikz={
        \node[label=below:\tit{n/m},
        draw,circle,
        scale=0.75,
        fit to=tree]{};
        }
            [\tsc{k}]
            [ϕP
                [\phantom{xxx}, roof, baseline]
            ]
        ]
      \end{forest}
      & \phantom{x} &
      \begin{forest} boom
        [\tsc{rel}P, s sep=15mm
            [\tsc{rel}P,
            tikz={
            \node[label=below:\tit{de},
            draw,circle,
            scale=0.75,
            fit to=tree]{};
            }
                [\phantom{xxx}, roof]
            ]
            [\tsc{k}P,
            tikz={
            \node[label=below:\tit{n/m},
            draw,circle,
            scale=0.75,
            fit to=tree]{};
            }
                [\tsc{k}]
                [ϕP
                    [\phantom{xxx}, roof, baseline]
                ]
            ]
        ]
      \end{forest}\\
      \bottomrule
  \end{tabular}
   \caption {\tsc{lh}-1 and \tsc{rp} in Old High German}
  \label{fig:rel-lh-ohg-1}
\end{figure}

Consider the first possible light head in Figure \ref{fig:rel-lh-ohg-1}.
These light heads (i.e. phi- and case-features) in Old High German are spelled out by a single morpheme, indicated by the circle around the structure. They are spelled out as \tit{n} or \tit{m}, depending on which case they realize.
Consider the relative pronoun in Figure \ref{fig:rel-lh-ohg-1}.
Relative pronouns in Old High German consist of two morphemes: the constituent that forms the light head (i.e. phi- and case features) and the \tsc{rel}P, again indicated by the circles. The \tsc{rel}P is spelled out as \tit{de}.
Throughout this chapter, I discuss the exact feature content of relative pronouns and light heads, I give lexical entries for them, and I show how these lexical entries form the constituents shown in Figure \ref{fig:rel-lh-ohg-1}.

In the second possible light head, the light head corresponds to the phi- and case-feature part of the relative pronoun plus an additional feature X. This feature X is also present in the morpheme that spells out the feature \tsc{rel}. The phi- and case-features are still spelled out by a portmanteau morpheme. The feature X is spelled out by a separate morpheme, which is the same morpheme that spells out X plus the feature \tsc{rel}. This means that the features of the relative pronoun and the light head are spelled out in such a way that they form the constituents shown in Figure \ref{fig:rel-lh-unres-simple-2}.

\begin{figure}[htbp]
  \center
  \begin{adjustbox}{max width=\textwidth}
  \begin{tabular}[b]{ccc}
      \toprule
      light head 2 & & relative pronoun \\
      \cmidrule(lr){1-1} \cmidrule(lr){3-3}
      \begin{forest} boom
      [XP, s sep = 20 mm
          [XP,
          tikz={
          \node[label=below:\tit{X},
          draw,circle,
          scale=0.85,
          fit to=tree]{};
          }
              [\phantom{xxx}, roof, baseline]
          ]
          [\tsc{k}P,
          tikz={
          \node[draw,circle,
          scale=0.85,
          fit to=tree]{};
          }
              [\tsc{k}]
              [ϕP
                  [\phantom{xxx}, roof, baseline]
              ]
          ]
      ]
      \end{forest}
      & \phantom{x} &
    \begin{forest} boom
      [\tsc{rel}P, s sep = 20 mm
          [\tsc{rel}P,
          tikz={
          \node[label=below:\tit{X},
          draw,circle,
          scale=0.85,
          fit to=tree]{};
          }
              [\tsc{rel}]
              [XP
                  [\phantom{xxx}, roof, baseline]
              ]
          ]
          [\tsc{k}P,
          tikz={
          \node[draw,circle,
          scale=0.85,
          fit to=tree]{};
          }
              [\tsc{k}]
              [ϕP
                  [\phantom{xxx}, roof, baseline]
              ]
          ]
      ]
    \end{forest}\\
      \bottomrule
  \end{tabular}
  \end{adjustbox}
   \caption {\tsc{lh}-2 and \tsc{rp} in the unrestricted type}
  \label{fig:rel-lh-unres-simple-2}
\end{figure}

First consider the situation in which the internal and the external case match. The light head consists of a phi- and case-feature portmanteau and a morpheme that spells out the feature X, which corresponds to phonological form \tit{X}. The relative pronoun consists of the same phi- and case-feature morpheme and a morpheme that spells out the feature X and the feature \tsc{rel}, which corresponds to the phonological form \tit{X} too. When the internal and the external case match, the phonological form corresponding to the phi- and case features is identical between the light head and the relative pronoun too. These lexical entries create such syntactic structures that the light head and the relative pronoun are formally identical. Since there is formal containment, one of the elements can be deleted, and the the other one surfaces with its case.

Consider now the situation in the external case wins the case competition. The light head consists of a phi- and case-feature portmanteau and a morpheme that spells out the feature X, which corresponds to phonological form \tit{X}. The relative pronoun consists of the same phi- and case-feature morpheme and a morpheme that spells out the feature X and the feature \tsc{rel}, which corresponds to the phonological form \tit{X} too. When the external case is more complex than the internal case (i.e. when the two cases differ), the phonological forms corresponding to the phi- and case features of the light head and the relative pronoun differ. However, the derivation in which the external case is more complex than the internal one goes through a stage in which the internal and the external case match. Therefore, at that stage, these lexical entries form such syntactic structures that the light head and the relative pronoun are formally identical. Since there is formal containment, one of the elements can be deleted, and the the other one surfaces with its case. Then, the more complex case is merged to the remaining element.

In Chapter \ref{ch:typology}, I showed that Old High German is a language of the unrestricted type. In this chapter, I show that Old High German has light heads and relative pronouns of type of structure described in Figure \ref{fig:rel-lh-unres-simple-2}. I give a compact version of the structures in Figure \ref{fig:rel-lh-ohg-2}.

\begin{figure}[htbp]
  \center
  \begin{tabular}[b]{ccc}
      \toprule
      light head & & relative pronoun \\
      \cmidrule(lr){1-1} \cmidrule(lr){3-3}
      \begin{forest} boom
      [XP, s sep = 20 mm
          [XP,
          tikz={
          \node[label=below:\tit{de},
          draw,circle,
          scale=0.85,
          fit to=tree]{};
          }
              [\phantom{xxx}, roof, baseline]
          ]
          [\tsc{k}P,
          tikz={
          \node[label=below:\tit{n/m},
          draw,circle,
          scale=0.85,
          fit to=tree]{};
          }
              [\tsc{k}]
              [ϕP
                  [\phantom{xxx}, roof, baseline]
              ]
          ]
      ]
      \end{forest}
      & \phantom{x} &
      \begin{forest} boom
        [\tsc{rel}P, s sep = 20 mm
            [\tsc{rel}P,
            tikz={
            \node[label=below:\tit{de},
            draw,circle,
            scale=0.85,
            fit to=tree]{};
            }
                [\tsc{rel}]
                [XP
                    [\phantom{xxx}, roof, baseline]
                ]
            ]
            [\tsc{k}P,
            tikz={
            \node[label=below:\tit{n/m},
            draw,circle,
            scale=0.75,
            fit to=tree]{};
            }
                [\tsc{k}]
                [ϕP
                    [\phantom{xxx}, roof, baseline]
                ]
            ]
        ]
      \end{forest}\\
      \bottomrule
  \end{tabular}
   \caption {\tsc{lh}-2 and \tsc{rp} in Old High German}
  \label{fig:rel-lh-ohg-2}
\end{figure}

Consider the first possible light head in Figure \ref{fig:rel-lh-ohg-2}.
These light heads in Old High German are spelled out by two morphemes, which are both circled. The morpheme that realizes the feature X is spelled out as \tit{de} and the phi-and case-features are spelled out as \tit{n} or \tit{m}, depending on which case they realize.

Consider the relative pronoun in Figure \ref{fig:rel-lh-ohg-2}.
Relative pronouns in Old High German are spelled out by the same two morphemes as the second possible light head. There is the phi- and case-feature morpheme (spelling out as \tit{n} or \tit{m}) and the morpheme that spells out the feature X and in the relative pronoun also the feature \tsc{rel} (spelling out as \tit{de}).
Throughout this chapter, I discuss the exact feature content of relative pronouns and light heads, I give lexical entries for them, and I show how these lexical entries form the constituents shown in Figure \ref{fig:rel-lh-ohg-2}.


This chapter is structured as follows.
First, I discuss the relative pronoun. I decompose the relative pronouns into the two morphemes I showed in Figure \ref{fig:rel-lh-ohg-1} and \ref{fig:rel-lh-ohg-2}, and I show which features each of the morphemes corresponds to. I illustrate how different morphemes are combined into the given constituents.
Then I discuss the two possible light heads. I argue that headless relatives in Old High German can be derived from two different types of light-headed relative clauses. The first type actually surfaces in the language. The second one is the type of light-headed relative clause that does not surface in the language, just like those in Modern German and Polish. %this makes a prediction that I will test
Finally, I compare the constituents of the light heads and the relative pronoun. I show that the first possible light head can be deleted when the internal case matches the external case or when the internal case is more complex than the external case via structural containment. The second possible light head can be deleted when the internal case matches the external case or when the internal case is more complex than the external case via formal containment. %some comments about the larger syntactic structure

\section{The Old High German German relative pronoun}

- relative pronoun, show that it's a D

What is different here, is that the relative pronoun is a \tsc{d}-pronoun instead of a \tsc{wh}.

Relative and demonstrative pronouns are syncretic in Old High German \pgcitep{braune2018}{338}. Table \ref{tbl:rel-dem-ohg} gives an overview of the forms in singular and plural, neuter, masculine and feminine and nominative, accusative and dative. The pronouns consist of two morphemes: a \tit{d} and suffix that differs per number, gender and case.\footnote{
\tit{d} can also be written as \tit{dh} and \tit{th}, \tit{ë} and \tit{ē} can also be \tit{e} and \tit{é} \pgcitep{braune2018}{339}.
}\footnote{
The suffix could also be further divided into a vowel and a suffix. As this is not relevant for the discussion here, I refrain from doing that.
}

\begin{table}[htbp]
 \center
 \caption {Relative/demonstrative pronouns in Old High German \pgcitep{braune2018}{339}}
  \begin{tabular}{cccc}
  \toprule
            & \ac{n}.\ac{sg}  & \ac{m}.\ac{sg}      & \ac{f}.\ac{sg}    \\
        \cmidrule{2-4}
  \ac{nom}  & d-aȥ            & d-ër                & d-iu               \\
  \ac{acc}  & d-aȥ            & d-ën                & d-ea/d-ia         \\
  \ac{dat}  & d-ëmu/d-ëmo     & d-ëmu/d-ëmo         & d-ëru/d-ëro       \\
  \bottomrule
            & \ac{n}.\ac{pl}  & \ac{m}.\ac{pl}      &  \ac{f}.\ac{pl}  \\
        \cmidrule{2-4}
  \ac{nom}  & d-iu            &  d-ē/d-ea/d-ia/d-ie & d-eo/-io         \\
  \ac{acc}  & d-iu            &  d-ē/d-ea/d-ia/d-ie & d-eo/-io         \\
  \ac{dat}  & d-ēm/d-ēn       &  d-ēm/d-ēn          & d-ēm/d-ēn        \\
    \bottomrule
  \end{tabular}
  \label{tbl:rel-dem-ohg}
\end{table}


The suffixes that combine with the \tit{d} in demonstrative and relative pronouns also appear on adjectives. This is illustrated in Table \ref{tbl:adj-ohg}.

\begin{table}[htbp]
 \center
 \caption {Adjectives on \tit{-a-/-ō-} in Old High German \pgcitealt{braune2018}{300}}
  \begin{tabular}{cccc}
  \toprule
            & \ac{n}.\ac{sg}    & \ac{m}.\ac{sg}      & \ac{f}.\ac{sg}    \\
    \cmidrule{2-4}
  \ac{nom}  & jung, jung-aȥ     & jung, jung-ēr       & jung, jung-iu     \\
  \ac{acc}  & jung, jung-aȥ     & jung-an             & jung-a            \\
  \ac{dat}  & jung-emu/jung-emo & jung-emu/jung-emo   & jung-eru/jung-ero \\
  \bottomrule
            & \ac{n}.\ac{pl}    & \ac{m}.\ac{pl}      &  \ac{f}.\ac{pl}   \\
      \cmidrule{2-4}
  \ac{nom}  & jung-iu           &  jung-e             & jung-o            \\
  \ac{acc}  & jung-iu           &  jung-e             & jung-o            \\
  \ac{dat}  & jung-ēm/jung-ēn   &  jung-ēm/jung-ēn    & jung-ēm/jung-ēn   \\
    \bottomrule
  \end{tabular}
  \label{tbl:adj-ohg}
\end{table}

I conclude from this that the suffix expresses features that are specific to being nominal, like number, gender and case. Not part of the suffix are features that are specific to being a demonstrative or relative pronoun, like anaphoricity and definiteness. I assume that these are expressed by the morpheme \tit{d}.

split the suffix up in two morphemes


In this section, I only discuss two forms: the nominative and accusative masculine singular relative and demonstrative pronoun. The nominative is \tit{dër} and the accusative is \tit{dën}. In what follows, I discuss the feature content of the morphemes \tit{d}, \tit{ër} and \tit{ën}. I start with the features that are expressed by the suffixes \tit{ër} and \tit{ën}.

This allows me to propose the following lexical entries for the two suffixes.


The \tit{d} morpheme corresponds to definiteness and anaphoricity. Anaphoricity establishes a relation with another element in the (linguistic) discourse. Definiteness encodes that the referent is specific.

\ex.
\begin{forest} boom
 [\tsc{d}P
     [\tsc{d}]
     [\tsc{ana}]
 ]
 {\draw (.east) node[right]{⇔ \tit{d}}; }
\end{forest}
\label{ex:ohg-d-lexicon}

So, the two relative pronouns look like this.\footnote{A question that arises here is how the case features can form a constituent to the exclusion of definiteness and anaphoricity. I come back to this issue in Chapter \ref{ch:discussion}.}


Headless relatives in which the relative pronoun starts with a \tit{d}, such as in Old High German, seem to be linked to individuating or definite readings and not to generalizing or indefinite readings \citep[cf.][]{fuss2017}. I illustrate this with the two examples I repeat from Chapter  \ref{ch:typology}.

Consider the example in \ref{ex:ohg-nom-acc-interpretation}, repeated from Chapter \ref{ch:typology}.
In this example, the author refers to the specific person which was talked about, and not to any or every person that was talked about.

\exg. Thíz ist \tbf{then} \tbf{sie} \tbf{zéllent}\\
\ac{dem}.\ac{sg}.\ac{n}.\ac{nom} be.\ac{pres}.3\ac{sg}\scsub{[nom]} \ac{rel}.\ac{sg}.\ac{m}.\ac{acc}
3\ac{pl}.\ac{m}.\ac{nom} tell.\ac{pres}.3\ac{pl}\scsub{[acc]}\\
`this is the one whom they talk about'\\
not: `this is whoever they talk about' \flushfill{Old High German, \ac{otfrid} III 16:50}\label{ex:ohg-nom-acc-interpretation}

Consider also the example in \ref{ex:ohg-nom-acc-interpretation}, repeated from Chapter \ref{ch:typology}.
In this example, the author refers to the specific person who spoke to someone, and not to any or every person who spoke to someone.

\exg. enti aer {ant uurta} demo \tbf{zaimo} \tbf{sprah}\\
and 3\ac{sg}.\ac{m}.\ac{nom} reply.\ac{pst}.3\ac{sg}\scsub{[dat]} \ac{rel}.\ac{sg}.\ac{m}.\ac{dat} {to 3\ac{sg}.\ac{m}.\ac{dat}} speak.\ac{pst}.3\ac{sg}\scsub{[nom]}\\
`and he replied to the one who spoke to him'\\
not: `and he replied to whoever spoke to him'
 \flushfill{Old High German, \ac{mons} 7:24, adapted from \pgcitealt{pittner1995}{199}}\label{ex:ohg-dat-nom-rep}

 Consider the light-headed relative in \ref{ex:ohg-double}. \tit{Thér} `\tsc{dem}.\tsc{sg}.\tsc{m}.\tsc{nom}' is the head of the relative clause, which is the external element. \tit{Then} `\tsc{rp}.\tsc{sg}.\tsc{m}.\tsc{acc}' is the relative pronoun in the relative clause, which is the internal element.

 \exg. eno nist thiz thér then ir suochet zi arslahanne?\\
  now {not be.3\ac{sg}} \tsc{dem}.\tsc{sg}.\tsc{n}.\tsc{nom} \tsc{dem}.\tsc{sg}.\tsc{m}.\tsc{nom}
  \tsc{rp}.\tsc{sg}.\tsc{m}.\tsc{acc} 2\ac{pl}.\tsc{nom} seek.2\tsc{pl} to kill.\tsc{inf}.\ac{sg}.\tsc{dat}\\
  `Isn't this now the one, who you seek to kill?'\label{ex:ohg-double}

 The difference between a light-headed relative and a headless relative is that in headless relatives, either the internal or the external is absent. The absent element is the one that has the least complex case. This shows the presence of two elements in Old High German is optional.\footnote{
 This sharply contrasts with headless relatives in Modern German, which are always ungrammatical when both the internal and external elements surface. I come back to this in Chapter \ref{ch:deriving-onlyinternal}.
 }
 In Old High German, there are three possible constructions: the internal and external element can both surface, only the internal element can surface and only the external element can surface. If only one of the two elements surfaces, this is the element that bears the most complex case, which is either the internal or the external one, as I have shown in Chapter \ref{ch:typology}. I assume that whether both or only one of the elements surfaces is determined by information structure. In \ref{ex:ohg-double}, the external element \tit{thér} `\tsc{dem}.\tsc{sg}.\tsc{m}.\tsc{nom}' is the candidate to be absent. However, it seems plausible that this is emphasized in this sentence and that it, therefore, cannot be absent.

 The light head in a light-headed relative is a demonstrative pronoun.











\section{The Old High German light head}

\subsection{The extra light head}

\subsection{The light head}

\subsection{Possible predictions}

  - possible prediction: ext>int = def, int>ext = wh, not what we see, show 4 examples






\section{Comparing constituents}\label{sec:comparing-ohg}

In this section, I compare the constituents of extra light heads to those of relative pronouns in Modern German. This is the worked out version of the comparisons in Section \ref{sec:basic-internal}. What is different here is that I show the comparison for Modern German specifically, and that I motivated the content of the constituents that are being compared.

I give three examples, in which the internal and external case vary.
I start with an example with matching cases, in which the internal and the external case are both accusative.
Then I give an example in which the internal dative case is more complex than the external accusative case.
I end with an example in which the external dative case is more complex than the internal accusative case.
I show that the first two examples are grammatical and the last one is not. I derive this by showing that only in the first two situations the light head forms a constituent within the relative pronoun in these cases, and that it can therefore then be deleted.





\subsection{With the extra light head}

I start with the situation in which the cases match.
Consider the example in \ref{ex:ohg-nom-nom-rep}, in which the internal nominative case competes against the external nominative case. The relative clause is marked in bold.
The internal case is nominative, as the predicate \tit{senten} `to send' takes nominative subjects. The relative pronoun \tit{dher} `\ac{rel}.\ac{sg}.\ac{m}.\ac{nom}' appears in the nominative case. This is the element that surfaces.
The external case is nominative as well, as the predicate \tit{queman} `to come' also takes nominative subjects. The light head \tit{r} `\tsc{elh}.\ac{sg}.\ac{m}.\ac{nom}' appears in the nominative case. It is placed between square brackets because it does not surface.

\exg. quham [r] \tbf{dher} \tbf{chisendit} \tbf{scolda} \tbf{uuerdhan}\\
 come.\ac{pst}.3\ac{sg}\scsub{[nom]} \ac{dem}.\ac{sg}.\ac{m}.\ac{nom} \ac{rel}.\ac{sg}.\ac{m}.\ac{nom} send.\ac{pst}.\ac{ptcp}\scsub{[nom]} should.\ac{pst}.3\ac{sg} become.\ac{inf}\\
 `the one, who should have been sent, came' \flushfill{Old High German, \ac{isid} 35:5}\label{ex:ohg-nom-nom-rep}

In Figure \ref{fig:ohg-int=ext}, I give the syntactic structure of the extra light head at the top and the syntactic structure of the relative pronoun at the bottom.

\begin{figure}[htbp]
  \center
  \begin{adjustbox}{max height=0.9\textheight}
  \begin{tabular}[b]{c}
        \toprule
        \tsc{nom} extra light head \tit{r}\\
        \cmidrule{1-1}
      \begin{forest} boom
        [\tsc{nom}P,
        tikz={
        \node[label=below:{\tit{r}},
        draw,circle,
        scale=0.8,
        fit to=tree]{};
        \node[draw,circle,
        dashed,
        scale=0.85,
        fill=DG,fill opacity=0.2,
        fit to=tree]{};
        }
            [\ac{f}1]
            [\tsc{ind}P
                [\phantom{xxx}, roof]
            ]
        ]
      \end{forest}
      \\
      \toprule
      \tsc{nom} relative pronoun \tit{dhe-r}
      \\
      \cmidrule{1-1}
          \begin{forest} boom
          [\tsc{rel}P
              [\tsc{rel}P
                  [\phantom{x}\tit{dhe}\phantom{x}, roof]
              ]
              [\tsc{nom}P,
              tikz={
              \node[label=below:{\tit{r}},
              draw,circle,
              scale=0.8,
              fit to=tree]{};
              \node[draw,circle,
              dashed,
              scale=0.85,
              fit to=tree]{};
              }
                  [\ac{f}1]
                  [\tsc{ind}P
                      [\phantom{xxx}, roof]
                  ]
              ]
          ]
        \end{forest}
        \\
      \bottomrule
  \end{tabular}
  \end{adjustbox}
  \caption {Old High German \tsc{ext}\scsub{nom} vs. \tsc{int}\scsub{nom} → \tit{dher}}
  \label{fig:ohg-int=ext}
\end{figure}

The relative pronoun consists of two morphemes: \tit{dhe} and \tit{r}.
The extra light head consists of a single morpheme: \tit{r}.
As usual, I circle the part of the structure that corresponds to a particular lexical entry, and I place the corresponding phonology under it, or I reduce the structure to a triangle, and I place the corresponding phonology under it.
I draw a dashed circle around each constituent that is a constituent in both the extra light head and the relative pronoun.

The extra light head consists of a single constituent: the \tsc{nom}P.
This \tsc{nom}P is also a constituent within the relative pronoun. Therefore, the extra light head can be deleted. I signal the deletion of the extra light head by marking the content of its circle gray.

I continue with the situation in which the internal case is the more complex one.
Consider the example in \ref{ex:ohg-nom-acc-rep}, in which the internal accusative case competes against the external nominative case. The relative clause is marked in bold.
The internal case is accusative, as the predicate \tit{zellen} `to tell' takes accusative objects. The relative pronoun \tit{then} `\ac{rel}.\ac{sg}.\ac{m}.\ac{acc}' appears in the accusative case. This is the element that surfaces.
The external case is nominative, as the predicate \tit{sin} `to be' takes nominative objects. The light head \tit{r} `\tsc{elh}.\ac{sg}.\ac{m}.\ac{nom}' appears in the nominative case. It is placed between square brackets because it does not surface.

\exg. Thíz ist [r] \tbf{then} \tbf{sie} \tbf{zéllent}\\
\ac{dem}.\ac{sg}.\ac{n}.\ac{nom} be.\ac{pres}.3\ac{sg}\scsub{[nom]} \ac{dem}.\ac{sg}.\ac{m}.\ac{nom} \ac{rel}.\ac{sg}.\ac{m}.\ac{acc} 3\ac{pl}.\ac{m}.\ac{nom} tell.\ac{pres}.3\ac{pl}\scsub{[acc]}\\
`this is the one whom they talk about' \flushfill{Old High German, \ac{otfrid} III 16:50}\label{ex:ohg-nom-acc-rep}

In Figure \ref{fig:ohg-int-wins}, I give the syntactic structure of the extra light head at the top and the syntactic structure of the relative pronoun at the bottom.

\begin{figure}[htbp]
  \center
  \begin{adjustbox}{max height=0.9\textheight}
  \begin{tabular}[b]{c}
      \toprule
      \tsc{nom} extra light head \tit{r}
      \\
      \cmidrule{1-1}
      \begin{forest} boom
        [\tsc{nom}P,
        tikz={
        \node[label=below:{\tit{r}},
        draw,circle,
        scale=0.8,
        fit to=tree]{};
        \node[draw,circle,
        dashed,
        scale=0.85,
        fill=DG,fill opacity=0.2,
        fit to=tree]{};
        }
            [\tsc{f}1]
            [\tsc{ind}P
                [\phantom{xxx}, roof]
            ]
        ]
      \end{forest}
      \\
      \toprule
      \tsc{acc} relative pronoun \tit{the-n}
      \\
      \cmidrule{1-1}
          \begin{forest} boom
            [\tsc{rel}P
                [\tsc{rel}P
                    [\phantom{x}\tit{the}\phantom{x}, roof]
                ]
                [\tsc{acc}P,
                tikz={
                \node[label=below:{\tit{n}},
                draw,circle,
                scale=0.85,
                fit to=tree]{};
                }
                    [\tsc{f}3]
                    [\tsc{acc}P,
                    tikz={
                    \node[draw,circle,
                    dashed,
                    scale=0.8,
                    fit to=tree]{};
                    }
                        [\ac{f}1]
                        [\tsc{ind}P
                            [\phantom{xxx}, roof]
                        ]
                    ]
                ]
            ]
        \end{forest}
        \\
      \bottomrule
  \end{tabular}
  \end{adjustbox}
   \caption {Old High German \tsc{ext}\scsub{nom} vs. \tsc{int}\scsub{acc} → \tit{then}}
  \label{fig:ohg-int-wins}
\end{figure}

The relative pronoun consists of two morphemes: \tit{the} and \tit{n}.
The extra light head consists of a single morpheme: \tit{r}.
Again, I circle the part of the structure that corresponds to a particular lexical entry, and I place the corresponding phonology under it, or I reduce the structure to a triangle, and I place the corresponding phonology under it.
I draw a dashed circle around each constituent that is a constituent in both the extra light head and the relative pronoun.

The extra light head consists of a single constituent: the \tsc{nom}P.
This \tsc{nom}P is also a constituent within the relative pronoun. Therefore, the extra light can be deleted. I signal the deletion of the extra light head by marking the content of its circle gray.

\subsection{With the light head}

I start with the situation in which the cases match.
Consider the example in \ref{ex:ohg-nom-nom-rep-1}, in which the internal nominative case competes against the external nominative case. The relative clause is marked in bold.
The internal case is nominative, as the predicate \tit{senten} `to send' takes nominative subjects. The relative pronoun \tit{dher} `\ac{rel}.\ac{sg}.\ac{m}.\ac{nom}' appears in the nominative case. This is the element that surfaces.
The external case is nominative as well, as the predicate \tit{queman} `to come' also takes nominative subjects. The light head \tit{r} `\tsc{elh}.\ac{sg}.\ac{m}.\ac{nom}' appears in the nominative case. It is placed between square brackets because it does not surface.

\exg. quham [r] \tbf{dher} \tbf{chisendit} \tbf{scolda} \tbf{uuerdhan}\\
 come.\ac{pst}.3\ac{sg}\scsub{[nom]} \ac{dem}.\ac{sg}.\ac{m}.\ac{nom} \ac{rel}.\ac{sg}.\ac{m}.\ac{nom} send.\ac{pst}.\ac{ptcp}\scsub{[nom]} should.\ac{pst}.3\ac{sg} become.\ac{inf}\\
 `the one, who should have been sent, came' \flushfill{Old High German, \ac{isid} 35:5}\label{ex:ohg-nom-nom-rep-1}

In Figure \ref{fig:ohg-int=ext-1}, I give the syntactic structure of the extra light head at the top and the syntactic structure of the relative pronoun at the bottom.

\begin{figure}[htbp]
  \center
  \begin{adjustbox}{max height=0.9\textheight}
  \begin{tabular}[b]{c}
        \toprule
        \tsc{nom} light head \tit{dhe-r}\\
        \cmidrule{1-1}
        \begin{forest} boom
          [\tsc{d}P, s sep=20mm,
          tikz={
          \node[draw,circle,
          dotted,
          scale=1,
          fit to=tree]{};
          }
              [\tsc{d}P,
              tikz={
              \node[label=below:\tit{dhe},
              draw,circle,
              scale=0.85,
              fit to=tree]{};
              }
                  [\phantom{xxx}, roof, baseline]
              ]
              [\tsc{nom}P,
              tikz={
              \node[label=below:\tit{r},
              draw,circle,
              scale=0.85,
              fit to=tree]{};
              }
                  [\tsc{f}1]
                  [\tsc{ind}P
                      [\phantom{xxx}, roof, baseline]
                  ]
              ]
          ]
        \end{forest}
      \\
      \toprule
      \tsc{nom} relative pronoun \tit{dhe-r}
      \\
      \cmidrule{1-1}
      \begin{forest} boom
        [\tsc{rel}P, s sep=20mm,
        tikz={
        \node[draw,circle,
        dotted,
        fill=DG,fill opacity=0.2,
        scale=1.05,
        fit to=tree]{};
        }
            [\tsc{rel}P,
            tikz={
            \node[label=below:\tit{dhe},
            draw,circle,
            scale=0.85,
            fit to=tree]{};
            }
                [\tsc{rel}]
                [\tsc{d}P
                    [\phantom{xxx}, roof, baseline]
                ]
            ]
            [\tsc{nom}P,
            tikz={
            \node[label=below:\tit{r},
            draw,circle,
            scale=0.85,
            fit to=tree]{};
            }
                [\tsc{f}1]
                [\tsc{ind}P
                    [\phantom{xxx}, roof, baseline]
                ]
            ]
        ]
      \end{forest}
        \\
      \bottomrule
  \end{tabular}
  \end{adjustbox}
  \caption {Old High German \tsc{ext}\scsub{nom} vs. \tsc{int}\scsub{nom} → \tit{dher}}
  \label{fig:ohg-int=ext-1}
\end{figure}

The relative pronoun consists of two morphemes: \tit{dhe} and \tit{r}.
The light head also consists of two morphemes: \tit{dhe} and \tit{r}.
As usual, I circle the part of the structure that corresponds to a particular lexical entry, and I place the corresponding phonology under it, or I reduce the structure to a triangle, and I place the corresponding phonology under it.
I draw a dotted circle around each constituent that is a constituent in both the light head and the relative pronoun.

The light head (the \tsc{d}P realized by \tit{dher}) is syncretic with the relative pronoun (the \tsc{rel}P realized by \tit{dher}).
As the two forms are entirely syncretic, either the light head or the relative pronoun can be deleted. I delete the relative pronoun here, as I discuss the situation in which the relative pronoun is deleted.
I illustrate this by marking the content of the dotted circle for the relative pronoun gray.









- ext wins doesn't work, so first larger syntactic structure: head needs to be up there, for instance cinque (he also says we need to have it up there)
then: yes, at some point in the derivation, and then merge the k2




I end with the situation in which the external case is the more complex one.
Consider the examples in \ref{ex:ohg-acc-nom-rep}, in which the internal nominative case competes against the external accusative case. The relative clause is marked in bold.
The internal case is nominative, as the predicate \tit{gisizzen} `to possess' takes nominative subjects. The relative pronoun \tit{dher} `\ac{rel}.\ac{sg}.\ac{m}.\ac{nom}' appears in the nominative case.
The external case is accusative, as the predicate \tit{bibringan} `to create' takes accusative objects. The light head \tit{dhen} `\ac{dem}.\ac{sg}.\ac{m}.\ac{acc}' appears in the accusative case. This is the element that surfaces.

\exg. ih bibringu fona iacobes samin endi fona iuda dhen [\tbf{dher}] \tbf{mina} \tbf{berga} \tbf{chisitzit}\\
1\ac{sg}.\ac{nom} {create}.\ac{pres}.1\ac{sg}\scsub{[acc]} of Jakob.\ac{gen} seed.\ac{sg}.\ac{dat} and of Judah.\ac{dat} \ac{rel}.\ac{sg}.\ac{m}.\ac{acc} my.\ac{acc}.\ac{m}.\ac{pl} mountain.\ac{acc}.\ac{pl} possess.\ac{pres}.3\ac{sg}\scsub{[nom]}\\
`I create of the seed of Jacob and of Judah the one, who possess my mountains' \flushfill{Old High German, \ac{isid} 34:3}\label{ex:ohg-acc-nom-rep}

In Figure \ref{fig:ohg-ext-wins}, I give the syntactic structure of the light head at the top and the syntactic structure of the relative pronoun at the bottom.

\begin{figure}[htbp]
  \center
  \begin{adjustbox}{max height=0.9\textheight}
  \begin{tabular}[b]{c}
        \toprule
        \tsc{nom} light head \tit{dhe-n}\\
        \cmidrule{1-1}
        \begin{forest} boom
          [\tsc{d}P, s sep=20mm,
          tikz={
          \node[draw,circle,
          dotted,
          scale=1,
          fit to=tree]{};
          }
              [\tsc{d}P,
              tikz={
              \node[label=below:\tit{dhe},
              draw,circle,
              scale=0.85,
              fit to=tree]{};
              }
                  [\phantom{xxx}, roof, baseline]
              ]
              [\tsc{acc}P,
              tikz={
              \node[label=below:\tit{n},
              draw,circle,
              scale=0.85,
              fit to=tree]{};
              }
                  [\tsc{f}2]
                  [\tsc{nom}P
                      [\tsc{f}1]
                      [\tsc{ind}P
                          [\phantom{xxx}, roof, baseline]
                      ]
                  ]
              ]
          ]
        \end{forest}
      \\
      \toprule
      \tsc{nom} relative pronoun \tit{dhe-r}
      \\
      \cmidrule{1-1}
      \begin{forest} boom
        [\tsc{rel}P, s sep=20mm,
        tikz={
        \node[draw,circle,
        dotted,
        fill=DG,fill opacity=0.2,
        scale=1.05,
        fit to=tree]{};
        }
            [\tsc{rel}P,
            tikz={
            \node[label=below:\tit{dhe},
            draw,circle,
            scale=0.85,
            fit to=tree]{};
            }
                [\tsc{rel}]
                [\tsc{d}P
                    [\phantom{xxx}, roof, baseline]
                ]
            ]
            [\tsc{acc}P,
            tikz={
            \node[label=below:\tit{r},
            draw,circle,
            scale=0.85,
            fit to=tree]{};
            }
                [\tsc{f}1]
                [\tsc{ind}P
                    [\phantom{xxx}, roof, baseline]
                ]
            ]
        ]
      \end{forest}
        \\
      \bottomrule
  \end{tabular}
  \end{adjustbox}
  \caption {Old High German \tsc{ext}\scsub{nom} vs. \tsc{int}\scsub{nom} → \tit{dher}}
  \label{fig:ohg-int=ext-1}
\end{figure}



\begin{figure}[htbp]
  \center
  \begin{adjustbox}{max height=0.9\textheight}
  \begin{adjustbox}{max width=\textwidth}
  \begin{tabular}[b]{c}
        \toprule
        \tsc{acc} extra light head \tit{dh-e-n} \\
        \cmidrule{1-1}
        \begin{forest} boompje
          [\tsc{d}P, s sep=20mm
              [\tsc{d}P,
              tikz={
              \node[label=below:\tit{dh},
              draw,circle,
              scale=0.8,
              fit to=tree]{};
              \node[draw,circle,
              dashed,
              scale=0.9,
              fit to=tree]{};
              }
                  [\tsc{d}, roof]
              ]
              [\tsc{acc}P, s sep=25mm
                  [\tsc{med}P,
                  tikz={
                  \node[label=below:\tit{e},
                  draw,circle,
                  scale=0.85,
                  fit to=tree]{};
                  \node[draw,circle,
                  dashed,
                  scale=0.9,
                  fit to=tree]{};
                  }
                      [\tsc{dx}\scsub{2}]
                      [\tsc{prox}P
                          [\tsc{dx}\scsub{1}]
                          [\tsc{ref}]
                      ]
                  ]
                  [\tsc{acc}P,
                  tikz={
                  \node[label=below:\tit{n},
                  draw,circle,
                  scale=0.95,
                  fit to=tree]{};
                  }
                      [\tsc{f}2]
                      [\tsc{nom}P,
                      tikz={
                      \node[draw,circle,
                      dashed,
                      scale=0.9,
                      fit to=tree]{};
                      }
                          [\tsc{f}1]
                          [\tsc{ind}P
                              [\tsc{ind}]
                              [\tsc{an}P
                                  [\tsc{an}]
                                  [\tsc{cl}P
                                      [\tsc{cl}]
                                  ]
                              ]
                          ]
                      ]
                  ]
              ]
          ]
        \end{forest}
        \\
        \toprule
        \tsc{nom} relative pronoun \tit{dh-e-r}
        \\
        \cmidrule{1-1}
            \begin{forest} boompje
              [\tsc{rp}P, s sep=15mm
                  [\tsc{rp}P,
                  tikz={
                  \node[label=below:\tit{dh},
                  draw,circle,
                  scale=0.95,
                  fill=DG,fill opacity=0.1,
                  fit to=tree]{};
                  }
                      [\tsc{rp}]
                      [\tsc{d}P,
                      tikz={
                      \node[draw,circle,
                      dashed,
                      fill=DG,fill opacity=0.2,
                      scale=0.8,
                      fit to=tree]{};
                      }
                          [\tsc{d}, roof]
                      ]
                  ]
                  [\tsc{nom}P, s sep=25mm
                      [\tsc{med}P,
                      tikz={
                      \node[label=below:\tit{e},
                      draw,circle,
                      scale=0.85,
                      fit to=tree]{};
                      \node[draw,circle,
                      dashed,
                      scale=0.9,
                      fill=DG,fill opacity=0.2,
                      fit to=tree]{};
                      }
                          [\tsc{dx}\scsub{2}]
                          [\tsc{prox}P
                              [\tsc{dx}\scsub{1}]
                              [\tsc{ref}]
                          ]
                      ]
                      [\tsc{nom}P,
                      tikz={
                      \node[label=below:\tit{r},
                      draw,circle,
                      scale=0.95,
                      fit to=tree]{};
                      \node[draw,circle,
                      dashed,
                      fill=DG,fill opacity=0.2,
                      scale=1,
                      fit to=tree]{};
                      }
                          [\tsc{f}1]
                          [\tsc{ind}P
                              [\tsc{ind}]
                              [\tsc{an}P
                                  [\tsc{an}]
                                  [\tsc{cl}P
                                      [\tsc{cl}]
                                  ]
                              ]
                          ]
                      ]
                  ]
              ]
          \end{forest}
          \\
      \bottomrule
  \end{tabular}
\end{adjustbox}
\end{adjustbox}
   \caption {Old High German \tsc{ext}\scsub{acc} vs. \tsc{int}\scsub{nom} → \tit{dhen}}
  \label{fig:ohg-ext-wins}
\end{figure}

The relative pronoun consists of three morphemes: \tit{dh}, \tit{e} and \tit{r}.
The light head consists of three morphemes: \tit{dh}, \tit{e} and \tit{n}.
Again, I circle the part of the structure that corresponds to a particular lexical entry, and I place the corresponding phonology under it.
I draw a dashed circle around each constituent that is a constituent in both the light head and the relative pronoun.
As each constituent of the light head is also a constituent within the relative pronoun or is syncretic with one, the relative pronoun can be absent. I illustrate this by marking the content of the dashed circles for the relative pronoun gray.

I explain this constituent by constituent.
I start with the right-most constituent of the relative pronoun head that spells out as \tit{r} (\tsc{nom}P). This constituent is also a constituent in the light head, contained in \tsc{acc}P.
I continue with the middle constituent of the relative pronoun that spells out as \tit{e} (\tsc{med}P). This constituent is also a constituent in the light head.
I end with the left-most constituent of the relative pronoun that spells out as \tit{d} {\tsc{rp}P}. This consituent is not contained in the light head, but it is syncretic with it. The \tsc{d}P is also spelled out as \tit{d}.
All three constituent of the light head are also a constituent within the relative pronoun or are syncretic with them, and the relative pronoun can be absent.





Gothic seems to be a variant of Old High German, in which there is also no single constituent containment. This time, the relative pronoun is not deleted by syncretism. Gothic has a separate suffix that spells out the feature \tsc{rel}. The light head deletes the relative pronoun, except for the suffix that spells out \tsc{rel}. The light head and the relative pronoun phonologically merge together, and the surface pronoun appears in the external case.
