% !TEX root = thesis.tex

\chapter{Deriving the unrestricted type}\label{ch:deriving-unrestricted}

In Chapter \ref{ch:the-basic-idea}, I suggested that languages of the unrestricted type have two possible light heads, which are part of the derivation under different circumstances. The first possible light head can part of the derivation used when the internal and external case match, and it appears when the internal case is more complex than the external one. The second possible light head can be part of the derivation when the internal and the external case too, and it appears when the external case is more complex than the internal one.

In the first possible light head, the light head corresponds to the phi- and case-feature part of the relative pronoun. The phi- and case-features are spelled out by a portmanteau morpheme, just as they are in the internal-only type of language. This means that the features of the relative pronoun and the light head are spelled out in such a way that they form the constituents shown in Figure \ref{fig:rel-lh-unres-simple}.

\begin{figure}[htbp]
  \center
  \begin{tabular}[b]{ccc}
      \toprule
      light head 1 & & relative pronoun \\
      \cmidrule(lr){1-1} \cmidrule(lr){3-3}
      \begin{forest} boom
      [\tsc{k}P,
      tikz={
      \node[draw,circle,
      scale=0.85,
      fit to=tree]{};
      }
          [\tsc{k}]
          [ϕP
              [\phantom{xxx}, roof, baseline]
          ]
      ]
      \end{forest}
      & \phantom{x} &
    \begin{forest} boom
      [\tsc{rel}P, s sep = 20 mm
          [\tsc{rel}P,
          tikz={
          \node[draw,circle,
          scale=0.85,
          fit to=tree]{};
          }
              [\phantom{xxx}, roof, baseline]
          ]
          [\tsc{k}P,
          tikz={
          \node[draw,circle,
          scale=0.85,
          fit to=tree]{};
          }
              [\tsc{k}]
              [ϕP
                  [\phantom{xxx}, roof, baseline]
              ]
          ]
      ]
    \end{forest}\\
      \bottomrule
  \end{tabular}
   \caption {\tsc{lh}-1 and \tsc{rp} in the unrestricted type}
  \label{fig:rel-lh-unres-simple-1}
\end{figure}

These lexical entries lead to a grammaticality pattern as shown in Table \ref{tbl:overview-unres-1}.

\begin{table}[htbp]
  \center
  \caption{Grammaticality in the unrestricted type (part 1)}
  \begin{adjustbox}{max width=\textwidth}
  \begin{tabular}{cccccc}
    \toprule
    situation           & \multicolumn{2}{c}{lexical entries}       & containment         & deleted             & surfacing           \\
    \cmidrule(lr){1-1}    \cmidrule(lr){2-3}                          \cmidrule(lr){4-4}    \cmidrule(lr){5-5}    \cmidrule(lr){6-6}
                        & \tsc{lh}            & \tsc{rp}            &                     &                     &                     \\
                          \cmidrule(lr){2-2}    \cmidrule(lr){3-3}
  \tsc{k}\scsub{int} = \tsc{k}\scsub{ext}               &
  [\tsc{k}\scsub{1}[ϕ]]                                 &
  [\tsc{rel}], [\tsc{k}\scsub{1}[ϕ]]                    &
  structure & \tsc{lh} & \tsc{rp}\scsub{int}            \\
  \tsc{k}\scsub{int} > \tsc{k}\scsub{ext}               &
  [\tsc{k}\scsub{1}[ϕ]]                                 &
  [\tsc{rel}], [\tsc{k}\scsub{2}[\tsc{k}\scsub{1}[ϕ]]]  &
  structure & \tsc{lh} & \tsc{rp}\scsub{int}            \\
  \bottomrule
  \end{tabular}
  \end{adjustbox}
\label{tbl:overview-unres-1}
\end{table}

First consider the situation in which the internal and the external case match. The situation here is identical to the one in the internal-only type of language. The light head consists of a phi- and case-feature portmanteau. The relative pronoun consists of the same morpheme plus an additional morpheme that spells out the feature \tsc{rel}. These lexical entries create such syntactic structures that the light head structurally forms a constituent within the relative pronoun. Therefore, the light head can be deleted, and the relative pronoun that bears the internal case surfaces.

Consider now the situation in the internal case wins the case competition. Here situation is identical to the one in the internal-only type of language too. The light head consists of a phi- and case-feature portmanteau. The relative pronoun consists of a phi- and case-feature portmanteau that contains at least one more case feature than the light head (\tsc{k}\scsub{2} in Figure \ref{tbl:overview-unres-1}) plus an additional morpheme that spells out the feature \tsc{rel}. These lexical entries create such syntactic structures that the light head structurally forms a constituent within the relative pronoun. Therefore, the light head can be deleted, and the relative pronoun that bears the internal case surfaces.

In Chapter \ref{ch:typology}, I showed that Old High German is a language of the unrestricted type. In this chapter, I show that Old High German has light heads and relative pronouns of type of structure described in Figure \ref{fig:rel-lh-unres-simple-1}. I give a compact version of the structures in Figure \ref{fig:rel-lh-ohg-1}.

\begin{figure}[htbp]
  \center
  \begin{tabular}[b]{ccc}
      \toprule
      light head & & relative pronoun \\
      \cmidrule(lr){1-1} \cmidrule(lr){3-3}
      \begin{forest} boom
        [\tsc{k}P,
        tikz={
        \node[label=below:\tit{n/m},
        draw,circle,
        scale=0.75,
        fit to=tree]{};
        }
            [\tsc{k}]
            [ϕP
                [\phantom{xxx}, roof, baseline]
            ]
        ]
      \end{forest}
      & \phantom{x} &
      \begin{forest} boom
        [\tsc{rel}P, s sep=15mm
            [\tsc{rel}P,
            tikz={
            \node[label=below:\tit{de},
            draw,circle,
            scale=0.75,
            fit to=tree]{};
            }
                [\phantom{xxx}, roof]
            ]
            [\tsc{k}P,
            tikz={
            \node[label=below:\tit{n/m},
            draw,circle,
            scale=0.75,
            fit to=tree]{};
            }
                [\tsc{k}]
                [ϕP
                    [\phantom{xxx}, roof, baseline]
                ]
            ]
        ]
      \end{forest}\\
      \bottomrule
  \end{tabular}
   \caption {\tsc{lh}-1 and \tsc{rp} in Old High German}
  \label{fig:rel-lh-ohg-1}
\end{figure}

Consider the first possible light head in Figure \ref{fig:rel-lh-ohg-1}.
These light heads (i.e. phi- and case-features) in Old High German are spelled out by a single morpheme, indicated by the circle around the structure. They are spelled out as \tit{n} or \tit{m}, depending on which case they realize.
Consider the relative pronoun in Figure \ref{fig:rel-lh-ohg-1}.
Relative pronouns in Old High German consist of two morphemes: the constituent that forms the light head (i.e. phi- and case features) and the \tsc{rel}P, again indicated by the circles. The \tsc{rel}P is spelled out as \tit{de}.
Throughout this chapter, I discuss the exact feature content of relative pronouns and light heads, I give lexical entries for them, and I show how these lexical entries form the constituents shown in Figure \ref{fig:rel-lh-ohg-1}.

In the second possible light head, the light head corresponds to the phi- and case-feature part of the relative pronoun plus an additional feature X. This feature X is also present in the morpheme that spells out the feature \tsc{rel}. The phi- and case-features are still spelled out by a portmanteau morpheme. The feature X is spelled out by a separate morpheme, which is the same morpheme that spells out X plus the feature \tsc{rel}. This means that the features of the relative pronoun and the light head are spelled out in such a way that they form the constituents shown in Figure \ref{fig:rel-lh-unres-simple-2}.

\begin{figure}[htbp]
  \center
  \begin{adjustbox}{max width=\textwidth}
  \begin{tabular}[b]{ccc}
      \toprule
      light head 2 & & relative pronoun \\
      \cmidrule(lr){1-1} \cmidrule(lr){3-3}
      \begin{forest} boom
      [XP, s sep = 20 mm
          [XP,
          tikz={
          \node[label=below:\tit{X},
          draw,circle,
          scale=0.85,
          fit to=tree]{};
          }
              [\phantom{xxx}, roof, baseline]
          ]
          [\tsc{k}P,
          tikz={
          \node[draw,circle,
          scale=0.85,
          fit to=tree]{};
          }
              [\tsc{k}]
              [ϕP
                  [\phantom{xxx}, roof, baseline]
              ]
          ]
      ]
      \end{forest}
      & \phantom{x} &
    \begin{forest} boom
      [\tsc{rel}P, s sep = 20 mm
          [\tsc{rel}P,
          tikz={
          \node[label=below:\tit{X},
          draw,circle,
          scale=0.85,
          fit to=tree]{};
          }
              [\tsc{rel}]
              [XP
                  [\phantom{xxx}, roof, baseline]
              ]
          ]
          [\tsc{k}P,
          tikz={
          \node[draw,circle,
          scale=0.85,
          fit to=tree]{};
          }
              [\tsc{k}]
              [ϕP
                  [\phantom{xxx}, roof, baseline]
              ]
          ]
      ]
    \end{forest}\\
      \bottomrule
  \end{tabular}
  \end{adjustbox}
   \caption {\tsc{lh}-2 and \tsc{rp} in the unrestricted type}
  \label{fig:rel-lh-unres-simple-2}
\end{figure}

These lexical entries lead to a grammaticality pattern as shown in Table \ref{tbl:overview-unres-2}.

\begin{table}[htbp]
  \center
  \caption{Grammaticality in the unrestricted type (part 2)}
  \begin{adjustbox}{max width=\textwidth}
  \begin{tabular}{cccccc}
    \toprule
    situation           & \multicolumn{2}{c}{lexical entries}       & containment         & deleted             & surfacing           \\
    \cmidrule(lr){1-1}    \cmidrule(lr){2-3}                          \cmidrule(lr){4-4}    \cmidrule(lr){5-5}    \cmidrule(lr){6-6}
                        & \tsc{lh}            & \tsc{rp}            &                     &                     &                     \\
                          \cmidrule(lr){2-2}    \cmidrule(lr){3-3}
  \tsc{k}\scsub{int} = \tsc{k}\scsub{ext}               &
  \tit{/X/}, \tit{/Y/}                                  &
  \tit{/X/}, \tit{/Y/}                                  &
  form & \tsc{rp} & \tsc{lh}\scsub{ext}                 \\
  \tsc{k}\scsub{int} < \tsc{k}\scsub{ext}               &
  \tit{/X/}, \tit{/Y/}                                  &
  \tit{/X/}, \tit{/Y/}                                  &
  form & \tsc{rp} & \tsc{lh}\scsub{ext}                 \\
  \bottomrule
  \end{tabular}
  \end{adjustbox}
\label{tbl:overview-unres-2}
\end{table}

First consider the situation in which the internal and the external case match. The light head consists of a phi- and case-feature portmanteau and a morpheme that spells out the feature X, which corresponds to phonological form \tit{X}. The relative pronoun consists of the same phi- and case-feature morpheme and a morpheme that spells out the feature X and the feature \tsc{rel}, which corresponds to the phonological form \tit{X} too. When the internal and the external case match, the phonological form corresponding to the phi- and case features is identical between the light head and the relative pronoun too. These lexical entries create such syntactic structures that the light head and the relative pronoun are formally identical. Since there is formal containment, one of the elements can be deleted, and the the other one surfaces with its case.

Consider now the situation in the external case wins the case competition. The light head consists of a phi- and case-feature portmanteau and a morpheme that spells out the feature X, which corresponds to phonological form \tit{X}. The relative pronoun consists of the same phi- and case-feature morpheme and a morpheme that spells out the feature X and the feature \tsc{rel}, which corresponds to the phonological form \tit{X} too. When the external case is more complex than the internal case (i.e. when the two cases differ), the phonological forms corresponding to the phi- and case features of the light head and the relative pronoun differ. However, the derivation in which the external case is more complex than the internal one goes through a stage in which the internal and the external case match. Therefore, at that stage, these lexical entries form such syntactic structures that the light head and the relative pronoun are formally identical. Since there is formal containment, one of the elements can be deleted, and the the other one surfaces with its case. Then, the more complex case is merged to the remaining element.

In Chapter \ref{ch:typology}, I showed that Old High German is a language of the unrestricted type. In this chapter, I show that Old High German has light heads and relative pronouns of type of structure described in Figure \ref{fig:rel-lh-unres-simple-2}. I give a compact version of the structures in Figure \ref{fig:rel-lh-ohg-2}.

\begin{figure}[htbp]
  \center
  \begin{tabular}[b]{ccc}
      \toprule
      light head & & relative pronoun \\
      \cmidrule(lr){1-1} \cmidrule(lr){3-3}
      \begin{forest} boom
      [XP, s sep = 20 mm
          [XP,
          tikz={
          \node[label=below:\tit{de},
          draw,circle,
          scale=0.85,
          fit to=tree]{};
          }
              [\phantom{xxx}, roof, baseline]
          ]
          [\tsc{k}P,
          tikz={
          \node[label=below:\tit{n/m},
          draw,circle,
          scale=0.85,
          fit to=tree]{};
          }
              [\tsc{k}]
              [ϕP
                  [\phantom{xxx}, roof, baseline]
              ]
          ]
      ]
      \end{forest}
      & \phantom{x} &
      \begin{forest} boom
        [\tsc{rel}P, s sep = 20 mm
            [\tsc{rel}P,
            tikz={
            \node[label=below:\tit{de},
            draw,circle,
            scale=0.85,
            fit to=tree]{};
            }
                [\tsc{rel}]
                [XP
                    [\phantom{xxx}, roof, baseline]
                ]
            ]
            [\tsc{k}P,
            tikz={
            \node[label=below:\tit{n/m},
            draw,circle,
            scale=0.75,
            fit to=tree]{};
            }
                [\tsc{k}]
                [ϕP
                    [\phantom{xxx}, roof, baseline]
                ]
            ]
        ]
      \end{forest}\\
      \bottomrule
  \end{tabular}
   \caption {\tsc{lh}-2 and \tsc{rp} in Old High German}
  \label{fig:rel-lh-ohg-2}
\end{figure}

Consider the first possible light head in Figure \ref{fig:rel-lh-ohg-2}.
These light heads in Old High German are spelled out by two morphemes, which are both circled. The morpheme that realizes the feature X is spelled out as \tit{de} and the phi-and case-features are spelled out as \tit{n} or \tit{m}, depending on which case they realize.

Consider the relative pronoun in Figure \ref{fig:rel-lh-ohg-2}.
Relative pronouns in Old High German are spelled out by the same two morphemes as the second possible light head. There is the phi- and case-feature morpheme (spelling out as \tit{n} or \tit{m}) and the morpheme that spells out the feature X and in the relative pronoun also the feature \tsc{rel} (spelling out as \tit{de}).
Throughout this chapter, I discuss the exact feature content of relative pronouns and light heads, I give lexical entries for them, and I show how these lexical entries form the constituents shown in Figure \ref{fig:rel-lh-ohg-2}.

This chapter is structured as follows.
First, I discuss the relative pronoun. I decompose the relative pronouns into the two morphemes I showed in Figure \ref{fig:rel-lh-ohg-1} and \ref{fig:rel-lh-ohg-2}, and I show which features each of the morphemes corresponds to. I illustrate how different morphemes are combined into the given constituents.
Then I discuss the two possible light heads. I argue that headless relatives in Old High German can be derived from two different types of light-headed relative clauses. The first type actually surfaces in the language. The second one is the type of light-headed relative clause that does not surface in the language, just like those in Modern German and Polish.
Next, I compare the constituents of the two different light heads and the relative pronoun. I show that the first possible light head can be deleted when the internal case and external case match and when the internal case is more complex than the external case via structural containment. The second possible light head can be deleted when the internal case and external case match and when the internal case is more complex than the external case via formal containment. In order to illustrate how this works, I need to make a few assumptions about the larger syntactic structure of headless relative clauses explicit.
%Finally, I test a prediction that follows from assuming the two possible light heads and I show that this is not really borne out.

\section{The Old High German German relative pronoun}

- relative pronoun, show that it's a D

What is different here, is that the relative pronoun is a \tsc{d}-pronoun instead of a \tsc{wh}.

Relative and demonstrative pronouns are syncretic in Old High German \pgcitep{braune2018}{338}. Table \ref{tbl:rel-dem-ohg} gives an overview of the forms in singular and plural, neuter, masculine and feminine and nominative, accusative and dative. The pronouns consist of two morphemes: a \tit{d} and suffix that differs per number, gender and case.\footnote{
\tit{d} can also be written as \tit{dh} and \tit{th}, \tit{ë} and \tit{ē} can also be \tit{e} and \tit{é} \pgcitep{braune2018}{339}.
}\footnote{
The suffix could also be further divided into a vowel and a suffix. As this is not relevant for the discussion here, I refrain from doing that.
}

\begin{table}[htbp]
 \center
 \caption {Relative/demonstrative pronouns in Old High German \pgcitep{braune2018}{339}}
  \begin{tabular}{cccc}
  \toprule
            & \ac{n}.\ac{sg}  & \ac{m}.\ac{sg}      & \tsc{f}.\ac{sg}    \\
        \cmidrule{2-4}
  \ac{nom}  & d-aȥ            & d-ër                & d-iu               \\
  \ac{acc}  & d-aȥ            & d-ën                & d-ea/d-ia         \\
  \ac{dat}  & d-ëmu/d-ëmo     & d-ëmu/d-ëmo         & d-ëru/d-ëro       \\
  \bottomrule
            & \ac{n}.\ac{pl}  & \ac{m}.\ac{pl}      &  \tsc{f}.\ac{pl}  \\
        \cmidrule{2-4}
  \ac{nom}  & d-iu            &  d-ē/d-ea/d-ia/d-ie & d-eo/-io         \\
  \ac{acc}  & d-iu            &  d-ē/d-ea/d-ia/d-ie & d-eo/-io         \\
  \ac{dat}  & d-ēm/d-ēn       &  d-ēm/d-ēn          & d-ēm/d-ēn        \\
    \bottomrule
  \end{tabular}
  \label{tbl:rel-dem-ohg}
\end{table}


The suffixes that combine with the \tit{d} in demonstrative and relative pronouns also appear on adjectives. This is illustrated in Table \ref{tbl:adj-ohg}.

\begin{table}[htbp]
 \center
 \caption {Adjectives on \tit{-a-/-ō-} in Old High German \pgcitealt{braune2018}{300}}
  \begin{tabular}{cccc}
  \toprule
            & \ac{n}.\ac{sg}    & \ac{m}.\ac{sg}      & \tsc{f}.\ac{sg}    \\
    \cmidrule{2-4}
  \ac{nom}  & jung, jung-aȥ     & jung, jung-ēr       & jung, jung-iu     \\
  \ac{acc}  & jung, jung-aȥ     & jung-an             & jung-a            \\
  \ac{dat}  & jung-emu/jung-emo & jung-emu/jung-emo   & jung-eru/jung-ero \\
  \bottomrule
            & \ac{n}.\ac{pl}    & \ac{m}.\ac{pl}      &  \tsc{f}.\ac{pl}   \\
      \cmidrule{2-4}
  \ac{nom}  & jung-iu           &  jung-e             & jung-o            \\
  \ac{acc}  & jung-iu           &  jung-e             & jung-o            \\
  \ac{dat}  & jung-ēm/jung-ēn   &  jung-ēm/jung-ēn    & jung-ēm/jung-ēn   \\
    \bottomrule
  \end{tabular}
  \label{tbl:adj-ohg}
\end{table}

I conclude from this that the suffix expresses features that are specific to being nominal, like number, gender and case. Not part of the suffix are features that are specific to being a demonstrative or relative pronoun, like anaphoricity and definiteness. I assume that these are expressed by the morpheme \tit{d}.

split the suffix up in two morphemes


In this section, I only discuss two forms: the nominative and accusative masculine singular relative and demonstrative pronoun. The nominative is \tit{dër} and the accusative is \tit{dën}. In what follows, I discuss the feature content of the morphemes \tit{d}, \tit{ër} and \tit{ën}. I start with the features that are expressed by the suffixes \tit{ër} and \tit{ën}.

This allows me to propose the following lexical entries for the two suffixes.


The \tit{d} morpheme corresponds to definiteness and anaphoricity. Anaphoricity establishes a relation with another element in the (linguistic) discourse. Definiteness encodes that the referent is specific.

\ex.
\begin{forest} boom
 [\tsc{d}P
     [\tsc{d}]
     [\tsc{ana}]
 ]
 {\draw (.east) node[right]{⇔ \tit{d}}; }
\end{forest}
\label{ex:ohg-d-lexicon}














\section{The Old High German light head}


\subsection{The extra light head}

Headless relatives in which the relative pronoun starts with a \tit{d}, such as in Old High German, seem to be linked to individuating or definite readings and not to generalizing or indefinite readings \citep[cf.][]{fuss2017}. I illustrate this with the two examples I repeat from Chapter  \ref{ch:typology}.

Consider the example in \ref{ex:ohg-nom-acc-interpretation}, repeated from Chapter \ref{ch:typology}.
In this example, the author refers to the specific person which was talked about, and not to any or every person that was talked about.

%int = acc, ext = nom, so extra light head, but individuation so light head expected
\exg. Thíz ist \tbf{then} \tbf{sie} \tbf{zéllent}\\
\ac{dem}.\ac{sg}.\ac{n}.\ac{nom} be.\ac{pres}.3\ac{sg}\scsub{[nom]} \ac{rel}.\ac{sg}.\ac{m}.\ac{acc}
3\ac{pl}.\ac{m}.\ac{nom} tell.\ac{pres}.3\ac{pl}\scsub{[acc]}\\
`this is the one whom they talk about'\\
not: `this is whoever they talk about' \flushfill{Old High German, \ac{otfrid} III 16:50}\label{ex:ohg-nom-acc-interpretation}

Consider also the example in \ref{ex:ohg-nom-acc-interpretation}, repeated from Chapter \ref{ch:typology}.
In this example, the author refers to the specific person who spoke to someone, and not to any or every person who spoke to someone.

%int = nom, ext = dat, so light head, and individuation so light head also expected
\exg. enti aer {ant uurta} demo \tbf{zaimo} \tbf{sprah}\\
and 3\ac{sg}.\ac{m}.\ac{nom} reply.\ac{pst}.3\ac{sg}\scsub{[dat]} \ac{rel}.\ac{sg}.\ac{m}.\ac{dat} {to 3\ac{sg}.\ac{m}.\ac{dat}} speak.\ac{pst}.3\ac{sg}\scsub{[nom]}\\
`and he replied to the one who spoke to him'\\
not: `and he replied to whoever spoke to him'
 \flushfill{Old High German, \ac{mons} 7:24, adapted from \pgcitealt{pittner1995}{199}}\label{ex:ohg-dat-nom-rep}

\subsection{The light head}

Old High German is special because the relative pronoun in its headless relatives is syncretic with the relative pronoun in its light-headed relatives.\footnote{
What about Modern German \tit{der} - \tit{der}? Modern German has two different relative pronouns, so there is actually the choice!
}


This light head story never works for Modern German or Polish because for them the relative pronoun and the light head are not syncretic.

Consider the light-headed relative in \ref{ex:ohg-double}. \tit{Thér} `\tsc{dem}.\tsc{sg}.\tsc{m}.\tsc{nom}' is the head of the relative clause, which is the external element. \tit{Then} `\tsc{rp}.\tsc{sg}.\tsc{m}.\tsc{acc}' is the relative pronoun in the relative clause, which is the internal element.

\exg. eno nist thiz thér then ir suochet zi arslahanne?\\
 now {not be.3\ac{sg}} \tsc{dem}.\tsc{sg}.\tsc{n}.\tsc{nom} \tsc{dem}.\tsc{sg}.\tsc{m}.\tsc{nom}
 \tsc{rp}.\tsc{sg}.\tsc{m}.\tsc{acc} 2\ac{pl}.\tsc{nom} seek.2\tsc{pl} to kill.\tsc{inf}.\ac{sg}.\tsc{dat}\\
 `Isn't this now the one, who you seek to kill?'\label{ex:ohg-double}

I assume that whether both or only one of the elements surfaces is determined by information structure. In \ref{ex:ohg-double}, the external element \tit{thér} `\tsc{dem}.\tsc{sg}.\tsc{m}.\tsc{nom}' is the candidate to be absent. However, it seems plausible that this is emphasized in this sentence and that it, therefore, cannot be absent.



\section{Comparing constituents}\label{sec:comparing-ohg}

In this section, I compare the constituents of extra light heads and light heads to those of relative pronouns in Old High German. This is the worked out version of the comparisons in Section \ref{sec:basic-unrestricted}. What is different here is that I show the comparison for Old High German specifically, and that I motivated the content of the constituents that are being compared.

I give three examples, in which the internal and external case vary.
I start with an example with matching cases, in which the internal and the external case are both nominative. I show that the grammaticality of the example can be derived by either taking the extra light head or by taking the light head as the present light head.
Then I give an example in which the external accusative case is more complex than the internal nominative case. I show that the grammaticality of this example can only be derived by taking the light head as the present light head and not the extra light head. Before I can properly do that, I need to take a small detour into the larger syntactic structure of headless relatives.
I end with an example in which the internal accusative case is more complex than the external nominative case. I show that the grammaticality of this example can only be derived by taking the extra light head as the present light head and not the light head.


I start with the situation in which the cases match.
Consider the example in \ref{ex:ohg-nom-nom-rep}, in which the internal nominative case competes against the external nominative case. The relative clause is marked in bold. \ref{ex:ohg-nom-nom-elh} shows the example with the extra light head as the present light head and \ref{ex:ohg-nom-nom-lh} shows the example with the light head as the present light head.
The internal case is nominative, as the predicate \tit{senten} `to send' takes nominative subjects.
In both examples, the relative pronoun \tit{dher} `\ac{rel}.\ac{sg}.\ac{m}.\ac{nom}' appears in the nominative case.
The external case is nominative as well, as the predicate \tit{queman} `to come' also takes nominative subjects.
In \ref{ex:ohg-nom-nom-elh}, the extra light head \tit{r} `\tsc{elh}.\ac{sg}.\ac{m}.\ac{nom}' appears in the nominative case. It is placed between square brackets because it does not surface.
In \ref{ex:ohg-nom-nom-lh}, the light head \tit{dher} `\tsc{dem}.\ac{sg}.\ac{m}.\ac{nom}' appears in the nominative case. Here the relative pronoun is placed between square brackets because it does not surface.

\ex.\label{ex:ohg-nom-nom-rep}
\ag. quham [r] \tbf{dher} \tbf{chisendit} \tbf{scolda} \tbf{uuerdhan}\\
 come.\ac{pst}.3\ac{sg}\scsub{[nom]} \ac{elh}.\ac{sg}.\ac{m}.\ac{nom} \ac{rel}.\ac{sg}.\ac{m}.\ac{nom} send.\ac{pst}.\ac{ptcp}\scsub{[nom]} should.\ac{pst}.3\ac{sg} become.\ac{inf}\\
 `the one, who should have been sent, came' \flushfill{Old High German, \ac{isid} 35:5}\label{ex:ohg-nom-nom-elh}
\bg. quham dher [\tbf{dher}] \tbf{chisendit} \tbf{scolda} \tbf{uuerdhan}\\
 come.\ac{pst}.3\ac{sg}\scsub{[nom]} \ac{dem}.\ac{sg}.\ac{m}.\ac{nom} \ac{rel}.\ac{sg}.\ac{m}.\ac{nom} send.\ac{pst}.\ac{ptcp}\scsub{[nom]} should.\ac{pst}.3\ac{sg} become.\ac{inf}\\
 `the one, who should have been sent, came' \flushfill{Old High German, \ac{isid} 35:5}\label{ex:ohg-nom-nom-lh}

Both examples in \ref{ex:ohg-nom-nom-rep} can be the light-headed relative clause that the headless relative is derived from. First I show the comparison of the two constituents for \ref{ex:ohg-nom-nom-elh} and then the one for \ref{ex:ohg-nom-nom-lh}.

In Figure \ref{fig:ohg-int=ext-elh}, I give the syntactic structure of the extra light head at the top and the syntactic structure of the relative pronoun at the bottom.

\begin{figure}[htbp]
  \center
  \begin{adjustbox}{max height=0.9\textheight}
  \begin{tabular}[b]{c}
        \toprule
        \tsc{nom} extra light head \tit{r}\\
        \cmidrule{1-1}
      \begin{forest} boom
        [\tsc{nom}P,
        tikz={
        \node[label=below:{\tit{r}},
        draw,circle,
        scale=0.8,
        fit to=tree]{};
        \node[draw,circle,
        dashed,
        scale=0.85,
        fill=DG,fill opacity=0.2,
        fit to=tree]{};
        }
            [\tsc{f}1]
            [\tsc{ind}P
                [\phantom{xxx}, roof]
            ]
        ]
      \end{forest}
      \\
      \toprule
      \tsc{nom} relative pronoun \tit{dhe-r}
      \\
      \cmidrule{1-1}
          \begin{forest} boom
          [\tsc{rel}P
              [\tsc{rel}P
                  [\phantom{x}\tit{dhe}\phantom{x}, roof]
              ]
              [\tsc{nom}P,
              tikz={
              \node[label=below:{\tit{r}},
              draw,circle,
              scale=0.8,
              fit to=tree]{};
              \node[draw,circle,
              dashed,
              scale=0.85,
              fit to=tree]{};
              }
                  [\tsc{f}1]
                  [\tsc{ind}P
                      [\phantom{xxx}, roof]
                  ]
              ]
          ]
        \end{forest}
        \\
      \bottomrule
  \end{tabular}
  \end{adjustbox}
  \caption {Old High German \tsc{ext}\scsub{nom} vs. \tsc{int}\scsub{nom} → \tit{dher} (\tsc{elh})}
  \label{fig:ohg-int=ext-elh}
\end{figure}

The relative pronoun consists of two morphemes: \tit{dhe} and \tit{r}.
The extra light head consists of a single morpheme: \tit{r}.
As usual, I circle the part of the structure that corresponds to a particular lexical entry, and I place the corresponding phonology under it, or I reduce the structure to a triangle, and I place the corresponding phonology under it.
I draw a dashed circle around each constituent that is a constituent in both the extra light head and the relative pronoun.

The extra light head consists of a single constituent: the \tsc{nom}P.
This \tsc{nom}P is also a constituent within the relative pronoun. Therefore, the extra light head can be deleted. I signal the deletion of the extra light head by marking the content of its circle gray.

In Figure \ref{fig:ohg-int=ext-lh}, I give the syntactic structure of the light head at the top and the syntactic structure of the relative pronoun at the bottom.

\begin{figure}[htbp]
  \center
  \begin{adjustbox}{max height=0.9\textheight}
  \begin{tabular}[b]{c}
        \toprule
        \tsc{nom} light head \tit{dhe-r}\\
        \cmidrule{1-1}
        \begin{forest} boom
          [\tsc{d}P, s sep=20mm,
          tikz={
          \node[draw,
          constituent-deletion,yshift=-0.4cm,
          dotted,
          scale=1.25,
          fit to=tree]{};
          }
              [\tsc{d}P,
              tikz={
              \node[label=below:\tit{dhe},
              draw,circle,
              scale=0.85,
              fit to=tree]{};
              }
                  [\phantom{xxx}, roof, baseline]
              ]
              [\tsc{nom}P,
              tikz={
              \node[label=below:\tit{r},
              draw,circle,
              scale=0.85,
              fit to=tree]{};
              }
                  [\tsc{f}1]
                  [\tsc{ind}P
                      [\phantom{xxx}, roof, baseline]
                  ]
              ]
          ]
        \end{forest}
      \\
      \toprule
      \tsc{nom} relative pronoun \tit{dhe-r}
      \\
      \cmidrule{1-1}
      \begin{forest} boom
        [\tsc{rel}P, s sep=20mm,
        tikz={
        \node[draw,
        constituent-deletion,yshift=-0.4cm,
        dotted,
        fill=DG,fill opacity=0.2,
        scale=1.2,
        fit to=tree]{};
        }
            [\tsc{rel}P,
            tikz={
            \node[label=below:\tit{dhe},
            draw,circle,
            scale=0.85,
            fit to=tree]{};
            }
                [\tsc{rel}]
                [\tsc{d}P
                    [\phantom{xxx}, roof, baseline]
                ]
            ]
            [\tsc{nom}P,
            tikz={
            \node[label=below:\tit{r},
            draw,circle,
            scale=0.85,
            fit to=tree]{};
            }
                [\tsc{f}1]
                [\tsc{ind}P
                    [\phantom{xxx}, roof, baseline]
                ]
            ]
        ]
      \end{forest}
        \\
      \bottomrule
  \end{tabular}
  \end{adjustbox}
  \caption {Old High German \tsc{ext}\scsub{nom} vs. \tsc{int}\scsub{nom} → \tit{dher} (\tsc{elh})}
  \label{fig:ohg-int=ext-lh}
\end{figure}

The relative pronoun consists of two morphemes: \tit{dhe} and \tit{r}.
The light head also consists of two morphemes: \tit{dhe} and \tit{r}.
I circle the part of the structure that corresponds to a particular lexical entry, and I place the corresponding phonology under it, or I reduce the structure to a triangle, and I place the corresponding phonology under it.
I draw a dotted circle around each constituent that is a constituent in both the light head and the relative pronoun.

The light head (the \tsc{d}P realized by \tit{dher}) is syncretic with the relative pronoun (the \tsc{rel}P realized by \tit{dher}).
As the two forms are entirely syncretic, either the light head or the relative pronoun can be deleted. I delete the relative pronoun here, as I discuss the situation in which the relative pronoun is deleted.
I illustrate this by marking the content of the dotted circle for the relative pronoun gray.


The external case is nominative as well, as the predicate \tit{queman} `to come' also takes nominative subjects.
In \ref{ex:ohg-nom-nom-elh}, the extra light head \tit{r} `\tsc{elh}.\ac{sg}.\ac{m}.\ac{nom}' appears in the nominative case. It is placed between square brackets because it does not surface.
In \ref{ex:ohg-nom-nom-lh}, the light head \tit{dher} `\tsc{dem}.\ac{sg}.\ac{m}.\ac{nom}' appears in the nominative case. Here the relative pronoun is placed between square brackets because it does not surface.

I continue with the situation in which the external case is the more complex one.
Consider the examples in \ref{ex:ohg-acc-nom-rep}, in which the internal nominative case competes against the external accusative case. The relative clause is marked in bold. \ref{ex:ohg-acc-nom-elh} shows the example with the extra light head as the present light head and \ref{ex:ohg-acc-nom-lh} shows the example with the light head as the present light head.
The internal case is nominative, as the predicate \tit{gisizzen} `to possess' takes nominative subjects.
In both examples, the relative pronoun \tit{dher} `\ac{rel}.\ac{sg}.\ac{m}.\ac{nom}' appears in the nominative case.
The external case is accusative, as the predicate \tit{bibringan} `to create' takes accusative objects.
In \ref{ex:ohg-acc-nom-elh}, the extra light head \tit{n} `\tsc{elh}.\ac{sg}.\ac{m}.\ac{acc}' appears in the accusative case. It is placed between square brackets because it does not surface.
In \ref{ex:ohg-acc-nom-lh}, the light head \tit{dhen} `\tsc{dem}.\ac{sg}.\ac{m}.\ac{acc}' appears in the accusative case. Here the relative pronoun is placed between square brackets because it does not surface.

\ex.\label{ex:ohg-acc-nom-rep}
\ag. *ih bibringu fona iacobes samin endi fona iuda [n] \tbf{dher} \tbf{mina} \tbf{berga} \tbf{chisitzit}\\
1\ac{sg}.\ac{nom} {create}.\ac{pres}.1\ac{sg}\scsub{[acc]} of Jakob.\ac{gen} seed.\ac{sg}.\ac{dat} and of Judah.\ac{dat} \ac{rel}.\ac{sg}.\ac{m}.\ac{acc} my.\ac{acc}.\ac{m}.\ac{pl} mountain.\ac{acc}.\ac{pl} possess.\ac{pres}.3\ac{sg}\scsub{[nom]}\\
`I create of the seed of Jacob and of Judah the one, who possess my mountains' \flushfill{Old High German, \ac{isid} 34:3}\label{ex:ohg-acc-nom-elh}
\bg. ih bibringu fona iacobes samin endi fona iuda dhen [\tbf{dher}] \tbf{mina} \tbf{berga} \tbf{chisitzit}\\
1\ac{sg}.\ac{nom} {create}.\ac{pres}.1\ac{sg}\scsub{[acc]} of Jakob.\ac{gen} seed.\ac{sg}.\ac{dat} and of Judah.\ac{dat} \ac{rel}.\ac{sg}.\ac{m}.\ac{acc} my.\ac{acc}.\ac{m}.\ac{pl} mountain.\ac{acc}.\ac{pl} possess.\ac{pres}.3\ac{sg}\scsub{[nom]}\\
`I create of the seed of Jacob and of Judah the one, who possess my mountains' \flushfill{Old High German, \ac{isid} 34:3}\label{ex:ohg-acc-nom-lh}

Only \ref{ex:ohg-acc-nom-lh} can be the light-headed relative that the headless relative is derived from. First I show by comparing the constituents that no headless relative can be derived from the \ref{ex:ohg-acc-nom-elh}. Then I show the comparison of the two constituents for \ref{ex:ohg-acc-nom-lh}, which does derive a grammatical result.

In Figure \ref{fig:mg-ext-wins-elh}, I give the syntactic structure of the extra light head at the top and the syntactic structure of the relative pronoun at the bottom.

\begin{figure}[htbp]
  \center
  \begin{adjustbox}{max height=0.9\textheight}
  \begin{tabular}[b]{c}
      \toprule
      \tsc{acc} extra light head \tit{n}
      \\
      \cmidrule{1-1}
      \begin{forest} boom
        [\tsc{acc}P,
        tikz={
        \node[label=below:{\tit{n}},
        draw,circle,
        scale=0.85,
        fit to=tree]{};
        }
            [\tsc{f}2]
            [\tsc{acc}P,
            tikz={
            \node[draw,circle,
            dashed,
            scale=0.8,
            fit to=tree]{};
            }
                [\ac{f}1]
                [\tsc{ind}P
                    [\phantom{xxx}, roof]
                ]
            ]
        ]
      \end{forest}
      \\
      \toprule
      \tsc{nom} relative pronoun \tit{dhe-r}
      \\
      \cmidrule{1-1}
          \begin{forest} boom
            [\tsc{rel}P
                [\tsc{rel}P
                    [\phantom{x}\tit{dhe}\phantom{x}, roof]
                ]
                [\tsc{acc}P,
                tikz={
                \node[label=below:{\tit{r}},
                draw,circle,
                scale=0.8,
                fit to=tree]{};
                \node[draw,circle,
                dashed,
                scale=0.85,
                fit to=tree]{};
                }
                    [\ac{f}1]
                    [\tsc{ind}P
                        [\phantom{xxx}, roof]
                    ]
                ]
            ]
        \end{forest}
        \\
      \bottomrule
  \end{tabular}
  \end{adjustbox}
   \caption {Old High German \tsc{ext}\scsub{acc} vs. \tsc{int}\scsub{nom} ↛ \tit{n}/\tit{dher} (\tsc{elh})}
  \label{fig:ohg-ext-wins-elh}
\end{figure}

The relative pronoun consists of two morphemes: \tit{dhe} and \tit{r}.
The extra light head consists of a single morpheme: \tit{n}.
Again, I circle the part of the structure that corresponds to a particular lexical entry, and I place the corresponding phonology under it, or I reduce the structure to a triangle, and I place the corresponding phonology under it.
I draw a dashed circle around each constituent that is a constituent in both the extra light head and the relative pronoun.

The extra light head consists of a single constituent: the \tsc{acc}P.
In this case, the relative pronoun does not contain this constituent. The relative pronoun only contains the \tsc{nom}P, and it lacks the \tsc{f}2 that makes a \tsc{acc}P. Since the weaker feature containment requirement is not met, the stronger constituent containment requirement cannot be met either.
The extra light head also does not contain all constituents or features that the relative pronoun contains, because it lacks the complete constituent and \tsc{rel}P.
Therefore, the extra light cannot be deleted, and the relative pronoun cannot be deleted either.


In what follows, I show that the light-headed relative that contains the light head is the structure that the headless relative is derived from. I cannot show this by directly comparing the constituents that the light head and the relative pronoun consist of. I need to make some assumptions I am making regarding the larger syntactic structure explicit. Before I do that, and I derive the grammatical result, I show that a direct comparison of the light head and the relative pronoun does not lead to a grammatical result.

In Figure \ref{fig:ohg-ext-wins-lh}, I give the syntactic structure of the light head at the top and the syntactic structure of the relative pronoun at the bottom.

\begin{figure}[htbp]
  \center
  \begin{adjustbox}{max height=0.9\textheight}
  \begin{tabular}[b]{c}
        \toprule
        \tsc{nom} light head \tit{dhe-n}\\
        \cmidrule{1-1}
        \begin{forest} boom
          [\tsc{d}P, s sep=20mm,
          tikz={
          \node[draw,
          constituent-deletion,yshift=-0.4cm,
          dotted,
          scale=1.25,
          fit to=tree]{};
          }
              [\tsc{d}P,
              tikz={
              \node[label=below:\tit{dhe},
              draw,circle,
              scale=0.85,
              fit to=tree]{};
              }
                  [\phantom{xxx}, roof, baseline]
              ]
              [\tsc{acc}P,
              tikz={
              \node[label=below:\tit{r},
              draw,circle,
              scale=0.85,
              fit to=tree]{};
              }
                  [\tsc{f}2]
                  [\tsc{nom}P
                      [\tsc{f}1]
                      [\tsc{ind}P
                          [\phantom{xxx}, roof, baseline]
                      ]
                  ]
              ]
          ]
        \end{forest}
      \\
      \toprule
      \tsc{nom} relative pronoun \tit{dhe-r}
      \\
      \cmidrule{1-1}
      \begin{forest} boom
        [\tsc{rel}P, s sep=20mm,
        tikz={
        \node[draw,
        constituent-deletion,yshift=-0.4cm,
        dotted,
        fill=DG,fill opacity=0.2,
        scale=1.2,
        fit to=tree]{};
        }
            [\tsc{rel}P,
            tikz={
            \node[label=below:\tit{dhe},
            draw,circle,
            scale=0.85,
            fit to=tree]{};
            }
                [\tsc{rel}]
                [\tsc{d}P
                    [\phantom{xxx}, roof, baseline]
                ]
            ]
            [\tsc{nom}P,
            tikz={
            \node[label=below:\tit{r},
            draw,circle,
            scale=0.85,
            fit to=tree]{};
            }
                [\tsc{f}1]
                [\tsc{ind}P
                    [\phantom{xxx}, roof, baseline]
                ]
            ]
        ]
      \end{forest}
        \\
      \bottomrule
  \end{tabular}
  \end{adjustbox}
  \caption {Old High German \tsc{ext}\scsub{acc} vs. \tsc{int}\scsub{nom} ↛ \tit{dhen}/\tit{ther} (\tsc{lh})}
  \label{fig:ohg-ext-wins-lh}
\end{figure}

this does not work, but the example is grammatical!
so by simply comparing the two, we're not gonna get there

as I already alluded to in the basic idea, there is a stage in the derivation, in which the two cases are both nominative, and on that moment, the relative pronoun is deleted

if we start talking about `stages in the derivation' we need to know more about the derivation.
So far, I have only be talking about the light head and the relative pronoun separately. I have not placed them in a larger syntactic structure. For the discussion so far that has also not been relevant: whatever is going on, it is the same for all languages.

now that I need to make reference to a stage in the derivation, we need to know what the larger syntactic structure looks like.

I assume that a syntactic structure of a light-headed relative looks as in X\footnote{
I actually assume that a light-headed relative with an extra light looks as in X

\ex. here an example with a non-D with the head in the low position

here explain what is in the example.
This is also what Cinque says: non-definite heads are low, and definite heads are high.
Two questions follow from such an analysis: (1) how do the case features end up down there, and (2) what triggers the movement of the light head to the higher position. About (1), Cinque says that it is feature percolation, and I follow that intuition. Technically, what's happening is backtracking, opening up the different workspaces, which leads to the case features finding a match on the element to the left of the relative clause. Concerning (2), Cinque says it's movement, I'm not sure what it's triggered by. I don't know what it is. If it's movement, then it can be triggered by spellout or by features. I don't see how either of them should work. It could be connected to the formation of a complex spec. It seems that as soon the spec is there, the light head also moves up, and the complex spec does not attach to the relative clause. I leave this for future research.
}

\ex. here an example of a high D and the relative clause below it

There is a D, which appears higher in the structure than the relative clause, etc. etc. explanation
the relative clause is already complete, including case features
This structure for light-headed relatives is also assumed by cf. Cinque etc. etc.
the features that are merged last in building a light head are the case features. first we have a D without case features, and then the case features are merged on by one. this means that we have a stage in the dervation dathat looks like:

\ex. no cases, including relative pronoun and relative clause

\ex. only nominative case

now there's deletion!

then we merge the next case feature, and we get a more complex external case

note that we also have c-command for the deletion! great!\footnote{
coming back the extra light head, we also have c-command there, under the defininteino of kayne
}





- ext wins doesn't work, so first larger syntactic structure: head needs to be up there, for instance cinque (he also says we need to have it up there)
then: yes, at some point in the derivation, and then merge the k2

%detour to larger syntactic structure


%internal case more complex

I continue with the situation in which the internal case is the more complex one.
Consider the example in \ref{ex:ohg-nom-acc-rep}, in which the internal accusative case competes against the external nominative case. The relative clause is marked in bold.
The internal case is accusative, as the predicate \tit{zellen} `to tell' takes accusative objects. The relative pronoun \tit{then} `\ac{rel}.\ac{sg}.\ac{m}.\ac{acc}' appears in the accusative case. This is the element that surfaces.
The external case is nominative, as the predicate \tit{sin} `to be' takes nominative objects. The light head \tit{r} `\tsc{elh}.\ac{sg}.\ac{m}.\ac{nom}' appears in the nominative case. It is placed between square brackets because it does not surface.

\exg. Thíz ist [r] \tbf{then} \tbf{sie} \tbf{zéllent}\\
\ac{dem}.\ac{sg}.\ac{n}.\ac{nom} be.\ac{pres}.3\ac{sg}\scsub{[nom]} \ac{dem}.\ac{sg}.\ac{m}.\ac{nom} \ac{rel}.\ac{sg}.\ac{m}.\ac{acc} 3\ac{pl}.\ac{m}.\ac{nom} tell.\ac{pres}.3\ac{pl}\scsub{[acc]}\\
`this is the one whom they talk about' \flushfill{Old High German, \ac{otfrid} III 16:50}\label{ex:ohg-nom-acc-rep}

%I show how this can never work, because it's the wrong one that's being deleted!
\exg. Thíz ist ther [\tbf{then}] \tbf{sie} \tbf{zéllent}\\
\ac{dem}.\ac{sg}.\ac{n}.\ac{nom} be.\ac{pres}.3\ac{sg}\scsub{[nom]} \ac{dem}.\ac{sg}.\ac{m}.\ac{nom} \ac{rel}.\ac{sg}.\ac{m}.\ac{acc} 3\ac{pl}.\ac{m}.\ac{nom} tell.\ac{pres}.3\ac{pl}\scsub{[acc]}\\
`this is the one whom they talk about' \flushfill{Old High German, \ac{otfrid} III 16:50}\label{ex:ohg-nom-acc-rep}

In Figure \ref{fig:ohg-int-wins}, I give the syntactic structure of the extra light head at the top and the syntactic structure of the relative pronoun at the bottom.

\begin{figure}[htbp]
  \center
  \begin{adjustbox}{max height=0.9\textheight}
  \begin{tabular}[b]{c}
      \toprule
      \tsc{nom} extra light head \tit{r}
      \\
      \cmidrule{1-1}
      \begin{forest} boom
        [\tsc{nom}P,
        tikz={
        \node[label=below:{\tit{r}},
        draw,circle,
        scale=0.8,
        fit to=tree]{};
        \node[draw,circle,
        dashed,
        scale=0.85,
        fill=DG,fill opacity=0.2,
        fit to=tree]{};
        }
            [\tsc{f}1]
            [\tsc{ind}P
                [\phantom{xxx}, roof]
            ]
        ]
      \end{forest}
      \\
      \toprule
      \tsc{acc} relative pronoun \tit{the-n}
      \\
      \cmidrule{1-1}
          \begin{forest} boom
            [\tsc{rel}P
                [\tsc{rel}P
                    [\phantom{x}\tit{the}\phantom{x}, roof]
                ]
                [\tsc{acc}P,
                tikz={
                \node[label=below:{\tit{n}},
                draw,circle,
                scale=0.85,
                fit to=tree]{};
                }
                    [\tsc{f}3]
                    [\tsc{acc}P,
                    tikz={
                    \node[draw,circle,
                    dashed,
                    scale=0.8,
                    fit to=tree]{};
                    }
                        [\tsc{f}1]
                        [\tsc{ind}P
                            [\phantom{xxx}, roof]
                        ]
                    ]
                ]
            ]
        \end{forest}
        \\
      \bottomrule
  \end{tabular}
  \end{adjustbox}
   \caption {Old High German \tsc{ext}\scsub{nom} vs. \tsc{int}\scsub{acc} → \tit{then}}
  \label{fig:ohg-int-wins}
\end{figure}

The relative pronoun consists of two morphemes: \tit{the} and \tit{n}.
The extra light head consists of a single morpheme: \tit{r}.
Again, I circle the part of the structure that corresponds to a particular lexical entry, and I place the corresponding phonology under it, or I reduce the structure to a triangle, and I place the corresponding phonology under it.
I draw a dashed circle around each constituent that is a constituent in both the extra light head and the relative pronoun.

The extra light head consists of a single constituent: the \tsc{nom}P.
This \tsc{nom}P is also a constituent within the relative pronoun. Therefore, the extra light can be deleted. I signal the deletion of the extra light head by marking the content of its circle gray.





\section{Possible predictions}

  - possible prediction: ext>int = def, int>ext = wh, not what we see, show 4 examples



Consider the example in \ref{ex:ohg-nom-acc-interpretation}, repeated from Chapter \ref{ch:typology}.
In this example, the author refers to the specific person which was talked about, and not to any or every person that was talked about.

%int = acc, ext = nom, so extra light head, but individuation so light head expected
\exg. Thíz ist \tbf{then} \tbf{sie} \tbf{zéllent}\\
\ac{dem}.\ac{sg}.\ac{n}.\ac{nom} be.\ac{pres}.3\ac{sg}\scsub{[nom]} \ac{rel}.\ac{sg}.\ac{m}.\ac{acc}
3\ac{pl}.\ac{m}.\ac{nom} tell.\ac{pres}.3\ac{pl}\scsub{[acc]}\\
`this is the one whom they talk about'\\
not: `this is whoever they talk about' \flushfill{Old High German, \ac{otfrid} III 16:50}\label{ex:ohg-nom-acc-interpretation}

Consider also the example in \ref{ex:ohg-nom-acc-interpretation}, repeated from Chapter \ref{ch:typology}.
In this example, the author refers to the specific person who spoke to someone, and not to any or every person who spoke to someone.

%int = nom, ext = dat, so light head, and individuation so light head also expected
\exg. enti aer {ant uurta} demo \tbf{zaimo} \tbf{sprah}\\
and 3\ac{sg}.\ac{m}.\ac{nom} reply.\ac{pst}.3\ac{sg}\scsub{[dat]} \ac{rel}.\ac{sg}.\ac{m}.\ac{dat} {to 3\ac{sg}.\ac{m}.\ac{dat}} speak.\ac{pst}.3\ac{sg}\scsub{[nom]}\\
`and he replied to the one who spoke to him'\\
not: `and he replied to whoever spoke to him'
 \flushfill{Old High German, \ac{mons} 7:24, adapted from \pgcitealt{pittner1995}{199}}\label{ex:ohg-dat-nom-rep}

\phantom{x}
