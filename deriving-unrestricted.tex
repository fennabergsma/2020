% !TEX root = thesis.tex

\chapter{Deriving the unrestricted type}\label{ch:deriving-unrestricted}

In Chapter \ref{ch:the-basic-idea}, I suggested that languages of the unrestricted type have two possible light heads, which are part of the derivation under different circumstances.
The first possible light head derives the pattern correctly for for the situation in which cases match and the situation in which internal case is more complex than the external case.
The second possible light head derives the pattern correctly for for the situation in which cases match and the situation in which external case is more complex than the internal case.

The first possible light head has the same internal syntax as the extra light head in internal-only languages, such as Modern German. It is spelled out by a portmanteau for phi and case features. The relative pronoun is spelled out by that same portmanteau plus a separate lexical entry that spells out the feature \tsc{rel}. This means that the internal syntax of light heads and relative pronouns looks as shown in Figure \ref{fig:rel-lh-unres-simple-1}.

\begin{figure}[htbp]
  \center
  \begin{tabular}[b]{ccc}
      \toprule
      light head 1 & & relative pronoun \\
      \cmidrule(lr){1-1} \cmidrule(lr){3-3}
      \begin{forest} boom
      [\tsc{k}P,
      tikz={
      \node[draw,circle,
      scale=0.85,
      fit to=tree]{};
      }
          [\tsc{k}]
          [ϕP
              [\phantom{xxx}, roof, baseline]
          ]
      ]
      \end{forest}
      & \phantom{x} &
    \begin{forest} boom
      [\tsc{rel}P, s sep = 17.5 mm
          [\tsc{rel}P,
          tikz={
          \node[draw,circle,
          scale=0.85,
          fit to=tree]{};
          }
              [\phantom{xxx}, roof, baseline]
          ]
          [\tsc{k}P,
          tikz={
          \node[draw,circle,
          scale=0.85,
          fit to=tree]{};
          }
              [\tsc{k}]
              [ϕP
                  [\phantom{xxx}, roof, baseline]
              ]
          ]
      ]
    \end{forest}\\
      \bottomrule
  \end{tabular}
   \caption {\tsc{lh}-1 and \tsc{rp} in the unrestricted type}
  \label{fig:rel-lh-unres-simple-1}
\end{figure}

These lexical entries lead to the grammaticality pattern shown in Table  \ref{tbl:overview-unres-1}.

\begin{table}[htbp]
  \center
  \caption{Grammaticality in the unrestricted type (part 1)}
  \begin{adjustbox}{max width=\textwidth}
  \begin{tabular}{cccccc}
    \toprule
    situation           & \multicolumn{2}{c}{lexical entries}       & containment         & deleted             & surfacing           \\
    \cmidrule(lr){1-1}    \cmidrule(lr){2-3}                          \cmidrule(lr){4-4}    \cmidrule(lr){5-5}    \cmidrule(lr){6-6}
                        & \tsc{lh}            & \tsc{rp}            &                     &                     &                     \\
                          \cmidrule(lr){2-2}    \cmidrule(lr){3-3}
  \tsc{k}\scsub{int} = \tsc{k}\scsub{ext}               &
  [\tsc{k}\scsub{1}[ϕ]]                                 &
  [\tsc{rel}], [\tsc{k}\scsub{1}[ϕ]]                    &
  structure & \tsc{lh} & \tsc{rp}\scsub{int}            \\
  \tsc{k}\scsub{int} > \tsc{k}\scsub{ext}               &
  [\tsc{k}\scsub{1}[ϕ]]                                 &
  [\tsc{rel}], [\tsc{k}\scsub{2}[\tsc{k}\scsub{1}[ϕ]]]  &
  structure & \tsc{lh} & \tsc{rp}\scsub{int}            \\
  \tsc{k}\scsub{int} < \tsc{k}\scsub{ext}               &
  [\tsc{rel}], [\tsc{k}\scsub{1}[ϕ]]                    &
  [\tsc{k}\scsub{2}[\tsc{k}\scsub{1}[ϕ]]]               &
  no & none & *                                         \\
  \bottomrule
  \end{tabular}
  \end{adjustbox}
\label{tbl:overview-unres-1}
\end{table}

Consider first the situation in which the internal and the external case match. The situation here is identical to the one in the internal-only type of language. The light head consists of a phi and case feature portmanteau. The relative pronoun consists of the same morpheme plus an additional morpheme that spells out the feature \tsc{rel}. The lexical entries create a syntactic structure such that the light head is structurally contained within the relative pronoun. Therefore, the light head can be deleted, and the relative pronoun surfaces, bearing the internal case.

Consider now the situation in the internal case wins the case competition. Here the situation is identical to the one in the internal-only type of language too. The light head consists of a phi and case feature portmanteau. The relative pronoun consists of a phi and case feature portmanteau that contains at least one more case feature than the light head (\tsc{k}\scsub{2} in Figure \ref{tbl:overview-unres-1}) plus an additional morpheme that spells out the feature \tsc{rel}. The lexical entries create a syntactic structure such that the light head is structurally contained within the relative pronoun. Therefore, the light head can be deleted, and the relative pronoun surfaces, bearing the internal case.

Consider now the situation in the internal case wins the case competition. Also here the situation is identical to the one in the internal-only type of language. The relative pronoun consists of a phi and case feature portmanteau and an additional morpheme that spells out the feature \tsc{rel}. Compared to the relative pronoun, the light head lacks the morpheme that spells out \tsc{rel}, and it contains at least one more case feature (\tsc{k}\scsub{2} in Figure \ref{tbl:overview-unres-1}). The lexical entries create a syntactic structure such that neither the light head nor the relative pronoun is a constituent that is contained within the other element. Therefore, none of the elements can be deleted, and there is no headless relative construction possible.

In Chapter \ref{ch:typology}, I showed that Old High German is a language of the unrestricted type. In this chapter, I show that Old High German has light heads and relative pronouns of type of structure described in Figure \ref{fig:rel-lh-unres-simple-1}. I give a compact version of the structures in Figure \ref{fig:rel-lh-ohg-1}.

\begin{figure}[htbp]
  \center
  \begin{tabular}[b]{ccc}
      \toprule
      light head & & relative pronoun \\
      \cmidrule(lr){1-1} \cmidrule(lr){3-3}
      \begin{forest} boom
        [\tsc{k}P,
        tikz={
        \node[label=below:\tit{ër/ën},
        draw,circle,
        scale=0.75,
        fit to=tree]{};
        }
            [\tsc{k}]
            [ϕP
                [\phantom{xxx}, roof, baseline]
            ]
        ]
      \end{forest}
      & \phantom{x} &
      \begin{forest} boom
        [\tsc{rel}P, s sep=15mm
            [\tsc{rel}P,
            tikz={
            \node[label=below:\tit{d},
            draw,circle,
            scale=0.75,
            fit to=tree]{};
            }
                [\phantom{xxx}, roof]
            ]
            [\tsc{k}P,
            tikz={
            \node[label=below:\tit{ër/ën},
            draw,circle,
            scale=0.75,
            fit to=tree]{};
            }
                [\tsc{k}]
                [ϕP
                    [\phantom{xxx}, roof, baseline]
                ]
            ]
        ]
      \end{forest}\\
      \bottomrule
  \end{tabular}
   \caption {\tsc{lh}-1 and \tsc{rp} in Old High German}
  \label{fig:rel-lh-ohg-1}
\end{figure}

Consider the first possible light head in Figure \ref{fig:rel-lh-ohg-1}.
These light heads (i.e. phi and case features) in Old High German are spelled out by a single morpheme, indicated by the circle around the structure. They are spelled out as \tit{ër} or \tit{ën}, depending on which case they realize.
Consider the relative pronoun in Figure \ref{fig:rel-lh-ohg-1}.
Relative pronouns in Old High German consist of two morphemes: the constituent that forms the light head (i.e. phi and case features) and the \tsc{rel}P, again indicated by the circles. The constituent that forms the light head has the same spellout as in the light head (\tit{ën} or \tit{m}), and the \tsc{rel}P is spelled out as \tit{d}.
Throughout this chapter, I discuss the exact feature content of light heads and relative pronouns, I give lexical entries for them, and I show how these lexical entries lead to the internal syntax of light heads and relative pronoun as shown in Figure \ref{fig:rel-lh-ohg-1}.

The second possible light head differs from the first possible head in that it contains an additional feature X. The phi and case features are still spelled out by the phi and case portmanteau. The additional feature X is spelled out by its own lexical entry.
The relative pronoun is spelled out by that same portmanteau plus a separate lexical entry that spells out the feature X and the feature \tsc{rel}. This means that the internal syntax of light heads and relative pronouns looks as shown in Figure \ref{fig:rel-lh-unres-simple-1}.

\begin{figure}[htbp]
  \center
  \begin{adjustbox}{max width=\textwidth}
  \begin{tabular}[b]{ccc}
      \toprule
      light head 2 & & relative pronoun \\
      \cmidrule(lr){1-1} \cmidrule(lr){3-3}
      \begin{forest} boom
      [XP, s sep = 17.5 mm
          [XP,
          tikz={
          \node[label=below:\tit{X},
          draw,circle,
          scale=0.85,
          fit to=tree]{};
          }
              [\phantom{xxx}, roof, baseline]
          ]
          [\tsc{k}P,
          tikz={
          \node[draw,circle,
          scale=0.85,
          fit to=tree]{};
          }
              [\tsc{k}]
              [ϕP
                  [\phantom{xxx}, roof, baseline]
              ]
          ]
      ]
      \end{forest}
      & \phantom{x} &
    \begin{forest} boom
      [\tsc{rel}P, s sep = 17.5 mm
          [\tsc{rel}P,
          tikz={
          \node[label=below:\tit{X},
          draw,circle,
          scale=0.85,
          fit to=tree]{};
          }
              [\tsc{rel}]
              [XP
                  [\phantom{xxx}, roof, baseline]
              ]
          ]
          [\tsc{k}P,
          tikz={
          \node[draw,circle,
          scale=0.85,
          fit to=tree]{};
          }
              [\tsc{k}]
              [ϕP
                  [\phantom{xxx}, roof, baseline]
              ]
          ]
      ]
    \end{forest}\\
      \bottomrule
  \end{tabular}
  \end{adjustbox}
   \caption {\tsc{lh}-2 and \tsc{rp} in the unrestricted type}
  \label{fig:rel-lh-unres-simple-2}
\end{figure}

These lexical entries lead to the grammaticality pattern shown in Table \ref{tbl:overview-unres-2}.

\begin{table}[htbp]
  \center
  \caption{Grammaticality in the unrestricted type (part 2)}
  \begin{adjustbox}{max width=\textwidth}
  \begin{tabular}{cccccc}
    \toprule
    situation           & \multicolumn{2}{c}{lexical entries}       & containment         & deleted             & surfacing           \\
    \cmidrule(lr){1-1}    \cmidrule(lr){2-3}                          \cmidrule(lr){4-4}    \cmidrule(lr){5-5}    \cmidrule(lr){6-6}
                        & \tsc{lh}-2           & \tsc{rp}            &                     &                     &                     \\
                          \cmidrule(lr){2-2}    \cmidrule(lr){3-3}
  \tsc{k}\scsub{int} = \tsc{k}\scsub{ext}               &
  \tit{/X/}, \tit{/Y/}                                  &
  \tit{/X/}, \tit{/Y/}                                  &
  form & \tsc{rp} & \tsc{lh}\scsub{ext}                 \\
  \tsc{k}\scsub{int} > \tsc{k}\scsub{ext}               &
  \tit{/X/}, \tit{/Y/}                                  &
  \tit{/X/}, \tit{/Z/}                                  &
  no & none & *                                         \\
  \tsc{k}\scsub{int} < \tsc{k}\scsub{ext}               &
  \tit{/X/}, \tit{/Y/}                                  &
  \tit{/X/}, \tit{/Y/}                                  &
  form & \tsc{rp} & \tsc{lh}\scsub{ext}                 \\
  \bottomrule
  \end{tabular}
  \end{adjustbox}
\label{tbl:overview-unres-2}
\end{table}

Consider first the situation in which the internal and the external case match. The light head consists of a phi and case feature portmanteau plus a morpheme that spells out the feature X, which corresponds to phonological form \tit{X}. The relative pronoun consists of the same phi plus case feature morpheme and a morpheme that spells out the feature X and the feature \tsc{rel}, which corresponds to the phonological form \tit{X} too. The lexical entries create a syntactic structure such that the light head and the relative pronoun are syncretic, so they both form formally contained within the other element. Therefore, the one of the elements can be deleted, and the other one surfaces, bearing the internal and external case.

Consider now the situation in the internal case wins the case competition. The light head consists of a phi and case feature portmanteau plus a morpheme that spells out the feature X, which corresponds to phonological form \tit{X}. The relative pronoun consists of a phi and case feature portmanteau that contains at least one more case feature than the light head (\tsc{k}\scsub{2} in Figure \ref{tbl:overview-unres-2}) plus a morpheme that spells out the feature X and the feature \tsc{rel}, which corresponds to the phonological form \tit{X} too.
The lexical entries create a syntactic structure such that neither the light head nor the relative pronoun is a constituent that is contained within the other element. Therefore, none of the elements can be deleted, and there is no headless relative construction possible.

Finally, consider the situation in which the external case wins the case competition. The relative pronoun consists of the same phi plus case feature morpheme and a morpheme that spells out the feature X and the feature \tsc{rel}, which corresponds to the phonological form \tit{X}. Compared to the relative pronoun, the light head lacks the feature \tsc{rel} and only the feature X spells out as \tit{X}, and it contains at least one more case feature (\tsc{k}\scsub{2} in Figure \ref{tbl:overview-intonly}). The lexical entries create a syntactic structure such that neither the light head nor the relative pronoun is a constituent that is contained within the other element. Therefore, none of the elements can be deleted, and there is no headless relative construction possible.
However, the derivation in which the external case is more complex than the internal one goes through a stage in which the internal and the external case match. Therefore, at that stage, these lexical entries create a syntactic structure such that the light head and the relative pronoun are syncretic, so the relative pronoun forms formally contained within the light head. Therefore, the relative pronoun can be deleted, and the light head remains, bearing external case. Then, the remaining complex case features are merged to the light head, and the light head surfaces, bearing the more complex external case.

In Chapter \ref{ch:typology}, I showed that Old High German is a language of the unrestricted type. In this chapter, I show that Old High German has light heads and relative pronouns of type of structure described in Figure \ref{fig:rel-lh-unres-simple-2}. The feature I so far called X is replaced here by D. I give a compact version of the structures in Figure \ref{fig:rel-lh-ohg-2}.

\begin{figure}[htbp]
  \center
  \begin{adjustbox}{max width=\textwidth}
  \begin{tabular}[b]{ccc}
      \toprule
      light head 2 & & relative pronoun \\
      \cmidrule(lr){1-1} \cmidrule(lr){3-3}
      \begin{forest} boom
      [DP, s sep = 17.5 mm
          [DP,
          tikz={
          \node[label=below:\tit{d},
          draw,circle,
          scale=0.85,
          fit to=tree]{};
          }
              [\phantom{xxx}, roof, baseline]
          ]
          [\tsc{k}P,
          tikz={
          \node[label=below:\tit{ër/ën},
          draw,circle,
          scale=0.85,
          fit to=tree]{};
          }
              [\tsc{k}]
              [ϕP
                  [\phantom{xxx}, roof, baseline]
              ]
          ]
      ]
      \end{forest}
      & \phantom{x} &
      \begin{forest} boom
        [\tsc{rel}P, s sep = 17.5 mm
            [\tsc{rel}P,
            tikz={
            \node[label=below:\tit{d},
            draw,circle,
            scale=0.85,
            fit to=tree]{};
            }
                [\tsc{rel}]
                [DP
                    [\phantom{xxx}, roof, baseline]
                ]
            ]
            [\tsc{k}P,
            tikz={
            \node[label=below:\tit{ër/ën},
            draw,circle,
            scale=0.75,
            fit to=tree]{};
            }
                [\tsc{k}]
                [ϕP
                    [\phantom{xxx}, roof, baseline]
                ]
            ]
        ]
      \end{forest}\\
      \bottomrule
  \end{tabular}
\end{adjustbox}
   \caption {\tsc{lh}-2 and \tsc{rp} in Old High German}
  \label{fig:rel-lh-ohg-2}
\end{figure}

Consider the first possible light head in Figure \ref{fig:rel-lh-ohg-2}.
Light heads (i.e. the phi and case features and the feature D) in Old High German are spelled out by two morphemes, which are both circled. The feature D is spelled out as \tit{d} and the phi and case features are spelled out as \tit{ër} or \tit{ën}, depending on which case they realize.
Consider the relative pronoun in Figure \ref{fig:rel-lh-ohg-2}.
Relative pronouns in Old High German consist of two morphemes: the constituent that spells out phi and case features and the constituent that spells out the feature D and the feature \tsc{rel}, again indicated by the circles. The constituent that spells out phi and case features has the same spellout as in the light head (\tit{ër} or \tit{ën}), and the \tsc{rel}P is spelled out as the XP in the light head: as \tit{d}.
Throughout this chapter, I discuss the exact feature content of light heads and relative pronouns, I give lexical entries for them, and I show how these lexical entries lead to the internal syntax of light heads and relative pronoun as shown in Figure \ref{fig:rel-lh-ohg-2}.


The chapter is structured as follows.
First, I discuss the relative pronoun. I decompose it into the two morphemes I showed in Figure \ref{fig:rel-lh-ohg-1} and Figure \ref{fig:rel-lh-ohg-1}. Then I show which features each of the morphemes corresponds to.
Then I discuss the two possible light heads. The first possible light head is one that does not surface as a light head in Old High German light-headed relatives, just as I argued for for Modern German and Polish. I show that the light head corresponds to one of the morphemes of the relative pronoun (the \tsc{k}P in Figure \ref{fig:rel-lh-ohg-1}).
The features that form the Old High German light head and relative pronoun are largely the same ones that form the Modern German light head and relative pronoun. The only difference in features is the the \tsc{wh} operator feature from Modern German and Polish relative pronouns is replaced by the feature D in Old High German. The second difference between the two languages is how the features are spelled out.
The second possible head does surface as a light head in a Old High German light-headed relatives. This light head corresponds the \tsc{k}P in the relative pronoun plus the additional feature D (see Figure \ref{fig:rel-lh-ohg-2}). The feature D is the only feature that is different in Old High German light heads compared to light heads in Modern German and Polish.
Next, I compare the internal syntax of the extra light head and the light head to that of the relative pronoun. I show that the first possible light head can be deleted when the internal case and external case match and when the internal case is more complex than the external case via structural containment. The second possible light head can be deleted when the internal case and external case match and when the internal case is more complex than the external case via formal containment. In order to illustrate how this works, I need to make a few assumptions about the larger syntactic structure of headless relative clauses explicit.
Finally, I return to the two different light heads and discuss differences in interpretation between the different sources of the headless relatives.
I also discuss the larger syntactic structure of headless relatives in a bit more detail and I show that this also holds for Modern German and Polish.


\section{The Old High German German relative pronoun}\label{sec:ohg-rel}

In the introduction of this chapter, I suggested in Figure \ref{fig:rel-lh-ohg-1} that the internal syntax of relative pronouns in Old High German looks as shown in \ref{ex:simple-unres-rp-1}.

\ex.\label{ex:simple-unres-rp-1}
\begin{forest} boom
  [\tsc{rel}P, s sep=15mm
      [\tsc{rel}P,
      tikz={
      \node[label=below:\tit{d},
      draw,circle,
      scale=0.75,
      fit to=tree]{};
      }
          [\phantom{xxx}, roof]
      ]
      [\tsc{k}P,
      tikz={
      \node[label=below:\tit{ër/ën},
      draw,circle,
      scale=0.75,
      fit to=tree]{};
      }
          [\tsc{k}]
          [ϕP
              [\phantom{xxx}, roof]
          ]
      ]
  ]
\end{forest}

In Figure \ref{fig:rel-lh-ohg-1}, I suggested that the internal syntax of relative pronouns in Old High German looks as shown in \ref{ex:simple-unres-rp-2}.

\ex.\label{ex:simple-unres-rp-2}
\begin{forest} boom
  [\tsc{rel}P, s sep = 17.5 mm
      [\tsc{rel}P,
      tikz={
      \node[label=below:\tit{d},
      draw,circle,
      scale=0.85,
      fit to=tree]{};
      }
          [\tsc{rel}]
          [DP
              [\phantom{xxx}, roof, baseline]
          ]
      ]
      [\tsc{k}P,
      tikz={
      \node[label=below:\tit{ër/ën},
      draw,circle,
      scale=0.75,
      fit to=tree]{};
      }
          [\tsc{k}]
          [ϕP
              [\phantom{xxx}, roof]
          ]
      ]
  ]
\end{forest}

In this chapter, I show that both structures show the internal syntax of Old High German relative pronouns. The structure in \ref{ex:simple-unres-rp-2} is just a bit more detailed version of \ref{ex:simple-unres-rp-1}.
As I also showed in Chapter \ref{ch:deriving-onlyinternal} for Modern German and in Chapter \ref{ch:deriving-matching} for Polish, relative pronouns contain more features than only \tsc{rel}, ϕ and \tsc{k}.
In this section, I show that Old High German relative pronouns consist of the same features, except for that the operator feature \tsc{wh} is replaced by the feature D.
Still, the crucial claim I made in Chapter \ref{ch:the-basic-idea} remains unchanged: unrestricted languages (of which Old High German is an example) have a portmanteau for the features that correspond to phi and case features and a morpheme that spells out the features the first light head does not contain.
I show the complete structure that I work towards in this section in \ref{ex:ohg-rp}.

\ex.\label{ex:ohg-rp}
\begin{adjustbox}{max width=0.9\textwidth}
\begin{forest} boom
  [\tsc{rel}P, s sep=40mm
      [\tsc{rel}P,
      tikz={
      \node[label=below:\tit{d},
      draw,circle,
      scale=0.95,
      fit to=tree]{};
      }
          [\tsc{rel}]
          [DP
              [D]
              [\tsc{ind}P
                  [\tsc{ind}]
                  [\tsc{an}]
              ]
          ]
      ]
      [\tsc{k}P,
      tikz={
      \node[label=below:\tit{ër/ën},
      draw,circle,
      scale=0.95,
      fit to=tree]{};
      }
          [\tsc{k}]
          [\tsc{ind}P
              [\tsc{ind}]
              [\tsc{an}P
                  [\tsc{an}]
                  [\tsc{cl}P
                      [\tsc{cl}]
                      [\tsc{ref}]
                  ]
              ]
          ]
      ]
  ]
\end{forest}
\end{adjustbox}

I discuss two relative pronouns: the animate nominative and the animate accusative. These are the two forms that I compare the internal syntax of in Section \ref{sec:comparing-ohg}. I show them in \ref{ex:ohg-rels}.

\ex.\label{ex:ohg-rels}
\ag. d-ër\\
 `\tsc{rp}.\tsc{an}.\tsc{nom}'\\
\bg. d-ën\\
 `\tsc{rp}.\tsc{an}.\tsc{acc}'\\

I decompose the relative pronouns into two morphemes: the \tit{d} and the final consonant (\tit{ër} or \tit{ën}). For each morpheme, I discuss which features they spell out, I give their lexical entries, and I show how I construct the relative pronouns by combining the separate morphemes.

I start with the suffixes: \tit{ër} and \tit{ën}.
These two morphemes correspond to what I called the phi and case feature portmanteau in Chapter \ref{ch:the-basic-idea} and the introduction to this chapter.
I argue that the phi features actually correspond to gender features, number features and pronominal features. Adding this all up, I claim that the final consonants correspond to number features, gender features, pronominal features and case features. Consider Table \ref{tbl:rel-dem-ohg}, which shows Old High German relative pronouns in two numbers, three genders and three cases.\footnote{
\tit{d} can also be written as \tit{dh} and \tit{th}, \tit{ë} and \tit{ē} can also be \tit{e} and \tit{é} \pgcitep{braune2018}{339}.
}

\begin{table}[htbp]
 \center
 \caption {Relative pronouns in Old High German \pgcitep{braune2018}{339}}
  \begin{tabular}{cccc}
  \toprule
            & \ac{n}.\ac{sg}  & \ac{m}.\ac{sg}      & \tsc{f}.\ac{sg}    \\
        \cmidrule{2-4}
  \ac{nom}  & d-aȥ            & d-ër                & d-iu               \\
  \ac{acc}  & d-aȥ            & d-ën                & d-ea/d-ia         \\
  \ac{dat}  & d-ëmu/d-ëmo     & d-ëmu/d-ëmo         & d-ëru/d-ëro       \\
  \bottomrule
            & \ac{n}.\ac{pl}  & \ac{m}.\ac{pl}      &  \tsc{f}.\ac{pl}  \\
        \cmidrule{2-4}
  \ac{nom}  & d-iu            &  d-ē/d-ea/d-ia/d-ie & d-eo/-io         \\
  \ac{acc}  & d-iu            &  d-ē/d-ea/d-ia/d-ie & d-eo/-io         \\
  \ac{dat}  & d-ēm/d-ēn       &  d-ēm/d-ēn          & d-ēm/d-ēn        \\
    \bottomrule
  \end{tabular}
  \label{tbl:rel-dem-ohg}
\end{table}

The suffixes in Table \ref{tbl:rel-dem-ohg} change depending on number, gender and case.
These different suffixes can be observed in several contexts besides relative pronouns. Table \ref{tbl:adj-ohg} gives an overview of the adjective \tit{jung} `young' in Old High German.

\begin{table}[htbp]
 \center
 \caption {Adjectives on \tit{-a-/-ō-} in Old High German \pgcitealt{braune2018}{300}}
  \begin{tabular}{cccc}
  \toprule
            & \ac{n}.\ac{sg}    & \ac{m}.\ac{sg}      & \tsc{f}.\ac{sg}    \\
    \cmidrule{2-4}
  \ac{nom}  & jung, jung-aȥ     & jung, jung-ēr       & jung, jung-iu     \\
  \ac{acc}  & jung, jung-aȥ     & jung-an             & jung-a            \\
  \ac{dat}  & jung-emu/jung-emo & jung-emu/jung-emo   & jung-eru/jung-ero \\
  \bottomrule
            & \ac{n}.\ac{pl}    & \ac{m}.\ac{pl}      &  \tsc{f}.\ac{pl}   \\
      \cmidrule{2-4}
  \ac{nom}  & jung-iu           &  jung-e             & jung-o            \\
  \ac{acc}  & jung-iu           &  jung-e             & jung-o            \\
  \ac{dat}  & jung-ēm/jung-ēn   &  jung-ēm/jung-ēn    & jung-ēm/jung-ēn   \\
    \bottomrule
  \end{tabular}
  \label{tbl:adj-ohg}
\end{table}

For some forms, the table gives two different forms, the first one being nominal inflection and the second one being pronominal inflection \citep{braune2018}.
The pronominal endings are the same as can be observed in the Table \ref{tbl:rel-dem-ohg}.
Note here that the situation in Old High German is slightly from the one in Modern German, in which only the final consonant expresses gender, number and case features.

Besides gender, number and case features, I assume that the suffix also contains pronominal features. I do so not only because the suffix is called pronominal inflection (\tit{Pronominalflexion}) in the literature \pgcitep{braune2018}{338}, but also because it appears in other pronominal forms too, such as possessives \pgcitep{braune2018}{337-338}, demonstratives with the \tit{dës}-stem \pgcitep{braune2018}{342} interrogatives \pgcitep{braune2018}{345}.

I give the lexical entries for \tit{ër} and \tit{ën} in \ref{ex:ohg-entry-ër} and \ref{ex:ohg-entry-ën}.
The \tit{ër} is the nominative masculine singular, so it spells out the features \tsc{ref}, \tsc{cl}, \tsc{an}, \tsc{ind} and \ac{f}1. The \tit{ën} is the accusative masculine singular, so it spells out the features that the \tit{ën} spells out plus \ac{f}2.

\ex.\label{ex:ohg-entries-ër-ën}
\a.\label{ex:ohg-entry-ër}
\begin{forest} boom
  [\tsc{nom}P
      [\ac{f}1]
      [\tsc{ind}P
          [\tsc{ind}]
          [\tsc{an}P
              [\tsc{an}]
              [\tsc{cl}P
                  [\tsc{cl}]
                  [\tsc{ref}]
              ]
          ]
      ]
  ]
  {\draw (.east) node[right]{⇔ \tit{ër}}; }
\end{forest}
\b.\label{ex:ohg-entry-ën}
\begin{forest} boom
  [\tsc{acc}P
      [\ac{f}2]
      [\tsc{nom}P
          [\ac{f}1]
          [\tsc{ind}P
              [\tsc{ind}]
              [\tsc{an}P
                  [\tsc{an}]
                  [\tsc{cl}P
                      [\tsc{cl}]
                      [\tsc{ref}]
                  ]
              ]
          ]
      ]
  ]
  {\draw (.east) node[right]{⇔ \tit{ën}}; }
\end{forest}

I continue with the morpheme \tit{d}. This morpheme corresponds to what I called the \tsc{rel}-feature in Chapter \ref{ch:the-basic-idea} and in the introduction to this chapter. I argue that this morpheme actually spells out the feature \tsc{rel}, the feature D and number and gender features. Notice here that Old High German relative pronouns differ from Modern German and Polish relative pronouns in that they do not contain the operator feature \tsc{wh}. Instead, Old High German relative pronouns contain the feature D.

Relative and demonstrative pronouns are syncretic in Old High German \pgcitep{braune2018}{338}. They contain the morpheme \tit{d}, which is responsible for establishing a definite reference. The feature \tsc{rel} is present to establish a relation.
I assume that \tit{d} also contains the features \tsc{ind}. For this I do not have any independent support. I make this assumption to allow myself to build a complex specifier.

In sum, the morpheme \tit{d} corresponds to the features D, \tsc{rel} and \tsc{ind} as shown in \ref{ex:ohg-entry-d}.

\ex. \begin{forest} boom
  [\tsc{rel}P
      [\tsc{rel}]
      [DP
          [D]
          [\tsc{ind}]
      ]
  ]
  {\draw (.east) node[right]{⇔ \tit{d}}; }
\end{forest}\label{ex:ohg-entry-d}

In what follows, I show how the Old High German relative pronouns are constructed. I follow the same functional sequence as I did for Modern German and Polish, except for substituting the feature \tsc{wh} by the feature D. I give the functional sequence in \ref{ex:fseq-d-rel}.

\ex.\label{ex:fseq-d-rel}
\begin{adjustbox}{max width=0.9\textwidth}
\begin{forest} boom
   [\tsc{k}P
       [\tsc{k}]
       [\tsc{rel}P
           [\tsc{rel}]
           [\tsc{wh}P
               [\tsc{wh}]
               [\tsc{ind}P
                   [\tsc{ind}]
                   [\tsc{an}P
                       [\tsc{an}]
                       [\tsc{cl}P
                           [\tsc{cl}]
                           [\tsc{ref}]
                       ]
                   ]
               ]
           ]
       ]
   ]
\end{forest}
\end{adjustbox}

Of course, the spellout procedure is identical. The outcome is different because of the different lexical entries Old High German has. I repeat the available lexical entries in \ref{ex:ohg-entries-all-rp}.

\ex.\label{ex:ohg-entries-all-rp}
\a.\label{ex:ohg-entry-ër-rep-rp}
\begin{forest} boom
  [\tsc{nom}P
      [\ac{f}1]
      [\tsc{ind}P
          [\tsc{ind}]
          [\tsc{an}P
              [\tsc{an}]
              [\tsc{cl}P
                  [\tsc{cl}]
                  [\tsc{ref}]
              ]
          ]
      ]
  ]
  {\draw (.east) node[right]{⇔ \tit{ër}}; }
\end{forest}
\b.\label{ex:ohg-entry-ën-rep-rp}
\begin{forest} boom
  [\tsc{acc}P
      [\ac{f}2]
      [\tsc{nom}P
          [\ac{f}1]
          [\tsc{ind}P
              [\tsc{ind}]
              [\tsc{an}P
                  [\tsc{an}]
                  [\tsc{cl}P
                      [\tsc{cl}]
                      [\tsc{ref}]
                  ]
              ]
          ]
      ]
  ]
  {\draw (.east) node[right]{⇔ \tit{ën}}; }
\end{forest}
\b.\label{ex:ohg-entry-d-rep-rp}
\begin{forest} boom
  [\tsc{rel}P
      [\tsc{rel}]
      [DP
          [D]
          [\tsc{ind}]
      ]
  ]
  {\draw (.east) node[right]{⇔ \tit{d}}; }
\end{forest}

Starting from the bottom, the first two features that are merged are \tsc{ref} and \tsc{cl}, creating a \tsc{cl}P.
The syntactic structure forms a constituent in the lexical tree in \ref{ex:ohg-entry-ër-rep-rp}, which corresponds to \tit{r}.
Therefore, the \tsc{cl}P is spelled out as \tit{ër}, which I do not show here.
Then, the feature \tsc{an} is merged, and a \tsc{an}P is created.
The syntactic structure forms a constituent in the lexical tree in \ref{ex:ohg-entry-ër-rep-rp}.
Therefore, the \tsc{an}P is spelled out as \tit{ër}, which I do not show here either.
Then, the feature \tsc{ind} is merged, and a \tsc{ind}P is created.
The syntactic structure forms a constituent in the lexical tree in \ref{ex:ohg-entry-ër-rep-rp}.
Therefore, the \tsc{ind}P is spelled out as \tit{ër}, which I show in

\ex.\label{ex:ohg-spellout-ër-ind}
\begin{forest} boom
  [\tsc{ind}P,
  tikz={
  \node[label=below:\tit{ër},
  draw,circle,
  scale=0.95,
  fit to=tree]{};
  }
      [\tsc{ind}]
      [\tsc{an}P
          [\tsc{an}]
          [\tsc{cl}P
              [\tsc{cl}]
               [\tsc{ref}]
          ]
      ]
  ]
\end{forest}

Next, the feature D is merged.
The derivation for this feature resembles the derivation of \tsc{wh} in Modern German and Polish.
The feature is merged with the existing syntactic structure, creating a DP.
This structure does not form a constituent in any of the lexical trees in the language's lexicon, and neither of the spellout driven movements leads to a successful spellout.
Therefore, in a second workspace, the feature D is merged with the feature \tsc{ind} (the previous syntactic feature on the functional sequence) into a DP. This syntactic structure forms a constituent in the lexical tree in \ref{ex:ohg-entry-d-rep-rp}, which corresponds to the \tit{d}.
Therefore, the DP is spelled out as \tit{d}. The newly created phrase is merged as a whole with the already existing structure, and projects to the top node, as shown in \ref{ex:ohg-spellout-dp}.

\ex.\label{ex:ohg-spellout-dp}
\begin{forest} boom
  [DP, s sep=15mm
      [DP,
      tikz={
      \node[label=below:\tit{d},
      draw,circle,
      scale=0.95,
      fit to=tree]{};
      }
          [D]
          [\tsc{ind}]
      ]
      [\tsc{ind}P,
      tikz={
      \node[label=below:\tit{ër},
      draw,circle,
      scale=0.95,
      fit to=tree]{};
      }
          [\tsc{ind}]
          [\tsc{an}P
              [\tsc{an}]
              [\tsc{cl}P
                  [\tsc{cl}]
                  [\tsc{ref}]
              ]
          ]
      ]
  ]
\end{forest}

The next feature in the functional sequence is the feature \tsc{rel}. The derivation for this feature resembles the derivation of \tsc{rel} in Modern German and Polish.
The feature is merged with the existing syntactic structure, creating a \tsc{rel}P.
This structure does not form a constituent in any of the lexical trees in the language's lexicon, and neither of the spellout driven movements leads to a successful spellout.
Backtracking leads to splitting up the DP from the \tsc{ind}P.
The feature \tsc{rel} is merged in both workspaces, so with DP and and with \tsc{ind}P. The spellout of \tsc{rel} is successful when it is combined with the DP.
It namely forms a constituent in the lexical tree in \ref{ex:ohg-entry-d-rep-rp}, which corresponds to the \tit{d}.
The \tsc{rel}P is spelled out as \tit{d}, and it is merged back to the existing syntactic structure.

For the nominative relative pronoun, the last feature is merged: the \tsc{f}1. This feature should somehow end up merging with \tsc{ind}P, because it forms a constituent in the lexical tree in \ref{ex:ohg-entry-ër-rep-rp}, which corresponds to the \tit{ër}.
This is achieved via Backtracking in which phrases are split up and going through the Spellout Algorithm. I go through the derivation step by step.
The feature \tsc{f}1 is merged with the existing syntactic structure, creating a \tsc{nom}P.
This structure does not form a constituent in any of the lexical trees in the language's lexicon, and neither of the spellout driven movements leads to a successful spellout.
Backtracking leads to splitting up the \tsc{rel}P from the \tsc{ind}P.
The feature \tsc{f}1 is merged in both workspaces, so with the \tsc{rel}P and and with the \tsc{ind}P. The spellout of \tsc{f}1 is successful when it is combined with the \tsc{ind}P.
It namely forms a constituent in the lexical tree in \ref{ex:ohg-entry-ër-rep-rp}, which corresponds to the \tit{ër}.
The \tsc{nom}P is spelled out as \tit{ër}, and all constituents are merged back into the existing syntactic structure, as shown in \ref{ex:ohg-spellout-rel-nom}.

\ex.\label{ex:ohg-spellout-rel-nom}
\begin{adjustbox}{max width=0.9\textwidth}
\begin{forest} boom
      [\tsc{rel}P, s sep=30mm
          [\tsc{rel}P,
          tikz={
          \node[label=below:\tit{d},
          draw,circle,
          scale=0.95,
          fit to=tree]{};
          }
              [\tsc{rel}]
              [DP
                  [D]
                  [\tsc{ind}]
              ]
          ]
          [\tsc{nom}P,
          tikz={
          \node[label=below:\tit{ër},
          draw,circle,
          scale=0.95,
          fit to=tree]{};
          }
              [\ac{f}1]
              [\tsc{ind}P
                  [\tsc{ind}]
                  [\tsc{an}P
                      [\tsc{an}]
                      [\tsc{cl}P
                          [\tsc{cl}]
                          [\tsc{ref}]
                      ]
                  ]
              ]
          ]
      ]
  ]
\end{forest}
\end{adjustbox}

For the accusative relative pronoun, the last feature is merged: the \tsc{f}2. The derivation for \tsc{f}2 resembles the derivation of \tsc{f}1. The feature is merged with the existing syntactic structure, creating a \tsc{acc}P.
This structure does not form a constituent in any of the lexical trees in the language's lexicon, and neither of the spellout driven movements leads to a successful spellout.
Backtracking leads to splitting up the \tsc{rel}P from the \tsc{nom}P.
The feature \tsc{f}2 is merged in both workspaces, so with the \tsc{rel}P and and with the \tsc{nom}P. The spellout of \tsc{f}2 is successful when it is combined with the \tsc{nom}P.
It namely forms a constituent in the lexical tree in \ref{ex:ohg-entry-ën-rep-rp}, which corresponds to the \tit{ën}. The \tsc{acc}P is spelled out as \tit{ën}, and all constituents are merged back into the existing syntactic structure, as shown in \ref{ex:ohg-spellout-rel-acc}.

\ex.\label{ex:ohg-spellout-rel-acc}
\begin{adjustbox}{max width=0.9\textwidth}
\begin{forest} boom
      [\tsc{rel}P, s sep=30mm
          [\tsc{rel}P,
          tikz={
          \node[label=below:\tit{d},
          draw,circle,
          scale=0.95,
          fit to=tree]{};
          }
              [\tsc{rel}]
              [DP
                  [D]
                  [\tsc{ind}]
              ]
          ]
          [\tsc{acc}P,
          tikz={
          \node[label=below:\tit{ën},
          draw,circle,
          scale=0.95,
          fit to=tree]{};
          }
              [\ac{f}2]
              [\tsc{nom}P
                  [\ac{f}1]
                  [\tsc{ind}P
                      [\tsc{ind}]
                      [\tsc{an}P
                          [\tsc{an}]
                          [\tsc{cl}P
                              [\tsc{cl}]
                              [\tsc{ref}]
                          ]
                      ]
                  ]
              ]
          ]
      ]
  ]
\end{forest}
\end{adjustbox}

To summarize, I decomposed the relative pronoun into the two morphemes: \tit{d} and the suffix (\tit{ër} and \tit{ën}). I showed which features each of the morphemes spells out and what the internal syntax looks like that they are combined into. It is this internal syntax that determines whether the light head can be deleted or not.


\section{The Old High German light heads}

I have suggested that headless relatives are derived from light-headed relatives. The light head or the relative pronoun can be deleted when either of them is a constituent that is contained within the other one. In the introduction of this chapter, I claimed that Old High German has two possible light heads. Therefore I will also claim that there are two types of light-headed relatives that are the source of the headless relative.

For Modern German and Polish, I considered two kinds of headless relatives as the potential source of the headless relative.
The first possible scenario is that the deletion is optional and that the headless relative is derived from an existing light headed relative.
The second possible scenario is that the deletion of the light head is obligatory and that the headless relative is derived from a light-headed relative that does not surface.
I concluded for Modern German and Polish that the second scenario is the one that is attested in the languages.
For Old High German I assume that headless relatives can be derived from both types of light-headed relatives.

In Section \ref{sec:ohg-elh}, I introduce the extra light head as the first possible light head. In Section \ref{sec:ohg-lh}, I introduce the light head as the second possible light head.


\subsection{The extra light head}\label{sec:ohg-elh}

In the introduction of this chapter, I claimed that the internal syntax of the first possible light head is as shown in \ref{ex:ohg-lh-complex-1}.

\ex.\label{ex:ohg-lh-complex-1}
\begin{forest} boom
  [\tsc{k}P,
  tikz={
  \node[label=below:\tit{ër/ën},
  draw,circle,
  scale=0.95,
  fit to=tree]{};
  }
      [\tsc{k}]
      [ϕP
          [\phantom{xxx}, roof]
      ]
  ]
\end{forest}

In Chapter \ref{ch:the-basic-idea}, I suggested that the first possible light head in the unrestricted type of language consist of at least two features: ϕ and \tsc{k}.
In this section, I determine the exact feature content of the light head.
Like I suggested in Chapter \ref{ch:deriving-onlyinternal} for Modern German and Polish, I end up claiming that the phi and case features of the relative pronoun is the light head in headless relatives. I show the complete structure that I work towards in this section in \ref{ex:pol-elh}.

\ex.
\begin{forest} boom
  [\tsc{k}P,
  tikz={
  \node[label=below:\tit{ër/ën},
  draw,circle,
  scale=0.95,
  fit to=tree]{};
  }
      [\tsc{k}]
      [\tsc{ind}P
          [\tsc{ind}]
          [\tsc{an}P
              [\tsc{an}]
              [\tsc{cl}P
                  [\tsc{cl}]
                  [\tsc{ref}]
              ]
          ]
      ]
  ]
\end{forest}

As I mentioned in the introduction of this section, headless relatives in Old High German can be derived from two different light-headed relative constructions: one that surfaces in the language and one that does not surface in the language.
In this section I discuss the second one, the light-headed relative that does not surface in the language. This light-headed relative has the extra light head as light head, just like the ones that are attested in Modern German and Polish.\footnote{
In the sections on extra light heads in Modern German and Polish I discussed the possible interpretations of headless relatives in these languages. In this section I do not do that for For Old High German, as I do not have this information for the extinct language.
In Section \ref{sec:coming-back} I briefly touch upon different interpretations of headless relatives, based on contexts in which they appear. I relate the different interpretations to the different light-headed relatives that the headless relatives are derived from.
}

In the remainder of this section, I discuss the two extra light heads that I compare the internal syntax of in Section \ref{sec:comparing-ohg}. The are the nominative masculine and the accusative masculine, shown in \ref{ex:ohg-elhs}.

\ex.\label{ex:ohg-elhs}
\ag. ër\\
 \tsc{elh}.\tsc{m}.\tsc{nom}\\
\bg. ën\\
 \tsc{elh}.\tsc{m}.\tsc{acc}\\

 Just as in Modern German and Polish, the functional sequence for the extra light head is as shown in \ref{ex:fseq-elh-rep-ohg}.

 \ex.\label{ex:fseq-elh-rep-ohg}
 \begin{forest} boom
   [\tsc{k}P
       [\tsc{k}]
       [\tsc{ind}P
           [\tsc{ind}]
           [\tsc{an}P
               [\tsc{an}]
               [\tsc{cl}P
                   [\tsc{cl}]
                   [\tsc{ref}]
               ]
           ]
       ]
   ]
 \end{forest}

 The functional sequence contains the pronominal feature \tsc{ref}, the gender features \tsc{cl} and \tsc{an}, the number feature \tsc{ind} and case features \tsc{k}.

 I introduced the lexical entries that are required to spell out these features in Section \ref{sec:ohg-rel}. I repeat them in \ref{ex:ohg-entries-rep-elh}.

 \ex.\label{ex:ohg-entries-rep-elh}
 \a.\label{ex:ohg-entry-ër-rep-elh}
 \begin{forest} boom
   [\tsc{nom}P
       [\ac{f}1]
       [\tsc{ind}P
           [\tsc{ind}]
           [\tsc{an}P
               [\tsc{an}]
               [\tsc{cl}P
                   [\tsc{cl}]
                   [\tsc{ref}]
               ]
           ]
       ]
   ]
   {\draw (.east) node[right]{⇔ \tit{ër}}; }
 \end{forest}
\b.\label{ex:ohg-entry-ën-rep-elh}
 \begin{forest} boom
   [\tsc{acc}P
       [\ac{f}2]
       [\tsc{nom}P
           [\ac{f}1]
           [\tsc{ind}P
               [\tsc{ind}]
               [\tsc{an}P
                   [\tsc{an}]
                   [\tsc{cl}P
                       [\tsc{cl}]
                       [\tsc{ref}]
                   ]
               ]
           ]
       ]
   ]
   {\draw (.east) node[right]{⇔ \tit{ën}}; }
 \end{forest}

In what follows, I construct the Old High German extra light heads. Until the feature \tsc{ind}, the derivation is identical to the one of the relative pronoun. I give the syntactic structure at that point in \ref{ex:ohg-spellout-ër-ind-rep}.

\ex.\label{ex:ohg-spellout-ër-ind-rep}
\begin{forest} boom
  [\tsc{ind}P,
  tikz={
  \node[label=below:\tit{ër},
  draw,circle,
  scale=0.9,
  fit to=tree]{};
  }
      [\tsc{ind}]
      [\tsc{an}P
          [\tsc{an}]
          [\tsc{cl}P
              [\tsc{cl}]
               [\tsc{ref}]
          ]
      ]
  ]
\end{forest}

The last feature that is merged for the nominative extra light head is the \tsc{f}1.
It is merged, and the \tsc{nom}P is created.
The syntactic structure forms a constituent in the lexical tree in \ref{ex:ohg-entry-ër-rep-elh}.
Therefore, the \tsc{nom}P is spelled out as \tit{ër}, as shown in \ref{ex:ohg-elh-nom}.

\ex.\label{ex:ohg-elh-nom}
\begin{forest} boom
  [\tsc{nom}P,
  tikz={
  \node[label=below:\tit{ër},
  draw,circle,
  scale=0.95,
  fit to=tree]{};
  }
      [\tsc{f}1]
      [\tsc{ind}P
          [\tsc{ind}]
          [\tsc{an}P
              [\tsc{an}]
              [\tsc{cl}P
                  [\tsc{cl}]
                   [\tsc{ref}]
              ]
          ]
      ]
  ]
\end{forest}

For the accusative extra light head, one more feature is merged: the \tsc{f}2.
It is merged, and the \tsc{acc}P is created.
The syntactic structure forms a constituent in the lexical tree in \ref{ex:ohg-entry-ën-rep-elh}.
Therefore, the \tsc{acc}P is spelled out as \tit{ën}, as shown in \ref{ex:ohg-elh-acc}.

\ex.\label{ex:ohg-elh-acc}
\begin{forest} boom
  [\tsc{acc}P,
  tikz={
  \node[label=below:\tit{ën},
  draw,circle,
  scale=0.95,
  fit to=tree]{};
  }
      [\tsc{f}2]
      [\tsc{nom}P
          [\tsc{f}1]
          [\tsc{ind}P
              [\tsc{ind}]
              [\tsc{an}P
                  [\tsc{an}]
                  [\tsc{cl}P
                      [\tsc{cl}]
                       [\tsc{ref}]
                  ]
              ]
          ]
      ]
  ]
\end{forest}

In sum, Old High German headless relatives can be derived from a light-headed relative with an extra light head, just like in Modern German and Polish. This extra light head is spelled out by a single phi and case feature portmanteau. The lexical entries used to spell this light head out are also used to spell out part of the internal syntax of the relative pronoun.



\subsection{The light head}\label{sec:ohg-lh}

In the introduction of this chapter, I claimed that the internal syntax of the second possible light head is as shown in \ref{ex:ohg-lh-complex-2}.

\ex.\label{ex:ohg-lh-complex-2}
\begin{forest} boom
  [DP, s sep = 17.5 mm
      [DP,
      tikz={
      \node[label=below:\tit{d},
      draw,circle,
      scale=0.85,
      fit to=tree]{};
      }
          [\phantom{xxx}, roof]
      ]
      [\tsc{k}P,
      tikz={
      \node[label=below:\tit{ër/ën},
      draw,circle,
      scale=0.75,
      fit to=tree]{};
      }
          [\tsc{k}]
          [ϕP
              [\phantom{xxx}, roof]
          ]
      ]
  ]
\end{forest}

In Chapter \ref{ch:the-basic-idea}, I suggested that the second possible light head in the unrestricted type of language consist of at least three features: D, ϕ and \tsc{k}.
In this section, I determine the exact feature content of the light head.

Like I suggested in Chapter \ref{ch:the-basic-idea}, I end up claiming that the feature D and the phi and case features of the relative pronoun is the light head in headless relatives. I show the complete structure that I work towards in this section in \ref{ex:ohg-lh}.

\ex.\label{ex:ohg-lh}
\begin{forest} boom
  [DP, s sep = 17.5 mm
      [DP,
          tikz={
          \node[label=below:\tit{d},
          draw,circle,
          scale=0.85,
          fit to=tree]{};
          }
          [\phantom{xxx}, roof, baseline]
      ]
      [\tsc{k}P,
      tikz={
      \node[label=below:\tit{ër/ën},
      draw,circle,
      scale=0.75,
      fit to=tree]{};
      }
          [\tsc{k}]
          [ϕP
              [\phantom{xxx}, roof, baseline]
          ]
      ]
  ]
\end{forest}

As I mentioned in the introduction of this section, headless relatives in Old High German can be derived from two different light-headed relative constructions: one that surfaces in the language and one that does not surface in the language.
In this section I discuss the first one, the light-headed relative that also surfaces in the language. This light-headed relative has the light head that functions as the light head. This light-headed relative is never the source of a headless relative in Modern German or Polish.


I give an example of a Old High German light-headed relative in \ref{ex:ohg-lh-rel}
\tit{Thér} `\tsc{dem}.\tsc{sg}.\tsc{m}.\tsc{nom}' is the light head that is the head of the relative clause.
\tit{Then} `\tsc{rp}.\tsc{sg}.\tsc{m}.\tsc{acc}' is the relative pronoun in the relative clause.\footnote{
I assume that whether both or only one of the elements surfaces is determined by information structure. In \ref{ex:ohg-lh-rel}, the light head \tit{thér} `\tsc{dem}.\tsc{sg}.\tsc{m}.\tsc{nom}' is the candidate to be absent, as it bears a less complex case than the relative pronoun. However, it seems plausible that the light head is emphasized in this sentence and that it, therefore, cannot be absent.
}

\exg. eno nist thiz thér then ir suochet zi arslahanne?\\
 now {not be.3\ac{sg}} \tsc{dem}.\tsc{sg}.\tsc{ën}.\tsc{nom} \tsc{dem}.\tsc{sg}.\tsc{m}.\tsc{nom}
 \tsc{rp}.\tsc{sg}.\tsc{m}.\tsc{acc} 2\ac{pl}.\tsc{nom} seek.2\tsc{pl} to kill.\tsc{inf}.\ac{sg}.\tsc{dat}\\
 `Isn't this now the one, who you seek to kill?'\label{ex:ohg-lh-rel}

As \ref{ex:ohg-lh-rel} shows and mentioned earlier in this chapter, relative pronouns and demonstrative pronouns are syncretic in Old High German. Both of them start with a \tit{d}, followed by a phi and case feature morpheme. Crucially, this syncretism leads Old High German to be an unrestricted type of language. In Modern German and in Polish, relative pronouns and demonstratives are not syncretic.\footnote{
An exception is..

First, German only had the d-pronoun and attraction. The pattern of attraction that came with that pronoun is ext only.
At some point, German invented the wh-pronoun. Helmut showed how it emerged. With that came the other pattern: int only. Some people lost the attraction (but everybody kept the d-pronoun) and with that the pattern disappeared.
So the patterns in headless relatives follow from the relative pronouns in the language.

Why are all languages of the `matching' type dead languages?
Was it a common thing that wh-pronouns were not used as relative pronouns?

Wouldn't we now not expect that Modern German patterns with Old High German wrt attraction in headed constructions. Yes, we would. And yes, this is exactly what we see. Paper by Bader on case attraction.

First there was only the relative pronoun with a D. Then we did case competition with this one, in both directions. Later, we only did it with the wh, and we only had internal left. Because this competitor was introduced, the case competition with D disappeared.
} Therefore, the relative pronoun cannot be deleted via formal containment.

In the remainder of this section, I discuss the two light heads that I compare the internal syntax of in Section \ref{sec:comparing-ohg}. The are the nominative masculine and the accusative masculine, shown in \ref{ex:ohg-lhs}.

\ex.\label{ex:ohg-lhs}
\ag. d-ër\\
 \tsc{lh}.\tsc{m}.\tsc{nom}\\
\bg. d-ën\\
 \tsc{lh}.\tsc{m}.\tsc{acc}\\

The functional sequence for the light head is as shown in \ref{ex:fseq-lh-rep-ohg}.

\ex.\label{ex:fseq-lh-rep-ohg}
 \begin{forest} boom
   [\tsc{k}P
       [\tsc{k}]
       [DP
           [D]
           [\tsc{ind}P
               [\tsc{ind}]
               [\tsc{an}P
                   [\tsc{an}]
                   [\tsc{cl}P
                       [\tsc{cl}]
                       [\tsc{ref}]
                   ]
               ]
           ]
       ]
   ]
\end{forest}

The functional sequence contains the pronominal feature \tsc{ref}, the gender features \tsc{cl} and \tsc{an}, the number feature \tsc{ind}, the definite feature D and case features \tsc{k}.

I introduced the lexical entries that are required to spell out these features in Section \ref{sec:ohg-rel}. I repeat them in \ref{ex:ohg-entries-rep-lh}.

 \ex.\label{ex:ohg-entries-rep-lh}
 \a.\label{ex:ohg-entry-ër-rep-lh}
 \begin{forest} boom
   [\tsc{nom}P
       [\ac{f}1]
       [\tsc{ind}P
           [\tsc{ind}]
           [\tsc{an}P
               [\tsc{an}]
               [\tsc{cl}P
                   [\tsc{cl}]
                   [\tsc{ref}]
               ]
           ]
       ]
   ]
   {\draw (.east) node[right]{⇔ \tit{ër}}; }
 \end{forest}
\b.\label{ex:ohg-entry-ën-rep-lh}
 \begin{forest} boom
   [\tsc{acc}P
       [\ac{f}2]
       [\tsc{nom}P
           [\ac{f}1]
           [\tsc{ind}P
               [\tsc{ind}]
               [\tsc{an}P
                   [\tsc{an}]
                   [\tsc{cl}P
                       [\tsc{cl}]
                       [\tsc{ref}]
                   ]
               ]
           ]
       ]
   ]
   {\draw (.east) node[right]{⇔ \tit{ën}}; }
 \end{forest}
\b.\label{ex:ohg-entry-d-rep-lh}
 \begin{forest} boom
   [\tsc{rel}P
       [\tsc{rel}]
       [DP
           [D]
           [\tsc{ind}]
       ]
   ]
   {\draw (.east) node[right]{⇔ \tit{d}}; }
 \end{forest}

In what follows, I construct the Old High German light heads. Until the feature D, the derivation is identical to the one of the relative pronoun. I give the syntactic structure at that point in \ref{ex:ohg-spellout-dp-rep}.

\ex.\label{ex:ohg-spellout-dp-rep}
\begin{forest} boom
  [DP, s sep=15mm
      [DP,
      tikz={
      \node[label=below:\tit{d},
      draw,circle,
      scale=0.95,
      fit to=tree]{};
      }
          [D]
          [\tsc{ind}]
      ]
      [\tsc{ind}P,
      tikz={
      \node[label=below:\tit{ër},
      draw,circle,
      scale=0.95,
      fit to=tree]{};
      }
          [\tsc{ind}]
          [\tsc{an}P
              [\tsc{an}]
              [\tsc{cl}P
                  [\tsc{cl}]
                  [\tsc{ref}]
              ]
          ]
      ]
  ]
\end{forest}

For the nominative light head, the last feature is merged: the \tsc{f}1. This feature should somehow end up merging with \tsc{ind}P, because it forms a constituent in the lexical tree in \ref{ex:ohg-entry-ër-rep-lh}, which corresponds to the \tit{ër}.
This is achieved via Backtracking in which phrases are split up and going through the Spellout Algorithm. I go through the derivation step by step.
The feature \tsc{f}1 is merged with the existing syntactic structure, creating a \tsc{nom}P.
This structure does not form a constituent in any of the lexical trees in the language's lexicon, and neither of the spellout driven movements leads to a successful spellout.
Backtracking leads to splitting up the DP from the \tsc{ind}P.
The feature \tsc{f}1 is merged in both workspaces, so with the DP and and with the \tsc{ind}P. The spellout of \tsc{f}1 is successful when it is combined with the \tsc{ind}P.
It namely forms a constituent in the lexical tree in \ref{ex:ohg-entry-ër-rep-lh}, which corresponds to the \tit{ër}.
The \tsc{nom}P is spelled out as \tit{ër}, and all constituents are merged back into the existing syntactic structure, as shown in \ref{ex:ohg-spellout-dp-nom}.

\ex.\label{ex:ohg-spellout-dp-nom}
\begin{adjustbox}{max width=0.9\textwidth}
\begin{forest} boom
      [DP, s sep=20mm
          [DP,
          tikz={
          \node[label=below:\tit{d},
          draw,circle,
          scale=0.95,
          fit to=tree]{};
          }
              [D]
              [\tsc{ind}]
          ]
          [\tsc{nom}P,
          tikz={
          \node[label=below:\tit{ër},
          draw,circle,
          scale=0.95,
          fit to=tree]{};
          }
              [\ac{f}1]
              [\tsc{ind}P
                  [\tsc{ind}]
                  [\tsc{an}P
                      [\tsc{an}]
                      [\tsc{cl}P
                          [\tsc{cl}]
                          [\tsc{ref}]
                      ]
                  ]
              ]
          ]
      ]
  ]
\end{forest}
\end{adjustbox}

For the accusative light head pronoun, the last feature is merged: the \tsc{f}2. The derivation for \tsc{f}2 resembles the derivation of \tsc{f}1. The feature is merged with the existing syntactic structure, creating a \tsc{acc}P.
This structure does not form a constituent in any of the lexical trees in the language's lexicon, and neither of the spellout driven movements leads to a successful spellout.
Backtracking leads to splitting up the DP from the \tsc{nom}P.
The feature \tsc{f}2 is merged in both workspaces, so with the DP and and with the \tsc{nom}P. The spellout of \tsc{f}2 is successful when it is combined with the \tsc{nom}P.
It namely forms a constituent in the lexical tree in \ref{ex:ohg-entry-ën-rep-lh}, which corresponds to the \tit{ën}. The \tsc{acc}P is spelled out as \tit{ën}, and all constituents are merged back into the existing syntactic structure, as shown in \ref{ex:ohg-spellout-dp-acc}.

\ex.\label{ex:ohg-spellout-dp-acc}
\begin{adjustbox}{max width=0.9\textwidth}
\begin{forest} boom
      [DP, s sep=20mm
          [DP,
          tikz={
          \node[label=below:\tit{d},
          draw,circle,
          scale=0.95,
          fit to=tree]{};
          }
              [D]
              [\tsc{ind}]
          ]
          [\tsc{acc}P,
          tikz={
          \node[label=below:\tit{ën},
          draw,circle,
          scale=0.95,
          fit to=tree]{};
          }
              [\ac{f}2]
              [\tsc{nom}P
                  [\ac{f}1]
                  [\tsc{ind}P
                      [\tsc{ind}]
                      [\tsc{an}P
                          [\tsc{an}]
                          [\tsc{cl}P
                              [\tsc{cl}]
                              [\tsc{ref}]
                          ]
                      ]
                  ]
              ]
          ]
      ]
  ]
\end{forest}
\end{adjustbox}

In sum, Old High German headless relatives can be derived from a light-headed relative with a light head. This light head is spelled out by a morpheme that spells out the definite feature and a phi and case feature portmanteau. The lexical entries used to spell this light head out are also used to spell out the relative pronoun, as the light head and the relative pronoun are syncretic.


\section{Comparing light heads and relative pronouns}\label{sec:comparing-ohg}

In this section, I compare the internal syntax of extra light heads and light heads to the internal syntax of relative pronouns in Old High German. This is the worked out version of the comparisons in Section \ref{sec:basic-unrestricted}. What is different here is that I show the comparison for Old High German specifically, and that the content of the internal syntax that is being compared is motivated earlier in this chapter.

I give three examples, in which the internal and external case vary.
I start with an example with matching cases, in which the internal and the external case are both nominative. I show that the grammaticality of the example can be derived by either taking the extra light head or by taking the light head as being part of the light-headed relative that the headless relative is derived from.
Then I give an example in which the external accusative case is more complex than the internal nominative case. I show that the grammaticality of this example can only be derived by taking the light head as being part of the light-headed relative that the headless relative is derived from and not the extra light head. Before I can properly do that, I take a necessary short detour into the larger syntactic structure of headless relatives.
I end with an example in which the internal accusative case is more complex than the external nominative case. I show that the grammaticality of this example can only be derived by taking the extra light head as being part of the light-headed relative that the headless relative is derived from and not the light head.

I start with the situation in which the cases match.
Consider the example in \ref{ex:ohg-nom-nom-rep}, in which the internal nominative case competes against the external nominative case. The relative clause is marked in bold. \ref{ex:ohg-nom-nom-elh} shows the example with the extra light head and \ref{ex:ohg-nom-nom-lh} shows the example with the light head.
The internal case is nominative, as the predicate \tit{senten} `to send' takes nominative subjects.
In both examples, the relative pronoun \tit{dher} `\ac{rel}.\ac{sg}.\ac{m}.\ac{nom}' appears in the nominative case.
The external case is nominative as well, as the predicate \tit{queman} `to come' also takes nominative subjects.
In \ref{ex:ohg-nom-nom-elh}, the extra light head \tit{er} `\tsc{elh}.\ac{sg}.\ac{m}.\ac{nom}' appears in the nominative case. It is placed between square brackets because it does not surface.
In \ref{ex:ohg-nom-nom-lh}, the light head \tit{dher} `\tsc{dem}.\ac{sg}.\ac{m}.\ac{nom}' appears in the nominative case. Here the relative pronoun is placed between square brackets because it does not surface.

\ex.\label{ex:ohg-nom-nom-rep}
\ag. quham [er] \tbf{dher} \tbf{chisendit} \tbf{scolda} \tbf{uuerdhan}\\
 come.\ac{pst}.3\ac{sg}\scsub{[nom]} \ac{elh}.\ac{sg}.\ac{m}.\ac{nom} \ac{rel}.\ac{sg}.\ac{m}.\ac{nom} send.\ac{pst}.\ac{ptcp}\scsub{[nom]} should.\ac{pst}.3\ac{sg} become.\ac{inf}\\
 `the one, who should have been sent, came' \flushfill{Old High German, \ac{isid} 35:5}\label{ex:ohg-nom-nom-elh}
\bg. quham dher [\tbf{dher}] \tbf{chisendit} \tbf{scolda} \tbf{uuerdhan}\\
 come.\ac{pst}.3\ac{sg}\scsub{[nom]} \ac{dem}.\ac{sg}.\ac{m}.\ac{nom} \ac{rel}.\ac{sg}.\ac{m}.\ac{nom} send.\ac{pst}.\ac{ptcp}\scsub{[nom]} should.\ac{pst}.3\ac{sg} become.\ac{inf}\\
 `the one, who should have been sent, came' \flushfill{Old High German, \ac{isid} 35:5}\label{ex:ohg-nom-nom-lh}

Both examples in \ref{ex:ohg-nom-nom-rep} can be the source that the headless relative is derived from. First I show the comparison of the internal syntax of the extra light head and relative pronoun in \ref{ex:ohg-nom-nom-elh}. Then I show the comparison of the internal syntax of the light head and the relative pronoun in \ref{ex:ohg-nom-nom-lh}.

In Figure \ref{fig:ohg-int=ext-elh}, I give the syntactic structure of the extra light head at the top and the syntactic structure of the relative pronoun at the bottom.

\begin{figure}[htbp]
  \center
  \begin{adjustbox}{max height=0.9\textheight}
  \begin{tabular}[b]{c}
        \toprule
        \tsc{nom} extra light head \tit{er}\\
        \cmidrule{1-1}
      \begin{forest} boom
        [\tsc{nom}P,
        tikz={
        \node[label=below:{\tit{er}},
        draw,circle,
        scale=0.8,
        fit to=tree]{};
        \node[draw,circle,
        dashed,
        scale=0.85,
        fill=DG,fill opacity=0.2,
        fit to=tree]{};
        }
            [\tsc{f}1]
            [\tsc{ind}P
                [\phantom{xxx}, roof]
            ]
        ]
      \end{forest}
      \\
      \toprule
      \tsc{nom} relative pronoun \tit{dh-er}
      \\
      \cmidrule{1-1}
          \begin{forest} boom
          [\tsc{rel}P
              [\tsc{rel}P
                  [\phantom{x}\tit{dh}\phantom{x}, roof]
              ]
              [\tsc{nom}P,
              tikz={
              \node[label=below:{\tit{er}},
              draw,circle,
              scale=0.8,
              fit to=tree]{};
              \node[draw,circle,
              dashed,
              scale=0.85,
              fit to=tree]{};
              }
                  [\tsc{f}1]
                  [\tsc{ind}P
                      [\phantom{xxx}, roof]
                  ]
              ]
          ]
        \end{forest}
        \\
      \bottomrule
  \end{tabular}
  \end{adjustbox}
  \caption {Old High German \tsc{ext}\scsub{nom} vs. \tsc{int}\scsub{nom} → \tit{dher} (\tsc{elh})}
  \label{fig:ohg-int=ext-elh}
\end{figure}

The relative pronoun consists of two morphemes: \tit{dh} and \tit{er}.
The extra light head consists of a single morpheme: \tit{er}.
As usual, I circle the part of the structure that corresponds to a particular lexical entry, or I reduce the structure to a triangle, and I place the corresponding phonology below it.
I draw a dashed circle around the biggest possible element that is structurally contained in both the extra light head and the relative pronoun.

The extra light head consists of a single morpheme: the \tsc{nom}P.
This \tsc{nom}P is structurally contained within the relative pronoun. Therefore, the extra light head can be deleted. I signal the deletion of the extra light head by marking the content of its circle gray.
The surface pronoun is the relative pronoun that bears the internal case: \tit{dher}.

In Figure \ref{fig:ohg-int=ext-lh}, I give the syntactic structure of the light head at the top and the syntactic structure of the relative pronoun at the bottom.

\begin{figure}[htbp]
  \center
  \begin{adjustbox}{max height=0.9\textheight}
  \begin{tabular}[b]{c}
        \toprule
        \tsc{nom} light head \tit{dh-er}\\
        \cmidrule{1-1}
        \begin{forest} boom
          [DP, s sep=15mm,
          tikz={
          \node[draw,
          constituent-deletion,yshift=-0.3cm,
          dotted,
          scale=1.35,
          fit to=tree]{};
          }
              [DP,
              tikz={
              \node[label=below:\tit{dh},
              draw,circle,
              scale=0.85,
              fit to=tree]{};
              }
                  [\phantom{xxx}, roof, baseline]
              ]
              [\tsc{nom}P,
              tikz={
              \node[label=below:\tit{er},
              draw,circle,
              scale=0.85,
              fit to=tree]{};
              }
                  [\tsc{f}1]
                  [\tsc{ind}P
                      [\phantom{xxx}, roof, baseline]
                  ]
              ]
          ]
        \end{forest}
      \\
      \toprule
      \tsc{nom} relative pronoun \tit{dh-er}
      \\
      \cmidrule{1-1}
      \begin{forest} boom
        [\tsc{rel}P, s sep=17.5mm,
        tikz={
        \node[draw,
        constituent-deletion,yshift=-0.4cm,
        dotted,
        fill=DG,fill opacity=0.2,
        scale=1.25,
        fit to=tree]{};
        }
            [\tsc{rel}P,
            tikz={
            \node[label=below:\tit{dh},
            draw,circle,
            scale=0.85,
            fit to=tree]{};
            }
                [\tsc{rel}]
                [DP
                    [\phantom{xxx}, roof, baseline]
                ]
            ]
            [\tsc{nom}P,
            tikz={
            \node[label=below:\tit{er},
            draw,circle,
            scale=0.85,
            fit to=tree]{};
            }
                [\tsc{f}1]
                [\tsc{ind}P
                    [\phantom{xxx}, roof, baseline]
                ]
            ]
        ]
      \end{forest}
        \\
      \bottomrule
  \end{tabular}
  \end{adjustbox}
  \caption {Old High German \tsc{ext}\scsub{nom} vs. \tsc{int}\scsub{nom} → \tit{dher} (\tsc{lh})}
  \label{fig:ohg-int=ext-lh}
\end{figure}

The relative pronoun consists of two morphemes: \tit{dh} and \tit{er}.
The light head also consists of two morphemes: \tit{dh} and \tit{er}.
Again, I circle the part of the structure that corresponds to a particular lexical entry, or I reduce the structure to a triangle, and I place the corresponding phonology below it.
I draw a dotted circle around the biggest possible element that formally contained in both the light head and the relative pronoun.

The relative pronoun (the \tsc{rel}P realized by \tit{dher}) is formally contained within the light head (the DP realized by \tit{dher}).
Therefore, the extra light head can be deleted. I signal the deletion of the extra light head by marking the content of its circle gray.
The surface pronoun is the light head that bears the external case: \tit{dher}.\footnote{
The same holds the other way around: the light head (the DP realized by \tit{dher}) is formally contained within the relative pronoun (the \tsc{rel}P realized by \tit{dher}). Therefore, with the information I have given so far, it could also be that the light head is deleted. In Section \ref{sec:coming-back} I discuss the larger syntactic structure of headless relatives and I show in this case only the relative pronoun can be deleted because of c-command relations.
}

I continue with the situation in which the external case is the more complex one.
Consider the examples in \ref{ex:ohg-acc-nom-rep}, in which the internal nominative case competes against the external accusative case. The relative clause is marked in bold. \ref{ex:ohg-acc-nom-elh} shows the example with the extra light head and \ref{ex:ohg-acc-nom-lh} shows the example with the light head.
The internal case is nominative, as the predicate \tit{gisizzen} `to possess' takes nominative subjects.
In both examples, the relative pronoun \tit{dher} `\ac{rel}.\ac{sg}.\ac{m}.\ac{nom}' appears in the nominative case.
The external case is accusative, as the predicate \tit{bibringan} `to create' takes accusative objects.
In \ref{ex:ohg-acc-nom-elh}, the extra light head \tit{ën} `\tsc{elh}.\ac{sg}.\ac{m}.\ac{acc}' appears in the accusative case. It is placed between square brackets because it does not surface.
In \ref{ex:ohg-acc-nom-lh}, the light head \tit{dhen} `\tsc{dem}.\ac{sg}.\ac{m}.\ac{acc}' appears in the accusative case. Here the relative pronoun is placed between square brackets because it does not surface.

\ex.\label{ex:ohg-acc-nom-rep}
\ag. *ih bibringu fona iacobes samin endi fona iuda [en] \tbf{dher} \tbf{mina} \tbf{berga} \tbf{chisitzit}\\
1\ac{sg}.\ac{nom} {create}.\ac{pres}.1\ac{sg}\scsub{[acc]} of Jakob.\ac{gen} seed.\ac{sg}.\ac{dat} and of Judah.\ac{dat} \ac{rel}.\ac{sg}.\ac{m}.\ac{acc} my.\ac{acc}.\ac{m}.\ac{pl} mountain.\ac{acc}.\ac{pl} possess.\ac{pres}.3\ac{sg}\scsub{[nom]}\\
`I create of the seed of Jacob and of Judah the one, who possess my mountains' \flushfill{Old High German, \ac{isid} 34:3}\label{ex:ohg-acc-nom-elh}
\bg. ih bibringu fona iacobes samin endi fona iuda dhen [\tbf{dher}] \tbf{mina} \tbf{berga} \tbf{chisitzit}\\
1\ac{sg}.\ac{nom} {create}.\ac{pres}.1\ac{sg}\scsub{[acc]} of Jakob.\ac{gen} seed.\ac{sg}.\ac{dat} and of Judah.\ac{dat} \ac{rel}.\ac{sg}.\ac{m}.\ac{acc} my.\ac{acc}.\ac{m}.\ac{pl} mountain.\ac{acc}.\ac{pl} possess.\ac{pres}.3\ac{sg}\scsub{[nom]}\\
`I create of the seed of Jacob and of Judah the one, who possess my mountains' \flushfill{Old High German, \ac{isid} 34:3}\label{ex:ohg-acc-nom-lh}

Only \ref{ex:ohg-acc-nom-lh} can be the source that the headless relative is derived from. First I show that no headless relative can be derived from the \ref{ex:ohg-acc-nom-elh}. Then I show the comparison of the two internal syntax of the two forms in \ref{ex:ohg-acc-nom-lh}, which does derive a grammatical result.

In Figure \ref{fig:ohg-ext-wins-elh}, I give the syntactic structure of the extra light head at the top and the syntactic structure of the relative pronoun at the bottom.

\begin{figure}[htbp]
  \center
  \begin{adjustbox}{max height=0.9\textheight}
  \begin{tabular}[b]{c}
      \toprule
      \tsc{acc} extra light head \tit{en}
      \\
      \cmidrule{1-1}
      \begin{forest} boom
        [\tsc{acc}P,
        tikz={
        \node[label=below:{\tit{en}},
        draw,circle,
        scale=0.85,
        fit to=tree]{};
        }
            [\tsc{f}2]
            [\tsc{acc}P,
            tikz={
            \node[draw,circle,
            dashed,
            scale=0.8,
            fit to=tree]{};
            }
                [\ac{f}1]
                [\tsc{ind}P
                    [\phantom{xxx}, roof]
                ]
            ]
        ]
      \end{forest}
      \\
      \toprule
      \tsc{nom} relative pronoun \tit{dh-er}
      \\
      \cmidrule{1-1}
          \begin{forest} boom
            [\tsc{rel}P
                [\tsc{rel}P
                    [\phantom{x}\tit{dh}\phantom{x}, roof]
                ]
                [\tsc{acc}P,
                tikz={
                \node[label=below:{\tit{er}},
                draw,circle,
                scale=0.8,
                fit to=tree]{};
                \node[draw,circle,
                dashed,
                scale=0.85,
                fit to=tree]{};
                }
                    [\ac{f}1]
                    [\tsc{ind}P
                        [\phantom{xxx}, roof]
                    ]
                ]
            ]
        \end{forest}
        \\
      \bottomrule
  \end{tabular}
  \end{adjustbox}
   \caption {Old High German \tsc{ext}\scsub{acc} vs. \tsc{int}\scsub{nom} ↛ \tit{en}/\tit{dher} (\tsc{elh})}
  \label{fig:ohg-ext-wins-elh}
\end{figure}

The relative pronoun consists of two morphemes: \tit{dh} and \tit{en}.
The extra light head consists of a single morpheme: \tit{er}.
Again, I circle the part of the structure that corresponds to a particular lexical entry, or I reduce the structure to a triangle, and I place the corresponding phonology below it.
I draw a dashed circle around the biggest possible element that is structurally contained in both the extra light head and the relative pronoun.

In this case, the light head is not structurally contained within the relative pronoun.
The extra light head consists of a single morpheme: the \tsc{acc}P.
The relative pronoun only contains the \tsc{nom}P, and it lacks the \tsc{f}2 that makes a \tsc{acc}P. Since the weaker feature containment requirement is not met, the stronger constituent containment requirement cannot be met either.

The relative pronoun is not structurally contained within the light head. It namely lacks the complete constituent and \tsc{rel}P.
Therefore, the extra light cannot be deleted, and the relative pronoun cannot be deleted either.
As a result, the light-headed relative with the extra light head cannot be the source of the headless relative.

In Figure \ref{fig:ohg-ext-wins-lh}, I give the syntactic structure of the light head at the top and the syntactic structure of the relative pronoun at the bottom.

\begin{figure}[htbp]
  \center
  \begin{adjustbox}{max height=0.9\textheight}
  \begin{tabular}[b]{c}
        \toprule
        \tsc{acc} light head \tit{dh-en}\\
        \cmidrule{1-1}
        \begin{forest} boom
          [DP, s sep=15mm,
          tikz={
          \node[draw,
          constituent-deletion,yshift=-0.3cm,
          dotted,
          scale=1.35,
          fit to=tree]{};
          }
              [DP,
              tikz={
              \node[label=below:\tit{dh},
              draw,circle,
              scale=0.85,
              fit to=tree]{};
              }
                  [\phantom{xxx}, roof, baseline]
              ]
              [\tsc{acc}P,
              tikz={
              \node[label=below:\tit{en},
              draw,circle,
              scale=0.85,
              fit to=tree]{};
              }
                  [\tsc{f}2]
                  [\tsc{nom}P
                      [\tsc{f}1]
                      [\tsc{ind}P
                          [\phantom{xxx}, roof, baseline]
                      ]
                  ]
              ]
          ]
        \end{forest}
      \\
      \toprule
      \tsc{nom} relative pronoun \tit{dh-er}
      \\
      \cmidrule{1-1}
      \begin{forest} boom
        [\tsc{rel}P, s sep=17.5mm,
        tikz={
        \node[draw,
        constituent-deletion,yshift=-0.4cm,
        dotted,
        scale=1.25,
        fit to=tree]{};
        }
            [\tsc{rel}P,
            tikz={
            \node[label=below:\tit{dh},
            draw,circle,
            scale=0.85,
            fit to=tree]{};
            }
                [\tsc{rel}]
                [DP
                    [\phantom{xxx}, roof, baseline]
                ]
            ]
            [\tsc{nom}P,
            tikz={
            \node[label=below:\tit{er},
            draw,circle,
            scale=0.85,
            fit to=tree]{};
            }
                [\tsc{f}1]
                [\tsc{ind}P
                    [\phantom{xxx}, roof, baseline]
                ]
            ]
        ]
      \end{forest}
        \\
      \bottomrule
  \end{tabular}
  \end{adjustbox}
  \caption {Old High German \tsc{ext}\scsub{acc} vs. \tsc{int}\scsub{nom} ↛ \tit{dhën}/\tit{dhër} (\tsc{lh})}
  \label{fig:ohg-ext-wins-lh}
\end{figure}

The relative pronoun consists of two morphemes: \tit{dh} and \tit{er}.
The light head also consists of two morphemes: \tit{dh} and \tit{en}.
Again, I circle the part of the structure that corresponds to a particular lexical entry, or I reduce the structure to a triangle, and I place the corresponding phonology below it.
I draw a dotted circle around the biggest possible element that is formally contained in both the light head and the relative pronoun.

The light head is realized as \tit{dhen}, and the relative pronoun is realized as \tit{dher}.
The light head is not formally contained within the relative pronoun, and the relative pronoun is not formally contained within the light head.
Therefore, the extra light cannot be deleted, and the relative pronoun cannot be deleted either.
The inevitable result seems to be that the light-headed relative with the light head cannot be the source of the headless relative.
This is not what the data suggests, however, as a more complex case is allowed to surface in Old High German.

we need to look at the larger syntactic structure. repeat example here

\ex. only with den der

I draw a tree of the light head and the relative clause

Consider the larger syntactic structure in \ref{ex:ohg-syntax-ext-wins} that represents part of the sentence in \ref{ex:ohg-acc-nom-lh}.

\ex.\label{ex:ohg-syntax-ext-wins}
 \begin{adjustbox}{max width=0.9\textwidth}
\begin{forest} boom
[, s sep=20mm
    [DP, s sep=15mm,
    tikz={
    \node[draw,
    constituent-deletion,yshift=-0.3cm,
    dotted,
    scale=1.3,
    fit to=tree]{};
    }
        [DP,
        tikz={
        \node[label=below:\tit{dh},
        draw,circle,
        scale=0.85,
        fit to=tree]{};
        }
            [\phantom{xxx}, roof]
        ]
        [\tsc{acc}P,
        tikz={
        \node[label=below:\tit{en},
        draw,circle,
        scale=0.85,
        fit to=tree]{};
        }
            [\tsc{f}2]
            [\tsc{nom}P
                [\tsc{f}1]
                [\tsc{ind}P
                    [\phantom{xxx}, roof]
                ]
            ]
        ]
    ]
    [CP, s sep=15mm
        [\tsc{rel}P, s sep=15mm,
        tikz={
        \node[draw,
        constituent-deletion,yshift=-0.4cm,
        dotted,
        scale=1.25,
        fit to=tree]{};
        }
            [\tsc{rel}P,
            tikz={
            \node[label=below:\tit{dh},
            draw,circle,
            scale=0.85,
            fit to=tree]{};
            }
                [\tsc{rel}]
                [DP
                    [\phantom{xxx}, roof]
                ]
            ]
            [\tsc{nom}P,
            tikz={
            \node[label=below:\tit{er},
            draw,circle,
            scale=0.85,
            fit to=tree]{};
            }
                [\tsc{f}1]
                [\tsc{ind}P
                    [\phantom{xxx}, roof]
                ]
            ]
        ]
        [CP
             [\tit{mina berga chisitzit}, roof]
        ]
    ]
]
\end{forest}
\end{adjustbox}

The DP on the left represents the light head from Figure X, the RelP in the middle is the relative pronoun from Figure X. The lower CP on the right contains relative clause besides for the relative pronoun.

This structure has come into being by merging features one by one. The last feature that has been merged is \tsc{f}2, that made the relative pronoun an \tsc{acc}P. Remember from the functional sequence in X that the case features are the highest features, so they are the last ones to be merged.\footnote{
These features end up within the left DP via Backtracking, splitting up the separate workspaces (DP and CP), further Backtracking, splitting up DP and \tsc{acc}P and also \tsc{rel}P and CP. The first can be spelled out first on the \tsc{acc}P, so that is where it is realized.
}
This means that one step in the derivation ago, the syntactic structure looked as in \ref{ex:ohg-syntax-ext=int}.

\ex.\label{ex:ohg-syntax-ext=int}
\begin{adjustbox}{max width=0.9\textwidth}
\begin{forest} boom
[, s sep=20mm
    [DP, s sep=15mm,
    tikz={
    \node[draw,
    constituent-deletion,yshift=-0.3cm,
    dotted,
    scale=1.3,
    fit to=tree]{};
    }
        [DP,
        tikz={
        \node[label=below:\tit{dh},
        draw,circle,
        scale=0.85,
        fit to=tree]{};
        }
            [\phantom{xxx}, roof]
        ]
        [\tsc{nom}P,
        tikz={
        \node[label=below:\tit{er},
        draw,circle,
        scale=0.85,
        fit to=tree]{};
        }
            [\tsc{f}1]
            [\tsc{ind}P
                [\phantom{xxx}, roof]
            ]
        ]
    ]
    [CP, s sep=15mm
        [\tsc{rel}P, s sep=15mm,
        tikz={
        \node[draw,
        constituent-deletion,yshift=-0.4cm,
        dotted,
        fill=DG,fill opacity=0.2,
        scale=1.25,
        fit to=tree]{};
        }
            [\tsc{rel}P,
            tikz={
            \node[label=below:\tit{dh},
            draw,circle,
            scale=0.85,
            fit to=tree]{};
            }
                [\tsc{rel}]
                [DP
                    [\phantom{xxx}, roof]
                ]
            ]
            [\tsc{nom}P,
            tikz={
            \node[label=below:\tit{er},
            draw,circle,
            scale=0.85,
            fit to=tree]{};
            }
                [\tsc{f}1]
                [\tsc{ind}P
                    [\phantom{xxx}, roof]
                ]
            ]
        ]
        [CP
             [\tit{mina berga chisitzit}, roof]
        ]
    ]
]
\end{forest}
\end{adjustbox}

Comparing the internal syntax of the light head and the relative pronoun at this stage.. see Figure X
I describe in words what happens there.

Then the feature \tsc{f}2 is merged, and we see the effect of if had the external case deleted the internal case.

%%blabla finish this



I end with the situation in which the internal case is the more complex one.
Consider the examples in \ref{ex:ohg-nom-acc-rep}, in which the internal accusative case competes against the external nominative case. The relative clause is marked in bold. \ref{ex:ohg-nom-acc-elh} shows the example with the extra light head and \ref{ex:ohg-acc-nom-lh} shows the example with the light head.
The internal case is accusative, as the predicate \tit{zellen} `to tell' takes accusative objects.
In both examples, the relative pronoun \tit{then} `\ac{rel}.\ac{sg}.\ac{m}.\ac{acc}' appears in the accusative case.
In \ref{ex:ohg-nom-acc-elh}, the extra light head \tit{ër} `\tsc{elh}.\ac{sg}.\ac{m}.\ac{nom}' appears in the nominative case. It is placed between square brackets because it does not surface.
In \ref{ex:ohg-nom-acc-lh}, the light head \tit{dher} `\tsc{dem}.\ac{sg}.\ac{m}.\ac{nom}' appears in the nominative case. Here the relative pronoun is placed between square brackets because it does not surface.

\exg. Thíz ist [er] \tbf{then} \tbf{sie} \tbf{zéllent}\\
\ac{dem}.\ac{sg}.\ac{n}.\ac{nom} be.\ac{pres}.3\ac{sg}\scsub{[nom]} \ac{dem}.\ac{sg}.\ac{m}.\ac{nom} \ac{rel}.\ac{sg}.\ac{m}.\ac{acc} 3\ac{pl}.\ac{m}.\ac{nom} tell.\ac{pres}.3\ac{pl}\scsub{[acc]}\\
`this is the one whom they talk about' \flushfill{Old High German, \ac{otfrid} III 16:50}\label{ex:ohg-nom-acc-elh}
\exg. *Thíz ist ther [\tbf{then}] \tbf{sie} \tbf{zéllent}\\
\ac{dem}.\ac{sg}.\ac{n}.\ac{nom} be.\ac{pres}.3\ac{sg}\scsub{[nom]} \ac{dem}.\ac{sg}.\ac{m}.\ac{nom} \ac{rel}.\ac{sg}.\ac{m}.\ac{acc} 3\ac{pl}.\ac{m}.\ac{nom} tell.\ac{pres}.3\ac{pl}\scsub{[acc]}\\
`this is the one whom they talk about' \flushfill{Old High German, \ac{otfrid} III 16:50}\label{ex:ohg-nom-acc-lh}

Only \ref{ex:ohg-nom-acc-lh} can be the source that the headless relative is derived from. First I show the comparison of the two internal syntax of the two forms in \ref{ex:ohg-nom-acc-lh}, which does derive a grammatical result. Then I show that no headless relative can be derived from the \ref{ex:ohg-acc-nom-elh}.

In Figure \ref{fig:ohg-int-wins-elh}, I give the syntactic structure of the extra light head at the top and the syntactic structure of the relative pronoun at the bottom.

\begin{figure}[htbp]
  \center
  \begin{adjustbox}{max height=0.9\textheight}
  \begin{tabular}[b]{c}
      \toprule
      \tsc{nom} extra light head \tit{er}
      \\
      \cmidrule{1-1}
      \begin{forest} boom
        [\tsc{nom}P,
        tikz={
        \node[label=below:{\tit{er}},
        draw,circle,
        scale=0.8,
        fit to=tree]{};
        \node[draw,circle,
        dashed,
        scale=0.85,
        fill=DG,fill opacity=0.2,
        fit to=tree]{};
        }
            [\tsc{f}1]
            [\tsc{ind}P
                [\phantom{xxx}, roof]
            ]
        ]
      \end{forest}
      \\
      \toprule
      \tsc{acc} relative pronoun \tit{th-en}
      \\
      \cmidrule{1-1}
          \begin{forest} boom
            [\tsc{rel}P
                [\tsc{rel}P
                    [\phantom{x}\tit{th}\phantom{x}, roof]
                ]
                [\tsc{acc}P,
                tikz={
                \node[label=below:{\tit{en}},
                draw,circle,
                scale=0.85,
                fit to=tree]{};
                }
                    [\tsc{f}3]
                    [\tsc{acc}P,
                    tikz={
                    \node[draw,circle,
                    dashed,
                    scale=0.8,
                    fit to=tree]{};
                    }
                        [\tsc{f}1]
                        [\tsc{ind}P
                            [\phantom{xxx}, roof]
                        ]
                    ]
                ]
            ]
        \end{forest}
        \\
      \bottomrule
  \end{tabular}
  \end{adjustbox}
   \caption {Old High German \tsc{ext}\scsub{nom} vs. \tsc{int}\scsub{acc} → \tit{then}}
  \label{fig:ohg-int-wins}
\end{figure}

The relative pronoun consists of two morphemes: \tit{th} and \tit{en}.
The extra light head consists of a single morpheme: \tit{er}.
Again, I circle the part of the structure that corresponds to a particular lexical entry, or I reduce the structure to a triangle, and I place the corresponding phonology below it.
I draw a dashed circle around the biggest possible element that is structurally a constituent in both the extra light head and the relative pronoun.

The extra light head consists of a single morpheme: the \tsc{nom}P.
This \tsc{nom}P is structurally contained within the relative pronoun. Therefore, the extra light can be deleted. I signal the deletion of the extra light head by marking the content of its circle gray.
The surface pronoun is the relative pronoun that bears the internal case: \tit{then}.

In Figure \ref{fig:ohg-int-wins-lh}, I give the syntactic structure of the light head at the top and the syntactic structure of the relative pronoun at the bottom.

\begin{figure}[htbp]
  \center
  \begin{adjustbox}{max height=0.9\textheight}
  \begin{tabular}[b]{c}
        \toprule
        \tsc{nom} light head \tit{th-er}\\
        \cmidrule{1-1}
        \begin{forest} boom
          [DP, s sep=15mm,
          tikz={
          \node[draw,
          constituent-deletion,yshift=-0.3cm,
          dotted,
          scale=1.35,
          fit to=tree]{};
          }
              [DP,
              tikz={
              \node[label=below:\tit{th},
              draw,circle,
              scale=0.85,
              fit to=tree]{};
              }
                  [\phantom{xxx}, roof, baseline]
              ]
              [\tsc{nom}P,
              tikz={
              \node[label=below:\tit{er},
              draw,circle,
              scale=0.85,
              fit to=tree]{};
              }
                  [\tsc{f}1]
                  [\tsc{ind}P
                      [\phantom{xxx}, roof, baseline]
                  ]
              ]
          ]
        \end{forest}
      \\
      \toprule
      \tsc{acc} relative pronoun \tit{dh-en}
      \\
      \cmidrule{1-1}
      \begin{forest} boom
        [\tsc{rel}P, s sep=17.5mm,
        tikz={
        \node[draw,
        constituent-deletion,yshift=-0.4cm,
        dotted,
        scale=1.25,
        fit to=tree]{};
        }
            [\tsc{rel}P,
            tikz={
            \node[label=below:\tit{th},
            draw,circle,
            scale=0.85,
            fit to=tree]{};
            }
                [\tsc{rel}]
                [DP
                    [\phantom{xxx}, roof, baseline]
                ]
            ]
            [\tsc{acc}P,
            tikz={
            \node[label=below:\tit{er},
            draw,circle,
            scale=0.85,
            fit to=tree]{};
            }
                [\tsc{f}2]
                [\tsc{nom}P
                    [\tsc{f}1]
                    [\tsc{ind}P
                        [\phantom{xxx}, roof, baseline]
                    ]
                ]
            ]
        ]
      \end{forest}
        \\
      \bottomrule
  \end{tabular}
  \end{adjustbox}
  \caption {Old High German \tsc{ext}\scsub{nom} vs. \tsc{int}\scsub{acc} ↛ \tit{ther}/\tit{then} (\tsc{lh})}
  \label{fig:ohg-ext-wins-lh}
\end{figure}

The relative pronoun consists of two morphemes: \tit{dh} and \tit{en}.
The light head also consists of two morphemes: \tit{dh} and \tit{er}.
Again, I circle the part of the structure that corresponds to a particular lexical entry, or I reduce the structure to a triangle, and I place the corresponding phonology below it.
I draw a dotted circle around the biggest possible element that formally contained in both the light head and the relative pronoun.

The light head is realized as \tit{dher}, and the relative pronoun is realized as \tit{dhen}.
The light head is not formally contained within the relative pronoun, and the relative pronoun is not formally contained within the light head.
Therefore, the extra light cannot be deleted, and the relative pronoun cannot be deleted either.
As a result, the light-headed relative with the extra light head cannot be the source of the headless relative.

Also this at some point in the derivation does not work. It's the relative pronoun that has too many cases, so taking of case features from that one does not help.



\section{Coming back to the light heads}\label{sec:coming-back}



I assume that a syntactic structure of a light-headed relative looks as in X\footnote{
I actually assume that a light-headed relative with an extra light looks as in X

\ex. here an example with a non-D with the head in the low position

here explain what is in the example.
This is also what Cinque says: non-definite heads are low, and definite heads are high.
Two questions follow from such an analysis: (1) how do the case features end up down there, and (2) what triggers the movement of the light head to the higher position. About (1), Cinque says that it is feature percolation, and I follow that intuition. Technically, what's happening is backtracking, opening up the different workspaces, which leads to the case features finding a match on the element to the left of the relative clause. Concerning (2), Cinque says it's movement, I'm not sure what it's triggered by. I don't know what it is. If it's movement, then it can be triggered by spellout or by features. I don't see how either of them should work. It could be connected to the formation of a complex spec. It seems that as soon the spec is there, the light head also moves up, and the complex spec does not attach to the relative clause. I leave this for future research.
}

\ex. here an example of a high D and the relative clause below it

There is a D, which appears higher in the structure than the relative clause, etc. etc. explanation
the relative clause is already complete, including case features
This structure for light-headed relatives is also assumed by cf. Cinque etc. etc.
the features that are merged last in building a light head are the case features. first we have a D without case features, and then the case features are merged on by one. this means that we have a stage in the dervation dathat looks like:

\ex. no cases, including relative pronoun and relative clause

\ex. only nominative case

now there's deletion!

then we merge the next case feature, and we get a more complex external case

note that we also have c-command for the deletion! great!\footnote{
coming back the extra light head, we also have c-command there, under the defininteino of kayne
}


Headless relatives in which the relative pronoun starts with a \tit{d}, such as in Old High German, seem to be linked to individuating or definite readings and not to generalizing or indefinite readings \citep[cf.][]{fuss2017}. I illustrate this with the two examples I repeat from Chapter  \ref{ch:typology}.

Consider the example in \ref{ex:ohg-nom-acc-interpretation}, repeated from Chapter \ref{ch:typology}.
In this example, the author refers to the specific person which was talked about, and not to any or every person that was talked about.

%int = acc, ext = nom, so extra light head, but individuation so light head expected
\exg. Thíz ist \tbf{then} \tbf{sie} \tbf{zéllent}\\
\ac{dem}.\ac{sg}.\ac{n}.\ac{nom} be.\ac{pres}.3\ac{sg}\scsub{[nom]} \ac{rel}.\ac{sg}.\ac{m}.\ac{acc}
3\ac{pl}.\ac{m}.\ac{nom} tell.\ac{pres}.3\ac{pl}\scsub{[acc]}\\
`this is the one whom they talk about'\\
not: `this is whoever they talk about' \flushfill{Old High German, \ac{otfrid} III 16:50}\label{ex:ohg-nom-acc-interpretation-1}

Consider also the example in \ref{ex:ohg-nom-acc-interpretation}, repeated from Chapter \ref{ch:typology}.
In this example, the author refers to the specific person who spoke to someone, and not to any or every person who spoke to someone.

%int = nom, ext = dat, so light head, and individuation so light head also expected
\exg. enti aer {ant uurta} demo \tbf{zaimo} \tbf{sprah}\\
and 3\ac{sg}.\ac{m}.\ac{nom} reply.\ac{pst}.3\ac{sg}\scsub{[dat]} \ac{rel}.\ac{sg}.\ac{m}.\ac{dat} {to 3\ac{sg}.\ac{m}.\ac{dat}} speak.\ac{pst}.3\ac{sg}\scsub{[nom]}\\
`and he replied to the one who spoke to him'\\
not: `and he replied to whoever spoke to him'
 \flushfill{Old High German, \ac{mons} 7:24, adapted from \pgcitealt{pittner1995}{199}}\label{ex:ohg-dat-nom-rep-1}











  - possible prediction: ext>int = def, int>ext = wh, not what we see, show 4 examples



Consider the example in \ref{ex:ohg-nom-acc-interpretation}, repeated from Chapter \ref{ch:typology}.
In this example, the author refers to the specific person which was talked about, and not to any or every person that was talked about.

%int = acc, ext = nom, so extra light head, but individuation so light head expected
\exg. Thíz ist \tbf{then} \tbf{sie} \tbf{zéllent}\\
\ac{dem}.\ac{sg}.\ac{n}.\ac{nom} be.\ac{pres}.3\ac{sg}\scsub{[nom]} \ac{rel}.\ac{sg}.\ac{m}.\ac{acc}
3\ac{pl}.\ac{m}.\ac{nom} tell.\ac{pres}.3\ac{pl}\scsub{[acc]}\\
`this is the one whom they talk about'\\
not: `this is whoever they talk about' \flushfill{Old High German, \ac{otfrid} III 16:50}\label{ex:ohg-nom-acc-interpretation}

Consider also the example in \ref{ex:ohg-nom-acc-interpretation}, repeated from Chapter \ref{ch:typology}.
In this example, the author refers to the specific person who spoke to someone, and not to any or every person who spoke to someone.

%int = nom, ext = dat, so light head, and individuation so light head also expected
\exg. enti aer {ant uurta} demo \tbf{zaimo} \tbf{sprah}\\
and 3\ac{sg}.\ac{m}.\ac{nom} reply.\ac{pst}.3\ac{sg}\scsub{[dat]} \ac{rel}.\ac{sg}.\ac{m}.\ac{dat} {to 3\ac{sg}.\ac{m}.\ac{dat}} speak.\ac{pst}.3\ac{sg}\scsub{[nom]}\\
`and he replied to the one who spoke to him'\\
not: `and he replied to whoever spoke to him'
 \flushfill{Old High German, \ac{mons} 7:24, adapted from \pgcitealt{pittner1995}{199}}\label{ex:ohg-dat-nom-rep}










 Old High German is special because the relative pronoun in its headless relatives is syncretic with the relative pronoun in its light-headed relatives.

 This light head story never works for Modern German or Polish because for them the relative pronoun and the light head are not syncretic: \tit{den}-\tsc{wen} and \tit{tego}-\tit{kogo}


 According to Cinque, every type of relative clause in every language is underlyingly double-headed. Evidence for this claim comes from languages that show this morphologically. An example from Kombai is given in \ref{ex:kombai}. The head of the relative clause is \tit{doü} `sago', and it appears inside the relative clause and outside.

 \exg. [\tbf{doü} adiyan-o-no] \tbf{doü} deyalukhe\\
  sago give.3\tsc{pl}.\tsc{nonfut}-{tr}-\tsc{conn} sago finished.\tsc{adj}\\
  `The sago that they gave is finished.' \flushfill{Kombai, \pgcitealt{vries1993}{78}}\label{ex:kombai}

 The internal and external instances of \tit{doü} correspond to the internal and external element I assume to be there in the headless relatives.

 \ref{ex:double-syntax} shows the syntactic structure of the sentence in \ref{ex:kombai}.

 \ex.
 \begin{forest} boom
 [CP
    [FP
       [CP
           [\tsc{int}
              [\tit{doü}, roof]
           ]
           [CP
               [\tit{adiyan-o-no}, roof]
           ]
       ]
       [\tsc{ext}
          [\tit{doü}, roof]
       ]
    ]
    [VP
       [\tit{deyalukhe}, roof]
    ]
 ]
 \end{forest}\label{ex:double-syntax}

 In most languages one of the two heads is deleted throughout the derivation.

 According to \citealt{cinqueforthcoming}, the internal element can delete the external element, because the internal element c-commands the external element. This is c-command according to Kayne's definition of it: the internal element is in the specifier of the specifier of the FP.

 \ex.
 \begin{forest} boom
 [
    [CP
        [\tsc{int}
           [\phantom{xxx}, roof]
        ]
        [CP
            [\phantom{xxx}, roof]
        ]
    ]
    [\tsc{ext}
       [\phantom{xxx}, roof]
    ]
 ]
 \end{forest}\label{ex:cinque-int-wins}

 In order for the internal element to be able to delete the external element, a movement needs to take place. The external element moves over the relative clause.\footnote{
 What remains unclear is what the trigger is for the movement of the external element over relative clause is.
 }
 From this position, the external element can delete the internal one, because the external element c-commands the internal one.

 \ex.
 \begin{forest} boom
 [
     [\tsc{ext}
        [\phantom{xxx}, roof]
     ]
     [FP
        [CP
            [\tsc{int}
               [\phantom{xxx}, roof]
            ]
            [CP
                [\phantom{xxx}, roof]
            ]
        ]
        [\tit{t\scsub{ext}}]
     ]
 ]
 \end{forest}

 Also talk about D here, and that maybe Old High German deletes a thing without a D when the internal thing wins. does that also have a not so definite interpretation?


 In the previous section I introduced the relative pronoun as the internal element. This means that the other element is the external element. This section starts with the observation that there actually are languages in which two elements surface in so-called double-headed relative clauses. In these languages, the external head is a subset of the internal head, and that some features like D and case are necessarily excluded in the external head. I adopt this insight, and I apply it to the headless relative situation. I propose that the external head in headless relatives is a copy of a specific part of the relative pronoun.

 As I said earlier, I need two elements to do case competition with. In headless relatives, I only see a single one surfacing. However, some languages actually show two elements surfacing. Here there are two copies of the element, one inside the relative clause, one outside of the relative clause.

 \exg. [\tbf{doü} adiyan-o-no] \tbf{doü} deyalukhe\\
  sago give.3\tsc{pl}.\tsc{nonfut}-{tr}-\tsc{conn} sago finished.\tsc{adj}\\
  `The sago that they gave is finished.' \flushfill{Kombai, \pgcitealt{vries1993}{78}}

The external element is not always an exact copy of the element inside of the relative clause. An example from Kombai shows that the element outside of the relative clause can also be a subset of what the element inside of the relative clause is. Here I give two examples, there is an \tit{old man} and a \tit{person}, and there is \tit{pig} and a \tit{thing}.

 \ex.
 \ag. [\tbf{yare} gamo khereja bogi-n-o] \tbf{rumu} na-momof-a\\
  {old man} join.\ac{ss} work do.\ac{dur}.3\ac{sg}.\ac{nf}-\ac{tr}-\ac{conn} person my-uncle-\ac{pred}\\
  `The old man, who is joining the work, is my uncle.' 77
 \bg. [\tbf{ai} fali-khano] \tbf{ro} nagu-n-ay-a.\\
  pig carry-go.3\tsc{pl}.\tsc{nf} thing our-\tsc{tr}-pig-\ac{pred}\\
  `The pig they took away, is ours.' \flushfill{Kombai, \pgcitealt{vries1993}{77}}

 Let me now apply what we have seen so far to headless relatives. Headless relatives do not have an overt NP, so this cannot be copied. However, there is the relative pronoun which is specified for number, gender, case, etc. Are all of these features copied onto the external element? The copy is the portion of the nominal extended projection c-commanded by the relative clause. A headless relative is a restrictive relative clause. Therefore, there is no D and no case.

 Is it possible to add features onto the external head after it has been copied? Yes, for example D, as the example shows, but also case.

 \exg. Junya-wa [Ayaka-ga \tbf{ringo}-o mui-ta] sono \tbf{ringo}-o tabe-ta.\\
 Junya-\ac{top} Ayaka-\ac{nom} apple-\ac{acc} peel-\ac{pst} that apple-\ac{acc} eat-\ac{pst}\\
 ‘Junya ate the apples that Ayaka peeled.’ \flushfill{Japanese, \pgcitealt{erlewine2016}{2}}

 In sum, the external element is a copy of a subset of the features of the relative pronoun. Definiteness and case are not copied. New features can be merged onto the external element.


\phantom{x}

\section{Summary}
