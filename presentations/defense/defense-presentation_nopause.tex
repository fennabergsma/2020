\documentclass[xcolor=dvipsnames,10pt]{beamer}

\usepackage{../fenna-files-presentation/packages}
\usepackage{../fenna-files-presentation/commands}
\addbibresource{../fenna-files-presentation/references.bib}

\usepackage{multicol}

\geometry{paperwidth=140mm,paperheight=105mm}
%
% Choose how your presentation looks.
%
% For more themes, color themes and font themes, see:
% http://deic.uab.es/~iblanes/beamer_gallery/index_by_theme.html
%
\mode<presentation>
{
\usetheme{Berlin}      % or try Darmstadt, Madrid, Warsaw, ...
\definecolor{goethe}{rgb}{0,0.37,0.66}
\usecolortheme[named=goethe]{structure}
	% \usefonttheme{serif}  % or try serif, structurebold, ...
\setbeamertemplate{navigation symbols}{}
\setbeamertemplate{caption}[numbered]
\resetcounteronoverlays{exx}
}

\renewcommand{\eachwordone}{\sffamily}
\renewcommand{\eachwordtwo}{\sffamily}
\renewcommand{\eachwordthree}{\sffamily}

%gets rid of bottom navigation bars
\setbeamertemplate{footline}{}

%gets rid of navigation symbols
% \setbeamertemplate{navigation symbols}{}


\addtobeamertemplate{navigation symbols}{}{%
\usebeamerfont{footline}%
\usebeamercolor[fg]{footline}%
\hspace{1em}%
\insertframenumber/\inserttotalframenumber
}


\title{Case competition in headless relatives}
\author{Fenna Bergsma}
\date{\today}
\institute{Goethe-Universität Frankfurt}



\begin{document}

\begin{frame}
\titlepage

\centering{
\includegraphics[width=0.4\textwidth]{../fenna-files-presentation/dfg-logo}
\hspace{2cm}
\includegraphics[width=0.3\textwidth]{../fenna-files-presentation/goethe-logo}
}
\end{frame}


\begin{frame}{Topic}

\pause

\exg. Ich {lade ein}, \textbf<3->{wem} \textbf<3->{auch} \textbf<3->{Maria} \textbf<3->{vertraut}.\\
 1\tsc{sg}.{\tsc{nom}} invite.\tsc{pres}.1\tsc{sg}\textcolor<6->{LimeGreen}{\scsub{[acc]}} \tsc{rel}.\tsc{an}.\textcolor<8->{red}{\tsc{dat}} also Maria.{\tsc{nom}} trust.\tsc{pres}.3\tsc{sg}\textcolor<7->{red}{\scsub{[dat]}}\\
 `I invite whoever Maria also trusts.'

\only<4>{
\exg. Ich lade die Person ein, \textbf<3->{dem} \textbf<3->{Maria} \textbf<3->{vertraut}.\\
 1\tsc{sg}.{\tsc{nom}} invite.\tsc{pres}.1\tsc{sg}{\scsub{[acc]}} the person {} \tsc{rel}.\tsc{an}.{\tsc{dat}} Maria.{\tsc{nom}} trust.\tsc{pres}.3\tsc{sg}{\scsub{[dat]}}\\
 `I invite the person that Maria trusts.'
 }

\onslide<9>
\exg. *Ich {lade ein}, wen \tbf{auch} \tbf{Maria} \tbf{vertraut}.\\
 1\tsc{sg}.{\tsc{nom}} invite.\tsc{pres}.1\tsc{sg}\textcolor{LimeGreen}{\scsub{[acc]}} \tsc{rel}.\tsc{an}.\textcolor{LimeGreen}{\tsc{acc}} also Maria.{\tsc{nom}} trust.\tsc{pres}.3\tsc{sg}\textcolor{red}{\scsub{[dat]}}\\
 `I invite whoever Maria also trusts.'

\end{frame}


\begin{frame}{Two aspects}

\pause

\begin{itemize}
  \item the winner of the competition  \onslide<4->{→ is stable across languages}\pause
  \item whether the winner gets approved \onslide<4->{→ differs across languages}
\end{itemize}

\pause
\pause

\vspace{1em}

\begin{enumerate}
  \item show the generalizations\pause
  \item derive the generalizations
\end{enumerate}

\end{frame}



\begin{frame}{The winner of the competition --- German}

\onslide<2->{
\exg. Ich {lade ein}, \tbf{wem} \tbf{auch} \tbf{Maria} \tbf{vertraut}. \\
1\tsc{sg}.{\tsc{nom}} invite.\tsc{pres}.1\tsc{sg}\textcolor{LimeGreen}{\scsub{[acc]}} \tsc{rel}.\tsc{an}.\textcolor{red}{\tsc{dat}} also Maria.{\tsc{nom}} trust.\tsc{pres}.3\tsc{sg}\textcolor{red}{\scsub{[dat]}}\\
`I invite whoever Maria also trusts.'
}

\only<3-4>{
\exg. *Ich {lade ein}, wen \tbf{auch} \tbf{Maria} \tbf{vertraut}. \\
1\tsc{sg}.{\tsc{nom}} invite.\tsc{pres}.1\tsc{sg}\textcolor{LimeGreen}{\scsub{[acc]}} \tsc{rel}.\tsc{an}.\textcolor{LimeGreen}{\tsc{acc}} also Maria.{\tsc{nom}} trust.\tsc{pres}.3\tsc{sg}\textcolor{red}{\scsub{[dat]}}\\
`I invite whoever Maria also trusts.'
}

\onslide<5->{
\exg. Uns besucht, \tbf{wen} \tbf{Maria} \tbf{mag}.\\
2\tsc{pl}.{\tsc{acc}} visit.\tsc{pres}.3\tsc{sg}\textcolor{Turquoise}{\scsub{[nom]}} \tsc{rel}.\tsc{an}.\textcolor{LimeGreen}{\tsc{acc}} Maria.{\tsc{nom}} like.\tsc{pres}.3\tsc{sg}\textcolor{LimeGreen}{\scsub{[acc]}}\\
`Who visits us, Maria likes.'
}

\only<6-7>{
\exg. *Uns besucht, wer \tbf{Maria} \tbf{mag}.\\
2\tsc{pl}.{\tsc{acc}} visit.\tsc{pres}.3\tsc{sg}\textcolor{Turquoise}{\scsub{[nom]}} \tsc{rel}.\tsc{an}.\textcolor{Turquoise}{\tsc{nom}} Maria.{\tsc{nom}} like.\tsc{pres}.3\tsc{sg}\textcolor{LimeGreen}{\scsub{[acc]}}\\
`Who visits us, Maria likes.'
}

\onslide<8-10>{
\exg. Uns besucht, \tbf{wem} \tbf{Maria} \tbf{vertraut}.\\
2\tsc{pl}.{\tsc{acc}} visit.\tsc{pres}.3\tsc{sg}\textcolor{Turquoise}{\scsub{[nom]}} \tsc{rel}.\tsc{an}.\textcolor{red}{\tsc{dat}} Maria.{\tsc{nom}} trust.\tsc{pres}.3\tsc{sg}\textcolor{red}{\scsub{[dat]}}\\
`Who visits us, Maria trusts.'
}

\only<9>{
\exg. *Uns besucht, wer \tbf{Maria} \tbf{vertraut}.\\
2\tsc{pl}.{\tsc{acc}} visit.\tsc{pres}.3\tsc{sg}\textcolor{Turquoise}{\scsub{[nom]}} \tsc{rel}.\tsc{an}.\textcolor{Turquoise}{\tsc{nom}} Maria.{\tsc{nom}} trust.\tsc{pres}.3\tsc{sg}\textcolor{red}{\scsub{[dat]}}\\
`Who visits us, Maria trusts.'
}

\onslide<7->{\tsc{nom} <} \onslide<4->{\tsc{acc} < \tsc{dat}}

\end{frame}



\begin{frame}{The winner of the competition --- other languages}

\pause

\tsc{nom} < \tsc{acc} < \tsc{dat}

\pause

\exg. \tbf{hòn} \tbf{hoi} \tbf{theoì} \tbf{philoũsin} apothnḗͅskei néos\\
\tsc{rp}.\tsc{sg}.\tsc{m}.\textcolor{LimeGreen}{\tsc{acc}} the god.\tsc{pl} love.3\tsc{pl}\textcolor{LimeGreen}{\scsub{[acc]}} die.3\tsc{sg}\textcolor{Turquoise}{\scsub{[nom]}} young\\
`He, whom the gods love, dies young.' \flushfill{Classical Greek, Menander, The Double Deceiver 125}\label{ex:ag-nom-acc}

\pause

\exg. \tbf{themo} \tbf{min} \tbf{uuirdit} \tbf{forlazan}, min minnot\\
\tsc{rp}.\tsc{sg}.\tsc{m}.\textcolor{red}{\tsc{dat}} less become.\tsc{pres}.3\tsc{sg} read.\tsc{inf}\textcolor{red}{\scsub{[dat]}} less love.\tsc{pres}.3\tsc{sg}\textcolor{Turquoise}{\scsub{[nom]}}\\
`whom less is read, loves less' \flushfill{Old High German, Tatian 138:13}\label{ex:ohg-nom-dat}

\pause

\exg. ei galaubjaiþ þamm \tbf{-ei} \tbf{insandida} \tbf{jains}\\
that believe.\tsc{opt}.\tsc{pres}.2\tsc{pl}\textcolor{red}{\scsub{[dat]}} \tsc{rel}.\tsc{sg}.\tsc{m}.\textcolor{red}{\tsc{dat}} -\tsc{comp} {send}.\tsc{pret}.3\tsc{sg}\textcolor{LimeGreen}{\scsub{[acc]}} \tsc{dem}.\tsc{sg}.\tsc{m}.\tsc{nom}\\
`that you believe in him whom he sent' \flushfill{Gothic, John 6:29}\label{ex:gothic-dat-acc}

\end{frame}



\begin{frame}{Whether the winner gets approved --- German}

\onslide<2->{
\exg. Ich {lade ein}, \tbf{wem} \tbf{auch} \tbf{Maria} \tbf{vertraut}.\\
1\tsc{sg}.{\tsc{nom}} invite.\tsc{pres}.1\tsc{sg}\textcolor{LimeGreen}{\scsub{[acc]}} \tsc{rel}.\tsc{an}.\textcolor{red}{\tsc{dat}} also Maria.{\tsc{nom}} trust.\tsc{pres}.3\tsc{sg}\textcolor{red}{\scsub{[dat]}}\\
`I invite whoever Maria also trusts.'
}

\only<3>{
\exg. *Ich {lade ein}, wen \tbf{auch} \tbf{Maria} \tbf{vertraut}.\\
1\tsc{sg}.{\tsc{nom}} invite.\tsc{pres}.1\tsc{sg}\textcolor{LimeGreen}{\scsub{[acc]}} \tsc{rel}.\tsc{an}.\textcolor{LimeGreen}{\tsc{acc}} also Maria.{\tsc{nom}} trust.\tsc{pres}.3\tsc{sg}\textcolor{red}{\scsub{[dat]}}\\
`I invite whoever Maria also trusts.'
}

\onslide<4->{
\exg. *Ich vertraue, wem \tbf{auch} \tbf{Maria} \tbf{mag}. \\
1\tsc{sg}.{\tsc{nom}} trust.\tsc{pres}.1\tsc{sg}\textcolor{red}{\scsub{[dat]}} \tsc{rel}.\tsc{an}.\textcolor{red}{\tsc{dat}} also Maria.{\tsc{nom}} like.\tsc{pres}.3\tsc{sg}\textcolor{LimeGreen}{\scsub{[acc]}}\\
`I trust whoever Maria also likes.'
}

\onslide<5->{
\exg. *Ich vertraue, \tbf{wen} \tbf{auch} \tbf{Maria} \tbf{mag}. \\
1\tsc{sg}.{\tsc{nom}} trust.\tsc{pres}.1\tsc{sg}\textcolor{red}{\scsub{[dat]}} \tsc{rel}.\tsc{an}.\textcolor{LimeGreen}{\tsc{acc}} also Maria.{\tsc{nom}} like.\tsc{pres}.3\tsc{sg}\textcolor{LimeGreen}{\scsub{[acc]}}\\
`I trust whoever Maria also likes.'
}

\only<6>{
internal vs. external case
}

\onslide<7->{
the internal case (\tsc{int}) gets approved when it wins the case competition}\onslide<8->{,
the external case (\tsc{ext}) does not
}

\end{frame}


\begin{frame}{Whether the winner gets approved --- German schema}

  \begin{table}[H]
    \center
    \begin{tabular}{c|c|c|c}
      \toprule
      \textsubscript{\tsc{int}} \textsuperscript{\tsc{ext}}
             & [\tsc{nom}]
             & [\tsc{acc}]
             & [\tsc{dat}]
             \\ \cmidrule{1-4}
         [\tsc{nom}]
             & \tsc{nom}
             & \cellcolor{LG}*
             & \cellcolor{LG}*
             \\ \cmidrule{1-4}
         [\tsc{acc}]
             & \cellcolor{DG}\tsc{acc}
             & \tsc{acc}
             & \cellcolor{LG}*
             \\ \cmidrule{1-4}
         [\tsc{dat}]
             & \cellcolor{DG}\tsc{dat}
             & \cellcolor{DG}\tsc{dat}
             & \tsc{dat}
             \\
       \bottomrule
    \end{tabular}
      \label{tbl:case-competition-only-int}
  \end{table}

\end{frame}



\begin{frame}{Whether the winner gets approved --- Old High German}

\pause

  \exg. \tbf{themo} \tbf{min} \tbf{uuirdit} \tbf{forlazan}, min minnot\\
  \tsc{rp}.\tsc{sg}.\tsc{m}.\textcolor{red}{\tsc{dat}} less become.\tsc{pres}.3\tsc{sg} read.\tsc{inf}\textcolor{red}{\scsub{[dat]}} less love.\tsc{pres}.3\tsc{sg}\textcolor{Turquoise}{\scsub{[nom]}}\\
  `whom less is read, loves less' \flushfill{Old High German, Tatian 138:13}\label{ex:ohg-nom-dat}

\pause

  \exg. enti aer {ant uurta} demo \tbf{zaimo} \tbf{sprah}\\
  and 3\tsc{sg}.\tsc{m}.\tsc{nom} reply.\tsc{pst}.3\tsc{sg}\textcolor{red}{\scsub{[dat]}} \tsc{rp}.\tsc{sg}.\tsc{m}.\textcolor{red}{\tsc{dat}} {to 3\tsc{sg}.\tsc{m}.\tsc{dat}} speak.\tsc{pst}.3\tsc{sg}\textcolor{Turquoise}{\scsub{[nom]}}\\
  `and he replied to the one who spoke to him' \flushfill{Old High German, \tsc{mons} 7:24, adapted from \pgcitealt{pittner1995}{199}}\label{ex:ohg-dat-nom}

\pause

  \begin{table}[H]
    \center
    \begin{tabular}{c|c|c|c}
      \toprule
      \textsubscript{\tsc{int}} \textsuperscript{\tsc{ext}}
             & [\tsc{nom}]
             & [\tsc{acc}]
             & [\tsc{dat}]
             \\ \cmidrule{1-4}
         [\tsc{nom}]
             & \tsc{nom}
             & \cellcolor{LG}\tsc{acc}
             & \cellcolor{LG}\tsc{dat}
             \\ \cmidrule{1-4}
         [\tsc{acc}]
             & \cellcolor{DG}\tsc{acc}
             & \tsc{acc}
             & \cellcolor{LG}\tsc{dat}
             \\ \cmidrule{1-4}
         [\tsc{dat}]
             & \cellcolor{DG}\tsc{dat}
             & \cellcolor{DG}\tsc{dat}
             & \tsc{dat}
             \\
       \bottomrule
    \end{tabular}
      \label{tbl:case-competition-int-ext}
  \end{table}

\end{frame}


\begin{frame}{Whether the winner gets approved --- Polish}

\pause

  \exg. *Jan lubi \tbf{komu} \tbf{-kolkwiek} \tbf{dokucza}.\\
  Jan like.\tsc{3sg}\textcolor{LimeGreen}{\scsub{[acc]}} \tsc{rel}.\textcolor{red}{\tsc{dat}}.\tsc{an}.\tsc{sg} ever tease.\tsc{3sg}\textcolor{red}{\scsub{[dat]}}\\
  `Jan likes whoever he teases.'

\pause

  \exg. *Jan ufa komu \tbf{-kolkwiek} \tbf{wpuścil} \tbf{do} \tbf{domu}.\\
  Jan trust.\tsc{3sg}\textcolor{red}{\scsub{[dat]}} \tsc{rel}.\textcolor{red}{\tsc{dat}}.\tsc{an}.\tsc{sg} ever let.\tsc{3sg}\textcolor{LimeGreen}{\scsub{[acc]}} to home\\
  `Jan trusts whoever he let into the house.'

\pause

  \begin{table}[H]
    \center
    \begin{tabular}{c|c|c|c}
      \toprule
      \textsubscript{\tsc{int}} \textsuperscript{\tsc{ext}}
             & [\tsc{nom}]
             & [\tsc{acc}]
             & [\tsc{dat}]
             \\ \cmidrule{1-4}
         [\tsc{nom}]
             & \tsc{nom}
             & \cellcolor{LG}*
             & \cellcolor{LG}*
             \\ \cmidrule{1-4}
         [\tsc{acc}]
             & \cellcolor{DG}*
             & \tsc{acc}
             & \cellcolor{LG}*
             \\ \cmidrule{1-4}
         [\tsc{dat}]
             & \cellcolor{DG}*
             & \cellcolor{DG}*
             & \tsc{dat}
             \\
       \bottomrule
    \end{tabular}
      \label{tbl:case-competition-none}
  \end{table}

\end{frame}


\begin{frame}{Whether the winner gets approved --- overview}

\begin{multicols}{2}

\onslide<2->{
  \begin{table}[H]
    \center
    \begin{tabular}{c|c|c|c}
      \toprule
      \textsubscript{\tsc{int}} \textsuperscript{\tsc{ext}}
             & [\tsc{nom}]
             & [\tsc{acc}]
             & [\tsc{dat}]
             \\ \cmidrule{1-4}
         [\tsc{nom}]
             & \tsc{nom}
             & \cellcolor{LG}*
             & \cellcolor{LG}*
             \\ \cmidrule{1-4}
         [\tsc{acc}]
             & \cellcolor{DG}\tsc{acc}
             & \tsc{acc}
             & \cellcolor{LG}*
             \\ \cmidrule{1-4}
         [\tsc{dat}]
             & \cellcolor{DG}\tsc{dat}
             & \cellcolor{DG}\tsc{dat}
             & \tsc{dat}
             \\
       \bottomrule
    \end{tabular}
      \label{tbl:case-competition-only-int}
  \end{table}
}

\onslide<2->{
  \begin{table}[H]
    \center
    \begin{tabular}{c|c|c|c}
      \toprule
      \textsubscript{\tsc{int}} \textsuperscript{\tsc{ext}}
             & [\tsc{nom}]
             & [\tsc{acc}]
             & [\tsc{dat}]
             \\ \cmidrule{1-4}
         [\tsc{nom}]
             & \tsc{nom}
             & \cellcolor{LG}\tsc{acc}
             & \cellcolor{LG}\tsc{dat}
             \\ \cmidrule{1-4}
         [\tsc{acc}]
             & \cellcolor{DG}\tsc{acc}
             & \tsc{acc}
             & \cellcolor{LG}\tsc{dat}
             \\ \cmidrule{1-4}
         [\tsc{dat}]
             & \cellcolor{DG}\tsc{dat}
             & \cellcolor{DG}\tsc{dat}
             & \tsc{dat}
             \\
       \bottomrule
    \end{tabular}
      \label{tbl:case-competition-int-ext}
  \end{table}
}

\onslide<3-5>{
  \begin{table}[H]
    \center
    \begin{tabular}{c|c|c|c}
      \toprule
      \textsubscript{\tsc{int}} \textsuperscript{\tsc{ext}}
             & [\tsc{nom}]
             & [\tsc{acc}]
             & [\tsc{dat}]
             \\ \cmidrule{1-4}
         [\tsc{nom}]
             & \tsc{nom}
             & \cellcolor{LG}*
             & \cellcolor{LG}*
             \\ \cmidrule{1-4}
         [\tsc{acc}]
             & \cellcolor{DG}*
             & \tsc{acc}
             & \cellcolor{LG}*
             \\ \cmidrule{1-4}
         [\tsc{dat}]
             & \cellcolor{DG}*
             & \cellcolor{DG}*
             & \tsc{dat}
             \\
       \bottomrule
    \end{tabular}
      \label{tbl:case-competition-none}
  \end{table}
}

\only<4>{
  \begin{table}[H]
    \center
    \begin{tabular}{c|c|c|c}
      \toprule
      \textsubscript{\tsc{int}} \textsuperscript{\tsc{ext}}
             & [\tsc{nom}]
             & [\tsc{acc}]
             & [\tsc{dat}]
             \\ \cmidrule{1-4}
         [\tsc{nom}]
             & \tsc{nom}
             & \cellcolor{LG}\tsc{acc}
             & \cellcolor{LG}\tsc{dat}
             \\ \cmidrule{1-4}
         [\tsc{acc}]
             & \cellcolor{DG}*
             & \tsc{acc}
             & \cellcolor{LG}\tsc{dat}
             \\ \cmidrule{1-4}
         [\tsc{dat}]
             & \cellcolor{DG}*
             & \cellcolor{DG}*
             & \tsc{dat}
             \\
       \bottomrule
    \end{tabular}
      \label{tbl:case-competition-only-ext}
  \end{table}
}

  \end{multicols}

\end{frame}



\begin{frame}{Two generalizations}
\pause

\begin{itemize}
  \item The winner of the case competition → is stable across languages\\ \pause
  \tsc{nom} < \tsc{acc} < \tsc{dat}\pause
  \item Whether the winner gets approved → differs across languages\\ \pause
  \tsc{int}/\tsc{ext} approved\pause
  \begin{itemize}
    \item \tsc{int} approved
    \item \tsc{int} + \tsc{ext} approved
    \item none approved
  \end{itemize}
\end{itemize}\pause

\vspace{1em}

\begin{enumerate}
  \item show the generalizations\pause
  \item derive the generalizations
\end{enumerate}

\end{frame}



\begin{frame}{Generalization 1: \tsc{nom} < \tsc{acc} < \tsc{dat}}

\pause

\begin{multicols}{2}

\begin{table}[H]
  \center
  \begin{tabular}{cl}
  \toprule
             & 3\tsc{sg} \\
             \cmidrule{2-2}
  \tsc{nom}
             & \onslide<2->{luw}                                    \\
  \tsc{acc}  & \onslide<3->{luw\tbf{-e:l}}                          \\
  \tsc{dat}  & \onslide<4->{luw\tbf{-e:l}\tcol{DG}{\tbf{-na}}}      \\
  \bottomrule
  \end{tabular}
\end{table}

\vspace{1em}

\center

\onslide<8->{
\begin{forest} boom
  [\tsc{dat}P
      [\tsc{k}3]
      [\tsc{acc}P
          [\tsc{k}2]
          [\tsc{nom}P
              [\tsc{k}1]
              [XP]
          ]
      ]
  ]
\end{forest}
}

\vspace{1em}

\begin{itemize}
  \item \onslide<5->{syncretism patterns (ref)}
  \item \onslide<6->{agreement (ref)}
  \item \onslide<7->{relativization (ref)}
\end{itemize}

\vspace{2em}

\onslide<9->{
\begin{forest} boom
[\tsc{dat}P
    [\tsc{acc}P
        [\tsc{nom}P,
        tikz={
        \node[label={below:\tit{luw}},
        draw,circle,
        scale=0.875,
        fit to=tree]{};
        }
            [3\tsc{sg}P]
            [\tsc{k}1]
        ]
        [\tsc{acc}P,
        tikz={
        \node[label={below:\tit{e:l}},
        draw,circle,
        scale=0.775,
        fit to=tree]{};
        }
            [\tsc{k}2]
        ]
    ]
    [\tsc{dat}P,
    tikz={
    \node[label={below:\tit{na}},
    draw,circle,
    scale=0.775,
    fit to=tree]{};
    }
        [\tsc{k}3]
    ]
]
\end{forest}
}

\end{multicols}

\end{frame}


\begin{frame}{The winning case contains the losing case}

\center

  \begin{forest} boom
    [\tsc{dat}P,
    tikz={
    \onslide<2-3>{
    \node[draw,circle,transparent,
    fill=DG,fill opacity=0.2,
    scale=0.875,
    fit to=tree]{};
    }
    }
        [\tsc{k}3]
          [\tsc{acc}P,
          tikz={
          \onslide<2>{
          \node[draw,circle,transparent,
          fill=DG,fill opacity=0.4,
          scale=0.825,
          fit to=tree]{};
          }
          \onslide<4>{
          \node[draw,circle,transparent,
          fill=DG,fill opacity=0.2,
          scale=0.875,
          fit to=tree]{};
          }
          }
            [\tsc{k}2]
            [\tsc{nom}P,
            tikz={
            \onslide<3-4>{
            \node[draw,circle,transparent,
            fill=DG,fill opacity=0.4,
            scale=0.775,
            fit to=tree]{};
            }
            }
                [\tsc{k}1]
                [XP
                    [\phantom{xxx}, roof]
                ]
            ]
        ]
    ]
  \end{forest}\label{ex:dat-contains-acc}

\end{frame}


\begin{frame}{Generalization 2: \tsc{int}/\tsc{ext} approved}

\pause

  \begin{figure}[H]
    \centering
    \begin{tabular}[b]{c}
      \toprule
      \begin{tikzpicture}[node distance=1.5cm]
        \node (question2) [question]
        {\tsc{int} approved};
            \node (outcome2) [outcome, below of=question2, xshift=-2cm, yshift=-0.5cm]
            {matching};
                \node (example2) [example, below of=outcome2]
                {e.g. Polish\\\phantom{x}\\\phantom{x}};
            \node (question3) [question, below of=question2, xshift=2.5cm, yshift=-1cm]
            {\tsc{ext} approved};
                \node (outcome3) [outcome, below of=question3, xshift=-2cm, yshift=-0.5cm]
                {internal-only};
                    \node (example3) [example, below of=outcome3]
                    {e.g. Modern German\\\phantom{x}};
                \node (outcome4) [outcome, below of=question3, xshift=2cm, yshift=-0.5cm]
                {unrestricted};
                    \node (example4) [example, below of=outcome4]
                    {e.g. Gothic, Old High German, Classical Greek};

      \draw [arrow] (question2) -- node[anchor=east] {no} (outcome2);
      \draw [arrow] (question2) -- node[anchor=west] {yes} (question3);
      \draw [arrow] (question3) -- node[anchor=east] {no} (outcome3);
      \draw [arrow] (question3) -- node[anchor=west] {yes} (outcome4);

      \pause\node[yshift=-0.5cm,
      draw,circle,color=red,
      scale=1,line width=2pt,
      fit=(question2)]{};
      \end{tikzpicture}\\
      \bottomrule
    \end{tabular}
      \label{fig:two-parameters}
  \end{figure}

\end{frame}



\begin{frame}{Borer-Chomsky Conjecture}

  Borer-Chomsky Conjecture: the lexicon is the source of language variation

\end{frame}


\begin{frame}{Assumptions}

\begin{itemize}
  \item \onslide<2->{headless relatives are derived from light-headed relatives}
  \only<3>{\\
  \vspace{1em}
  light head\scsub{ext} [relative pronoun\scsub{int} ... ]\label{ex:light+rel}
  }
  \item \onslide<4->{deletion takes place when the light head is contained in the relative pronoun}
  \item \onslide<5->{the relative pronoun contains the features of the light head plus an additional one}\\
  \onslide<6->{
  \begin{figure}[H]
    \center
    \begin{tabular}[b]{ccc}
        \toprule
        light head & & relative pronoun \\
        \cmidrule(lr){1-1} \cmidrule(lr){3-3}
        \begin{forest} boom
        [\tsc{k}P,
            [\tsc{k}]
            [ϕP, baseline]
        ]
        \end{forest}
        & \phantom{x} &
      \begin{forest} boom
        [\tsc{rel}P
            [\tsc{rel}]
            [\tsc{k}P
                [\tsc{k}]
                [ϕP, baseline]
            ]
        ]
      \end{forest}\\
        \bottomrule
    \end{tabular}
    \label{fig:rel-lh-intonly-1}
  \end{figure}
  }
\end{itemize}

\end{frame}




\begin{frame}{First show German}

  \begin{figure}[H]
    \center
    \begin{tabular}[b]{ccc}
        \toprule
        light head & & relative pronoun \\
        \cmidrule(lr){1-1} \cmidrule(lr){3-3}
        \begin{forest} boom
          [\tsc{nom}P,
          tikz={
          \node[draw,circle,
          dashed,
          scale=0.85,
          fill=DG,fill opacity=0.2,
          fit to=tree]{};
          }
              [\tsc{k}1]
              [ϕP
                  [\phantom{xxx}, roof, baseline]
              ]
          ]
        \end{forest}
        & \phantom{x} &
        \begin{forest} boom
          [\tsc{rel}P
              [\tsc{rel}P
                  [\phantom{xxx}, roof, baseline]
              ]
              [\tsc{nom}P,
              tikz={
              \node[draw,circle,
              dashed,
              scale=0.85,
              fit to=tree]{};
              }
                  [\tsc{k}1]
                  [ϕP
                      [\phantom{xxx}, roof, baseline]
                  ]
              ]
          ]
        \end{forest}\\
        \bottomrule
    \end{tabular}
     \caption {\tsc{ext}\scsub{nom} vs. \tsc{int}\scsub{nom} in the internal-only type}
    \label{fig:nom-nom-intonly}
  \end{figure}


\end{frame}




\begin{frame}{Different lexical entries}

  \begin{forest} boom
    [\tsc{acc}P
        [\tsc{k}2]
        [\tsc{nom}P
            [\tsc{k}1]
            [3\tsc{sg}.\tsc{k}P
                [\phantom{xxx}, roof]
            ]
        ]
    ]
  \end{forest}

\begin{multicols}{2}

  \begin{forest} boom
  [\tsc{acc}P,
  tikz={
  \node[label=below:\tit{sie},
  draw,circle,
  scale=0.825,
  fit to=tree]{};
  }
      [\tsc{k}2]
      [\tsc{nom}P
          [\tsc{k}1]
          [3\tsc{sg}.\tsc{k}P
              [\phantom{xxx}, roof]
          ]
      ]
  ]
  \end{forest}

  \begin{forest} boom
  [\tsc{acc}P, s sep=15mm
      [\tsc{nom}P,
      tikz={
      \node[label={below:\tit{luw}},
      draw,circle,
      scale=0.775,
      fit to=tree]{};
      }
          [\tsc{k}1]
          [3\tsc{sg}P
              [\phantom{xxx}, roof]
          ]
      ]
      [\tsc{acc}P,
      tikz={
      \node[label={below:\tit{e:l}},
      draw,circle,
      scale=0.775,
      fit to=tree]{};
      }
       [\tsc{k}2]
      ]
  ]
  \end{forest}

\end{multicols}


\end{frame}



\begin{frame}


  \begin{table}[H]
    \center
    \caption{Different language types in different situations}
    \begin{tabular}{ccc}
      \toprule
    language type &   situation                               & surface element         \\
    \cmidrule(lr){1-1}  \cmidrule(lr){2-2} \cmidrule(lr){3-3}
    internal-only &   \tsc{k}\scsub{int} = \tsc{k}\scsub{ext} & \tsc{rp}\scsub{int/ext} \\
                  &   \tsc{k}\scsub{int} > \tsc{k}\scsub{ext} & \tsc{rp}\scsub{int}     \\
                  &   \tsc{k}\scsub{int} < \tsc{k}\scsub{ext} & *                       \\
                  \cmidrule(lr){2-2} \cmidrule(lr){3-3}
    matching      &   \tsc{k}\scsub{int} = \tsc{k}\scsub{ext} & \tsc{rp}\scsub{int/ext} \\
                  &   \tsc{k}\scsub{int} > \tsc{k}\scsub{ext} & *                       \\
                  &   \tsc{k}\scsub{int} < \tsc{k}\scsub{ext} & *                       \\
    \bottomrule
    \end{tabular}
    \label{tbl:overview-situations}
    \end{table}


\end{frame}


\begin{frame}


\end{frame}






\begin{frame}{References}

Citko, Barbara (2013). “Size matters: Multidominance and DP structure in Polish”. In: \tit{Talk at the th Poznan Linguistic Meeting}.\\
Daskalaki, Evangelia (2011). “Case Mis-Matching as Kase Stranding”. In: U\tit{niversity of Pennsylvania Working Papers in Linguistics.} Ed. by Lauren A. Friedman. Vol. 17. Philadelphia: Penn Linguistics Club, pp. 77–86.\\
Harbert, Wayne Eugene (1978). “Gothic syntax: a relational grammar”. PhD thesis. Urbana-Champaign.\\
Himmelreich, Anke (2017). “Case Matching Effects in Free Relatives and Parasitic Gaps: A Study on the Properties of Agree”. PhD thesis. Universität Leipzig.\\
Vogel, Ralf (2001). “Case Conflict in German Free Relative Constructions: An Optimality Theoretic Treatment”. In: \tit{Competition in Syntax.} Ed. by Gereon Müller and Wolfgang Sternefeld. Berlin: Mouton de Gruyter, pp. 341–375. doi: 10.1515/9783110829068.341.

\end{frame}


\end{document}
