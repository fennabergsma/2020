\documentclass[xcolor=dvipsnames,10pt]{beamer}

\usepackage{../fenna-files-presentation/packages}
\usepackage{../fenna-files-presentation/commands}
\addbibresource{../fenna-files-presentation/references.bib}

\usepackage{multicol}
\usepackage{adjustbox}
\usepackage{ulem}

\geometry{paperwidth=140mm,paperheight=105mm}
%
% Choose how your presentation looks.
%
% For more themes, color themes and font themes, see:
% http://deic.uab.es/~iblanes/beamer_gallery/index_by_theme.html
%
\mode<presentation>
{
\usetheme{Berlin}      % or try Darmstadt, Madrid, Warsaw, ...
\definecolor{goethe}{rgb}{0,0.37,0.66}
\usecolortheme[named=goethe]{structure}
	% \usefonttheme{serif}  % or try serif, structurebold, ...
\setbeamertemplate{navigation symbols}{}
\setbeamertemplate{caption}[numbered]
\resetcounteronoverlays{exx}
}

\renewcommand{\eachwordone}{\sffamily}
\renewcommand{\eachwordtwo}{\sffamily}
\renewcommand{\eachwordthree}{\sffamily}

%gets rid of bottom navigation bars
\setbeamertemplate{footline}{}

%gets rid of navigation symbols
\setbeamertemplate{navigation symbols}{}


\addtobeamertemplate{navigation symbols}{}{%
\usebeamerfont{footline}%
\usebeamercolor[fg]{footline}%
\hspace{1em}%
\insertframenumber/\inserttotalframenumber
}


\title{Case competition in headless relatives}
\author{Fenna Bergsma}
\date{\today}
\institute{\normalsize{Research Training Group `Nominal Modification'\\ Goethe-Universität Frankfurt}}



\begin{document}

\begin{frame}[t]
\titlepage

\centering{
\includegraphics[width=0.4\textwidth]{../fenna-files-presentation/dfg-logo}
\hspace{2cm}
\includegraphics[width=0.3\textwidth]{../fenna-files-presentation/goethe-logo}
}
\end{frame}

\begin{frame}[t]{Introduction}

\pause

\exg. Ich {lade ein}, \textbf<3->{wem} \textbf<3->{auch} \textbf<3->{Maria} \textbf<3->{vertraut}.\\
 I invite\textcolor<6->{LimeGreen}{\scsub{[acc]}} \tsc{rp}.\textcolor<8->{red}{\tsc{dat}} also Maria trust\textcolor<5->{red}{\scsub{[dat]}}\\
 `I invite whoever Maria also trusts.' \hfill{(Modern German, \pgcitealt{vogel2001}{344})}

\only<4-8>{
\exg. Ich lade die Person ein, \textbf<3->{der} \textbf<3->{Maria} \textbf<3->{vertraut}.\\
 I invite\textcolor<7-8>{LimeGreen}{\scsub{[acc]}}
 the.\textcolor<7-8>{LimeGreen}{\tsc{acc}} person {}
 \tsc{rp}.\textcolor<7-8>{red}{\tsc{dat}} Maria
 trust\textcolor<7-8>{red}{\scsub{[dat]}}\\
 `I invite the person that Maria trusts.'
 }

 \vspace{1em}

\onslide<10->{two factors determine grammaticality}

\begin{enumerate}
  \item \onslide<11->{the case of the relative pronoun}
  \begin{itemize}
      \onslide<12->{\item the winner is determined by \tbf{\tsc{nom} < \tsc{acc} < \tsc{dat}}}
      \onslide<13->{\item is stable across languages}
  \end{itemize}
  \item \onslide<14->{where the winning case comes from}
  \begin{itemize}
      \onslide<15->{\item \tbf{\tsc{int}/\tsc{ext}} case is allowed to surface when it wins}
      \onslide<16->{\item differs across languages}
  \end{itemize}
\end{enumerate}

\vspace{1em}

\onslide<17->{
\begin{itemize}
  \item \textbf<18>{illustrate the generalizations}\pause
  \item derive the generalizations
\end{itemize}
}

\end{frame}



\begin{frame}[t]{The case of the relative pronoun in Modern German (Vogel 2001)}

\pause

\center
\textbf<6-8,9-11,12>{\tsc{nom}} \textbf<6-8,9-11,12>{<} \textbf<3-5,6-8,12>{\tsc{acc}} \textbf<3-5,12>{<} \textbf<3-5,9-11,12>{\tsc{dat}}

\citep[cf.][]{harbert1978,pittner1995,vogel2001,grosu2003,caha2019}

\onslide<4->{
\exg. Ich {lade ein}, \tbf{wem} \tbf{auch} \tbf{Maria} \tbf{vertraut}. \\
I invite\textcolor{LimeGreen}{\scsub{[acc]}} \tsc{rp}.\textcolor{red}{\tsc{dat}} also Maria trust\textcolor{red}{\scsub{[dat]}}\\
`I invite whoever Maria also trusts.'
}

\only<5>{
\exg. *Ich {lade ein}, wen \tbf{auch} \tbf{Maria} \tbf{vertraut}. \\
I invite\textcolor{LimeGreen}{\scsub{[acc]}} \tsc{rp}.\textcolor{LimeGreen}{\tsc{acc}} also Maria trust\textcolor{red}{\scsub{[dat]}}\\
`I invite whoever Maria also trusts.'
}

\vspace{1em}

\onslide<7->{
\exg. Uns besucht, \tbf{wen} \tbf{Maria} \tbf{mag}.\\
us visit\textcolor{Turquoise}{\scsub{[nom]}} \tsc{rp}.\textcolor{LimeGreen}{\tsc{acc}} Maria like\textcolor{LimeGreen}{\scsub{[acc]}}\\
`Who visits us, Maria likes.'
}

\only<8>{
\exg. *Uns besucht, wer \tbf{Maria} \tbf{mag}.\\
us visit\textcolor{Turquoise}{\scsub{[nom]}} \tsc{rp}.\textcolor{Turquoise}{\tsc{nom}} Maria like\textcolor{LimeGreen}{\scsub{[acc]}}\\
`Who visits us, Maria likes.'
}

\vspace{1em}

\onslide<10-12>{
\exg. Uns besucht, \tbf{wem} \tbf{Maria} \tbf{vertraut}.\\
us visit\textcolor{Turquoise}{\scsub{[nom]}} \tsc{rp}.\textcolor{red}{\tsc{dat}} Maria trust\textcolor{red}{\scsub{[dat]}}\\
`Who visits us, Maria trusts.'
}

\only<11>{
\exg. *Uns besucht, wer \tbf{Maria} \tbf{vertraut}.\\
us visit\textcolor{Turquoise}{\scsub{[nom]}} \tsc{rp}.\textcolor{Turquoise}{\tsc{nom}} Maria trust\textcolor{red}{\scsub{[dat]}}\\
`Who visits us, Maria trusts.'
}

\end{frame}



\begin{frame}[t]{The case of the relative pronoun in other languages}

\pause

\center
\tsc{nom} < \tsc{acc} < \tsc{dat}

\pause

\exg. \tbf{hòn} \tbf{hoi} \tbf{theoì} \tbf{philoũsin} apothnḗͅskei néos\\
\tsc{rp}.\textcolor{LimeGreen}{\tsc{acc}} the god love\textcolor{LimeGreen}{\scsub{[acc]}} die\textcolor{Turquoise}{\scsub{[nom]}} young\\
`He, whom the gods love, dies young.' \flushfill{Classical Greek, Menander, The Double Deceiver 125}\label{ex:ag-nom-acc}

\pause

\exg. \tbf{themo} \tbf{min} \tbf{uuirdit} \tbf{forlazan}, min minnot\\
\tsc{rp}.\textcolor{red}{\tsc{dat}} less become read\textcolor{red}{\scsub{[dat]}} less love\textcolor{Turquoise}{\scsub{[nom]}}\\
`whom less is read, loves less' \flushfill{Old High German, Tatian 138:13}\label{ex:ohg-nom-dat}

\pause

\exg. ei galaubjaiþ þamm \tbf{-ei} \tbf{insandida} \tbf{jains}\\
that believe\textcolor{red}{\scsub{[dat]}} \tsc{rp}.\textcolor{red}{\tsc{dat}} -\tsc{comp} {send}\textcolor{LimeGreen}{\scsub{[acc]}} he\\
`that you believe in him whom he sent' \flushfill{Gothic, John 6:29}\label{ex:gothic-dat-acc}

\end{frame}



\begin{frame}[t]{Where the winning case comes from in Modern German}

\pause
{
\center

\begin{itemize}
  \item \textbf<4>{winning case =
  \tsc{int} case}
  \item \textbf<6>{winning case =
  \tsc{ext} case}

  \citep[cf.][]{grosu1994,himmelreich2017,cinqueforthcoming}
\end{itemize}
}


\onslide<3->{
\exg. Ich {lade ein}, \tbf{wem} \tbf{auch} \tbf{Maria} \tbf{vertraut}.\\
I invite\textcolor{LimeGreen}{\scsub{[acc]}} \tsc{rp}.\textcolor{red}{\tsc{dat}} also Maria trust\textcolor{red}{\scsub{[dat]}}\\
`I invite whoever Maria also trusts.'
}

\onslide<5->{
\exg. *Ich vertraue, wem \tbf{auch} \tbf{Maria} \tbf{mag}. \\
I trust\textcolor{red}{\scsub{[dat]}} \tsc{rp}.\textcolor{red}{\tsc{dat}} also Maria like\textcolor{LimeGreen}{\scsub{[acc]}}\\
`I trust whoever Maria also likes.'
}

\onslide<7->{
\tsc{int} case is allowed to surface when it wins the case competition,
\tsc{ext} case is not
}

\end{frame}


\begin{frame}[t]{Where the winning case comes from in Modern German}

\pause
the \tsc{int} case is allowed to surface when it wins the case competition,
the \tsc{ext} case is not

\pause
  \begin{table}[H]
    \center
    \begin{tabular}{c|c|c|c}
      \toprule
      \textsubscript{\tsc{int}} \textsuperscript{\tsc{ext}}
             & [\tsc{nom}]
             & [\tsc{acc}]
             & [\tsc{dat}]
             \\ \cmidrule{1-4}
         [\tsc{nom}]
             & \tsc{nom}
             & \cellcolor{LG}*
             & \cellcolor{LG}*
             \\ \cmidrule{1-4}
         [\tsc{acc}]
             & \cellcolor{DG}\tsc{acc}
             & \tsc{acc}
             & \cellcolor{LG}*
             \\ \cmidrule{1-4}
         [\tsc{dat}]
             & \cellcolor{DG}\tsc{dat}
             & \cellcolor{DG}\tsc{dat}
             & \tsc{dat}
             \\
       \bottomrule
    \end{tabular}
      \label{tbl:case-competition-only-int}
  \end{table}

\end{frame}



\begin{frame}[t]{Where the winning case comes from in Old High German}

\pause

both the \tsc{int} case and the \tsc{ext} case are allowed to surface when they win the case competition

\pause

  \exg. \tbf{themo} \tbf{min} \tbf{uuirdit} \tbf{forlazan}, min minnot\\
  \tsc{rp}.\textcolor{red}{\tsc{dat}} less become read\textcolor{red}{\scsub{[dat]}} less love\textcolor{Turquoise}{\scsub{[nom]}}\\
  `whom less is read, loves less' \flushfill{Old High German, Tatian 138:13}\label{ex:ohg-nom-dat}

\pause

  \exg. enti aer {ant uurta} demo \tbf{zaimo} \tbf{sprah}\\
  and he reply\textcolor{red}{\scsub{[dat]}} \tsc{rp}.\textcolor{red}{\tsc{dat}} {to him} speak\textcolor{Turquoise}{\scsub{[nom]}}\\
  `and he replied to the one who spoke to him' \flushfill{Old High German, \tsc{mons} 7:24, adapted from \pgcitealt{pittner1995}{199}}\label{ex:ohg-dat-nom}

\pause

  \begin{table}[H]
    \center
    \begin{tabular}{c|c|c|c}
      \toprule
      \textsubscript{\tsc{int}} \textsuperscript{\tsc{ext}}
             & [\tsc{nom}]
             & [\tsc{acc}]
             & [\tsc{dat}]
             \\ \cmidrule{1-4}
         [\tsc{nom}]
             & \tsc{nom}
             & \cellcolor{LG}\tsc{acc}
             & \cellcolor{LG}\tsc{dat}
             \\ \cmidrule{1-4}
         [\tsc{acc}]
             & \cellcolor{DG}\tsc{acc}
             & \tsc{acc}
             & \cellcolor{LG}\tsc{dat}
             \\ \cmidrule{1-4}
         [\tsc{dat}]
             & \cellcolor{DG}\tsc{dat}
             & \cellcolor{DG}\tsc{dat}
             & \tsc{dat}
             \\
       \bottomrule
    \end{tabular}
      \label{tbl:case-competition-int-ext}
  \end{table}

\end{frame}


\begin{frame}[t]{Where the winning case comes from in Polish}

\pause

neither the \tsc{int} case nor the \tsc{ext} case is allowed to surface when it wins the case competition

\pause

  \exg. *Jan lubi \tbf{komu} \tbf{-kolkwiek} \tbf{dokucza}.\\
  Jan like\textcolor{LimeGreen}{\scsub{[acc]}} \tsc{rp}.\textcolor{red}{\tsc{dat}} ever tease\textcolor{red}{\scsub{[dat]}}\\
  `Jan likes whoever he teases.'

\pause

  \exg. *Jan ufa komu \tbf{-kolkwiek} \tbf{wpuścil} \tbf{do} \tbf{domu}.\\
  Jan trust\textcolor{red}{\scsub{[dat]}} \tsc{rp}.\textcolor{red}{\tsc{dat}} ever let\textcolor{LimeGreen}{\scsub{[acc]}} to home\\
  `Jan trusts whoever he let into the house.'

\pause

  \begin{table}[H]
    \center
    \begin{tabular}{c|c|c|c}
      \toprule
      \textsubscript{\tsc{int}} \textsuperscript{\tsc{ext}}
             & [\tsc{nom}]
             & [\tsc{acc}]
             & [\tsc{dat}]
             \\ \cmidrule{1-4}
         [\tsc{nom}]
             & \tsc{nom}
             & \cellcolor{LG}*
             & \cellcolor{LG}*
             \\ \cmidrule{1-4}
         [\tsc{acc}]
             & \cellcolor{DG}*
             & \tsc{acc}
             & \cellcolor{LG}*
             \\ \cmidrule{1-4}
         [\tsc{dat}]
             & \cellcolor{DG}*
             & \cellcolor{DG}*
             & \tsc{dat}
             \\
       \bottomrule
    \end{tabular}
      \label{tbl:case-competition-none}
  \end{table}

\end{frame}


\begin{frame}[t]{Where the winning case comes from in overview}

%order

\begin{multicols}{2}

\onslide<2->{
  \begin{table}[H]
    \center
    \caption{Modern German pattern}
    \begin{tabular}{c|c|c|c}
      \toprule
      \textsubscript{\tsc{int}} \textsuperscript{\tsc{ext}}
             & [\tsc{nom}]
             & [\tsc{acc}]
             & [\tsc{dat}]
             \\ \cmidrule{1-4}
         [\tsc{nom}]
             & \tsc{nom}
             & \cellcolor{LG}*
             & \cellcolor{LG}*
             \\ \cmidrule{1-4}
         [\tsc{acc}]
             & \cellcolor{DG}\tsc{acc}
             & \tsc{acc}
             & \cellcolor{LG}*
             \\ \cmidrule{1-4}
         [\tsc{dat}]
             & \cellcolor{DG}\tsc{dat}
             & \cellcolor{DG}\tsc{dat}
             & \tsc{dat}
             \\
       \bottomrule
    \end{tabular}
      \label{tbl:case-competition-only-int}
  \end{table}
}

\onslide<4-6>{
  \begin{table}[H]
    \center
    \caption{Polish pattern}
    \begin{tabular}{c|c|c|c}
      \toprule
      \textsubscript{\tsc{int}} \textsuperscript{\tsc{ext}}
             & [\tsc{nom}]
             & [\tsc{acc}]
             & [\tsc{dat}]
             \\ \cmidrule{1-4}
         [\tsc{nom}]
             & \tsc{nom}
             & \cellcolor{LG}*
             & \cellcolor{LG}*
             \\ \cmidrule{1-4}
         [\tsc{acc}]
             & \cellcolor{DG}*
             & \tsc{acc}
             & \cellcolor{LG}*
             \\ \cmidrule{1-4}
         [\tsc{dat}]
             & \cellcolor{DG}*
             & \cellcolor{DG}*
             & \tsc{dat}
             \\
       \bottomrule
    \end{tabular}
      \label{tbl:case-competition-none}
  \end{table}
}

\onslide<3->{
  \begin{table}[H]
    \center
    \caption{Old High German pattern}
    \begin{tabular}{c|c|c|c}
      \toprule
      \textsubscript{\tsc{int}} \textsuperscript{\tsc{ext}}
             & [\tsc{nom}]
             & [\tsc{acc}]
             & [\tsc{dat}]
             \\ \cmidrule{1-4}
         [\tsc{nom}]
             & \tsc{nom}
             & \cellcolor{LG}\tsc{acc}
             & \cellcolor{LG}\tsc{dat}
             \\ \cmidrule{1-4}
         [\tsc{acc}]
             & \cellcolor{DG}\tsc{acc}
             & \tsc{acc}
             & \cellcolor{LG}\tsc{dat}
             \\ \cmidrule{1-4}
         [\tsc{dat}]
             & \cellcolor{DG}\tsc{dat}
             & \cellcolor{DG}\tsc{dat}
             & \tsc{dat}
             \\
       \bottomrule
    \end{tabular}
      \label{tbl:case-competition-int-ext}
  \end{table}
}

\onslide<5>{
  \begin{table}[H]
    \center
    \caption{unattested pattern}
    \begin{tabular}{c|c|c|c}
      \toprule
      \textsubscript{\tsc{int}} \textsuperscript{\tsc{ext}}
             & [\tsc{nom}]
             & [\tsc{acc}]
             & [\tsc{dat}]
             \\ \cmidrule{1-4}
         [\tsc{nom}]
             & \tsc{nom}
             & \cellcolor{LG}\tsc{acc}
             & \cellcolor{LG}\tsc{dat}
             \\ \cmidrule{1-4}
         [\tsc{acc}]
             & \cellcolor{DG}*
             & \tsc{acc}
             & \cellcolor{LG}\tsc{dat}
             \\ \cmidrule{1-4}
         [\tsc{dat}]
             & \cellcolor{DG}*
             & \cellcolor{DG}*
             & \tsc{dat}
             \\
       \bottomrule
    \end{tabular}
      \label{tbl:case-competition-only-ext}
  \end{table}
}

\end{multicols}

\end{frame}



\begin{frame}[t]{Two factors determine grammaticality}

\pause

\begin{enumerate}
  \item {the case of the relative pronoun}
  \begin{itemize}
      {\item the winner is determined by \tbf{\tsc{nom} < \tsc{acc} < \tsc{dat}}}
      {\item is stable across languages}
  \end{itemize}
  \item {where the winning case comes from}
  \begin{itemize}
      {\item \tbf{\tsc{int}/\tsc{ext}} case is allowed to surface when it wins}
      {\item differs across languages}
  \end{itemize}
\end{enumerate}

  \vspace{1em}

  {
  \begin{itemize}
    \item {illustrate the generalizations}
    \item \textbf<3>{derive the generalizations}
  \end{itemize}
  }

\end{frame}




\begin{frame}[t]{\tsc{nom} < \tsc{acc} < \tsc{dat}}

\pause

\begin{multicols}{2}

\begin{table}[H]
  \center
  \caption{Khanty 3\tsc{sg} pronouns (\pgcitealt{nikolaeva1999}{16} after \citealt{smith2019})}
  \begin{tabular}{cl}
  \toprule
             & 3\tsc{sg} \\
             \cmidrule{2-2}
  \tsc{nom}
             & \onslide<2->{luw}             \\
  \tsc{acc}  & \onslide<3->{luw-e:l}         \\
  \tsc{dat}  & \onslide<4->{luw-e:l-na}      \\
  \bottomrule
  \end{tabular}
\end{table}

\vspace{1em}

\begin{itemize}
  \item \onslide<5->{syncretism patterns \citep[cf.][]{baerman2005}}
  \item \onslide<6->{agreement \citep[cf.][]{moravcsik1978}}
  \item \onslide<7->{relativization \citep[cf.][]{keenan1977}}
\end{itemize}

\vspace{4em}

\center

\onslide<8->{
a single trigger
}

\vspace{2em}

\onslide<9->{
\begin{forest} boom
  [\tsc{dat}P
      [\tsc{k}3]
      [\tsc{acc}P
          [\tsc{k}2]
          [\tsc{nom}P
              [\tsc{k}1]
              [XP]
          ]
      ]
  ]
\end{forest}

\citep{caha2009,caha2019}
}

\vspace{1em}


\end{multicols}

\end{frame}


\begin{frame}[t]{The winning case contains the losing case}

\center

\pause

  \begin{forest} boom
    [\tsc{dat}P,
    tikz={
    \onslide<2-3>{
    \node[draw,circle,transparent,
    fill=DG,fill opacity=0.2,
    scale=0.875,
    fit to=tree]{};
    }
    }
        [\tsc{k}3]
          [\tsc{acc}P,
          tikz={
          \onslide<2>{
          \node[draw,circle,transparent,
          fill=DG,fill opacity=0.4,
          scale=0.825,
          fit to=tree]{};
          }
          \onslide<4>{
          \node[draw,circle,transparent,
          fill=DG,fill opacity=0.2,
          scale=0.825,
          fit to=tree]{};
          }
          }
            [\tsc{k}2]
            [\tsc{nom}P,
            tikz={
            \onslide<3-4>{
            \node[draw,circle,transparent,
            fill=DG,fill opacity=0.4,
            scale=0.775,
            fit to=tree]{};
            }
            }
                [\tsc{k}1]
                [XP
                    [\phantom{xxx}, roof]
                ]
            ]
        ]
    ]
  \end{forest}\label{ex:dat-contains-acc}


\end{frame}



\begin{frame}[t]{\tsc{int}/\tsc{ext}}

\pause

\begin{multicols}{2}

  \begin{table}[H]
    \center
    \caption{Modern German pattern}
    \begin{tabular}{c|c|c|c}
      \toprule
      \textsubscript{\tsc{int}} \textsuperscript{\tsc{ext}}
             & [\tsc{nom}]
             & [\tsc{acc}]
             & [\tsc{dat}]
             \\ \cmidrule{1-4}
         [\tsc{nom}]
             & \tsc{nom}
             & \cellcolor{LG}*
             & \cellcolor{LG}*
             \\ \cmidrule{1-4}
         [\tsc{acc}]
             & \cellcolor{DG}\tsc{acc}
             & \tsc{acc}
             & \cellcolor{LG}*
             \\ \cmidrule{1-4}
         [\tsc{dat}]
             & \cellcolor{DG}\tsc{dat}
             & \cellcolor{DG}\tsc{dat}
             & \tsc{dat}
             \\
       \bottomrule
    \end{tabular}
      \label{tbl:case-competition-only-int}
  \end{table}

  \onslide<2>{
    \begin{table}[H]
      \center
      \caption{Old High German pattern}
      \begin{tabular}{c|c|c|c}
        \toprule
        \textsubscript{\tsc{int}} \textsuperscript{\tsc{ext}}
               & [\tsc{nom}]
               & [\tsc{acc}]
               & [\tsc{dat}]
               \\ \cmidrule{1-4}
           [\tsc{nom}]
               & \tsc{nom}
               & \cellcolor{LG}\tsc{acc}
               & \cellcolor{LG}\tsc{dat}
               \\ \cmidrule{1-4}
           [\tsc{acc}]
               & \cellcolor{DG}\tsc{acc}
               & \tsc{acc}
               & \cellcolor{LG}\tsc{dat}
               \\ \cmidrule{1-4}
           [\tsc{dat}]
               & \cellcolor{DG}\tsc{dat}
               & \cellcolor{DG}\tsc{dat}
               & \tsc{dat}
               \\
         \bottomrule
      \end{tabular}
        \label{tbl:case-competition-int-ext}
    \end{table}
  }

  \begin{table}[H]
    \center
    \caption{Polish pattern}
    \begin{tabular}{c|c|c|c}
      \toprule
      \textsubscript{\tsc{int}} \textsuperscript{\tsc{ext}}
             & [\tsc{nom}]
             & [\tsc{acc}]
             & [\tsc{dat}]
             \\ \cmidrule{1-4}
         [\tsc{nom}]
             & \tsc{nom}
             & \cellcolor{LG}*
             & \cellcolor{LG}*
             \\ \cmidrule{1-4}
         [\tsc{acc}]
             & \cellcolor{DG}*
             & \tsc{acc}
             & \cellcolor{LG}*
             \\ \cmidrule{1-4}
         [\tsc{dat}]
             & \cellcolor{DG}*
             & \cellcolor{DG}*
             & \tsc{dat}
             \\
       \bottomrule
    \end{tabular}
      \label{tbl:case-competition-none}
  \end{table}

  \pause
  \pause

\end{multicols}

\vspace{-10em}

\begin{table}[H]
  \center
  \begin{tabular}{lc}
    \toprule
                      & \tsc{int} = allowed to surface  \\
       \cmidrule(lr){2-2}
       Modern German  & yes                             \\
       Polish         & no                              \\
     \bottomrule
  \end{tabular}
    \label{tbl:case-competition-none}
\end{table}

\vspace{1em}

\pause
  Borer-Chomsky Conjecture: the lexicon is the source of language variation

\end{frame}


\begin{frame}[t]{Assumptions}

\begin{itemize}
  \item \onslide<2->{headless relatives are derived from light-headed relatives, headed by a special type of light head}
  \only<3>{\\
  light head\scsub{ext} [relative pronoun\scsub{int} ... ]\label{ex:light+rel}
  }
  \onslide<4->{\\
  \sout{light head\scsub{ext}} [relative pronoun\scsub{int} ... ]\label{ex:light+rel}
  }
  \item \onslide<5->{deletion takes place when the relative pronoun contains the light head as a single constituent}
  \item \onslide<6->{the relative pronoun contains the features of the light head plus an additional one}\\
  \onslide<6->{
  \begin{figure}[H]
    \center
    \begin{tabular}[b]{ccc}
        \toprule
        light head & & relative pronoun \\
        \cmidrule(lr){1-1} \cmidrule(lr){3-3}
        \begin{forest} boom
        [\tsc{k}P,
            [\tsc{k}]
            [ϕP, baseline]
        ]
        \end{forest}
        & \phantom{x} &
      \begin{forest} boom
        [\tsc{rel}P
            [\tsc{rel}]
            [\tsc{k}P
                [\tsc{k}]
                [ϕP, baseline]
            ]
        ]
      \end{forest}\\
        \bottomrule
    \end{tabular}
    \label{fig:rel-lh-intonly-1}
  \end{figure}
  }
\end{itemize}

\onslide<7->{
lexical entries → internal syntax → containment → deletion → headless relative
}

\end{frame}


\begin{frame}[t]{Light head and relative pronoun in Modern German}

\pause

\begin{multicols}{2}

  \begin{table}[H]
    \center
    \begin{tabular}{c|c|c|c}
      \toprule
      \textsubscript{\tsc{int}} \textsuperscript{\tsc{ext}}
             & [\tsc{nom}]
             & [\tsc{acc}]
             & [\tsc{dat}]
             \\ \cmidrule{1-4}
         [\tsc{nom}]
             & \tsc{nom}
             & \cellcolor{LG}*
             & \cellcolor{LG}*
             \\ \cmidrule{1-4}
         [\tsc{acc}]
             & \cellcolor{DG}\tsc{acc}
             & \tsc{acc}
             & \cellcolor{LG}*
             \\ \cmidrule{1-4}
         [\tsc{dat}]
             & \cellcolor{DG}\tsc{dat}
             & \cellcolor{DG}\tsc{\textcolor<3->{red}{dat}}
             & \tsc{dat}
             \\
       \bottomrule
    \end{tabular}
      \label{tbl:case-competition-only-int}
  \end{table}

\vspace{2em}

\onslide<5->{

\center
lexicon

\vspace{1em}

\begin{adjustbox}{max width=0.5\textwidth}

\begin{forest} boom
  [\tsc{acc}P,baseline
      [\tsc{k}2]
      [\tsc{nom}P
          [\tsc{k}1]
          [ϕP
              [\phantom{xxx}, roof]
          ]
      ]
  ]
  {\draw (.east) node[right]{⇔ \tit{n}}; }
\end{forest}

\begin{forest} boom
  [\tsc{dat}P,baseline
      [\tsc{k}3]
      [\tsc{acc}P
          [\tsc{k}2]
          [\tsc{nom}P
              [\tsc{k}1]
              [ϕP
                  [\phantom{xxx}, roof]
              ]
          ]
      ]
  ]
  {\draw (.east) node[right]{⇔ \tit{m}}; }
\end{forest}

\begin{forest} boom
  [\tsc{rel}P,baseline
      [\phantom{xxx}, roof]
  ]
  {\draw (.east) node[right]{⇔ \tit{we}}; }
\end{forest}

\end{adjustbox}

}

\onslide<4->{

  \begin{table}[H]
    \center
    \caption{Modern German \tsc{lh} and \tsc{rp}}
    \begin{tabular}{cc}
    \toprule
    \tsc{lh} & \tsc{rp} \\
    \cmidrule{1-2}
    n        & we-m     \\
    \bottomrule
    \end{tabular}
  \end{table}

}

\vspace{3em}

\onslide<6->{
  \begin{figure}[H]
    \begin{adjustbox}{max height=0.3\textheight}
    \centering
    \begin{tabular}[b]{ccc}
        \toprule
        light head & & relative pronoun \\
        \cmidrule(lr){1-1} \cmidrule(lr){3-3}
        \begin{forest} boom
          [\tsc{k}P,
          tikz={
          \node[label=below:\tit{n},
          draw,circle,
          scale=0.75,
          fit to=tree]{};
          }
              [\tsc{k}]
              [ϕP
                  [\phantom{xxx}, roof, baseline]
              ]
          ]
        \end{forest}
        & \phantom{x} &
        \begin{forest} boom
          [\tsc{rel}P, s sep=15mm
              [\tsc{rel}P,
              tikz={
              \node[label=below:\tit{we},
              draw,circle,
              scale=0.75,
              fit to=tree]{};
              }
                  [\phantom{xxx}, roof]
              ]
              [\tsc{k}P,
              tikz={
              \node[label=below:\tit{m},
              draw,circle,
              scale=0.75,
              fit to=tree]{};
              }
                  [\tsc{k}]
                  [ϕP
                      [\phantom{xxx}, roof, baseline]
                  ]
              ]
          ]
        \end{forest}\\
        \bottomrule
    \end{tabular}
  \end{adjustbox}
    \label{fig:rel-lh-mg}
  \end{figure}
}

\end{multicols}

\end{frame}


% \begin{frame}[t]{\tsc{acc}\scsub{ext} vs. \tsc{acc}\scsub{int} in Modern German}
%
% \pause
%
% \exg. Ich lade \sout{n} ein, \tbf{wen} \tbf{auch} \tbf{Maria} \tbf{mag}.\\
%  I invite\scsub{[acc]} \tsc{lh}.\tsc{acc} {} \tsc{rp}.\tsc{acc} also Maria like\scsub{[acc]}\\
%  `I invite who Maria also likes.' \flushfill{Modern German, adapted from \pgcitealt{vogel2001}{344}}\label{ex:mg-acc-acc-rep}
%
% \pause
%
% \begin{figure}[H]
%   \begin{adjustbox}{max height=0.5\textheight}
%   \centering
%     \begin{tabular}[b]{ccc}
%         \toprule
%         light head \tit{n} & & relative pronoun \tit{we-n} \\
%         \cmidrule(lr){1-1} \cmidrule(lr){3-3}
%         \begin{forest} boom
%           [\tsc{acc}P,
%           tikz={
%           \node[label=below:\tit{n},
%           draw,circle,
%           scale=0.8,
%           fit to=tree]{};
%           \onslide<4>{
%           \node[draw,circle,
%           dashed,
%           scale=0.85,
%           fill=DG,fill opacity=0.2,
%           fit to=tree]{};
%           }
%           }
%               [\tsc{k}2]
%               [\tsc{nomP}
%                   [\tsc{k}1]
%                   [ϕP
%                       [\phantom{xxx}, roof, baseline]
%                   ]
%               ]
%           ]
%         \end{forest}
%         & \phantom{x} &
%         \begin{forest} boom
%           [\tsc{rel}P
%               [\tsc{rel}P,
%               tikz={
%               \node[label=below:\tit{we},
%               draw,circle,
%               scale=0.75,
%               fit to=tree]{};
%               }
%                   [\phantom{xxx}, roof, baseline]
%               ]
%               [\tsc{acc}P,
%               tikz={
%               \onslide<4>{
%               \node[draw,circle,
%               dashed,
%               scale=0.85,
%               fit to=tree]{};
%               }
%               \node[label=below:\tit{n},
%               draw,circle,
%               scale=0.8,
%               fit to=tree]{};
%               }
%                   [\tsc{k}2]
%                   [\tsc{nomP}
%                       [\tsc{k}1]
%                       [ϕP
%                           [\phantom{xxx}, roof, baseline]
%                       ]
%                   ]
%               ]
%           ]
%         \end{forest}\\
%         \bottomrule
%     \end{tabular}
%   \end{adjustbox}
%   \end{figure}
%
%
% \end{frame}



% \begin{frame}[t]{\tsc{ext}\scsub{dat} vs. \tsc{int}\scsub{acc} ↛ \tit{m}/\tit{wen}}
%
% \pause
%
% \exg. *Ich vertraue \sout{m}, \tbf{wen} \tbf{auch} \tbf{Maria} \tbf{mag}.\\
% I trust\scsub{[dat]} \tsc{lh}.\tsc{dat} \tsc{rp}.\tsc{acc} also Maria like\scsub{[acc]}\\
% `I trust whoever Maria also likes.' \flushfill{Modern German, adapted from \pgcitealt{vogel2001}{345}}\label{ex:mg-dat-acc-rep-lh}
%
% \pause
%
% \begin{figure}[H]
%   \begin{adjustbox}{max height=0.5\textheight}
%   \centering
%     \begin{tabular}[b]{ccc}
%         \toprule
%         light head \tit{m} & & relative pronoun \tit{we-n} \\
%         \cmidrule(lr){1-1} \cmidrule(lr){3-3}
%         \begin{forest} boom
%           [\tsc{dat}P,
%           tikz={
%           \node[label=below:\tit{m},
%           draw,circle,
%           scale=0.9,
%           fit to=tree]{};
%           }
%               [\tsc{k}3]
%               [\tsc{acc}P,
%               tikz={
%               \onslide<4>{
%               \node[draw,circle,
%               dashed,
%               scale=0.85,
%               fit to=tree]{};
%               }
%               }
%                   [\tsc{k}2]
%                   [\tsc{nomP}
%                       [\tsc{k}1]
%                       [ϕP
%                           [\phantom{xxx}, roof, baseline]
%                       ]
%                   ]
%               ]
%           ]
%         \end{forest}
%         & \phantom{x} &
%         \begin{forest} boom
%           [\tsc{rel}P
%               [\tsc{rel}P,
%               tikz={
%               \node[label=below:\tit{we},
%               draw,circle,
%               scale=0.75,
%               fit to=tree]{};
%               }
%                   [\phantom{xxx}, roof, baseline]
%               ]
%               [\tsc{acc}P,
%               tikz={
%               \onslide<4>{
%               \node[draw,circle,
%               dashed,
%               scale=0.85,
%               fit to=tree]{};
%               }
%               \node[label=below:\tit{n},
%               draw,circle,
%               scale=0.8,
%               fit to=tree]{};
%               }
%                   [\tsc{k}2]
%                   [\tsc{nomP}
%                       [\tsc{k}1]
%                       [ϕP
%                           [\phantom{xxx}, roof, baseline]
%                       ]
%                   ]
%               ]
%           ]
%         \end{forest}\\
%         \bottomrule
%     \end{tabular}
%     \label{fig:acc-nom-intonly}
%   \end{adjustbox}
%   \end{figure}
%
% \end{frame}




\begin{frame}[t]{\tsc{acc}\scsub{ext} vs. \tsc{dat}\scsub{int} in Modern German}

\pause

\exg. Ich lade \sout{n} ein, \tbf{wem} \tbf{auch} \tbf{Maria} \tbf{vertraut}.\\
I invite\scsub{[acc]} \tsc{lh}.\tsc{acc} {} \tsc{rp}.\tsc{dat} also Maria trust\scsub{[dat]}\\
`I invite whoever Maria also trusts.' \flushfill{Modern German, adapted from \pgcitealt{vogel2001}{344}}\label{ex:mg-acc-dat-rep}

\pause

\begin{figure}[H]
  \begin{adjustbox}{max height=0.5\textheight}
  \centering
    \begin{tabular}[b]{ccc}
        \toprule
        light head \tit{n} & & relative pronoun \tit{we-m}\\
        \cmidrule(lr){1-1} \cmidrule(lr){3-3}
        \begin{forest} boom
          [\tsc{acc}P,
          tikz={
          \onslide<4>{
          \node[draw,circle,
          dashed,
          scale=0.85,
          fill=DG,fill opacity=0.2,
          fit to=tree]{};
          }
          \node[label=below:\tit{n},
          draw,circle,
          scale=0.8,
          fit to=tree]{};
          }
              [\tsc{k}2]
              [\tsc{nomP}
                  [\tsc{k}1]
                  [ϕP
                      [\phantom{xxx}, roof, baseline]
                  ]
              ]
          ]
        \end{forest}
        & \phantom{x} &
        \begin{forest} boom
          [\tsc{rel}P, s sep =15mm
              [\tsc{rel}P,
              tikz={
              \node[label=below:\tit{we},
              draw,circle,
              scale=0.75,
              fit to=tree]{};
              }
                  [\phantom{xxx}, roof, baseline]
              ]
              [\tsc{dat}P,
              tikz={
              \node[label=below:\tit{m},
              draw,circle,
              scale=0.9,
              fit to=tree]{};
              }
                  [\tsc{k}3]
                  [\tsc{acc}P,
                  tikz={
                  \onslide<4>{
                  \node[draw,circle,
                  dashed,
                  scale=0.85,
                  fit to=tree]{};
                  }
                  }
                  [\tsc{k}2]
                      [\tsc{nomP}
                          [\tsc{k}1]
                          [ϕP
                              [\phantom{xxx}, roof, baseline]
                          ]
                      ]
                  ]
              ]
          ]
        \end{forest}\\
        \bottomrule
    \end{tabular}
    \label{fig:nom-acc-intonly}
  \end{adjustbox}
  \end{figure}

\end{frame}


\begin{frame}[t]{Light head and relative pronoun in Polish}

\pause

\begin{multicols}{2}

\begin{table}[H]
  \center
  \begin{tabular}{c|c|c|c}
    \toprule
    \textsubscript{\tsc{int}} \textsuperscript{\tsc{ext}}
           & [\tsc{nom}]
           & [\tsc{acc}]
           & [\tsc{dat}]
           \\ \cmidrule{1-4}
       [\tsc{nom}]
           & \tsc{nom}
           & \cellcolor{LG}*
           & \cellcolor{LG}*
           \\ \cmidrule{1-4}
       [\tsc{acc}]
           & \cellcolor{DG}*
           & \tsc{acc}
           & \cellcolor{LG}*
           \\ \cmidrule{1-4}
       [\tsc{dat}]
           & \cellcolor{DG}*
           & \cellcolor{DG}\textcolor<3->{red}{*}
           & \tsc{dat}
           \\
     \bottomrule
  \end{tabular}
    \label{tbl:case-competition-none}
\end{table}

\vspace{2em}

\onslide<5->{

\center
lexicon

\vspace{1em}

\begin{adjustbox}{max width=0.4\textwidth}

\begin{forest} boom
  [ϕP,baseline
      [\phantom{xxx}, roof]
  ]
  {\draw (.east) node[right]{⇔ \tit{o}}; }
\end{forest}

\begin{forest} boom
  [\tsc{acc}P,baseline
      [\tsc{k}2]
      [\tsc{nom}P
          [\tsc{k}1]
      ]
  ]
  {\draw (.east) node[right]{⇔ \tit{go}}; }
\end{forest}

\begin{forest} boom
  [\tsc{dat}P,baseline
      [\tsc{k}3]
      [\tsc{acc}P
          [\tsc{k}2]
          [\tsc{nom}P
              [\tsc{k}1]
          ]
      ]
  ]
  {\draw (.east) node[right]{⇔ \tit{mu}}; }
\end{forest}

\begin{forest} boom
  [\tsc{rel}P,baseline
      [\phantom{xxx}, roof]
  ]
  {\draw (.east) node[right]{⇔ \tit{k}}; }
\end{forest}

\end{adjustbox}

}

\onslide<4->{

  \begin{table}[H]
    \center
    \caption{Polish \tsc{lh} and \tsc{rp}}
    \begin{tabular}{cc}
    \toprule
     \tsc{lh} & \tsc{rp} \\
     \cmidrule{1-2}
     o-go     & k-o-mu   \\
    \bottomrule
    \end{tabular}
  \end{table}

}

\vspace{3em}

\onslide<6->{
\begin{figure}[H]
    \begin{adjustbox}{max height=0.3\textheight}
  \centering
  \begin{tabular}[b]{ccc}
      \toprule
      light head & & relative pronoun \\
      \cmidrule(lr){1-1} \cmidrule(lr){3-3}
      \begin{forest} boom
      [\tsc{k}P, s sep = 15 mm
          [ϕP,
          tikz={
          \node[label=below:\tit{o},
          draw,circle,
          scale=0.85,
          fit to=tree]{};
          }
              [\phantom{xxx}, roof]
          ]
          [\tsc{k}P,
          tikz={
          \node[label=below:\tit{go},
          draw,circle,
          scale=0.85,
          fit to=tree]{};
          }
              [\tsc{k}, baseline]
          ]
      ]
      \end{forest}
      & \phantom{x} &
    \begin{forest} boom
      [\tsc{rel}P
          [\tsc{rel}P,
          tikz={
          \node[label=below:\tit{k},
          draw,circle,
          scale=0.85,
          fit to=tree]{};
          }
              [\phantom{xxx}, roof, baseline]
          ]
          [\tsc{k}P, s sep = 15 mm
              [ϕP,
              tikz={
              \node[label=below:\tit{o},
              draw,circle,
              scale=0.85,
              fit to=tree]{};
              }
                  [\phantom{xxx}, roof]
              ]
              [\tsc{k}P,
              tikz={
              \node[label=below:\tit{mu},
              draw,circle,
              scale=0.85,
              fit to=tree]{};
              }
                  [\tsc{k1}, baseline]
              ]
          ]
      ]
    \end{forest}\\
      \bottomrule
  \end{tabular}
  \label{fig:rel-lh-matching}
\end{adjustbox}
\end{figure}
}

\end{multicols}

\end{frame}





\begin{frame}[t]{Comparing Polish to Modern German}

\pause

  \begin{figure}[H]
      \begin{adjustbox}{max height=0.4\textheight}
    \centering
    \begin{tabular}[b]{ccc}
        \toprule
        light head & & relative pronoun \\
        \cmidrule(lr){1-1} \cmidrule(lr){3-3}
        \begin{forest} boom
        [\tsc{k}P, s sep = 15 mm
            [ϕP,
            tikz={
            \node[label=below:\tit{o},
            draw,circle,
            scale=0.85,
            fit to=tree]{};
            }
                [\phantom{xxx}, roof]
            ]
            [\tsc{k}P,
            tikz={
            \node[label=below:\tit{go},
            draw,circle,
            scale=0.85,
            fit to=tree]{};
            }
                [\tsc{k}, baseline]
            ]
        ]
        \end{forest}
        & \phantom{x} &
      \begin{forest} boom
        [\tsc{rel}P
            [\tsc{rel}P,
            tikz={
            \node[label=below:\tit{k},
            draw,circle,
            scale=0.85,
            fit to=tree]{};
            }
                [\phantom{xxx}, roof, baseline]
            ]
            [\tsc{k}P, s sep = 15 mm
                [ϕP,
                tikz={
                \node[label=below:\tit{o},
                draw,circle,
                scale=0.85,
                fit to=tree]{};
                }
                    [\phantom{xxx}, roof]
                ]
                [\tsc{k}P,
                tikz={
                \node[label=below:\tit{mu},
                draw,circle,
                scale=0.85,
                fit to=tree]{};
                }
                    [\tsc{k1}, baseline]
                ]
            ]
        ]
      \end{forest}\\
        \bottomrule
    \end{tabular}
    \label{fig:rel-lh-matching}
  \end{adjustbox}
  \end{figure}


  \begin{figure}[H]
      \begin{adjustbox}{max height=0.4\textheight}
    \centering
    \begin{tabular}[b]{ccc}
        \toprule
        light head & & relative pronoun \\
        \cmidrule(lr){1-1} \cmidrule(lr){3-3}
        \begin{forest} boom
          [\tsc{k}P,
          tikz={
          \node[label=below:\tit{n},
          draw,circle,
          scale=0.75,
          fit to=tree]{};
          }
              [\tsc{k}]
              [ϕP
                  [\phantom{xxx}, roof, baseline]
              ]
          ]
        \end{forest}
        & \phantom{x} &
        \begin{forest} boom
          [\tsc{rel}P, s sep=15mm
              [\tsc{rel}P,
              tikz={
              \node[label=below:\tit{we},
              draw,circle,
              scale=0.75,
              fit to=tree]{};
              }
                  [\phantom{xxx}, roof]
              ]
              [\tsc{k}P,
              tikz={
              \node[label=below:\tit{m},
              draw,circle,
              scale=0.75,
              fit to=tree]{};
              }
                  [\tsc{k}]
                  [ϕP
                      [\phantom{xxx}, roof, baseline]
                  ]
              ]
          ]
        \end{forest}\\
        \bottomrule
    \end{tabular}
    \label{fig:rel-lh-mg}
  \end{adjustbox}
  \end{figure}


\end{frame}


% \begin{frame}[t]{\tsc{acc}\scsub{ext} vs. \tsc{acc}\scsub{int} in Polish}
%
% \pause
%
% \exg. Jan lubi \sout{ogo} \tbf{kogo} \tbf{-kolkwiek} \tbf{Maria} \tbf{lubi}.\\
% Jan like\scsub{[acc]} \tsc{lh}.\tsc{acc}  \tsc{rp}.\tsc{acc} ever Maria like\scsub{[acc]}\\
% `Jan likes whoever Maria likes.' \flushfill{Polish, adapted from \citealt{citko2013} after \pgcitealt{himmelreich2017}{17}}\label{ex:polish-acc-acc-rep}
%
% \pause
%
%   \begin{figure}[H]
%     \begin{adjustbox}{max height=0.5\textheight}
%     \centering
%     \begin{tabular}[b]{ccc}
%       \toprule
%       light head \tit{o-go} & & relative pronoun \tit{k-o-go} \\
%       \cmidrule(lr){1-1} \cmidrule(lr){3-3}
%       \begin{forest} boom
%         [\tsc{acc}P,
%         tikz={
%         \onslide<4>{
%         \node[
%         draw, circle,
%         fill=DG,fill opacity=0.2,
%         scale=0.95,
%         yshift=-0.5cm,
%         dashed,
%         fit to=tree]{};
%         }
%         }
%             [ϕP,
%             tikz={
%             \node[label=below:\tit{o},
%             draw,circle,
%             scale=0.85,
%             fit to=tree]{};
%             }
%                 [\phantom{xxx}, roof]
%             ]
%             [\tsc{acc}P,
%             tikz={
%             \node[label=below:\tit{go},
%             draw,circle,
%             scale=0.9,
%             fit to=tree]{};
%             }
%                 [\tsc{k}2]
%                 [\tsc{nom}P
%                     [\tsc{k}1]
%                 ]
%             ]
%         ]
%       \end{forest}
%       & \phantom{x} &
%       \begin{forest} boom
%         [\tsc{rel}P
%             [\tsc{rel}P,
%             tikz={
%             \node[label=below:\tit{k},
%             draw,circle,
%             scale=0.85,
%             fit to=tree]{};
%             }
%                 [\phantom{xxx}, roof]
%             ]
%             [\tsc{acc}P,
%             tikz={
%             \onslide<4>{
%             \node[
%             draw, circle,
%             scale=0.95,
%             yshift=-0.5cm,
%             dashed,
%             fit to=tree]{};
%             }
%             }
%                 [ϕP,
%                 tikz={
%                 \node[label=below:\tit{o},
%                 draw,circle,
%                 scale=0.85,
%                 fit to=tree]{};
%                 }
%                     [\phantom{xxx}, roof]
%                 ]
%                 [\tsc{acc}P,
%                 tikz={
%                 \node[label=below:\tit{go},
%                 draw,circle,
%                 scale=0.9,
%                 fit to=tree]{};
%                 }
%                     [\tsc{k}2]
%                     [\tsc{nom}P
%                         [\tsc{k}1]
%                     ]
%                 ]
%             ]
%         ]
%       \end{forest}\\
%       \bottomrule
%     \end{tabular}
%   \end{adjustbox}
%    \label{fig:nom-nom-matching}
%   \end{figure}
%
% \end{frame}



% \begin{frame}[t]{\tsc{ext}\scsub{dat} vs. \tsc{int}\scsub{acc} ↛ \tit{omu}/\tit{kogo} in Polish}
%
% \pause
%
% \exg. *Jan ufa \sout{omu} \tbf{kogo} \tbf{-kolkwiek} \tbf{wpuścil} \tbf{do} \tbf{domu}.\\
% Jan trust\scsub{[dat]} \tsc{elh}.\tsc{dat} \tsc{rp}.\tsc{acc} ever let\scsub{[acc]} to home\\
% `Jan trusts whoever he let into the house.' \flushfill{Polish, adapted from \citealt{citko2013} after \pgcitealt{himmelreich2017}{17}}\label{ex:polish-dat-acc-rel}
%
% \pause
%
%   \begin{figure}[H]
%     \begin{adjustbox}{max height=0.5\textheight}
%     \centering
%     \begin{tabular}[b]{ccc}
%       \toprule
%       light head \tit{o-mu} & & relative pronoun \tit{k-o-go} \\
%       \cmidrule(lr){1-1} \cmidrule(lr){3-3}
%       \begin{forest} boom
%         [\tsc{dat}P
%             [\tsc{ϕ}P,
%             tikz={
%             \onslide<4>{
%             \node[
%             draw,circle,
%             scale=0.85,
%             dashed,
%             fit to=tree]{};
%             }
%             \node[label=below:\tit{o},
%             draw,circle,
%             scale=0.85,
%             fit to=tree]{};
%             }
%                 [\phantom{xxx}, roof]
%             ]
%             [\tsc{dat}P,
%             tikz={
%             \node[label=below:\tit{mu},
%             draw,circle,
%             scale=0.95,
%             fit to=tree]{};
%             }
%                 [\tsc{k}3]
%                 [\tsc{acc}P,
%                 tikz={
%                 \onslide<4>{
%                 \node[
%                 draw,circle,
%                 scale=0.9,
%                 dashed,
%                 fit to=tree]{};
%                 }
%                 }
%                     [\tsc{k}2]
%                     [\tsc{nom}P
%                         [\tsc{k}1]
%                     ]
%                 ]
%             ]
%         ]
%         \end{forest}
%       & \phantom{x} &
%       \begin{forest} boom
%         [\tsc{rel}P
%             [\tsc{rel}P,
%             tikz={
%             \node[label=below:\tit{k},
%             draw,circle,
%             scale=0.85,
%             fit to=tree]{};
%             }
%                 [\phantom{xxx}, roof]
%             ]
%             [\tsc{acc}P
%                 [\tsc{ϕ}P,
%                 tikz={
%                 \onslide<4>{
%                 \node[
%                 draw,circle,
%                 scale=0.85,
%                 dashed,
%                 fit to=tree]{};
%                 }
%                 \node[label=below:\tit{o},
%                 draw,circle,
%                 scale=0.85,
%                 fit to=tree]{};
%                 }
%                     [\phantom{xxx}, roof]
%                 ]
%                 [\tsc{acc}P,
%                 tikz={
%                 \node[label=below:\tit{go},
%                 draw,circle,
%                 scale=0.9,
%                 fit to=tree]{};
%                 \onslide<4>{
%                 \node[
%                 draw,circle,
%                 scale=0.95,
%                 dashed,
%                 fit to=tree]{};
%                 }
%                 }
%                     [\tsc{k}2]
%                     [\tsc{nom}P
%                         [\tsc{k}1]
%                     ]
%                 ]
%             ]
%         ]
%       \end{forest}\\
%       \bottomrule
%     \end{tabular}
%   \end{adjustbox}
%    \label{fig:nom-acc-matching}
%   \end{figure}
%
% \end{frame}




\begin{frame}[t]{\tsc{acc}\scsub{ext} vs. \tsc{dat}\scsub{int} in Polish}

\pause

\exg. *Jan lubi \sout{ogo} \tbf{komu} \tbf{-kolkwiek} \tbf{dokucza}.\\
Jan like\scsub{[acc]} \tsc{lh}.\tsc{acc} \tsc{rp}.\tsc{dat} ever tease\scsub{[dat]}\\
`Jan likes whoever he teases.' \flushfill{Polish, adapted from \citealt{citko2013} after \pgcitealt{himmelreich2017}{17}}\label{ex:polish-acc-dat-rel}

\pause

  \begin{figure}[H]
    \begin{adjustbox}{max height=0.5\textheight}
    \centering
    \begin{tabular}[b]{ccc}
      \toprule
      light head \tit{o-go} & & relative pronoun \tit{k-o-mu}\\
      \cmidrule(lr){1-1} \cmidrule(lr){3-3}
      \begin{forest} boom
          [\tsc{acc}P, s sep= 12mm
              [\tsc{ϕ}P,
              tikz={
              \onslide<4>{
              \node[
              draw,circle,
              scale=0.9,
              dashed,
              fit to=tree]{};
              }
              \node[label=below:\tit{o},
              draw,circle,
              scale=0.85,
              fit to=tree]{};
              }
                  [\phantom{xxx}, roof]
              ]
              [\tsc{acc}P,
              tikz={
              \node[label=below:\tit{go},
              draw,circle,
              scale=0.9,
              fit to=tree]{};
              \onslide<4>{
              \node[
              draw,circle,
              scale=0.95,
              dashed,
              fit to=tree]{};
              }
              }
                  [\tsc{k}2]
                  [\tsc{nom}P
                      [\tsc{k}1]
                  ]
              ]
          ]
        \end{forest}
      & \phantom{x} &
      \begin{forest} boom
        [\tsc{rel}P
            [\tsc{rel}P,
            tikz={
            \node[label=below:\tit{k},
            draw,circle,
            scale=0.85,
            fit to=tree]{};
            }
                [\phantom{xxx}, roof]
            ]
            [\tsc{dat}P, s sep = 15 mm
                [\tsc{ϕ}P,
                tikz={
                \onslide<4>{
                \node[
                draw,circle,
                scale=0.9,
                dashed,
                fit to=tree]{};
                }
                \node[label=below:\tit{o},
                draw,circle,
                scale=0.85,
                fit to=tree]{};
                }
                    [\phantom{xxx}, roof]
                ]
                [\tsc{dat}P,
                tikz={
                \node[label=below:\tit{mu},
                draw,circle,
                scale=0.95,
                fit to=tree]{};
                }
                    [\tsc{k}3]
                    [\tsc{acc}P, tikz={
                    \onslide<4>{
                    \node[
                    draw,circle,
                    scale=0.9,
                    dashed,
                    fit to=tree]{};
                    }
                    }
                        [\tsc{k}2]
                        [\tsc{nom}P
                            [\tsc{k}1]
                        ]
                    ]
                ]
            ]
        ]
      \end{forest}\\
      \bottomrule
    \end{tabular}
  \end{adjustbox}
   \label{fig:nom-acc-matching}
  \end{figure}

\end{frame}




\begin{frame}[t]{\tsc{int}/\tsc{ext}}

  \pause

  \begin{table}[H]
    \center
    \begin{tabular}{lcc}
      \toprule
                        & \tsc{int} = allowed to surface  & \onslide<3-4>{ϕ + \tsc{k}}         \\
         \cmidrule(lr){2-3}
         Modern German  & yes                             & \onslide<3-4>{portmanteau}         \\
         Polish         & no                              & \onslide<3-4>{separate morphemes}  \\
       \bottomrule
    \end{tabular}
      \label{tbl:case-competition-none}
  \end{table}

  \vspace{1em}

  \pause
  \pause

\begin{multicols}{2}

  \begin{figure}[H]
    \begin{adjustbox}{max height=0.26\textheight}
    \centering
      \begin{tabular}[b]{ccc}
          \toprule
          light head \tit{n} & & relative pronoun \tit{we-m}\\
          \cmidrule(lr){1-1} \cmidrule(lr){3-3}
          \begin{forest} boom
            [\tsc{acc}P,
            tikz={
            \node[draw,circle,
            dashed,
            scale=0.85,
            fill=DG,fill opacity=0.2,
            fit to=tree]{};
            \node[label=below:\tit{n},
            draw,circle,
            scale=0.8,
            fit to=tree]{};
            }
                [\tsc{k}2]
                [\tsc{nomP}
                    [\tsc{k}1]
                    [ϕP
                        [\phantom{xxx}, roof, baseline]
                    ]
                ]
            ]
          \end{forest}
          & \phantom{x} &
          \begin{forest} boom
            [\tsc{rel}P, s sep =15mm
                [\tsc{rel}P,
                tikz={
                \node[label=below:\tit{we},
                draw,circle,
                scale=0.75,
                fit to=tree]{};
                }
                    [\phantom{xxx}, roof, baseline]
                ]
                [\tsc{dat}P,
                tikz={
                \node[label=below:\tit{m},
                draw,circle,
                scale=0.9,
                fit to=tree]{};
                }
                    [\tsc{k}3]
                    [\tsc{acc}P,
                    tikz={
                    \node[draw,circle,
                    dashed,
                    scale=0.85,
                    fit to=tree]{};
                    }
                    [\tsc{k}2]
                        [\tsc{nomP}
                            [\tsc{k}1]
                            [ϕP
                                [\phantom{xxx}, roof, baseline]
                            ]
                        ]
                    ]
                ]
            ]
          \end{forest}\\
          \bottomrule
      \end{tabular}
      \label{fig:nom-acc-intonly}
    \end{adjustbox}
    \end{figure}

  \begin{figure}[H]
    \begin{adjustbox}{max height=0.26\textheight}
    \centering
    \begin{tabular}[b]{ccc}
      \toprule
      light head \tit{o-go} & & relative pronoun \tit{k-o-mu}\\
      \cmidrule(lr){1-1} \cmidrule(lr){3-3}
      \begin{forest} boom
          [\tsc{acc}P, s sep = 17.5mm
              [\tsc{ϕ}P,
              tikz={
              \node[
              draw,circle,
              scale=0.9,
              dashed,
              fit to=tree]{};
              \node[label=below:\tit{o},
              draw,circle,
              scale=0.85,
              fit to=tree]{};
              }
                  [\phantom{xxx}, roof]
              ]
              [\tsc{acc}P,
              tikz={
              \node[label=below:\tit{go},
              draw,circle,
              scale=0.9,
              fit to=tree]{};
              \node[
              draw,circle,
              scale=0.95,
              dashed,
              fit to=tree]{};
              }
                  [\tsc{k}2]
                  [\tsc{nom}P
                      [\tsc{k}1]
                  ]
              ]
          ]
        \end{forest}
      & \phantom{x} &
      \begin{forest} boom
        [\tsc{rel}P, s sep=12.5 mm
            [\tsc{rel}P,
            tikz={
            \node[label=below:\tit{k},
            draw,circle,
            scale=0.85,
            fit to=tree]{};
            }
                [\phantom{xxx}, roof]
            ]
            [\tsc{dat}P, s sep=17.5 mm
                [\tsc{ϕ}P,
                tikz={
                \node[
                draw,circle,
                scale=0.9,
                dashed,
                fit to=tree]{};
                \node[label=below:\tit{o},
                draw,circle,
                scale=0.85,
                fit to=tree]{};
                }
                    [\phantom{xxx}, roof]
                ]
                [\tsc{dat}P,
                tikz={
                \node[label=below:\tit{mu},
                draw,circle,
                scale=0.95,
                fit to=tree]{};
                }
                    [\tsc{k}3]
                    [\tsc{acc}P, tikz={
                    \node[
                    draw,circle,
                    scale=0.9,
                    dashed,
                    fit to=tree]{};
                    }
                        [\tsc{k}2]
                        [\tsc{nom}P
                            [\tsc{k}1]
                        ]
                    ]
                ]
            ]
        ]
      \end{forest}\\
      \bottomrule
    \end{tabular}
  \end{adjustbox}
   \label{fig:nom-acc-matching}
  \end{figure}

\end{multicols}

\end{frame}


\begin{frame}[t]{Conclusion}

\pause

two factors influence grammaticality

\pause

\begin{itemize}
  \item {the case of the relative pronoun}
  \begin{itemize}
      {\item the winner is determined by \tbf{\tsc{nom} < \tsc{acc} < \tsc{dat}}}
      {\item is stable across languages}
  \end{itemize}
\end{itemize}

\pause

  \center
  follows from how case is organized in syntax [[[\tsc{nom}]\tsc{acc}]\tsc{dat}]

  winning case contains losing case

\vspace{1em}

\pause

\begin{itemize}
  \item {where the winning case comes from}
  \begin{itemize}
      {\item \tbf{\tsc{int}/\tsc{ext}} case is allowed to surface when it wins}
      {\item differs across languages}
  \end{itemize}
\end{itemize}

\pause

  \center
  \tsc{int} is allowed to surface =

  ϕ+\tsc{k} portmanteau =

  \tsc{rp} contains \tsc{lh} as a single constituent

\end{frame}


\begin{frame}[t]{References}

Citko, Barbara (2013). “Size matters: Multidominance and DP structure in Polish”. In: \tit{Talk at the th Poznan Linguistic Meeting}.\\
Daskalaki, Evangelia (2011). “Case Mis-Matching as Kase Stranding”. In: \tit{University of Pennsylvania Working Papers in Linguistics.} Ed. by Lauren A. Friedman. Vol. 17. Philadelphia: Penn Linguistics Club, pp. 77–86.\\
Harbert, Wayne Eugene (1978). “Gothic syntax: a relational grammar”. PhD thesis. Urbana-Champaign.\\
Himmelreich, Anke (2017). “Case Matching Effects in Free Relatives and Parasitic Gaps: A Study on the Properties of Agree”. PhD thesis. Universität Leipzig.\\
Vogel, Ralf (2001). “Case Conflict in Modern German Free Relative Constructions: An Optimality Theoretic Treatment”. In: \tit{Competition in Syntax.} Ed. by Gereon Müller and Wolfgang Sternefeld. Berlin: Mouton de Gruyter, pp. 341–375. doi: 10.1515/9783110829068.341.

\end{frame}


\end{document}
