% !TEX root = thesis.tex

\chapter{Case decomposition meets ellipsis}

The problem: so far people that account for headless relatives have made reference to this case hierarchy. they put them in their OT tables, let the fly in from the left in their syntax, whatever. What I want to do is unify all the instances of nom-acc-dat. I put nom-acc-dat in syntax. which is morphology.


\section{Problem with previous analyses of headless relatives}



The problem: so far people that account for headless relatives have made reference to this case hierachy. they put them in their OT tables, let the fly in from the left in their syntax, whatever.

What I do is start is start from morphology. There we have complex case: dat - acc - nom.
What we see in syntax is a by-product of the morphology, it´s a consequence, it´s an indirect relation. cause and effect
if the morphology is different, than so will the syntax



\section{Morphology}

\subsection{Case decomposition}

morphological containment

how can we account for that? well, all these morphemes have their own heads

\ex.
\begin{forest} boom
  [\ac{dat}P
      [\ac{dat}]
      [\ac{acc}P
          [\ac{acc}]
          [\ac{nom}P
              [\ac{nom}]
              [NP
                  [..., roof]
              ]
          ]
      ]
  ]
\end{forest}


\exg. phral -és -kə\\
 brother.\ac{nom} -\ac{acc} -\ac{dat}\\ \flushfill{Kalderaš Romani, \pgcitealt{boretzky1994}{31-46}}


\ex.
\begin{forest} boom
[\tsc{datP}
    [\ac{acc}P,name=accp2, s sep=20mm
        [\tsc{nomP},name=nomp,
        tikz={
        \node[label=below right:\tit{phral},
        draw,circle,
        xscale=0.8,yscale=1,
        fit=(nomp)(nom)(brother)(np)]{};
        }
            [\ac{nom},name=nom]
            [NP,name=np
                [\tsc{brother},name=brother, roof],baseline
            ]
        ]
        [\ac{acc}P,name=accp,
        tikz={
        \node[label={below right:\tit{-és}},
        draw,circle,
        xscale=0.7,yscale=0.9,
        fit=(acc)(accp)]{};
        }
         [\ac{acc},name=acc]
        ]
    ]
    [\tsc{datP},name=datp,
    tikz={
    \node[label={below right:\tit{-kə}},
    draw,circle,
    xscale=0.7,yscale=0.9,
    fit=(dat)(datp)]{};
    }
        [\ac{dat},name=dat]
    ]
]
\end{forest}

how to deal with it that \tit{phral} spells out brother and nominative, so spells out phrases already..

\ex.
\a.
\begin{forest} boom
  [\ac{nom}P
      [\ac{nom}]
      [NP
          [\tsc{brother}, roof]
      ]
  ]
  {\draw (.east) node[right]{⇔ \tit{phral}}; }
\end{forest}
\b. \begin{forest} boom
  [\ac{acc}P
      [\ac{acc}]
  ]
  {\draw (.east) node[right]{⇔ \tit{-és}}; }
\end{forest}
\b. \begin{forest} boom
  [\ac{dat}P
      [\ac{dat}]
  ]
  {\draw (.east) node[right]{⇔ \tit{-kə}}; }
\end{forest}



this is how this works in nanosyntax






\subsection{Phrasal spellout}

but what about the syncretism patterns?

we want to start out with the same syntax

with phrasal spellout, we spell out multiple heads at once

this is how that works in nanosyntax









\section{Ellipsis}

Ellipsis targets phrases

it does not delete elements one by one



\section{Reflex of morphology in syntax}


\subsection{Morphology}

\ex.
\begin{forest} boom
  [\tsc{datP},name=datp,
  tikz={
  \node[draw,circle,LG,
  xscale=0.8,yscale=1,
  fill opacity=0.2,
  fill=LG,
  fit=(datp)(dat)(nom)(x)]{};
  }
      [\ac{dat},name=dat]
        [\ac{acc}P,name=accp,
        tikz={
        \node[draw,circle,
        xscale=0.75,yscale=0.95,
        fill opacity=0.2,
        fill=DG,DG,
        fit=(accp)(acc)(nom)(np)(x)]{};
        }
          [\ac{acc},name=acc]
          [\tsc{nomP},name=nomp
              [\ac{nom},name=nom]
              [NP,name=np
                  [...,name=x, roof]
              ]
          ]
      ]
  ]
\end{forest}


\ex.
\begin{forest} boom
  [\tsc{datP},name=datp,
  tikz={
  \node[draw,circle,LG,
  xscale=0.8,yscale=1,
  fill opacity=0.2,
  fill=LG,
  fit=(datp)(dat)(nom)(x)]{};
  }
      [\ac{dat},name=dat]
      [\ac{acc}P,name=accp
          [\ac{acc},name=acc]
          [\tsc{nomP},name=nomp,
          tikz={
          \node[draw,circle,DG,
          xscale=0.7,yscale=0.9,
          fill opacity=0.2,
          fill=DG,
          fit=(nomp)(nom)(np)(x)]{};
          }
              [\ac{nom},name=nom]
              [NP,name=np
                  [...,name=x, roof]
              ]
          ]
      ]
  ]
\end{forest}



\ex.
\begin{forest} boom
      [\ac{acc}P,name=accp,
      tikz={
      \node[draw,circle,
      xscale=0.8,yscale=1,
      fill opacity=0.2,
      fill=LG,LG,
      fit=(accp)(acc)(nom)(x)]{};
      }
          [\ac{acc},name=acc]
          [\tsc{nomP},name=nomp,
          tikz={
          \node[draw,circle,DG,
          xscale=0.75,yscale=0.95,
          fill opacity=0.2,
          fill=DG,
          fit=(nomp)(nom)(x)]{};
          }
              [\ac{nom},name=nom]
              [NP,name=np
                  [...,name=x, roof]
              ]
          ]
      ]
  ]
\end{forest}



\subsection{Syntax}

\begin{table}[H]
  \center
	\caption {\ac{dat}P deletes \ac{acc}P}
		\begin{tabular}[b]{c c}
      \begin{forest} boom
        [\tsc{datP}
            [\ac{dat}]
              [\ac{acc}P,name=accp,
              tikz={
              \node[draw,circle,
              xscale=0.775,yscale=0.975,
              fit=(accp)(acc)(nom)(x)]{};
              }
                [\ac{acc},name=acc]
                [\tsc{nomP},name=nomp
                    [\ac{nom},name=nom]
                    [NP,name=np
                        [...,name=x, roof ,baseline]
                    ]
                ]
            ]
        ]
      \end{forest}
      &
      \begin{forest} boom
        [\textcolor{LG}{\tsc{accP}},name=accp,
        tikz={
        \node[draw,circle,
        xscale=0.775,yscale=0.975,
        fit=(accp)(acc)(nom)(x)]{};
        }
            [\textcolor{LG}{\ac{acc}},name=acc,edge=LG]
            [\textcolor{LG}{\tsc{nomP}},name=nomp,edge=LG
                [\textcolor{LG}{\ac{nom}},name=nom,edge=LG]
                [\textcolor{LG}{NP},name=np,edge=LG
                    [\textcolor{LG}{...},name=x,
                    roof, baseline, edge=LG
                    ]
                ]
            ]
        ]
      \end{forest} \\
  \end{tabular}
\end{table}

\begin{table}[H]
  \center
	\caption {\ac{dat}P deletes \ac{nom}P}
		\begin{tabular}[b]{cc}
      \begin{forest} boom
        [\tsc{datP}
            [\ac{dat}]
              [\ac{acc}P,name=accp
                [\ac{acc},name=acc]
                [\tsc{nomP},name=nomp,
                tikz={
                \node[draw,circle,
                xscale=0.75,yscale=0.95,
                fit=(nomp)(nom)(x)]{};
                }
                    [\ac{nom},name=nom]
                    [NP,name=np
                        [...,name=x, roof, baseline]
                    ]
                ]
            ]
        ]
      \end{forest}
      &
      \begin{forest} boom
        [\textcolor{LG}{\tsc{nomP}},name=nomp,
        tikz={
        \node[draw,circle,
        xscale=0.75,yscale=0.95,
        fit=(nomp)(nom)(x)]{};
        }
            [\textcolor{LG}{\ac{nom}},name=nom,edge=LG]
            [\textcolor{LG}{NP},name=np,baseline,edge=LG
                [\textcolor{LG}{...},name=x,
                roof,baseline,edge=LG
                ]
            ]
        ]
      \end{forest} \\
  \end{tabular}
\end{table}

\begin{table}[H]
  \center
	\caption {\ac{acc}P deletes \ac{nom}P}
		\begin{tabular}[b]{cc}
      \begin{forest} boom
          [\ac{acc}P,name=accp
              [\ac{acc},name=acc]
              [\tsc{nomP},name=nomp,
              tikz={
              \node[draw,circle,
              xscale=0.75,yscale=0.95,
              fit=(nomp)(nom)(x)]{};
              }
                  [\ac{nom},name=nom]
                  [NP,name=np
                      [...,name=x,
                      roof,baseline
                      ]
                  ]
              ]
          ]
      \end{forest}
      &
      \begin{forest} boom
        [\textcolor{LG}{\tsc{nomP}},name=nomp,
        tikz={
        \node[draw,circle,
        xscale=0.75,yscale=0.95,
        fit=(nomp)(nom)(x)]{};
        }
            [\textcolor{LG}{\ac{nom}},name=nom,
            edge=LG]
            [\textcolor{LG}{NP},name=np,
            edge=LG
                [\textcolor{LG}{...},name=x,
                roof,baseline,edge=LG
                ]
            ]
        ]
      \end{forest}\\
  \end{tabular}
\end{table}





\section{Similar analyses}

Himmelreich
