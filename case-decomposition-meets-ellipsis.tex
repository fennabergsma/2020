% !TEX root = thesis.tex

\chapter{Case decomposition meets ellipsis}

The problem: so far people that account for headless relatives have made reference to this case hierarchy. they put them in their OT tables, let the fly in from the left in their syntax, whatever. What I want to do is unify all the instances of nom-acc-dat. I put nom-acc-dat in syntax. which is morphology.


\section{Problem with previous analyses of headless relatives}



The problem: so far people that account for headless relatives have made reference to this case hierachy. they put them in their OT tables, let the fly in from the left in their syntax, whatever.

What I do is start is start from morphology. There we have complex case: dat - acc - nom.
What we see in syntax is a by-product of the morphology, it´s a consequence, it´s an indirect relation. cause and effect
if the morphology is different, than so will the syntax



\section{Morphology}

\subsection{Case decomposition}

\ex.
\begin{forest} boom
  [\ac{dat}P
      [\ac{dat}]
      [\ac{acc}P
          [\ac{acc}]
          [\ac{nom}P
              [\ac{nom}]
              [X]
          ]
      ]
  ]
\end{forest}


morphological containment





\subsection{Phrasal spellout}

Single morphemes spell out phrases

suppletion and syncretism







\section{Ellipsis}

Ellipsis targets phrases



\section{Relation between morphology and syntax}





\section{The intuition}

\subsection{Morphology}

\ex.
\begin{forest} boom
  [\tsc{datP},name=datp,
  tikz={
  \node[draw,circle,LG,
  xscale=0.9,yscale=0.9,
  fill opacity=0.2,
  fill=LG,
  fit=(datp)(dat)(nom)(x)]{};
  }
      [\tsc{dat},name=dat]
        [\tsc{acc}P,name=accp,
        tikz={
        \node[draw,circle,
        xscale=0.87,yscale=0.87,
        fill opacity=0.2,
        fill=DG,DG,
        fit=(accp)(acc)(nom)(x)]{};
        }
          [\tsc{acc},name=acc]
          [\tsc{nomP},name=nomp
              [\tsc{nom},name=nom]
              [XP,name=x
              ]
          ]
      ]
  ]
\end{forest}


\ex.
\begin{forest} boom
  [\tsc{datP},name=datp,
  tikz={
  \node[draw,circle,LG,
  xscale=0.93,yscale=0.93,
  fill opacity=0.2,
  fill=LG,
  fit=(datp)(dat)(nom)(x)]{};
  }
      [\tsc{dat},name=dat]
      [\tsc{acc}P,name=accp
          [\tsc{acc},name=acc]
          [\tsc{nomP},name=nomp,
          tikz={
          \node[draw,circle,DG,
          xscale=0.85,yscale=0.85,
          fill opacity=0.2,
          fill=DG,
          fit=(nomp)(nom)(x)]{};
          }
              [\tsc{nom},name=nom]
              [XP,name=x
              ]
          ]
      ]
  ]
\end{forest}



\ex.
\begin{forest} boom
      [\tsc{acc}P,name=accp,
      tikz={
      \node[draw,circle,
      xscale=0.9,yscale=0.9,
      fill opacity=0.2,
      fill=LG,LG,
      fit=(accp)(acc)(nom)(x)]{};
      }
          [\tsc{acc},name=acc]
          [\tsc{nomP},name=nomp,
          tikz={
          \node[draw,circle,DG,
          xscale=0.85,yscale=0.85,
          fill opacity=0.2,
          fill=DG,
          fit=(nomp)(nom)(x)]{};
          }
              [\tsc{nom},name=nom]
              [XP,name=x
              ]
          ]
      ]
  ]
\end{forest}



\subsection{Syntax}

\begin{table}[H]
  \center
	\caption {\tsc{dat}P deletes \tsc{acc}P}
		\begin{tabular}[b]{c c}
      \begin{forest} boom
        [\tsc{datP}
            [\tsc{dat}]
              [\tsc{acc}P,name=accp,
              tikz={
              \node[draw,circle,
              xscale=0.87,yscale=0.87,
              fit=(accp)(acc)(nom)(x)]{};
              }
                [\tsc{acc},name=acc]
                [\tsc{nomP},name=nomp
                    [\tsc{nom},name=nom]
                    [XP,name=x,baseline
                    ]
                ]
            ]
        ]
      \end{forest}
      &
      \begin{forest} boom
        [\textcolor{LG}{\tsc{accP}},name=accp,baseline,
        tikz={
        \node[draw,circle,
        xscale=0.87,yscale=0.87,
        fit=(accp)(acc)(nom)(x)]{};
        }
            [\textcolor{LG}{\tsc{acc}},name=acc,edge=LG]
            [\textcolor{LG}{\tsc{nomP}},name=nomp,edge=LG
                [\textcolor{LG}{\tsc{nom}},name=nom,edge=LG]
                [\textcolor{LG}{XP},name=x,baseline,edge=LG]
            ]
        ]
      \end{forest} \\
  \end{tabular}
\end{table}

\begin{table}[H]
  \center
	\caption {\tsc{dat}P deletes \tsc{nom}P}
		\begin{tabular}[b]{cc}
      \begin{forest} boom
        [\tsc{datP}
            [\tsc{dat}]
              [\tsc{acc}P,name=accp
                [\tsc{acc},name=acc]
                [\tsc{nomP},name=nomp,
                tikz={
                \node[draw,circle,
                xscale=0.87,yscale=0.87,
                fit=(nomp)(nom)(x)]{};
                }
                    [\tsc{nom},name=nom]
                    [XP,name=x,baseline
                    ]
                ]
            ]
        ]
      \end{forest}
      &
      \begin{forest} boom
        [\textcolor{LG}{\tsc{nomP}},name=nomp,
        tikz={
        \node[draw,circle,
        xscale=0.87,yscale=0.87,
        fit=(nomp)(nom)(x)]{};
        }
            [\textcolor{LG}{\tsc{nom}},name=nom,edge=LG]
            [\textcolor{LG}{XP},name=x,baseline,edge=LG]
        ]
      \end{forest} \\
  \end{tabular}
\end{table}

\begin{table}[H]
  \center
	\caption {\tsc{acc}P deletes \tsc{nom}P}
		\begin{tabular}[b]{cc}
      \begin{forest} boom
          [\tsc{acc}P,name=accp
              [\tsc{acc},name=acc]
              [\tsc{nomP},name=nomp,
              tikz={
              \node[draw,circle,
              xscale=0.87,yscale=0.87,
              fit=(nomp)(nom)(x)]{};
              }
                  [\tsc{nom},name=nom]
                  [XP,name=x,baseline]
              ]
          ]
      \end{forest}
      &
      \begin{forest} boom
        [\textcolor{LG}{\tsc{nomP}},name=nomp,
        tikz={
        \node[draw,circle,
        xscale=0.87,yscale=0.87,
        fit=(nomp)(nom)(x)]{};
        }
            [\textcolor{LG}{\tsc{nom}},name=nom,
            edge=LG]
            [\textcolor{LG}{XP},name=x,baseline,
            edge=LG]
        ]
      \end{forest}\\
  \end{tabular}
\end{table}





\section{Similar analyses}

Himmelreich
