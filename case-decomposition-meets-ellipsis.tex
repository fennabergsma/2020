% !TEX root = thesis.tex

\chapter{Case decomposition meets ellipsis}

At the beginning of the previous chapter I discussed the case scale active in headless relatives: \tsc{nom < acc < dat}. In most accounts for headless relatives (\citealt[cf.][]{pittner1995,vogel2001,grosu2003,harbert1978}, an exception to this is \citealt{himmelreich2017}) the case scale is stipulated. Headless relatives simply obey to that hierarchy.

``One of the reviewers notes that an explanation in terms of a Case hierarchy is rather stipulative. However, as far as I know, nobody has suggested a nonstipulative explanation for these facts.'' \pgcitep{pittner1995}{201} (footnote 4)

What I showed as well in the previous chapter is the pattern \tsc{nom < acc < dat} is reccurring. It can be observed in at least two more syntactic phenomena: agreement en relativization.\footnote{
In this dissertation I do not work out an account for these two syntactic phenomena. They merely serve as an illustration that the pattern is reflected in different syntactic phenomena.}
We also see it in morphology with syncretism patterns and formal containment. Morphology was also mentioned by Pittner.

``Furthermore, the Case hierarchies receive some independent support by morphology as shown by the various inflectional paradigms.'' \pgcitep{pittner1995}{201} (footnote 4)

Do we have an abstract hierarchy that we make reference to time and time again? In an ideal analysis, the hierarchy exists once and all the other occurrences are reflexes of the first one. That also means we want morphology and syntax be connected.

That's exactly what I'm going to propose. (this intuition has been worked out in a different way in Himmelreich's dissertation.)

First I show where the case hierarchy is, which is in morphology, with actual structural containment: \tsc{[[[nom]acc]dat]}. What we see in syntax is a by-product of the morphology, it's a consequence, it's an indirect relation. cause and effect


\section{Case decomposition}

The claim: case should be decomposed

(\citealt{caha2009,caha2013} and later \citealt[cf.][]{starke2009,bobaljik2012,mcfadden2018,smith2019,vanbaal2018})

\begin{table}[h]
  \center
	\caption {Decomposed cases}
		\begin{tabular}{ll}
    \toprule
    case      & features                      \\
    \midrule
    \tsc{nom} & \tsc{f}1                      \\
    \tsc{acc} & \tsc{f}1, \tsc{f}2            \\
    \tsc{dat} & \tsc{f}1, \tsc{f}2, \tsc{f}3  \\
    \bottomrule
    \end{tabular}
\end{table}

With that in place, case containment and syncretism does not longer come as a surprise.

The intuition

containment:

\exg. phral -és -kə\\
 brother.\ac{nom} -\ac{acc} -\ac{dat}\\ \flushfill{Kalderaš Romani, \pgcitealt{boretzky1994}{31-46}}

syncretism:

contigent zones, no ABA. spellout is not only exact match, but also a subset of the features can be a match


I show how this can be derived exactly, within Nanosyntax, the framework which this proposal works in.


\subsection{Case containment}

in general the organization of nano

insert figure with syntax, lexicon, pf and cf


only syntax can make complex syntactic structures

The specific implications of this claim are strengthened by the way Nanosyntax
treats complex feature structures. In particular, the core assumption is that
the only way how a complex feature structure can arise is for it to be assembled
by Merge. Complex feature structures are never the input to syntax, they are
always its product. Syntactic trees are therefore not assembled from some preexisting
‘feature bundles’ or any similar complex building blocks (lexical items).
The idea is that syntax starts from single, atomic, indivisible grammatical features,
ideally the same for all languages (cf. Cinque \& Rizzi 2008).


what's in the lexion?

Following Starke (2014), I understand lexical entries as nothing else but links
between well-formed syntactic representations, well formed phonological representations,
and/or conceptual representations. One of the motivations for adopting
this view is to have a ‘principled’ theory of the lexicon. What ‘principled’
means is not that lexical items are no longer arbitrary associations between syntax,
phonology and/or conceptual meaning; they still are. Rather, ‘principled’
is used in the sense that the format of the lexical item is restricted. Since wellformed
syntactic structures are constrained by principles, and lexical items link
such representations to their pronunciation (a phonological representation), it
follows that the representations in a lexical entry are constrained by universal
principles (the same ones that regulate what syntactic trees look like).


case tree

\ex.
\begin{forest} boom
  [\ac{dat}P
      [\ac{dat}]
      [\ac{acc}P
          [\ac{acc}]
          [\ac{nom}P
              [\ac{nom}]
              [NP
                  [..., roof]
              ]
          ]
      ]
  ]
\end{forest}

now let's look at the case decomposition example again

\begin{table}[ht]
  \center
  \caption {Containment pattern}
    % !TEX root = ../thesis.tex

\begin{tabular}{cl}
\toprule
          & 3\ac{sg} \\
          \cmidrule{2-2}
\ac{nom}
          & luw                                     \\
\ac{acc}  & luw\tbf{-e:l}                           \\
\ac{dat}  & luw\tbf{-e:l}\tcol{DG}{\tbf{-na}}       \\
\bottomrule
\end{tabular}

  \label{tbl:containment-derive}
\end{table}

first the nominative
\footnote{
I address the issue of phrasal spellout in the next section.
}

\ex.
\begin{forest} boom
  [\ac{nom}P
      [\ac{nom}]
      [NP
          [\tsc{brother}, roof]
      ]
  ]
  {\draw (.east) node[right]{⇔ \tit{phral}}; }
\end{forest}

\ex. \begin{forest} boom
[\tsc{nomP},name=nomp,
tikz={
\node[label=below right:\tit{phral},
draw,circle,
xscale=0.8,yscale=1,
fit=(nomp)(nom)(brother)(np)]{};
}
    [\ac{nom},name=nom]
    [NP,name=np
        [\tsc{brother},name=brother, roof],baseline
    ]
]
\end{forest}

then the accusative
\footnote{
I will come back to the movement later.
}

\ex. \begin{forest} boom
  [\ac{acc}P
      [\ac{acc}]
  ]
  {\draw (.east) node[right]{⇔ \tit{-és}}; }
\end{forest}

\ex. \begin{forest} boom
[\ac{acc}P,name=accp2, s sep=20mm
    [\tsc{nomP},name=nomp,
    tikz={
    \node[label=below right:\tit{phral},
    draw,circle,
    xscale=0.8,yscale=1,
    fit=(nomp)(nom)(brother)(np)]{};
    }
        [\ac{nom},name=nom]
        [NP,name=np
            [\tsc{brother},name=brother, roof],baseline
        ]
    ]
    [\ac{acc}P,name=accp,
    tikz={
    \node[label={below right:\tit{-és}},
    draw,circle,
    xscale=0.7,yscale=0.9,
    fit=(acc)(accp)]{};
    }
     [\ac{acc},name=acc]
    ]
]
\end{forest}

then dative


\ex. \begin{forest} boom
  [\ac{dat}P
      [\ac{dat}]
  ]
  {\draw (.east) node[right]{⇔ \tit{-kə}}; }
\end{forest}


\ex.
\begin{forest} boom
[\tsc{datP}
    [\ac{acc}P,name=accp2, s sep=20mm
        [\tsc{nomP},name=nomp,
        tikz={
        \node[label=below right:\tit{phral},
        draw,circle,
        xscale=0.8,yscale=1,
        fit=(nomp)(nom)(brother)(np)]{};
        }
            [\ac{nom},name=nom]
            [NP,name=np
                [\tsc{brother},name=brother, roof],baseline
            ]
        ]
        [\ac{acc}P,name=accp,
        tikz={
        \node[label={below right:\tit{-és}},
        draw,circle,
        xscale=0.7,yscale=0.9,
        fit=(acc)(accp)]{};
        }
         [\ac{acc},name=acc]
        ]
    ]
    [\tsc{datP},name=datp,
    tikz={
    \node[label={below right:\tit{-kə}},
    draw,circle,
    xscale=0.7,yscale=0.9,
    fit=(dat)(datp)]{};
    }
        [\ac{dat},name=dat]
    ]
]
\end{forest}



\subsection{Phrasal spellout}

basic background on matching in nano

\ex. \tbf{The Superset Principle} \citet{starke2009}:\\
A lexically stored tree matches a syntactic node iff the lexically stored tree contains the syntactic node.

What the Superset Principle states is that matching between a lexical tree and a syntactic tree does not require full identity, but ‘only’ containment.

\ex. \tbf{The Elsewhere Condition} (\citealt{kiparsky1973}, formulated as in \citealt{caha2019}):\\
When two entries can spell out a given node, the more specific entry wins. Under the Superset Principle governed insertion, the more specific entry is the one which has fewer unused features.

\begin{table}[ht]
  \center
  \caption {Syncretism pattern}
    % !TEX root = ../thesis.tex

\begin{tabular}{cccccccc}
  \toprule
      \multicolumn{3}{c}{pattern}
        & \ac{nom}
        & \ac{acc}
        & \ac{dat}
        & translation
        & language \\
  \cmidrule(lr){1-3} \cmidrule(lr){4-6} \cmidrule(lr){7-7} \cmidrule(lr){8-8}
      A & A & A
        & \cellcolor{LG}jullie
        & \cellcolor{LG}jullie
        & \cellcolor{LG}jullie
        & 2\ac{pl}
        & Dutch \\
      A & A & B
        & \cellcolor{LG}sie
        & \cellcolor{LG}sie
        & ihr
        & 3\ac{sg}.\ac{f}
        & German \\
      A & B & B
        & við
        & \cellcolor{LG}okkur
        & \cellcolor{LG}okkur
        & 1\ac{pl}
        & Icelandic \\
      A & B & C
        & tú
        & teg
        & tær
        & 2\ac{sg}
        & Faroese \\
      A & B & A
        & \cellcolor{LG}
        &
        & \cellcolor{LG}
        &
        & not attested \\
  \bottomrule
\end{tabular}

  \label{tbl:syncretisms-derive}
\end{table}


but what about the syncretism patterns?

we want to start out with the same syntax

with phrasal spellout, we spell out multiple heads at once

this is how that works in nanosyntax









\section{Ellipsis}

Ellipsis targets phrases

it does not delete elements one by one



\section{Reflex of morphology in syntax}


\subsection{Morphology}

\ex.
\begin{forest} boom
  [\tsc{datP},name=datp,
  tikz={
  \node[draw,circle,LG,
  xscale=0.8,yscale=1,
  fill opacity=0.2,
  fill=LG,
  fit=(datp)(dat)(nom)(x)]{};
  }
      [\ac{dat},name=dat]
        [\ac{acc}P,name=accp,
        tikz={
        \node[draw,circle,
        xscale=0.75,yscale=0.95,
        fill opacity=0.2,
        fill=DG,DG,
        fit=(accp)(acc)(nom)(np)(x)]{};
        }
          [\ac{acc},name=acc]
          [\tsc{nomP},name=nomp
              [\ac{nom},name=nom]
              [NP,name=np
                  [...,name=x, roof]
              ]
          ]
      ]
  ]
\end{forest}


\ex.
\begin{forest} boom
  [\tsc{datP},name=datp,
  tikz={
  \node[draw,circle,LG,
  xscale=0.8,yscale=1,
  fill opacity=0.2,
  fill=LG,
  fit=(datp)(dat)(nom)(x)]{};
  }
      [\ac{dat},name=dat]
      [\ac{acc}P,name=accp
          [\ac{acc},name=acc]
          [\tsc{nomP},name=nomp,
          tikz={
          \node[draw,circle,DG,
          xscale=0.7,yscale=0.9,
          fill opacity=0.2,
          fill=DG,
          fit=(nomp)(nom)(np)(x)]{};
          }
              [\ac{nom},name=nom]
              [NP,name=np
                  [...,name=x, roof]
              ]
          ]
      ]
  ]
\end{forest}



\ex.
\begin{forest} boom
      [\ac{acc}P,name=accp,
      tikz={
      \node[draw,circle,
      xscale=0.8,yscale=1,
      fill opacity=0.2,
      fill=LG,LG,
      fit=(accp)(acc)(nom)(x)]{};
      }
          [\ac{acc},name=acc]
          [\tsc{nomP},name=nomp,
          tikz={
          \node[draw,circle,DG,
          xscale=0.75,yscale=0.95,
          fill opacity=0.2,
          fill=DG,
          fit=(nomp)(nom)(x)]{};
          }
              [\ac{nom},name=nom]
              [NP,name=np
                  [...,name=x, roof]
              ]
          ]
      ]
  ]
\end{forest}



\subsection{Syntax}

\begin{table}[H]
  \center
	\caption {\ac{dat}P deletes \ac{acc}P}
		\begin{tabular}[b]{c c}
      \begin{forest} boom
        [\tsc{datP}
            [\ac{dat}]
              [\ac{acc}P,name=accp,
              tikz={
              \node[draw,circle,
              xscale=0.775,yscale=0.975,
              fit=(accp)(acc)(nom)(x)]{};
              }
                [\ac{acc},name=acc]
                [\tsc{nomP},name=nomp
                    [\ac{nom},name=nom]
                    [NP,name=np
                        [...,name=x, roof ,baseline]
                    ]
                ]
            ]
        ]
      \end{forest}
      &
      \begin{forest} boom
        [\textcolor{LG}{\tsc{accP}},name=accp,
        tikz={
        \node[draw,circle,
        xscale=0.775,yscale=0.975,
        fit=(accp)(acc)(nom)(x)]{};
        }
            [\textcolor{LG}{\ac{acc}},name=acc,edge=LG]
            [\textcolor{LG}{\tsc{nomP}},name=nomp,edge=LG
                [\textcolor{LG}{\ac{nom}},name=nom,edge=LG]
                [\textcolor{LG}{NP},name=np,edge=LG
                    [\textcolor{LG}{...},name=x,
                    roof, baseline, edge=LG
                    ]
                ]
            ]
        ]
      \end{forest} \\
  \end{tabular}
\end{table}

\begin{table}[H]
  \center
	\caption {\ac{dat}P deletes \ac{nom}P}
		\begin{tabular}[b]{cc}
      \begin{forest} boom
        [\tsc{datP}
            [\ac{dat}]
              [\ac{acc}P,name=accp
                [\ac{acc},name=acc]
                [\tsc{nomP},name=nomp,
                tikz={
                \node[draw,circle,
                xscale=0.75,yscale=0.95,
                fit=(nomp)(nom)(x)]{};
                }
                    [\ac{nom},name=nom]
                    [NP,name=np
                        [...,name=x, roof, baseline]
                    ]
                ]
            ]
        ]
      \end{forest}
      &
      \begin{forest} boom
        [\textcolor{LG}{\tsc{nomP}},name=nomp,
        tikz={
        \node[draw,circle,
        xscale=0.75,yscale=0.95,
        fit=(nomp)(nom)(x)]{};
        }
            [\textcolor{LG}{\ac{nom}},name=nom,edge=LG]
            [\textcolor{LG}{NP},name=np,baseline,edge=LG
                [\textcolor{LG}{...},name=x,
                roof,baseline,edge=LG
                ]
            ]
        ]
      \end{forest} \\
  \end{tabular}
\end{table}

\begin{table}[H]
  \center
	\caption {\ac{acc}P deletes \ac{nom}P}
		\begin{tabular}[b]{cc}
      \begin{forest} boom
          [\ac{acc}P,name=accp
              [\ac{acc},name=acc]
              [\tsc{nomP},name=nomp,
              tikz={
              \node[draw,circle,
              xscale=0.75,yscale=0.95,
              fit=(nomp)(nom)(x)]{};
              }
                  [\ac{nom},name=nom]
                  [NP,name=np
                      [...,name=x,
                      roof,baseline
                      ]
                  ]
              ]
          ]
      \end{forest}
      &
      \begin{forest} boom
        [\textcolor{LG}{\tsc{nomP}},name=nomp,
        tikz={
        \node[draw,circle,
        xscale=0.75,yscale=0.95,
        fit=(nomp)(nom)(x)]{};
        }
            [\textcolor{LG}{\ac{nom}},name=nom,
            edge=LG]
            [\textcolor{LG}{NP},name=np,
            edge=LG
                [\textcolor{LG}{...},name=x,
                roof,baseline,edge=LG
                ]
            ]
        ]
      \end{forest}\\
  \end{tabular}
\end{table}





\section{Similar analyses}

Himmelreich
