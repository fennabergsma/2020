% !TEX root = thesis.tex

\chapter{Case decomposition meets ellipsis}

The problem: so far people that account for headless relatives have made reference to this case hierachy. they put them in their OT tables, let the fly in from the left in their syntax, whatever. What I want to do is unify all the instances of nom-acc-dat. I put nom-acc-dat in syntax. which is morphology.


\section{Case decomposition}




Syntax = morphology

\ex.
\begin{forest} boom
  [\ac{dat}P
      [\ac{dat}]
      [\ac{acc}P
          [\ac{acc}]
          [\ac{nom}P
              [\ac{nom}]
              [X]
          ]
      ]
  ]
\end{forest}


\section{Elipsis}

Elipsis targets phrases




\section{Phrasal spellout}

Single morphemes spell out phrases


\section{The intuition}

\subsection{Cases contain each other}

\ex.
\begin{forest} boom
  [\tsc{datP},name=datp,
  tikz={
  \node[draw,circle,LG,
  xscale=0.9,yscale=0.9,
  fill opacity=0.2,
  fill=LG,
  fit=(datp)(dat)(nom)(x)]{};
  }
      [\tsc{dat},name=dat]
        [\tsc{acc}P,name=accp,
        tikz={
        \node[draw,circle,
        xscale=0.87,yscale=0.87,
        fill opacity=0.2,
        fill=DG,DG,
        fit=(accp)(acc)(nom)(x)]{};
        }
          [\tsc{acc},name=acc]
          [\tsc{nomP},name=nomp
              [\tsc{nom},name=nom]
              [XP,name=x
              ]
          ]
      ]
  ]
\end{forest}


\ex.
\begin{forest} boom
  [\tsc{datP},name=datp,
  tikz={
  \node[draw,circle,LG,
  xscale=0.93,yscale=0.93,
  fill opacity=0.2,
  fill=LG,
  fit=(datp)(dat)(nom)(x)]{};
  }
      [\tsc{dat},name=dat]
      [\tsc{acc}P,name=accp
          [\tsc{acc},name=acc]
          [\tsc{nomP},name=nomp,
          tikz={
          \node[draw,circle,DG,
          xscale=0.85,yscale=0.85,
          fill opacity=0.2,
          fill=DG,
          fit=(nomp)(nom)(x)]{};
          }
              [\tsc{nom},name=nom]
              [XP,name=x
              ]
          ]
      ]
  ]
\end{forest}



\ex.
\begin{forest} boom
      [\tsc{acc}P,name=accp,
      tikz={
      \node[draw,circle,
      xscale=0.9,yscale=0.9,
      fill opacity=0.2,
      fill=LG,LG,
      fit=(accp)(acc)(nom)(x)]{};
      }
          [\tsc{acc},name=acc]
          [\tsc{nomP},name=nomp,
          tikz={
          \node[draw,circle,DG,
          xscale=0.85,yscale=0.85,
          fill opacity=0.2,
          fill=DG,
          fit=(nomp)(nom)(x)]{};
          }
              [\tsc{nom},name=nom]
              [XP,name=x
              ]
          ]
      ]
  ]
\end{forest}



\subsection{Cases elide each other}

\begin{table}[H]
  \center
	\caption {\tsc{dat}P deletes \tsc{acc}P}
		\begin{tabular}[b]{c c}
      \begin{forest} boom
        [\tsc{datP}
            [\tsc{dat}]
              [\tsc{acc}P,name=accp,
              tikz={
              \node[draw,circle,
              xscale=0.87,yscale=0.87,
              fit=(accp)(acc)(nom)(x)]{};
              }
                [\tsc{acc},name=acc]
                [\tsc{nomP},name=nomp
                    [\tsc{nom},name=nom]
                    [XP,name=x,baseline
                    ]
                ]
            ]
        ]
      \end{forest}
      &
      \begin{forest} boom
        [\textcolor{LG}{\tsc{accP}},name=accp,baseline,
        tikz={
        \node[draw,circle,
        xscale=0.87,yscale=0.87,
        fit=(accp)(acc)(nom)(x)]{};
        }
            [\textcolor{LG}{\tsc{acc}},name=acc,edge=LG]
            [\textcolor{LG}{\tsc{nomP}},name=nomp,edge=LG
                [\textcolor{LG}{\tsc{nom}},name=nom,edge=LG]
                [\textcolor{LG}{XP},name=x,baseline,edge=LG]
            ]
        ]
      \end{forest} \\
  \end{tabular}
\end{table}

\begin{table}[H]
  \center
	\caption {\tsc{dat}P deletes \tsc{nom}P}
		\begin{tabular}[b]{cc}
      \begin{forest} boom
        [\tsc{datP}
            [\tsc{dat}]
              [\tsc{acc}P,name=accp
                [\tsc{acc},name=acc]
                [\tsc{nomP},name=nomp,
                tikz={
                \node[draw,circle,
                xscale=0.87,yscale=0.87,
                fit=(nomp)(nom)(x)]{};
                }
                    [\tsc{nom},name=nom]
                    [XP,name=x,baseline
                    ]
                ]
            ]
        ]
      \end{forest}
      &
      \begin{forest} boom
        [\textcolor{LG}{\tsc{nomP}},name=nomp,
        tikz={
        \node[draw,circle,
        xscale=0.87,yscale=0.87,
        fit=(nomp)(nom)(x)]{};
        }
            [\textcolor{LG}{\tsc{nom}},name=nom,edge=LG]
            [\textcolor{LG}{XP},name=x,baseline,edge=LG]
        ]
      \end{forest} \\
  \end{tabular}
\end{table}

\begin{table}[H]
  \center
	\caption {\tsc{acc}P deletes \tsc{nom}P}
		\begin{tabular}[b]{cc}
      \begin{forest} boom
          [\tsc{acc}P,name=accp
              [\tsc{acc},name=acc]
              [\tsc{nomP},name=nomp,
              tikz={
              \node[draw,circle,
              xscale=0.87,yscale=0.87,
              fit=(nomp)(nom)(x)]{};
              }
                  [\tsc{nom},name=nom]
                  [XP,name=x,baseline]
              ]
          ]
      \end{forest}
      &
      \begin{forest} boom
        [\textcolor{LG}{\tsc{nomP}},name=nomp,
        tikz={
        \node[draw,circle,
        xscale=0.87,yscale=0.87,
        fit=(nomp)(nom)(x)]{};
        }
            [\textcolor{LG}{\tsc{nom}},name=nom,
            edge=LG]
            [\textcolor{LG}{XP},name=x,baseline,
            edge=LG]
        ]
      \end{forest}\\
  \end{tabular}
\end{table}
