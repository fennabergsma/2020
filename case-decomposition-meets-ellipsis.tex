% !TEX root = thesis.tex

\chapter{Case decomposition and ellipsis}

At the beginning of the previous chapter I showed that the case scale \tsc{nom < acc < dat} appears in headless relatives. In most accounts for headless relatives (\citealt[cf.][]{pittner1995,vogel2001,grosu2003,harbert1978}, an exception to this is \citealt{himmelreich2017}) the case scale is stipulated. Headless relatives simply obey to that hierarchy. \pgcitet{pittner1995}{201:fn.4} makes this explicit: ``One of the reviewers notes that an explanation in terms of a Case hierarchy is rather stipulative. However, as far as I know, nobody has suggested a nonstipulative explanation for these facts.''

What I showed as well in the previous chapter is the case scale \tsc{nom < acc < dat} is a wide-spread phenomenon: it recurs. The scale can be observed in at least two more syntactic phenomena: agreement en relativization.\footnote{
In this dissertation I do not work out accounts for these two syntactic phenomena. They merely serve as an illustration that the pattern is reflected in other syntactic phenomena as well.}
The case scale also appears within morphology in syncretism patterns and formal containment. \pgcitet{pittner1995}{201:fn.4} makes this link to morphology as well: ``Furthermore, the Case hierarchies receive some independent support by morphology as shown by the various inflectional paradigms.''

I am not after a theory in which the case hierarchy is not something construction specific, and syntax and morphology both have their own case hierarchy. I argue that there is a single trigger that is responsible for the case scale in different subparts of language \citep[cf.][on numeral constructions]{caha2019}. Specifically, I show that the observed case scale naturally follows on the assumption that the case hierarchy is deeply anchored in syntax. The case scale in morphology and syntax are merely reflexes of how case is organized in language.\footnote{
\citet{himmelreich2017} works this intuition out in a different way.
}

In this chapter I show that the case scale can be derived is case is decomposed. The precise facts in headless relatives relatives follow after adding an ellipsis that deletes phrases.


\section{Case decomposition}

The intuition: case is complex

containment:

\exg. luw -e:l -na\\
 3\tsc{sg}.\ac{nom} -\ac{acc} -\ac{dat}\\ \flushfill{Khanty, \pgcitealt{nikolaeva1999}{16}}

syncretism:

contigent zones, no ABA. spellout is not only exact match, but also a subset of the features can be a match

I show how this can be derived, within Nanosyntax, the framework in which I work this proposal out.



\subsection{The basic idea}

\citet{caha2009,caha2013} (followed by \citealt[cf.][]{starke2009,bobaljik2012,mcfadden2018,smith2019,vanbaal2018}) has extensively argued that case should be decomposed into privative features. Specifically, the decomposition is cumulative: each case has a different number of case features, and the number grows monotonically.
This is illustrated in Table \ref{tbl:case-decomposed}. Accusative has all the features that nominative has (here \tsc{f}1) plus one extra (here \tsc{f}2). Dative has all the features accusative has (\tsc{f} and \tsc{f}2) plus one extra (\tsc{f}3).

\begin{table}[ht]
  \center
	\caption {Case decomposed}
		\begin{tabular}{ll}
    \toprule
    case      & features                      \\
    \midrule
    \tsc{nom} & \tsc{f}1                      \\
    \tsc{acc} & \tsc{f}1, \tsc{f}2            \\
    \tsc{dat} & \tsc{f}1, \tsc{f}2, \tsc{f}3  \\
    \bottomrule
    \end{tabular}
    \label{tbl:case-decomposed}
\end{table}

The decomposition in Table \ref{tbl:case-decomposed} forms the basis to derive the case scale effects observed in the previous chapter. The next section shows how case containment and syncretism effects follow naturally. Later in this chapter I show how the decomposition also derives the case competition facts in headless relatives.

%should I talk about alternatives and why they do not work?


\subsection{Deriving syncretism}

In this section I show how case syncretism patterns can be derived from the case decomposition in Table \ref{tbl:case-decomposed}. I repeat an example that shows the possible and impossible syncretism patterns in Table \ref{tbl:syncretisms-derive}.

\begin{table}[ht]
  \center
  \caption {Syncretism pattern}
    % !TEX root = ../thesis.tex

\begin{tabular}{cccccccc}
  \toprule
      \multicolumn{3}{c}{pattern}
        & \ac{nom}
        & \ac{acc}
        & \ac{dat}
        & translation
        & language \\
  \cmidrule(lr){1-3} \cmidrule(lr){4-6} \cmidrule(lr){7-7} \cmidrule(lr){8-8}
      A & A & A
        & \cellcolor{LG}jullie
        & \cellcolor{LG}jullie
        & \cellcolor{LG}jullie
        & 2\ac{pl}
        & Dutch \\
      A & A & B
        & \cellcolor{LG}sie
        & \cellcolor{LG}sie
        & ihr
        & 3\ac{sg}.\ac{f}
        & German \\
      A & B & B
        & við
        & \cellcolor{LG}okkur
        & \cellcolor{LG}okkur
        & 1\ac{pl}
        & Icelandic \\
      A & B & C
        & tú
        & teg
        & tær
        & 2\ac{sg}
        & Faroese \\
      A & B & A
        & \cellcolor{LG}
        &
        & \cellcolor{LG}
        &
        & not attested \\
  \bottomrule
\end{tabular}

  \label{tbl:syncretisms-derive}
\end{table}

The intuition is the following.
%In Nanosyntax (the framework I shall be using in this study), lexical
% items apply to feature sets not based on identity, but on the basis of a containment
% relation.

 % The morphological form of the pronouns mirror the cumulative feature decomposition given in Table \ref{tbl:case-decomposed}. That is, the accusative has the morphology that the nominative has (\tit{luw}) plus something extra (\tit{-e:l}). The dative has the morphology that the accusative has (\tit{luw-e:l}) plus something extra (\tit{luw-e:l}-na). In what follows I make it concrete how syntactic features translate to morphological form.


The proposal in this dissertation is worked out in Nanosyntax. I provide background assumptions on the framework whenever necessary to follow the discussion as I proceed. At this point I discuss the general architecture of Nanosyntax, its lexicon, phrasal spellout and matching between syntax and the lexicon.

The architecture of Nanosyntax is schematically shown in Figure \ref{fig:nano} \citep[from][]{vandenwyngaerd2020,caha2019}. In Nanosyntax, syntax starts with atomic features, and it builds complex syntactic trees. Specifically, there are no `feature bundles' that enter the syntax. The only way complex feature structures come to exist is a a result of merge.
After syntax (actually, each instance of merge), the syntactic structure is matched against the lexicon for pronunciation. The lexicon `translates' between syntactic representations on the one hand and phonology (PF) and concepts (CF) on the other hand. So, Nanosyntax is a late insertion model: (lexical) insertion takes place late, namely after syntax.

\begin{figure}[ht]
  \centering
  \begin{tikzpicture}[node distance = 1cm, auto]
    \tikzstyle{block} = [rectangle, draw, text width=5em, text centered, rounded corners, minimum height=2em]
    \tikzstyle{line} = [draw, -latex']
      \node [block] (syntax) {Syntax};
      \node [block, below of=syntax, node distance=1.5cm] (lexicon) {Lexicon};
      \node [block, below left=0.5cm and -1cm of lexicon] (pf) {PF};
      \node [block, below right=0.5cm and -1cm of lexicon] (cf) {CF};
      \path [line] (syntax) -- (lexicon);
      \path [line] (lexicon) -- (pf);
      \path [line] (lexicon) -- (cf);
  \end{tikzpicture}
  \caption{Nanosyntactic model of grammar}
  \label{fig:nano}
\end{figure}

In Nanosyntax, the lexicon is nothing but links between syntactic representations, phonological representations and conceptual representations \citep{starke2014}.\footnote{
The syntactic representation does not have to correspond to both a phonological and a conceptual representation. Syntactic representation that only correspond to a conceptual representations and not to phonological representations are (phrasal or clausal) idioms. Syntactic representations that only correspond to phonological representations but not to conceptual representations are for instance irregular plurals.
}
I leave the conceptual representation out of discussion for now, as it not relevant for the discussion here. The fact that only syntax can create complex feature structures also has a consequence for the content of the lexicon.
Syntactic structures are constrained by certain principles, such that only well-formed syntactic structures exist. Since the lexical items from the lexicon link syntactic representation to phonological and conceptual representation, these syntactic representations are constrained by the same principles as syntactic trees are. As a result, the lexicon only contains well-formed syntactic structures. The lexicon does not contain unstructured `feature bundles', because they could never be created by syntax.

Following this logic, the syntactic representation for a lexical entry as in \ref{ex:feature-set} cannot exist. The feature bundle cannot have entered syntax, because syntax starts with atomic features. It can also not be created by syntax, because complex structures can only be created with merge.

\ex. [ \tsc{f}1, \tsc{f}2, \tsc{f}3 ]\label{ex:feature-set}

Instead, a possible syntactic representation for a lexical entry looks as in \ref{ex:feature-structure}. The features are merged one by one in a binary syntactic structure.

\ex. \begin{forest} boom
  [\ac{dat}P
      [\ac{f}3]
      [\ac{acc}P
          [\ac{f}2]
          [\ac{f}1]
      ]
  ]
\end{forest}\label{ex:feature-structure}

This structure leads to the concept of phrasal spellout. Phrasal spellout is that not terminals but multiple syntactic heads (phrases) are realized with a single piece of phonology (i.e. a single morpheme). A necessary requirement is that these multiple syntactic heads form a constituent.

Let me illustrate all of the above with the Faroese pronouns from Table \ref{tbl:syncretisms-derive}. I simplify the situation in two respects. First, I do not show the internal complexity of the pronouns, including person and number features. Instead, I give a triangle, indicating that this is a complex syntactic structure. Second, in this simplified representation I consider the Faroese pronouns to be monomorphemic. I ignore the fact that all three pronouns clearly have the stem \tit{t} with a suffix on it.

The lexical entry for \tit{tú} `3\tsc{sg}.\tsc{nom}' is given in \ref{ex:faroese-tu-lexicon}. The syntactic representation consists of the complex syntactic structure that corresponds to the third person singular pronoun, and \tsc{f}1, making it a \tsc{nom}P. The phonological representation that is linked to the syntactic representation in \tit{tú}.

\ex.
\begin{forest} boom
  [\ac{nom}P
      [\ac{f}1]
      [NP
          [3\tsc{sg}, roof]
      ]
  ]
  {\draw (.east) node[right]{⇔ \tit{tú}}; }
\end{forest}
\label{ex:faroese-tu-lexicon}

The lexical entry for \tit{teg} `3\tsc{sg}.\tsc{acc}' is given in \ref{ex:faroese-teg-lexicon}. The syntactic representation contains all the features of the syntactic structure in \ref{ex:faroese-tu-lexicon}, plus \tsc{f}2, making it an \tsc{acc}P. The linked phonological representation is \tit{teg}.

\ex.
\begin{forest} boom
  [\ac{acc}P
      [\ac{f}2]
      [\ac{nom}P
          [\ac{f}1]
          [3\ac{sg}
              [..., roof]
          ]
      ]
  ]
  {\draw (.east) node[right]{⇔ \tit{teg}}; }
\end{forest}
\label{ex:faroese-teg-lexicon}

The lexical entry for \tit{tær} `3\tsc{sg}.\tsc{dat}' is given in \ref{ex:faroese-taer-lexicon}. The syntactic representation contains all the features of the syntactic structure in \ref{ex:faroese-teg-lexicon}, plus \tsc{f}3, making it an \tsc{dat}P. The linked phonological representation is \tit{tær}.

\ex.
\begin{forest} boom
  [\ac{dat}P
      [\ac{f}3]
      [\ac{acc}P
          [\ac{f}2]
          [\ac{nom}P
              [\ac{f}1]
              [3\ac{sg}
                  [..., roof]
              ]
          ]
      ]
  ]
  {\draw (.east) node[right]{⇔ \tit{tær}}; }
\end{forest}
\label{ex:faroese-taer-lexicon}

This brings me to the part of the discussion about matching. The syntactic structures and their phonological counterparts I gave in \ref{ex:faroese-tu-lexicon} to \ref{ex:faroese-taer-lexicon} are lexical entries. These are matched against the actual syntactic structure that is to be realized. Examples of syntactic structures are given in \ref{ex:faroese-spellout}.
The syntactic structure in \ref{ex:faroese-tu-spellout} matches the syntactic structure of the lexical entry in \ref{ex:faroese-tu-lexicon}. Therefore, this syntactic structure is spelled out as \tit{tu}. Throughout this dissertation I circle the part of the structure that corresponds to a particular lexical entry, and I place corresponding the phonology next to it.
\ref{ex:faroese-teg-spellout} matches the syntactic structure of the lexical entry in \ref{ex:faroese-teg-lexicon}, and it is spelled out as \tit{teg}.
\ref{ex:faroese-taer-spellout} matches the syntactic structure of the lexical entry in \ref{ex:faroese-taer-lexicon}, and it is spelled out as \tit{tær}.

\ex.\label{ex:faroese-spellout}
\a. \begin{forest} boom
[\tsc{nomP},name=nomp,
tikz={
\node[label=below right:\tit{tú},
draw,circle,
xscale=0.8,yscale=1,
fit=(nomp)(nom)(3sg)(np)]{};
}
    [\ac{nom},name=nom]
    [NP,name=np
        [\tsc{3sg},name=3sg, roof],baseline
    ]
]
\end{forest}
\label{ex:faroese-tu-spellout}
\b. \begin{forest} boom
[\tsc{accP},name=accp,
tikz={
\node[label=below right:\tit{teg},
draw,circle,
xscale=0.8,yscale=1,
fit=(accp)(acc)(3sg)(np)]{};
}
    [\tsc{acc},name=acc]
    [\tsc{nom}P,name=nomp
        [\ac{nom},name=nom]
        [NP,name=np
            [\tsc{3sg},name=3sg, roof],baseline
        ]
    ]
]
\end{forest}
\label{ex:faroese-teg-spellout}
\b. \begin{forest} boom
[\tsc{datP},name=datp,
tikz={
\node[label=below right:\tit{tær},
draw,circle,
xscale=0.8,yscale=1,
fit=(datp)(dat)(3sg)(np)]{};
}
    [\tsc{dat},name=dat]
    [\tsc{acc}P,name=accp
        [\tsc{acc},name=acc]
        [\tsc{nom}P,name=nomp
            [\ac{nom},name=nom]
            [NP,name=np
                [\tsc{3sg},name=3sg, roof],baseline
            ]
        ]
    ]
]
\end{forest}
\label{ex:faroese-taer-spellout}

In the Faroese examples above, the syntactic structure in the syntax always exactly matched the syntactic structure of the lexical entry. However, to be a successful match, identity is not a necessary requirement. Instead, matching relies on a containment relation. A lexical entry applies when it contains all features. This is formalized as in \ref{ex:superset-principle}.

\ex. \tbf{The Superset Principle} \citet{starke2009}:\\
A lexically stored tree matches a syntactic node iff the lexically stored tree contains the syntactic node.
\label{ex:superset-principle}

Let me illustrate this with the Dutch second person plural pronoun from Table \ref{tbl:syncretisms-derive}. This pronoun is syncretic between between the nominative, accusative and dative.
The lexicon only contains a single lexical entry, namely \ref{ex:dutch-jullie-lexicon}. The nominative, the accusative and the dative can all be spelled out with this lexical entry.

\ex.
\begin{forest} boom
  [\ac{dat}P
      [\ac{f}3]
      [\ac{acc}P
          [\ac{f}2]
          [\ac{nom}P
              [\ac{f}1]
              [2\ac{pl}
                  [..., roof]
              ]
          ]
      ]
  ]
  {\draw (.east) node[right]{⇔ \tit{jullie}}; }
\end{forest}
\label{ex:dutch-jullie-lexicon}

The syntactic structure of the dative, given in \ref{ex:dutch-jullie-spellout-dat}, is the least exciting of the three. It is an exact match to the syntactic structure in the lexical entry \ref{ex:dutch-jullie-lexicon}, and therefore, spelled out as \tit{jullie}.

\ex. \begin{forest} boom
[\tsc{datP},name=datp,
tikz={
\node[label=below right:\tit{jullie},
draw,circle,
xscale=0.8,yscale=1,
fit=(datp)(dat)(2pl)(np)]{};
}
    [\tsc{dat},name=dat]
    [\tsc{acc}P,name=accp
        [\tsc{acc},name=acc]
        [\tsc{nom}P,name=nomp
            [\ac{nom},name=nom]
            [NP,name=np
                [\tsc{2pl},name=2pl, roof],baseline
            ]
        ]
    ]
]
\end{forest}
\label{ex:dutch-jullie-spellout-dat}

The syntactic structure of the accusative second person plural is given in \ref{ex:dutch-jullie-spellout-acc}. This syntactic structure does not exactly match the syntactic structure in the lexical entry in \ref{ex:dutch-jullie-lexicon}. However, the syntactic structure in the lexical entry \tbf{contains} the syntactic structure of the accusative.
I repeated the lexical entry for \tit{jullie} in \ref{ex:dutch-jullie-lexicon-acc}, marking the subpart of the tree that matches the syntactic structure in gray.

\ex. \begin{forest} boom
  [\tsc{datP},name=datp
      [\tsc{dat},name=dat]
      [\tsc{acc}P,name=accp,
      tikz={
      \node[draw,circle,DG,
      fill=DG,fill opacity=0.2,
      xscale=0.8,yscale=1,
      fit=(accp)(acc)(2pl)(np)]{};
      }
          [\tsc{acc},name=acc]
          [\tsc{nom}P,name=nomp
              [\ac{nom},name=nom]
              [NP,name=np
                  [\tsc{2pl},name=2pl, roof],baseline
              ]
          ]
      ]
  ]
  {\draw (.east) node[right]{⇔ \tit{jullie}}; }
\end{forest}
\label{ex:dutch-jullie-lexicon-acc}

As a result, the accusative is spelled out as \tit{jullie}, shown in \ref{ex:dutch-jullie-spellout-acc}.

\ex. \begin{forest} boom
[\tsc{accP},name=accp,
tikz={
\node[label=below right:\tit{jullie},
draw,circle,
xscale=0.8,yscale=1,
fit=(accp)(acc)(np)(2pl)]{};
}
    [\tsc{acc},name=acc]
    [\tsc{nom}P,name=nomp
        [\ac{nom},name=nom]
        [NP,name=np
            [\tsc{2pl},name=2pl, roof],baseline
        ]
    ]
]
\end{forest}
\label{ex:dutch-jullie-spellout-acc}



The same holds for the nominative second person plural. The syntactic structure is given in \ref{ex:dutch-jullie-spellout-acc}. This syntactic structure does not exactly match the syntactic structure in the lexical entry in \ref{ex:dutch-jullie-lexicon}. However, again, the syntactic structure in the lexical entry \tbf{contains} the syntactic structure of the nominative.
I repeated the lexical entry for \tit{jullie} in \ref{ex:dutch-jullie-lexicon-nom}, marking the subpart of the tree that matches the syntactic structure in gray.

 \ex. \begin{forest} boom
   [\tsc{datP},name=datp
       [\tsc{dat},name=dat]
       [\tsc{acc}P,name=accp
           [\tsc{acc},name=acc]
           [\tsc{nom}P,name=nomp,
           tikz={
           \node[draw,circle,DG,
           fill=DG,fill opacity=0.2,
           xscale=0.8,yscale=1,
           fit=(nomp)(nom)(2pl)(np)]{};
           }
               [\ac{nom},name=nom]
               [NP,name=np
                   [\tsc{2pl},name=2pl, roof],baseline
               ]
           ]
       ]
   ]
   {\draw (.east) node[right]{⇔ \tit{jullie}}; }
 \end{forest}
 \label{ex:dutch-jullie-lexicon-nom}

As a result, the nominative is spelled out as \tit{jullie}, as shown in \ref{ex:dutch-jullie-spellout-nom}.

\ex.
\begin{forest} boom
[\tsc{nomP},name=nomp,
tikz={
\node[label=below right:\tit{jullie},
draw,circle,
xscale=0.8,yscale=1,
fit=(nomp)(nom)(2pl)(np)]{};
}
    [\ac{nom},name=nom]
    [NP,name=np
        [\tsc{2pl},name=2pl, roof],baseline
    ]
]
\end{forest}
 \label{ex:dutch-jullie-spellout-nom}


A question arises at this point. Why is the accusative and nominative in Faroese not spelled out by the lexical entry for the dative (and why is the nominative not spelled out by the lexical entry for the accusative)? These syntactic structures are namely contained in the lexical entry for the dative (and the accusative).
The reason for that comes from how competition between lexical entries is regulated. When two lexical entries compete, the best fit wins. The best fit is the lexical entry with the least unused features. This is formalized as in \ref{ex:elsewhere-condition}.

\ex. \tbf{The Elsewhere Condition} (\citealt{kiparsky1973}, formulated as in \citealt{caha2020}):\\
When two entries can spell out a given node, the more specific entry wins. Under the Superset Principle governed insertion, the more specific entry is the one which has fewer unused features.
\label{ex:elsewhere-condition}

Let me come back to the Faroese examples. Consider first again the syntactic structure in \ref{ex:faroese-tu-spellout}.
All the Faroese lexical entries \ref{ex:faroese-tu-lexicon}, \ref{ex:faroese-teg-lexicon} and \ref{ex:faroese-taer-lexicon} contain the syntactic structure for it.
\ref{ex:faroese-taer-lexicon} has two unused features: \tsc{f2} and \tsc{f3}. \ref{ex:faroese-teg-lexicon} has one unused feature: \tsc{f2}. Only \ref{ex:faroese-tu-lexicon} does not have any unused features, hence is wins the competition over the other two.

Regarding the syntactic structure \ref{ex:faroese-teg-spellout}, the lexical entries \ref{ex:faroese-teg-lexicon} and \ref{ex:faroese-taer-lexicon} contain the syntactic structure. \footnote{
\ref{ex:faroese-tu-lexicon} is not even a candidate here, because it does not contain the syntactic structure (it lacks \tsc{f2}.)
}
However, only \ref{ex:faroese-taer-lexicon} does not have any unused features, and it is inserted.

The Table in \ref{tbl:syncretisms-derive} contains two more attested patterns: the ABB in Icelandic and the AAB in German. In the remainder of this section I show how these two patterns are derived. I also how the system is unable to derive an ABA, which is a pattern unattested crosslinguistically.

Let us first consider the Icelanic pattern. For the first person plural, Icelandic has \tit{við} for the nominative and \tit{okkur} for the accusative and dative. Two lexical entries are needed for that. The first one in \ref{ex:icelandic-vid-lexicon} contain pronominal syntactic structure and \tsc{f}1, and corresponds to the phonology \tit{við}.
The second one is given in \ref{ex:icelandic-okkur-lexicon}. It contains in addition to what \ref{ex:icelandic-vid-lexicon} the features \tsc{f}2 and \tsc{f}3.

\ex.
\a.
\begin{forest} boom
  [\ac{nom}P
      [\ac{f}1]
      [NP
          [1\tsc{pl}, roof]
      ]
  ]
  {\draw (.east) node[right]{⇔ \tit{við}}; }
\end{forest}
\label{ex:icelandic-vid-lexicon}
\b.
\begin{forest} boom
  [\ac{dat}P
      [\ac{f}3]
      [\ac{acc}P
          [\ac{f}2]
          [\ac{nom}P
              [\ac{f}1]
              [1\tsc{pl}
                  [..., roof]
              ]
          ]
      ]
  ]
  {\draw (.east) node[right]{⇔ \tit{okkur}}; }
\end{forest}
\label{ex:icelandic-okkur-lexicon}

The syntactic structure in \ref{ex:icelandic-okkur-spellout-dat} is contained in the syntactic structure of the lexical entry \ref{ex:icelandic-okkur-lexicon}, and therefore, spelled out as \tit{okkur}.

\ex. \begin{forest} boom
[\tsc{datP},name=datp,
tikz={
\node[label=below right:\tit{okkur},
draw,circle,
xscale=0.8,yscale=1,
fit=(datp)(dat)(3sg)(np)]{};
}
    [\tsc{dat},name=dat]
    [\tsc{acc}P,name=accp
        [\tsc{acc},name=acc]
        [\tsc{nom}P,name=nomp
            [\ac{nom},name=nom]
            [NP,name=np
                [\tsc{3sg},name=3sg, roof],baseline
            ]
        ]
    ]
]
\end{forest}
\label{ex:icelandic-okkur-spellout-dat}

The syntactic structure in \ref{ex:icelandic-okkur-spellout-acc} is contained in the syntactic structure of the lexical entry \ref{ex:icelandic-okkur-lexicon}, and therefore, spelled out as \tit{okkur}.

\ex. \begin{forest} boom
[\tsc{accP},name=accp,
tikz={
\node[label=below right:\tit{okkur},
draw,circle,
xscale=0.8,yscale=1,
fit=(accp)(acc)(3sg)(np)]{};
}
    [\tsc{acc},name=acc]
    [\tsc{nom}P,name=nomp
        [\ac{nom},name=nom]
        [NP,name=np
            [\tsc{3sg},name=3sg, roof],baseline
        ]
    ]
]
\end{forest}
\label{ex:icelandic-okkur-spellout-acc}

The syntactic structure in \ref{ex:icelandic-vid-spellout} is contained in the syntactic structure of the lexical entry \ref{ex:icelandic-vid-lexicon} and in \ref{ex:icelandic-okkur-lexicon}.
The former, \ref{ex:icelandic-vid-lexicon}, has no unused features. The latter, \ref{ex:icelandic-okkur-lexicon}, has two unused features: \tsc{f}2 and \tsc{f}3.
Therefore, \ref{ex:icelandic-okkur-lexicon} wins the competition, and the syntactic structure is spelled out as \tit{við}.

\ex. \begin{forest} boom
[\tsc{nomP},name=nomp,
tikz={
\node[label=below right:\tit{við},
draw,circle,
xscale=0.8,yscale=1,
fit=(nomp)(nom)(3sg)(np)]{};
}
    [\ac{nom},name=nom]
    [NP,name=np
        [\tsc{3sg},name=3sg, roof],baseline
    ]
]
\end{forest}
\label{ex:icelandic-vid-spellout}



For the third person singular feminine, German uses \tit{sie} for nominative and accusative, and \tit{ihr} for dative. Two lexical entries are needed for that.
The first one in \ref{ex:german-sie-lexicon} contains pronominal syntactic structure, \tsc{f}1 and \tsc{f}2, and corresponds to the phonology \tit{sie}.
The second one is given in \ref{ex:german-ihr-lexicon}. It contains in addition to what \ref{ex:german-sie-lexicon} the feature \tsc{f}3.

\ex.
\a.
\begin{forest} boom
  [\ac{acc}P
      [\ac{f}2]
      [\ac{nom}P
          [\ac{f}1]
          [3\ac{sg}.\tsc{f}
              [..., roof]
          ]
      ]
  ]
  {\draw (.east) node[right]{⇔ \tit{sie}}; }
\end{forest}
\label{ex:german-sie-lexicon}
\b.
\begin{forest} boom
  [\ac{dat}P
      [\ac{f}3]
      [\ac{acc}P
          [\ac{f}2]
          [\ac{nom}P
              [\ac{f}1]
              [3\ac{sg}.\tsc{f}
                  [..., roof]
              ]
          ]
      ]
  ]
  {\draw (.east) node[right]{⇔ \tit{ihr}}; }
\end{forest}
\label{ex:german-ihr-lexicon}

The syntactic structure in \ref{ex:german-ihr-spellout} is contained in the syntactic structure of the lexical entry \ref{ex:german-ihr-lexicon}, and therefore, spelled out as \tit{ihr}.

\ex. \begin{forest} boom
[\tsc{datP},name=datp,
tikz={
\node[label=below right:\tit{ihr},
draw,circle,
xscale=0.8,yscale=1,
fit=(datp)(dat)(3sg)(np)]{};
}
    [\tsc{dat},name=dat]
    [\tsc{acc}P,name=accp
        [\tsc{acc},name=acc]
        [\tsc{nom}P,name=nomp
            [\ac{nom},name=nom]
            [NP,name=np
                [\tsc{3sg},name=3sg, roof],baseline
            ]
        ]
    ]
]
\end{forest}
\label{ex:german-ihr-spellout}

The syntactic structure in \ref{ex:german-sie-spellout-acc} is contained in the syntactic structure of the lexical entry \ref{ex:german-sie-lexicon} and in \ref{ex:german-ihr-lexicon}.
The former, \ref{ex:german-sie-lexicon}, has one no unused features. The latter, \ref{ex:german-ihr-lexicon}, has one unused feature: \tsc{f}3.
Therefore, \ref{ex:german-sie-lexicon} wins the competition, and the syntactic structurer is spelled out as \tit{sie}.

\ex. \begin{forest} boom
[\tsc{accP},name=accp,
tikz={
\node[label=below right:\tit{sie},
draw,circle,
xscale=0.8,yscale=1,
fit=(accp)(acc)(3sg)(np)]{};
}
    [\tsc{acc},name=acc]
    [\tsc{nom}P,name=nomp
        [\ac{nom},name=nom]
        [NP,name=np
            [\tsc{3sg},name=3sg, roof],baseline
        ]
    ]
]
\end{forest}
\label{ex:german-sie-spellout-acc}

The syntactic structure in \ref{ex:german-sie-spellout-nom} is contained in the syntactic structure of the lexical entry \ref{ex:german-sie-lexicon} and in \ref{ex:german-ihr-lexicon}.
The former, \ref{ex:german-sie-lexicon}, has one unused feature: \tsc{f}2. The latter, \ref{ex:german-ihr-lexicon}, has two unused features: \tsc{f}2 and \tsc{f}3. Therefore, \ref{ex:german-sie-lexicon} wins the competition, and the syntactic structurer is spelled out as \tit{sie}.

\ex. \begin{forest} boom
[\tsc{nomP},name=nomp,
tikz={
\node[label=below right:\tit{sie},
draw,circle,
xscale=0.8,yscale=1,
fit=(nomp)(nom)(3sg)(np)]{};
}
    [\ac{nom},name=nom]
    [NP,name=np
        [\tsc{3sg},name=3sg, roof],baseline
    ]
]
\end{forest}
\label{ex:german-sie-spellout-nom}

This last example also illustrates that the laid out system is unable to derive an ABA pattern. The unability of the system to derive such a pattern is a welcome one, since the pattern is unattested cross-linguistically. In an ABA pattern, the nominative and the dative are syncretic, to the exclusion of the accusative. Staying close to the German example, in this hypothetical example, the dative would be \tit{ihr}, the accusative \tit{sie}, and the nominative \tit{ihr} again.
This result could never be derived with the lexical entries given in \ref{ex:german-sie-lexicon} and \ref{ex:german-ihr-lexicon}. \tit{Ihr} is inserted for the dative and the cases contained in it (so accusative and nominative), unless a more specific lexical entry is found. \tit{Sie} is the more specific lexical entry that is found from the accusative on. From the accusative on, \tit{sie} will be inserted until a more specific entry is found. If no entry is specified for nominative, \tit{sie} will surface. \tit{Ihr} will not resurface, because the lexical entry for \tit{sie} will remain to be more specific.

%say how others do not account for this, at the place where it's necessary, so the end?

\subsection{Deriving case containment}

In this section I show how morphological case containment can be derived from the case decomposition in Table \ref{tbl:case-decomposed}. I repeat an example that shows morphological case containment in Table \ref{tbl:cont-khanty} \pgcitep{nikolaeva1999}{16}.

\begin{table}[ht]
  \center
  \caption {Containment in \tsc{3sg} in Khanty}
    % !TEX root = ../thesis.tex

\begin{tabular}{cl}
\toprule
          & 3\ac{sg} \\
          \cmidrule{2-2}
\ac{nom}
          & luw                                     \\
\ac{acc}  & luw\tbf{-e:l}                           \\
\ac{dat}  & luw\tbf{-e:l}\tcol{DG}{\tbf{-na}}       \\
\bottomrule
\end{tabular}

  \label{tbl:cont-khanty-3sg}
\end{table}

The intuition is the following. The morphological form of the pronouns mirror the cumulative feature decomposition given in Table \ref{tbl:case-decomposed}. That is, the accusative has the morphology that the nominative has (\tit{luw}) plus something extra (\tit{-e:l}). The dative has the morphology that the accusative has (\tit{luw-e:l}) plus something extra (\tit{luw-e:l}-na). In what follows I make it concrete how syntactic features translate to morphological form.

First, I give the lexical entry for the nominative third person singular. It contains pronominal syntactic structure and the feature \tsc{f}1. The lexical entry is given in \ref{ex:khanty-luw-lexicon}.

\ex.
\begin{forest} boom
  [\ac{nom}P
      [\ac{nom}]
      [3\ac{sg}
          [.., roof]
      ]
  ]
  {\draw (.east) node[right]{⇔ \tit{luw}}; }
\end{forest}\label{ex:khanty-luw-lexicon}

The syntactic structure in \ref{ex:khanty-luw-spellout} is contained in the syntactic structure of the lexical entry \ref{ex:khanty-luw-spellout}. The nominative is spelled out as \tit{luw}.

\ex. \begin{forest} boom
[\tsc{nomP},name=nomp,
tikz={
\node[label=below right:\tit{luw},
draw,circle,
xscale=0.8,yscale=1,
fit=(nomp)(nom)(3sg)(np)]{};
}
    [\ac{nom},name=nom]
    [NP,name=np
        [\tsc{3sg},name=3sg, roof],baseline
    ]
]
\end{forest}\label{ex:khanty-luw-spellout}

In the previous section I only gave examples in which the forms were syncretic or suppletive (i.e. formally unrelated). The examples from Khanty are different. The accusative pronoun formally contains the nominative pronoun. This can be modelled by letting \tsc{nom}P still be realized by the lexical entry for the nominative, and giving the accusative its own realization.\footnote{
Note that it is crucial here to have a theory in which the accusative contains the nominative. If not, it would be a surprise that the nominative is contained in the accusative.
}

Accordingly, I give the lexical entry for the accusative marker \tit{-e:l} in \ref{ex:khanty-luw-el}.

\ex. \begin{forest} boom
  [\ac{acc}P
      [\ac{acc}]
  ]
  {\draw (.east) node[right]{⇔ \tit{-e:l}}; }
\end{forest}\label{ex:khanty-el-lexicon}

So, \tit{luw-e:l} consists of two morphemes that both correspond to their own piece of syntactic structure: \tit{luw} and \tit{e:l}. But how do these two morphemes combine? This brings me to another detour into the Nanosyntactic theory. To understand how multiple morphemes combine, I need to address the spellout procedure in Nanosyntax, which happens according to a spellout algorithm. Part of this algorithm are two movement options, that create different constituents. The two movement options are Cinquan rollup movement and spec to spec movement.

Spellout in Nanosyntax only targets constituents. So it is impossible to let accP spell out \tit{e:l} without spelling out also everything below it.

\ex. \begin{forest} boom
[\ac{acc}P,name=accp, s sep=30mm,
tikz={
\node[label={below left:\tit{-e:l}},
draw,circle,rotate=45,
xscale=0.7,yscale=0.9,
fit=(acc)(accp)]{};
}
    [\ac{acc},name=acc]
    [\tsc{nomP},name=nomp,
    tikz={
    \node[label=below right:\tit{luw},
    draw,circle,
    xscale=0.8,yscale=1,
    fit=(nomp)(nom)(3sg)(np)]{};
    }
        [\ac{nom},name=nom]
        [NP,name=np
            [\tsc{3sg},name=3sg, roof],baseline
        ]
    ]
]
\end{forest}
\label{ex:khanty-el-luw-spellout}

The \tsc{acc}P can only be spelled out if the \tsc{nom}P has moved away. So like this:

\ex. \begin{forest} boom
[\ac{acc}P,name=accp2, s sep=20mm
    [\tsc{nomP},name=nomp,
    tikz={
    \node[label=below right:\tit{luw},
    draw,circle,
    xscale=0.8,yscale=1,
    fit=(nomp)(nom)(3sg)(np)]{};
    }
        [\ac{nom},name=nom]
        [NP,name=np
            [\tsc{3sg},name=3sg, roof],baseline
        ]
    ]
    [\ac{acc}P,name=accp,
    tikz={
    \node[label={below right:\tit{-e:l}},
    draw,circle,
    xscale=0.7,yscale=0.9,
    fit=(acc)(accp)]{};
    }
     [\ac{acc},name=acc]
    ]
]
\end{forest}
\label{ex:khanty-luw-el-spellout}

This is exactly one of the movement options that exists in the spellout algorithm.

Just as the accusative, the dative also has its own realization. The lexical entry for \tit{-na} is given in \ref{ex:khanty-na-lexion}.

\ex. \begin{forest} boom
  [\ac{dat}P
      [\ac{dat}]
  ]
  {\draw (.east) node[right]{⇔ \tit{-na}}; }
\end{forest}
\label{ex:khanty-na-lexion}

Again, because the spellout only targets constituents, it cannot be spelled out right after it has been merged, shown in \ref{ex:khanty-na-luw-el-spellout}.

\ex. \begin{forest} boom
[\ac{acc}P,name=accp2, s sep=20mm
    [\tsc{nomP},name=nomp,
    tikz={
    \node[label=below right:\tit{luw},
    draw,circle,
    xscale=0.8,yscale=1,
    fit=(nomp)(nom)(3sg)(np)]{};
    }
        [\ac{nom},name=nom]
        [NP,name=np
            [\tsc{3sg},name=3sg, roof],baseline
        ]
    ]
    [\ac{acc}P,name=accp,
    tikz={
    \node[label={below right:\tit{-e:l}},
    draw,circle,
    xscale=0.7,yscale=0.9,
    fit=(acc)(accp)]{};
    }
     [\ac{acc},name=acc]
    ]
]
\end{forest}
\label{ex:khanty-na-luw-el-spellout}

Instead, the whole complement needs to be moved out of the \tsc{dat}P. Now the lexical entry can be inserted.

\ex.
\begin{forest} boom
[\tsc{datP}
    [\ac{acc}P,name=accp2, s sep=20mm
        [\tsc{nomP},name=nomp,
        tikz={
        \node[label=below right:\tit{luw},
        draw,circle,
        xscale=0.8,yscale=1,
        fit=(nomp)(nom)(3sg)(np)]{};
        }
            [\ac{nom},name=nom]
            [NP,name=np
                [\tsc{3sg},name=3sg, roof],baseline
            ]
        ]
        [\ac{acc}P,name=accp,
        tikz={
        \node[label={below right:\tit{-e:l}},
        draw,circle,
        xscale=0.7,yscale=0.9,
        fit=(acc)(accp)]{};
        }
         [\ac{acc},name=acc]
        ]
    ]
    [\tsc{datP},name=datp,
    tikz={
    \node[label={below right:\tit{-na}},
    draw,circle,
    xscale=0.7,yscale=0.9,
    fit=(dat)(datp)]{};
    }
        [\ac{dat},name=dat]
    ]
]
\end{forest}
\label{ex:khanty-luw-el-na-spellout}





\section{Ellipsis}

Ellipsis targets phrases

it does not delete elements one by one

(1) Sluicing
Someone woke up early
but I have no idea who woke up early.

(2) VP ellipsis
John woke up early, and Mary did wake up early.

(3) Fragment answers
Q: Who woke up early?
A: John woke up early


(Õ) VP Ellipsis:
a. John was hassled by the police and Mary was [VP hassled by the police ] too.
b. John put his beer on the žfloor, so Mary did [VP put her beer on the žfloor ]





I will climb a mountain today but he will not [climb a mountain today]


\section{Reflex of morphology in syntax}


\subsection{Morphology}

\ex.
\begin{forest} boom
  [\tsc{datP},name=datp,
  tikz={
  \node[draw,circle,LG,
  xscale=0.8,yscale=1,
  fill opacity=0.2,
  fill=LG,
  fit=(datp)(dat)(nom)(x)]{};
  }
      [\ac{dat},name=dat]
        [\ac{acc}P,name=accp,
        tikz={
        \node[draw,circle,
        xscale=0.75,yscale=0.95,
        fill opacity=0.2,
        fill=DG,DG,
        fit=(accp)(acc)(nom)(np)(x)]{};
        }
          [\ac{acc},name=acc]
          [\tsc{nomP},name=nomp
              [\ac{nom},name=nom]
              [NP,name=np
                  [...,name=x, roof]
              ]
          ]
      ]
  ]
\end{forest}


\ex.
\begin{forest} boom
  [\tsc{datP},name=datp,
  tikz={
  \node[draw,circle,LG,
  xscale=0.8,yscale=1,
  fill opacity=0.2,
  fill=LG,
  fit=(datp)(dat)(nom)(x)]{};
  }
      [\ac{dat},name=dat]
      [\ac{acc}P,name=accp
          [\ac{acc},name=acc]
          [\tsc{nomP},name=nomp,
          tikz={
          \node[draw,circle,DG,
          xscale=0.7,yscale=0.9,
          fill opacity=0.2,
          fill=DG,
          fit=(nomp)(nom)(np)(x)]{};
          }
              [\ac{nom},name=nom]
              [NP,name=np
                  [...,name=x, roof]
              ]
          ]
      ]
  ]
\end{forest}



\ex.
\begin{forest} boom
      [\ac{acc}P,name=accp,
      tikz={
      \node[draw,circle,
      xscale=0.8,yscale=1,
      fill opacity=0.2,
      fill=LG,LG,
      fit=(accp)(acc)(nom)(x)]{};
      }
          [\ac{acc},name=acc]
          [\tsc{nomP},name=nomp,
          tikz={
          \node[draw,circle,DG,
          xscale=0.75,yscale=0.95,
          fill opacity=0.2,
          fill=DG,
          fit=(nomp)(nom)(x)]{};
          }
              [\ac{nom},name=nom]
              [NP,name=np
                  [...,name=x, roof]
              ]
          ]
      ]
  ]
\end{forest}



\subsection{Syntax}

\begin{table}[H]
  \center
	\caption {\ac{dat}P deletes \ac{acc}P}
		\begin{tabular}[b]{c c}
      \begin{forest} boom
        [\tsc{datP}
            [\ac{dat}]
              [\ac{acc}P,name=accp,
              tikz={
              \node[draw,circle,
              xscale=0.775,yscale=0.975,
              fit=(accp)(acc)(nom)(x)]{};
              }
                [\ac{acc},name=acc]
                [\tsc{nomP},name=nomp
                    [\ac{nom},name=nom]
                    [NP,name=np
                        [...,name=x, roof ,baseline]
                    ]
                ]
            ]
        ]
      \end{forest}
      &
      \begin{forest} boom
        [\textcolor{LG}{\tsc{accP}},name=accp,
        tikz={
        \node[draw,circle,
        xscale=0.775,yscale=0.975,
        fit=(accp)(acc)(nom)(x)]{};
        }
            [\textcolor{LG}{\ac{acc}},name=acc,edge=LG]
            [\textcolor{LG}{\tsc{nomP}},name=nomp,edge=LG
                [\textcolor{LG}{\ac{nom}},name=nom,edge=LG]
                [\textcolor{LG}{NP},name=np,edge=LG
                    [\textcolor{LG}{...},name=x,
                    roof, baseline, edge=LG
                    ]
                ]
            ]
        ]
      \end{forest} \\
  \end{tabular}
\end{table}

\begin{table}[H]
  \center
	\caption {\ac{dat}P deletes \ac{nom}P}
		\begin{tabular}[b]{cc}
      \begin{forest} boom
        [\tsc{datP}
            [\ac{dat}]
              [\ac{acc}P,name=accp
                [\ac{acc},name=acc]
                [\tsc{nomP},name=nomp,
                tikz={
                \node[draw,circle,
                xscale=0.75,yscale=0.95,
                fit=(nomp)(nom)(x)]{};
                }
                    [\ac{nom},name=nom]
                    [NP,name=np
                        [...,name=x, roof, baseline]
                    ]
                ]
            ]
        ]
      \end{forest}
      &
      \begin{forest} boom
        [\textcolor{LG}{\tsc{nomP}},name=nomp,
        tikz={
        \node[draw,circle,
        xscale=0.75,yscale=0.95,
        fit=(nomp)(nom)(x)]{};
        }
            [\textcolor{LG}{\ac{nom}},name=nom,edge=LG]
            [\textcolor{LG}{NP},name=np,baseline,edge=LG
                [\textcolor{LG}{...},name=x,
                roof,baseline,edge=LG
                ]
            ]
        ]
      \end{forest} \\
  \end{tabular}
\end{table}

\begin{table}[H]
  \center
	\caption {\ac{acc}P deletes \ac{nom}P}
		\begin{tabular}[b]{cc}
      \begin{forest} boom
          [\ac{acc}P,name=accp
              [\ac{acc},name=acc]
              [\tsc{nomP},name=nomp,
              tikz={
              \node[draw,circle,
              xscale=0.75,yscale=0.95,
              fit=(nomp)(nom)(x)]{};
              }
                  [\ac{nom},name=nom]
                  [NP,name=np
                      [...,name=x,
                      roof,baseline
                      ]
                  ]
              ]
          ]
      \end{forest}
      &
      \begin{forest} boom
        [\textcolor{LG}{\tsc{nomP}},name=nomp,
        tikz={
        \node[draw,circle,
        xscale=0.75,yscale=0.95,
        fit=(nomp)(nom)(x)]{};
        }
            [\textcolor{LG}{\ac{nom}},name=nom,
            edge=LG]
            [\textcolor{LG}{NP},name=np,
            edge=LG
                [\textcolor{LG}{...},name=x,
                roof,baseline,edge=LG
                ]
            ]
        ]
      \end{forest}\\
  \end{tabular}
\end{table}
