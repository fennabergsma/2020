% !TEX root = thesis.tex

\chapter{Appendix}


\section{Ungrammatical examples against case scale in \ac{mg}}

let me show that the claim I made for Gothic holds for \ac{mg} as well: \tsc{dat} wins over \tsc{acc} wins over \tsc{nom}.

Examples in which the internal case is dative, the external case is accusative and the relative pronoun appears in accusative case is ungrammatical.

\exg. *Ich {lade ein} wen \tbf{auch} \tbf{Maria} \tbf{vertraut}. \\
 I invite\scsub{[acc]} \tsc{rel}.\ac{acc}.\tsc{an} also Maria trusts\scsub{[dat]}.\\
 `I invite whoever Maria also trusts.' \flushfill{\pgcitealt{vogel2001}{344}}\label{ex:mg-acc-dat-u}

Examples in which the internal case is dative, the external case is accusative and the relative pronoun appears in accusative case is ungrammatical.

\exg. *Uns besucht wer \tbf{Maria} \tbf{vertraut}.\\
 us visits\scsub{[nom]} \tsc{rel}.\ac{nom}.\tsc{an} Maria trusts\scsub{[dat]}\\
 `Who visits us, Maria trusts.' \flushfill{\pgcitealt{vogel2001}{343}}

Examples in which the internal case is accusative, the external case is nominative and the relative pronoun appears in nominative case is ungrammatical.

\exg. *Uns besucht wer \tbf{Maria} \tbf{mag}.\\
 Us visits\scsub{[nom]} \tsc{rel}.\ac{nom}.\tsc{an} Maria.\ac{nom} likes\scsub{[acc]}\\
 `Who visits us likes Maria likes.' \flushfill{\pgcitealt{vogel2001}{343}}\label{ex:mg-nom-acc-u}

Examples in which the internal case is accusative, the external case is dative and the relative pronoun appears in accusative case is ungrammatical.

\exg. *Ich vertraue \tbf{wen} \tbf{auch} \tbf{Maria} \tbf{mag}. \\
 I trust\scsub{[dat]} \tsc{rel}.\ac{acc}.\tsc{an} also Maria likes\scsub{[acc]}.\\
 `I trust whoever Maria also likes.' \flushfill{\pgcitealt{vogel2001}{345}}

Examples in which the internal case is nominative, the external case is dative and the relative pronoun appears in nominative case is ungrammatical.

\exg. *Ich vertraue, \tbf{wer} \tbf{Hitchcock} \tbf{mag}.\\
 I trust\scsub{[dat]} \tsc{rel}.\ac{nom}.\tsc{an} Hitchcock likes\scsub{[nom]}\\
 `I trust who likes Hitchcock.' \flushfill{\pgcitealt{vogel2001}{345}}

Examples in which the internal case is nominative, the external case is accusative and the relative pronoun appears in nominative case is ungrammatical.

\exg. *Ich {lade ein}, \tbf{wer} \tbf{mir} \tbf{sympathisch} \tbf{ist}.\\
 I invite\scsub{[acc]} \tsc{rel}.\ac{nom}.\tsc{an} me nice is\scsub{[nom]}\\
 `I invite who I like.' \flushfill{\pgcitealt{vogel2001}{344}}

\phantom{x}
