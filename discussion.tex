% !TEX root = thesis.tex

\chapter{Discussion}\label{ch:discussion}


\section{Diachronic part}

First, German only had the d-pronoun and attraction. The pattern of attraction that came with that pronoun is ext only.
At some point, German invented the wh-pronoun. Helmut showed how it emerged. With that came the other pattern: int only. Some people lost the attraction (but everybody kept the d-pronoun) and with that the pattern disappeared.
So the patterns in headless relatives follow from the relative pronouns in the language.

Why are all languages of the `matching' type dead languages?
Was it a common thing that wh-pronouns were not used as relative pronouns?

Wouldn't we now not expect that Modern German patterns with Old High German wrt attraction in headed constructions. Yes, we would. And yes, this is exactly what we see. Paper by Bader on case attraction.

First there was only the relative pronoun with a D. Then we did case competition with this one, in both directions. Later, we only did it with the wh, and we only had internal left. Because this competitor was introduced, the case competition with D disappeared.


\section{Towards deriving the always-external pattern}

grosu: morphological distinctions correlate with `freedom'

Why \tsc{fem} does not have \tsc{wh}-pronouns?


\section{More languages}

\tit{valita} `choose' takes a partitive object

\ex. Valitsen mista sina piddt.
choose-I.el what-el you like-you.part
'I choose what you like.'

\tit{pitää} `like' takes elative objects

\exg. *Pidan mista sind valitset.\\
like-I.part what-el you choose-you.el\\
`I like what you choose.'

\exg. *Pidan mita sind valitset.\\
like-I.part what-el you choose-you.el\\
`I like what you choose.'




\section{The missing dative/accusative}

The accusative/dative example is missing from Gothic, Old High German and Ancient Greek. What if I take that seriously?
