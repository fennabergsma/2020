% !TEX root = thesis.tex

\chapter{Conclusion}\label{ch:discussio}

something happens in syntax
it can be explained if you look at morphology

Case competition in headless relatives has been described as some special property of a few special languages. Therefore, language-specific rules have been postulated to account for the data. My goal is to show that this phenomenon can be captured with `normal' syntactic processes, like ellipsis, c-command. The account makes predictions about how a language behaves based on the shape of its relative pronouns. And we see that case competition in headless relatives is actually more wide-spread than what has been assumed.


Table \ref{tbl:overview-rel-light} shows per language type which of the three options is chosen when the internal and external case differ.

\begin{table}[htbp]
  \center
  \caption{The surface pronoun with differing cases per language}
\begin{tabular}{cccc}
  \toprule
                & \tsc{k}\scsub{int} > \tsc{k}\scsub{ext} & \tsc{k}\scsub{ext} > \tsc{k}\scsub{int} &                  \\
                \cmidrule{2-3}
unrestricted    & relative pronoun\scsub{int}  & light head\scsub{ext} & Old High German  \\
internal-only   & relative pronoun\scsub{int}  & *                     & Modern German    \\
matching        & *                            & *                     & Polish           \\
external-only   & *                            & light head\scsub{ext} & not attested     \\
\bottomrule
\end{tabular}
\label{tbl:overview-rel-light}
\end{table}

The first column list the types of languages.
The second column shows the situation in which the internal case is the most complex. The relative pronoun that bears the internal case is the potential surface pronoun.
The third column shows the situation in which the external case is the most complex. The light head that bears the external case is the potential surface pronoun.
The asterix (*) indicates that there is no grammatical form for the surface pronoun.
The fourth column gives the example of the language type that I discuss in this chapter.
A language of the unrestricted type (like Old High German) allows both the internal case and the external case to surface when either of them wins the case competition. Either the light head with its external case or the relative pronoun with its internal case can be the surface pronoun.
A language of the internal-only type (like Modern German) allows only the internal case to surface when it wins the case competition, and it does not allow the external case to do so. The relative pronoun with its internal case can be the surface pronoun and the light head with its external case cannot.
A language of the matching type (like Polish) allows neither the internal nor the external case to surface when either of them wins the case competition. Neither the relative pronoun with its internal case nor the light head with its external case can be the surface pronoun.\footnote{
This holds for the situation in which the internal and external case differ. In Section \ref{ch:deriving-matching}, I show that the relative pronoun surfaces in matching contexts.
}
The language type that is not attested is the external-only type. That means that there is no language that allows only the external case to surface when it wins the case competition, and it does not allow the internal case to do so. In other words, there exist no language, in which the surface pronoun can only be the light head and not the relative pronoun.

What I have done in this section so far is reformulate the two descriptive parameters from Figure \ref{fig:two-parameters} into two other descriptive parameters.
Whether the internal case is allowed to surface corresponds to whether the relative pronoun surfaces. That implicates that the light head has been deleted and is therefore absent.
Similarly, whether the external case is allowed to surface corresponds to whether the light head surfaces. That implicates that the relative pronoun has been deleted and is therefore absent.
I show this in Figure \ref{fig:lexical-entries-rep}.

\begin{figure}[htbp]
  \centering
  \begin{tabular}[b]{c}
    \toprule
    \begin{tikzpicture}[node distance=1.5cm]
      \node (question2) [question]
      {ϕ+\tsc{k} portmanteau};
          \node (outcome2) [outcome, below of=question2, xshift=-2cm, yshift=-0.5cm]
          {matching};
              \node (example2) [example, below of=outcome2]
              {e.g. Polish\\\phantom{x}\\\phantom{x}};
          \node (question3) [question, below of=question2, xshift=2.5cm, yshift=-1cm]
          {\tsc{lh}-\tsc{rp} syncretism};
              \node (outcome3) [outcome, below of=question3, xshift=-2cm, yshift=-0.5cm]
              {internal-only};
                  \node (example3) [example, below of=outcome3]
                  {e.g. Modern German\\\phantom{x}};
              \node (outcome4) [outcome, below of=question3, xshift=2cm, yshift=-0.5cm]
              {unrestricted};
                  \node (example4) [example, below of=outcome4]
                  {e.g. Gothic, Old High German, Classical Greek};

    \draw [arrow] (question2) -- node[anchor=east] {no} (outcome2);
    \draw [arrow] (question2) -- node[anchor=west] {yes} (question3);
    \draw [arrow] (question3) -- node[anchor=east] {no} (outcome3);
    \draw [arrow] (question3) -- node[anchor=west] {yes} (outcome4);
    \end{tikzpicture}\\
    \bottomrule
  \end{tabular}
    \caption{Different lexical entries generate three language types (repeated)}
    \label{fig:lexical-entries-rep}
\end{figure}

Reformulating these parameters is not just restating the generalization in different terms. With this new formulation, I am able to identify the elements (i.e. the light head and the relative pronoun) that bear the internal and external cases. The difference between languages lies in whether or not it is possible to delete the light head (and with it the external case) and the relative pronoun (and with it the internal case).
