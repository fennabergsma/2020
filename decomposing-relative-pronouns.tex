% !TEX root = thesis.tex

\chapter{Decomposing relative pronouns}\label{ch:decomposing-relative-pronouns}

Putting all of this together in detail

What will lead to the different derivations from the last chapter? The only source of crosslinguistic variation: different lexical entries.

That's why I talk about the internal structure of the relative pronoun and the external head

So all the behavior we see in this section is derived from how the relative pronouns and the external head is specified in the lexicon.

OHG d: spec, rel
OHG wh: only interrogative

MG d: spec, rel
MG wh: interrogative, rel

P d: deix, rel
P wh: interrog, rel


\section{The lexical entries}


\subsection{The paradigms}

Is there any other observable difference between the languages? Yes! The shape of the relative pronouns.

This will be intuition later

\begin{itemize}
  \item when a language's light-headed relatives are \tsc{d} - \tsc{wh}, non-matching is never allowed
  \item with \tsc{wh} morphology, the internal case can win
  \item with \tsc{d} morphology (or is it when the relative pronoun can take a complement?), both the external and the internal can win
\end{itemize}


\begin{table}[H]
 \center
 \caption {Relative pronouns in headless relatives in Modern German}
  \begin{tabular}{ccc}
  \toprule
       & \ac{inan} & \ac{an} \\
        \cmidrule{2-3}
    \ac{nom}  & w-as     & w-er    \\
    \ac{acc}  & w-as     & w-en   \\
    \ac{dat}  & -      & w-em    \\
  \bottomrule
  \end{tabular}
\end{table}

\begin{table}[H]\label{tbl:paradigmohg}
 \center
 \caption {Relative pronouns in headless relatives in Old High German}
  \begin{tabular}{cccc}
  \toprule
       & \ac{n}.\ac{sg} & \ac{m}.\ac{sg}  & \ac{f}.\ac{sg} \\
        \cmidrule{2-4}
  \ac{nom} & d-aȥ           & d-ër          & d-iu      \\
  \ac{acc} & d-aȥ        & d-ën      & d-ea/-ia/(-ie) \\
  \ac{dat} & d-ëmu/-ëmo     & d-ëmu/-ëmo   & d-ëru/-ëro   \\
  \bottomrule
         & \ac{n}.\ac{pl} & \ac{m}.\ac{pl}   & \ac{f}.\ac{pl} \\
          \cmidrule{2-4}
    \ac{nom}  & d-iu/-ei      &  d-ē/-ea/-ia/-ie & d-eo/-io        \\
    \ac{acc}  & d-iu/-ei      &  d-ē/-ea/-ia/-ie & d-eo/-io        \\
    \ac{dat}  & d-ēm/-ēn      &  d-ēm/-ēn        & d-ēm/-ēn        \\
    \bottomrule
  \end{tabular}
\end{table}

Gothic relative pronouns are built from the demonstratives plus the complementizer \tit{ei}. Under \tit{ei}, two phonological processes take place. First, \tit{s} changes into \tit{z}, e.g. in \tit{þ-ōs} to \tit{þ-ōz-ei}. Second, on bisyllabic elements, final vowels disappear e.g. \tit{þ-ata} to \tit{þ-at-ei}.

\begin{table}[H]
	\center
	\caption {Gothic demonstratives}
		\begin{tabular}{cccc}
		\toprule
							& \ac{n}.\ac{sg} 	& \ac{m}.\ac{sg}	& \ac{f}.\ac{sg}  \\
		 						\cmidrule{2-4}
    \ac{nom} 	& þ-ata 	 			  & sa  			  		& sō		    			\\
    \ac{acc}	& þ-ata    	   		& þ-ana  	  	 		& þ-ō     				\\
    \ac{dat} 	& þ-amma 		   		& þ-amma  				& þ-i-z-ái  			\\
		\bottomrule
    					& \ac{n}.\ac{pl}	& \ac{m}.\ac{pl}	& \ac{f}.\ac{pl}	\\
						    \cmidrule{2-4}
    \ac{nom} 	& þ-ō		     			&	þ-ái   					&	þ-ōs	  				\\
    \ac{acc} 	& þ-ō    					&	þ-ans   				&	þ-ōs	   				\\
    \ac{dat} 	& þ-áim   				&	þ-áim    				&	þ-áim   				\\
    \bottomrule
		\end{tabular}
\end{table}

The suffixes that appear on demonstratives are also found on 3\tsc{sg} pronouns. The only difference is that the demonstratives attach to a \tit{þ(a?)}-stem and the pronouns attach to an \tit{i}-stem. This does not hold for all forms, some seem to be suppletive.

\begin{table}[H]
	\center
	\caption {Gothic \tsc{3sg} pronouns}
		\begin{tabular}{cccc}
		\toprule
							& \ac{n}.\ac{sg} 	& \ac{m}.\ac{sg}	& \ac{f}.\ac{sg}  \\
		 						\cmidrule{2-4}
    \ac{nom} 	& i-ta   	 			  & i-s  			  		& si		    			\\
    \ac{acc}	& i-ta    	   		& i-na  	  	 		& i-ja     				\\
    \ac{dat} 	& i-mma 		   		& i-mma  			   	& i-z-ái  	  		\\
		\bottomrule
    					& \ac{n}.\ac{pl}	& \ac{m}.\ac{pl}	& \ac{f}.\ac{pl}	\\
						    \cmidrule{2-4}
    \ac{nom} 	& i-ja  					&	eis    					&	i-jōs  					\\
    \ac{acc} 	& i-ja   					&	i-ns    				&	i-jōs 					\\
    \ac{dat} 	& i-m     				&	i-m      				&	i-m     				\\
    \bottomrule
		\end{tabular}
\end{table}




\subsection{Morphemes}

d-pronouns, wh-pronouns

give this big table and and show which features they express with gray

first phi and case features that form a bundle
then wh

So we also have a plural the

and here there is the complementizer extra

\subsection{The light head}

Which features are contained in the light head?

German \tit{das, was} - \tit{an dem}
\tit{as, was} - \tit{am}

so d is lacking

which feature is missing there? Florian Schwarz


\subsection{Features}




\section{The spellout procedure}

\ex. \tbf{Spellout Algorithm:}\\
Merge F and \label{ex:spellout}
 \a. Spell out FP.
 \b. If (a) fails, attempt movement of the spec of the complement of \tsc{f}, and retry (a).
 \b. If (b) fails, move the complement of \tsc{f}, and retry (a).

When a new match is found, it overrides previous spellouts.

\ex. \tbf{Cyclic Override} \citep{starke2018}:\\
Lexicalisation at a node XP overrides any previous match at a phrase contained in XP.

If the spellout procedure in \ref{ex:spellout} fails, backtracking takes place.

\ex. \tbf{Backtracking} \citep{starke2018}:\\
When spellout fails, go back to the previous cycle, and try the next option for that cycle.\label{ex:backtracking}

If backtracking also does not help, a specifier is constructed.

\ex. \tbf{Spec Formation} \citep{starke2018}:\\
If Merge F has failed to spell out (even after backtracking), try to spawn a new derivation providing the feature F and merge that with the current derivation, projecting the feature F at the top node.\label{ex:specformation}

\ex. Merge F, Move XP, Merge XP


illustrate this by building the pronouns
