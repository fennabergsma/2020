% !TEX root = thesis.tex

\chapter{Two attested patterns}

In Part \ref{part:case-facts} of this dissertation, I discussed a first aspect of case competition in headless relatives. There is a fixed scale that determines which case wins the case competition. This is the same case scale crosslinguistically. I repeat the case scale from Chapter \ref{ch:recurring} in \ref{ex:case-scale-two-patterns}.

\ex. \ac{nom} < \ac{acc} < \ac{dat}\label{ex:case-scale-two-patterns}

Also in Part \ref{part:case-facts}, I argued that a cumulative case decomposition can derive the case scale. This does not only hold for case competition in headless relatives, but also for syncretism patterns and morphological case containment patterns. In a cumulative case composition, the scale in \ref{ex:case-scale-two-patterns} can be interpreted as follows: an accusative contains all features a nominative contains plus one more. Similarly, a dative contains all features an accusative contains plus one. Therefore, a dative can be considered more complex than an accusative, and a accusative complex more than a nominative. In line with that, I refer to cases more to the right on the case scale as more complex cases than cases more to the left on the scale.

This part of the dissertation, Part \ref{part:variation}, focuses on a second aspect to headless relatives. This part is not stable crosslinguistically, but it differs across languages. Languages differ in whether they allow the internal case (the case from the relative clause) or the external case (the case from the main clause) to surface when they win the case competition. Metaphorically speaking, even though a case wins the case competition, it is a second matter whether it is allowed to come forward as a winner. Three patterns are logically possible for languages: (1) the internal and external case are both allowed to surface, (2) only the internal case is allowed to surface, and the external case is not, and (3) only the external case is allowed to surface, and the internal case is not.\footnote{
Theoretically, there a is fourth possibility: a language that lets neither the internal nor the external case surface when it wins. On the surface this language cannot be distinguished from a language that does not have case competition. In this section I do not discuss this type of language. I come back to it in Chapter \ref{ch:relativization}.
}
I show in this chapter that only two of the three patterns are attested across languages.

The next section introduces the patterns that are logically possible in languages with case competition. Section \ref{sec:pattern-i} to Section \ref{sec:pattern-iii} discuss the patterns one by one, and I give examples when the pattern is attested. In Section \ref{sec:potentiel-counterexamples} I make a sidestep to potential counterexamples to my generalization. I argue that these cases are actually languages that lack case competition to begin with.


\section{Three possible patterns}\label{sec:possible-patterns}

Case competition has two aspects. The first aspect is the topic of Part \ref{part:case-facts} of the dissertation. It concerns which case wins the case competition. This is decided by the same case scale for all languages. The second aspects is the topic of Part \ref{part:variation} of the dissertation. This one concerns whether the case that wins the case competition is actually allowed to surface. It namely differs per language whether they allow the internal and the external case do so.

Metaphorically, the second aspect can be described as a language-specific approval committee. The committee learns (from the first aspect) which case has wins the case competition. Then it can either approve this case or not approve it. This approval happens based on where the winning case comes from: (1) from internal to the relative clause (internal) or (2) external to the relative clause (external). It is determined per language whether it approves the internal case, the external case or both of them. The approval committee can only approve the winner of the competition or deny it, it cannot propose an alternative winner. In this metaphor, the approval of the committee means that a particular case is allowed to surface. If the case is not allowed to surface, the headless relative as a whole is ungrammatical.

Taking this all together, there are three patterns possible in languages. First, both the internal case and the external case are allowed to surface. Second, only the internal case is allowed to surface, and the external case is not. Third, only the external case is allowed to surface, and the internal case is not. In what follows, I introduce these possible patterns.

In Chapter \ref{ch:recurring}, I discussed headless relatives in Gothic. In this language, both the internal case and the external case are both allowed to surface when they win the case competition. I repeat a summary of the Gothic data in \ref{tbl:summary-gothic-repeated}. The examples corresponding to the cells in the table can be found in Section \ref{sec:pattern-rels}.

The left column shows the internal case between square brackets. The upper row shows the external case between square brackets. The other cells indicate the case of the relative pronoun. The diagonal is left blank, because these are instances in which the internal and external case match, and there is no case competition taking place.
The three cells in the lower left corner are the situations in which the internal case surfaces when it wins the competition. The three cells in the upper right corner are the situations in which the external case surfaces when it wins the competition. All these instances are grammatical in Gothic.
This shows that Gothic allows the internal and external case to surface when they win the case competition.

\begin{table}[H]
  \center
  \caption{Summary of Gothic headless relatives (repeated)}
    % !TEX root = ../thesis.tex

\begin{tabular}{c|c|c|c}
  \toprule
      \textsubscript{\ac{int}} \textsuperscript{\ac{ext}}
        & [\ac{nom}]
        & [\ac{acc}]
        & [\ac{dat}]
        \\ \cmidrule{1-4}
    [\ac{nom}]
        & \ac{nom}
        & \ac{acc}
        & \ac{dat}
        \\ \cmidrule{1-4}
    [\ac{acc}]
        & \ac{acc}
        & \ac{acc}
        & \ac{dat}
        \\ \cmidrule{1-4}
    [\ac{dat}]
        & \ac{dat}
        & (\ac{dat})
        & \ac{dat}
        \\
  \bottomrule
\end{tabular}

    \label{tbl:summary-gothic-repeated}
\end{table}

The second possible pattern is that of a language allows that the internal case to surface when it wins the case competition, but it does not allow the external case to. In this type of language, only the internal case gets to surface if it is more complex than the external one. If the external case is more complex, it is not allowed to surface, and the headless relative construction is ungrammatical.

Table \ref{tbl:case-competition-only-int} illustrates what a the pattern for such a language looks like. Only half of situations from Gothic remains. The other half is ungrammatical.
The three cells in the lower left corner are the situations in which the internal case surfaces when it wins the competition. Just like in Gothic, these instances are grammatical.
The three cells in the upper right corner are the situations in which the external case surfaces when it wins the competition. These instances are not grammatical for this type of language. The reasoning behind that is that the language does not allow the external case to surface when it wins the case competition.

\begin{table}[H]
  \center
  \caption{Only internal case allowed}
  \begin{tabular}{c|c|c|c}
    \toprule
    \textsubscript{\ac{int}} \textsuperscript{\ac{ext}}
           & [\ac{nom}]
           & [\ac{acc}]
           & [\ac{dat}]
           \\ \cmidrule{1-4}
       [\ac{nom}]
           &
           & *
           & *
           \\ \cmidrule{1-4}
       [\ac{acc}]
           & \ac{acc}
           &
           & *
           \\ \cmidrule{1-4}
       [\ac{dat}]
           & \ac{dat}
           & \ac{dat}
           &
           \\
     \bottomrule
  \end{tabular}
    \label{tbl:case-competition-only-int}
\end{table}

The third possible pattern is that of a language that allows the external case to surface when it wins the case competition, but it does not allow the internal case to. In this type of language, only the external case gets to surface if it is more complex. If the internal case is more complex, it is not allowed to surface, and the headless relative construction is ungrammatical.

Table \ref{tbl:case-competition-only-ext} illustrates what a the pattern for such a language looks like. It is the mirror image of the second type of language.
The three cells in the lower left corner are the situations in which the internal case surfaces when it wins the competition. Unlike Gothic and the second pattern, these instances are not grammatical for this type of language. The reasoning behind that is that the language does not allow the internal case to surface when it wins the case competition.
The three cells in the upper right corner are the situations in which the external case surfaces when it wins the competition. Just like in Gothic, these instances are grammatical.

\begin{table}[H]
  \center
  \caption{Only external case allowed}
  \begin{tabular}{c|c|c|c}
    \toprule
    \textsubscript{\ac{int}} \textsuperscript{\ac{ext}}
           & [\ac{nom}]
           & [\ac{acc}]
           & [\ac{dat}]
           \\ \cmidrule{1-4}
       [\ac{nom}]
           &
           & \ac{acc}
           & \ac{dat}
           \\ \cmidrule{1-4}
       [\ac{acc}]
           & *
           &
           & \ac{dat}
           \\ \cmidrule{1-4}
       [\ac{dat}]
           & *
           & *
           &
           \\
     \bottomrule
  \end{tabular}
    \label{tbl:case-competition-only-ext}
\end{table}

In this chapter I show that two of these three patterns are attested crosslinguistically. The first pattern, in which the internal and external case can surface, is the one already exemplified by Gothic in Part \ref{part:case-facts}. In Section \ref{sec:pattern-i}, I show that there is another language that shows the same pattern: Old High German. The second pattern, in which only the internal case can surface, is exemplified by Modern German in Section \ref{sec:pattern-ii}. To my knowledge, there is no language in which only the external case can surface when it wins the case competition. This is discussed in \ref{sec:pattern-iii}.

\section{Internal and external case allowed}\label{sec:pattern-i}

This section discusses the situation in which the internal case and the external case are allowed to surface when they win the case competition. In Section \ref{sec:possible-patterns} I showed that Gothic exemplifies this pattern. I summarized the findings, repeated from Chapter \ref{ch:recurring} \citep{harbert1978}. In this section I show another language that shows the case pattern: Old High German. The presented data is the result of my own research.

In Old High German, the relative pronoun is allowed to surface in the internal or external case. This conclusion follows from my own research of the texts Der althochdeutsche Isidor, The Monsee fragments, Otfrid's Evangelienbuch and Tatian in ANNIS \citep{krause2016}.\footnote{
Old High German is widely discussed in the literature because of its case attraction in headed relatives \citep[cf.][]{pittner1995}, a phenomenon that seems related to case competition in headless relatives (see Section \ref{sec:attraction} for why attraction is not further discussed in this dissertation).
A common observation is that case attraction in headed relatives in Old High German adheres to the case scale. The same is claimed for headless relatives.
What, to my knowledge, has not been studied systematically is whether Old High German headless relative allow the internal or the external case to surface when it wins the case competition. This is what I investigated in my work.
}
The examples follow the spelling and the detailed glosses in ANNIS. The translations are my own.

I start with the competition between accusative and nominative. Following the case scale, the relative pronoun appears in accusative case and never in nominative. As Old High German allows the internal and external case to surface, the accusative surfaces if it is the internal case and if it is the external case.

Consider the example in \ref{ex:ohg-acc-nom}. In this example, the internal nominative case competes against the external accusative case.
The internal case is nominative, as the predicate \tit{gisizzen} `to possess' takes nominative subjects.
The external case is accusative, as the predicate \tit{bibringan} `to create' takes accusative objects.
The relative pronoun \tit{dhen} `\ac{rel}.\ac{sg}.\ac{m}.\ac{acc}' appears in the external case: the accusative. The relative pronoun is not marked in bold, just like as the main clause, showing that the relative pronoun patterns with the main clause.\footnote{
At the end of this section I discuss a counterexample to the case scale, in which the internal case is nominative, the external case is accusative, and the relative pronoun appears in the nominative case.
}

\exg. ih bibringu fona iacobes samin endi fona iuda dhen \tbf{mina} \tbf{berga} \tbf{chisitzit}\\
1\ac{sg}.\ac{nom} {create}.\ac{pres}.1\ac{sg}\scsub{[acc]} of Jakob.\ac{gen} seed.\ac{sg}.\ac{dat} and of Judah.\ac{dat} \ac{rel}.\ac{sg}.\ac{m}.\ac{acc} my.\ac{acc}.\ac{m}.\ac{pl} mountain.\ac{acc}.\ac{pl} possess.\ac{pres}.3\ac{sg}\scsub{[nom]}\\
`I create of the seed of Jacob and of Judah the one, who possess my mountains' \flushfill{Old High German, \ac{isid} 34:3}\label{ex:ohg-acc-nom}

Consider the example in \ref{ex:ohg-nom-acc}. In this example, the internal accusative case competes against the external nominative case.
The internal case is accusative, as the predicate \tit{zellen} `to tell' takes accusative objects.
The external case is nominative, as the predicate \tit{sin} `to be' takes nominative objects.
The relative pronoun \tit{then} `\ac{rel}.\ac{sg}.\ac{m}.\ac{acc}' appears in the internal case: the accusative. The relative pronoun is marked in bold, just like as the relative clause, showing that the relative pronoun patterns with the relative clause.
Examples in which the internal case is accusative, the external case is nominative and the relative pronoun appears in nominative case are unattested.

\exg. thíz ist \tbf{then} \tbf{sie} \tbf{zéllent}\\
\ac{dem}.\ac{sg}.\ac{n}.\ac{nom} be.\ac{pres}.3\ac{sg}\scsub{[nom]} \ac{rel}.\ac{sg}.\ac{m}.\ac{acc} 3\ac{pl}.\ac{m}.\ac{nom} tell.\ac{pres}.3\ac{pl}\scsub{[acc]}\\
`this is the one whom they talk about' \flushfill{Old High German, \ac{otfrid} III 16:50}\label{ex:ohg-nom-acc}

I continue with the competition between dative and nominative. Following the case scale, the relative pronoun appears in dative case and never in nominative. As Old High German allows the internal and external case to surface, the dative surfaces if it is the internal case and if it is the external case.

Consider the example in \ref{ex:ohg-dat-nom}. In this example, the internal nominative case competes against the external dative case.
The internal case is nominative, as the predicate \tit{sprehhan} `to speak' takes nominative subjects.
The external case is dative, as the predicate \tit{antwurten} `to reply' takes dative objects.
The relative pronoun \tit{demo} `\ac{rel}.\ac{sg}.\ac{m}.\ac{dat}' appears in the external case: the dative. The relative pronoun is not marked in bold, just like as the main clause, showing that the relative pronoun patterns with the main clause.
Examples in which the internal case is nominative, the external case is dative and the relative pronoun appears in nominative case are unattested.

\exg. enti aer {ant uurta} demo \tbf{zaimo} \tbf{sprah}\\
and 3\ac{sg}.\ac{m}.\ac{nom} reply.\ac{pst}.3\ac{sg}\scsub{[dat]} \ac{rel}.\ac{sg}.\ac{m}.\ac{dat} {to 3\ac{sg}.\ac{m}.\ac{dat}} speak.\ac{pst}.3\ac{sg}\scsub{[nom]}\\
`and he replied to the one who spoke to him' \flushfill{Old High German, \ac{mons} 7:24, adapted from \pgcitealt{pittner1995}{199}}\label{ex:ohg-dat-nom}

Consider the example in \ref{ex:ohg-nom-dat}. In this example, the internal dative case competes against the external nominative case.
The internal case is dative, as the predicate \tit{forlazan} `to read' takes dative indirect objects.
The external case is nominative, as the predicate \tit{minnon} `to love' takes nominative subjects.
The relative pronoun \tit{themo} `\ac{rel}.\ac{sg}.\ac{m}.\ac{dat}' appears in the internal case: the dative. The relative pronoun is marked in bold, just like as the relative clause, showing that the relative pronoun patterns with the relative clause.
Examples in which the internal case is dative, the external case is nominative and the relative pronoun appears in nominative case are unattested.

\exg. \tbf{themo} \tbf{min} \tbf{uuirdit} \tbf{forlazan}, min minnot\\
\ac{rel}.\ac{sg}.\ac{m}.\ac{dat} less become.\ac{pres}.3\ac{sg} read.\ac{inf}\scsub{[dat]} less love.\ac{pres}.3\ac{sg}\scsub{[nom]}\\
`to whom less is read, loves less' \flushfill{Old High German, \ac{tatian} 138:13}\label{ex:ohg-nom-dat}

I end with the competition between dative and accusative. Following the case scale, the relative pronoun appears in dative case and never in accusative. As Old High German allows the internal and external case to surface, the dative surfaces if it is the internal case and if it is the external case.

Consider the example in \ref{ex:ohg-dat-acc}. In this example, the internal accusative case competes against the external dative case.
The internal case is nominative, as the predicate \tit{zellen} `to tell' takes accusative objects.
The external case is dative, as the comparative of the adjective \tit{furiro} `great' takes dative objects.
The relative pronoun \tit{thên} `\ac{rel}.\ac{pl}.\ac{m}.\ac{dat}' appears in the external case: the dative. The relative pronoun is not marked in bold, just like as the main clause, showing that the relative pronoun patterns with the main clause.
Examples in which the internal case is accusative, the external case is dative and the relative pronoun appears in accusative case are unattested.

\exg. bis -tú nu {zi wáre} furira Ábrahame? ouh thén \tbf{man} \tbf{hiar} \tbf{nu} \tbf{zálta}\\
be.\ac{pres}.2\ac{sg} -2\ac{sg}.\ac{nom} now truly {great}.\ac{cmpr}\scsub{[dat]} Abraham.\ac{dat} and \ac{rel}.\ac{pl}.\ac{m}.\ac{dat} one.\ac{nom}.\ac{m}.\ac{sg} here now tell.\ac{pst}.3\ac{sg}\scsub{[acc]}\\
`are you now truly greater than Abraham? and than those, who one talked about here now' \flushfill{Old High German, \ac{otfrid} III 18:33}\label{ex:ohg-dat-acc}

--there is no instance of internal dative and external accusative with two predicates. possibly an example with a dative preposition will be inserted here--

A summary of the data is given in \ref{tbl:summary-old-high-german}.
The three cells in the lower left corner are the situations in which the internal case surfaces when it wins the competition. They correspond to the examples \ref{ex:ohg-nom-acc}, \ref{ex:ohg-nom-dat} and \ref{ex:ohg-acc-dat}.
The three cells in the upper right corner are the situations in which the external case surfaces when it wins the competition. They correspond to the examples \ref{ex:ohg-acc-nom}, \ref{ex:ohg-dat-nom} and \ref{ex:ohg-dat-acc}.

\begin{table}[H]
  \center
  \caption{Summary of Old High German headless relatives}
  \begin{tabular}{c|c|c|c}
    \toprule
        \textsubscript{\ac{int}} \textsuperscript{\ac{ext}}
          & [\ac{nom}]
          & [\ac{acc}]
          & [\ac{dat}]
          \\ \cmidrule{1-4}
      [\ac{nom}]
          &
          & \ac{acc}
          & \ac{dat}
          \\ \cmidrule{1-4}
      [\ac{acc}]
          & \ac{acc}
          &
          & \ac{dat}
          \\ \cmidrule{1-4}
      [\ac{dat}]
          & \ac{dat}
          & \ac{dat}
          &
          \\
    \bottomrule
  \end{tabular}
    \label{tbl:summary-old-high-german}
\end{table}

In my research I encountered a single counterexample to the pattern I just described.
Consider the example in \ref{ex:ohg-counterexample}. In this example, the internal nominative case competes against the external accusative case.
The internal case is nominative, as the predicate \tit{giheilen} `to save' takes nominative subjects.
The external case is accusative, as the predicate \tit{beran} `to bear' takes accusative objects.
Surprisingly, the relative pronoun \tit{thér} `\ac{rel}.\ac{sg}.\ac{m}.\ac{nom}' appears in the internal case: the nominative, which is the less complex of the two cases. The relative pronoun is marked in bold, just like as the relative clause, showing that the relative pronoun patterns with the relative clause.

\exg. tház si uns béran scolti \tbf{thér} \tbf{unsih} \tbf{gihéilti}\\
 that 3\ac{sg}.\ac{f}.\ac{nom} 1\ac{pl}.\ac{dat} bear.\ac{inf}\scsub{[acc]} should.\ac{subj}.\ac{pst}.3\ac{sg} \ac{rel}.\ac{sg}.\ac{m}.\ac{nom} 1\ac{pl}.\ac{acc} save.\ac{sbjv}.\ac{pst}.3\ac{sg}\scsub{[nom]}\\
 `that she should have beared for us the one, who had saved us' \flushfill{Old High German, \ac{otfrid} I 3:38}\label{ex:ohg-counterexample}

This example is unexpected, because the less complex case (the nominative) wins and not the more complex case (the accusative).
The only explanation for this I can see is a functional one. The \tit{thér} `\ac{rel}.\ac{sg}.\ac{m}.\ac{nom}' in \ref{ex:ohg-counterexample} refers to Jesus. In the relative clause he is the subject of \tit{unsih gihéilti} `had saved us', hence the internal nominative case. In the main clause he is the object of \tit{tház si uns béran scolti} `that she should have beared', hence the external accusative case.
Letting the relative pronoun surface in the internal case could be interpreted as emphasizing the role of Jesus as a savior, rather than him being the object of being given birth to. In line with that reasoning, it is expected that certain grammatical facts more often deviate from regular patterns if Jesus is involved. I leave investigating this prediction for future research.
Of course, this does not answer the question of what happens to the accusative case required by the external predicate. It also does not explain why not another emphasizing strategy is used, for instance forming a light-headed relative, which would leave space for two cases.
I acknowledge this example as a counterexample to the pattern I describe, but I do not change my generalization, as this is a single occurrence.

Leaving the counterexample aside, I conclude that Gothic and Old High German are both instances of languages that allow the internal and the external case to surface. The relative pronoun surfaces simply in the case that wins the case competition.


\section{Only internal case allowed}\label{sec:pattern-ii}

This section discusses the situation in which only the internal case is allowed to surface when it wins the case competition. When the external case wins the case competition, the result is ungrammatical. Schematically, this looks as in Table \ref{tbl:case-competition-only-int-repeated}.

\begin{table}[H]
  \center
  \caption{Only internal case allowed}
  \begin{tabular}{c|c|c|c}
    \toprule
    \textsubscript{\ac{int}} \textsuperscript{\ac{ext}}
           & [\ac{nom}]
           & [\ac{acc}]
           & [\ac{dat}]
           \\ \cmidrule{1-4}
       [\ac{nom}]
           &
           & *
           & *
           \\ \cmidrule{1-4}
       [\ac{acc}]
           & \ac{acc}
           &
           & *
           \\ \cmidrule{1-4}
       [\ac{dat}]
           & \ac{dat}
           & \ac{dat}
           &
           \\
     \bottomrule
  \end{tabular}
    \label{tbl:case-competition-only-int-repeated}
\end{table}

An example of a language that shows this pattern is Modern German. In this section I discuss the Modern German data, based on the research of \citet{vogel2001}.

I start with the competition between accusative and nominative. Following the case scale, the relative pronoun appears in accusative case and never in nominative. Following the internal-only requirement, only if the accusative case is the internal case, the sentence is grammatical.

Consider the example in \ref{ex:mg-acc-nom}. In this example, the internal nominative case competes against the external accusative case.
The internal case is nominative, as the predicate \tit{sein} `to be' takes nominative subjects.
The external case is accusative, as the predicate \tit{einladen} `to invite' takes accusative objects.
The relative pronoun \tit{wen} `\ac{rel}.\ac{an}.\ac{acc}' appears in the external case: the accusative. The relative pronoun is not marked in bold, just like as the main clause, showing that the relative pronoun patterns with the main clause.
The example adheres to the case scale, but the more complex case (here the accusative) is not the internal case. As only the internal can win the case competition in Modern German, the example in ungrammatical.

\exg. *Ich {lade ein}, wen \tbf{mir} \tbf{sympathisch} \tbf{ist}.\\
 1\ac{sg}.\ac{nom} invite.\ac{pres}.1\ac{sg}\scsub{[acc]} \ac{rel}.\ac{an}.\ac{acc} 1\ac{sg}.\ac{dat} nice be.\ac{pres}.3\ac{sg}\scsub{[nom]}\\
 `I invite who I like.' \flushfill{Modern German, adapted from \pgcitealt{vogel2001}{344}}\label{ex:mg-acc-nom}

This example in \ref{ex:mg-acc-nom-u} is identical to \ref{ex:mg-acc-nom}, except for that the relative pronoun appears in the external less complex nominative case. This example is also ungrammatical: in addition to the more complex case not being the internal case, the relative pronoun also does not appear in the more complex case (the accusative) but in the less complex case (the nominative).

\exg. *Ich {lade ein}, wer \tbf{mir} \tbf{sympathisch} \tbf{ist}.\\
 1\ac{sg}.\ac{nom} invite.\ac{pres}.1\ac{sg}\scsub{[acc]} \ac{rel}.\ac{an}.\ac{nom} 1\ac{sg}.\ac{dat} nice be.\ac{pres}.3\ac{sg}\scsub{[nom]}\\
 `I invite who I like.' \flushfill{Modern German, adapted from \pgcitealt{vogel2001}{344}}\label{ex:mg-acc-nom-u}

Consider the example in \ref{ex:mg-nom-acc}. In this example, the internal accusative case competes against the external nominative case.
The internal case is accusative, as the predicate \tit{mögen} `to like' takes accusative objects.
The external case is nominative, as the predicate \tit{besuchen} `to visit' takes nominative subjects.
The relative pronoun \tit{wen} `\ac{rel}.\ac{an}.\ac{acc}' appears in the internal case: the accusative. The relative pronoun is marked in bold, just like as the relative clause, showing that the relative pronoun patterns with the relative clause.
The example adheres to the case scale, and the more complex case (here the accusative) is the internal case, so the example is grammatical.

\exg. Uns besucht \tbf{wen} \tbf{Maria} \tbf{mag}.\\
 2\ac{pl}.\ac{acc} visit.\ac{pres}.3\ac{sg}\scsub{[nom]} \ac{rel}.\ac{an}.\ac{acc} Maria.\ac{nom} like.\ac{pres}.3\ac{sg}\scsub{[acc]}\\
 `Who visits us, Maria likes.' \flushfill{Modern German, adapted from \pgcitealt{vogel2001}{343}}\label{ex:mg-nom-acc}

This example in \ref{ex:mg-nom-acc-u} is identical to \ref{ex:mg-nom-acc}, except for that the relative pronoun appears in the external less complex nominative case. This example is ungrammatical: although the internal case is more complex, the relative pronoun appears in the less complex case (the nominative) and not in the more complex case (the accusative).

\exg. *Uns besucht \tbf{wer} \tbf{Maria} \tbf{mag}.\\
 2\ac{pl}.\ac{acc} visit.\ac{pres}.3\ac{sg}\scsub{[nom]} \ac{rel}.\ac{an}.\ac{nom} Maria.\ac{nom} like.\ac{pres}.3\ac{sg}\scsub{[acc]}\\
 `Who visits us, Maria likes.' \flushfill{Modern German, adapted from \pgcitealt{vogel2001}{343}}\label{ex:mg-nom-acc-u}

I continue with the competition between dative and nominative. Following the case scale, the relative pronoun appears in dative case and never in nominative. Following the internal-only requirement, only if the dative case is the internal case, the sentence is grammatical.

Consider the example in \ref{ex:mg-dat-nom}. In this example, the internal nominative case competes against the external dative case.
The internal case is nominative, as the predicate \tit{mögen} `to like' takes nominative subjects.
The external case is dative, as the predicate \tit{vertrauen} `to trust' takes dative objects.
The relative pronoun \tit{wem} `\ac{rel}.\ac{an}.\ac{dat}' appears in the external case: the dative. The relative pronoun is not marked in bold, just like as the main clause, showing that the relative pronoun patterns with the main clause.
The example adheres to the case cale, but the more complex case (here the dative) is not the internal case. As only the internal can win the case competition in Modern German, the example in ungrammatical.

\exg. *Ich vertraue, wem \tbf{Hitchcock} \tbf{mag}.\\
1\ac{sg}.\ac{nom} trust.\ac{pres}.1\ac{sg}\scsub{[dat]} \ac{rel}.\ac{an}.\ac{dat} Hitchcock.\ac{acc} like.\ac{pres}.3\ac{sg}\scsub{[nom]}\\
`I trust who likes Hitchcock.' \flushfill{Modern German, adapted from \pgcitealt{vogel2001}{345}}\label{ex:mg-dat-nom}

The example in \ref{ex:mg-dat-nom-u} is identical to \ref{ex:mg-dat-nom}, except for that the relative pronoun appears in the external less complex nominative case. This example is also ungrammatical: in addition to the more complex case not being the internal case, the relative pronoun also does not appear in the more complex case (the dative) but in the less complex case (the nominative).

\exg. *Ich vertraue, wer \tbf{Hitchcock} \tbf{mag}.\\
1\ac{sg}.\ac{nom} trust.\ac{pres}.1\ac{sg}\scsub{[dat]} \ac{rel}.\ac{an}.\ac{nom} Hitchcock.\ac{acc} like.\ac{pres}.3\ac{sg}\scsub{[nom]}\\
`I trust who likes Hitchcock.' \flushfill{Modern German, adapted from \pgcitealt{vogel2001}{345}}\label{ex:mg-dat-nom-u}

Consider the example in \ref{ex:mg-nom-dat}. In this example, the internal dative case competes against the external nominative case.
The internal case is dative, as the predicate \tit{vertrauen} `to trust' takes dative objects.
The external case is nominative, as the predicate \tit{besuchen} `to visit' takes nominative subjects.
The relative pronoun \tit{wem} `\ac{rel}.\ac{an}.\ac{dat}' appears in the internal case: the dative. The relative pronoun is marked in bold, just like as the relative clause, showing that the relative pronoun patterns with the relative clause.
The example adheres to the case scale, and the more complex case (here the dative) is the internal case, so the example is grammatical.

\exg. Uns besucht \tbf{wem} \tbf{Maria} \tbf{vertraut}.\\
2\ac{pl}.\ac{acc} visit.\ac{pres}.3\ac{sg}\scsub{[nom]} \ac{rel}.\ac{an}.\ac{dat} Maria.\ac{nom} trust.\ac{pres}.3\ac{sg}\scsub{[dat]}\\
`Who visits us, Maria trusts.' \flushfill{Modern German, adapted from \pgcitealt{vogel2001}{343}}\label{ex:mg-nom-dat}

This example in \ref{ex:mg-nom-dat-u} is identical to \ref{ex:mg-nom-dat}, except for that the relative pronoun appears in the external less complex nominative case. This example is ungrammatical: although the internal case is more complex, the relative pronoun appears in the less complex case (the nominative) and not in the more complex case (the dative).

\exg. *Uns besucht \tbf{wer} \tbf{Maria} \tbf{vertraut}.\\
2\ac{pl}.\ac{acc} visit.\ac{pres}.3\ac{sg}\scsub{[nom]} \ac{rel}.\ac{an}.\ac{nom} Maria.\ac{nom} trust.\ac{pres}.3\ac{sg}\scsub{[dat]}\\
`Who visits us, Maria trusts.' \flushfill{Modern German, adapted from \pgcitealt{vogel2001}{343}}\label{ex:mg-nom-dat-u}

I end with the competition between dative and accusative. Following the case scale, the relative pronoun appears in dative case and never in accusative. Following the internal-only requirement, only if the dative case is the internal case, the sentence is grammatical.

Consider the example in \ref{ex:mg-dat-acc}. In this example, the internal accusative case competes against the external dative case.
The internal case is accusative, as the predicate \tit{mögen} `to like' takes accusative objects.
The external case is dative, as the predicate \tit{vertrauen} `to trust' takes dative objects.
The relative pronoun \tit{wem} `\ac{rel}.\ac{an}.\ac{dat}' appears in the external case: the dative. The relative pronoun is not marked in bold, just like as the main clause, showing that the relative pronoun patterns with the main clause.
The example adheres to the case scale, but the more complex case (here the dative) is not the internal case. As only the internal can win the case competition in Modern German, the example in ungrammatical.

\exg. *Ich vertraue wem \tbf{auch} \tbf{Maria} \tbf{mag}. \\
1\ac{sg}.\ac{nom} trust.\ac{pres}.1\ac{sg}\scsub{[dat]} \ac{rel}.\ac{an}.\ac{dat} also Maria.\ac{nom} like.\ac{pres}.3\ac{sg}\scsub{[acc]}.\\
`I trust whoever Maria also likes.' \flushfill{Modern German, adapted from \pgcitealt{vogel2001}{345}}\label{ex:mg-dat-acc}

The example in \ref{ex:mg-dat-acc-u} is identical to \ref{ex:mg-dat-acc}, except for that the relative pronoun appears in the external less complex accusative case. This example is also ungrammatical: in addition to the more complex case not being the internal case, the relative pronoun also does not appear in the more complex case (the dative) but in the less complex case (the accusative).

\exg. *Ich vertraue wen \tbf{auch} \tbf{Maria} \tbf{mag}. \\
1\ac{sg}.\ac{nom} trust.\ac{pres}.1\ac{sg}\scsub{[dat]} \ac{rel}.\ac{an}.\ac{acc} also Maria.\ac{nom} like.\ac{pres}.3\ac{sg}\scsub{[acc]}.\\
`I trust whoever Maria also likes.' \flushfill{Modern German, adapted from \pgcitealt{vogel2001}{345}}\label{ex:mg-dat-acc-u}

Consider the example in \ref{ex:mg-acc-dat}. In this example, the internal dative case competes against the external accusative case.
The internal case is dative, as the predicate \tit{vertrauen} `to trust' takes dative objects.
The external case is accusative, as the predicate \tit{einladen} `to invite' takes accusative objects.
The relative pronoun \tit{wem} `\ac{rel}.\ac{an}.\ac{dat}' appears in the internal case: the dative. The relative pronoun is marked in bold, just like as the relative clause, showing that the relative pronoun patterns with the relative clause.
The example adheres to the case scale, and the more complex case (here the dative) is the internal case, so the example is grammatical.

\exg. Ich {lade ein} \tbf{wem} \tbf{auch} \tbf{Maria} \tbf{vertraut}. \\
1\ac{sg}.\ac{nom} invite.\ac{pres}.1\ac{sg}\scsub{[acc]} \ac{rel}.\ac{an}.\ac{dat} also Maria.\ac{nom} trust.\ac{pres}.3\ac{sg}\scsub{[dat]}.\\
`I invite whoever Maria also trusts.' \flushfill{Modern German, adapted from \pgcitealt{vogel2001}{344}}\label{ex:mg-acc-dat}

This example in \ref{ex:mg-acc-dat-u} is identical to \ref{ex:mg-acc-dat}, except for that the relative pronoun appears in the external less complex accusative case. This example is ungrammatical: although the internal case is more complex, the relative pronoun appears in the less complex case (the accusative) and not in the more complex case (the dative).

\exg. *Ich {lade ein} \tbf{wen} \tbf{auch} \tbf{Maria} \tbf{vertraut}. \\
1\ac{sg}.\ac{nom} invite.\ac{pres}.1\ac{sg}\scsub{[acc]} \ac{rel}.\ac{an}.\ac{acc} also Maria.\ac{nom} trust.\ac{pres}.3\ac{sg}\scsub{[dat]}.\\
`I invite whoever Maria also trusts.' \flushfill{Modern German, adapted from \pgcitealt{vogel2001}{344}}\label{ex:mg-acc-dat-u}

A summary of the data is given in \ref{tbl:summary-modern-german}.
The three cells in the lower left corner are the situations in which the internal case surfaces when it wins the competition. They correspond to the examples \ref{ex:mg-nom-acc}, \ref{ex:mg-nom-dat} and \ref{ex:mg-acc-dat}.
The three cells in the upper right corner are the situations in which the external case surfaces when it wins the competition. As Modern German does not allow the winner of the case competition to surface when it appears in the external case, these instances are ungrammatical. They correspond to the examples \ref{ex:mg-acc-nom}, \ref{ex:mg-dat-nom} and \ref{ex:mg-dat-acc}.

\begin{table}[H]
  \center
  \caption{Summary of Modern German headless relatives}
  \begin{tabular}{c|c|c|c}
    \toprule
   \textsubscript{\ac{int}} \textsuperscript{\ac{ext}}
          & [\ac{nom}]
          & [\ac{acc}]
          & [\ac{dat}]
          \\ \cmidrule{1-4}
      [\ac{nom}]
          &
          & *
          & *
          \\ \cmidrule{1-4}
      [\ac{acc}]
          & \ac{acc}
          &
          & *
          \\ \cmidrule{1-4}
      [\ac{dat}]
          & \ac{dat}
          & \ac{dat}
          &
          \\
    \bottomrule
  \end{tabular}
  \label{tbl:summary-modern-german}
\end{table}

In sum, Modern German is an instance of a language that allows only the internal case to surface. The relative pronoun surfaces in the more complex case, but only if this more complex case is the internal case.

\section{Only external case allowed}\label{sec:pattern-iii}

This section discusses the situation in which only the external case is allowed to surface when it wins the case competition. When the internal case wins the case competition, the result is ungrammatical. Schematically, this looks as in Table \ref{tbl:case-competition-only-ext-repeated}.

\begin{table}[H]
  \center
  \caption{Only external case allowed}
  \begin{tabular}{c|c|c|c}
    \toprule
    \textsubscript{\ac{int}} \textsuperscript{\ac{ext}}
           & [\ac{nom}]
           & [\ac{acc}]
           & [\ac{dat}]
           \\ \cmidrule{1-4}
       [\ac{nom}]
           &
           & \ac{acc}
           & \ac{dat}
           \\ \cmidrule{1-4}
       [\ac{acc}]
           & *
           &
           & \ac{dat}
           \\ \cmidrule{1-4}
       [\ac{dat}]
           & *
           & *
           &
           \\
     \bottomrule
  \end{tabular}
    \label{tbl:case-competition-only-ext-repeated}
\end{table}

To my knowledge, this pattern is not attested in any (extinct) natural language. For Ancient Greek is has been claimed in the literature that it follows this pattern. I show that Ancient Greek actually patterns with Gothic and Old High German.

It has been claimed that Ancient Greek only allows the external case to surface when it wins the case competition  \citep[cf.][]{cinqueforthcoming}. It does indeed seem to be the case that examples in which the external case wins over the internal case are more frequent than examples in which the internal case wins over the external case (see \citealt{kakarikos2014} for numerous examples of cases).\footnote{
In this dissertation I do not address the question of why certain constructions and configurations are more frequent than others. My goal is to set up a system that generates the grammatical patterns and excludes the ungrammatical or unattested patterns.
} I give an example of such a situation, in which a more complex external case wins over a less complex internal case.

Consider the example in \ref{ex:ag-dat-acc}. In this example, the internal accusative case competes against the external dative case.
The internal case is accusative, as the predicate \tit{tíktō} `to give birth to' takes accusative objects.
The external case is dative, as the predicate \tit{ékhō} `to provide' takes dative indirect objects.
The relative pronoun \tit{hō̃ͅ} `\ac{rel}.\ac{sg}.\ac{m}.\ac{acc}' appears in the internal case: the accusative. The relative pronoun is marked in bold, just like as the relative clause, showing that the relative pronoun patterns with the relative clause.

\exg. pãn {tò tekòn} trophḕn ékhei hō̃ͅ \tbf{án} \tbf{tékēͅ}\\
any parent.\ac{sg}.\ac{nom} food.\ac{sg}.\ac{acc} provide.\ac{pres}.3\ac{sg} \ac{rel}.\ac{sg}.\ac{m}.\ac{dat} \ac{mod} {gives birth}.\ac{aor}.3\ac{sg}\\
`any parent provides food to what he would have given birth to' \flushfill{Ancient Greek, \ac{pl.men} 237e, adapted from \pgcitealt{kakarikos2014}{292}}\label{ex:ag-dat-acc}

This example is compatible with the picture of Ancient Greek only allowing the external case to surface when they win the competition. In Table \ref{tbl:case-competition-ag-poss1}, I marked the example \ref{ex:ag-dat-acc} in gray in the external-only pattern taken from the beginning of this section.

\begin{table}[H]
  \center
  \caption{Ancient Greek possibility 1}
  \begin{tabular}{c|c|c|c}
    \toprule
    \textsubscript{\ac{int}} \textsuperscript{\ac{ext}}
           & [\ac{nom}]
           & [\ac{acc}]
           & [\ac{dat}]
           \\ \cmidrule{1-4}
       [\ac{nom}]
           &
           & \ac{acc}
           & \ac{dat}
           \\ \cmidrule{1-4}
       [\ac{acc}]
           & *
           &
           & \cellcolor{LG}\ac{dat}
           \\ \cmidrule{1-4}
       [\ac{dat}]
           & *
           & *
           &
           \\
     \bottomrule
  \end{tabular}
    \label{tbl:case-competition-ag-poss1}
\end{table}

However, the example is also compatible with the picture of Ancient Greek allowing the internal and the external case to surface when they win the case competition. In Table \ref{tbl:case-competition-ag-poss2}, I marked the example \ref{ex:ag-dat-acc} in gray in the internal-and-external pattern taken from the beginning of this section.

\begin{table}[H]
  \center
  \caption{Ancient Greek possibility 2}
  \begin{tabular}{c|c|c|c}
    \toprule
    \textsubscript{\ac{int}} \textsuperscript{\ac{ext}}
           & [\ac{nom}]
           & [\ac{acc}]
           & [\ac{dat}]
           \\ \cmidrule{1-4}
       [\ac{nom}]
           &
           & \ac{acc}
           & \ac{dat}
           \\ \cmidrule{1-4}
       [\ac{acc}]
           & \ac{acc}
           &
           & \cellcolor{LG}\ac{dat}
           \\ \cmidrule{1-4}
       [\ac{dat}]
           & \ac{dat}
           & \ac{dat}
           &
           \\
     \bottomrule
  \end{tabular}
    \label{tbl:case-competition-ag-poss2}
\end{table}

What sets Table \ref{tbl:case-competition-ag-poss1} and Table \ref{tbl:case-competition-ag-poss2} apart is the lower left corner of the table. These are cases in which the internal case wins the case competition.
In Table \ref{tbl:case-competition-ag-poss1} these examples are not allowed to surface, and in Table \ref{tbl:case-competition-ag-poss2} they are.
I give an example in which a more complex internal case wins over a less complex external case. This indicates that Ancient Greek cannot be of the type shown in Table \ref{tbl:case-competition-ag-poss1}, but is has to be of the type shown in Table \ref{tbl:case-competition-ag-poss2}.

Consider the example in \ref{ex:ag-nom-acc}. In this example, the internal accusative case competes against the external nominative case.
The internal case is accusative, as the predicate \tit{philéō} `to love' takes accusative objects.
The external case is nominative, as the predicate \tit{apothnḗiskō} `to die' takes nominative subjects.
The relative pronoun \tit{hòn} `\ac{rel}.\ac{sg}.\ac{m}.\ac{acc}' appears in the internal case: the accusative. The relative pronoun is marked in bold, just like as the relative clause, showing that the relative pronoun patterns with the relative clause.\footnote{
The sentence in \ref{ex:ag-nom-acc} can also be analyzed as a headed relative, in which the relative clause modified the phonologically empty subject of \tit{apothnḗiskō} `to die'. Then, however, more needs to be said about how it is possible for a relative clause to modify a phonologically empty element.
}

\exg. \tbf{hòn} \tbf{hoi} \tbf{theoì} \tbf{philoũsin} apothnḗͅskei néos\\
\ac{rel}.\ac{sg}.\ac{m}.\ac{acc} the god.\ac{pl} love.3\ac{pl}\scsub{acc} die.3\ac{sg}\scsub{nom} young\\
`He, whom the gods love, dies young.' \flushfill{Ancient Greek, \ac{men.dd}, 125}\label{ex:ag-nom-acc}

This example shows that Ancient Greek is not an instance of the third possible pattern, in which only the external case is allowed to surface. Instead, as illustrated by Table \ref{tbl:case-competition-ancient-greek}, the language allows the external case (marked light grey) and the internal case (marked dark grey) to surface when they win the case competition.

\begin{table}[H]
  \center
  \caption{Only external case allowed}
  \begin{tabular}{c|c|c|c}
    \toprule
    \textsubscript{\ac{int}} \textsuperscript{\ac{ext}}
           & [\ac{nom}]
           & [\ac{acc}]
           & [\ac{dat}]
           \\ \cmidrule{1-4}
       [\ac{nom}]
           &
           & \ac{acc}
           & \ac{dat}
           \\ \cmidrule{1-4}
       [\ac{acc}]
           & \cellcolor{DG}\ac{acc}
           &
           & \cellcolor{LG}\ac{dat}
           \\ \cmidrule{1-4}
       [\ac{dat}]
           & \ac{dat}
           & \ac{dat}
           &
           \\
     \bottomrule
  \end{tabular}
    \label{tbl:case-competition-ancient-greek}
\end{table}

To my knowledge, there is no language in which only the external case is allowed to surface when it wins the case competition, and the internal case is not. Ancient Greek, which has been mentioned as an instance of this pattern, actually patterns with Gothic and Old High German in that is allows both the internal and the external case to surface.

\section{Aside: languages without case competition}\label{sec:potentiel-counterexamples}

Modern Greek and Old English: potential counterexamples.

In this chapter so far, I discussed languages that show case competition. There are also languages that do not show any case competition. In these languages, the two cases do not compete to show their case on the relative pronoun. It is irrelevant how the two cases relate to each other on the case scale. Where the case comes from determines the case of the relative pronoun: the internal case or the external case.

Table \ref{tbl:no-case-competition-int} shows the pattern of a language in which the relative pronoun always appears in the internal case. For instance, in the second row, the internal case is nominative and the external case is either accusative or dative. The relative pronoun appears in the nominative. It is irrelevant here that the nominative is less complex than the accusative and the dative. There is no case competition taking place. To my knowledge, this type is not attested in any natural language.

\begin{table}[H]
  \center
  \caption{Always internal case}
  \begin{tabular}{c|c|c|c}
    \toprule
   \textsubscript{\ac{int}} \textsuperscript{\ac{ext}}
          & [\ac{nom}]
          & [\ac{acc}]
          & [\ac{dat}]
          \\ \cmidrule{1-4}
      [\ac{nom}]
          &
          & \ac{nom}
          & \ac{nom}
          \\ \cmidrule{1-4}
      [\ac{acc}]
          & \ac{acc}
          &
          & \ac{acc}
          \\ \cmidrule{1-4}
      [\ac{dat}]
          & \ac{dat}
          & \ac{dat}
          &
          \\
    \bottomrule
  \end{tabular}
  \label{tbl:no-case-competition-int}
\end{table}

Table \ref{tbl:no-case-competition-ext} shows the pattern of a language in which the relative pronoun always appears in the external case. For instance, in the second column, the external case is nominative and the internal case is either accusative or dative. The relative pronoun appears in the nominative. It is irrelevant here that the nominative is less complex than the accusative and the dative. There is no case competition taking place.

\begin{table}[H]
  \center
  \caption{Always external case}
  \begin{tabular}{c|c|c|c}
    \toprule
   \textsubscript{\ac{int}} \textsuperscript{\ac{ext}}
          & [\ac{nom}]
          & [\ac{acc}]
          & [\ac{dat}]
          \\ \cmidrule{1-4}
      [\ac{nom}]
          &
          & \ac{acc}
          & \ac{dat}
          \\ \cmidrule{1-4}
      [\ac{acc}]
          & \ac{nom}
          &
          & \ac{dat}
          \\ \cmidrule{1-4}
      [\ac{dat}]
          & \ac{nom}
          & \ac{acc}
          &
          \\
    \bottomrule
  \end{tabular}
  \label{tbl:no-case-competition-ext}
\end{table}

In this section I discuss two languages in which the relative pronoun always appears in the external case. I show that these languages do not show any case competition. In other words, these languages are of the type in Table \ref{tbl:no-case-competition-ext} and not of the type I discussed in Section \ref{sec:pattern-iii}.





I start with Old English.

Consider the example in \ref{ex:oe-dat-nom}.
The internal case is nominative, as the predicate \tit{gegyltan} `to sin' takes nominative subjects.
The external case is accusative, as the predicate \tit{for-gifan} `to forgive' takes dative objects.
The relative pronoun \tit{ðam} `\tsc{rel}.\ac{dat}.\tsc{pl}' appears in the external case: the dative. The relative pronoun is not marked in bold, unlike the relative clause, showing that the relative pronoun patterns with the main clause.

\exg. ðæt is, ðæt man for-gife, ðam \tbf{ðe} \tbf{wið} \tbf{hine} \tbf{gegylte}\\
 that is that one forgive.\tsc{subj}.\tsc{sg}\scsub{dat} \tsc{rel}.\ac{dat}.\tsc{pl} \tsc{comp} against 3\tsc{sg.m.acc} sin.\tsc{3sg}\scsub{nom}\\
 `that is, that one₂ forgive him₁, who sins against him₂' \flushfill{Old English, adapted from \pgcitealt{harbert1983}{549}} \label{ex:oe-dat-nom}

This example is compatible with the picture of Old English being a case competition language that only allows the external case to surface. In Table \ref{tbl:oe-poss1}, I marked the example \ref{ex:oe-dat-nom} in gray in the external-only pattern taken from Section \ref{sec:pattern-iii}.

 \begin{table}[H]
   \center
   \caption{Old English possibility 1}
   \begin{tabular}{c|c|c|c}
     \toprule
     \textsubscript{\ac{int}} \textsuperscript{\ac{ext}}
            & [\ac{nom}]
            & [\ac{acc}]
            & [\ac{dat}]
            \\ \cmidrule{1-4}
        [\ac{nom}]
            &
            & \ac{acc}
            & \cellcolor{LG}\ac{dat}
            \\ \cmidrule{1-4}
        [\ac{acc}]
            & *
            &
            & \ac{dat}
            \\ \cmidrule{1-4}
        [\ac{dat}]
            & *
            & *
            &
            \\
      \bottomrule
   \end{tabular}
     \label{tbl:oe-poss1}
 \end{table}

However, the example is also compatible with the picture of Old English being a language without case competition and letting the relative pronoun appear in the external case. In Table \ref{tbl:oe-poss2}, I marked the example \ref{ex:oe-dat-nom} in gray in the always-external pattern repeated from \ref{tbl:no-case-competition-ext}.

 \begin{table}[H]
   \center
   \caption{Old English possibility 2}
   \begin{tabular}{c|c|c|c}
     \toprule
    \textsubscript{\ac{int}} \textsuperscript{\ac{ext}}
           & [\ac{nom}]
           & [\ac{acc}]
           & [\ac{dat}]
           \\ \cmidrule{1-4}
       [\ac{nom}]
           &
           & \ac{acc}
           & \cellcolor{LG}\ac{dat}
           \\ \cmidrule{1-4}
       [\ac{acc}]
           & \ac{nom}
           &
           & \ac{dat}
           \\ \cmidrule{1-4}
       [\ac{dat}]
           & \ac{nom}
           & \ac{acc}
           &
           \\
     \bottomrule
   \end{tabular}
   \label{tbl:oe-poss2}
 \end{table}

What sets Table \ref{tbl:oe-poss1} and Table \ref{tbl:oe-poss2} apart is the upper right corner of the table. These are cases in which the internal case wins the case competition.
In Table \ref{tbl:oe-poss1} these examples are not allowed to surface, and in Table \ref{tbl:oe-poss2} they are.
I give an example in which the internal case is more complex than the external one. The relative pronoun surfaces in the less complex external case. This indicates that Old English cannot be of the type shown in Table \ref{tbl:oe-poss1}, but is has to be of the type shown in Table \ref{tbl:oe-poss2}.

Consider the example in \ref{ex:oe-acc-dat}.
The internal case is dative, as the preposition \tit{onuppan} `upon' takes dative objects.
The external case is accusative, as the predicate \tit{tōbrȳsan} `to pulversize' takes accusative objects.
The relative pronoun \tit{ðone} `\ac{rel}.\ac{sg}.\ac{acc}' appears in the external case: the accusative.
The example appears in the external case, although it is the less complex case of the two. The example is grammatical, because Old English does not show case competition, so the case scale is irrelevant. As long as the relative pronoun appears in the external case, the headless relative is grammatical.

\exg. he tobryst ðone \tbf{ðe} \tbf{he} \tbf{onuppan} \tbf{fylð}\\
 it pulverizes\scsub{[acc]} \ac{rel}.\ac{sg}.\ac{acc} \ac{comp} it upon\scsub{[dat]} falls\\
`It pulverizes him whom it falls upon.' \flushfill{Old English, adapted from \pgcitealt{harbert1983}{550}} \label{ex:oe-acc-dat}

This example shows that Old English is not an instance of the pattern in Section \ref{sec:pattern-iii}, in which only the external case is allowed to surface. Instead, as illustrated by Table \ref{tbl:no-case-competition-old-english}, the language does not have any case competition. The relative pronoun appears in the external case, irrespective whether this the more complex case (marked light grey) or the less complex case (marked dark grey).

\begin{table}[H]
  \center
  \caption{Old English possibility 2}
  \begin{tabular}{c|c|c|c}
    \toprule
   \textsubscript{\ac{int}} \textsuperscript{\ac{ext}}
          & [\ac{nom}]
          & [\ac{acc}]
          & [\ac{dat}]
          \\ \cmidrule{1-4}
      [\ac{nom}]
          &
          & \ac{acc}
          & \cellcolor{LG}\ac{dat}
          \\ \cmidrule{1-4}
      [\ac{acc}]
          & \ac{nom}
          &
          & \ac{dat}
          \\ \cmidrule{1-4}
      [\ac{dat}]
          & \ac{nom}
          & \cellcolor{DG}\ac{acc}
          &
          \\
    \bottomrule
  \end{tabular}
  \label{tbl:no-case-competition-old-english}
\end{table}

%%there is a third possibility: gothic/ohg%%



The same pattern appears in Modern Greek.


I give examples from Modern Greek and Old English. In both languages, the relative pronoun takes the external case. I start with Modern Greek.

Consider the example in \ref{ex:greek-gen-nom}. In this example, the internal nominative case competes against the external genitive case.
The internal case is nominative, as the predicate \tit{voíθisó} `to help' takes nominative subjects.
The external case is accusative, as the predicate \tit{eðósó} `to give' takes genitive objects.
The relative pronoun \tit{hòn} `\tsc{rel}.\tsc{gen}.\tsc{m}.\tsc{pl}' appears in the external case: the genitive. The relative pronoun is not marked in bold, unlike the relative clause, showing that the relative pronoun patterns with the main clause.

\exg. ´Eðósa leftá ópju \tbf{me} \tbf{voíθise}.\\
give.\tsc{past}.1\tsc{sg}\scsub{gen} money \tsc{rel}.\tsc{gen}.\tsc{m}.\tsc{pl} \tsc{cl}.1\tsc{sg}.\tsc{acc} help.\tsc{past}.3\tsc{sg}\scsub{nom}\\
`I gave money to whoever helped me.' \flushfill{Modern Greek, adapted from \pgcitealt{daskalaki2011}{80}}\label{ex:greek-gen-nom}

The Modern Greek relative pronoun appears in the external case. Moreover, the external case is the more complex case than the internal case. In \ref{ex:greek-gen-nom}, the genitive is more complex than the nominative.
So far, the data pattern is consistent with the third pattern: the language only allows the external case to surface when it wins the case competition. The next examples, however, falsify this hypothesis. Examples in which the internal case is more complex are not ungrammatical. Rather, the relative pronoun still appears in the external (less complex) case.




Consider the example in \ref{ex:greek-nom-acc}.
The internal case is accusative, as the predicate \tit{irθó} `to invite' takes accusative objects.
The external case is accusative, as the predicate \tit{kálesó} `to come' takes nominative subjects.
The relative pronoun \tit{ópji} `\ac{rel}.\ac{pl}.\ac{m}.\ac{nom}' appears in the external case: the nominative.
The example appears in the external case, but does not adhere to the case scale: the relative pronoun appears in the less complex external  case (the nominative), and not in the more complex internal case (the accusative).
The example is grammatical, because in Modern Greek the relative pronoun appears in the external case. The case scale is irrelevant for the case in which the relative pronoun appears in Modern Greek .

\exg. Irθan ópji \tbf{káleses}.\\
come.\ac{pst}.3\ac{pl}\scsub{nom} \ac{rel}.\ac{pl}.\ac{m}.\ac{nom} invite.\ac{pst}.2\ac{sg}\scsub{acc}\\
`Whoever you invited came.'\flushfill{Modern Greek, adapted from \pgcitealt{daskalaki2011}{80}}\label{ex:greek-nom-acc}

This example in \ref{ex:greek-nom-acc-u} is identical to \ref{ex:greek-nom-acc}, except for that the relative pronoun appears in the internal  more complex case. This example is ungrammatical: the relative pronoun does not appear in the external case. The fact that the internal case is more complex is irrelevant.

\exg. *Irθan \tbf{ópjus} \tbf{káleses}.\\
come.\ac{pst}.3\ac{pl}\scsub{nom} \ac{rel}.\ac{pl}.\ac{m}.\ac{acc} invite.\ac{pst}.2\ac{sg}\scsub{acc}\\
`Whoever you invited came.'\flushfill{Modern Greek, adapted from \pgcitealt{daskalaki2011}{79}}\label{ex:greek-nom-acc-u}

When the internal case is genitive instead of accusative, a clitic is added to the sentence to make it grammatical.
Consider the example in \ref{ex:greek-nom-gen}.
The internal case is genitive, as the predicate \tit{eðósó} `to give' takes genitive objects.
The external case is accusative, as the predicate \tit{efχarístisó} `to thank' takes nominative subjects.
The relative pronoun \tit{ópjon} `\ac{rel}.\ac{pl}.\ac{m}.\ac{nom}' appears in the internal case: the genitive. The relative pronoun is not marked in bold, unlike the relative clause, showing that the relative pronoun patterns with the main clause.
The example appears in the external case, but does not adhere to the case scale: the relative pronoun appears in the less complex external case (the nominative), and not in the more complex internal case (the genitive). The relative clause obligatorily contains the genitive clitic \tit{tus} `\ac{cl}.3\ac{pl}.\ac{gen}'.\footnote{
In Modern German, it is possible to insert a light head to resolve a situation with a more complex external case. However, then the relative pronoun has to change as well (from a \tsc{wh}-pronoun into a \tsc{d}-pronoun). I assume this is a different construction, and the Modern Greek one with the clitic inserted is not.
}
The example is grammatical, because in Modern Greek the relative pronoun appears in the external case. The case scale is irrelevant for the case in which the relative pronoun appears in Modern Greek .

\exg. Me efχarístisan ópji \tbf{tus} \tbf{íχa} \tbf{ðósi} \tbf{leftá}.\\
 \ac{cl}.1\ac{sg}.\ac{acc} thank.\ac{pst}.3\ac{pl}\scsub{nom} \ac{rel}.\ac{pl}.\ac{m}.\ac{nom} \ac{cl}.3\ac{pl}.\ac{gen} have.\ac{pst}.1\ac{sg} give.\ac{ptcp}\scsub{gen} money\\
 `Whoever I had given money to, thanked me.'\flushfill{Modern Greek, adapted from \pgcitealt{daskalaki2011}{80}}\label{ex:greek-nom-gen}

In sum, in the relative pronoun in Old English and Modern Greek appears in the the external case, and it does not adhere to the case scale.

Taking this all together, I have not encountered a language that only allows the external case to surface when it wins the case competition. Ancient Greek allows the internal or the external case to surface when it wins the case competition. Modern Greek and Old English are examples of languages in which the relative pronoun appears in the external case, and that do not take the case scale into account.


\section{Summary}\label{sec:summary-2-patterns}

In case competition in headless relatives two factors play a role. The first one is which case wins the case competition. It is a crosslinguistically stable fact that this is determiner by the case scale in \ref{ex:case-scale-two-patterns-sum}, repeated from Chapter \ref{ch:recurring}. A case more to the right on the scale wins over a case more to the left on the scale.

\ex. \ac{nom} < \ac{acc} < \ac{dat}\label{ex:case-scale-two-patterns-sum}

The second factor is whether the internal and the external case are allowed to surface when they wins the case competition. This differs across languages. There are three possible patterns: (1) a pattern in which the external or the internal case are allowed to surface when they win, (2) a pattern in which only the internal case is allowed to surface when it wins, and (3) a pattern in which only the external case is allowed to surface when it wins.

Gothic and Old High German are examples of languages of the first type. Modern German is an example of a language of the second type. To my knowledge, the third pattern is not attested. It is impossible to prove that this pattern does not exist (or has not existed) in any natural language, and it could be an accidental gap. However, in line with the available data so far, I set up a system in the next section that derives the two attested patterns, and excludes the third one.
