% !TEX root = thesis.tex

\chapter{A case competition typology}

In Part \ref{part:case-facts} of this dissertation, I discussed a first aspect of case competition in headless relatives. There is a fixed scale that determines which case wins the case competition. This is the same case scale crosslinguistically. I repeat the case scale from Chapter \ref{ch:recurring} in \ref{ex:case-scale-two-patterns}.

\ex. \ac{nom} < \ac{acc} < \ac{dat}\label{ex:case-scale-two-patterns}

Also in Part \ref{part:case-facts}, I argued that a cumulative case decomposition can derive the case scale. This does not only hold for case competition in headless relatives, but also for syncretism patterns and morphological case containment patterns. In a cumulative case composition, the scale in \ref{ex:case-scale-two-patterns} can be interpreted as follows: an accusative contains all features a nominative contains plus one more. Similarly, a dative contains all features an accusative contains plus one. Therefore, a dative can be considered more complex than an accusative, and a accusative complex more than a nominative. In line with that, I refer to cases more to the right on the case scale as more complex cases than cases more to the left on the scale.

This part of the dissertation, Part \ref{part:variation}, focuses on a second aspect to headless relatives. This part is not stable crosslinguistically, but it differs across languages. Languages differ in whether they allow the internal case (the case from the relative clause) or the external case (the case from the main clause) to surface when they win the case competition. Metaphorically speaking, even though a case wins the case competition, it is a second matter whether it is allowed to come forward as a winner. Four patterns are logically possible for languages: (1) either the internal case or the external case is allowed to surface, (2) only the internal case is allowed to surface, and the external case is not, (3) only the external case is allowed to surface, and the internal case is not, (4) neither the internal nor the external case is allowed to surface.\footnote{
On the surface, the last pattern cannot be distinguished from a language that does not have case competition and does not allow for any case mismatches. I come back to this mater later, where I argue that there actually is case competition in play.
}
I show in this chapter that one of these logically possible patterns is not attested in any natural language.

The next section introduces the patterns that are logically possible in languages with case competition. Section \ref{sec:pattern-i} to Section \ref{sec:pattern-iv} discuss the patterns one by one, and I give examples when the pattern is attested. In Section \ref{sec:potentiel-counterexamples} I make a sidestep to potential counterexamples to my generalization. I argue that these cases are actually languages that lack case competition to begin with.


\section{Four possible patterns}\label{sec:possible-patterns}

Case competition has two aspects. The first aspect is the topic of Part \ref{part:case-facts} of the dissertation. It concerns which case wins the case competition. This is decided by the same case scale for all languages. The second aspects is the topic of Part \ref{part:variation} of the dissertation. This one concerns whether the case that wins the case competition is actually allowed to surface. It namely differs per language whether they allow the internal and the external case do so.

Metaphorically, the second aspect can be described as a language-specific approval committee. The committee learns (from the first aspect) which case wins the case competition. Then it can either approve this case or not approve it. This approval happens based on where the winning case comes from: (1) from inside to the relative clause (internal) or (2) from outside to the relative clause (external). It is determined per language whether it approves the internal case, the external case or both of them. The approval committee can only approve the winner of the competition or deny it, it cannot propose an alternative winner. In this metaphor, the approval of the committee means that a particular case is allowed to surface. When the case is not allowed to surface, the headless relative as a whole is ungrammatical.

Taking this all together, there are four patterns possible in languages. First, either the internal case or the external case is allowed to surface. Second, only the internal case is allowed to surface, and the external case is not. Third, only the external case is allowed to surface, and the internal case is not. Fourth, neither the internal nor the external case is allowed to surface. In what follows, I introduce these possible patterns.

The first possible pattern is that of a language that allows the internal case and the external case to surface when either of them wins the case competition. This is might look familiar, because it is the pattern that Gothic shows, which I discussed in Chapter \ref{ch:recurring}. Table \ref{tbl:case-competition-only-int} (repeated from Table \ref{tbl:summary-gothic-simple}) illustrates what the pattern for such a language looks like.

The left column shows the internal case between square brackets. The upper row shows the external case between square brackets. The other cells indicate the case of the relative pronoun. The diagonal is crossed out, because these are instances in which the internal and external case match, and there is no case competition taking place.
The three cells in the lower left corner are the situations in which the internal case surfaces when it wins the competition. The three cells in the upper right corner are the situations in which the external case surfaces when it wins the competition. All these instances are grammatical.

\begin{table}[H]
  \center
  \caption{Either internal or external case allowed}
  \begin{tabular}{c|c|c|c}
    \toprule
    \textsubscript{\ac{int}} \textsuperscript{\ac{ext}}
           & [\ac{nom}]
           & [\ac{acc}]
           & [\ac{dat}]
           \\ \cmidrule{1-4}
       [\ac{nom}]
           & \ac{nom}
           & \ac{acc}
           & \ac{dat}
           \\ \cmidrule{1-4}
       [\ac{acc}]
           & \ac{acc}
           & \ac{acc}
           & \ac{dat}
           \\ \cmidrule{1-4}
       [\ac{dat}]
           & \ac{dat}
           & \ac{dat}
           & \ac{dat}
           \\
     \bottomrule
  \end{tabular}
    \label{tbl:case-competition-int-ext}
\end{table}

The second possible pattern is that of a language that allows the internal case to surface when it wins the case competition, but it does not allow the external case to do so. In this type of language, the internal case gets to surface when it is more complex than the external one. When the external case is more complex, it is not allowed to surface, and the headless relative construction is ungrammatical.

Table \ref{tbl:case-competition-only-int} illustrates what the pattern for such a language looks like. Only half of the situations from the first pattern remains. The other half is ungrammatical.
The three cells in the lower left corner are the situations in which the internal case surfaces when it wins the competition. Just as in the first pattern, these instances are grammatical.
The three cells in the upper right corner are the situations in which the external case surfaces when it wins the competition. These instances are not grammatical for this type of language. The reasoning behind that is that the language does not allow the external case to surface when it wins the case competition.

\begin{table}[H]
  \center
  \caption{Only internal case allowed}
  \begin{tabular}{c|c|c|c}
    \toprule
    \textsubscript{\ac{int}} \textsuperscript{\ac{ext}}
           & [\ac{nom}]
           & [\ac{acc}]
           & [\ac{dat}]
           \\ \cmidrule{1-4}
       [\ac{nom}]
           & \ac{nom}
           & *
           & *
           \\ \cmidrule{1-4}
       [\ac{acc}]
           & \ac{acc}
           & \ac{acc}
           & *
           \\ \cmidrule{1-4}
       [\ac{dat}]
           & \ac{dat}
           & \ac{dat}
           & \ac{dat}
           \\
     \bottomrule
  \end{tabular}
    \label{tbl:case-competition-only-int}
\end{table}

The third possible pattern is that of a language that allows the external case to surface when it wins the case competition, but it does not allow the internal case to do so. In this type of language, only the external case gets to surface when it is more complex. When the internal case is more complex, it is not allowed to surface, and the headless relative construction is ungrammatical.

Table \ref{tbl:case-competition-only-ext} illustrates what the pattern for such a language looks like. It is the mirror image of the second type of language.
The three cells in the lower left corner are the situations in which the internal case surfaces when it wins the competition. Unlike in the first and the second pattern, these instances are not grammatical for this type of language. The reasoning behind that is that the language does not allow the internal case to surface when it wins the case competition.
The three cells in the upper right corner are the situations in which the external case surfaces when it wins the competition. Just as the first pattern, these instances are grammatical.

\begin{table}[H]
  \center
  \caption{Only external case allowed}
  \begin{tabular}{c|c|c|c}
    \toprule
    \textsubscript{\ac{int}} \textsuperscript{\ac{ext}}
           & [\ac{nom}]
           & [\ac{acc}]
           & [\ac{dat}]
           \\ \cmidrule{1-4}
       [\ac{nom}]
           & \ac{nom}
           & \ac{acc}
           & \ac{dat}
           \\ \cmidrule{1-4}
       [\ac{acc}]
           & *
           & \ac{acc}
           & \ac{dat}
           \\ \cmidrule{1-4}
       [\ac{dat}]
           & *
           & *
           & \ac{dat}
           \\
     \bottomrule
  \end{tabular}
    \label{tbl:case-competition-only-ext}
\end{table}

The fourth possible pattern is that of a language that allows neither the internal nor the external case to surface when it wins the case competition. When the internal case or the external is more complex, it is not allowed to surface, and the headless relative construction is ungrammatical. In other words, when the internal and the external case differ, there is no grammatical headless relative construction possible.\footnote{
On the surface, this pattern cannot be distinguished from a pattern that does not have case competition and does not allow for any case mismatches. I understand `a language with case competition' as a language that compares the internal and external case in its headless relatives. The source of ungrammaticality for the cells in Table \ref{tbl:case-competition-none} can only come from the comparing the internal and external case. For this to go through, the language is required to have headless relatives in some configurations, for instance when the cases match. In Section \ref{sec:aside-potentiel-counterexamples} I discuss languages in which the internal and external case are not compared to each other.
}

Table \ref{tbl:case-competition-none} illustrates what the pattern for such a language looks like.
The three cells in the lower left corner are the situations in which the internal case surfaces when it wins the competition. Just as the third pattern, but unlike the first and the second pattern, these instances are not grammatical for this type of language. The reasoning behind that is that the language does not allow the internal case to surface when it wins the case competition.
The three cells in the upper right corner are the situations in which the external case surfaces when it wins the competition. Just as the second pattern, but unlike the first and the third pattern, these instances are not grammatical for this type of language. The reasoning behind that is that the language does not allow the external case to surface when it wins the case competition.

\begin{table}[H]
  \center
  \caption{Neither internal nor external allowed}
  \begin{tabular}{c|c|c|c}
    \toprule
    \textsubscript{\ac{int}} \textsuperscript{\ac{ext}}
           & [\ac{nom}]
           & [\ac{acc}]
           & [\ac{dat}]
           \\ \cmidrule{1-4}
       [\ac{nom}]
           & \ac{nom}
           & *
           & *
           \\ \cmidrule{1-4}
       [\ac{acc}]
           & *
           & \ac{acc}
           & *
           \\ \cmidrule{1-4}
       [\ac{dat}]
           & *
           & *
           & \ac{dat}
           \\
     \bottomrule
  \end{tabular}
    \label{tbl:case-competition-none}
\end{table}

In this chapter I show that three of these four patterns are attested crosslinguistically. Section \ref{sec:pattern-i} shows that the first pattern, in which either the internal case or the external case can surface, is exemplified by Gothic (repeated from Chapter \ref{ch:recurring}) and by Old High German. The second pattern, in which only the internal case can surface, is illustrated by Modern German in Section \ref{sec:pattern-ii}. To my knowledge, there is no language in which only the external case can surface when it wins the case competition. This is discussed in \ref{sec:pattern-iii}. Section \ref{sec:pattern-iv} shows a language that allows neither the internal nor the external case to surface when it wins the competition: Polish.


\section{Internal and external case allowed}\label{sec:pattern-i}

This section discusses the situation in which the internal case or the external case is allowed to surface when either of them wins the case competition. Schematically, this looks as in Table \ref{tbl:case-competition-int-ext-repeated} (repeated from Table \ref{tbl:case-competition-int-ext}).

\begin{table}[H]
  \center
  \caption{Internal and external case allowed (repeated)}
  \begin{tabular}{c|c|c|c}
    \toprule
    \textsubscript{\ac{int}} \textsuperscript{\ac{ext}}
           & [\ac{nom}]
           & [\ac{acc}]
           & [\ac{dat}]
           \\ \cmidrule{1-4}
       [\ac{nom}]
           & \ac{nom}
           & \ac{acc}
           & \ac{dat}
           \\ \cmidrule{1-4}
       [\ac{acc}]
           & \ac{acc}
           & \ac{acc}
           & \ac{dat}
           \\ \cmidrule{1-4}
       [\ac{dat}]
           & \ac{dat}
           & \ac{dat}
           & \ac{dat}
           \\
     \bottomrule
  \end{tabular}
    \label{tbl:case-competition-int-ext-repeated}
\end{table}

Two examples of languages that show this pattern are Gothic and Old High German. I repeat the findings from Gothic (from \ref{ch:recurring}), and I present the data for Old High German, which is the result of my own research

In \ref{ch:recurring}, I discussed case competition between nominative, accusative and dative case in Gothic headless relatives, based on the work of \citet{harbert1978}. I repeat the results in Table \ref{tbl:summary-gothic-repeated} from Table \ref{tbl:summary-gothic-simple}.

In Gothic, the relative pronoun is allowed to surface in either the internal case or the external case. The three cells in the lower left corner are the situations in which the internal case surfaces when it wins the competition. The three cells in the upper right corner are the situations in which the external case surfaces when it wins the competition. All these instances are grammatical. The examples corresponding to the cells in Table \ref{tbl:summary-gothic-repeated} can be found in Section \ref{sec:pattern-rels}.

\begin{table}[H]
  \center
  \caption{Internal and external case allowed (repeated)}
    % !TEX root = ../thesis.tex

\begin{tabular}{c|c|c|c}
  \toprule
      \textsubscript{\ac{int}} \textsuperscript{\ac{ext}}
        & [\ac{nom}]
        & [\ac{acc}]
        & [\ac{dat}]
        \\ \cmidrule{1-4}
    [\ac{nom}]
        & \ac{nom}
        & \ac{acc}
        & \ac{dat}
        \\ \cmidrule{1-4}
    [\ac{acc}]
        & \ac{acc}
        & \ac{acc}
        & \ac{dat}
        \\ \cmidrule{1-4}
    [\ac{dat}]
        & \ac{dat}
        & (\ac{dat})
        & \ac{dat}
        \\
  \bottomrule
\end{tabular}

    \label{tbl:summary-gothic-repeated}
\end{table}

Old High German is another instance of a language in which the relative pronoun is allowed to surface in either the internal case or the external case. This conclusion follows from my own research of the texts `Der althochdeutsche Isidor', `The Monsee fragments', `Otfrid's Evangelienbuch' and `Tatian' in ANNIS \citep{krause2016}.\footnote{
Old High German is widely discussed in the literature because of its case attraction in headed relatives \citep[cf.][]{pittner1995}, a phenomenon that seems related to case competition in headless relatives (see Section \ref{sec:attraction} for why attraction is not further discussed in this dissertation).
A common observation is that case attraction in headed relatives in Old High German adheres to the case scale. The same is claimed for headless relatives.
What, to my knowledge, has not been studied systematically is whether Old High German headless relatives allow the internal case and the external case to surface when either of them wins the case competition. This is what I investigated in my work.
}
The examples follow the spelling and the detailed glosses in ANNIS. The translations are my own.

I start with the competition between accusative and nominative. Following the case scale, the relative pronoun appears in accusative case and never in nominative. As Old High German allows the internal and external case to surface, the accusative surfaces when it is the internal case and when it is the external case.

Consider the example in \ref{ex:ohg-acc-nom}. In this example, the internal nominative case competes against the external accusative case.
The internal case is nominative, as the predicate \tit{gisizzen} `to possess' takes nominative subjects.
The external case is accusative, as the predicate \tit{bibringan} `to create' takes accusative objects.
The relative pronoun \tit{dhen} `\ac{rel}.\ac{sg}.\ac{m}.\ac{acc}' appears in the external case: the accusative. The relative pronoun is not marked in bold, just as the main clause, showing that the relative pronoun patterns with the main clause.\footnote{
At the end of this section I discuss a counterexample to the case scale, in which the internal case is nominative, the external case is accusative, and the relative pronoun appears in the nominative case.
}

\exg. ih bibringu fona iacobes samin endi fona iuda dhen \tbf{mina} \tbf{berga} \tbf{chisitzit}\\
1\ac{sg}.\ac{nom} {create}.\ac{pres}.1\ac{sg}\scsub{[acc]} of Jakob.\ac{gen} seed.\ac{sg}.\ac{dat} and of Judah.\ac{dat} \ac{rel}.\ac{sg}.\ac{m}.\ac{acc} my.\ac{acc}.\ac{m}.\ac{pl} mountain.\ac{acc}.\ac{pl} possess.\ac{pres}.3\ac{sg}\scsub{[nom]}\\
`I create of the seed of Jacob and of Judah the one, who possess my mountains' \flushfill{Old High German, \ac{isid} 34:3}\label{ex:ohg-acc-nom}

Consider the example in \ref{ex:ohg-nom-acc}. In this example, the internal accusative case competes against the external nominative case.
The internal case is accusative, as the predicate \tit{zellen} `to tell' takes accusative objects.
The external case is nominative, as the predicate \tit{sin} `to be' takes nominative objects.
The relative pronoun \tit{then} `\ac{rel}.\ac{sg}.\ac{m}.\ac{acc}' appears in the internal case: the accusative. The relative pronoun is marked in bold, just as the relative clause, showing that the relative pronoun patterns with the relative clause.
Examples in which the internal case is accusative, the external case is nominative and the relative pronoun appears in nominative case are unattested.

\exg. thíz ist \tbf{then} \tbf{sie} \tbf{zéllent}\\
\ac{dem}.\ac{sg}.\ac{n}.\ac{nom} be.\ac{pres}.3\ac{sg}\scsub{[nom]} \ac{rel}.\ac{sg}.\ac{m}.\ac{acc} 3\ac{pl}.\ac{m}.\ac{nom} tell.\ac{pres}.3\ac{pl}\scsub{[acc]}\\
`this is the one whom they talk about' \flushfill{Old High German, \ac{otfrid} III 16:50}\label{ex:ohg-nom-acc}

The two examples in which nominative and accusative compete are highlighted in Table \ref{tbl:summary-old-high-german-nom-acc}. The light gray marking corresponds to \ref{ex:ohg-acc-nom}, in which the external accusative wins over the internal nominative, and the relative pronoun is allowed to surface in the accusative case. The dark gray marking corresponds to \ref{ex:ohg-nom-acc}, in which the internal accusative wins over the external nominative, and the relative pronoun is allowed to surface in the accusative case.

\begin{table}[H]
  \center
  \caption{Summary of Old High German headless relatives (\ac{nom} --- \ac{acc})}
  \begin{tabular}{c|c|c|c}
    \toprule
        \textsubscript{\ac{int}} \textsuperscript{\ac{ext}}
          & [\ac{nom}]
          & [\ac{acc}]
          & [\ac{dat}]
          \\ \cmidrule{1-4}
      [\ac{nom}]
          & \ac{nom}
          & \cellcolor{LG}\ac{acc}
          & \ac{dat}
          \\ \cmidrule{1-4}
      [\ac{acc}]
          & \cellcolor{DG}\ac{acc}
          & \ac{acc}
          & \ac{dat}
          \\ \cmidrule{1-4}
      [\ac{dat}]
          & \ac{dat}
          & \ac{dat}
          & \ac{dat}
          \\
    \bottomrule
  \end{tabular}
    \label{tbl:summary-old-high-german-nom-acc}
\end{table}

I continue with the competition between dative and nominative. Following the case scale, the relative pronoun appears in dative case and never in nominative. As Old High German allows the internal and the external case to surface, the dative surfaces when it is the internal case and when it is the external case.

Consider the example in \ref{ex:ohg-dat-nom}. In this example, the internal nominative case competes against the external dative case.
The internal case is nominative, as the predicate \tit{sprehhan} `to speak' takes nominative subjects.
The external case is dative, as the predicate \tit{antwurten} `to reply' takes dative objects.
The relative pronoun \tit{demo} `\ac{rel}.\ac{sg}.\ac{m}.\ac{dat}' appears in the external case: the dative. The relative pronoun is not marked in bold, just as the main clause, showing that the relative pronoun patterns with the main clause.
Examples in which the internal case is nominative, the external case is dative and the relative pronoun appears in nominative case are unattested.

\exg. enti aer {ant uurta} demo \tbf{zaimo} \tbf{sprah}\\
and 3\ac{sg}.\ac{m}.\ac{nom} reply.\ac{pst}.3\ac{sg}\scsub{[dat]} \ac{rel}.\ac{sg}.\ac{m}.\ac{dat} {to 3\ac{sg}.\ac{m}.\ac{dat}} speak.\ac{pst}.3\ac{sg}\scsub{[nom]}\\
`and he replied to the one who spoke to him' \flushfill{Old High German, \ac{mons} 7:24, adapted from \pgcitealt{pittner1995}{199}}\label{ex:ohg-dat-nom}

Consider the example in \ref{ex:ohg-nom-dat}. In this example, the internal dative case competes against the external nominative case.
The internal case is dative, as the predicate \tit{forlazan} `to read' takes dative indirect objects.
The external case is nominative, as the predicate \tit{minnon} `to love' takes nominative subjects.
The relative pronoun \tit{themo} `\ac{rel}.\ac{sg}.\ac{m}.\ac{dat}' appears in the internal case: the dative. The relative pronoun is marked in bold, just as the relative clause, showing that the relative pronoun patterns with the relative clause.
Examples in which the internal case is dative, the external case is nominative and the relative pronoun appears in nominative case are unattested.

\exg. \tbf{themo} \tbf{min} \tbf{uuirdit} \tbf{forlazan}, min minnot\\
\ac{rel}.\ac{sg}.\ac{m}.\ac{dat} less become.\ac{pres}.3\ac{sg} read.\ac{inf}\scsub{[dat]} less love.\ac{pres}.3\ac{sg}\scsub{[nom]}\\
`to whom less is read, loves less' \flushfill{Old High German, \ac{tatian} 138:13}\label{ex:ohg-nom-dat}

The two examples in which nominative and dative compete are highlighted in Table \ref{tbl:summary-old-high-german-nom-dat}. The light gray marking corresponds to \ref{ex:ohg-dat-nom}, in which the external dative wins over the internal nominative, and the relative pronoun is allowed to surface in the dative case. The dark gray marking corresponds to \ref{ex:ohg-nom-dat}, in which the internal dative wins over the external nominative, and the relative pronoun is allowed to surface in the dative case.

\begin{table}[H]
  \center
  \caption{Summary of Old High German headless relatives (\ac{nom} --- \ac{dat})}
  \begin{tabular}{c|c|c|c}
    \toprule
        \textsubscript{\ac{int}} \textsuperscript{\ac{ext}}
          & [\ac{nom}]
          & [\ac{acc}]
          & [\ac{dat}]
          \\ \cmidrule{1-4}
      [\ac{nom}]
          & \ac{nom}
          & \ac{acc}
          & \cellcolor{LG}\ac{dat}
          \\ \cmidrule{1-4}
      [\ac{acc}]
          & \ac{acc}
          & \ac{acc}
          & \ac{dat}
          \\ \cmidrule{1-4}
      [\ac{dat}]
          & \cellcolor{DG}\ac{dat}
          & \ac{dat}
          & \ac{dat}
          \\
    \bottomrule
  \end{tabular}
    \label{tbl:summary-old-high-german-nom-dat}
\end{table}

I end with the competition between dative and accusative. Following the case scale, the relative pronoun appears in dative case and never in accusative. As Old High German allows the internal and the external case to surface, the dative surfaces when it is the internal case and when it is the external case.

Consider the example in \ref{ex:ohg-dat-acc}. In this example, the internal accusative case competes against the external dative case.
The internal case is accusative, as the predicate \tit{zellen} `to tell' takes accusative objects.
The external case is dative, as the comparative of the adjective \tit{furiro} `great' takes dative objects.
The relative pronoun \tit{thên} `\ac{rel}.\ac{pl}.\ac{m}.\ac{dat}' appears in the external case: the dative. The relative pronoun is not marked in bold, just as the main clause, showing that the relative pronoun patterns with the main clause.
Examples in which the internal case is accusative, the external case is dative and the relative pronoun appears in accusative case are unattested.

\exg. bis -tú nu {zi wáre} furira Ábrahame? ouh thén \tbf{man} \tbf{hiar} \tbf{nu} \tbf{zálta}\\
be.\ac{pres}.2\ac{sg} -2\ac{sg}.\ac{nom} now truly {great}.\ac{cmpr}\scsub{[dat]} Abraham.\ac{dat} and \ac{rel}.\ac{pl}.\ac{m}.\ac{dat} one.\ac{nom}.\ac{m}.\ac{sg} here now tell.\ac{pst}.3\ac{sg}\scsub{[acc]}\\
`are you now truly greater than Abraham? and than those, who one talked about here now' \flushfill{Old High German, \ac{otfrid} III 18:33}\label{ex:ohg-dat-acc}

Consider the example in \ref{ex:ohg-acc-dat}. In this example, the internal dative case competes against the external accusative case.
The internal case is dative, as the predicate \tit{zawen} `to tell' takes dative subjects.
The external case is accusative, as the predicate \tit{weizan} `to know' takes accusative objects.
The relative pronoun \tit{thémo} `\ac{rel}.\ac{sg}.\ac{m}.\ac{dat}' appears in the external case: the dative. The relative pronoun is marked in bold, just as the relative clause, showing that the relative pronoun patterns with the relative clause.
Examples in which the internal case is accusative, the external case is dative and the relative pronoun appears in accusative case are unattested.

\exg. weiz \tbf{thémo} \tbf{ouh} \tbf{baz} \tbf{záweta}\\
know.1\ac{sg}\scsub{[acc]} \ac{rel}.\ac{sg}.\ac{m}.\ac{dat} also better manage.\ac{pst}.3\ac{sg}\scsub{[dat]}\\
`I know the one who also managed it better' \flushfill{Old High German, \ac{otfrid} V 5:5}\label{ex:ohg-acc-dat}

The two examples in which accusative and dative compete are highlighted in Table \ref{tbl:summary-old-high-german-acc-dat}. The light gray marking corresponds to \ref{ex:ohg-dat-acc}, in which the external dative wins over the internal accusative, and the relative pronoun is allowed to surface in the dative case. The dark gray marking corresponds to \ref{ex:ohg-acc-dat}, in which the internal dative wins over the external accusative, and the relative pronoun is allowed to surface in the dative case.

\begin{table}[H]
  \center
  \caption{Summary of Old High German headless relatives (\ac{acc} --- \ac{dat})}
  \begin{tabular}{c|c|c|c}
    \toprule
        \textsubscript{\ac{int}} \textsuperscript{\ac{ext}}
          & [\ac{nom}]
          & [\ac{acc}]
          & [\ac{dat}]
          \\ \cmidrule{1-4}
      [\ac{nom}]
          & \ac{nom}
          & \ac{acc}
          & \ac{dat}
          \\ \cmidrule{1-4}
      [\ac{acc}]
          & \ac{acc}
          & \ac{acc}
          & \cellcolor{LG}\ac{dat}
          \\ \cmidrule{1-4}
      [\ac{dat}]
          & \ac{dat}
          & \cellcolor{DG}\ac{dat}
          & \ac{dat}
          \\
    \bottomrule
  \end{tabular}
    \label{tbl:summary-old-high-german-acc-dat}
\end{table}

In my research I encountered a single counterexample to the pattern I just described.
Consider the example in \ref{ex:ohg-counterexample}. In this example, the internal nominative case competes against the external accusative case.
The internal case is nominative, as the predicate \tit{giheilen} `to save' takes nominative subjects.
The external case is accusative, as the predicate \tit{beran} `to bear' takes accusative objects.
Surprisingly, the relative pronoun \tit{thér} `\ac{rel}.\ac{sg}.\ac{m}.\ac{nom}' appears in the internal case: the nominative, which is the less complex of the two cases. The relative pronoun is marked in bold, just as the relative clause, showing that the relative pronoun patterns with the relative clause.

\exg. tház si uns béran scolti \tbf{thér} \tbf{unsih} \tbf{gihéilti}\\
 that 3\ac{sg}.\ac{f}.\ac{nom} 1\ac{pl}.\ac{dat} bear.\ac{inf}\scsub{[acc]} should.\ac{subj}.\ac{pst}.3\ac{sg} \ac{rel}.\ac{sg}.\ac{m}.\ac{nom} 1\ac{pl}.\ac{acc} save.\ac{sbjv}.\ac{pst}.3\ac{sg}\scsub{[nom]}\\
 `that she should have beared for us the one, who had saved us' \flushfill{Old High German, \ac{otfrid} I 3:38}\label{ex:ohg-counterexample}

This example is unexpected, because the less complex case (the nominative) wins and not the more complex case (the accusative).
The only explanation for this I can see is a functional one. The \tit{thér} `\ac{rel}.\ac{sg}.\ac{m}.\ac{nom}' in \ref{ex:ohg-counterexample} refers to Jesus. In the relative clause he is the subject of \tit{unsih gihéilti} `had saved us', hence the internal nominative case. In the main clause he is the object of \tit{tház si uns béran scolti} `that she should have beared', hence the external accusative case.
Letting the relative pronoun surface in the internal case could be interpreted as emphasizing the role of Jesus as a savior, rather than him being the object of being given birth to. In line with that reasoning, it is expected that certain grammatical facts more often deviate from regular patterns if Jesus is involved. I leave investigating this prediction for future research.
Of course, this does not answer the question of what happens to the accusative case required by the external predicate. It also does not explain why not another emphasizing strategy is used, for instance forming a light-headed relative, which would leave space for two cases.
I acknowledge this example as a counterexample to the pattern I describe, but I do not change my generalization, as this is a single occurrence.

Leaving the counterexample aside, I conclude that Gothic and Old High German are both instances of languages that allow the internal and the external case to surface. The relative pronoun surfaces in the case that wins the case competition.


\section{Only internal case allowed}\label{sec:pattern-ii}

This section discusses the situation in which only the internal case is allowed to surface when it wins the case competition. When the external case wins the case competition, the result is ungrammatical. Schematically, this looks as in Table \ref{tbl:case-competition-only-int-repeated} (repeated from Table \ref{tbl:case-competition-only-int}).

\begin{table}[H]
  \center
  \caption{Only internal case allowed (repeated)}
  \begin{tabular}{c|c|c|c}
    \toprule
    \textsubscript{\ac{int}} \textsuperscript{\ac{ext}}
           & [\ac{nom}]
           & [\ac{acc}]
           & [\ac{dat}]
           \\ \cmidrule{1-4}
       [\ac{nom}]
           & \ac{nom}
           & *
           & *
           \\ \cmidrule{1-4}
       [\ac{acc}]
           & \ac{acc}
           & \ac{acc}
           & *
           \\ \cmidrule{1-4}
       [\ac{dat}]
           & \ac{dat}
           & \ac{dat}
           & \ac{dat}
           \\
     \bottomrule
  \end{tabular}
    \label{tbl:case-competition-only-int-repeated}
\end{table}

An example of a language that shows this pattern is Modern German. In this section I discuss the Modern German data, based on the research of \citet{vogel2001}. %adapted from.. glosses translations judgements

\exg. Uns besucht wer Maria mag\\
us visits who-NOM Maria-ACC likes\\ 343

\exg. Ich lade ein, wen auch Maria mag\\
I invite who-ACC also Maria likes'\\ 344

\exg. Ich vertraue, wem Maria gefällt\\
I trust who-DAT Maria-ΝΟΜ pleases\\ p. 345

I start with the competition between accusative and nominative. Following the case scale, the relative pronoun appears in accusative case and never in nominative. Following the internal-only requirement, only when the accusative case is the internal case, the sentence is grammatical.

I start with the situation in which the external case wins the competition, and there is no grammatical outcome possible.
Consider the example in \ref{ex:mg-acc-nom}. In this example, the internal nominative case competes against the external accusative case.
The internal case is nominative, as the predicate \tit{sein} `to be' takes nominative subjects.
The external case is accusative, as the predicate \tit{einladen} `to invite' takes accusative objects.
The relative pronoun \tit{wen} `\ac{rel}.\ac{an}.\ac{acc}' appears in the external case: the accusative. The relative pronoun is not marked in bold, just as the main clause, showing that the relative pronoun patterns with the main clause.
The example adheres to the case scale, but the more complex case (here the accusative) is not the internal case. As only the internal can win the case competition in Modern German, the example in ungrammatical.

\exg. *Ich {lade ein}, wen \tbf{mir} \tbf{sympathisch} \tbf{ist}.\\
 1\ac{sg}.\ac{nom} invite.\ac{pres}.1\ac{sg}\scsub{[acc]} \ac{rel}.\ac{an}.\ac{acc} 1\ac{sg}.\ac{dat} nice be.\ac{pres}.3\ac{sg}\scsub{[nom]}\\
 `I invite who I like.' \flushfill{Modern German, adapted from \pgcitealt{vogel2001}{344}}\label{ex:mg-acc-nom}

The example in \ref{ex:mg-acc-nom-u} is identical to \ref{ex:mg-acc-nom}, except for that the relative pronoun appears in the external less complex nominative case. This example is also ungrammatical: in addition to the more complex case not being the internal case, the relative pronoun also does not appear in the more complex case (the accusative) but in the less complex case (the nominative).

\exg. *Ich {lade ein}, wer \tbf{mir} \tbf{sympathisch} \tbf{ist}.\\
 1\ac{sg}.\ac{nom} invite.\ac{pres}.1\ac{sg}\scsub{[acc]} \ac{rel}.\ac{an}.\ac{nom} 1\ac{sg}.\ac{dat} nice be.\ac{pres}.3\ac{sg}\scsub{[nom]}\\
 `I invite who I like.' \flushfill{Modern German, adapted from \pgcitealt{vogel2001}{344}}\label{ex:mg-acc-nom-u}

Now I turn to the situation in which the internal case wins the competition, and there is a grammatical outcome possible.
Consider the example in \ref{ex:mg-nom-acc}. In this example, the internal accusative case competes against the external nominative case.
The internal case is accusative, as the predicate \tit{mögen} `to like' takes accusative objects.
The external case is nominative, as the predicate \tit{besuchen} `to visit' takes nominative subjects.
The relative pronoun \tit{wen} `\ac{rel}.\ac{an}.\ac{acc}' appears in the internal case: the accusative. The relative pronoun is marked in bold, just as the relative clause, showing that the relative pronoun patterns with the relative clause.
The example adheres to the case scale, and the more complex case (here the accusative) is the internal case, so the example is grammatical.

\exg. Uns besucht \tbf{wen} \tbf{Maria} \tbf{mag}.\\
 2\ac{pl}.\ac{acc} visit.\ac{pres}.3\ac{sg}\scsub{[nom]} \ac{rel}.\ac{an}.\ac{acc} Maria.\ac{nom} like.\ac{pres}.3\ac{sg}\scsub{[acc]}\\
 `Who visits us, Maria likes.' \flushfill{Modern German, adapted from \pgcitealt{vogel2001}{343}}\label{ex:mg-nom-acc}

The example in \ref{ex:mg-nom-acc-u} is identical to \ref{ex:mg-nom-acc}, except for that the relative pronoun appears in the external less complex nominative case. This example is ungrammatical: although the internal case is more complex, the relative pronoun appears in the less complex case (the nominative) and not in the more complex case (the accusative).

\exg. *Uns besucht \tbf{wer} \tbf{Maria} \tbf{mag}.\\
 2\ac{pl}.\ac{acc} visit.\ac{pres}.3\ac{sg}\scsub{[nom]} \ac{rel}.\ac{an}.\ac{nom} Maria.\ac{nom} like.\ac{pres}.3\ac{sg}\scsub{[acc]}\\
 `Who visits us, Maria likes.' \flushfill{Modern German, adapted from \pgcitealt{vogel2001}{343}}\label{ex:mg-nom-acc-u}

The two examples in which nominative and accusative compete are highlighted in Table \ref{tbl:case-competition-mg-nom-acc}. The light gray marking corresponds to \ref{ex:mg-acc-nom}, in which the external accusative wins over the internal nominative, but the relative pronoun is not allowed to surface in the accusative case (or in the losing nominative case). The dark gray marking corresponds to \ref{ex:mg-nom-acc}, in which the internal accusative wins over the external nominative, and the relative pronoun is allowed to surface in the accusative case (and in the losing nominative case).

 \begin{table}[H]
   \center
   \caption{Summary of Modern German headless relatives (\ac{nom} --- \ac{acc})}
   \begin{tabular}{c|c|c|c}
     \toprule
     \textsubscript{\ac{int}} \textsuperscript{\ac{ext}}
            & [\ac{nom}]
            & [\ac{acc}]
            & [\ac{dat}]
            \\ \cmidrule{1-4}
        [\ac{nom}]
            & \ac{nom}
            & \cellcolor{LG}*
            & *
            \\ \cmidrule{1-4}
        [\ac{acc}]
            & \cellcolor{DG}\ac{acc}
            & \ac{acc}
            & *
            \\ \cmidrule{1-4}
        [\ac{dat}]
            & \ac{dat}
            & \ac{dat}
            & \ac{dat}
            \\
      \bottomrule
   \end{tabular}
     \label{tbl:case-competition-mg-nom-acc}
 \end{table}

I continue with the competition between dative and nominative. Following the case scale, the relative pronoun appears in dative case and never in nominative. Following the internal-only requirement, only when the dative case is the internal case, the sentence is grammatical.

I start again with the situation in which the external case wins the competition, and there is no grammatical outcome possible.
Consider the example in \ref{ex:mg-dat-nom}. In this example, the internal nominative case competes against the external dative case.
The internal case is nominative, as the predicate \tit{mögen} `to like' takes nominative subjects.
The external case is dative, as the predicate \tit{vertrauen} `to trust' takes dative objects.
The relative pronoun \tit{wem} `\ac{rel}.\ac{an}.\ac{dat}' appears in the external case: the dative. The relative pronoun is not marked in bold, just as the main clause, showing that the relative pronoun patterns with the main clause.
The example adheres to the case cale, but the more complex case (here the dative) is not the internal case. As only the internal can win the case competition in Modern German, the example in ungrammatical.

\exg. *Ich vertraue, wem \tbf{Hitchcock} \tbf{mag}.\\
1\ac{sg}.\ac{nom} trust.\ac{pres}.1\ac{sg}\scsub{[dat]} \ac{rel}.\ac{an}.\ac{dat} Hitchcock.\ac{acc} like.\ac{pres}.3\ac{sg}\scsub{[nom]}\\
`I trust who likes Hitchcock.' \flushfill{Modern German, adapted from \pgcitealt{vogel2001}{345}}\label{ex:mg-dat-nom}

The example in \ref{ex:mg-dat-nom-u} is identical to \ref{ex:mg-dat-nom}, except for that the relative pronoun appears in the external less complex nominative case. This example is also ungrammatical: in addition to the more complex case not being the internal case, the relative pronoun also does not appear in the more complex case (the dative) but in the less complex case (the nominative).

\exg. *Ich vertraue, wer \tbf{Hitchcock} \tbf{mag}.\\
1\ac{sg}.\ac{nom} trust.\ac{pres}.1\ac{sg}\scsub{[dat]} \ac{rel}.\ac{an}.\ac{nom} Hitchcock.\ac{acc} like.\ac{pres}.3\ac{sg}\scsub{[nom]}\\
`I trust who likes Hitchcock.' \flushfill{Modern German, adapted from \pgcitealt{vogel2001}{345}}\label{ex:mg-dat-nom-u}

Now I turn again to the situation in which the internal case wins the competition, and there is a grammatical outcome possible.
Consider the example in \ref{ex:mg-nom-dat}. In this example, the internal dative case competes against the external nominative case.
The internal case is dative, as the predicate \tit{vertrauen} `to trust' takes dative objects.
The external case is nominative, as the predicate \tit{besuchen} `to visit' takes nominative subjects.
The relative pronoun \tit{wem} `\ac{rel}.\ac{an}.\ac{dat}' appears in the internal case: the dative. The relative pronoun is marked in bold, just as the relative clause, showing that the relative pronoun patterns with the relative clause.
The example adheres to the case scale, and the more complex case (here the dative) is the internal case, so the example is grammatical.

\exg. Uns besucht \tbf{wem} \tbf{Maria} \tbf{vertraut}.\\
2\ac{pl}.\ac{acc} visit.\ac{pres}.3\ac{sg}\scsub{[nom]} \ac{rel}.\ac{an}.\ac{dat} Maria.\ac{nom} trust.\ac{pres}.3\ac{sg}\scsub{[dat]}\\
`Who visits us, Maria trusts.' \flushfill{Modern German, adapted from \pgcitealt{vogel2001}{343}}\label{ex:mg-nom-dat}

The example in \ref{ex:mg-nom-dat-u} is identical to \ref{ex:mg-nom-dat}, except for that the relative pronoun appears in the external less complex nominative case. This example is ungrammatical: although the internal case is more complex, the relative pronoun appears in the less complex case (the nominative) and not in the more complex case (the dative).

\exg. *Uns besucht \tbf{wer} \tbf{Maria} \tbf{vertraut}.\\
2\ac{pl}.\ac{acc} visit.\ac{pres}.3\ac{sg}\scsub{[nom]} \ac{rel}.\ac{an}.\ac{nom} Maria.\ac{nom} trust.\ac{pres}.3\ac{sg}\scsub{[dat]}\\
`Who visits us, Maria trusts.' \flushfill{Modern German, adapted from \pgcitealt{vogel2001}{343}}\label{ex:mg-nom-dat-u}

The two examples in which nominative and dative compete are highlighted in Table \ref{tbl:case-competition-mg-nom-dat}. The light gray marking corresponds to \ref{ex:mg-dat-nom}, in which the external dative wins over the internal nominative, but the relative pronoun is not allowed to surface in the dative case (or in the losing nominative case). The dark gray marking corresponds to \ref{ex:mg-nom-dat}, in which the internal dative wins over the external nominative, and the relative pronoun is allowed to surface in the dative case (and in the losing nominative case).

\begin{table}[H]
  \center
  \caption{Summary of Modern German headless relatives (\ac{nom} --- \ac{dat})}
  \begin{tabular}{c|c|c|c}
    \toprule
    \textsubscript{\ac{int}} \textsuperscript{\ac{ext}}
           & [\ac{nom}]
           & [\ac{acc}]
           & [\ac{dat}]
           \\ \cmidrule{1-4}
       [\ac{nom}]
           & \ac{nom}
           & *
           & \cellcolor{LG}*
           \\ \cmidrule{1-4}
       [\ac{acc}]
           & \ac{acc}
           & \ac{acc}
           & *
           \\ \cmidrule{1-4}
       [\ac{dat}]
           & \cellcolor{DG}\ac{dat}
           & \ac{dat}
           & \ac{dat}
           \\
     \bottomrule
  \end{tabular}
    \label{tbl:case-competition-mg-nom-dat}
\end{table}

I end with the competition between dative and accusative. Following the case scale, the relative pronoun appears in dative case and never in accusative. Following the internal-only requirement, only when the dative case is the internal case, the sentence is grammatical.

I start again with the situation in which the external case wins the competition, and there is no grammatical outcome possible.
Consider the example in \ref{ex:mg-dat-acc}. In this example, the internal accusative case competes against the external dative case.
The internal case is accusative, as the predicate \tit{mögen} `to like' takes accusative objects.
The external case is dative, as the predicate \tit{vertrauen} `to trust' takes dative objects.
The relative pronoun \tit{wem} `\ac{rel}.\ac{an}.\ac{dat}' appears in the external case: the dative. The relative pronoun is not marked in bold, just as the main clause, showing that the relative pronoun patterns with the main clause.
The example adheres to the case scale, but the more complex case (here the dative) is not the internal case. As only the internal can win the case competition in Modern German, the example in ungrammatical.

\exg. *Ich vertraue wem \tbf{auch} \tbf{Maria} \tbf{mag}. \\
1\ac{sg}.\ac{nom} trust.\ac{pres}.1\ac{sg}\scsub{[dat]} \ac{rel}.\ac{an}.\ac{dat} also Maria.\ac{nom} like.\ac{pres}.3\ac{sg}\scsub{[acc]}.\\
`I trust whoever Maria also likes.' \flushfill{Modern German, adapted from \pgcitealt{vogel2001}{345}}\label{ex:mg-dat-acc}

The example in \ref{ex:mg-dat-acc-u} is identical to \ref{ex:mg-dat-acc}, except for that the relative pronoun appears in the external less complex accusative case. This example is also ungrammatical: in addition to the more complex case not being the internal case, the relative pronoun also does not appear in the more complex case (the dative) but in the less complex case (the accusative).

\exg. *Ich vertraue wen \tbf{auch} \tbf{Maria} \tbf{mag}. \\
1\ac{sg}.\ac{nom} trust.\ac{pres}.1\ac{sg}\scsub{[dat]} \ac{rel}.\ac{an}.\ac{acc} also Maria.\ac{nom} like.\ac{pres}.3\ac{sg}\scsub{[acc]}.\\
`I trust whoever Maria also likes.' \flushfill{Modern German, adapted from \pgcitealt{vogel2001}{345}}\label{ex:mg-dat-acc-u}

Now I turn again to the situation in which the internal case wins the competition, and there is a grammatical outcome possible.
Consider the example in \ref{ex:mg-acc-dat}. In this example, the internal dative case competes against the external accusative case.
The internal case is dative, as the predicate \tit{vertrauen} `to trust' takes dative objects.
The external case is accusative, as the predicate \tit{einladen} `to invite' takes accusative objects.
The relative pronoun \tit{wem} `\ac{rel}.\ac{an}.\ac{dat}' appears in the internal case: the dative. The relative pronoun is marked in bold, just as the relative clause, showing that the relative pronoun patterns with the relative clause.
The example adheres to the case scale, and the more complex case (here the dative) is the internal case, so the example is grammatical.

\exg. Ich {lade ein} \tbf{wem} \tbf{auch} \tbf{Maria} \tbf{vertraut}. \\
1\ac{sg}.\ac{nom} invite.\ac{pres}.1\ac{sg}\scsub{[acc]} \ac{rel}.\ac{an}.\ac{dat} also Maria.\ac{nom} trust.\ac{pres}.3\ac{sg}\scsub{[dat]}.\\
`I invite whoever Maria also trusts.' \flushfill{Modern German, adapted from \pgcitealt{vogel2001}{344}}\label{ex:mg-acc-dat}

The example in \ref{ex:mg-acc-dat-u} is identical to \ref{ex:mg-acc-dat}, except for that the relative pronoun appears in the external less complex accusative case. This example is ungrammatical: although the internal case is more complex, the relative pronoun appears in the less complex case (the accusative) and not in the more complex case (the dative).

\exg. *Ich {lade ein} \tbf{wen} \tbf{auch} \tbf{Maria} \tbf{vertraut}. \\
1\ac{sg}.\ac{nom} invite.\ac{pres}.1\ac{sg}\scsub{[acc]} \ac{rel}.\ac{an}.\ac{acc} also Maria.\ac{nom} trust.\ac{pres}.3\ac{sg}\scsub{[dat]}.\\
`I invite whoever Maria also trusts.' \flushfill{Modern German, adapted from \pgcitealt{vogel2001}{344}}\label{ex:mg-acc-dat-u}

The two examples in which nominative and dative compete are highlighted in Table \ref{tbl:case-competition-mg-acc-dat}. The light gray marking corresponds to \ref{ex:mg-dat-nom}, in which the external dative wins over the internal nominative, but the relative pronoun is not allowed to surface in the dative case (or in the losing accusative case). The dark gray marking corresponds to \ref{ex:mg-acc-dat}, in which the internal dative wins over the external accusative, and the relative pronoun is allowed to surface in the dative case (and in the losing accusative case).

\begin{table}[H]
  \center
  \caption{Summary of Modern German headless relatives (\ac{acc} --- \ac{dat})}
  \begin{tabular}{c|c|c|c}
    \toprule
    \textsubscript{\ac{int}} \textsuperscript{\ac{ext}}
           & [\ac{nom}]
           & [\ac{acc}]
           & [\ac{dat}]
           \\ \cmidrule{1-4}
       [\ac{nom}]
           & \ac{nom}
           & *
           & *
           \\ \cmidrule{1-4}
       [\ac{acc}]
           & \ac{acc}
           & \ac{acc}
           & \cellcolor{LG}*
           \\ \cmidrule{1-4}
       [\ac{dat}]
           & \ac{dat}
           & \cellcolor{DG}\ac{dat}
           & \ac{dat}
           \\
     \bottomrule
  \end{tabular}
    \label{tbl:case-competition-mg-acc-dat}
\end{table}

In sum, Modern German is an instance of a language that only allows the internal case to surface. The relative pronoun surfaces in the more complex case, but only when this more complex case is the internal case.




\section{Only external case allowed}\label{sec:pattern-iii}

This section discusses the situation in which only the external case is allowed to surface when it wins the case competition. When the internal case wins the case competition, the result is ungrammatical. Schematically, this looks as in Table \ref{tbl:case-competition-only-ext-repeated} (repeated from Table \ref{tbl:case-competition-only-ext}).

\begin{table}[H]
  \center
  \caption{Only external case allowed (repeated)}
  \begin{tabular}{c|c|c|c}
    \toprule
    \textsubscript{\ac{int}} \textsuperscript{\ac{ext}}
           & [\ac{nom}]
           & [\ac{acc}]
           & [\ac{dat}]
           \\ \cmidrule{1-4}
       [\ac{nom}]
           & \ac{nom}
           & \ac{acc}
           & \ac{dat}
           \\ \cmidrule{1-4}
       [\ac{acc}]
           & *
           & \ac{acc}
           & \ac{dat}
           \\ \cmidrule{1-4}
       [\ac{dat}]
           & *
           & *
           & \ac{dat}
           \\
     \bottomrule
  \end{tabular}
    \label{tbl:case-competition-only-ext-repeated}
\end{table}

To my knowledge, this pattern is not attested in any natural language, whether extinct or alive. For Classical Greek is has been claimed in the literature that it follows this pattern. I show that Classical Greek actually patterns with Gothic and Old High German.

It has been claimed that Classical Greek only allows the external case to surface when it wins the case competition  \citep[cf.][]{cinqueforthcoming}.
%Classical Greek has been mentioned as `internal only' and as `both works' (Grosu). I show that
It does indeed seem to be the case that examples in which the external case wins over the internal case are more frequent than examples in which the internal case wins over the external case (see \citealt{kakarikos2014} for numerous examples of these cases).\footnote{
In this dissertation I do not address the question of why certain constructions and configurations are more frequent than others. My goal is to set up a system that generates the grammatical patterns and excludes the ungrammatical or unattested patterns.
} I start with an example of such a situation, in which a more complex external case wins over a less complex internal case.

Consider the example in \ref{ex:ag-dat-acc}. In this example, the internal accusative case competes against the external dative case.
The internal case is accusative, as the predicate \tit{tíktō} `to give birth to' takes accusative objects.
The external case is dative, as the predicate \tit{ékhō} `to provide' takes dative indirect objects.
The relative pronoun \tit{hō̃ͅ} `\ac{rel}.\ac{sg}.\ac{m}.\ac{acc}' appears in the internal case: the accusative. The relative pronoun is not marked in bold, unlike as the relative clause, showing that the relative pronoun patterns with the main clause.

\exg. pãn {tò tekòn} trophḕn ékhei hō̃ͅ \tbf{án} \tbf{tékēͅ}\\
any parent.\ac{sg}.\ac{nom} food.\ac{sg}.\ac{acc} provide.\ac{pres}.3\ac{sg} \ac{rel}.\ac{sg}.\ac{m}.\ac{dat} \ac{mod} {gives birth}.\ac{aor}.3\ac{sg}\\
`any parent provides food to what he would have given birth to' \flushfill{Classical Greek, \ac{pl.men} 237e, adapted from \pgcitealt{kakarikos2014}{292}}\label{ex:ag-dat-acc}

This example is compatible with the picture of Classical Greek only allowing the external case to surface when it wins the competition. In Table \ref{tbl:case-competition-ag-poss1}, I mark the example \ref{ex:ag-dat-acc} in gray in the external-only pattern taken from the beginning of this section.

\begin{table}[H]
  \center
  \caption{Classical Greek possibility 1}
  \begin{tabular}{c|c|c|c}
    \toprule
    \textsubscript{\ac{int}} \textsuperscript{\ac{ext}}
           & [\ac{nom}]
           & [\ac{acc}]
           & [\ac{dat}]
           \\ \cmidrule{1-4}
       [\ac{nom}]
           & \ac{nom}
           & \ac{acc}
           & \ac{dat}
           \\ \cmidrule{1-4}
       [\ac{acc}]
           & *
           & \ac{acc}
           & \cellcolor{LG}\ac{dat}
           \\ \cmidrule{1-4}
       [\ac{dat}]
           & *
           & *
           & \ac{dat}
           \\
     \bottomrule
  \end{tabular}
    \label{tbl:case-competition-ag-poss1}
\end{table}

However, the example is also compatible with the picture of Classical Greek allowing the internal or the external case to surface when either of them wins the case competition. In Table \ref{tbl:case-competition-ag-poss2}, I mark the example \ref{ex:ag-dat-acc} in gray in the internal-and-external pattern taken from Section \ref{sec:pattern-i}

\begin{table}[H]
  \center
  \caption{Classical Greek possibility 2}
  \begin{tabular}{c|c|c|c}
    \toprule
    \textsubscript{\ac{int}} \textsuperscript{\ac{ext}}
           & [\ac{nom}]
           & [\ac{acc}]
           & [\ac{dat}]
           \\ \cmidrule{1-4}
       [\ac{nom}]
           & \ac{nom}
           & \ac{acc}
           & \ac{dat}
           \\ \cmidrule{1-4}
       [\ac{acc}]
           & \ac{acc}
           & \ac{acc}
           & \cellcolor{LG}\ac{dat}
           \\ \cmidrule{1-4}
       [\ac{dat}]
           & \ac{dat}
           & \ac{dat}
           & \ac{dat}
           \\
     \bottomrule
  \end{tabular}
    \label{tbl:case-competition-ag-poss2}
\end{table}

What sets Table \ref{tbl:case-competition-ag-poss1} and Table \ref{tbl:case-competition-ag-poss2} apart is the lower left corner of the table. These are cases in which the internal case wins the case competition.
In Table \ref{tbl:case-competition-ag-poss1} these examples are not allowed to surface, and in Table \ref{tbl:case-competition-ag-poss2} they are.
In what follows, I give an example in which a more complex internal case wins over a less complex external case. This indicates that Classical Greek cannot be of the type shown in Table \ref{tbl:case-competition-ag-poss1}, but is has to be of the type shown in Table \ref{tbl:case-competition-ag-poss2}. In other words, it is not of the type that only allows the external case to surface when it wins the case competition.

Consider the example in \ref{ex:ag-nom-acc}. In this example, the internal accusative case competes against the external nominative case.
The internal case is accusative, as the predicate \tit{philéō} `to love' takes accusative objects.
The external case is nominative, as the predicate \tit{apothnḗiskō} `to die' takes nominative subjects.
The relative pronoun \tit{hòn} `\ac{rel}.\ac{sg}.\ac{m}.\ac{acc}' appears in the internal case: the accusative. The relative pronoun is marked in bold, just as the relative clause, showing that the relative pronoun patterns with the relative clause.\footnote{
The sentence in \ref{ex:ag-nom-acc} can also be analyzed as a headed relative, in which the relative clause modifies the phonologically empty subject of \tit{apothnḗiskō} `to die'. Then, however, more needs to be said about how it is possible for a relative clause to modify a phonologically empty element.
}

\exg. \tbf{hòn} \tbf{hoi} \tbf{theoì} \tbf{philoũsin} apothnḗͅskei néos\\
\ac{rel}.\ac{sg}.\ac{m}.\ac{acc} the god.\ac{pl} love.3\ac{pl}\scsub{[acc]} die.3\ac{sg}\scsub{[nom]} young\\
`He, whom the gods love, dies young.' \flushfill{Classical Greek, \ac{men.dd}, 125}\label{ex:ag-nom-acc}

This example shows that Classical Greek is not an instance of the third possible pattern, in which only the external case is allowed to surface. Instead, as illustrated by Table \ref{tbl:case-competition-classical-greek}, the language allows the external case (marked light gray) and the internal case (marked dark gray) to surface when either of them wins the case competition.

\begin{table}[H]
  \center
  \caption{Summary of Classical Greek headless relatives}
  \begin{tabular}{c|c|c|c}
    \toprule
    \textsubscript{\ac{int}} \textsuperscript{\ac{ext}}
           & [\ac{nom}]
           & [\ac{acc}]
           & [\ac{dat}]
           \\ \cmidrule{1-4}
       [\ac{nom}]
           & \ac{nom}
           & \ac{acc}
           & \ac{dat}
           \\ \cmidrule{1-4}
       [\ac{acc}]
           & \cellcolor{DG}\ac{acc}
           & \ac{acc}
           & \cellcolor{LG}\ac{dat}
           \\ \cmidrule{1-4}
       [\ac{dat}]
           & \ac{dat}
           & \ac{dat}
           & \ac{dat}
           \\
     \bottomrule
  \end{tabular}
    \label{tbl:case-competition-classical-greek}
\end{table}

I do not give examples that correspond to the cells not highlighted in Table \ref{tbl:case-competition-classical-greek}. The only kind of system that is compatible with the examples given is the one in which the internal or external case is allowed to surface when either of them wins the case competition. For more examples in which the external case wins, I refer the reader to \pgcitet{kakarikos2014}{292-294}. An example in which the external dative wins over the internal nominative can be found in \citet{noussia2015}. I am not aware of an example in which the internal dative wins over the external accusative.

To sum up, to my knowledge, there is no language in which only the external case is allowed to surface when it wins the case competition, and the internal case is not. Classical Greek, which has been mentioned in the literature as an instance of this pattern, actually patterns with Gothic and Old High German in that is allows the internal and the external case to surface.



\section{Internal and external case not allowed}\label{sec:pattern-iv}

This section discusses the situation in which neither the internal nor the external case is allowed to surface when it wins the case competition. When the internal case or the external is more complex, it is not allowed to surface, and the headless relative construction is ungrammatical. In other words, when the internal and the external case differ, there is no grammatical headless relative construction possible. Schematically, this looks as in Table \ref{tbl:case-competition-none-repeated} (repeated from Table \ref{tbl:case-competition-none}).

\begin{table}[H]
  \center
  \caption{Neither internal nor external allowed (repeated)}
  \begin{tabular}{c|c|c|c}
    \toprule
    \textsubscript{\ac{int}} \textsuperscript{\ac{ext}}
           & [\ac{nom}]
           & [\ac{acc}]
           & [\ac{dat}]
           \\ \cmidrule{1-4}
       [\ac{nom}]
           & \ac{nom}
           & *
           & *
           \\ \cmidrule{1-4}
       [\ac{acc}]
           & *
           & \ac{acc}
           & *
           \\ \cmidrule{1-4}
       [\ac{dat}]
           & *
           & *
           & \ac{dat}
           \\
     \bottomrule
  \end{tabular}
    \label{tbl:case-competition-none-repeated}
\end{table}

An example of a language that shows this pattern is Polish. In this section I discuss the Polish data, based on the research of \citet{citko2013} after \citet{himmelreich2017}. I only discuss the case competition between accusative and dative, as only this data is discussed. This does not change anything about the point I am making here: the only kind of system that is compatible with the examples given is the one in which neither the internal nor the external case is allowed to surface.

I give examples from the case competition between accusative and dative. According to the case scale, the dative would win over the accusative. However, as neither the internal nor the external case are allowed to surface when they win the case competition, all examples are ungrammatical.

I start with the situation in which the external case wins the competition, and there is no grammatical outcome possible.
Consider the example in \ref{ex:polish-dat-acc}. In this example, the internal accusative case competes against the external dative case.
The internal case is nominative, as the predicate \tit{wpuścić} `to let' takes accusative subjects.
The external case is accusative, as the predicate \tit{ufać} `to trust' takes dative objects.
The relative pronoun \tit{komu} `\ac{rel}.\ac{an}.\ac{dat}' appears in the external case: the dative. The relative pronoun is not marked in bold, just as the main clause, showing that the relative pronoun patterns with the main clause.
The example adheres to the case scale, but the the external case is not allowed to surface when it wins the case competition. Therefore, the example is ungrammatical.

\exg. *Jan ufa komu \tbf{-kolkwiek} \tbf{wpuścil} \tbf{do} \tbf{domu}.\\
Jan trust.\tsc{3sg}\scsub{dat} \tsc{rel}.\tsc{dat}.\tsc{m}.\tsc{sg} ever let.\tsc{3sg}\scsub{acc} to home\\
`Jan trusts whoever he let into the house.' \flushfill{Polish, adapted from \citealt{citko2013} after \citealt{himmelreich2017}{17}}\label{ex:polish-dat-acc}

The example in \ref{ex:polish-dat-acc-u} is identical to \ref{ex:polish-dat-acc}, except for that the relative pronoun appears in the internal less complex accusative case. This example is also ungrammatical: the internal case is less complex, and the internal case is not allowed to surface when it wins the case competition.

\exg. *Jan ufa \tbf{kogo} \tbf{-kolkwiek} \tbf{wpuścil} \tbf{do} \tbf{domu}.\\
Jan trust.\tsc{3sg}\scsub{dat} \tsc{rel}.\tsc{acc}.\tsc{m}.\tsc{sg} ever let.\tsc{3sg}\scsub{acc} to home\\
`Jan trusts whoever he let into the house.' \flushfill{Polish, adapted from \citealt{citko2013} after \citealt{himmelreich2017}{17}}\label{ex:polish-dat-acc-u}

Now I turn to the situation in which the internal case wins the competition, and there is also no grammatical outcome possible.
Consider the example in \ref{ex:mg-acc-dat}. In this example, the internal dative case competes against the external accusative case.
The internal case is dative, as the predicate \tit{dokuczać} `to tease' takes dative objects.
The external case is accusative, as the predicate \tit{lubić} `to like' takes accusative subjects.
The relative pronoun \tit{komu} `\ac{rel}.\ac{an}.\ac{dat}' appears in the internal case: the dative. The relative pronoun is marked in bold, just as the relative clause, showing that the relative pronoun patterns with the relative clause.
The example adheres to the case scale, but the the internal case is (just as the external case) not allowed to surface when it wins the case competition. Therefore, the example is ungrammatical.

\exg. *Jan lubi \tbf{komu} \tbf{-kolkwiek} \tbf{dokucza}.\\
Jan like.\tsc{3sg}\scsub{acc} \tsc{rel}.\tsc{dat}.\tsc{m}.\tsc{sg} ever tease.\tsc{3sg}\scsub{dat}\\
`Jan likes whoever he teases.' \flushfill{Polish, adapted from \citealt{citko2013} after \citealt{himmelreich2017}{17}}\label{ex:polish-acc-dat}

The example in \ref{ex:polish-acc-dat-u} is identical to \ref{ex:polish-acc-dat}, except for that the relative pronoun appears in the external less complex accusative case. This example is also ungrammatical: the external case is less complex, and the external case is not allowed to surface when it wins the case competition.

\exg. *Jan lubi kogo \tbf{-kolkwiek} \tbf{dokucza}.\\
Jan like.\tsc{3sg}\scsub{acc} \tsc{rel}.\tsc{acc}.\tsc{m}.\tsc{sg} ever tease.\tsc{3sg}\scsub{dat}\\
`Jan likes whoever he teases.' \flushfill{Polish, adapted from \citealt{citko2013} after \citealt{himmelreich2017}{17}}\label{ex:polish-acc-dat-u}

The two examples in which nominative and dative compete are highlighted in Table \ref{tbl:case-competition-polish}. The light gray marking corresponds to \ref{ex:mg-dat-acc}, in which the external dative wins over the internal accusative, but the relative pronoun is not allowed to surface in the dative case (or in the losing accusative case). The dark gray marking corresponds to \ref{ex:polish-acc-dat}, in which the internal dative wins over the external accusative, but the relative pronoun is not allowed to surface in the dative case (or in the losing accusative case).

\begin{table}[H]
  \center
  \caption{Summary of Polish headless relatives}
  \begin{tabular}{c|c|c}
    \toprule
    \textsubscript{\ac{int}} \textsuperscript{\ac{ext}}
           & [\ac{acc}]
           & [\ac{dat}]
           \\ \cmidrule{1-3}
       [\ac{acc}]
           & \ac{acc}
           & \cellcolor{LG}{*}
           \\ \cmidrule{1-3}
       [\ac{dat}]
           & \cellcolor{DG}{*}
           & \ac{dat}
           \\
     \bottomrule
  \end{tabular}
    \label{tbl:case-competition-polish}
\end{table}

In sum, Polish is an instance of a language that allows neither the internal case nor the external case to surface. When the internal and the external case differ in Polish, there is no way to form a grammatical headless relative construction.

\section{Summary}\label{sec:summary-2-patterns}

In case competition in headless relatives two aspects play a role. The first one is which case wins the case competition. It is a crosslinguistically stable fact that this is determiner by the case scale in \ref{ex:case-scale-two-patterns-sum}, repeated from Chapter \ref{ch:recurring}. A case more to the right on the scale wins over a case more to the left on the scale.

\ex. \ac{nom} < \ac{acc} < \ac{dat}\label{ex:case-scale-two-patterns-sum}

The second aspect is whether the internal and the external case are allowed to surface when they wins the case competition. This differs across languages. There are four possible patterns: (1) a pattern in which the external or the internal case are allowed to surface when either of them wins, (2) a pattern in which only the internal case is allowed to surface when it wins, (3) a pattern in which only the external case is allowed to surface when it wins, and (4) a pattern in which neither the internal nor the external case is allowed to surface.

Gothic, Old High German and Classical Greek are examples of languages of the first type. Modern German is an example of a language of the second type. Polish is an example of a language of the fourth type. To my knowledge, the third pattern is not attested. A summary of the patterns and languages is given in Table \ref{tbl:possible-headless-relatives-case-competition}.

\ac{ext}, \ac{int}

\begin{table}[H]
  \center
  \caption{Possible patterns in headless relatives with case competition}
    \begin{tabular}{cc|ccc}
    \toprule
    \multicolumn{2}{c}{\ac{int}>\ac{ext}}   & \multicolumn{2}{|c}{\ac{ext}>\ac{int}} &                                           \\
    \cmidrule(lr){1-2}                      \cmidrule(lr){3-4}
    \ac{int}            & \ac{ext}          & \ac{int}          & \ac{ext}            & language                                  \\
    \cmidrule(lr){1-1}  \cmidrule(lr){2-2}  \cmidrule(lr){3-3}  \cmidrule(lr){4-4}    \cmidrule(lr){5-5}
    ✔                   & *                 & *                 & ✔                   & Old High German                           \\
    ✔                   & *                 & *                 & *                   & Modern German                             \\
    {*}                 & *                 & *                 & ✔                   & n.a.                                      \\
    {*}                 & *                 & *                 & *                   & Polish                                    \\
    \bottomrule
  \end{tabular}
    \label{tbl:possible-headless-relatives-case-competition}
\end{table}

It is impossible to prove that this pattern does not exist (or has not existed) in any natural language, and it could be an accidental gap. However, in line with the available data so far, I set up a system in the next section that derives the three attested patterns, and excludes the fourth one.

\begin{figure}[H]
  \centering
    \begin{tikzpicture}[node distance=2.5cm]
    \node (question2) [question, xshift=2.5cm]
    {\ac{int} as winner?};
        \node (outcome2) [outcome, below of=question2, xshift=-2.5cm]
        {ungrammatical};
            \node (example2) [example, below of=outcome2, yshift=1.4cm]
            {\footnotesize{e.g. Polish}};
        \node (question3) [question, below of=question2, xshift=2.5cm]
        {\ac{ext} as winner?};
            \node (outcome3) [outcome, below of=question3, xshift=-2.5cm]
            {\ac{rel} = \ac{int} + complex};
                \node (example3) [example, below of=outcome3, yshift=1.4cm]
                {\footnotesize{e.g. Modern German}};
            \node (outcome4) [outcome, below of=question3, xshift=2.5cm]
            {\ac{rel} = complex};
                \node (example3) [example, below of=outcome4, yshift=1cm]
                {\footnotesize{e.g. Gothic, Old High German, Classical Greek}};

    \draw [arrow] (question2) -- node[anchor=east] {no} (outcome2);
    \draw [arrow] (question2) -- node[anchor=west] {yes} (question3);
    \draw [arrow] (question3) -- node[anchor=east] {no} (outcome3);
    \draw [arrow] (question3) -- node[anchor=west] {yes} (outcome4);
    \end{tikzpicture}

    \caption{Overview attested headless relatives with case competition}
    \label{fig:attested-headless-relatives-case-competition}
\end{figure}




\section{Aside: languages without case competition}\label{sec:aside-potentiel-counterexamples}

Two languages that come close to being of the third type discussed in Section \ref{sec:pattern-iii} are Old English and Modern Greek. In this section I show that these two languages are actually languages that lack case competition.

In this chapter so far, I discussed languages that show case competition. There are also languages that do not show any case competition. In these languages, the internal case and the external case do not compete to show their case on the relative pronoun. It is irrelevant how the two cases relate to each other on the case scale. Instead, it is fixed per language whether the relative pronoun appears in the external or the internal case. Logically, there are two possible languages: one that lets the relative pronoun appear in the internal case, and one that lets the relative pronoun appear in the external case.

Table \ref{tbl:no-case-competition-int} shows the pattern of a language in which the relative pronoun always appears in the internal case. In the second row, the internal case is nominative and the external case is either accusative or dative. The relative pronoun appears in the nominative. It is irrelevant here that the nominative is less complex than the accusative and the dative, because there is no case competition taking place. The third row shows that the relative pronoun always appears in the accusative when the internal case is the accusative, and the fourth row shows the same for the dative. To my knowledge, this type is not attested in any natural language.

\begin{table}[H]
  \center
  \caption{Always internal case}
  \begin{tabular}{c|c|c|c}
    \toprule
   \textsubscript{\ac{int}} \textsuperscript{\ac{ext}}
          & [\ac{nom}]
          & [\ac{acc}]
          & [\ac{dat}]
          \\ \cmidrule{1-4}
      [\ac{nom}]
          & \ac{nom}
          & \ac{nom}
          & \ac{nom}
          \\ \cmidrule{1-4}
      [\ac{acc}]
          & \ac{acc}
          & \ac{acc}
          & \ac{acc}
          \\ \cmidrule{1-4}
      [\ac{dat}]
          & \ac{dat}
          & \ac{dat}
          & \ac{dat}
          \\
    \bottomrule
  \end{tabular}
  \label{tbl:no-case-competition-int}
\end{table}

Table \ref{tbl:no-case-competition-ext} shows the pattern of a language in which the relative pronoun always appears in the external case. In the second column, the external case is nominative and the internal case is either accusative or dative. The relative pronoun appears in the nominative. It is irrelevant here that the nominative is less complex than the accusative and the dative, because there is no case competition taking place. The third column shows that the relative pronoun always appears in the accusative when the external case is the accusative, and the fourth column shows the same for the dative.

\begin{table}[H]
  \center
  \caption{Always external case}
  \begin{tabular}{c|c|c|c}
    \toprule
   \textsubscript{\ac{int}} \textsuperscript{\ac{ext}}
          & [\ac{nom}]
          & [\ac{acc}]
          & [\ac{dat}]
          \\ \cmidrule{1-4}
      [\ac{nom}]
          & \ac{nom}
          & \ac{acc}
          & \ac{dat}
          \\ \cmidrule{1-4}
      [\ac{acc}]
          & \ac{nom}
          & \ac{acc}
          & \ac{dat}
          \\ \cmidrule{1-4}
      [\ac{dat}]
          & \ac{nom}
          & \ac{acc}
          & \ac{dat}
          \\
    \bottomrule
  \end{tabular}
  \label{tbl:no-case-competition-ext}
\end{table}


\subsection{Always external case}

In this section I discuss two languages in which the relative pronoun always appears in the external case. I show that these languages do not show any case competition. In other words, these languages are of the type shown in Table \ref{tbl:no-case-competition-ext} and not of the type I discussed in Section \ref{sec:pattern-iii} (or of the one in Section \ref{sec:pattern-i}).

I start with Old English. I give an example in which the external case is more complex than the internal case and the relative pronoun appears in the more complex external case.

Consider the example in \ref{ex:oe-dat-nom}.
The internal case is nominative, as the predicate \tit{gegyltan} `to sin' takes nominative subjects.
The external case is accusative, as the predicate \tit{for-gifan} `to forgive' takes dative objects.
The relative pronoun \tit{ðam} `\ac{rel}.\ac{dat}.\ac{pl}' appears in the external case: the dative. The relative pronoun is not marked in bold, unlike the relative clause, showing that the relative pronoun patterns with the main clause.

\exg. ðæt is, ðæt man for-gife, ðam \tbf{ðe} \tbf{wið} \tbf{hine} \tbf{gegylte}\\
 that is that one forgive.\ac{subj}.\ac{sg}\scsub{[dat]} \ac{rel}.\ac{dat}.\ac{pl} \ac{comp} against 3\ac{sg}.\ac{m}.\ac{acc} sin.3\ac{sg}\scsub{[nom]}\\
 `that is, that one₂ forgive him₁, who sins against him₂' \flushfill{Old English, adapted from \pgcitealt{harbert1983}{549}} \label{ex:oe-dat-nom}

This example is compatible with three patterns. First, Old English could be a case competition language that only allows the external case to surface. In Table \ref{tbl:oe-poss1}, I mark the example \ref{ex:oe-dat-nom} in gray in the external-only pattern taken from Section \ref{sec:pattern-iii}.

 \begin{table}[H]
   \center
   \caption{Old English possibility 1}
   \begin{tabular}{c|c|c|c}
     \toprule
     \textsubscript{\ac{int}} \textsuperscript{\ac{ext}}
            & [\ac{nom}]
            & [\ac{acc}]
            & [\ac{dat}]
            \\ \cmidrule{1-4}
        [\ac{nom}]
            & \ac{nom}
            & \ac{acc}
            & \cellcolor{LG}\ac{dat}
            \\ \cmidrule{1-4}
        [\ac{acc}]
            & *
            & \ac{acc}
            & \ac{dat}
            \\ \cmidrule{1-4}
        [\ac{dat}]
            & *
            & *
            & \ac{dat}
            \\
      \bottomrule
   \end{tabular}
     \label{tbl:oe-poss1}
 \end{table}

Second, Old English could be a case competition language that allows the internal case and external case to surface. In Table \ref{tbl:oe-poss2}, I mark the example \ref{ex:oe-dat-nom} in gray in the internal-and-external pattern repeated from Section \ref{sec:pattern-i}.

  \begin{table}[H]
    \center
    \caption{Old English possibility 2}
    \begin{tabular}{c|c|c|c}
      \toprule
      \textsubscript{\ac{int}} \textsuperscript{\ac{ext}}
             & [\ac{nom}]
             & [\ac{acc}]
             & [\ac{dat}]
             \\ \cmidrule{1-4}
         [\ac{nom}]
             & \ac{nom}
             & \ac{acc}
             & \cellcolor{LG}\ac{dat}
             \\ \cmidrule{1-4}
         [\ac{acc}]
             & \ac{acc}
             & \ac{acc}
             & \ac{dat}
             \\ \cmidrule{1-4}
         [\ac{dat}]
             & \ac{dat}
             & \ac{dat}
             & \ac{dat}
             \\
       \bottomrule
    \end{tabular}
      \label{tbl:oe-poss2}
  \end{table}

Third, Old English could be a language without case competition that lets the relative pronoun appear in the external case. In Table \ref{tbl:oe-poss3}, I mark the example \ref{ex:oe-dat-nom} in gray in the always-external pattern repeated from Table \ref{tbl:no-case-competition-ext}.

 \begin{table}[H]
   \center
   \caption{Old English possibility 3}
   \begin{tabular}{c|c|c|c}
     \toprule
    \textsubscript{\ac{int}} \textsuperscript{\ac{ext}}
           & [\ac{nom}]
           & [\ac{acc}]
           & [\ac{dat}]
           \\ \cmidrule{1-4}
       [\ac{nom}]
           & \ac{nom}
           & \ac{acc}
           & \cellcolor{LG}\ac{dat}
           \\ \cmidrule{1-4}
       [\ac{acc}]
           & \ac{nom}
           & \ac{acc}
           & \ac{dat}
           \\ \cmidrule{1-4}
       [\ac{dat}]
           & \ac{nom}
           & \ac{acc}
           & \ac{dat}
           \\
     \bottomrule
   \end{tabular}
   \label{tbl:oe-poss3}
 \end{table}

What sets Table \ref{tbl:oe-poss1}, Table \ref{tbl:oe-poss2} and Table \ref{tbl:oe-poss3} apart is the lower left corner of the table. These are cases in which the internal case is more complex than the external case.

In Table \ref{tbl:oe-poss1} the winning case is not allowed to surface, and there is no grammatical headless relative possible. If this is the pattern that Old English shows, then it would be a language with case competition that only allows the external case to surface, i.e. it would be of the type of Section \ref{sec:pattern-iii} I claimed did not exist.

In Table \ref{tbl:oe-poss2} and in Table \ref{tbl:oe-poss3} there is a relative pronoun that can surface, but the case of the relative pronouns differs. In Table \ref{tbl:oe-poss2}, the relative pronoun surfaces in the more complex case that wins the case competition: the internal case. In Table \ref{tbl:oe-poss3}, there is no case competition taking place, and the relative pronoun surfaces in the external case.

In the example that follows I show that Old English is of the type in Table \ref{tbl:oe-poss3}. I give an example in which the internal case is more complex than the external one. Nevertheless, the relative pronoun surfaces in the less complex external case. Old English is namely a language without case competition that lets the relative pronoun surface in the external case.

Consider the example in \ref{ex:oe-acc-dat}.
The internal case is dative, as the preposition \tit{onuppan} `upon' takes dative objects.
The external case is accusative, as the predicate \tit{tōbrȳsan} `to pulversize' takes accusative objects.
The relative pronoun \tit{ðone} `\ac{rel}.\ac{sg}.\ac{acc}' appears in the external case: the accusative.
The relative pronoun appears in the external case, although it is the less complex case of the two. The example is grammatical, because Old English does not show case competition, so the case scale is irrelevant. As long as the relative pronoun appears in the external case, the headless relative is grammatical.

\exg. he tobryst ðone \tbf{ðe} \tbf{he} \tbf{onuppan} \tbf{fylð}\\
 it pulverizes\scsub{[acc]} \ac{rel}.\ac{sg}.\ac{acc} \ac{comp} it upon\scsub{[dat]} falls\\
`It pulverizes him whom it falls upon.' \flushfill{Old English, adapted from \pgcitealt{harbert1983}{550}} \label{ex:oe-acc-dat}

This example shows that Old English is not an instance of the pattern in Section \ref{sec:pattern-iii}, in which only the external case is allowed to surface. Instead, as illustrated by Table \ref{tbl:no-case-competition-old-english}, the language does not have any case competition. The relative pronoun appears in the external case: the external case can be the more complex case, illustrated by the example in \ref{ex:oe-dat-nom}, marked here in light gray, or the external case can be the less complex case, illustrated by the example in \ref{ex:oe-acc-dat}, marked here in dark gray.

\begin{table}[H]
  \center
  \caption{Summary of Old English headless relatives}
  \begin{tabular}{c|c|c|c}
    \toprule
   \textsubscript{\ac{int}} \textsuperscript{\ac{ext}}
          & [\ac{nom}]
          & [\ac{acc}]
          & [\ac{dat}]
          \\ \cmidrule{1-4}
      [\ac{nom}]
          & \ac{nom}
          & \ac{acc}
          & \cellcolor{LG}\ac{dat}
          \\ \cmidrule{1-4}
      [\ac{acc}]
          & \ac{nom}
          & \ac{acc}
          & \ac{dat}
          \\ \cmidrule{1-4}
      [\ac{dat}]
          & \ac{nom}
          & \cellcolor{DG}\ac{acc}
          & \ac{dat}
          \\
    \bottomrule
  \end{tabular}
  \label{tbl:no-case-competition-old-english}
\end{table}

I do not give examples that correspond to the cells not highlighted in Table \ref{tbl:no-case-competition-old-english}. The only kind of system that is compatible with the examples given is the one in which the relative pronoun always appears in the external case.

The same pattern appears in Modern Greek. The only difference is that Modern Greek has the genitive, and not the dative. I start again with an example in which the external case is more complex than the internal case and the relative pronoun appears in the more complex external case.

Consider the example in \ref{ex:greek-acc-nom}.
The internal case is nominative, as the predicate \tit{voíθisó} `to help' takes nominative subjects.
The external case is accusative, as the predicate \tit{efχarístisó} `to thank' takes accusative objects.
The relative pronoun \tit{ópjus} `\ac{rel}.\ac{pl}.\ac{m}.\ac{acc}' appears in the external case: the accusative. The relative pronoun is not marked in bold, unlike the relative clause, showing that the relative pronoun patterns with the main clause.

\exg. Efχarístisa ópjus \tbf{me} \tbf{voíϑisan}.\\
thank.\ac{pst}.3\ac{pl}\scsub{[acc]} \ac{rel}.\ac{pl}.\ac{m}.\ac{acc} \ac{cl}.1\ac{sg}.\ac{acc} help.\ac{pst}.3\ac{pl}\scsub{[nom]}\\
`I thanked whoever helped me.' \flushfill{Modern Greek, adapted from \pgcitealt{daskalaki2011}{80}}\label{ex:greek-acc-nom}

This example is compatible with three patterns. First, Modern Greek could be a case competition language that only allows the external case to surface. In Table \ref{tbl:greek-poss1}, I mark the example \ref{ex:greek-acc-nom} in gray in the external-only pattern taken from Section \ref{sec:pattern-iii}.

 \begin{table}[H]
   \center
   \caption{Modern Greek possibility 1}
   \begin{tabular}{c|c|c|c}
     \toprule
     \textsubscript{\ac{int}} \textsuperscript{\ac{ext}}
            & [\ac{nom}]
            & [\ac{acc}]
            & [\ac{gen}]
            \\ \cmidrule{1-4}
        [\ac{nom}]
            & \ac{nom}
            & \cellcolor{LG}\ac{acc}
            & \ac{gen}
            \\ \cmidrule{1-4}
        [\ac{acc}]
            & *
            & \ac{acc}
            & \ac{gen}
            \\ \cmidrule{1-4}
        [\ac{gen}]
            & *
            & *
            & \ac{dat}
            \\
      \bottomrule
   \end{tabular}
     \label{tbl:greek-poss1}
 \end{table}

Second, Modern Greek could be a case competition language that allows the internal case and external case to surface. In Table \ref{tbl:greek-poss2}, I mark the example \ref{ex:greek-acc-nom} in gray in the internal-and-external pattern repeated from Section \ref{sec:pattern-i}.

  \begin{table}[H]
    \center
    \caption{Modern Greek possibility 2}
    \begin{tabular}{c|c|c|c}
      \toprule
      \textsubscript{\ac{int}} \textsuperscript{\ac{ext}}
             & [\ac{nom}]
             & [\ac{acc}]
             & [\ac{gen}]
             \\ \cmidrule{1-4}
         [\ac{nom}]
             & \ac{nom}
             & \cellcolor{LG}\ac{acc}
             & \ac{gen}
             \\ \cmidrule{1-4}
         [\ac{acc}]
             & \ac{acc}
             & \ac{acc}
             & \ac{gen}
             \\ \cmidrule{1-4}
         [\ac{gen}]
             & \ac{gen}
             & \ac{gen}
             & \ac{dat}
             \\
       \bottomrule
    \end{tabular}
      \label{tbl:greek-poss2}
  \end{table}

Third, Modern Greek could be a language without case competition that lets the relative pronoun appear in the external case. In Table \ref{tbl:greek-poss3}, I mark the example \ref{ex:greek-acc-nom} in gray in the always-external pattern repeated from Table \ref{tbl:no-case-competition-ext}.

 \begin{table}[H]
   \center
   \caption{Modern Greek possibility 3}
   \begin{tabular}{c|c|c|c}
     \toprule
    \textsubscript{\ac{int}} \textsuperscript{\ac{ext}}
           & [\ac{nom}]
           & [\ac{acc}]
           & [\ac{gen}]
           \\ \cmidrule{1-4}
       [\ac{nom}]
           & \ac{nom}
           & \cellcolor{LG}\ac{acc}
           & \ac{gen}
           \\ \cmidrule{1-4}
       [\ac{acc}]
           & \ac{nom}
           & \ac{acc}
           & \ac{gen}
           \\ \cmidrule{1-4}
       [\ac{gen}]
           & \ac{nom}
           & \ac{acc}
           & \ac{dat}
           \\
     \bottomrule
   \end{tabular}
   \label{tbl:greek-poss3}
 \end{table}

What sets Table \ref{tbl:greek-poss1}, Table \ref{tbl:greek-poss2} and Table \ref{tbl:greek-poss3} apart is the lower left corner of the table. These are cases in which the internal case is more complex than the external case.

In Table \ref{tbl:greek-poss1} the winning case is not allowed to surface, and there is no grammatical headless relative possible. If this is the pattern that Modern Greek shows, then it would be a language with case competition that only allows the external case to surface, i.e. it would be of the type of Section \ref{sec:pattern-iii} I claimed did not exist.

In Table \ref{tbl:greek-poss2} and in Table \ref{tbl:greek-poss3} there is a relative pronoun that can surface, but the case of the relative pronouns differs. In Table \ref{tbl:greek-poss2}, the relative pronoun surfaces in the more complex case that wins the case competition: the internal case. In Table \ref{tbl:greek-poss3}, there is no case competition taking place, and the relative pronoun surfaces in the external case.

In the example that follows I show that Modern Greek is of the type in Table \ref{tbl:greek-poss3}. I give an example in which the internal case is more complex than the external one. Nevertheless, the relative pronoun surfaces in the less complex external case. Modern Greek is namely a language without case competition that lets the relative pronoun surface in the external case.

Consider the example in \ref{ex:greek-nom-acc}.
The internal case is accusative, as the predicate \tit{irθó} `to invite' takes accusative objects.
The external case is accusative, as the predicate \tit{kálesó} `to come' takes nominative subjects.
The relative pronoun \tit{ópji} `\ac{rel}.\ac{pl}.\ac{m}.\ac{nom}' appears in the external case: the nominative.
The relative pronoun appears in the external case, although it is the less complex case of the two. The example is grammatical, because Modern Greek does not show case competition, so the case scale is irrelevant. As long as the relative pronoun appears in the external case, the headless relative is grammatical.

\exg. Irθan ópji \tbf{káleses}.\\
come.\ac{pst}.3\ac{pl}\scsub{[nom]} \ac{rel}.\ac{pl}.\ac{m}.\ac{nom} invite.\ac{pst}.2\ac{sg}\scsub{[acc]}\\
`Whoever you invited came.'\flushfill{Modern Greek, adapted from \pgcitealt{daskalaki2011}{80}}\label{ex:greek-nom-acc}

The example in \ref{ex:greek-nom-acc-u} is identical to \ref{ex:greek-nom-acc}, except for that the relative pronoun appears in the internal  more complex case. This example is ungrammatical: the relative pronoun does not appear in the external case. The fact that the internal case is more complex is irrelevant.

\exg. *Irθan \tbf{ópjus} \tbf{káleses}.\\
come.\ac{pst}.3\ac{pl}\scsub{[nom]} \ac{rel}.\ac{pl}.\ac{m}.\ac{acc} invite.\ac{pst}.2\ac{sg}\scsub{[acc]}\\
`Whoever you invited came.'\flushfill{Modern Greek, adapted from \pgcitealt{daskalaki2011}{79}}\label{ex:greek-nom-acc-u}

This example shows that Modern Greek is not an instance of the pattern in Section \ref{sec:pattern-iii}, in which only the external case is allowed to surface. Instead, as illustrated by Table \ref{tbl:no-case-competition-greek}, the language does not have any case competition. The relative pronoun appears in the external case: the external case can be the more complex case, illustrated by the example in \ref{ex:greek-acc-nom}, marked here in light gray, or the external case can be the less complex case, illustrated by the example in \ref{ex:greek-nom-acc}, marked here in dark gray.

\begin{table}[H]
  \center
  \caption{Summary of Modern Greek headless relatives}
  \begin{tabular}{c|c|c|c}
    \toprule
   \textsubscript{\ac{int}} \textsuperscript{\ac{ext}}
          & [\ac{nom}]
          & [\ac{acc}]
          & [\ac{gen}]
          \\ \cmidrule{1-4}
      [\ac{nom}]
          & \ac{nom}
          & \cellcolor{LG}\ac{acc}
          & \ac{gen}
          \\ \cmidrule{1-4}
      [\ac{acc}]
          & \cellcolor{DG}\ac{nom}
          & \ac{acc}
          & \ac{gen}
          \\ \cmidrule{1-4}
      [\ac{gen}]
          & \ac{nom}
          & \ac{acc}
          & \ac{dat}
          \\
    \bottomrule
  \end{tabular}
  \label{tbl:no-case-competition-greek}
\end{table}

There is something more to be said about the situation in Modern Greek. When the internal case is genitive instead of accusative, a clitic is added to the sentence to make it grammatical \citep{daskalaki2011}.

Consider the example in \ref{ex:greek-nom-gen}.
The internal case is genitive, as the predicate \tit{eðósó} `to give' takes genitive objects.
The external case is accusative, as the predicate \tit{efχarístisó} `to thank' takes nominative subjects.
The relative pronoun \tit{ópjon} `\ac{rel}.\ac{pl}.\ac{m}.\ac{nom}' appears in the external case: the nominative.
The relative pronoun appears in the external case, although it is the less complex case of the two. The example is grammatical, because Modern Greek does not show case competition, so the case scale is irrelevant. As long as the relative pronoun appears in the external case, the headless relative is grammatical. In addition, the relative clause obligatorily contains the genitive clitic \tit{tus} `\ac{cl}.3\ac{pl}.\ac{gen}'.\footnote{
In Modern German, it is possible to insert a light head to resolve a situation with a more complex external case. However, then the relative pronoun has to change as well (from a \tsc{wh}-pronoun into a \tsc{d}-pronoun). I assume this is a different construction, and the Modern Greek one with the clitic inserted is not.
}

\exg. Me efχarístisan ópji \tbf{tus} \tbf{íχa} \tbf{ðósi} \tbf{leftá}.\\
 \ac{cl}.1\ac{sg}.\ac{acc} thank.\ac{pst}.3\ac{pl}\scsub{[nom]} \ac{rel}.\ac{pl}.\ac{m}.\ac{nom} \ac{cl}.3\ac{pl}.\ac{gen} have.\ac{pst}.1\ac{sg} give.\ac{ptcp}\scsub{[gen]} money\\
 `Whoever I had given money to, thanked me.'\flushfill{Modern Greek, adapted from \pgcitealt{daskalaki2011}{80}}\label{ex:greek-nom-gen}

This once again confirms the picture of Modern Greek always letting the relative pronoun surface in the external case. Even if the internal case is clearly more complex, the relative pronoun surfaces in the external case. The internal case is taken care of by the clitic, which independent of the relative clause construction.

I do not give examples that correspond to the cells not highlighted in Table \ref{tbl:no-case-competition-greek}. The only kind of system that is compatible with the examples given is the one in which the relative pronoun always appears in the external case. For more examples that illustrate this pattern, I refer the reader to \pgcitet{daskalaki2011}{79-80}.

Taking this all together, I have not encountered a language that only allows the external case to surface when it wins the case competition. Modern Greek and Old English come close, because they always let the relative pronoun surface in the external case. The crucial difference with the non-attested pattern is that they do not show any case competition.


\subsection{A typology of headless relatives}

I combine


\begin{table}[H]
  \center
  \caption{Always internal case}
  \begin{tabular}{c|c|c|c}
    \toprule
   \textsubscript{\ac{int}} \textsuperscript{\ac{ext}}
          & [\ac{nom}]
          & [\ac{acc}]
          & [\ac{dat}]
          \\ \cmidrule{1-4}
      [\ac{nom}]
          & \ac{nom}
          & \ac{nom}
          & \ac{nom}
          \\ \cmidrule{1-4}
      [\ac{acc}]
          & \ac{acc}
          & \ac{acc}
          & \ac{acc}
          \\ \cmidrule{1-4}
      [\ac{dat}]
          & \ac{dat}
          & \ac{dat}
          & \ac{dat}
          \\
    \bottomrule
  \end{tabular}
  \label{tbl:no-case-competition-int}
\end{table}

\begin{table}[H]
  \center
  \caption{Always external case}
  \begin{tabular}{c|c|c|c}
    \toprule
   \textsubscript{\ac{int}} \textsuperscript{\ac{ext}}
          & [\ac{nom}]
          & [\ac{acc}]
          & [\ac{dat}]
          \\ \cmidrule{1-4}
      [\ac{nom}]
          & \ac{nom}
          & \ac{acc}
          & \ac{dat}
          \\ \cmidrule{1-4}
      [\ac{acc}]
          & \ac{nom}
          & \ac{acc}
          & \ac{dat}
          \\ \cmidrule{1-4}
      [\ac{dat}]
          & \ac{nom}
          & \ac{acc}
          & \ac{dat}
          \\
    \bottomrule
  \end{tabular}
  \label{tbl:no-case-competition-ext}
\end{table}


\begin{table}[H]
  \center
  \caption{Either internal or external case allowed}
  \begin{tabular}{c|c|c|c}
    \toprule
    \textsubscript{\ac{int}} \textsuperscript{\ac{ext}}
           & [\ac{nom}]
           & [\ac{acc}]
           & [\ac{dat}]
           \\ \cmidrule{1-4}
       [\ac{nom}]
           & \ac{nom}
           & \ac{acc}
           & \ac{dat}
           \\ \cmidrule{1-4}
       [\ac{acc}]
           & \ac{acc}
           & \ac{acc}
           & \ac{dat}
           \\ \cmidrule{1-4}
       [\ac{dat}]
           & \ac{dat}
           & \ac{dat}
           & \ac{dat}
           \\
     \bottomrule
  \end{tabular}
    \label{tbl:case-competition-int-ext-typology}
\end{table}


\begin{table}[H]
  \center
  \caption{Only internal case allowed}
  \begin{tabular}{c|c|c|c}
    \toprule
    \textsubscript{\ac{int}} \textsuperscript{\ac{ext}}
           & [\ac{nom}]
           & [\ac{acc}]
           & [\ac{dat}]
           \\ \cmidrule{1-4}
       [\ac{nom}]
           & \ac{nom}
           & *
           & *
           \\ \cmidrule{1-4}
       [\ac{acc}]
           & \ac{acc}
           & \ac{acc}
           & *
           \\ \cmidrule{1-4}
       [\ac{dat}]
           & \ac{dat}
           & \ac{dat}
           & \ac{dat}
           \\
     \bottomrule
  \end{tabular}
    \label{tbl:case-competition-only-int-typology}
\end{table}


\begin{table}[H]
  \center
  \caption{Only external case allowed}
  \begin{tabular}{c|c|c|c}
    \toprule
    \textsubscript{\ac{int}} \textsuperscript{\ac{ext}}
           & [\ac{nom}]
           & [\ac{acc}]
           & [\ac{dat}]
           \\ \cmidrule{1-4}
       [\ac{nom}]
           & \ac{nom}
           & \ac{acc}
           & \ac{dat}
           \\ \cmidrule{1-4}
       [\ac{acc}]
           & *
           & \ac{acc}
           & \ac{dat}
           \\ \cmidrule{1-4}
       [\ac{dat}]
           & *
           & *
           & \ac{dat}
           \\
     \bottomrule
  \end{tabular}
    \label{tbl:case-competition-only-ext-typology}
\end{table}


\begin{table}[H]
  \center
  \caption{Neither internal nor external allowed}
  \begin{tabular}{c|c|c|c}
    \toprule
    \textsubscript{\ac{int}} \textsuperscript{\ac{ext}}
           & [\ac{nom}]
           & [\ac{acc}]
           & [\ac{dat}]
           \\ \cmidrule{1-4}
       [\ac{nom}]
           & \ac{nom}
           & *
           & *
           \\ \cmidrule{1-4}
       [\ac{acc}]
           & *
           & \ac{acc}
           & *
           \\ \cmidrule{1-4}
       [\ac{dat}]
           & *
           & *
           & \ac{dat}
           \\
     \bottomrule
  \end{tabular}
    \label{tbl:case-competition-none-typology}
\end{table}







\begin{table}[H]
  \center
  \caption{Possible patterns in headless relatives}
    \begin{tabular}{ccc|ccc}
    \toprule
      &   \multicolumn{2}{c}{\ac{int}>\ac{ext}}   & \multicolumn{2}{|c}{\ac{ext}>\ac{int}}  &                  \\
          \cmidrule(lr){2-3}                      \cmidrule(lr){4-5}
      &   \ac{int}            & \ac{ext}          & \ac{int}          & \ac{ext}            & language         \\
          \cmidrule(lr){2-2}  \cmidrule(lr){3-3}  \cmidrule(lr){4-4}  \cmidrule(lr){5-5}    \cmidrule(lr){6-6}
    1 &   ✔                   & *                 & ✔               & *                     & n.a.             \\
    2 &   ✔                   & *                 & *               & ✔                     & Old High German  \\
    3 &   ✔                   & *                 & *               & *                     & Modern German    \\
    4 &   {*}                 & ✔                 & ✔               & *                     & n.a.             \\
    5 &   {*}                 & ✔                 & *               & ✔                     & Old English      \\
    6 &   {*}                 & ✔                 & *               & *                     & n.a.             \\
    7 &   {*}                 & *                 & ✔               & *                     & n.a.             \\
    8 &   {*}                 & *                 & *               & ✔                     & n.a.             \\
    9 &   {*}                 & *                 & *               & *                     & Polish           \\
    \bottomrule
  \end{tabular}
    \label{tbl:possible-headless-relatives}
\end{table}

\begin{figure}[H]
  \centering
    \begin{tikzpicture}[node distance=2cm]
    \node (question1) [question]
    {case\\ competition?};
        \node (question4) [question, below of=question1, xshift=-3cm, yshift=-1cm]
        {\ac{int}/\ac{ext}?};
            \node (outcome5) [outcome, below of=question4, xshift=-2cm]
            {\ac{int}};
                \node (example5) [example, below of=outcome5, yshift=0.5cm]
                {\footnotesize{n.a.\\\phantom{x}\\\phantom{x}}};
            \node (outcome1) [outcome, below of=question4, xshift=2cm]
            {\ac{ext}};
                \node (example1) [example, below of=outcome1, yshift=0.5cm]
                {\footnotesize{e.g. Old English, Modern Greek\\\phantom{x}}};
        \node (question2) [question, below of=question1, xshift=3cm, yshift=-1cm]
        {\ac{int} as winner?};
            \node (outcome2) [outcome, below of=question2, xshift=-2cm]
            {*};
                \node (example2) [example, below of=outcome2, yshift=0.5cm]
                {\footnotesize{e.g. Polish\\\phantom{x}\\\phantom{x}}};
            \node (question3) [question, below of=question2, xshift=2.5cm, yshift=-0.5cm]
            {\ac{ext} as winner?};
                \node (outcome3) [outcome, below of=question3, xshift=-2cm]
                {\ac{int} + complex};
                    \node (example3) [example, below of=outcome3, yshift=0.5cm]
                    {\footnotesize{e.g. Modern German\\\phantom{x}\\\phantom{x}}};
                \node (outcome4) [outcome, below of=question3, xshift=2cm]
                {complex};
                    \node (example3) [example, below of=outcome4, yshift=0.5cm]
                    {\footnotesize{e.g. Gothic, Old High German, Classical Greek}};

    \draw [arrow] (question1) -- node[anchor=west] {yes} (question2);
    \draw [arrow] (question2) -- node[anchor=east] {no} (outcome2);
    \draw [arrow] (question2) -- node[anchor=west] {yes} (question3);
    \draw [arrow] (question3) -- node[anchor=east] {no} (outcome3);
    \draw [arrow] (question3) -- node[anchor=west] {yes} (outcome4);
    \draw [arrow] (question1) -- node[anchor=east] {no} (question4);
    \draw [arrow] (question4) -- node[anchor=east] {\tsc{int}} (outcome5);
    \draw [arrow] (question4) -- node[anchor=west] {\tsc{ext}} (outcome1);
    \end{tikzpicture}

    \caption{Overview attested headless relatives}
    \label{fig:typology1}
\end{figure}

hallo
