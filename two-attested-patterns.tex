% !TEX root = thesis.tex

\chapter{Two attested patterns}

I showed that headless relatives crosslinguistically make reference to the case scale: if two cases are in competition, it is always the same that wins.
However, there is a second aspect to headless relatives, which differs across languages. That is, sometimes the case competition does not even take place, or rather, there is no winner to the competition. I show that whether is not the case competition takes place depends on the type of relative pronoun that is used. I distinguish two types of relative pronouns: \tsc{wh}-pronouns and \tsc{d}-pronouns.

In what follows I first discuss the different possibilities for languages that show case competition.
There are languages in which both the internal and the external case can win (like Gothic). There are also languages in which only the internal case can win. Crucially, there is no language in which a case competition takes place but only the external case can win.
I summarize this in a table:

\begin{table}[H]
 \center
 \caption {Variation}
  \begin{tabular}{ccc}
  \toprule
        & \ac{int}>\ac{ext}  & \ac{ext}>\ac{int} \\
        \cmidrule{2-3}
  Gothic OHG  & ✔          & ✔         \\
  \ac{mg}  & ✔           & *         \\
  n.a.     & *          & ✔         \\
  \bottomrule
  \end{tabular}
\end{table}

Now there are also languages that do not do case competition. There are two types I found so far too:
There are languages in which the external case always surfaces, no matter the internal case. These languages are for instance Old English and and Greek.
Another type of language does not allow for any type of mismatch.
What I have not seen so far: a language in which the internal case surfaces, no matter what happens with the external one.

\section{Internal or external wins}

\subsection{Gothic}

\begin{table}[H]
  \center
  \caption{Summary Gothic headless relatives (repeated)}
    % !TEX root = ../thesis.tex

\begin{tabular}{c|c|c|c}
  \toprule
      \textsubscript{\ac{int}} \textsuperscript{\ac{ext}}
        & [\ac{nom}]
        & [\ac{acc}]
        & [\ac{dat}]
        \\ \cmidrule{1-4}
    [\ac{nom}]
        & \ac{nom}
        & \ac{acc}
        & \ac{dat}
        \\ \cmidrule{1-4}
    [\ac{acc}]
        & \ac{acc}
        & \ac{acc}
        & \ac{dat}
        \\ \cmidrule{1-4}
    [\ac{dat}]
        & \ac{dat}
        & (\ac{dat})
        & \ac{dat}
        \\
  \bottomrule
\end{tabular}

    \label{tbl:summary-gothic-repeated}
\end{table}


\subsection{Old High German}

% Old High German is widely discussed in the literature because of its case attraction (references).
%
% \ex. example case attraction
%
% by extension, examples with external case surfacing in headless relatives are also discussed
% to the best of my knowledge, there is no systematic study of direction: int to ext or ext to int

Old High German shows the same pattern as Gothic. The relative pronoun can surface in the internal or the external case, depending on which case is more to the right on the case scale. This conclusion follows from my own research of the texts Der althochdeutsche Isidor, The Monsee fragments, Otfrid's Evangelienbuch and Tatian in ANNIS3.\footnote{
https://korpling.german.hu-berlin.de/annis3/
} The examples below follow the spelling and the detailed glosses in ANNIS3. The translations are my own.

Consider the example in \ref{ex:ohg-acc-nom}. In this example, the internal case is nominative and the external case is accusative.
The internal case is nominative. The predicate \tit{gisizzen} `to possess' takes nominative subjects.
The external case is accusative. The predicate \tit{bibringan} `to give rise to' takes accusative objects.
The relative pronoun \tit{dhen} `\ac{rel}.\ac{acc}.\ac{m}.\ac{sg}' appears in the external case: the accusative. The relative pronoun is not marked in bold, just like as the main clause, showing that the relative pronoun patterns with the main clause.

% Context: the voice of God speaks through him (I think Jesaja):

\exg. ``Ih bibringu fona iacobes samin endi fona iuda dhen \tbf{mina} \tbf{berga} \tbf{chisitzit}.''\\
1\tsc{sg}.\tsc{nom} {give rise to/send}.\tsc{1sg}\scsub{[acc]} from Jakob.\tsc{gen} Samuel.\tsc{dat} and from Judas.\tsc{abl} \tsc{rel}.\tsc{acc}.\tsc{m}.\tsc{sg} my.\tsc{acc}.\tsc{m}.\tsc{pl} mountain.\tsc{acc}.\tsc{pl} possess.3\tsc{sg}\scsub{[nom]}\\
`I give rise to from Samuel of Jakob and from Judas him who possesses my mountains.' \flushfill{\ac{ohg}, \ac{isid} 34:3}\label{ex:ohg-acc-nom}\\
% Lat `Educam de Iacob semine et de Iuda possedentem montes meos.'

Consider the example in \ref{ex:ohg-dat-nom}. In this example, the internal case is nominative and the external case is dative.
The internal case is nominative. The predicate \tit{sprehhan} `to speak' takes nominative subjects.
The external case is dative. The predicate \tit{antwurten} `to reply' takes dative objects.
The relative pronoun \tit{demo} `\ac{rel}.\ac{dat}.\ac{m}.\ac{sg}' appears in the external case: the dative. The relative pronoun is not marked in bold, just like as the main clause, showing that the relative pronoun patterns with the main clause.

\exg. Enti aer {ant uurta} demo \tbf{zaimo} \tbf{sprah}, quad:\\
and 3\tsc{sg.m.nom} reply.\tsc{3sg}.\tsc{past}\scsub{[dat]} \tsc{rel}.\ac{dat}.\tsc{m}.\tsc{sg} {to 3\tsc{sg.m.dat}} speak.\tsc{3sg}.\tsc{past}\scsub{[nom]} say.\tsc{past}.\tsc{3sg}\\
`And he replied to the one who spoke to him, he said:' \flushfill{\ac{ohg}, \ac{mons} 7:24, adapted from \pgcitealt{pittner1995}{199}}\label{ex:ohg-dat-nom}

Consider the example in \ref{ex:ohg-dat-acc}. In this example, the internal case is accusative and the external case is dative.
The internal case is nominative. The predicate \tit{zellen} `to report' takes accusative objects.
The external case is dative. The comparative of the adjective \tit{furiro} `great' takes dative objects.
The relative pronoun \tit{thên} `\ac{rel}.\ac{dat}.\ac{pl}' appears in the external case: the dative. The relative pronoun is not marked in bold, just like as the main clause, showing that the relative pronoun patterns with the main clause.

\exg. Bistú nu {zi wáre} furira Ábrahame? Ouh thén \tbf{man} \tbf{hiar} \tbf{nu} \tbf{zálta} joh sie álle tod bifálta?\\
{be.2\tsc{sg} \tsc{2sg.nom}} now really {great}.\tsc{com}\scsub{[dat]} Abraham.\tsc{dat} and \tsc{rel}.\ac{dat}.\tsc{m}.\tsc{pl} one.\tsc{m.sg.nom} here now report.\tsc{past}.\tsc{3sg}\scsub{[acc]}
and \tsc{3pl}.\tsc{acc} all.\tsc{3pl}.\tsc{acc} death.\tsc{nom}.\tsc{sg} attack.\tsc{past}.\tsc{3sg}\\
`Are you now really greater than Abraham? And than those who one reported here now and death attacked them all?' \flushfill{\ac{ohg}, \ac{otfrid} III 18:33}\label{ex:ohg-dat-acc}

Consider the example in \ref{ex:ohg-nom-acc}. In this example, the internal case is accusative and the external case is nominative.
The internal case is accusative. The predicates \tit{zellen} `to tell' and \tit{slahan} `to kill' takes accusative objects.
The external case is nominative. The predicate \tit{sin} `to be' takes nominative objects.
The relative pronoun \tit{then} `\ac{rel}.\ac{acc}.\ac{m}.\ac{sg}' appears in the internal case: the accusative. The relative pronoun is marked in bold, just like as the relative clause, showing that the relative pronoun patterns with the relative clause.

\exg. thíz ist \tbf{then} \tbf{sie} \tbf{zéllent} \tbf{joh} \tbf{then} \tbf{sie} \tbf{sláhan} \tbf{wollent}!\\
this.\tsc{nom} be.3\tsc{sg}\scsub{[nom]} \tsc{rel}.\tsc{acc}.\tsc{m}.\tsc{sg} \tsc{3pl}.\tsc{masc}.\tsc{nom} tell.\tsc{3pl}\scsub{[acc]}
and \tsc{rel}.\tsc{acc}.\tsc{m}.\tsc{sg} \tsc{3pl}.\tsc{masc}.\tsc{nom} kill\scsub{[acc]} want.\tsc{3pl}\\
`This is the one whom they talk about and whom they want to kill.' \flushfill{\ac{ohg}, \ac{otfrid} III 16:50}\label{ex:ohg-nom-acc}

Consider the example in \ref{ex:ohg-nom-dat}. In this example, the internal case is dative and the external case is nominative.
The internal case is dative. The predicate \tit{forlazan} `to read' takes dative indirect objects.
The external case is nominative. The predicate \tit{minnon} `to love' takes nominative subjects.
The relative pronoun \tit{themo} `\ac{rel}.\ac{dat}.\ac{m}.\ac{sg}' appears in the internal case: the dative. The relative pronoun is marked in bold, just like as the relative clause, showing that the relative pronoun patterns with the relative clause.

\exg. \tbf{themo} \tbf{min} \tbf{uuirdit} \tbf{forlazan}, min minnot\\
\tsc{rel}.\tsc{dat}.\tsc{m}.\tsc{sg} less become.\tsc{3sg} read\scsub{[dat]} less love.\tsc{3sg}\scsub{[nom]}\\
`To whom less is read, loves less.' \flushfill{\ac{ohg}, \ac{tatian} 138:13}\label{ex:ohg-nom-dat}

Consider the example in \ref{ex:ohg-acc-dat}. In this example, the internal case is dative and the external case is accusative.
The internal case is dative. The predicate \tit{gituon} `to do' takes dative indirect objects.
The external case is nominative. The predicate \tit{queman} `to come' takes nominative subjects.
The relative pronoun \tit{themo} `\ac{rel}.\ac{dat}.\ac{m}.\ac{sg}' appears in the internal case: the dative. The relative pronoun is marked in bold, just like as the relative clause, showing that the relative pronoun patterns with the relative clause.

% Previous sentence: \tit{Ther mán sih thaz gilérit,thia gilóuba in ínan kerit --giduat er húgu sinan in éwon filu blídan;}
% ``the man teaches himself that, he focusses on the belief in himself, he does his own belief forever with a lot of enthusiasm.''

\exg. \tbf{Themo} \tbf{avur} \tbf{tház} \tbf{ni} \tbf{gidúat}, quimit séragaz muat, joh wónot inan úbari gotes ábulgi!\\
\tsc{rel}.\tsc{dat}.\tsc{m}.\tsc{sg} but \tsc{dem}.\tsc{acc}.\tsc{n}.\tsc{sg} not do.3\tsc{sg}\scsub{[dat?]} come.3\tsc{sg}\scsub{[acc?]} sad.\tsc{nom}.\tsc{sg} heart.\tsc{nom}.\tsc{sg}
and live.3\tsc{sg} 3.\tsc{sg}.\tsc{m}.\tsc{acc} over god.\tsc{gen}.\tsc{sg} rage.\tsc{sg}\\
`But he does not do that to him, the sad heart comes him, and he lives in God's rage!' \flushfill{\ac{ohg}, \ac{otfrid} II 3:37}\label{ex:ohg-acc-dat}

There is a counterexample.
Consider the example in \ref{ex:ohg-counterexample}. In this example, the internal case is nominative and the external case is accusative. Surprisingly, the relative pronoun appears in the case lower on the case scale.
The internal case is nominative. The predicate \tit{giheilen} `to save' takes nominative subjects.
The external case is accusative. The predicate \tit{beran} `to bear' takes accusative objects.
The relative pronoun \tit{thér} `\ac{rel}.\ac{nom}.\ac{m}.\ac{sg}' appears in the internal case: the nominative. The relative pronoun is marked in bold, just like as the relative clause, showing that the relative pronoun patterns with the relative clause.

\exg. tház si uns béran scolti \tbf{thér} \tbf{unsih} \tbf{gihéilti}\\
 that 3\tsc{sg.f.nom} 1\tsc{pl.dat} bear\scsub{[acc]} should.\tsc{subj}.\tsc{past}.\tsc{3sg} \tsc{rel}.\tsc{nom}.\tsc{m}.\tsc{sg} 1\tsc{pl.acc} save.\tsc{subj.past}.\tsc{3sg}\scsub{[nom]}\\
 `that she should bear for us him who will save us' \flushfill{\ac{ohg}, \ac{otfrid} I 3:38}\label{ex:ohg-counterexample}

\phantom{x}


\section{Internal case wins}

There is really case competition: see appendix


In \ac{mg}, only internal case can win.

Internal is grammatical.

Consider the example in \ref{ex:mg-acc-dat}. In this example, the internal case is dative and the external case is accusative.
The internal case is dative. The predicate \tit{vertrauen} `to trust' takes dative objects.
The external case is accusative. The predicate \tit{einladen} `to invite' takes accusative objects.
The relative pronoun \tit{wem} `\tsc{rel}.\ac{dat}.\tsc{an}' appears in the internal case: the dative. The relative pronoun is marked in bold, just like as the relative clause, showing that the relative pronoun patterns with the relative clause.

\exg. Ich {lade ein} \tbf{wem} \tbf{auch} \tbf{Maria} \tbf{vertraut}. \\
I.\ac{nom} invite.1\tsc{sg}\scsub{[acc]} \tsc{rel}.\ac{dat}.\tsc{an} also Maria.\ac{nom} trust.3\tsc{sg}\scsub{[dat]}.\\
`I invite whoever Maria also trusts.' \flushfill{\ac{mg}, adapted from \pgcitealt{vogel2001}{344}}\label{ex:mg-acc-dat}

Consider the example in \ref{ex:mg-nom-dat}. In this example, the internal case is dative and the external case is nominative.
The internal case is dative. The predicate \tit{vertrauen} `to trust' takes dative objects.
The external case is nominative. The predicate \tit{besuchen} `to visit' takes nominative subjects.
The relative pronoun \tit{wem} `\tsc{rel}.\ac{dat}.\tsc{an}' appears in the internal case: the dative. The relative pronoun is marked in bold, just like as the relative clause, showing that the relative pronoun patterns with the relative clause.

\exg. Uns besucht \tbf{wem} \tbf{Maria} \tbf{vertraut}.\\
we.\ac{acc} visit.3\ac{sg}\scsub{[nom]} \tsc{rel}.\ac{dat}.\tsc{an} Maria.\ac{nom} trust.3\ac{sg}\scsub{[dat]}\\
`Who visits us, Maria trusts.' \flushfill{\ac{mg}, adapted from \pgcitealt{vogel2001}{343}}\label{ex:mg-nom-dat}

Consider the example in \ref{ex:mg-nom-acc}. In this example, the internal case is accusative and the external case is nominative.
The internal case is accusative. The predicate \tit{mögen} `to like' takes accusative objects.
The external case is nominative. The predicate \tit{besuchen} `to visit' takes nominative subjects.
The relative pronoun \tit{wen} `\tsc{rel}.\ac{acc}.\tsc{an}' appears in the internal case: the accusative. The relative pronoun is marked in bold, just like as the relative clause, showing that the relative pronoun patterns with the relative clause.

\exg. Uns besucht \tbf{wen} \tbf{Maria} \tbf{mag}.\\
 we.\ac{acc} visit.3\ac{sg}\scsub{[nom]} \tsc{rel}.\ac{acc}.\tsc{an} Maria.\ac{nom} like.3\ac{sg}\scsub{[acc]}\\
 `Who visits us, Maria likes.' \flushfill{\ac{mg}, adapted from \pgcitealt{vogel2001}{343}}\label{ex:mg-nom-acc}

External is ungrammatical.

Consider the example in \ref{ex:mg-dat-acc}. In this example, the internal case is accusative and the external case is dative.
The internal case is accusative. The predicate \tit{mögen} `to like' takes accusative objects.
The external case is dative. The predicate \tit{vertrauen} `to trust' takes dative objects.
The relative pronoun \tit{wem} `\tsc{rel}.\ac{dat}.\tsc{an}' appears in the external case: the dative. The relative pronoun is not marked in bold, just like as the main clause, showing that the relative pronoun patterns with the main clause.
This is ungrammatical in German: only the internal case can win.

\exg. *Ich vertraue wem \tbf{auch} \tbf{Maria} \tbf{mag}. \\
I.\ac{nom} trust.1\ac{sg}\scsub{[dat]} \tsc{rel}.\ac{dat}.\tsc{an} also Maria.\ac{nom} like.3\ac{sg}\scsub{[acc]}.\\
`I trust whoever Maria also likes.' \flushfill{\ac{mg}, adapted from \pgcitealt{vogel2001}{345}}\label{ex:mg-dat-acc}

Consider the example in \ref{ex:mg-dat-nom}. In this example, the internal case is nominative and the external case is dative.
The internal case is nominative. The predicate \tit{mögen} `to like' takes nominative subjects.
The external case is dative. The predicate \tit{vertrauen} `to trust' takes dative objects.
The relative pronoun \tit{wem} `\tsc{rel}.\ac{dat}.\tsc{an}' appears in the external case: the dative. The relative pronoun is not marked in bold, just like as the main clause, showing that the relative pronoun patterns with the main clause.
This is ungrammatical in German: only the internal case can win.

\exg. *Ich vertraue, wem \tbf{Hitchcock} \tbf{mag}.\\
I.\ac{nom} trust.1\ac{sg}\scsub{[dat]} \tsc{rel}.\ac{dat}.\tsc{an} Hitchcock.\ac{acc} like.3\ac{sg}\scsub{[nom]}\\
`I trust who likes Hitchcock.' \flushfill{\ac{mg}, adapted from \pgcitealt{vogel2001}{345}}\label{ex:mg-dat-nom}

Consider the example in \ref{ex:mg-acc-nom}. In this example, the internal case is nominative and the external case is accusative.
The internal case is nominative. The predicate \tit{sein} `to be' takes nominative subjects.
The external case is accusative. The predicate \tit{einladen} `to invite' takes accusative objects.
The relative pronoun \tit{wen} `\tsc{rel}.\ac{acc}.\tsc{an}' appears in the external case: the accusative. The relative pronoun is not marked in bold, just like as the main clause, showing that the relative pronoun patterns with the main clause.
This is ungrammatical in German: only the internal case can win.

\exg. *Ich {lade ein}, wen \tbf{mir} \tbf{sympathisch} \tbf{ist}.\\
I.\ac{nom} invite.1\ac{sg}\scsub{[acc]} \tsc{rel}.\ac{acc}.\tsc{an} I.\ac{dat} nice be.3\ac{sg}\scsub{[nom]}\\
`I invite who I like.' \flushfill{\ac{mg}, adapted from \pgcitealt{vogel2001}{344}}\label{ex:mg-acc-nom}

To summarize

\begin{table}[H]
  \center
  \caption{Summary Modern German headless relatives}
  \begin{tabular}{c|c|c|c}
    \toprule
   \textsubscript{\ac{int}} \textsuperscript{\ac{ext}}
          & [\ac{nom}]
          & [\ac{acc}]
          & [\ac{dat}]
          \\ \cmidrule{1-4}
      [\ac{nom}]
          &
          &
          &
          \\ \cmidrule{1-4}
      [\ac{acc}]
          & \ac{acc}
          &
          &
          \\ \cmidrule{1-4}
      [\ac{dat}]
          & \ac{dat}
          & \ac{dat}
          &
          \\
    \bottomrule
  \end{tabular}
\end{table}




\section{Languages without case attraction}

Old English, Old Icelandic both show: \tsc{d}-pronoun plus invariant relativizer + always take case from the main clause

Greek

Ancient Greek also actually has both directions (see \cite{vanriemsdijk2006}), so there is actually not a language with external only + hierarchy effects


% The nativeness of the headless relative constructions under
% consideration, and in particular those represented by group (I), can be
% established for the moment by citing instances in which they occur independently
% of the Greek, e.g., in which the Gothic modified relative clause
% reflects a Greek participial construction or a noun, or instances in which
% they occur in opposition to the Greek. Evidence of the first type is
% found in Mk 10:32, Luk 3:13, Rom 14:19, II Cor 8:11, Col 3:2, and elsewhere.

Polish and Italian have \tsc{wh}-pronouns but do not allow for conflicts


What does not exist: external only


\section{Patterns without case competition}

\subsection{No mismatch allowed}

pro-drop

\ex.
\ag. *Jan ufa kogokolkwiek wpu{\'s}cil do domu.\\
Jan trusts\scsub{dat} who.\tsc{acc} let\scsub{acc} to home\\
`Jan trusts whoever he let into the house.'
\bg. *Jan ufa komukolkwiek wpu{\'s}cil do domu.\\
Jan trusts\scsub{dat} who.\tsc{dat} let\scsub{acc} to home\\
`Jan trusts whoever he let into the house.'

\ex.
\ag. *Jan lubi kogokolkwiek dokucza.\\
Jan likes\scsub{acc} who.\tsc{acc} teases\scsub{dat}\\
`Jan likes whoever he teases.'
\bg. *Jan lubi komukolkwiek dokucza.\\
Jan likes\scsub{acc} who.\tsc{dat} teases\scsub{dat}\\
`Jan likes whoever he teases.'

very small table, only acc and dat

\subsection{External case surfaces}


\subsubsection{Greek}

Dasalaki: `no case competition'

free relative pronoun takes main clause case

first one without clitic

\ex.
\ag. Irϑan ópji káleses.\\
came.3\tsc{pl}\scsub{nom} who.\tsc{nom.pl} invited.2\tsc{sg}\scsub{acc}\\
`Whoever you invited came.' \hfill m:\tsc{nom}, e:\tsc{acc}, PL
\bg. *Irϑan ópjus káleses.\\
came.3\tsc{pl}\scsub{nom} who.\tsc{acc.pl} invited.2\tsc{sg}\scsub{acc}\\
`Whoever you invited came.' \hfill m:\tsc{nom}, e:\tsc{acc}

\ex.
\ag. *Me efχarístisan ópjon íχa ósi leftá.\\
\tsc{cl.1sg.acc} thanked.3\tsc{pl}\scsub{nom} who.\tsc{gen.pl} had.1\tsc{sg}\scsub{gen} given money\\
`Whoever I had given money to thanked me.' \hfill m:\tsc{nom}, e:\tsc{gen}
\bg. Me efχarístisan ópji tus íχa δósi leftá.\\
\tsc{cl.1sg.acc} thanked.3\tsc{pl}\scsub{nom} who.\tsc{nom.pl} \tsc{cl.3pl.gen} had.1\tsc{sg}\scsub{gen} given money\\
`Whoever I had given money to thanked me.' \hfill m:\tsc{nom}, e:\tsc{gen}

second and third with clitic

\ex.
\ag. *Γnórisa ópju éδosan tin ipotrofia.\\
met.1\tsc{sg}\scsub{acc} who.\tsc{gen.sg} gave.3\tsc{pl}\scsub{gen} the scholarship.\tsc{acc}\\
`I met whoever they gave the scholarship to.' \hfill m:\tsc{acc}, e:\tsc{gen}
\bg. Γnórisa ópjon tu éδosan tin ipotrofia.\\
met.1\tsc{sg}\scsub{acc} who.\tsc{acc.sg} \tsc{cl.3sg.gen} gave.3\tsc{pl}\scsub{gen} the scholarship.\tsc{acc}\\
`I met whoever they gave the scholarship to.' \hfill m:\tsc{acc}, e:\tsc{gen}



\subsubsection{Old English}

Harbert

old english: matrix clause decides, going against case competition

\tit{tōbrȳsan} `to pulversize' takes accusative objects, \tit{onuppan} `upon' takes dative objects.

\exg. he tobryst ðone \tbf{ðe} \tbf{he} \tbf{onuppan} \tbf{fylð}\\
 it pulverizes\scsub{[acc]}  \tsc{rel}.\ac{acc} \tsc{compl} it upon\scsub{[dat]} falls\\
`It pulverizes him whom it falls upon.' adapted from St.Mat. 1249 after xx

\subsection{No internal case surfaces}

o
%
% only internal case when it's dislocated
% and done de bu nu hafst, nis and pron-acc compl you now have not-is se bin wer he your husband 'And him who you now have, he is not your husband.' Alc.P.V. 37
% but then there is this additional element, so a different construction
%
