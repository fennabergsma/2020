% !TEX root = thesis.tex

\chapter{Two attested patterns}

In Part \ref{part:case-facts} of this dissertation, I discussed a first aspect of case competition in headless relatives. When there is case competition, reference is always made to the same case scale: there is a fixed order that determines which case wins. This is the same case scale crosslinguistically. I repeat the case scale from Part \ref{part:case-facts} in \ref{ex:case-scale-two-patterns}.

\ex. \ac{nom} < \ac{acc} < \ac{dat}\label{ex:case-scale-two-patterns}

Also in Part \ref{part:case-facts}, that a cumulative case decomposition can derive the case scale, not only in case competition in headless relatives, but also in syncretism patterns, morphological case containment. In other words, an accusative contains all features a nominative contains plus one more. Similarly, a dative contains all features an accusative contains plus one. Therefore, a dative can be considered more complex than an accusative, and a accusative more than a nominative. In line with that, I refer to cases more to the right on the case as more complex cases than cases more to the left on the scale.

This part, Part \ref{part:variation}, of the dissertation focuses on a second aspect to headless relatives. This part is not stable crosslinguistically, but it differs across languages. Languages differ in whether they allow the internal case or the external case to surface when they win the case competition. Logically speaking, there are three patterns possible for languages that show case competition.\footnote{
Theoretically, there a is fourth possibility: a language that has case competition in its headless relatives, but it lets neither the internal nor the external case surface when it wins. On the surface this language cannot be distinguished from a language that does not have case competition to begin with. In this section I do not discuss this type of language. I come back to it in Chapter \ref{ch:relativization}.
}

The first possible pattern is that a language allows for the internal or the external case to surface when it wins the competition. In this type of language, any case can be the internal and external case, and the more complex case of the two surfaces.
The second possible pattern is that a language allows the internal case to surface when it wins the case competition, but it does not allow the external case to. In this type of language, only the internal case gets to surface if it is more complex than the external one. If the external case is more complex, it does not surface, and it is not possible to form a grammatical headless relative construction.
The third possible pattern is that a language allows the external case to surface when it wins the case competition, but it does not allow the internal case to. In this type of language, only the external case gets to surface if it is more complex. If the internal case is more complex, it does not surface, and it is not possible to form a grammatical headless relative construction.

In this chapter I show that two of these three patterns are attested crosslinguistically. The first pattern, in which the internal or external case can surface, is the one already exemplified by Gothic. I show that there is another language that shows the same pattern: Old High German. The second pattern, in which only the internal case can surface, is exemplified by Modern German. There is no language in which only the external case can surface when it wins the case competition. A summary of the pattens is given in Table \ref{tbl:competition-summary}.

\begin{table}[H]
 \center
 \caption {Variation}
  \begin{tabular}{ccc}
  \toprule
                    & \tsc{int}>\tsc{ext}  & \tsc{ext}>\tsc{int} \\
                    \cmidrule{2-3}
  Gothic, \ac{ohg}  & ✔                  & ✔                 \\
  \ac{mg}           & ✔                  & *                 \\
  n.a.              & *                  & ✔                 \\
  \bottomrule
\end{tabular}\label{tbl:competition-summary}
\end{table}


\section{Internal or external wins}

The first type of language that is attested is the one that allows the internal case or the external case to surface when it wins the case competition. In this section I give a summary of the findings in Gothic \citep{harbert1978}, repeated from Chapter \ref{ch:recurring}. I also discuss data from Old High German, which is the result of my own research.

As discussed in Chapter \ref{ch:recurring}, Gothic is a language that shows case competition. The relative pronoun surfaces in the more complex case, following the case scale. That is, accusative wins over nominative, dative wins over nominative, and dative wins over accusative.
Gothic allows the internal case or the external case to surface when it wins the case competition. This is summarized in \ref{tbl:summary-gothic-repeated}. The left column shows the internal case between square brackets. The upper row shows the external case between square brackets. The other cells indicate the case of the relative pronoun.
The three cells in the lower left corner are the situations in which the internal case surfaces when it wins the competition. The three cells in the upper right corner are the situations in which the external case surfaces when it wins the competition.
The examples corresponding to the cells in the table can be found in Section \ref{sec:pattern-rels}.

\begin{table}[H]
  \center
  \caption{Summary Gothic headless relatives (repeated)}
    % !TEX root = ../thesis.tex

\begin{tabular}{c|c|c|c}
  \toprule
      \textsubscript{\ac{int}} \textsuperscript{\ac{ext}}
        & [\ac{nom}]
        & [\ac{acc}]
        & [\ac{dat}]
        \\ \cmidrule{1-4}
    [\ac{nom}]
        & \ac{nom}
        & \ac{acc}
        & \ac{dat}
        \\ \cmidrule{1-4}
    [\ac{acc}]
        & \ac{acc}
        & \ac{acc}
        & \ac{dat}
        \\ \cmidrule{1-4}
    [\ac{dat}]
        & \ac{dat}
        & (\ac{dat})
        & \ac{dat}
        \\
  \bottomrule
\end{tabular}

    \label{tbl:summary-gothic-repeated}
\end{table}

A language that shows the same pattern as Gothic is Old High German. The relative pronoun can surface in the internal or the external case, depending on which case is more complex. This conclusion follows from my own research of the texts Der althochdeutsche Isidor, The Monsee fragments, Otfrid's Evangelienbuch and Tatian in ANNIS \citep{krause2016}.\footnote{
Old High German is widely discussed in the literature because of its case attraction in headed relatives \citep[cf.][]{pittner1995}, a phenomenon that seems related to case competition in headless relatives. A common conclusion is that case attraction in headed relatives in Old High German adheres to the case scale. The same is claimed for headless relatives.
What, to the best of my knowledge, has not been systematically studied is whether Old High German headless relative allow the internal or the external case to surface when it wins the case competition. This is what I investigated in my work.
}
The examples follow the spelling and the detailed glosses in ANNIS. The translations are my own.

I start with the competition between accusative and nominative. Following the case scale, the relative pronoun appears in accusative case and never in nominative.

Consider the example in \ref{ex:ohg-acc-nom}. In this example, the internal case is nominative and the external case is accusative.
The internal case is nominative. The predicate \tit{gisizzen} `to possess' takes nominative subjects.
The external case is accusative. The predicate \tit{bibringan} `to give rise to' takes accusative objects.
The relative pronoun \tit{dhen} `\ac{rel}.\ac{acc}.\ac{m}.\ac{sg}' appears in the external case: the accusative. The relative pronoun is not marked in bold, just like as the main clause, showing that the relative pronoun patterns with the main clause.

% Context: the voice of God speaks through him (I think Jesaja):

\exg. ``Ih bibringu fona iacobes samin endi fona iuda dhen \tbf{mina} \tbf{berga} \tbf{chisitzit}.''\\
1\tsc{sg}.\tsc{nom} {give rise to/send}.\tsc{1sg}\scsub{[acc]} from Jakob.\tsc{gen} Samuel.\tsc{dat} and from Judas.\tsc{abl} \tsc{rel}.\tsc{acc}.\tsc{m}.\tsc{sg} my.\tsc{acc}.\tsc{m}.\tsc{pl} mountain.\tsc{acc}.\tsc{pl} possess.3\tsc{sg}\scsub{[nom]}\\
`I give rise to from Samuel of Jakob and from Judas him who possesses my mountains.' \flushfill{\ac{ohg}, \ac{isid} 34:3}\label{ex:ohg-acc-nom}\\
% Lat `Educam de Iacob semine et de Iuda possedentem montes meos.'

Consider the example in \ref{ex:ohg-nom-acc}. In this example, the internal case is accusative and the external case is nominative.
The internal case is accusative. The predicates \tit{zellen} `to tell' and \tit{slahan} `to kill' takes accusative objects.
The external case is nominative. The predicate \tit{sin} `to be' takes nominative objects.
The relative pronoun \tit{then} `\ac{rel}.\ac{acc}.\ac{m}.\ac{sg}' appears in the internal case: the accusative. The relative pronoun is marked in bold, just like as the relative clause, showing that the relative pronoun patterns with the relative clause.
Examples in which the internal case is accusative, the external case is nominative and the relative pronoun appears in nominative case are unattested.

hi

\exg. thíz ist \tbf{then} \tbf{sie} \tbf{zéllent} \tbf{joh} \tbf{then} \tbf{sie} \tbf{sláhan} \tbf{wollent}!\\
this.\tsc{nom} be.3\tsc{sg}\scsub{[nom]} \tsc{rel}.\tsc{acc}.\tsc{m}.\tsc{sg} \tsc{3pl}.\tsc{masc}.\tsc{nom} tell.\tsc{3pl}\scsub{[acc]}
and \tsc{rel}.\tsc{acc}.\tsc{m}.\tsc{sg} \tsc{3pl}.\tsc{masc}.\tsc{nom} kill\scsub{[acc]} want.\tsc{3pl}\\
`This is the one whom they talk about and whom they want to kill.' \flushfill{\ac{ohg}, \ac{otfrid} III 16:50}\label{ex:ohg-nom-acc}

I continue with the competition between dative and nominative. Following the case scale, the relative pronoun appears in dative case and never in nominative.

Consider the example in \ref{ex:ohg-dat-nom}. In this example, the internal case is nominative and the external case is dative.
The internal case is nominative. The predicate \tit{sprehhan} `to speak' takes nominative subjects.
The external case is dative. The predicate \tit{antwurten} `to reply' takes dative objects.
The relative pronoun \tit{demo} `\ac{rel}.\ac{dat}.\ac{m}.\ac{sg}' appears in the external case: the dative. The relative pronoun is not marked in bold, just like as the main clause, showing that the relative pronoun patterns with the main clause.
Examples in which the internal case is nominative, the external case is dative and the relative pronoun appears in nominative case are unattested.

\exg. Enti aer {ant uurta} demo \tbf{zaimo} \tbf{sprah}, quad:\\
and 3\tsc{sg.m.nom} reply.\tsc{3sg}.\tsc{past}\scsub{[dat]} \tsc{rel}.\ac{dat}.\tsc{m}.\tsc{sg} {to 3\tsc{sg.m.dat}} speak.\tsc{3sg}.\tsc{past}\scsub{[nom]} say.\tsc{past}.\tsc{3sg}\\
`And he replied to the one who spoke to him, he said:' \flushfill{\ac{ohg}, \ac{mons} 7:24, adapted from \pgcitealt{pittner1995}{199}}\label{ex:ohg-dat-nom}

Consider the example in \ref{ex:ohg-nom-dat}. In this example, the internal case is dative and the external case is nominative.
The internal case is dative. The predicate \tit{forlazan} `to read' takes dative indirect objects.
The external case is nominative. The predicate \tit{minnon} `to love' takes nominative subjects.
The relative pronoun \tit{themo} `\ac{rel}.\ac{dat}.\ac{m}.\ac{sg}' appears in the internal case: the dative. The relative pronoun is marked in bold, just like as the relative clause, showing that the relative pronoun patterns with the relative clause.
Examples in which the internal case is dative, the external case is nominative and the relative pronoun appears in nominative case are unattested.

\exg. \tbf{themo} \tbf{min} \tbf{uuirdit} \tbf{forlazan}, min minnot\\
\tsc{rel}.\tsc{dat}.\tsc{m}.\tsc{sg} less become.\tsc{3sg} read\scsub{[dat]} less love.\tsc{3sg}\scsub{[nom]}\\
`To whom less is read, loves less.' \flushfill{\ac{ohg}, \ac{tatian} 138:13}\label{ex:ohg-nom-dat}

I end with the competition between dative and accusative. Following the case scale, the relative pronoun appears in dative case and never in accusative.

Consider the example in \ref{ex:ohg-dat-acc}. In this example, the internal case is accusative and the external case is dative.
The internal case is nominative. The predicate \tit{zellen} `to report' takes accusative objects.
The external case is dative. The comparative of the adjective \tit{furiro} `great' takes dative objects.
The relative pronoun \tit{thên} `\ac{rel}.\ac{dat}.\ac{pl}' appears in the external case: the dative. The relative pronoun is not marked in bold, just like as the main clause, showing that the relative pronoun patterns with the main clause.
Examples in which the internal case is accusative, the external case is dative and the relative pronoun appears in accusative case are unattested.

\exg. Bistú nu {zi wáre} furira Ábrahame? Ouh thén \tbf{man} \tbf{hiar} \tbf{nu} \tbf{zálta} joh sie álle tod bifálta?\\
{be.2\tsc{sg} \tsc{2sg.nom}} now really {great}.\tsc{com}\scsub{[dat]} Abraham.\tsc{dat} and \tsc{rel}.\ac{dat}.\tsc{m}.\tsc{pl} one.\tsc{m.sg.nom} here now report.\tsc{past}.\tsc{3sg}\scsub{[acc]}
and \tsc{3pl}.\tsc{acc} all.\tsc{3pl}.\tsc{acc} death.\tsc{nom}.\tsc{sg} attack.\tsc{past}.\tsc{3sg}\\
`Are you now really greater than Abraham? And than those who one reported here now and death attacked them all?' \flushfill{\ac{ohg}, \ac{otfrid} III 18:33}\label{ex:ohg-dat-acc}

Consider the example in \ref{ex:ohg-acc-dat}. In this example, the internal case is dative and the external case is accusative.
The internal case is dative. The predicate \tit{gituon} `to do' takes dative indirect objects.
The external case is nominative. The predicate \tit{queman} `to come' takes nominative subjects.
The relative pronoun \tit{themo} `\ac{rel}.\ac{dat}.\ac{m}.\ac{sg}' appears in the internal case: the dative. The relative pronoun is marked in bold, just like as the relative clause, showing that the relative pronoun patterns with the relative clause.
Examples in which the internal case is dative, the external case is accusative and the relative pronoun appears in accusative case are unattested.

% Previous sentence: \tit{Ther mán sih thaz gilérit,thia gilóuba in ínan kerit --giduat er húgu sinan in éwon filu blídan;}
% ``the man teaches himself that, he focusses on the belief in himself, he does his own belief forever with a lot of enthusiasm.''

\exg. \tbf{Themo} \tbf{avur} \tbf{tház} \tbf{ni} \tbf{gidúat}, quimit séragaz muat, joh wónot inan úbari gotes ábulgi!\\
\tsc{rel}.\tsc{dat}.\tsc{m}.\tsc{sg} but \tsc{dem}.\tsc{acc}.\tsc{n}.\tsc{sg} not do.3\tsc{sg}\scsub{[dat?]} come.3\tsc{sg}\scsub{[acc?]} sad.\tsc{nom}.\tsc{sg} heart.\tsc{nom}.\tsc{sg}
and live.3\tsc{sg} 3.\tsc{sg}.\tsc{m}.\tsc{acc} over god.\tsc{gen}.\tsc{sg} rage.\tsc{sg}\\
`But he does not do that to him, the sad heart comes him, and he lives in God's rage!' \flushfill{\ac{ohg}, \ac{otfrid} II 3:37}\label{ex:ohg-acc-dat}

In my research I found a single counterexample to the generalizations made.
Consider the example in \ref{ex:ohg-counterexample}. In this example, the internal case is nominative and the external case is accusative. Surprisingly, the relative pronoun appears in the case lower on the case scale: the nominative.
The internal case is nominative. The predicate \tit{giheilen} `to save' takes nominative subjects.
The external case is accusative. The predicate \tit{beran} `to bear' takes accusative objects.
The relative pronoun \tit{thér} `\ac{rel}.\ac{nom}.\ac{m}.\ac{sg}' appears in the internal case: the nominative. The relative pronoun is marked in bold, just like as the relative clause, showing that the relative pronoun patterns with the relative clause.

\exg. tház si uns béran scolti \tbf{thér} \tbf{unsih} \tbf{gihéilti}\\
 that 3\tsc{sg.f.nom} 1\tsc{pl.dat} bear\scsub{[acc]} should.\tsc{subj}.\tsc{past}.\tsc{3sg} \tsc{rel}.\tsc{nom}.\tsc{m}.\tsc{sg} 1\tsc{pl.acc} save.\tsc{subj.past}.\tsc{3sg}\scsub{[nom]}\\
 `that she should bear for us him who will save us' \flushfill{\ac{ohg}, \ac{otfrid} I 3:38}\label{ex:ohg-counterexample}

%this is the only counterexample I have found. it could be there because of jezus.


To summarize \footnote{
I left the counterexample out of this overview.
}

\begin{table}[H]
  \center
  \caption{Summary Old High German headless relatives}
  \begin{tabular}{c|c|c|c}
    \toprule
        \textsubscript{\tsc{int}} \textsuperscript{\tsc{ext}}
          & [\ac{nom}]
          & [\ac{acc}]
          & [\ac{dat}]
          \\ \cmidrule{1-4}
      [\ac{nom}]
          &
          & \ac{acc}
          & \ac{dat}
          \\ \cmidrule{1-4}
      [\ac{acc}]
          & \ac{acc}
          &
          & \ac{dat}
          \\ \cmidrule{1-4}
      [\ac{dat}]
          & \ac{dat}
          & \ac{dat}
          &
          \\
    \bottomrule
  \end{tabular}
    \label{tbl:summary-old-high-german}
\end{table}

In sum, Gothic and Old High German both allow the internal or the external case to surface, as long as it is the most complex case.

\section{Only internal case wins}

Modern German differs from Gothic and Old High German. Just like the two other languages, Modern German adheres to the case scale: the more complex case wins the case competition. However, Modern German only allows the winner of the case competition to surface when it is the internal case. When the external case wins the competition, the result is ungrammatical.

The description of Modern German is mostly based on \citep{vogel2001}.

I start with the competition between accusative and nominative. Following the case scale, the relative pronoun appears in accusative case and never in nominative. Following the internal-only requirement, only if the accusative case is the internal case, the sentence is grammatical.

Consider the example in \ref{ex:mg-acc-nom}. In this example, the internal case is nominative and the external case is accusative.
The internal case is nominative. The predicate \tit{sein} `to be' takes nominative subjects.
The external case is accusative. The predicate \tit{einladen} `to invite' takes accusative objects.
The relative pronoun \tit{wen} `\tsc{rel}.\ac{acc}.\tsc{an}' appears in the external case: the accusative. The relative pronoun is not marked in bold, just like as the main clause, showing that the relative pronoun patterns with the main clause.
The example adheres to the case cale, but the more complex case (here the accusative) is not the internal case. As only the internal can win the case competition in Modern German, the example in ungrammatical.

\exg. *Ich {lade ein}, wen \tbf{mir} \tbf{sympathisch} \tbf{ist}.\\
I.\ac{nom} invite.1\ac{sg}\scsub{[acc]} \tsc{rel}.\ac{acc}.\tsc{an} I.\ac{dat} nice be.3\ac{sg}\scsub{[nom]}\\
`I invite who I like.' \flushfill{\ac{mg}, adapted from \pgcitealt{vogel2001}{344}}\label{ex:mg-acc-nom}

Changing the case of the relative pronoun to the nominative still gives an ungrammatical result, as illustrated in \ref{ex:mg-acc-nom-u}. This in addition violates the case scale: nominative is less a less complex case than accusative.

\exg. *Ich {lade ein}, wer \tbf{mir} \tbf{sympathisch} \tbf{ist}.\\
I.\ac{nom} invite.1\ac{sg}\scsub{[acc]} \tsc{rel}.\ac{nom}.\tsc{an} I.\ac{dat} nice be.3\ac{sg}\scsub{[nom]}\\
`I invite who I like.' \flushfill{\ac{mg}, adapted from \pgcitealt{vogel2001}{344}}\label{ex:mg-acc-nom-u}

Consider the example in \ref{ex:mg-nom-acc}. In this example, the internal case is accusative and the external case is nominative.
The internal case is accusative. The predicate \tit{mögen} `to like' takes accusative objects.
The external case is nominative. The predicate \tit{besuchen} `to visit' takes nominative subjects.
The relative pronoun \tit{wen} `\tsc{rel}.\ac{acc}.\tsc{an}' appears in the internal case: the accusative. The relative pronoun is marked in bold, just like as the relative clause, showing that the relative pronoun patterns with the relative clause.

\exg. Uns besucht \tbf{wen} \tbf{Maria} \tbf{mag}.\\
 we.\ac{acc} visit.3\ac{sg}\scsub{[nom]} \tsc{rel}.\ac{acc}.\tsc{an} Maria.\ac{nom} like.3\ac{sg}\scsub{[acc]}\\
 `Who visits us, Maria likes.' \flushfill{\ac{mg}, adapted from \pgcitealt{vogel2001}{343}}\label{ex:mg-nom-acc}

Changing the case of the relative pronoun to the nominative gives an ungrammatical result, as illustrated in \ref{ex:mg-nom-acc-u}. This violates the case scale: nominative is less a less complex case than accusative.

\exg. *Uns besucht \tbf{wer} \tbf{Maria} \tbf{mag}.\\
 we.\ac{acc} visit.3\ac{sg}\scsub{[nom]} \tsc{rel}.\ac{nom}.\tsc{an} Maria.\ac{nom} like.3\ac{sg}\scsub{[acc]}\\
 `Who visits us, Maria likes.' \flushfill{\ac{mg}, adapted from \pgcitealt{vogel2001}{343}}\label{ex:mg-nom-acc-u}

I continue with the competition between dative and nominative. Following the case scale, the relative pronoun appears in dative case and never in nominative. Following the internal-only requirement, only if the dative case is the internal case, the sentence is grammatical.

Consider the example in \ref{ex:mg-dat-nom}. In this example, the internal case is nominative and the external case is dative.
The internal case is nominative. The predicate \tit{mögen} `to like' takes nominative subjects.
The external case is dative. The predicate \tit{vertrauen} `to trust' takes dative objects.
The relative pronoun \tit{wem} `\tsc{rel}.\ac{dat}.\tsc{an}' appears in the external case: the dative. The relative pronoun is not marked in bold, just like as the main clause, showing that the relative pronoun patterns with the main clause.
The example adheres to the case cale, but the more complex case (here the dative) is not the internal case. As only the internal can win the case competition in Modern German, the example in ungrammatical.

\exg. *Ich vertraue, wem \tbf{Hitchcock} \tbf{mag}.\\
I.\ac{nom} trust.1\ac{sg}\scsub{[dat]} \tsc{rel}.\ac{dat}.\tsc{an} Hitchcock.\ac{acc} like.3\ac{sg}\scsub{[nom]}\\
`I trust who likes Hitchcock.' \flushfill{\ac{mg}, adapted from \pgcitealt{vogel2001}{345}}\label{ex:mg-dat-nom}

Changing the case of the relative pronoun to the nominative still gives an ungrammatical result, as illustrated in \ref{ex:mg-dat-nom-u}. This in addition violates the case scale: nominative is less a less complex case than dative.

\exg. *Ich vertraue, wer \tbf{Hitchcock} \tbf{mag}.\\
I.\ac{nom} trust.1\ac{sg}\scsub{[dat]} \tsc{rel}.\ac{nom}.\tsc{an} Hitchcock.\ac{acc} like.3\ac{sg}\scsub{[nom]}\\
`I trust who likes Hitchcock.' \flushfill{\ac{mg}, adapted from \pgcitealt{vogel2001}{345}}\label{ex:mg-dat-nom-u}

Consider the example in \ref{ex:mg-nom-dat}. In this example, the internal case is dative and the external case is nominative.
The internal case is dative. The predicate \tit{vertrauen} `to trust' takes dative objects.
The external case is nominative. The predicate \tit{besuchen} `to visit' takes nominative subjects.
The relative pronoun \tit{wem} `\tsc{rel}.\ac{dat}.\tsc{an}' appears in the internal case: the dative. The relative pronoun is marked in bold, just like as the relative clause, showing that the relative pronoun patterns with the relative clause.

\exg. Uns besucht \tbf{wem} \tbf{Maria} \tbf{vertraut}.\\
we.\ac{acc} visit.3\ac{sg}\scsub{[nom]} \tsc{rel}.\ac{dat}.\tsc{an} Maria.\ac{nom} trust.3\ac{sg}\scsub{[dat]}\\
`Who visits us, Maria trusts.' \flushfill{\ac{mg}, adapted from \pgcitealt{vogel2001}{343}}\label{ex:mg-nom-dat}

Changing the case of the relative pronoun to the nominative gives an ungrammatical result, as illustrated in \ref{ex:mg-nom-dat-u}. This violates the case scale: nominative is less a less complex case than dative.

\exg. *Uns besucht \tbf{wer} \tbf{Maria} \tbf{vertraut}.\\
we.\ac{acc} visit.3\ac{sg}\scsub{[nom]} \tsc{rel}.\ac{nom}.\tsc{an} Maria.\ac{nom} trust.3\ac{sg}\scsub{[dat]}\\
`Who visits us, Maria trusts.' \flushfill{\ac{mg}, adapted from \pgcitealt{vogel2001}{343}}\label{ex:mg-nom-dat-u}

I end with the competition between dative and accusative. Following the case scale, the relative pronoun appears in dative case and never in accusative. Following the internal-only requirement, only if the dative case is the internal case, the sentence is grammatical.

Consider the example in \ref{ex:mg-dat-acc}. In this example, the internal case is accusative and the external case is dative.
The internal case is accusative. The predicate \tit{mögen} `to like' takes accusative objects.
The external case is dative. The predicate \tit{vertrauen} `to trust' takes dative objects.
The relative pronoun \tit{wem} `\tsc{rel}.\ac{dat}.\tsc{an}' appears in the external case: the dative. The relative pronoun is not marked in bold, just like as the main clause, showing that the relative pronoun patterns with the main clause.
The example adheres to the case scale, but the more complex case (here the dative) is not the internal case. As only the internal can win the case competition in Modern German, the example in ungrammatical.

\exg. *Ich vertraue wem \tbf{auch} \tbf{Maria} \tbf{mag}. \\
I.\ac{nom} trust.1\ac{sg}\scsub{[dat]} \tsc{rel}.\ac{dat}.\tsc{an} also Maria.\ac{nom} like.3\ac{sg}\scsub{[acc]}.\\
`I trust whoever Maria also likes.' \flushfill{\ac{mg}, adapted from \pgcitealt{vogel2001}{345}}\label{ex:mg-dat-acc}

Changing the case of the relative pronoun to the accusative still gives an ungrammatical result, as illustrated in \ref{ex:mg-dat-acc-u}. This in addition violates the case scale: accusative is less a less complex case than dative.

\exg. *Ich vertraue wen \tbf{auch} \tbf{Maria} \tbf{mag}. \\
I.\ac{nom} trust.1\ac{sg}\scsub{[dat]} \tsc{rel}.\ac{acc}.\tsc{an} also Maria.\ac{nom} like.3\ac{sg}\scsub{[acc]}.\\
`I trust whoever Maria also likes.' \flushfill{\ac{mg}, adapted from \pgcitealt{vogel2001}{345}}\label{ex:mg-dat-acc-u}

Consider the example in \ref{ex:mg-acc-dat}. In this example, the internal case is dative and the external case is accusative.
The internal case is dative. The predicate \tit{vertrauen} `to trust' takes dative objects.
The external case is accusative. The predicate \tit{einladen} `to invite' takes accusative objects.
The relative pronoun \tit{wem} `\tsc{rel}.\ac{dat}.\tsc{an}' appears in the internal case: the dative. The relative pronoun is marked in bold, just like as the relative clause, showing that the relative pronoun patterns with the relative clause.

\exg. Ich {lade ein} \tbf{wem} \tbf{auch} \tbf{Maria} \tbf{vertraut}. \\
I.\ac{nom} invite.1\tsc{sg}\scsub{[acc]} \tsc{rel}.\ac{dat}.\tsc{an} also Maria.\ac{nom} trust.3\tsc{sg}\scsub{[dat]}.\\
`I invite whoever Maria also trusts.' \flushfill{\ac{mg}, adapted from \pgcitealt{vogel2001}{344}}\label{ex:mg-acc-dat}

Changing the case of the relative pronoun to the accusative gives an ungrammatical result, as illustrated in \ref{ex:mg-acc-dat-u}. This violates the case scale: accusative is less a less complex case than dative.

\exg. *Ich {lade ein} \tbf{wen} \tbf{auch} \tbf{Maria} \tbf{vertraut}. \\
I.\ac{nom} invite.1\tsc{sg}\scsub{[acc]} \tsc{rel}.\ac{acc}.\tsc{an} also Maria.\ac{nom} trust.3\tsc{sg}\scsub{[dat]}.\\
`I invite whoever Maria also trusts.' \flushfill{\ac{mg}, adapted from \pgcitealt{vogel2001}{344}}\label{ex:mg-acc-dat-u}


To summarize

\begin{table}[H]
  \center
  \caption{Summary Modern German headless relatives}
  \begin{tabular}{c|c|c|c}
    \toprule
   \textsubscript{\tsc{int}} \textsuperscript{\tsc{ext}}
          & [\ac{nom}]
          & [\ac{acc}]
          & [\ac{dat}]
          \\ \cmidrule{1-4}
      [\ac{nom}]
          &
          &
          &
          \\ \cmidrule{1-4}
      [\ac{acc}]
          & \ac{acc}/(\ac{nom})
          &
          &
          \\ \cmidrule{1-4}
      [\ac{dat}]
          & \ac{dat}
          & \ac{dat}
          &
          \\
    \bottomrule
  \end{tabular}
\end{table}

In sum, Modern German has two requirements for case competition in its headless relatives. First, the relative pronoun surfaces in the more complex case, following the case scale. Second, the case competition can only be won by the internal case, not the external case.

\section{No only external case wins}

Logically, a third possibility exists. However, so the best of my knowledge, this pattern is not attested in any (former) natural language.

It has been said about Old Greek (cf. in \citealt{cinque2005}, who stated the same about Gothic) (see a lot of examples here \citealt{kakarikos2014}), but here is an example of why that's not the case:. Although they seem to be more rare, there are also example in which the internal case surfaces.

Consider..
\tit{philéō} `to love' takes accusative, \tit{apothnḗiskō} `to die'

\exg. hòn hoi theoì philoũsin apothnḗͅskei néos\\
\tsc{rel}.\tsc{acc}.\tsc{m}.\tsc{sg} the god.\tsc{pl} love.3\tsc{pl} die.3\tsc{sg} young\\
`He whom the gods love dies young.' (Menander, The Double  Deceiver, fragment 125)

Two other languages come close, but these languages adhere to the case scale.

Dasalaki: `no case competition'

`meet' takes accusative case, `give' takes genitive case. Going with the case competition would be: let the relative pronoun appear in genitive case. This is ungrammatical. Instead, the relative pronoun appear in the external case, and the internal case gets a clitic.

\ex.
\ag. *Γnórisa \tbf{ópju} \tbf{éδosan} \tbf{tin} \tbf{ipotrofia}.\\
met.1\tsc{sg}\scsub{acc} \tsc{rel}.\tsc{gen}.\tsc{sg}.\tsc{m} gave.3\tsc{pl}\scsub{gen} the scholarship.\tsc{acc}\\
`I met whoever they gave the scholarship to.'
\bg. Γnórisa ópjon \tbf{tu} \tbf{éδosan} \tbf{tin} \tbf{ipotrofia}.\\
met.1\tsc{sg}\scsub{acc} who.\tsc{acc.sg} \tsc{cl.3sg.gen} gave.3\tsc{pl}\scsub{gen} the scholarship.\tsc{acc}\\
`I met whoever they gave the scholarship to.'

bladiebla \citep{harbert1983}

old english: matrix clause decides, going against case competition

\tit{tōbrȳsan} `to pulversize' takes accusative objects, \tit{onuppan} `upon' takes dative objects. But we see that the relative pronoun appears in accusative case

\exg. he tobryst ðone \tbf{ðe} \tbf{he} \tbf{onuppan} \tbf{fylð}\\
 it pulverizes\scsub{[acc]} \tsc{rel}.\ac{acc} \tsc{compl} it upon\scsub{[dat]} falls\\
`It pulverizes him whom it falls upon.' \flushfill{Old English, adapted from \pgcitealt{harbert1983}{550}}\label{ex:old-english}

 adapted from St.Mat. 1249 after xx  550

It is of course impossible to prove the non-existence of anything, so it could be an accident, but I am going to set up an system in which this does not exist.

%old english looks pretty similar to gothic here. Both have a complementizer, both have a relative pronoun preceding it. what I'll say later is that the old english complementizer also can also spell out case features and the gothic one cannot.

\section{Summary}

Case competition: two patterns. Gothic and Old High German allow both directions, Modern German only allows the internal case to surface. To the best of my knowledge, there are no languages in which only the external case can surface.
