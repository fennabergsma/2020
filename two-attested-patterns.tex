% !TEX root = thesis.tex

\chapter{Two attested patterns}

In Part \ref{part:case-facts} of this dissertation, I discussed a first aspect of case competition in headless relatives. There is a fixed scale that determines which case wins the case competition. This is the same case scale crosslinguistically. I repeat the case scale from Part \ref{ch:recurring} in \ref{ex:case-scale-two-patterns}.

\ex. \ac{nom} < \ac{acc} < \ac{dat}\label{ex:case-scale-two-patterns}

Also in Part \ref{part:case-facts}, I argued that a cumulative case decomposition can derive the case scale. This does not only hold for case competition in headless relatives, but also for syncretism patterns and morphological case containment patterns. In a cumulative case composition, the scale in \ref{ex:case-scale-two-patterns} can be interpreted as follows: an accusative contains all features a nominative contains plus one more. Similarly, a dative contains all features an accusative contains plus one. Therefore, a dative can be considered more complex than an accusative, and a accusative complex more than a nominative. In line with that, I refer to cases more to the right on the case as more complex cases than cases more to the left on the scale.

This part of the dissertation, Part \ref{part:variation}, focuses on a second aspect to headless relatives. This part is not stable crosslinguistically, but it differs across languages. Languages differ in whether they allow the internal case (the case from the relative clause) or the external case (the case from the main clause) to surface when they win the case competition. Logically speaking, there are three patterns possible for languages that show case competition.\footnote{
Theoretically, there a is fourth possibility: a language that has case competition in its headless relatives, but it lets neither the internal nor the external case surface when it wins. On the surface this language cannot be distinguished from a language that does not have case competition to begin with. In this section I do not discuss this type of language. I come back to it in Chapter \ref{ch:relativization}.
}
I show in this chapter that only two of the three patterns are attested across languages.

The next section introduces the patterns that are logically possible in languages with case competition. Section \ref{sec:pattern-i} to Section \ref{sec:pattern-iii} discuss the patterns one by one, and they give examples when the pattern is attested. In Section \ref{sec:potentiel-counterexamples} I make a sidestep to potential counterexamples, that turn out to be languages that lack case competition to begin with.


\section{Possible patterns with case competition}

There are two processes going on.

The first possible pattern is that a language allows for the internal or the external case to surface when it wins the competition. In this type of language, any case on the case scale can be the internal and external case, and the more complex case of the two surfaces.

\begin{table}[H]
  \center
  \caption{Case competition pattern I (internal or external)}
  \begin{tabular}{c|c|c|c}
    \toprule
        \textsubscript{\ac{int}} \textsuperscript{\ac{ext}}
          & [\ac{nom}]
          & [\ac{acc}]
          & [\ac{dat}]
          \\ \cmidrule{1-4}
      [\ac{nom}]
          &
          & \ac{acc}
          & \ac{dat}
          \\ \cmidrule{1-4}
      [\ac{acc}]
          & \ac{acc}
          &
          & \ac{dat}
          \\ \cmidrule{1-4}
      [\ac{dat}]
          & \ac{dat}
          & \ac{dat}
          &
          \\
    \bottomrule
  \end{tabular}
    \label{tbl:case-competition-pattern-i}
\end{table}

The second possible pattern is that a language allows the internal case to surface when it wins the case competition, but it does not allow the external case to. In this type of language, only the internal case gets to surface if it is more complex than the external one. If the external case is more complex, it does not surface, and it is not possible to form a grammatical headless relative construction.

\begin{table}[H]
  \center
  \caption{Case competition pattern II (internal only)}
  \begin{tabular}{c|c|c|c}
    \toprule
    \textsubscript{\ac{int}} \textsuperscript{\ac{ext}}
           & [\ac{nom}]
           & [\ac{acc}]
           & [\ac{dat}]
           \\ \cmidrule{1-4}
       [\ac{nom}]
           &
           & *
           & *
           \\ \cmidrule{1-4}
       [\ac{acc}]
           & \ac{acc}
           &
           & *
           \\ \cmidrule{1-4}
       [\ac{dat}]
           & \ac{dat}
           & \ac{dat}
           &
           \\
     \bottomrule
  \end{tabular}
    \label{tbl:case-competition-pattern-ii}
\end{table}

The third possible pattern is that a language allows the external case to surface when it wins the case competition, but it does not allow the internal case to. In this type of language, only the external case gets to surface if it is more complex. If the internal case is more complex, it does not surface, and it is not possible to form a grammatical headless relative construction.

\begin{table}[H]
  \center
  \caption{Case competition pattern III (external only)}
  \begin{tabular}{c|c|c|c}
    \toprule
    \textsubscript{\ac{int}} \textsuperscript{\ac{ext}}
           & [\ac{nom}]
           & [\ac{acc}]
           & [\ac{dat}]
           \\ \cmidrule{1-4}
       [\ac{nom}]
           &
           & \ac{acc}
           & \ac{dat}
           \\ \cmidrule{1-4}
       [\ac{acc}]
           & *
           &
           & \ac{dat}
           \\ \cmidrule{1-4}
       [\ac{dat}]
           & *
           & *
           &
           \\
     \bottomrule
  \end{tabular}
    \label{tbl:case-competition-pattern-iii}
\end{table}

In this chapter I show that two of these three patterns are attested crosslinguistically. The first pattern, in which the internal or external case can surface, is the one already exemplified by Gothic in Part \ref{part:case-facts}. I show that there is another language that shows the same pattern: Old High German. The second pattern, in which only the internal case can surface, is exemplified by Modern German. To my knowledge, there is no language in which only the external case can surface when it wins the case competition. A summary of the pattens is given in Table \ref{tbl:competition-summary}.

\begin{table}[H]
 \center
 \caption {Variation}
  \begin{tabular}{ccc}
  \toprule
                    & \ac{int}>\ac{ext}  & \ac{ext}>\ac{int} \\
                    \cmidrule{2-3}
  Gothic, \ac{ohg}  & ✔                  & ✔                 \\
  \ac{mg}           & ✔                  & *                 \\
  n.a.              & *                  & ✔                 \\
  \bottomrule
\end{tabular}\label{tbl:competition-summary}
\end{table}

\section{Pattern I: internal or external}\label{sec:pattern-i}

The first type of language that is attested is the one that allows the internal case or the external case to surface when it wins the case competition. Two languages that show this pattern are Gothic and Old High German. In this section I first give a summary of the findings in Gothic \citep{harbert1978}, repeated from Chapter \ref{ch:recurring}. Second, I discuss data from Old High German, which is the result of my own research.

As discussed in Chapter \ref{ch:recurring}, Gothic is a language that shows case competition. The relative pronoun surfaces in the more complex case, following the case scale. That is, accusative wins over nominative, dative wins over nominative, and dative wins over accusative.
Gothic allows the internal case or the external case to surface when it wins the case competition. This is summarized in \ref{tbl:summary-gothic-repeated}. The left column shows the internal case between square brackets. The upper row shows the external case between square brackets. The other cells indicate the case of the relative pronoun. The diagonal is left blank, because these are instances in which the internal and external case match, and there is no case competition taking place.
The three cells in the lower left corner are the situations in which the internal case surfaces when it wins the competition. The three cells in the upper right corner are the situations in which the external case surfaces when it wins the competition.
The examples corresponding to the cells in the table can be found in Section \ref{sec:pattern-rels}.

\begin{table}[H]
  \center
  \caption{Summary Gothic headless relatives (repeated)}
    % !TEX root = ../thesis.tex

\begin{tabular}{c|c|c|c}
  \toprule
      \textsubscript{\ac{int}} \textsuperscript{\ac{ext}}
        & [\ac{nom}]
        & [\ac{acc}]
        & [\ac{dat}]
        \\ \cmidrule{1-4}
    [\ac{nom}]
        & \ac{nom}
        & \ac{acc}
        & \ac{dat}
        \\ \cmidrule{1-4}
    [\ac{acc}]
        & \ac{acc}
        & \ac{acc}
        & \ac{dat}
        \\ \cmidrule{1-4}
    [\ac{dat}]
        & \ac{dat}
        & (\ac{dat})
        & \ac{dat}
        \\
  \bottomrule
\end{tabular}

    \label{tbl:summary-gothic-repeated}
\end{table}

A language that shows the same pattern as Gothic is Old High German. The relative pronoun can surface in the internal or the external case, depending on which case is more complex. This conclusion follows from my own research of the texts Der althochdeutsche Isidor, The Monsee fragments, Otfrid's Evangelienbuch and Tatian in ANNIS \citep{krause2016}.\footnote{
Old High German is widely discussed in the literature because of its case attraction in headed relatives \citep[cf.][]{pittner1995}, a phenomenon that seems related to case competition in headless relatives (see Section \ref{sec:attraction} for why attraction is not further discussed in this dissertation).
A common observation is that case attraction in headed relatives in Old High German adheres to the case scale. The same is claimed for headless relatives.
What, to my knowledge, has not been systematically studied is whether Old High German headless relative allow the internal or the external case to surface when it wins the case competition. This is what I investigated in my work.
}
The examples follow the spelling and the detailed glosses in ANNIS. The translations are my own.

I start with the competition between accusative and nominative. Following the case scale, the relative pronoun appears in accusative case and never in nominative.

Consider the example in \ref{ex:ohg-acc-nom}. In this example, the internal nominative case competes against the external accusative case.
The internal case is nominative, as the predicate \tit{gisizzen} `to possess' takes nominative subjects.
The external case is accusative, as the predicate \tit{bibringan} `to create' takes accusative objects.
The relative pronoun \tit{dhen} `\ac{rel}.\ac{acc}.\ac{m}.\ac{sg}' appears in the external case: the accusative. The relative pronoun is not marked in bold, just like as the main clause, showing that the relative pronoun patterns with the main clause.\footnote{
At the end of this section I discuss a counterexample to the case scale, in which the internal case is nominative, the external case is accusative, and the relative pronoun appears in the nominative case.
}


\exg. ih bibringu fona iacobes samin endi fona iuda dhen \tbf{mina} \tbf{berga} \tbf{chisitzit}\\
1\ac{sg}.\ac{nom} {educate}.1\ac{sg}\scsub{[acc]} of Jakob.\ac{gen} seed.\ac{dat} and of Judah.\ac{dat} \ac{rel}.\ac{acc}.\ac{m}.\ac{sg} my.\ac{acc}.\ac{m}.\ac{pl} mountain.\ac{acc}.\ac{pl} possess.3\ac{sg}\scsub{[nom]}\\
`I create of the seed of Jacob and of Judah the one, who possess my mountains' \flushfill{\ac{ohg}, \ac{isid} 34:3}\label{ex:ohg-acc-nom}

Consider the example in \ref{ex:ohg-nom-acc}. In this example, the internal accusative case competes against the external nominative case.
The internal case is accusative, as the predicate \tit{zellen} `to tell' takes accusative objects.
The external case is nominative, as the predicate \tit{sin} `to be' takes nominative objects.
The relative pronoun \tit{then} `\ac{rel}.\ac{acc}.\ac{m}.\ac{sg}' appears in the internal case: the accusative. The relative pronoun is marked in bold, just like as the relative clause, showing that the relative pronoun patterns with the relative clause.
Examples in which the internal case is accusative, the external case is nominative and the relative pronoun appears in nominative case are unattested.

\exg. thíz ist \tbf{then} \tbf{sie} \tbf{zéllent}\\
this.\ac{nom} be.3\ac{sg}\scsub{[nom]} \ac{rel}.\ac{acc}.\ac{m}.\ac{sg} 3\ac{pl}.\ac{m}.\ac{nom} tell.3\ac{pl}\scsub{[acc]}\\
`this is the one whom they talk about' \flushfill{\ac{ohg}, \ac{otfrid} III 16:50}\label{ex:ohg-nom-acc}

I continue with the competition between dative and nominative. Following the case scale, the relative pronoun appears in dative case and never in nominative.

Consider the example in \ref{ex:ohg-dat-nom}. In this example, the internal nominative case competes against the external dative case.
The internal case is nominative, as the predicate \tit{sprehhan} `to speak' takes nominative subjects.
The external case is dative, as the predicate \tit{antwurten} `to reply' takes dative objects.
The relative pronoun \tit{demo} `\ac{rel}.\ac{dat}.\ac{m}.\ac{sg}' appears in the external case: the dative. The relative pronoun is not marked in bold, just like as the main clause, showing that the relative pronoun patterns with the main clause.
Examples in which the internal case is nominative, the external case is dative and the relative pronoun appears in nominative case are unattested.

\exg. enti aer {ant uurta} demo \tbf{zaimo} \tbf{sprah}\\
and 3\ac{sg}.\ac{m}.\ac{nom} reply.3\ac{sg}.\ac{pst}\scsub{[dat]} \ac{rel}.\ac{dat}.\ac{m}.\ac{sg} {to 3\ac{sg}.\ac{m}.\ac{dat}} speak.3\ac{sg}.\ac{pst}\scsub{[nom]}\\
`and he replied to the one who spoke to him' \flushfill{\ac{ohg}, \ac{mons} 7:24, adapted from \pgcitealt{pittner1995}{199}}\label{ex:ohg-dat-nom}

Consider the example in \ref{ex:ohg-nom-dat}. In this example, the internal dative case competes against the external nominative case.
The internal case is dative, as the predicate \tit{forlazan} `to read' takes dative indirect objects.
The external case is nominative, as the predicate \tit{minnon} `to love' takes nominative subjects.
The relative pronoun \tit{themo} `\ac{rel}.\ac{dat}.\ac{m}.\ac{sg}' appears in the internal case: the dative. The relative pronoun is marked in bold, just like as the relative clause, showing that the relative pronoun patterns with the relative clause.
Examples in which the internal case is dative, the external case is nominative and the relative pronoun appears in nominative case are unattested.

\exg. \tbf{themo} \tbf{min} \tbf{uuirdit} \tbf{forlazan}, min minnot\\
\ac{rel}.\ac{dat}.\ac{m}.\ac{sg} less become.3\ac{sg} read\scsub{[dat]} less love.3\ac{sg}\scsub{[nom]}\\
`to whom less is read, loves less' \flushfill{\ac{ohg}, \ac{tatian} 138:13}\label{ex:ohg-nom-dat}

I end with the competition between dative and accusative. Following the case scale, the relative pronoun appears in dative case and never in accusative.

Consider the example in \ref{ex:ohg-dat-acc}. In this example, the internal accusative case competes against the external dative case.
The internal case is nominative, as the predicate \tit{zellen} `to tell' takes accusative objects.
The external case is dative, as the comparative of the adjective \tit{furiro} `great' takes dative objects.
The relative pronoun \tit{thên} `\ac{rel}.\ac{dat}.\ac{pl}' appears in the external case: the dative. The relative pronoun is not marked in bold, just like as the main clause, showing that the relative pronoun patterns with the main clause.
Examples in which the internal case is accusative, the external case is dative and the relative pronoun appears in accusative case are unattested.

\exg. bist -ú nu {zi wáre} furira Ábrahame? ouh thén \tbf{man} \tbf{hiar} \tbf{nu} \tbf{zálta}\\
be.2\ac{sg} -2\ac{sg}.\ac{nom} now really {great}.\ac{cmpr}\scsub{[dat]} Abraham.\ac{dat} and \ac{rel}.\ac{dat}.\ac{m}.\ac{pl} one.\ac{nom}.\ac{m}.\ac{sg} here now tell.\ac{pst}.3\ac{sg}\scsub{[acc]}\\
`are you now really greater than Abraham? and than those, who one talked about here now' \flushfill{\ac{ohg}, \ac{otfrid} III 18:33}\label{ex:ohg-dat-acc}

Consider the example in \ref{ex:ohg-acc-dat}. In this example, the internal dative case competes against the external accusative case.
The internal case is dative, as the predicate \tit{gituon} `to do' takes dative indirect objects.
The external case is nominative, as the predicate \tit{queman} `to come' takes nominative subjects.
The relative pronoun \tit{themo} `\ac{rel}.\ac{dat}.\ac{m}.\ac{sg}' appears in the internal case: the dative. The relative pronoun is marked in bold, just like as the relative clause, showing that the relative pronoun patterns with the relative clause.
Examples in which the internal case is dative, the external case is accusative and the relative pronoun appears in accusative case are unattested.

\exg. \tbf{themo} \tbf{avur} \tbf{tház} \tbf{ni} \tbf{gidúat}, quimit séragaz muat\\
\ac{rel}.\ac{dat}.\ac{m}.\ac{sg} but \ac{dem}.\ac{acc}.\ac{n}.\ac{sg} not do.3\ac{sg}\scsub{[dat]} approach.3\ac{sg}\scsub{[acc]} sad.\ac{nom}.\ac{sg} heart.\ac{nom}.\ac{sg}\\
`the sad heart approaches the one, whom he does not do that to' \flushfill{\ac{ohg}, \ac{otfrid} II 3:37}\label{ex:ohg-acc-dat}

A summary of the data is given in \ref{tbl:summary-old-high-german}.
The three cells in the lower left corner are the situations in which the internal case surfaces when it wins the competition. They correspond to the examples \ref{ex:ohg-nom-acc}, \ref{ex:ohg-nom-dat} and \ref{ex:ohg-acc-dat}.
The three cells in the upper right corner are the situations in which the external case surfaces when it wins the competition. They correspond to the examples \ref{ex:ohg-acc-nom}, \ref{ex:ohg-dat-nom} and \ref{ex:ohg-dat-acc}.

\begin{table}[H]
  \center
  \caption{Summary Old High German headless relatives}
  \begin{tabular}{c|c|c|c}
    \toprule
        \textsubscript{\ac{int}} \textsuperscript{\ac{ext}}
          & [\ac{nom}]
          & [\ac{acc}]
          & [\ac{dat}]
          \\ \cmidrule{1-4}
      [\ac{nom}]
          &
          & \ac{acc}
          & \ac{dat}
          \\ \cmidrule{1-4}
      [\ac{acc}]
          & \ac{acc}
          &
          & \ac{dat}
          \\ \cmidrule{1-4}
      [\ac{dat}]
          & \ac{dat}
          & \ac{dat}
          &
          \\
    \bottomrule
  \end{tabular}
    \label{tbl:summary-old-high-german}
\end{table}

In my research I encountered a single counterexample to the pattern I just described.
Consider the example in \ref{ex:ohg-counterexample}. In this example, the internal nominative case competes against the external accusative case.
The internal case is nominative, as the predicate \tit{giheilen} `to save' takes nominative subjects.
The external case is accusative, as the predicate \tit{beran} `to bear' takes accusative objects.
Surprisingly, the relative pronoun \tit{thér} `\ac{rel}.\ac{nom}.\ac{m}.\ac{sg}' appears in the internal case: the nominative, which is the less complex of the two cases. The relative pronoun is marked in bold, just like as the relative clause, showing that the relative pronoun patterns with the relative clause.

\exg. tház si uns béran scolti \tbf{thér} \tbf{unsih} \tbf{gihéilti}\\
 that 3\ac{sg}.\ac{f}.\ac{nom} 1\ac{pl}.\ac{dat} bear\scsub{[acc]} should.\ac{subj}.\ac{pst}.3\ac{sg} \ac{rel}.\ac{nom}.\ac{m}.\ac{sg} 1\ac{pl}.\ac{acc} save.\ac{sbjv}.\ac{pst}.3\ac{sg}\scsub{[nom]}\\
 `that she should have beared for us the one, who had saved us' \flushfill{\ac{ohg}, \ac{otfrid} I 3:38}\label{ex:ohg-counterexample}

This example is unexpected, because the less complex case (the nominative) wins and not the more complex case (the accusative).
The only explanation for this I can see is a functional one. The \tit{thér} `\ac{rel}.\ac{nom}.\ac{m}.\ac{sg}' in \ref{ex:ohg-counterexample} refers to Jesus. In the relative clause he is the subject of \tit{unsih gihéilti} `will save us', hence the internal nominative case. In the main clause he is the object of \tit{tház si uns béran scolti} `that she should bear', hence the external accusative case.
Letting the relative pronoun surface in the internal case could be interpreted as emphasizing the role of Jesus as a savior, rather than him being the object of being given birth to. In line with that reasoning, it is expected that certain grammatical facts more often deviate from regular patterns if Jesus is involved. I leave investigating this prediction for future research.
Of course, this does not answer the question of what happens to the accusative case required by the external predicate. It also does not explain why not another emphasizing strategy is used, for instance forming a light-headed relative, which would leave space for two cases.
I acknowledge this example as a counterexample to the pattern I describe, but I do not change my generalization, as this is a single occurrence.

Leaving the counterexample aside, I conclude that Gothic and Old High German are both instances languages that allow the internal or the external case to surface in case competition. The deciding factor for which case surfaces is the complexity of the case: the most complex case wins and surfaces.


\section{Pattern II: only internal}\label{sec:pattern-ii}

The second type of language that is attested is the one that only allows the internal case to surface when it wins the case competition. An example of a language that shows this pattern is Modern German. In this section I discuss the Modern German data, based on the research of \citet{vogel2001}.

Modern German behaves just like Gothic and Old High German in that it adheres to the case scale: the more complex case wins the case competition. Modern German differs from the two other languages in that it only allows the winner of the case competition to surface when it is the internal case. When the external case wins the competition, the result is ungrammatical.

I start with the competition between accusative and nominative. Following the case scale, the relative pronoun appears in accusative case and never in nominative. Following the internal-only requirement, only if the accusative case is the internal case, the sentence is grammatical.

Consider the example in \ref{ex:mg-acc-nom}. In this example, the internal nominative case competes against the external accusative case.
The internal case is nominative, as the predicate \tit{sein} `to be' takes nominative subjects.
The external case is accusative, as the predicate \tit{einladen} `to invite' takes accusative objects.
The relative pronoun \tit{wen} `\ac{rel}.\ac{acc}.\ac{an}' appears in the external case: the accusative. The relative pronoun is not marked in bold, just like as the main clause, showing that the relative pronoun patterns with the main clause.
The example adheres to the case scale, but the more complex case (here the accusative) is not the internal case. As only the internal can win the case competition in Modern German, the example in ungrammatical.

\exg. *Ich {lade ein}, wen \tbf{mir} \tbf{sympathisch} \tbf{ist}.\\
I.\ac{nom} invite.1\ac{sg}\scsub{[acc]} \ac{rel}.\ac{acc}.\ac{an} I.\ac{dat} nice be.3\ac{sg}\scsub{[nom]}\\
`I invite who I like.' \flushfill{\ac{mg}, adapted from \pgcitealt{vogel2001}{344}}\label{ex:mg-acc-nom}

This example in \ref{ex:mg-acc-nom-u} is identical to \ref{ex:mg-acc-nom}, except for that the relative pronoun appears in the external less complex nominative case. This example is also ungrammatical: in addition to the more complex case not being the internal case, the relative pronoun also does not appear in the more complex case (the accusative) but in the less complex case (the nominative).

\exg. *Ich {lade ein}, wer \tbf{mir} \tbf{sympathisch} \tbf{ist}.\\
I.\ac{nom} invite.1\ac{sg}\scsub{[acc]} \ac{rel}.\ac{nom}.\ac{an} I.\ac{dat} nice be.3\ac{sg}\scsub{[nom]}\\
`I invite who I like.' \flushfill{\ac{mg}, adapted from \pgcitealt{vogel2001}{344}}\label{ex:mg-acc-nom-u}

Consider the example in \ref{ex:mg-nom-acc}. In this example, the internal accusative case competes against the external nominative case.
The internal case is accusative, as the predicate \tit{mögen} `to like' takes accusative objects.
The external case is nominative, as the predicate \tit{besuchen} `to visit' takes nominative subjects.
The relative pronoun \tit{wen} `\ac{rel}.\ac{acc}.\ac{an}' appears in the internal case: the accusative. The relative pronoun is marked in bold, just like as the relative clause, showing that the relative pronoun patterns with the relative clause.
The example adheres to the case scale, and the more complex case (here the accusative) is the internal case, so the example is grammatical.

\exg. Uns besucht \tbf{wen} \tbf{Maria} \tbf{mag}.\\
 we.\ac{acc} visit.3\ac{sg}\scsub{[nom]} \ac{rel}.\ac{acc}.\ac{an} Maria.\ac{nom} like.3\ac{sg}\scsub{[acc]}\\
 `Who visits us, Maria likes.' \flushfill{\ac{mg}, adapted from \pgcitealt{vogel2001}{343}}\label{ex:mg-nom-acc}

This example in \ref{ex:mg-nom-acc-u} is identical to \ref{ex:mg-nom-acc}, except for that the relative pronoun appears in the external less complex nominative case. This example is ungrammatical: although the internal case is more complex, the relative pronoun appears in the less complex case (the nominative) and not in the more complex case (the accusative).

\exg. *Uns besucht \tbf{wer} \tbf{Maria} \tbf{mag}.\\
 we.\ac{acc} visit.3\ac{sg}\scsub{[nom]} \ac{rel}.\ac{nom}.\ac{an} Maria.\ac{nom} like.3\ac{sg}\scsub{[acc]}\\
 `Who visits us, Maria likes.' \flushfill{\ac{mg}, adapted from \pgcitealt{vogel2001}{343}}\label{ex:mg-nom-acc-u}

I continue with the competition between dative and nominative. Following the case scale, the relative pronoun appears in dative case and never in nominative. Following the internal-only requirement, only if the dative case is the internal case, the sentence is grammatical.

Consider the example in \ref{ex:mg-dat-nom}. In this example, the internal nominative case competes against the external dative case.
The internal case is nominative, as the predicate \tit{mögen} `to like' takes nominative subjects.
The external case is dative, as the predicate \tit{vertrauen} `to trust' takes dative objects.
The relative pronoun \tit{wem} `\ac{rel}.\ac{dat}.\ac{an}' appears in the external case: the dative. The relative pronoun is not marked in bold, just like as the main clause, showing that the relative pronoun patterns with the main clause.
The example adheres to the case cale, but the more complex case (here the dative) is not the internal case. As only the internal can win the case competition in Modern German, the example in ungrammatical.

\exg. *Ich vertraue, wem \tbf{Hitchcock} \tbf{mag}.\\
I.\ac{nom} trust.1\ac{sg}\scsub{[dat]} \ac{rel}.\ac{dat}.\ac{an} Hitchcock.\ac{acc} like.3\ac{sg}\scsub{[nom]}\\
`I trust who likes Hitchcock.' \flushfill{\ac{mg}, adapted from \pgcitealt{vogel2001}{345}}\label{ex:mg-dat-nom}

The example in \ref{ex:mg-dat-nom-u} is identical to \ref{ex:mg-dat-nom}, except for that the relative pronoun appears in the external less complex nominative case. This example is also ungrammatical: in addition to the more complex case not being the internal case, the relative pronoun also does not appear in the more complex case (the dative) but in the less complex case (the nominative).

\exg. *Ich vertraue, wer \tbf{Hitchcock} \tbf{mag}.\\
I.\ac{nom} trust.1\ac{sg}\scsub{[dat]} \ac{rel}.\ac{nom}.\ac{an} Hitchcock.\ac{acc} like.3\ac{sg}\scsub{[nom]}\\
`I trust who likes Hitchcock.' \flushfill{\ac{mg}, adapted from \pgcitealt{vogel2001}{345}}\label{ex:mg-dat-nom-u}

Consider the example in \ref{ex:mg-nom-dat}. In this example, the internal dative case competes against the external nominative case.
The internal case is dative, as the predicate \tit{vertrauen} `to trust' takes dative objects.
The external case is nominative, as the predicate \tit{besuchen} `to visit' takes nominative subjects.
The relative pronoun \tit{wem} `\ac{rel}.\ac{dat}.\ac{an}' appears in the internal case: the dative. The relative pronoun is marked in bold, just like as the relative clause, showing that the relative pronoun patterns with the relative clause.
The example adheres to the case scale, and the more complex case (here the dative) is the internal case, so the example is grammatical.

\exg. Uns besucht \tbf{wem} \tbf{Maria} \tbf{vertraut}.\\
we.\ac{acc} visit.3\ac{sg}\scsub{[nom]} \ac{rel}.\ac{dat}.\ac{an} Maria.\ac{nom} trust.3\ac{sg}\scsub{[dat]}\\
`Who visits us, Maria trusts.' \flushfill{\ac{mg}, adapted from \pgcitealt{vogel2001}{343}}\label{ex:mg-nom-dat}

This example in \ref{ex:mg-nom-dat-u} is identical to \ref{ex:mg-nom-dat}, except for that the relative pronoun appears in the external less complex nominative case. This example is ungrammatical: although the internal case is more complex, the relative pronoun appears in the less complex case (the nominative) and not in the more complex case (the dative).

\exg. *Uns besucht \tbf{wer} \tbf{Maria} \tbf{vertraut}.\\
we.\ac{acc} visit.3\ac{sg}\scsub{[nom]} \ac{rel}.\ac{nom}.\ac{an} Maria.\ac{nom} trust.3\ac{sg}\scsub{[dat]}\\
`Who visits us, Maria trusts.' \flushfill{\ac{mg}, adapted from \pgcitealt{vogel2001}{343}}\label{ex:mg-nom-dat-u}

I end with the competition between dative and accusative. Following the case scale, the relative pronoun appears in dative case and never in accusative. Following the internal-only requirement, only if the dative case is the internal case, the sentence is grammatical.

Consider the example in \ref{ex:mg-dat-acc}. In this example, the internal accusative case competes against the external dative case.
The internal case is accusative, as the predicate \tit{mögen} `to like' takes accusative objects.
The external case is dative, as the predicate \tit{vertrauen} `to trust' takes dative objects.
The relative pronoun \tit{wem} `\ac{rel}.\ac{dat}.\ac{an}' appears in the external case: the dative. The relative pronoun is not marked in bold, just like as the main clause, showing that the relative pronoun patterns with the main clause.
The example adheres to the case scale, but the more complex case (here the dative) is not the internal case. As only the internal can win the case competition in Modern German, the example in ungrammatical.

\exg. *Ich vertraue wem \tbf{auch} \tbf{Maria} \tbf{mag}. \\
I.\ac{nom} trust.1\ac{sg}\scsub{[dat]} \ac{rel}.\ac{dat}.\ac{an} also Maria.\ac{nom} like.3\ac{sg}\scsub{[acc]}.\\
`I trust whoever Maria also likes.' \flushfill{\ac{mg}, adapted from \pgcitealt{vogel2001}{345}}\label{ex:mg-dat-acc}

The example in \ref{ex:mg-dat-acc-u} is identical to \ref{ex:mg-dat-acc}, except for that the relative pronoun appears in the external less complex accusative case. This example is also ungrammatical: in addition to the more complex case not being the internal case, the relative pronoun also does not appear in the more complex case (the dative) but in the less complex case (the accusative).

\exg. *Ich vertraue wen \tbf{auch} \tbf{Maria} \tbf{mag}. \\
I.\ac{nom} trust.1\ac{sg}\scsub{[dat]} \ac{rel}.\ac{acc}.\ac{an} also Maria.\ac{nom} like.3\ac{sg}\scsub{[acc]}.\\
`I trust whoever Maria also likes.' \flushfill{\ac{mg}, adapted from \pgcitealt{vogel2001}{345}}\label{ex:mg-dat-acc-u}

Consider the example in \ref{ex:mg-acc-dat}. In this example, the internal dative case competes against the external accusative case.
The internal case is dative, as the predicate \tit{vertrauen} `to trust' takes dative objects.
The external case is accusative, as the predicate \tit{einladen} `to invite' takes accusative objects.
The relative pronoun \tit{wem} `\ac{rel}.\ac{dat}.\ac{an}' appears in the internal case: the dative. The relative pronoun is marked in bold, just like as the relative clause, showing that the relative pronoun patterns with the relative clause.
The example adheres to the case scale, and the more complex case (here the dative) is the internal case, so the example is grammatical.

\exg. Ich {lade ein} \tbf{wem} \tbf{auch} \tbf{Maria} \tbf{vertraut}. \\
I.\ac{nom} invite.1\ac{sg}\scsub{[acc]} \ac{rel}.\ac{dat}.\ac{an} also Maria.\ac{nom} trust.3\ac{sg}\scsub{[dat]}.\\
`I invite whoever Maria also trusts.' \flushfill{\ac{mg}, adapted from \pgcitealt{vogel2001}{344}}\label{ex:mg-acc-dat}

This example in \ref{ex:mg-acc-dat-u} is identical to \ref{ex:mg-acc-dat}, except for that the relative pronoun appears in the external less complex accusative case. This example is ungrammatical: although the internal case is more complex, the relative pronoun appears in the less complex case (the accusative) and not in the more complex case (the dative).

\exg. *Ich {lade ein} \tbf{wen} \tbf{auch} \tbf{Maria} \tbf{vertraut}. \\
I.\ac{nom} invite.1\ac{sg}\scsub{[acc]} \ac{rel}.\ac{acc}.\ac{an} also Maria.\ac{nom} trust.3\ac{sg}\scsub{[dat]}.\\
`I invite whoever Maria also trusts.' \flushfill{\ac{mg}, adapted from \pgcitealt{vogel2001}{344}}\label{ex:mg-acc-dat-u}

A summary of the data is given in \ref{tbl:summary-modern-german}.
The three cells in the lower left corner are the situations in which the internal case surfaces when it wins the competition. They correspond to the examples \ref{ex:mg-nom-acc}, \ref{ex:mg-nom-dat} and \ref{ex:mg-acc-dat}.
The three cells in the upper right corner are the situations in which the external case surfaces when it wins the competition. As Modern German does not allow the winner of the case competition to surface when it appears in the external case, these instances are ungrammatical. They correspond to the examples \ref{ex:mg-acc-nom}, \ref{ex:mg-dat-nom} and \ref{ex:mg-dat-acc}.

\begin{table}[H]
  \center
  \caption{Summary Modern German headless relatives}
  \begin{tabular}{c|c|c|c}
    \toprule
   \textsubscript{\ac{int}} \textsuperscript{\ac{ext}}
          & [\ac{nom}]
          & [\ac{acc}]
          & [\ac{dat}]
          \\ \cmidrule{1-4}
      [\ac{nom}]
          &
          & *
          & *
          \\ \cmidrule{1-4}
      [\ac{acc}]
          & \ac{acc}
          &
          & *
          \\ \cmidrule{1-4}
      [\ac{dat}]
          & \ac{dat}
          & \ac{dat}
          &
          \\
    \bottomrule
  \end{tabular}
  \label{tbl:summary-modern-german}
\end{table}

In sum, Modern German has two requirements for case competition in its headless relatives. First, the relative pronoun surfaces in the more complex case, following the case scale. Second, the case competition can only be won by the internal case, and not by the external case.

\section{Pattern III: only external}\label{sec:pattern-iii}

Logically, a third type of language exist. That is a language that only allows the external case to surface when it wins the case competition. This pattern looks as follows.



To my knowledge, this pattern is not attested in any (extinct) natural language. In this section I discuss Ancient Greek, for which is has been claimed in the literature that it follows the pattern. I give examples that show that Ancient Greek actually patterns with Gothic and Old High German.

It has been claimed that Ancient Greek only allows the external case to surface when it wins the case competition  \citep[cf.][]{cinqueforthcoming}. It does indeed seem to be the case that examples in which the external case wins over the internal case are more frequent (see \citealt{kakarikos2014} for a number of them with different combinations of cases coming together) than examples in which the internal case wins over the external case.\footnote{
In this dissertation I do not address the question of why certain constructions and configurations are more frequent than others. My goal is to set up a system that generates the grammatical patterns and excludes the ungrammatical or unattested patterns.
}
However, such examples exist. I give an example in which the internal accusative wins over the external nominative.

Consider the example in \ref{ex:ancient-greek}.
The internal case is nominative, as the predicate \tit{philéō} `to love' takes accusative objects.
The external case is accusative, as the predicate \tit{apothnḗiskō} `to die' takes nominative subjects.
The relative pronoun \tit{hòn} `\ac{rel}.\ac{acc}.\ac{m}.\ac{sg}' appears in the internal case: the accusative. The relative pronoun is marked in bold, just like as the relative clause, showing that the relative pronoun patterns with the relative clause.

\exg. \tbf{hòn} \tbf{hoi} \tbf{theoì} \tbf{philoũsin} apothnḗͅskei néos\\
\ac{rel}.\ac{acc}.\ac{m}.\ac{sg} the god.\ac{pl} love.3\ac{pl}\scsub{acc} die.3\ac{sg}\scsub{nom} young\\
`He whom the gods love dies young.' (Menander, The Double Deceiver, fragment 125)\label{ex:ancient-greek}

To conclude, Ancient Greek patterns with Gothic and Old High German.



\section{Aside: languages without case competition}\label{sec:potentiel-counterexamples}









I also discuss languages in which the relative pronouns appear in the external case. However, they are also not languages of the third type, as they do not show case competition.



There are examples of languages in which the relative pronoun always surfaces in the external case. However, they do not show case competition.
I show that this is the case for Old English and for Modern Greek.

I start with Old English.
Consider the example in \ref{ex:oe-acc-dat}.
The internal case is dative, as the preposition \tit{onuppan} `upon' takes dative objects.
The external case is accusative, as the predicate \tit{tōbrȳsan} `to pulversize' takes accusative objects.
The relative pronoun \tit{ðone} `\ac{rel}.\ac{acc}.\ac{sg}' appears in the external case: the accusative.
The example appears in the external case, but does not adhere to the case scale: the relative pronoun appears in the less complex external  case (the accusative), and not in the more complex internal case (the dative).
The example is grammatical, because in Old English the relative pronoun appears in the external case. The case scale is irrelevant for the case in which the relative pronoun appears in Old English.

\exg. he tobryst ðone \tbf{ðe} \tbf{he} \tbf{onuppan} \tbf{fylð}\\
 it pulverizes\scsub{[acc]} \ac{rel}.\ac{acc} \ac{comp} it upon\scsub{[dat]} falls\\
`It pulverizes him whom it falls upon.' \flushfill{Old English, adapted from \pgcitealt{harbert1983}{550}} \label{ex:oe-acc-dat}

The same pattern appears in Modern Greek.
Consider the example in \ref{ex:greek-nom-acc}.
The internal case is accusative, as the predicate \tit{irθó} `to invite' takes accusative objects.
The external case is accusative, as the predicate \tit{kálesó} `to come' takes nominative subjects.
The relative pronoun \tit{ópji} `\ac{rel}.\ac{nom}.\ac{pl}.\ac{m}' appears in the external case: the nominative.
The example appears in the external case, but does not adhere to the case scale: the relative pronoun appears in the less complex external  case (the nominative), and not in the more complex internal case (the accusative).
The example is grammatical, because in Modern Greek the relative pronoun appears in the external case. The case scale is irrelevant for the case in which the relative pronoun appears in Modern Greek .

\exg. Irθan ópji \tbf{káleses}.\\
come.\ac{pst}.3\ac{pl}\scsub{nom} \ac{rel}.\ac{nom}.\ac{pl}.\ac{m} invite.\ac{pst}.2\ac{sg}\scsub{acc}\\
`Whoever you invited came.'\flushfill{Modern Greek, adapted from \pgcitealt{daskalaki2011}{80}}\label{ex:greek-nom-acc}

This example in \ref{ex:greek-nom-acc-u} is identical to \ref{ex:greek-nom-acc}, except for that the relative pronoun appears in the internal  more complex case. This example is ungrammatical: the relative pronoun does not appear in the external case. The fact that the internal case is more complex is irrelevant.

\exg. *Irθan \tbf{ópjus} \tbf{káleses}.\\
come.\ac{pst}.3\ac{pl}\scsub{nom} \ac{rel}.\ac{acc}.\ac{pl}.\ac{m} invite.\ac{pst}.2\ac{sg}\scsub{acc}\\
`Whoever you invited came.'\flushfill{Modern Greek, adapted from \pgcitealt{daskalaki2011}{79}}\label{ex:greek-nom-acc-u}

When the internal case is genitive instead of accusative, a clitic is added to the sentence to make it grammatical.
Consider the example in \ref{ex:greek-nom-gen}.
The internal case is genitive, as the predicate \tit{eðósó} `to give' takes genitive objects.
The external case is accusative, as the predicate \tit{efχarístisó} `to thank' takes nominative subjects.
The relative pronoun \tit{ópjon} `\ac{rel}.\ac{nom}.\ac{pl}.\ac{m}' appears in the internal case: the genitive. The relative pronoun is not marked in bold, unlike the relative clause, showing that the relative pronoun patterns with the main clause.
The example appears in the external case, but does not adhere to the case scale: the relative pronoun appears in the less complex external case (the nominative), and not in the more complex internal case (the genitive). The relative clause obligatorily contains the genitive clitic \tit{tus} `\ac{cl}.\ac{gen}.3\ac{pl}'.\footnote{
In Modern German, it is possible to insert a light head to resolve a situation with a more complex external case. However, then the relative pronoun has to change as well (from a \tsc{wh}-pronoun into a \tsc{d}-pronoun). I assume this is a different construction, and the Modern Greek one with the clitic inserted is not.
}
The example is grammatical, because in Modern Greek the relative pronoun appears in the external case. The case scale is irrelevant for the case in which the relative pronoun appears in Modern Greek .

\exg. Me efχarístisan ópji \tbf{tus} \tbf{íχa} \tbf{ðósi} \tbf{leftá}.\\
 \ac{cl}.1\ac{sg}.\ac{acc} thank.\ac{pst}.3\ac{pl}\scsub{nom} \ac{rel}.\ac{nom}.\ac{pl}.\ac{m} \ac{cl}.\ac{gen}.3\ac{pl} have.\ac{pst}.1\ac{sg} give.\ac{ptcp}\scsub{gen} money\\
 `Whoever I had given money to, thanked me.'\flushfill{Modern Greek, adapted from \pgcitealt{daskalaki2011}{80}}\label{ex:greek-nom-gen}

In sum, in the relative pronoun in Old English and Modern Greek appears in the the external case, and it does not adhere to the case scale.

Taking this all together, I have not encountered a language that only allows the external case to surface when it wins the case competition. Ancient Greek allows the internal or the external case to surface when it wins the case competition. Modern Greek and Old English are examples of languages in which the relative pronoun appears in the external case, and that do not take the case scale into account.


\section{Summary}\label{sec:summary-2-patterns}

In case competition in headless relatives two factors play a role. The first one is which case wins the case competition. It is a crosslinguistically stable fact that this is determiner by the case scale in \ref{ex:case-scale-two-patterns-sum}, repeated from Chapter \ref{ch:recurring}. A case more to the right on the scale wins over a case more to the left on the scale.

\ex. \ac{nom} < \ac{acc} < \ac{dat}\label{ex:case-scale-two-patterns-sum}

The second factor is whether the internal and the external case are allowed to surface when they wins the case competition. This differs across languages. There are three possible patterns: (1) a pattern in which the external or the internal case are allowed to surface when they win, (2) a pattern in which only the internal case is allowed to surface when it wins, and (3) a pattern in which only the external case is allowed to surface when it wins.

Gothic and Old High German are examples of languages of the first type. Modern German is an example of a language of the second type. To my knowledge, the third pattern is not attested. It is impossible to prove that this pattern does not exist (or has not existed) in any natural language, and it could be an accidental gap. However, in line with the available data so far, I set up a system in the next section that derives the two attested patterns, and excludes the third one.
