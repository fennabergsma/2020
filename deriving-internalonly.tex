% !TEX root = thesis.tex

\chapter{Deriving the internal-only type}\label{ch:deriving-onlyinternal}

Languages of the internal-only type can be summarizes as in Table \ref{tbl:rel-lh-mg}.

\begin{table}[htbp]
  \center
  \caption{Grammaticality in the internal-only type}
\begin{tabular}{cc}
  \toprule
                                        & surface pronoun         \\
  \cmidrule(lr){2-2}
\tsc{k}\scsub{int} = \tsc{k}\scsub{ext} & \tsc{rp}\scsub{int/ext} \\
\tsc{k}\scsub{int} > \tsc{k}\scsub{ext} & \tsc{rp}\scsub{int}     \\
\tsc{k}\scsub{int} < \tsc{k}\scsub{ext} & *                       \\
\bottomrule
\end{tabular}
\label{tbl:rel-lh-mg}
\end{table}

When the internal and the external case match, and there is a tie, the relative pronoun surfaces in the this particular case (just like in all other language types).
When the internal case wins the case competition, this type of language allows the internal case to surface. This means that the relative pronoun with a more complex internal case can be the surface pronoun.
When the external case wins the case competition, this type of language does not allow the external case to surface. This means that the light head with a more complex external case cannot be the surface pronoun.

In Chapter \ref{ch:the-basic-idea}, I suggested that the relative pronoun contains at least one feature more than the light head.
In the internal-only type of language, these features are spelled out in such a way that they form the constituency shown in Figure \ref{fig:rel-lh-intonly-simple}.

\begin{figure}[htbp]
  \center
  \begin{tabular}[b]{ccc}
      \toprule
      light head & & relative pronoun \\
      \cmidrule(lr){1-1} \cmidrule(lr){3-3}
      \begin{forest} boom
      [\tsc{k}P,
          [\tsc{k}]
          [ϕP
              [\phantom{xxx}, roof, baseline]
          ]
      ]
      \end{forest}
      & \phantom{x} &
    \begin{forest} boom
      [\tsc{rel}P
          [\tsc{rel}P
              [\phantom{xxx}, roof, baseline]
          ]
          [\tsc{k}P
              [\tsc{k}]
              [ϕP
                  [\phantom{xxx}, roof, baseline]
              ]
          ]
      ]
    \end{forest}\\
      \bottomrule
  \end{tabular}
   \caption {\tsc{lh} and \tsc{rp} in the internal-only type}
  \label{fig:rel-lh-intonly-simple}
\end{figure}

When the internal and the external case match, the relative pronoun can delete the light head, because the light head forms a single constituent within the relative pronoun.
When the internal case is more complex than the external case, the relative pronoun can still delete the light head, because it still forms a constituent within the relative pronoun.
When the external case is more complex than the internal case, the relative pronoun is not a single constituent within the light head. The relative pronoun contains features that are not part of the light head. As a result, there is no grammatical form to surface when the internal case is more complex.

In this chapter, I show that Modern German light heads and relative pronouns have this type of structure. I give a compact version of the structures in Figure \ref{fig:rel-lh-mg}.

\begin{figure}[htbp]
  \center
  \begin{tabular}[b]{ccc}
      \toprule
      light head & & relative pronoun \\
      \cmidrule(lr){1-1} \cmidrule(lr){3-3}
      \begin{forest} boom
        [\tsc{k}P,
        tikz={
        \node[label=below:\tit{r/n/m},
        draw,circle,
        scale=0.75,
        fit to=tree]{};
        }
            [\tsc{k}]
            [ϕP
                [\phantom{xxx}, roof, baseline]
            ]
        ]
      \end{forest}
      & \phantom{x} &
      \begin{forest} boom
        [\tsc{rel}P
            [\tsc{rel}P,
            tikz={
            \node[label=below:\tit{w},
            draw,circle,
            scale=0.75,
            fit to=tree]{};
            }
                [\phantom{xxx}, roof]
            ]
            [\tsc{k}P,
            tikz={
            \node[label=below:\tit{r/n/m},
            draw,circle,
            scale=0.75,
            fit to=tree]{};
            }
                [\tsc{k}]
                [ϕP
                    [\phantom{xxx}, roof, baseline]
                ]
            ]
        ]
      \end{forest}\\
      \bottomrule
  \end{tabular}
   \caption {\tsc{lh} and \tsc{rp} in Modern German}
  \label{fig:rel-lh-mg}
\end{figure}

Consider the light head in Figure \ref{fig:rel-lh-mg}.
Light heads in Modern German are spelled out by a single morpheme, indicated by the circle around the structure. They are spelled out as \tit{r}, \tit{n} or \tit{m} depending on which case they realize.

Consider the relative pronoun in Figure \ref{fig:rel-lh-mg}.
Relative pronouns in Modern German consist of two morphemes: the \tsc{k}P (which is the constituent that forms the light head) and the \tsc{rel}P, again indicated by the circles.

Throughout this chapter, I discuss what the exact feature content of relative pronouns and light heads is. I motivate that these features form the constituents shown in Figure \ref{fig:rel-lh-mg}.

The chapter is structured as follows.
First, I discuss the relative pronoun. I decompose the relative pronouns into the three morphemes I showed in Figure \ref{fig:rel-lh-mg}, and I show which features each of the morphemes corresponds to. I illustrate how different morphemes are combined using Nanosyntax.
Then I discuss the light head. I argue that Modern German headless relatives are derived from a type of light-headed relative clause that does not surface in the language. I show that the light head corresponds to one of the morphemes of the relative pronoun (the \tsc{k}P in Figure \ref{fig:rel-lh-mg}).
Finally, I compare the constituents of the light head and the relative pronoun. I show that the relative pronoun can delete the light head when the internal case matches the external case or when the internal case is more complex than the external one. When the external case is more complex, I show that none of the elements can delete the other one.


\section{The Modern German relative pronoun}\label{sec:mg-rel}

In Chapter \ref{ch:constituent-containment}, I argued that in the internal-only type, the features of the relative pronoun are spelled out in such a way that they form the constituency shown in Figure \ref{ex:simple-intonly-rp}.

\ex.\label{ex:simple-intonly-rp}
\begin{forest} boom
  [\tsc{rel}P
      [\tsc{rel}P,
      tikz={
      \node[label=below:\tit{w},
      draw,circle,
      scale=0.75,
      fit to=tree]{};
      }
          [\phantom{xxx}, roof]
      ]
      [\tsc{k}P,
      tikz={
      \node[label=below:\tit{r/n/m},
      draw,circle,
      scale=0.75,
      fit to=tree]{};
      }
          [\tsc{k}]
          [ϕP
              [\phantom{xxx}, roof, baseline]
          ]
      ]
  ]
\end{forest}

In this section, I give decompose the Modern German relative pronouns and I determine which features they correspond to.
I carefully establish the feature content of the relative pronoun, because the features that I introduce for Modern German are present in the same way in the other two language types.
In the end, I argue that Modern German relative pronoun have the structure shown in \ref{ex:mg-rp}.

\ex.\label{ex:mg-rp}
\begin{adjustbox}{max width=0.9\textwidth}
\begin{forest} boom
  [\tsc{rel}P, s sep=18mm
      [\tsc{rel}P,
      tikz={
      \node[label=below:\tit{we},
      draw,circle,
      scale=0.9,
      fit to=tree]{};
      }
          [\tsc{rel}]
          [\tsc{wh}P
              [\tsc{wh}]
              [\tsc{c}]
          ]
      ]
      [\tsc{k}P,
      tikz={
      \node[label=below:\tit{n/m},
      draw,circle,
      scale=0.95,
      fit to=tree]{};
      }
          [\tsc{k}]
          [\tsc{c}P
              [\tsc{c}]
              [\tsc{ind}P
                  [\tsc{ind}]
                  [\tsc{an}P
                      [\tsc{an}]
                      [\tsc{cl}P
                          [\tsc{cl}]
                          [ΣP
                              [Σ]
                              [\tsc{ref}]
                          ]
                      ]
                  ]
              ]
          ]
      ]
  ]
\end{forest}
\end{adjustbox}

I discuss two relative pronouns: the animate accusative and the animate dative. These are the two forms that I compare the constituents of in Section \ref{sec:comparing-mg}. I show them in \ref{ex:mg-rels}.

\ex.\label{ex:mg-rels}
\ag. we-n\\
 `\tsc{rp}.\tsc{an}.\tsc{acc}'\\
\bg. we-m\\
 `\tsc{rp}.\tsc{an}.\tsc{dat}'\\

I decompose the relative pronouns in three morphemes: the \tit{we} and the final consonant (\tit{n} or \tit{m}). For each morpheme, I discuss which features they spell out, and I give their lexical entries. In the next section, I show how I construct the relative pronouns by combining the separate morphemes.

I start with the final consonants: \tit{n} and \tit{m}. I argue that these consonants spell out gender features, number features, case features and pronominal features. Consider Table \ref{tbl:mg-paradigm-wh-rels}.

\begin{table}[htbp]
\center
\caption {Modern German relative pronouns \pgcitep{durrell2011}{5.3.3}}
\begin{tabular}{ccc}
\toprule
            & \ac{an} & \tsc{inan}\\
  \cmidrule{2-3}
  \ac{nom}  & we-r    & wa-s     \\
  \ac{acc}  & we-n    & wa-s     \\
  \ac{dat}  & we-m    & -        \\
\bottomrule
\end{tabular}
\label{tbl:mg-paradigm-wh-rels}
\end{table}

The final consonants depending on animateness and case.\footnote{
The vowel also differs between animateness. I return to this point when I discuss the feature content of the \tit{e}.
}
The differing final consonant can be observed in several contexts besides relative pronouns. Table \ref{tbl:mg-dieser} gives an overview of the demonstrative \tit{dieser} `this' in Modern German in two numbers, three genders and three cases.\footnote{
The vowel preceding the final consonant is written as \tit{e}. I write it as \tit{ə}, because this is how it is pronounced. I make this distinction to emphasize that this differs from the vowel used in the relative pronouns.
}\footnote{
Notice that the animate forms in Table \ref{tbl:mg-paradigm-wh-rels} are the masculine forms in Table \ref{tbl:mg-dieser} and that the inanimate forms in Table \ref{tbl:mg-paradigm-wh-rels} are the neuter forms in Table \ref{tbl:mg-dieser}. This is a pattern that appears more often.
}

\begin{table}[htbp]
\center
\caption {Modern German demonstrative \tit{dieser} `this' \pgcitep{durrell2011}{Table 5.2}}
 \begin{tabular}{ccccc}
 \toprule
             & \tsc{m}.\tsc{sg} & \tsc{n}.\tsc{sg} & \ac{f}.\tsc{sg} & \tsc{pl} \\
   \cmidrule{2-5}
   \ac{nom}  & dies-ə-r        & dies-ə-s         & dies-ə          & dies-ə   \\
   \ac{acc}  & dies-ə-n        & dies-ə-s         & dies-ə          & dies-ə   \\
   \ac{dat}  & dies-ə-m        & dies-ə-m         & dies-ə-r        & dies-ə-n \\
 \bottomrule
 \end{tabular}
 \label{tbl:mg-dieser}
\end{table}

Table \ref{tbl:mg-dieser} shows that the final consonant differs depending on gender, number and case. There is no vowel that differs between the different forms. I conclude from this that the consonant realizes features having to do with gender, number and case.

For number and gender, I adopt the features that are distinguished by \citet{harley2002} for pronouns. The feature \tsc{cl} corresponds to a gender feature, which is inanimate or neuter if it is not combined with any other features. Combining \tsc{cl} with the feature \tsc{an} gives the animate or masculine gender.\footnote{
If the features \tsc{cl} and \tsc{an} are combined with the feature \tsc{fem}, it becomes the feminine gender.
}
The feature \tsc{ind} corresponds to number, which is singular if it is not combined with any other features.

For case, I adopt the features of \citet{caha2009}, already introduced in Chapter \ref{ch:decomposition}. The feature \ac{f}1 and \tsc{f}2 corresponds to the accusative, and the features \ac{f}1, \ac{f}2 and \tsc{f}3 correspond to the dative.

I continue with the pronominal features. Another context in which the final consonants appear is in their use as a pronoun. More specifically, the final consonants correspond to the weak pronoun in Modern German. In \ref{ex:mg-weak}, I give examples of the masculine accusative singular and masculine dative singular.

\ex.\label{ex:mg-weak}
\ag. Ich wollte n gestern schon anrufen.\\
 I wanted 3\tsc{sg}.\tsc{m}.\tsc{acc}.\tsc{wk} yesterday already call\\
 `I already wanted to call him yesterday.'
\bg. Ich helfe m sein Fahrrad zu reparieren.\\
 I help 3\tsc{sg}.\tsc{m}.\tsc{dat}.\tsc{wk} his bike to repare\\
 `I help him reparing his bike.'

\citet{cardinaletti1994} split pronouns in three classes: strong pronouns, weak pronouns and clitics. There are several tests that distinguish the types from each other. In what follows, I show that the pronouns in \ref{ex:mg-weak} are neither strong pronouns nor clitics, and therefore, should be classified as weak pronouns. The tests I are are taken from \citet{cardinaletti1994}.

First, I show that \tit{n} and \tit{m} are not strong pronouns because of how they behave under coordination and under focus.
Strong pronouns can be coordinated. The examples in \ref{ex:wk-pron-coord} illustrate that \tit{n} and \tit{m} cannot be coordinated.

\ex.\label{ex:wk-pron-coord}
\ag. *Ich wollte Jan und n gestern schon anrufen.\\
 I wanted Jan and 3\tsc{sg}.\tsc{m}.\tsc{acc}.\tsc{wk} yesterday already call\\
 `I already wanted to call Jan and him yesterday.'
\bg. *Ich helfe Jan und m sein Fahrrad zu reparieren.\\
 I help Jan and 3\tsc{sg}.\tsc{m}.\tsc{acc}.\tsc{wk} his bike to repare\\
 `I help Jan and him repairing his bike.'

Strong pronouns can be focused.
The examples in \ref{ex:wk-pron-focus} show that \tit{n} and \tit{m} cannot be focused.

\ex.\label{ex:wk-pron-focus}
\ag. *Ich wollte nur n anrufen.\\
 I wanted only 3\tsc{sg}.\tsc{m}.\tsc{acc}.\tsc{wk} call\\
 `I wanted to call only him.'
\bg. *Ich helfe nur m sein Fahrrad zu reparieren.\\
 I help only 3\tsc{sg}.\tsc{m}.\tsc{dat}.\tsc{wk} his bike to repare\\
 `I help only him repairing his bike.'

Second, I show that the consonant is not a clitic because of how they behave with respect to following dative objects and combining with prepositions.
Clitics can either follow a dative object or precede it. Strong and weak pronouns can only follow it.
The examples in \ref{ex:wk-pron-order-n} and \ref{ex:wk-pron-order-m} show that \tit{n} and \tit{m} can only follow the dative object \tit{Jan}.

\ex.\label{ex:wk-pron-order-n}
\ag. .. dass Ursel Jan n empfohlen hat.\\
 {} that Ursel Jan 3\tsc{sg}.\tsc{m}.\tsc{acc}.\tsc{wk} recommended has\\
 `that Ursel recommended him to Jan.'
\bg. *.. dass Ursel n Jan empfohlen hat.\\
{} that Ursel 3\tsc{sg}.\tsc{m}.\tsc{acc}.\tsc{wk} Jan recommended has\\
`that Ursel recommended him to Jan.'

\ex.\label{ex:wk-pron-order-m}
\ag. .. dass Ursel m Jan empfohlen hat.\\
 {} that Ursel 3\tsc{sg}.\tsc{m}.\tsc{dat}.\tsc{wk} Jan recommended has\\
 `that Ursel recommended Jan to him.'
\bg. *.. dass Ursel Jan m empfohlen hat.\\
{} that Ursel Jan 3\tsc{sg}.\tsc{m}.\tsc{dat}.\tsc{wk} recommended has\\
`that Ursel recommended Jan to him.'

Clitics cannot combine with prepositions.
The examples in \ref{ex:wk-pron-prep-n} and \ref{ex:wk-pron-prep-m} show that \tit{n} and \tit{m} can combine with prepositions.

\ex.\label{ex:wk-pron-prep-n}
\ag. Ich habe schon ein Geschenk für n gekauft.\\
 I have already a gift for 3\tsc{sg}.\tsc{m}.\tsc{acc}.\tsc{wk} bought\\
 `I already bought a gift for him.'
\bg. Ich bin schnell auf n zu gelaufen.\\
 I am fast on 3\tsc{sg}.\tsc{m}.\tsc{acc}.\tsc{wk} to walked\\
 `I walked toward him fast.'

\ex.\label{ex:wk-pron-prep-m}
\ag. Ich war mit m im Wald wandern.\\
 I have already a gift for 3\tsc{sg}.\tsc{m}.\tsc{dat}.\tsc{wk} bought\\
 `I was hiking with him in the woods.'
\bg. Ich habe Blumen von m bekommen.\\
 I have flowers from 3\tsc{sg}.\tsc{m}.\tsc{dat}.\tsc{wk} received\\
 `I received flowers from him.'

In sum, \tit{n} and \tit{m} are not strong pronouns and not clitics, so they are weak pronouns.
This means that the forms also contain pronominal features.
I propose that two pronominal features are present: \tsc{ref} and Σ.
\citet{harley2002} claim that all pronouns contain the feature \tsc{ref}, because they are referential expressions. The feature Σ is present because the consonants are weak pronouns \citep{cardinaletti1994}.\footnote{
I assume that clitics lack the features \tsc{ref} (which corresponds to the LP in \pgcitealt{cardinaletti1994}{61}) and the feature Σ. Strong pronouns have, in addition to \tsc{ref} and Σ, another feature (C in terms of \pgcitealt{cardinaletti1994}{61}).
}

I give the lexical entries for \tit{n} and \tit{m} in \ref{ex:mg-entry-n} and \ref{ex:mg-entry-m}.
The \tit{n} is the nominative masculine singular, so it spells out the features \tsc{ref}, Σ, \tsc{cl}, \tsc{an}, \tsc{ind} \ac{f}1 and \tsc{f}2. The \tit{m} is the accusative masculine singular, so it spells out the features that the \tit{r} spells out plus \ac{f}3.

\ex.\label{ex:mg-entries-nm}
\a.\label{ex:mg-entry-n}
\begin{forest} boom
  [\tsc{acc}P
      [\ac{f}2]
      [\tsc{nom}P
          [\ac{f}1]
          [\tsc{ind}P
              [\tsc{ind}]
              [\tsc{an}P
                  [\tsc{an}]
                  [\tsc{cl}P
                      [\tsc{cl}]
                      [ΣP
                          [Σ]
                          [\tsc{ref}]
                      ]
                  ]
              ]
          ]
      ]
  ]
  {\draw (.east) node[right]{⇔ \tit{n}}; }
\end{forest}
\b.\label{ex:mg-entry-m}
\begin{forest} boom
  [\tsc{dat}P
      [\ac{f}3]
      [\tsc{acc}P
          [\ac{f}2]
          [\tsc{nom}P
              [\ac{f}1]
              [\tsc{ind}P
                  [\tsc{ind}]
                  [\tsc{an}P
                      [\tsc{an}]
                      [\tsc{cl}P
                          [\tsc{cl}]
                          [ΣP
                              [Σ]
                              [\tsc{ref}]
                          ]
                      ]
                  ]
              ]
          ]
      ]
  ]
  {\draw (.east) node[right]{⇔ \tit{m}}; }
\end{forest}

Note that the ordering of the features here is not random. I motivate this later on in this section.

I continue with the \tit{e}. This morpheme is present (\tsc{wh}-)relative pronouns, but also in demonstratives, as shown in \ref{tbl:mg-paradigm-dem}.\footnote{
Note that \tsc{wh}-relative pronouns, unlike the demonstratives, do not have a feminine form for the relative pronouns in Table \ref{tbl:mg-paradigm-wh-rels}. Demonstratives also have plural forms, and \tsc{wh}-relative pronouns do not. As far as I know, this holds for all relative pronouns in languages of the internal-only type (cf. also for Finnish, even though it makes a lot of morphological distinctions) and of the matching type. Relative pronouns in languages of the unrestricted type do inflect for feminine, as well as always-external languages. In Chapter \ref{ch:discussion} I return to this observation in relation with the always-external languages.
}

\begin{table}[H]
\center
\caption {Modern German demonstrative pronouns \pgcitep{durrell2011}{5.4.1}} %dieser source
 \begin{tabular}{ccccc}
 \toprule
             & \ac{m}  & \tsc{n} & \ac{f} & \ac{pl} \\
   \cmidrule{2-4}
   \ac{nom}  & d-e-r   & d-a-s   & d-ie   & d-ie    \\
   \ac{acc}  & d-e-n   & d-a-s   & d-ie   & d-ie    \\
   \ac{dat}  & d-e-m   & d-e-m   & d-e-r  & d-e-n   \\
 \bottomrule
 \end{tabular}
 \label{tbl:mg-paradigm-dem}
\end{table}

I suggest that the morpheme \tit{e} spells out deixis features, pronominal features and gender and number features. I start with the deixis features.

In relative pronouns it does not express spatial deixis, but discourse deixis: it establishes a relation with an antecedent.
Generally, three types of deixis are distinguished: proximal, medial and distal. I argue that \tit{e} in the relative pronoun corresponds to the medial. Generally speaking, \tsc{wh}-pronouns combine with the medial or the distal. English has morphological evidence for this claim. Demonstratives in English can combine with either the proximal or this medial/distal, as shown in \ref{ex:english-dem}.

\ex.\label{ex:english-dem}
 \ag. this\\
 \tsc{dem}.\tsc{prox}\\
 \bg. that\\
 \tsc{dem}.\tsc{med/dist}\\

\tsc{wh}-pronouns combine with the medial/distal and are ungrammatical when combined with the proximal, shown in \ref{ex:english-wh}.

\ex.\label{ex:english-wh}
 \ag. *whis\\
 \tsc{wh}.\tsc{prox}\\
 \bg. what\\
 \tsc{wh}.\tsc{med/dist}\\

The use of the medial in \tsc{wh}-pronouns can be understood conceptually if one connects spatial deixis to discourse deixis \citep[cf.][]{colasanti2019}. The proximal is spatially near the speaker, and it refers to knowledge that the speaker possesses. The medial is spatially near the hearer, and it refers to knowledge that the hearer possesses. The distal is spatially away from the speaker and the hearer, and refers to knowledge that neither of them possess. In \tsc{wh}-pronouns, the speaker is not aware of the knowledge, so the use of the proximal is excluded. Since I do not have explicit evidence for the presence of the distal, I assume that it is the medial that combines with the \tsc{wh}-pronoun.

I adopt the features for deixis distinguished by \citet{lander2018}. The feature \tsc{dx}\scsub{1} corresponds to the proximal, the features \tsc{dx}\scsub{1} and \tsc{dx}\scsub{2} correspond to the medial, and the features \tsc{dx}\scsub{1}, \tsc{dx}\scsub{2} and \tsc{dx}\scsub{3} correspond to the distal.
The difference between the proximal, the medial and the distal cannot be observed in Modern German, because it is syncretic all of them \pgcitep{lander2018}{387}, see Table \ref{tbl:mg-paradigm-dem}.

I continue with gender and number features. Table \ref{tbl:mg-paradigm-dem} shows that the vowel changes depending on gender and number. The masculine singular uses the \tit{e}, the neuter singular uses the \tit{a}, the feminine singular uses the \tit{ie} and the plural uses for all genders the \tit{ie}.\footnote{
Note that the dative forms in all gender and numbers have the \tit{e}, which I assigned to masculine gender. This holds for the genitive forms too, which I have not given here. I see these as arguments in favor the following analysis.

The phonological part of the lexical entries in \ref{ex:mg-entries-nm} should not be \tit{n} and \tit{m} but \tit{en} and \tit{em}.
Additionally, the lexical entry carries additional information: the lexical entries in \ref{ex:mg-entries-nm} only bring a single slot for a single consonant.
The lexical entry for the neuter singular nominative and accusative is \tit{as} but only the \tit{s} surfaces due the single consonant slot.
The feminine singular and the plurals do not have a weak pronoun and they do not have a marker in forms like \tit{diese} `this' (see Table \ref{tbl:mg-dieser}), because their lexical entry does not contain a consonant.

Under this analysis, the lexical entry does not correspond to the phonology \tit{e}. Its phonological contribution is to bring a slot for a vowel. The lexical entry is then also not specified for gender, which correctly captures the observation that \tit{e} appears outside of the masculine.

As this matter is not relevant for the core of my analysis, I put it aside for now. For ease of exposition I assign a phonological exponent to each lexical entry.
}
Therefore, I suggest that \tit{e} spells out the features \tsc{an} and \tsc{ind}.

Finally, I assume that the \tit{e} corresponds to the feature \tsc{c}, a feature that strong pronouns have in addition to \tsc{ref} and Σ. Demonstratives, which contain the \tit{e} can namely be coordinated and focused, which is typical for strong pronouns \citep{cardinaletti1994}. I give an example of a coordinated demonstrative and a focused demonstrative in \ref{ex:mg-e-strong}.\footnote{
The feature \tsc{c} could in principle also be contained in the morpheme \tit{d}.
}

\ex.\label{ex:mg-e-strong}
\ag. Ich wollte Jan und den gestern schon anrufen.\\
 I wanted Jan and \tsc{dem}.\tsc{sg}.\tsc{m}.\tsc{acc}. yesterday already call\\
 `I already wanted to call Jan and him yesterday.'
\bg. Ich helfe nur dem sein Fahrrad zu reparieren.\\
 I help only \tsc{dem}.\tsc{sg}.\tsc{m}.\tsc{dat} his bike to repare\\
 `I help only him repairing his bike.'

In sum, the \tit{e} spells out the features \tsc{an}, \tsc{ind}, \tsc{c}, \tsc{dx}\scsub{1}, \tsc{dx}\scsub{2} and \tsc{dx}\scsub{3}, as shown in \ref{ex:mg-entry-e}.

\ex.\label{ex:mg-entry-e}
\begin{forest} boom
  [\tsc{dist}P
      [\tsc{dx}\scsub{3}]
      [\tsc{med}P
          [\tsc{dx}\scsub{2}]
          [\tsc{prox}P
              [\tsc{dx}\scsub{1}]
              [\tsc{c}P
                  [\tsc{c}]
                  [\tsc{ind}P
                      [\tsc{ind}]
                      [\tsc{an}]
                  ]
              ]
          ]
      ]
  ]
  {\draw (.east) node[right]{⇔ \tit{e}}; }
\end{forest}

This leaves the morpheme \tit{w} of the relative pronoun. I showed in this section that \tit{w} combines with the same endings as the \tit{d} does in demonstratives (or relative pronouns in headed relatives).
This identifies the \tit{d} and, more importantly for the discussion here, the \tit{w} as a separate morpheme. I suggest that the morpheme \tit{w} spells out three features: \tsc{wh}, \tsc{rel} and \tsc{dx}\scsub{2}.

The first feature I refer to as \tsc{wh}. This is a feature that \tsc{wh}-pronouns, such as \tsc{wh}-relative pronouns and interrogatives, share. The \tsc{wh}-element triggers the construction of a set of alternatives in the sense of \citet{rooth1985,rooth1992} \citep{hachem2015}. This contrasts with the \tsc{d} in Table \ref{tbl:mg-paradigm-dem}, which is responsible for establishing a definite reference.
The second relevant feature is \tsc{rel}, which establishes a relation.
The third feature is \tsc{dx}\scsub{2}, which ?

In sum, the \tit{w} spells out the features \tsc{wh} \tsc{rel} and \tsc{dx}\scsub{2}, as shown in \ref{ex:mg-entry-w}.

\ex. \begin{forest} boom
  [\tsc{rel}P
      [\tsc{rel}]
      [\tsc{wh}P
          [\tsc{wh}]
          [\tsc{dx}\scsub{2}]
      ]
  ]
  {\draw (.east) node[right]{⇔ \tit{w}}; }
\end{forest}\label{ex:mg-entry-w}

At this point, I gave lexical entries for each of the morphemes (in \ref{ex:mg-entry-n}, \ref{ex:mg-entry-m}, \ref{ex:mg-entry-e} and \ref{ex:mg-entry-w})
and I showed what the relative pronouns as a whole look like (in \ref{ex:mg-rp}).
What is still needed, is a theory for combining these morphemes into a relative pronoun. This theory should determine which morphemes should be combined with each other in which order. Ideally, the theory is not language-specific, but the same for all languages. In what follows I show how this is accomplished in Nanosyntax. Readers who are not interested in the precise mechanics can proceed directly to Section \ref{sec:light-mg}.


\section{Combining morphemes in Nanosyntax}

The way Nanosyntax combines different morphemes is not by glueing them together directly from the lexicon. Instead, features are merged one by one using two components that drive the derivation. These two components are (1) a functional sequence, in which the features that need to be merged and their order in which they are merged are specified, and (2) the Spellout Algorithm, which describes the spellout procedure. The lexical entries that are available within a language interact with the derivation in such a way that the morphemes get combined in the right way. Note that the functional sequence and the Spellout Algorithm are stable across languages. The only difference between languages lies in their lexical entries.

\ref{ex:fseq-wh-rel} shows the functional sequence for relative pronouns. It gives all features it contains and their hierarchical ordering.

\ex.\label{ex:fseq-wh-rel}
\begin{adjustbox}{max width=0.9\textwidth}
\begin{forest} for tree={s sep=13mm, inner sep=0, l=0}
   [\tsc{k}P
       [\tsc{k}]
       [\tsc{rel}P
           [\tsc{rel}]
           [\tsc{wh}P
               [\tsc{wh}]
               [\tsc{med}P
                   [\tsc{dx}\scsub{2}]
                   [\tsc{prox}P
                       [\tsc{dx}\scsub{1}]
                       [\tsc{c}P
                           [\tsc{c}]
                           [\tsc{ind}P
                               [\tsc{ind}]
                               [\tsc{an}P
                                   [\tsc{an}]
                                   [\tsc{cl}P
                                       [\tsc{cl}]
                                       [ΣP
                                            [Σ]
                                            [\tsc{ref}]
                                       ]
                                   ]
                               ]
                           ]
                       ]
                   ]
               ]
           ]
       ]
   ]
\end{forest}
\end{adjustbox}

Starting from the bottom, these are pronominal features \tsc{ref} and Σ, deixis features \tsc{dx}\scsub{1} and \tsc{dx}\scsub{2}, gender features \tsc{cl} and \tsc{an}, a number feature \tsc{ind}, the strong pronoun feature \tsc{c}, operator features \tsc{wh} and \tsc{rel} and case features \tsc{k}.\footnote{
The \tsc{k}P in this functional sequence is a placeholder for multiple case projections.
When the extra light head is the accusative, the \tsc{k}P consists of the features \tsc{f}1 and \tsc{f}2, and they form the \tsc{acc}P.
When the extra light head is the dative, the \tsc{k}P consists of the features \tsc{f}1, \tsc{f}2 and \tsc{f}3, and they form the \tsc{dat}P.
}
This order is independently supported by work in the literature. Both Picallo and Kramer argue that number is hierarchically higher than gender. Case is agreed to be higher than number (cf. Bittner and Hale).

\ex.
\a. of those children
\b. of which children

Before I derive construct the relative pronouns, I explain how the spellout procedure in Nanosyntax works. Features (Fs) are merged one by one according to the functional sequence, starting from the bottom. After each instance of merge, the constructed phrase must be spelled out, as stated in \ref{ex:cyclic-phrasal-spellout}.

\ex. \tbf{Cyclic phrasal spellout} \citep{caha2020a}\\
Spellout must successfuly apply to the output of every Merge F operation. After successful spellout, the derivation may terminate, or proceed to another round of Merge F.\label{ex:cyclic-phrasal-spellout}

Spellout is successful when the phrase that contains the newly merged feature forms a constituents in a lexical tree that is part of the language's lexicon.
When the new feature is merged, it forms a phrase with all features merged so far.
If this created phrase cannot be spelled out successfuly (i.e. when it does not form a constituent in a lexical tree), there are two movement operations possible that modify the syntactic structure in such a way that the newly merged feature becomes part of a different syntactic structure.
These movements are triggered because spellout needs to successully apply, and, therefore, they are called spellout-driven movements.
A Spellout Algorithm specifies which movement operations apply and in which order this happens. I give it in \ref{ex:spellout-algorithm}.

\ex. \tbf{Spellout Algorithm} (as in \citealt{caha2020a}, based on \citealt{starke2018})\label{ex:spellout-algorithm}
 \a. Merge F and spell out.\label{ex:spellout-algorithm-phrasal}
 \b. If (a) fails, move the Spec of the complement and spell out.\label{ex:spellout-algorithm-spec}
 \b. If (b) fails, move the complement of F and spell out.\label{ex:spellout-algorithm-comp}

I informally reformulate what is in \ref{ex:spellout-algorithm}. I start with the first line in \ref{ex:spellout-algorithm-phrasal}. This says that a feature F is merged, and the newly created phrase FP is attempted to spell out.
The next two lines, \ref{ex:spellout-algorithm-spec} and \ref{ex:spellout-algorithm-comp}, describe the two types of rescue movements that take place when the spellout in \ref{ex:spellout-algorithm-phrasal} fails (i.e. when there is no match in the lexicon).
In the discussion about Modern German, only the first line leads to successful spellout. In the next section in which I discuss Polish derivations, second and third line also lead to successful spellouts. I give the full algorithm here to give the complete picture from the start.

If these two movement operations still do not lead to a successful spellout, there are two more derivational options possible: Backtracking and Spec Formation. I return to these options later in this section, when they are relevant in the derivation of Modern German relative pronouns.

I start constructing the nominative relative pronoun. Starting from the bottom of the functional sequence, the first two features that are merged at \tsc{ref} and Σ, creating a ΣP.

\ex.
\begin{forest} boom
  [ΣP
       [Σ]
       [\tsc{ref}]
  ]
\end{forest}

The syntactic structure forms a constituent in the lexical tree in \ref{ex:mg-entry-n-rep}, repeated from \ref{ex:mg-entry-n}, which corresponds to the \tit{n}.

\ex.
\begin{forest} boom
  [\tsc{acc}P
      [\tsc{f}2]
      [\tsc{nom}P
          [\ac{f}1]
          [\tsc{ind}P
              [\tsc{ind}]
              [\tsc{an}P
                  [\tsc{an}]
                  [\tsc{cl}P
                      [\tsc{cl}]
                      [ΣP
                          [Σ]
                          [\tsc{ref}]
                      ]
                  ]
              ]
          ]
      ]
  ]
  {\draw (.east) node[right]{⇔ \tit{n}}; }
\end{forest}
\label{ex:mg-entry-n-rep}

Therefore, the ΣP is spelled out as \tit{n}. As usual, I mark this by circling the part of the structure that corresponds to the lexical entry, and placing the corresponding phonology under it.
This spellout option corresponds to \ref{ex:spellout-algorithm-phrasal} in the Spellout Algorithm.

\ex.\label{ex:mg-spellout-n-ref-sigma}
\begin{forest} boom
  [ΣP,
  tikz={
  \node[label=below:\tit{n},
  draw,circle,
  scale=0.9,
  fit to=tree]{};
  }
       [Σ]
       [\tsc{ref}]
  ]
\end{forest}

There are more features in the functional sequence, so the next feature is merged.
This next feature is the feature \tsc{cl}, and a \tsc{cl}P is created.
The syntactic structure forms a constituent in the lexical tree in \ref{ex:mg-entry-n-rep}.
Therefore, the \tsc{cl}P is spelled out as \tit{n}, shown in \ref{ex:mg-spellout-n-sigma-cl}.

\ex.\label{ex:mg-spellout-n-sigma-cl}
\begin{forest} boom
  [\tsc{cl}P,
  tikz={
  \node[label=below:\tit{n},
  draw,circle,
  scale=0.9,
  fit to=tree]{};
  }
      [\tsc{cl}]
      [ΣP
           [Σ]
           [\tsc{ref}]
      ]
  ]
\end{forest}

The features \tsc{an} and \tsc{ind} are merged and spelled out in the same way.
First, the feature \tsc{an} is merged, and a \tsc{an}P is created.
The syntactic structure forms a constituent in the lexical tree in \ref{ex:mg-entry-n-rep}.
Therefore, the \tsc{an}P is spelled out as \tit{n}.
Then, the feature \tsc{ind} is merged, and a \tsc{ind}P is created.
The syntactic structure forms a constituent in the lexical tree in \ref{ex:mg-entry-n-rep}.
Therefore, the \tsc{ind}P is spelled out as \tit{n}, shown in \ref{ex:mg-spellout-n-ind}.

\ex.\label{ex:mg-spellout-n-ind}
\begin{forest} boom
  [\tsc{ind}P,
  tikz={
  \node[label=below:\tit{n},
  draw,circle,
  scale=0.95,
  fit to=tree]{};
  }
      [\tsc{ind}]
      [\tsc{an}P
          [\tsc{an}]
          [\tsc{cl}P
              [\tsc{cl}]
              [ΣP
                   [Σ]
                   [\tsc{ref}]
              ]
          ]
      ]
  ]
\end{forest}

The next feature in the functional sequence is the feature \tsc{c}. This feature can not be spelled out as the other ones before. I show that in what follows.
The feature \tsc{c} is merged, and a \tsc{c}P is created, as shown in \ref{ex:mg-spellout-c-phrasal}

\ex.\label{ex:mg-spellout-c-phrasal}
\begin{forest} boom
  [\tsc{c}P
      [\tsc{c}]
      [\tsc{ind}P
          [\tsc{ind}]
          [\tsc{an}P
              [\tsc{an}]
              [\tsc{cl}P
                  [\tsc{cl}]
                  [ΣP
                       [Σ]
                       [\tsc{ref}]
                  ]
              ]
          ]
      ]
  ]
\end{forest}

This syntactic structure does not form a constituent in the lexical tree in \ref{ex:mg-entry-n-rep}. There is also no other lexical tree that contains the structure in \ref{ex:mg-spellout-c-phrasal} as a constituent. Therefore, there is no successful spellout for the syntactic structure in the derivational step in which the structure is spelled out as a single phrase (\ref{ex:spellout-algorithm-phrasal} in the Spellout Algorithm).

The first movement option in the Spellout Algorithm is moving the specifier, as described in \ref{ex:spellout-algorithm-spec}. As there is no specifier in this structure, so the first movement option is irrelevant.
The second movement option in the Spellout Algorithm is moving the complement, as described in \ref{ex:spellout-algorithm-comp}. In this case, the complement of \tsc{c}, the \tsc{ind}P, is moved to the specifier of \tsc{ind}P. As this movement option does not lead to a successful match, I do not discuss it here. I come back to it in Chapter \ref{ch:deriving-matching}, in which it does lead to a successful match.

As I mentioned earlier, there are two more derivational options possible: Backtracking and Spec Formation. Derivationally, Backtracking comes first. However, since this does not lead to a successful spellout here I first introduce Spec Formation first and I return to Backtracking later. Spec Formation is a last resort operation, when the feature cannot be spelled out by any of the preceding options. It is formalized as in \ref{ex:merge-spec}.

\ex.\label{ex:merge-spec}
\tbf{Spec Formation} \citep{starke2018}:\\
If Merge F has failed to spell out (even after Backtracking), try to spawn a new derivation providing F and merge that with the current derivation, projecting F to the top node.

To reformulate this informally, if none of the preceding spellout options led to a successful spellout, a last resort operation applies. The feature that has not been spelled out yet, is merged with some other features (to which I come back next) in a separate workspace. Crucially, the phrase that is created is contained in a lexical tree in the language's lexicon. Finally, the feature is spelled out successfully. The newly created phrase (the spec) is merged as a whole with the already existing structure.

Now I come back to the `other' features that the feature is merged with to create a phrase that can be spelled out. This cannot be just any feature. What is crucial here again is the functional sequence. The newly merged feature is merged with features that precede it in this sequence. This can be a single feature or more than one. I illustrate this with the Modern German relative pronouns.

For \tsc{c} this means that it is merged with \tsc{ind}. Then, the lexicon is checked for a lexical tree that contains the phrase \tsc{c}P that contains \tsc{c} and \tsc{ind}, as shown in \ref{ex:mg-spellout-c-ind}.

\ex.\label{ex:mg-spellout-c-ind}
\begin{forest} boom
  [\tsc{c}P
      [\tsc{c}]
      [\tsc{ind}]
  ]
\end{forest}

This syntactic structure does not form a constituent in any of the lexical trees in the language's lexicon.
Therefore, the feature \tsc{c} combines not only with the feature merged before it, but with a phrase that consists of the two features merged before it: \tsc{ind} and \tsc{an}. I give the phrase this creates in \ref{ex:mg-structure-c-ind-an}.

\ex.\label{ex:mg-structure-c-ind-an}
\begin{forest} boom
  [\tsc{c}P
      [\tsc{c}]
      [\tsc{ind}P
          [\tsc{ind}]
          [\tsc{an}]
      ]
  ]
\end{forest}

This syntactic structure forms a constituent in the lexical tree in \ref{ex:mg-entry-e-rep}, repeated from \ref{ex:mg-entry-e}, which corresponds to the \tit{e}.

\ex.
\begin{forest} boom
  [\tsc{dist}P
      [\tsc{dx}\scsub{3}]
      [\tsc{med}P
          [\tsc{dx}\scsub{2}]
          [\tsc{prox}P
              [\tsc{dx}\scsub{1}]
              [\tsc{c}P
                  [\tsc{c}]
                  [\tsc{ind}P
                      [\tsc{ind}]
                      [\tsc{an}]
                  ]
              ]
          ]
      ]
  ]
  {\draw (.east) node[right]{⇔ \tit{e}}; }
  \label{ex:mg-entry-e-rep}
\end{forest}

Therefore, the \tsc{c}P is spelled out as \tit{e}, as shown in \ref{ex:mg-spellout-cp-ind}.

\ex.\label{ex:mg-spellout-cp-ind}
\begin{forest} boom
  [\tsc{c}P,
   tikz={
   \node[label=below:\tit{e},
   draw,circle,
   scale=0.9,
   fit to=tree]{};
   }
      [\tsc{c}]
      [\tsc{ind}P
          [\tsc{ind}]
          [\tsc{an}]
      ]
  ]
\end{forest}

The newly created phrase is merged as a whole with the already existing structure. As specified in \ref{ex:merge-spec}, the feature \tsc{c} projects to the top node. I show the results in \ref{ex:mg-spellout-cp}.

\ex.\label{ex:mg-spellout-cp}
\begin{adjustbox}{max width=0.9\textwidth}
\begin{forest} boom
  [\tsc{c}P, s sep = 30mm
      [\tsc{c}P,
       tikz={
       \node[label=below:\tit{e},
       draw,circle,
       scale=0.9,
       fit to=tree]{};
       }
          [\tsc{c}]
          [\tsc{ind}P
              [\tsc{ind}]
              [\tsc{an}]
          ]
      ]
      [\tsc{ind}P,
      tikz={
      \node[label=below:\tit{n},
      draw,circle,
      scale=1,
      fit to=tree]{};
      }
          [\tsc{ind}]
          [\tsc{an}P
              [\tsc{an}]
              [\tsc{cl}P
                  [\tsc{cl}]
                  [ΣP
                       [Σ]
                       [\tsc{ref}]
                  ]
              ]
          ]
      ]
  ]
\end{forest}
\end{adjustbox}

Notice here that there is an overlap of multiple features between the phrase on the right and the phrase on the left.\footnote{
There are three different proposals on Spec Formation.
\citet{caha2019} argue that there can only be a single feature overlap between the two phrases.
\citet{de2018} argue that there cannot be any overlap at all. The features that used in the second workspace are removed from the structure in the main workspace.
In this dissertation, I work with the proposal in \citet{starke2018}, in which the the overlap between the phrase on the left and the phrase on the right can also be more than a single feature. This is the only proposal of the three that allows me to derive all the forms I encounter.
}

The next feature in the functional sequence is the feature \tsc{dx}\scsub{1}. As always, it merged to the existing syntactic structure, which is now the \tsc{c}P. The result is the \tsc{prox}P shown in \ref{ex:mg-spellout-prox-phrasal}.

\ex.\label{ex:mg-spellout-prox-phrasal}
\begin{adjustbox}{max width=0.9\textwidth}
\begin{forest} boom
  [\tsc{prox}P
      [\tsc{dx}\scsub{1}]
      [\tsc{c}P, s sep = 30mm
          [\tsc{c}P,
           tikz={
           \node[label=below:\tit{e},
           draw,circle,
           scale=0.9,
           fit to=tree]{};
           }
              [\tsc{c}]
              [\tsc{ind}P
                  [\tsc{ind}]
                  [\tsc{an}]
              ]
          ]
          [\tsc{ind}P,
          tikz={
          \node[label=below:\tit{n},
          draw,circle,
          scale=0.95,
          fit to=tree]{};
          }
              [\tsc{ind}]
              [\tsc{an}P
                  [\tsc{an}]
                  [\tsc{cl}P
                      [\tsc{cl}]
                      [ΣP
                           [Σ]
                           [\tsc{ref}]
                      ]
                  ]
              ]
          ]
      ]
  ]
\end{forest}
\end{adjustbox}

This whole structure does not form a constituent in any of the lexical trees in the language's lexicon. Neither of the spellout driven movement operations leads to a successful spellout. This means that, once again, the derivation reaches a point at which one of the two more possible derivational options come into play. As I mentioned before, Backtracking comes first, and this is the operation that leads to a successful spellout here.

Consider the syntactic structure in \ref{ex:mg-spellout-prox-phrasal} again. The feature \tsc{dx}\scsub{1} is merged with the highest \tsc{c}P. In this position it cannot be spelled out.
Consider now the lexical entry in \ref{ex:mg-entry-e-rep}. This is a lexical tree that contains \tsc{dx}\scsub{1}. This means that the feature \tsc{dx}\scsub{1} somehow needs to end up in the Spec that has just been merged.
I follow \citet{caha2019} who proposes that this happens via Backtracking. He argues that the main idea of Backtracking is that a feature is merged with a different tree than the one it was merged with before, as stated in \ref{ex:backtracking}.\footnote{
In this dissertation I do not discuss the effect that Backtracking `normally' has, namely to try a different spellout option at the previous cycle. That does not mean that I assume it is not part of the derivation: I actually assume it a step that attempted is. I refrain from mentioning it, because this does not lead to a successful spellout in any of the derivations I discuss.
}

\ex. \tbf{The logic of backtracking} \pgcitep{caha2019}{198}\\\label{ex:backtracking}
When spellout of F fails, go back to the previous cycle, and provide a different configuration for Merge F.

Imagine a situation in which the previous feature was spelled out with a complex specifier and the next feature reaches the derivational option Backtracking. This is exactly the situation that arises after \tsc{dx}\scsub{1} is merged. Providing a different configuration means splitting up the two phrases, and then merging the feature again. Specifically, I adopt the proposal in which the features is merged in both workspaces, as stated in \ref{ex:multiple-merge}.

\ex. \tbf{Multiple Merge} \pgcitep{caha2019}{227}\\\label{ex:multiple-merge}
When backtracking reopens multiple workspaces, merge F in each such workspace.

For the example under discussion, the situation looks as in \ref{ex:mg-dx1-2x}.

\ex.\label{ex:mg-dx1-2x}
\a.\label{ex:mg-cp-dx2}
\begin{forest} boom
  [\tsc{prox}P
      [\tsc{dx}\scsub{1}]
      [\tsc{c}P
          [\tsc{c}]
          [\tsc{ind}P
              [\tsc{ind}]
              [\tsc{an}]
          ]
      ]
  ]
\end{forest}
\b.\label{ex:mg-indp-dx2}
\begin{forest} boom
  [\tsc{prox}P
      [\tsc{dx}\scsub{1}]
      [\tsc{ind}P
          [\tsc{ind}]
          [\tsc{an}P
              [\tsc{an}]
              [\tsc{cl}P
                  [\tsc{cl}]
                  [ΣP
                       [Σ]
                       [\tsc{ref}]
                  ]
              ]
          ]
      ]
  ]
\end{forest}

The feature \tsc{dx}\scsub{1} is merged in both workspaces, so it combines with the \tsc{c}P in \ref{ex:mg-cp-dx2} and with the \tsc{ind}P \ref{ex:mg-indp-dx2}. Spellout has to be successful in at least one of the two workspaces. From here on, the derivation proceeds, as usual, according to the Spellout Algorithm, with the only difference that it happens in two workspaces simultaneously.

In the case of \ref{ex:mg-dx1-2x}, the spellout of \tsc{dx}\scsub{1} is successful in the syntactic structure in \ref{ex:mg-cp-dx2}.
This syntactic structure namely forms a constituent in the lexical tree in \ref{ex:mg-entry-e-rep}, which corresponds to the \tit{e}. As spellout has succeeded, the workspaces can be merged back together. The result is shown in \ref{ex:mg-spellout-proxp-inspec}.

\ex.\label{ex:mg-spellout-proxp-inspec}
\begin{adjustbox}{max width=0.9\textwidth}
\begin{forest} boom
  [\tsc{prox}P, s sep = 40mm
      [\tsc{prox}P,
       tikz={
       \node[label=below:\tit{e},
       draw,circle,
       scale=0.9,
       fit to=tree]{};
       }
          [\tsc{dx}\scsub{1}]
          [\tsc{c}P
              [\tsc{c}]
              [\tsc{ind}P
                  [\tsc{ind}]
                  [\tsc{an}]
              ]
          ]
      ]
      [\tsc{ind}P,
      tikz={
      \node[label=below:\tit{n},
      draw,circle,
      scale=0.95,
      fit to=tree]{};
      }
          [\tsc{ind}]
          [\tsc{an}P
              [\tsc{an}]
              [\tsc{cl}P
                  [\tsc{cl}]
                  [ΣP
                       [Σ]
                       [\tsc{ref}]
                  ]
              ]
          ]
      ]
  ]
\end{forest}
\end{adjustbox}

The next feature on the functional sequence is \tsc{dx}\scsub{2}. The derivation for \tsc{dx}\scsub{2} resembles the derivation of \tsc{dx}\scsub{1}.
The feature is merged with the existing syntactic structure, creating a \tsc{med}P.
This structure does not form a constituent in any of the lexical trees in the language's lexicon, and neither of the spellout driven movements leads to a successful spellout.
Backtracking leads splitting up the \tsc{prox}P from the the \tsc{ind}P.
The feature \tsc{dx}\scsub{2} is merged in both workspaces, so with \tsc{prox}P and and with \tsc{ind}P. The spellout of \tsc{dx}\scsub{2} is successful when it is combined with the \tsc{prox}P.
It namely forms a constituent in the lexical tree in \ref{ex:mg-entry-e-rep}, which corresponds to the \tit{e}.
The \tsc{med}P is spelled out as \tit{e}, and the \tsc{med}P is merged back to the existing syntactic structure, as shown in \ref{ex:mg-spellout-medp}.

\ex.\label{ex:mg-spellout-medp}
\begin{adjustbox}{max width=0.9\textwidth}
\begin{forest} boom
  [\tsc{med}P, s sep = 50mm
      [\tsc{med}P,
       tikz={
       \node[label=below:\tit{e},
       draw,circle,
       scale=0.95,
       fit to=tree]{};
       }
          [\tsc{dx}\scsub{2}]
          [\tsc{prox}P
              [\tsc{dx}\scsub{1}]
              [\tsc{c}P
                  [\tsc{c}]
                  [\tsc{ind}P
                      [\tsc{ind}]
                      [\tsc{an}]
                  ]
              ]
          ]
      ]
      [\tsc{ind}P,
      tikz={
      \node[label=below:\tit{n},
      draw,circle,
      scale=0.95,
      fit to=tree]{};
      }
          [\tsc{ind}]
          [\tsc{an}P
              [\tsc{an}]
              [\tsc{cl}P
                  [\tsc{cl}]
                  [ΣP
                       [Σ]
                       [\tsc{ref}]
                  ]
              ]
          ]
      ]
  ]
\end{forest}
\end{adjustbox}

The next feature on the functional sequence is \tsc{wh}. The derivation for \tsc{wh} resembles the derivation of \tsc{dx}\scsub{1}.
The feature is merged with the existing syntactic structure, creating a \tsc{wh}P.
This structure does not form a constituent in any of the lexical trees in the language's lexicon, and neither of the spellout driven movements leads to a successful spellout. Backtracking also does not lead to a successful spellout.
Therefore, in a second workspace, the feature \tsc{wh} is merged with the feature \tsc{dx}\scsub{2} (the previous syntactic feature on the functional sequence) into a \tsc{wh}P.
This syntactic structure forms a constituent in the lexical tree in \ref{ex:mg-entry-w-rep}, repeated from \ref{ex:mg-entry-w}, which corresponds to the \tit{w}.

\ex.\label{ex:mg-entry-w-rep}
\begin{forest} boom
  [\tsc{rel}P
      [\tsc{rel}]
      [\tsc{wh}P
          [\tsc{wh}]
          [\tsc{dx}\scsub{2}]
      ]
  ]
  {\draw (.east) node[right]{⇔ \tit{w}}; }
\end{forest}

Therefore, the \tsc{wh}P is spelled out as \tit{w}, and \tsc{wh}P is merged back to the existing syntactic structure, as shown in \ref{ex:mg-spellout-whp}.

\ex.\label{ex:mg-spellout-whp}
\begin{adjustbox}{max width=0.9\textwidth}
\begin{forest} boom
  [\tsc{wh}P, s sep = 20mm
      [\tsc{wh}P,
      tikz={
      \node[label=below:\tit{w},
      draw,circle,
      scale=1,
      fit to=tree]{};
      }
          [\tsc{wh}]
          [\tsc{dx\scsub{2}}]
      ]
      [\tsc{med}P, s sep = 50mm
          [\tsc{med}P,
           tikz={
           \node[label=below:\tit{e},
           draw,circle,
           scale=0.95,
           fit to=tree]{};
           }
              [\tsc{dx}\scsub{2}]
              [\tsc{prox}P
                  [\tsc{dx}\scsub{1}]
                  [\tsc{c}P
                      [\tsc{c}]
                      [\tsc{ind}P
                          [\tsc{ind}]
                          [\tsc{an}]
                      ]
                  ]
              ]
          ]
          [\tsc{ind}P,
          tikz={
          \node[label=below:\tit{n},
          draw,circle,
          scale=0.95,
          fit to=tree]{};
          }
              [\tsc{ind}]
              [\tsc{an}P
                  [\tsc{an}]
                  [\tsc{cl}P
                      [\tsc{cl}]
                      [ΣP
                           [Σ]
                           [\tsc{ref}]
                      ]
                  ]
              ]
          ]
      ]
  ]
\end{forest}
\end{adjustbox}

The next feature on the functional sequence is \tsc{rel}. The derivation for \tsc{rel} resembles the derivation of \tsc{dx}\scsub{1} and \tsc{dx}\scsub{2}.
The feature is merged with the existing syntactic structure, creating a \tsc{rel}P.
This structure does not form a constituent in any of the lexical trees in the language's lexicon, and neither of the spellout driven movements leads to a successful spellout.
Backtracking leads splitting up the \tsc{wh}P from the (higher) \tsc{med}P (which contains the lower \tsc{med}P and the \tsc{ind}P).
The feature \tsc{rel} is merged in both workspaces, so with \tsc{wh}P and and with \tsc{med}P. The spellout of \tsc{rel} is successful when it is combined with the \tsc{wh}P.
It namely forms a constituent in the lexical tree in \ref{ex:mg-entry-w-rep}, which corresponds to the \tit{w}.
The \tsc{rel}P is spelled out as \tit{w}, and the \tsc{rel}P is merged back to the existing syntactic structure, as shown in \ref{ex:mg-spellout-relp}.

\ex.\label{ex:mg-spellout-relp}
\begin{adjustbox}{max width=0.9\textwidth}
\begin{forest} boom
  [\tsc{rel}P, s sep = 20mm
      [\tsc{rel}P,
      tikz={
      \node[label=below:\tit{w},
      draw,circle,
      scale=0.9,
      fit to=tree]{};
      }
          [\tsc{rel}]
          [\tsc{wh}P
              [\tsc{wh}]
              [\tsc{dx\scsub{2}}]
          ]
      ]
      [\tsc{med}P, s sep = 50mm
          [\tsc{med}P,
           tikz={
           \node[label=below:\tit{e},
           draw,circle,
           scale=0.95,
           fit to=tree]{};
           }
              [\tsc{dx}\scsub{2}]
              [\tsc{prox}P
                  [\tsc{dx}\scsub{1}]
                  [\tsc{c}P
                      [\tsc{c}]
                      [\tsc{ind}P
                          [\tsc{ind}]
                          [\tsc{an}]
                      ]
                  ]
              ]
          ]
          [\tsc{ind}P,
          tikz={
          \node[label=below:\tit{n},
          draw,circle,
          scale=0.95,
          fit to=tree]{};
          }
              [\tsc{ind}]
              [\tsc{an}P
                  [\tsc{an}]
                  [\tsc{cl}P
                      [\tsc{cl}]
                      [ΣP
                           [Σ]
                           [\tsc{ref}]
                      ]
                  ]
              ]
          ]
      ]
  ]
\end{forest}
\end{adjustbox}

The next feature on the functional sequence is \tsc{f}1. This feature should somehow end up merging with \tsc{ind}P, because it forms a constituent in the lexical tree in \ref{ex:mg-entry-n-rep}, which corresponds to the \tit{n}. This is achieved via two instances of Backtracking in which phrases are split up. I go through the derivation step by step.

The feature \tsc{f}1 is merged with the existing syntactic structure, creating a \tsc{nom}P.
This structure does not form a constituent in any of the lexical trees in the language's lexicon, and neither of the spellout driven movements leads to a successful spellout.
Backtracking leads splitting up the \tsc{rel}P from the (higher) \tsc{med}P (which contains the lower \tsc{med}P and the \tsc{ind}P).
The feature \tsc{f}1 is merged in both workspaces, so with the \tsc{rel}P and and with the \tsc{med}P. None of these phrases form a constituent in any of the lexical trees in the language's lexicon, and neither of the spellout driven movements leads to a successful spellout.

Further Backtracking leads to splitting up the \tsc{med}P from the \tsc{ind}P.
The feature \tsc{f}1 is merged in both workspaces, so with the \tsc{med}P and and with the \tsc{ind}P. The spellout of \tsc{f}1 is successful when it is combined with the \tsc{ind}P.
It namely forms a constituent in the lexical tree in \ref{ex:mg-entry-n-rep}, which corresponds to the \tit{n}.
The \tsc{nom}P is spelled out as \tit{r}, and all constituents are merged back into the existing syntactic structure, as shown in \ref{ex:mg-spellout-rel-nom}.

\ex.\label{ex:mg-spellout-rel-nom}
\begin{adjustbox}{max width=0.9\textwidth}
\begin{forest} boom
  [\tsc{rel}P, s sep=20mm
      [\tsc{rel}P,
      tikz={
      \node[label=below:\tit{w},
      draw,circle,
      scale=0.9,
      fit to=tree]{};
      }
          [\tsc{rel}]
          [\tsc{wh}P
              [\tsc{wh}]
              [\tsc{dx}\scsub{2}]
          ]
      ]
      [\tsc{med}P, s sep=55mm
          [\tsc{med}P,
          tikz={
          \node[label=below:\tit{e},
          draw,circle,
          scale=0.95,
          fit to=tree]{};
          }
              [\tsc{dx}\scsub{2}]
              [\tsc{prox}P
                  [\tsc{dx}\scsub{1}]
                  [\tsc{c}P
                      [\tsc{c}]
                      [\tsc{ind}P
                          [\tsc{ind}]
                          [\tsc{an}]
                      ]
                  ]
              ]
          ]
          [\tsc{nom}P,
          tikz={
          \node[label=below:\tit{n},
          draw,circle,
          scale=0.95,
          fit to=tree]{};
          }
              [\ac{f}1]
              [\tsc{ind}P
                  [\tsc{ind}]
                  [\tsc{an}P
                      [\tsc{an}]
                      [\tsc{cl}P
                          [\tsc{cl}]
                          [ΣP
                              [Σ]
                              [\tsc{ref}]
                          ]
                      ]
                  ]
              ]
          ]
      ]
  ]
\end{forest}
\end{adjustbox}

For the accusative relative pronoun, the last feature is merged: the \tsc{f}2. The derivation for \tsc{f}2 resembles the derivation of \tsc{f}1. The feature is merged with the existing syntactic structure, creating a \tsc{acc}P.
This structure does not form a constituent in any of the lexical trees in the language's lexicon, and neither of the spellout driven movements leads to a successful spellout.
Backtracking leads splitting up the \tsc{rel}P from the (higher) \tsc{med}P (which contains the lower \tsc{med}P and the \tsc{nom}P).
The feature \tsc{f}2 is merged in both workspaces, so with the \tsc{wh}P and and with the \tsc{med}P. None of these phrases form a constituent in any of the lexical trees in the language's lexicon, and neither of the spellout driven movements leads to a successful spellout.

Further Backtracking leads to splitting up the \tsc{med}P from the \tsc{nom}P.
The feature \tsc{f}2 is merged in both workspaces, so with the \tsc{med}P and and with the \tsc{nom}P. The spellout of \tsc{f}2 is successful when it is combined with the \tsc{nom}P.
It namely forms a constituent in the lexical tree in \ref{ex:mg-entry-n-rep}, which corresponds to the \tit{n}. The \tsc{acc}P is spelled out as \tit{n}, and all constituents are merged back into the existing syntactic structure, as shown in \ref{ex:mg-spellout-rel-acc}.

\ex.\label{ex:mg-spellout-rel-acc}
\begin{adjustbox}{max width=0.9\textwidth}
\begin{forest} boom
  [\tsc{rel}P, s sep=20mm
      [\tsc{rel}P,
      tikz={
      \node[label=below:\tit{w},
      draw,circle,
      scale=0.9,
      fit to=tree]{};
      }
          [\tsc{rel}]
          [\tsc{wh}P
              [\tsc{wh}]
              [\tsc{dx}\scsub{2}]
          ]
      ]
      [\tsc{med}P, s sep=60mm
          [\tsc{med}P,
          tikz={
          \node[label=below:\tit{e},
          draw,circle,
          scale=1,
          fit to=tree]{};
          }
              [\tsc{dx}\scsub{2}]
              [\tsc{prox}P
                  [\tsc{dx}\scsub{1}]
                  [\tsc{c}P
                      [\tsc{c}]
                      [\tsc{ind}P
                          [\tsc{ind}]
                          [\tsc{an}]
                      ]
                  ]
              ]
          ]
          [\tsc{acc}P,
          tikz={
          \node[label=below:\tit{n},
          draw,circle,
          scale=0.95,
          fit to=tree]{};
          }
              [\ac{f}2]
              [\tsc{nom}P
                  [\ac{f}1]
                  [\tsc{ind}P
                      [\tsc{ind}]
                      [\tsc{an}P
                          [\tsc{an}]
                          [\tsc{cl}P
                              [\tsc{cl}]
                              [ΣP
                                  [Σ]
                                  [\tsc{ref}]
                              ]
                          ]
                      ]
                  ]
              ]
          ]
      ]
  ]
\end{forest}
\end{adjustbox}

For the dative relative pronoun, one more feature is merged: the \tsc{f}3. The derivation for \tsc{f}3 resembles the derivation of \tsc{f}1 and \tsc{f}2. The feature is merged with the existing syntactic structure, creating a \tsc{dat}P.
This structure does not form a constituent in any of the lexical trees in the language's lexicon, and neither of the spellout driven movements leads to a successful spellout.
Backtracking leads splitting up the \tsc{rel}P from the (higher) \tsc{med}P (which contains the lower \tsc{med}P and the \tsc{acc}P).
The feature \tsc{f}3 is merged in both workspaces, so with the \tsc{wh}P and and with the \tsc{med}P. None of these phrases form a constituent in any of the lexical trees in the language's lexicon, and neither of the spellout driven movements leads to a successful spellout.

Further Backtracking leads to splitting up the \tsc{med}P from the \tsc{acc}P.
The feature \tsc{f}3 is merged in both workspaces, so with the \tsc{med}P and and with the \tsc{acc}P. The spellout of \tsc{f}3 is successful when it is combined with the \tsc{acc}P.
It namely forms a constituent in the lexical tree in \ref{ex:mg-entry-m-rep}, repeated from \ref{ex:mg-entry-m}, which corresponds to the \tit{m}.

\ex.\label{ex:mg-entry-m-rep}
\begin{forest} boom
  [\tsc{dat}P
      [\tsc{f}3]
      [\tsc{acc}P
          [\ac{f}2]
          [\tsc{nom}P
              [\ac{f}1]
              [\tsc{ind}P
                  [\tsc{ind}]
                  [\tsc{an}P
                      [\tsc{an}]
                      [\tsc{cl}P
                          [\tsc{cl}]
                          [ΣP
                              [Σ]
                              [\tsc{ref}]
                          ]
                      ]
                  ]
              ]
          ]
      ]
  ]
  {\draw (.east) node[right]{⇔ \tit{m}}; }
\end{forest}

The \tsc{acc}P is spelled out as \tit{m}, and all constituents are merged back into the existing syntactic structure, as shown in \ref{ex:mg-spellout-rel-dat}.

\ex.\label{ex:mg-spellout-rel-dat}
\begin{adjustbox}{max width=0.9\textwidth}
\begin{forest} boom
  [\tsc{rel}P, s sep=20mm
      [\tsc{rel}P,
      tikz={
      \node[label=below:\tit{w},
      draw,circle,
      scale=0.9,
      fit to=tree]{};
      }
          [\tsc{rel}]
          [\tsc{wh}P
              [\tsc{wh}]
              [\tsc{dx}\scsub{2}]
          ]
      ]
      [\tsc{med}P, s sep=60mm
          [\tsc{med}P,
          tikz={
          \node[label=below:\tit{e},
          draw,circle,
          scale=1,
          fit to=tree]{};
          }
              [\tsc{dx}\scsub{2}]
              [\tsc{prox}P
                  [\tsc{dx}\scsub{1}]
                  [\tsc{c}P
                      [\tsc{c}]
                      [\tsc{ind}P
                          [\tsc{ind}]
                          [\tsc{an}]
                      ]
                  ]
              ]
          ]
          [\tsc{dat}P,
          tikz={
          \node[label=below:\tit{m},
          draw,circle,
          scale=0.95,
          fit to=tree]{};
          }
              [\tsc{f}3]
              [\tsc{acc}P
                  [\ac{f}2]
                  [\tsc{nom}P
                      [\ac{f}1]
                      [\tsc{ind}P
                          [\tsc{ind}]
                          [\tsc{an}P
                              [\tsc{an}]
                              [\tsc{cl}P
                                  [\tsc{cl}]
                                  [ΣP
                                      [Σ]
                                      [\tsc{ref}]
                                  ]
                              ]
                          ]
                      ]
                  ]
              ]
          ]
      ]
  ]
\end{forest}
\end{adjustbox}

To summarize, I decomposed the relative pronoun into the three morphemes \tit{w}, \tit{e} and the final consonant (\tit{n} and \tit{m}). I showed which features each of the morphemes spells out, and in which constituents the features are combined. It is these constituency that determine whether the relative pronoun can delete the light head or not.

\section{The Modern German (extra) light head}\label{sec:light-mg}

In Chapter \ref{ch:constituent-containment}, I argued that headless relatives are derived from light-headed relatives. The relative pronoun can delete the light head when the light head forms a constituent within the relative pronoun. In internal-only languages, the relative pronoun can delete the light head as long as the external case is not more complex than the internal case. In internal-only languages, features of the relative pronoun and light head are spelled out in such a way that they form the constituency shown in Figure \ref{fig:rel-lh-intonly-simple-rep}.

\begin{figure}[htbp]
  \center
  \begin{tabular}[b]{ccc}
      \toprule
      light head & & relative pronoun \\
      \cmidrule(lr){1-1} \cmidrule(lr){3-3}
      \begin{forest} boom
      [\tsc{k}P,
          [\tsc{k}]
          [ϕP
              [\phantom{x}ϕ\phantom{x}, roof, baseline]
          ]
      ]
      \end{forest}
      & \phantom{x} &
    \begin{forest} boom
      [\tsc{rel}P
          [\tsc{rel}]
          [\tsc{k}P
              [\tsc{k}]
              [ϕP
                  [\phantom{x}ϕ\phantom{x}, roof, baseline]
              ]
          ]
      ]
    \end{forest}\\
      \bottomrule
  \end{tabular}
   \caption {\tsc{lh} and \tsc{rp} in the internal-only type}
  \label{fig:rel-lh-intonly-simple-rep}
\end{figure}

In the previous section, I showed that relative pronouns in Modern German are constituently structured as in Figure \ref{fig:rel-lh-intonly-simple-rep}. I give a compact version of it in \ref{ex:mg-rp-rep}.

\ex.\label{ex:mg-rp-rep}
\begin{forest} boom
  [\tsc{rel}P
      [\tsc{rel}P,
      tikz={
      \node[
      draw,circle,
      scale=0.75,
      fit to=tree]{};
      }
          [\phantom{xxx}, roof]
      ]
      [\tsc{med}P, s sep = 15mm
          [\tsc{med}P,
          tikz={
          \node[
          draw,circle,
          scale=0.75,
          fit to=tree]{};
          }
              [\phantom{xxx}, roof]
          ]
          [\tsc{k}P,
          tikz={
          \node[
          draw,circle,
          scale=0.75,
          fit to=tree]{};
          }
              [\tsc{k}]
              [\tsc{ind}P
                  [\phantom{xxx}, roof]
              ]
          ]
      ]
  ]
\end{forest}

In this section, I show that light heads in Modern German are constituently structured as in Figure \ref{fig:rel-lh-intonly-simple-rep}. The light head in Modern German forms a constituent within the relative pronoun. I give the structure of the Modern German light head in \ref{ex:mg-lh}.

\ex.\label{ex:mg-lh}
\begin{forest} boom
    [\tsc{k}P,
    tikz={
    \node[label=below:\tit{r/n},
    draw,circle,
    scale=0.95,
    fit to=tree]{};
    }
        [\tsc{k}]
        [\tsc{ind}P
            [\tsc{ind}]
            [\tsc{an}P
                [\tsc{an}]
                [\tsc{cl}P
                    [\tsc{cl}]
                    [ΣP
                        [Σ]
                        [\tsc{ref}]
                    ]
                ]
            ]
        ]
    ]
\end{forest}

Before I dive into the feature content of the light head, I first need to identify it, as it does not surface in headless relatives.
I consider two kinds of light-headed relatives as the source of the headless relative.
First, the deletion of the light head is optional, and the light-headed relative is derived from an existing light-headed relative.
Second, the deletion of the light head is obligatory, and the light-headed relative is derived from a light-headed relative that does not surfaces in Modern German.
I consider the first option first, and I give two reasons against it.
However, to identify the exact input structure, I take the light head from the existing light-headed relative as a point of departure, and I modify it in such a way that it is appropriate as a light head for a headless relative.

I give an example of a Modern German light-headed relative in \ref{ex:mg-den-wen}.\footnote{
Modern German also has another light-headed relative, in which the relative pronoun is the \tsc{d}-pronoun. I give an example in \ref{ex:mg-den-den}.

\exg. Jan umarmt den \tbf{den} \tbf{er} \tbf{mag}.\\
Jan hugs \tsc{d}.\tsc{m}.\tsc{sg}.\tsc{acc} \tsc{rp}.\tsc{m}.\tsc{sg}.\tsc{acc} he likes\\
`Jan hugs the man that he likes.'\label{ex:mg-den-den}

This relative pronoun generally appears in headed relatives, shown in \ref{ex:mg-den-headed}.

\exg. Jan umarmt den Mann \tbf{den} \tbf{er} \tbf{mag}.\\
Jan hugs \tsc{d}.\tsc{m}.\tsc{sg}.\tsc{acc} man \tsc{rp}.\tsc{m}.\tsc{sg}.\tsc{acc} he likes\\
`Jan hugs the man that he likes.'\label{ex:mg-den-headed}

I directly exclude the possibility that Modern German headless relatives are derived from these light-headed relatives, because they appear with the incorrect relative pronoun.
}

\exg. Jan umarmt den \tbf{wen} \tbf{er} \tbf{mag}.\\
Jan hugs \tsc{dem}.\tsc{m}.\tsc{sg}.\tsc{acc} \tsc{rp}.\tsc{an}.\tsc{acc} he likes\\
`Jan hugs the man that he likes.'\label{ex:mg-den-wen}

In \ref{ex:mg-den-wen}, the relative pronoun is the \tsc{wh}-pronoun \tit{wen} `\tsc{rp}.\tsc{an}.\tsc{acc}', and the light head is the \tsc{d}-pronoun \tit{den} `\tsc{dem}.\tsc{m}.\tsc{sg}.\tsc{acc}'. For easy reference, I call this light-headed relative the \tit{den}-\tit{wen} relative.

One hypothesis is that the demonstrative \tit{den} `\tsc{dem}.\tsc{m}.\tsc{sg}.\tsc{acc}' is deleted from the light-headed relative in \ref{ex:mg-den-wen} and that the headless relative in \ref{ex:mg-wen} remains.\footnote{
This is exactly what \citet{hanink2018} argues for. She claims that the feature content of the light head matches the feature content of the relative pronoun. Therefore, the light head is by default deleted. Only if the light head carries an extra focus feature it surfaces.
}
For easy reference, I call this headless relative the \tit{wen} relative.

\exg. Jan umarmt \tbf{wen} \tbf{er} \tbf{mag}.\\
Jan hugs \tsc{rp}.\tsc{an}.\tsc{acc} he likes\\
`Jan hugs who he likes.'\label{ex:mg-wen}

I give two arguments against this hypothesis. First, in headless relatives the morpheme \tit{auch immer} `ever' can appear, as shown in \ref{ex:mg-wh-for-headless-ever}.

\exg. Jan unarmt \tbf{wen} {\tbf{auch} \tbf{immer}} \tbf{er} \tbf{mag}.\\
Jan hugs \tsc{rp}.\tsc{an}.\tsc{acc} ever he likes\\
`Jan hugs whoever he likes.'\label{ex:mg-wh-for-headless-ever}

Light-headed relatives do not allow this morpheme to be inserted, illustrated in \ref{ex:mg-wh-for-headed-ever}.

\exg. *Jan unarmt den \tbf{wen} {\tbf{auch} \tbf{immer}} \tbf{er} \tbf{mag}.\\
Jan hugs \tsc{dem}.\tsc{m}.\tsc{sg}.\tsc{acc} \tsc{rp}.\tsc{an}.\tsc{acc} ever he likes\\
`Jan hugs him whoever he likes.'\label{ex:mg-wh-for-headed-ever}%source?

I assume that the headless relative is not derived from an ungrammatical structure.\footnote{
I am aware that such an analysis is common for sluicing.
}

The second argument against the \tit{den}-\tit{wen} relative being the source of the \tit{wen} relative comes from the interpretation differences between the two. Broadly speaking, the \tit{wen} relative has two interpretations (see \citealt{s̆imík2020} for a recent elaborate overview on the semantics of free relatives). The \tit{den}-\tit{wen} has only one of them. I show this schematically in Table \ref{tbl:mg-interpretations}.

\begin{table}[htbp]
  \center
  \caption{Intepretations of \tit{wen} and \tit{den}-\tit{wen} relatives}
\begin{tabular}{ccc}
  \toprule
                & \tit{wen} & \tit{den}-\tit{wen} \\
                \cmidrule{2-3}
definite-like   & ✔         & ✔                   \\
universal-like  & ✔         & *                   \\
\bottomrule
\end{tabular}
\label{tbl:mg-interpretations}
\end{table}

The first interpretation of the \tit{wen} relative is a definite-like one. This interpretation corresponds to a definite description: Jan hugs the person that he likes. Consider the context which facilitates a definite-interpretation and the repeated \tit{den}-\tit{wen} and \tit{wen} relative in \ref{ex:mg-context-def}.

\ex.
\a. Context: Yesterday Jan met with two friends. He likes one of them. The other one he does not like so much.\label{ex:mg-context-def}
\bg. Jan umarmt den \tbf{wen} \tbf{er} \tbf{mag}.\\
Jan hugs \tsc{dem}.\tsc{m}.\tsc{sg}.\tsc{acc} \tsc{rp}.\tsc{an}.\tsc{acc} he likes\\
`Jan hugs who he likes.'
\bg. Jan umarmt \tbf{wen} \tbf{er} \tbf{mag}.\\
Jan hugs \tsc{rp}.\tsc{an}.\tsc{acc} he likes\\
`Jan hugs who he likes.'

The interpretation is available for the \tit{wen} relative and for the \tit{den}-\tit{wen} relative.

The second interpretation of the \tit{wen} relative is a universal-like one. This interpretation corresponds to a universal quantifier: Jan hugs everybody that he likes. Consider the context which facilitates a universal-interpretation and the repeated \tit{den}-\tit{wen} and \tit{wen} relative in \ref{ex:mg-context-univ}.

\ex.
\a. Jan has a general habit of hugging everybody that he likes.\label{ex:mg-context-univ}
\bg. \#Jan umarmt den \tbf{wen} \tbf{er} \tbf{mag}.\\
Jan hugs \tsc{dem}.\tsc{m}.\tsc{sg}.\tsc{acc} \tsc{rp}.\tsc{an}.\tsc{acc} he likes\\
`Jan hugs who he likes.'
\bg. Jan umarmt \tbf{wen} \tbf{er} \tbf{mag}.\\
Jan hugs \tsc{rp}.\tsc{an}.\tsc{acc} he likes\\
`Jan hugs who he likes.'

This interpretation is available for the \tit{wen} relative, but not for the \tit{den}-\tit{wen} relative.

There are some indications that the universal-like interpretation of headless relatives is the main interpretation that should be accounted for.
First, informants have reported to me that headless relatives with case mismatches become more acceptable in the universal-like interpretation compared to the definite-like interpretation.
Second, \pgcitet{s̆imík2020}{4} notes that some languages do not easily allow for the definite-like interpretation of headless relatives with an \tit{ever}-morpheme. There is no language documented that does not allow for the universal-like interpretation, but does allow the definite-like interpretation.

In sum, there are two arguments against the \tit{den}-\tit{wen} relative being the source of the \tit{wen} relative. In what follows, I show how the presence of \tit{den} leads to having only the definite-like interpretation. I suggest that the problem lies in the feature content of the light head \tit{den}. I point out how the feature content should be modified such that it is a suitable light head.

The light head in the \tit{den}-\tit{wen} relative is a demonstrative. A demonstrative refers back to a linguistic or extra-linguistic antecedent. Consider the context in \ref{ex:mg-context-def} again. The demonstrative \tit{den} in the \tit{den}-\tit{wen} relative refers back to the friend of Jan that he likes, and the construction is grammatical. Now consider the context in \ref{ex:mg-context-univ} again. In this case, there is no antecedent for the demonstrative \tit{den} to refer back to, and the structure is infelicitous.

I zoom in on the internal structure of the demonstrative \tit{den} to investigate what it is about the demonstrative that forces the definite-like interpretation. The demonstrative consists of the three morphemes \tit{d}, \tit{e} and \tit{n}. Two of its morphemes are identical to the \tsc{wh}-relative pronoun: (1) \tit{n}, which spells out pronominal, number, gender and case features, and (2) the \tit{e} which spells out deictic features. One morpheme differs: the \tit{d}, which establishes a definite reference. The two morphemes that force the definite-interpretation are the \tit{d} and the \tit{e}. The \tit{e} establishes a reference, and the \tit{d} makes this reference a definite one.

So far, I established that the \tit{den}-\tit{wen} relative cannot be the source from which the headless relative is derived. However, there must be some structure that is the source. I propose that this is a light-headed relative in which the head is even lighter than the head in the \tit{den}-\tit{wen} relative: it is an extra light head.

I propose that the extra light head is the element that is left once the morphemes \tit{d} and \tit{e} are absent. This is the morpheme that is the final consonant of the relative pronoun. I give the extra light-headed relative from which the \tit{wen}-relative is derived in \ref{ex:mg-real-base}. The brackets around the light head indicate that it is obligatorily deleted.\footnote{
The light head and the extra light head I discuss resemble the strong and weak definite in \citet{schwarz2009}, at least morphologically (although my light head is always obligatorily deleted). \posscitet{schwarz2009} strong definite is anaphoric in nature, and the weak definite encodes uniqueness. I give an example of a strong definite in \ref{ex:mg-florian-strong}. The strong definite is \tit{dem} that precedes \tit{Freund} `friend'. It refers back to the linguistic antecedent \tit{einen Freund} `a friend'.

\exg. Hans hat heute einen Freund zum Essen mit nach Hause gebracht. Er hat uns vorher ein Foto von dem Freund gezeigt.\\
Hans has today a friend {to the} dinner with to home brought he has us beforehand a photo of the\scsub{strong} friend shown\\
`Hans brought a friend home for dinner today. He had shown us a photo of the friend beforehand.'\label{ex:mg-florian-strong}

Weak definites are used when situational uniqueness is involved. This uniqueness can be global or within a restricted domain. I give two examples in \ref{ex:mg-florian-weak}. In \ref{ex:mg-florian-weak-hund}, the dog is unique in this specific situation of the break-in. In \ref{ex:mg-florian-weak-mond}, the moon is unique for us people on the planet.

\ex.\label{ex:mg-florian-weak}
\ag. Der Einbrecher ist {zum Glück} vom Hund verjagt worden.\\
the burglar is luckily {by the\scsub{weak}} dog {chased away} been\\
`Luckily, the burglar was chased away by the dog.'\label{ex:mg-florian-weak-hund}
\bg. Armstrong flog als erster zum Mond.\\
Armstrong flew as {first one} {to the\scsub{weak}} moon\\
`Armstrong was the first one to fly to the moon.' \flushfill{Modern German, \pgcitealt{schwarz2009}{40}}\label{ex:mg-florian-weak-mond}

The meaning of \posscitet{schwarz2009} strong definite seems similar to the meaning of the light head in the \tit{den}-\tit{wen} relative.
I do not see right away how the extra light head in headless relatives could encode uniqueness. One possibility is that the feature content of his and my form differs slightly after all. Another possibility is that the fact that his form combines with a preposition and an overt nouns leads to a change in interpretation.
}

\exg. Jan umarmt [n] \tbf{wen} \tbf{er} \tbf{mag}.\\
Jan hugs \tsc{lh}.\tsc{an}.\tsc{acc} \tsc{rp}.\tsc{an}.\tsc{acc} he likes\\
`Jan hugs who he likes.'\label{ex:mg-real-base}

In the remainder of this section, I discuss the two extra light heads that I compare the constituents of in Section \ref{sec:comparing-mg}. The are the accusative animate and the dative animate, shown in \ref{ex:mg-lhs}.

\ex.\label{ex:mg-lhs}
\ag. n\\
 \tsc{lh}.\tsc{an}.\tsc{acc}\\
\bg. m\\
 \tsc{lh}.\tsc{an}.\tsc{dat}\\

In Chapter \ref{ch:constituent-containment}, I suggested that the relative pronoun contains at least one features more than the extra light head. In my proposal, it is actually five features, namely \tsc{wh}, \tsc{rel}, \tsc{dx}\scsub{1}, \tsc{dx}\scsub{2} and \tsc{c}. This leaves the functional for the extra light head as shown in \ref{ex:fseq-elh}.

\ex.\label{ex:fseq-elh}
\begin{forest} boom
  [\tsc{k}P
      [\tsc{k}]
      [\tsc{ind}P
          [\tsc{ind}]
          [\tsc{an}P
              [\tsc{an}]
              [\tsc{cl}P
                  [\tsc{cl}]
                  [ΣP
                      [Σ]
                      [\tsc{ref}]
                  ]
              ]
          ]
      ]
  ]
\end{forest}

It contains the pronominal features \tsc{ref} and Σ, the gender features \tsc{cl} and \tsc{an}, the number feature \tsc{ind} and case features \tsc{k}.

The two lexical entries that are required to spell these extra light heads out are the final consonants I introduced the lexical entries for in Section \ref{sec:mg-rel}. I repeat them from \ref{ex:mg-entries-nm} in \ref{ex:mg-entries-nm-rep}.

\ex.\label{ex:mg-entries-nm-rep}
\a.\label{ex:mg-entry-n-rep1}
 \begin{forest} boom
   [\tsc{acc}P
       [\ac{f}2]
       [\tsc{nom}P
           [\ac{f}1]
           [\tsc{ind}P
               [\tsc{ind}]
               [\tsc{an}P
                   [\tsc{an}]
                   [\tsc{cl}P
                       [\tsc{cl}]
                       [ΣP
                           [Σ]
                           [\tsc{ref}]
                       ]
                   ]
               ]
           ]
       ]
   ]
   {\draw (.east) node[right]{⇔ \tit{n}}; }
 \end{forest}
\b.\label{ex:mg-entry-m-rep1}
 \begin{forest} boom
   [\tsc{dat}P
       [\ac{f}3]
       [\tsc{acc}P
           [\ac{f}2]
           [\tsc{nom}P
               [\ac{f}1]
               [\tsc{ind}P
                   [\tsc{ind}]
                   [\tsc{an}P
                       [\tsc{an}]
                       [\tsc{cl}P
                           [\tsc{cl}]
                           [ΣP
                               [Σ]
                               [\tsc{ref}]
                           ]
                       ]
                   ]
               ]
           ]
       ]
   ]
   {\draw (.east) node[right]{⇔ \tit{m}}; }
 \end{forest}

The derivations of the extra light heads are straight-forward ones. The features are merged one by one, and after each new phrase is created, it is spelled out as a whole. I still go through them step by step.

First, the features \tsc{ref} and Σ are merged, and the ΣP is created.
The syntactic structure forms a constituent in the lexical tree in \ref{ex:mg-entry-n-rep1}.
Therefore, the ΣP is spelled out as \tit{n}.
Then, the feature \tsc{cl} is merged, and the \tsc{cl}P is created.
The syntactic structure forms a constituent in the lexical tree in \ref{ex:mg-entry-n-rep1}.
Therefore, the \tsc{cl}P is spelled out as \tit{n}.
Exactly the same happens for the features \tsc{an}, \tsc{ind} and \tsc{f}1 and
They are merged, they form a constituent in the lexical tree in \ref{ex:mg-entry-n-rep1}, and they are spelled out as \tit{n}.

The last feature that is merged for the accusative extra light head is the \tsc{f}2.
It is merged, and the \tsc{acc}P is created.
The syntactic structure forms a constituent in the lexical tree in \ref{ex:mg-entry-n-rep1}.
Therefore, the \tsc{acc}P is spelled out as \tit{n}, as shown in \ref{ex:mg-elh-acc}.

\ex. \begin{forest} boom
    [\tsc{acc}P,
    tikz={
    \node[label=below:\tit{n},
    draw,circle,
    scale=0.95,
    fit to=tree]{};
    }
        [\tsc{f}2]
        [\tsc{nom}P
            [\ac{f}1]
            [\tsc{ind}P
                [\tsc{ind}]
                [\tsc{an}P
                    [\tsc{an}]
                    [\tsc{cl}P
                        [\tsc{cl}]
                        [ΣP
                            [Σ]
                            [\tsc{ref}]
                        ]
                    ]
                ]
            ]
        ]
    ]
\end{forest}
\label{ex:mg-elh-acc}

For the dative extra light head another feature is merged: the \tsc{f}3.
The feature \tsc{f}3 is merged, and the \tsc{dat}P is created.
The syntactic structure forms a constituent in the lexical tree in \ref{ex:mg-entry-m-rep1}.
Therefore, the \tsc{dat}P is spelled out as \tit{m}, as shown in \ref{ex:mg-elh-dat}.

\ex. \label{ex:mg-elh-dat}
\begin{forest} boom
[\tsc{dat}P,
tikz={
\node[label=below:\tit{m},
draw,circle,
scale=1,
fit to=tree]{};
}
    [\tsc{f}3]
    [\tsc{acc}P
        [\ac{f}2]
        [\tsc{nom}P
            [\ac{f}1]
            [\tsc{ind}P
                [\tsc{ind}]
                [\tsc{an}P
                    [\tsc{an}]
                    [\tsc{cl}P
                        [\tsc{cl}]
                        [ΣP
                            [Σ]
                            [\tsc{ref}]
                        ]
                    ]
                ]
            ]
        ]
    ]
]
\end{forest}


In sum, I argued that extra light heads consists of a single constituent. This constituent is also a constituent within the light head.

% At first sight it seems like \citet{fuss2014} discuss a exception to this claim, namely headless relatives with \tsc{d}-pronouns. However, they claim that these headless relatives are actually light-headed relatives in which one of two syncretic elements is deleted by haplology.


\section{Comparing Modern German constituents}\label{sec:comparing-mg}

In this section, I compare the constituents of extra light heads to those of relative pronouns in Modern German. I give three examples, in which the internal and external case vary.
I start with an example with matching cases: the internal and the external case are both accusative.
Then I give an example in which the internal case is more complex than the external case: the internal case is the dative and the external case is the accusative.
I end with an example in which the external case is more complex than the internal case: the internal case is the accusative and the external case is the dative.
In Modern German, a internal-only language, the first two example is grammatical. I derive this by showing that the relative pronoun can delete the light head as long as its case is not less complex. In these situations, the light head forms namely a constituent within the relative pronoun.

I start with the matching cases.
Consider the example in \ref{ex:mg-acc-acc-rep}, in which the internal accusative case competes against the external accusative case. The relative clause is marked in bold.
The internal case is accusative, as the predicate \tit{mögen} `to like' takes accusative objects. The relative pronoun \tit{wen} `\ac{rel}.\ac{an}.\ac{acc}' appears in the accusative case. This is the element that surfaces.
The external case is accusative as well, as the predicate \tit{einladen} `to invite' also takes accusative objects. The extra light head \tit{n} `\ac{elh}.\ac{an}.\ac{acc}' appears in the accusative case. It is placed between square brackets because it does not surface.

\exg. Ich {lade ein} [n], \tbf{wen} \tbf{auch} \tbf{Maria} \tbf{mag}.\\
 1\ac{sg}.\ac{nom} invite.\ac{pres}.1\ac{sg}\scsub{[acc]} \tsc{elh}.\ac{an}.\ac{acc} \tsc{rp}.\ac{an}.\ac{acc} Maria.\ac{nom} like.\ac{pres}.3\ac{sg}\scsub{[acc]}\\
 `I invite who Maria also likes.' \flushfill{Modern German, adapted from \pgcitealt{vogel2001}{344}}\label{ex:mg-acc-acc-rep}

In Figure \ref{fig:mg-int=ext}, I give the syntactic structure of the extra light head at the top and the syntactic structure of the relative pronoun at the bottom.

\begin{figure}[htbp]
  \center
  \begin{adjustbox}{max height=0.9\textheight}
  \begin{tabular}[b]{c}
        \toprule
        \tsc{acc} extra light head \tit{n}\\
        \cmidrule{1-1}
      \begin{forest} boom
        [\tsc{acc}P,
        tikz={
        \node[label=below:{\tit{n}},
        draw,circle,
        scale=0.8,
        fit to=tree]{};
        \node[draw,circle,
        dashed,
        scale=0.85,
        fill=DG,fill opacity=0.2,
        fit to=tree]{};
        }
            [\tsc{f}2]
            [\tsc{nom}P
                [\ac{f}1]
                [\tsc{ind}P
                    [\phantom{xxx}, roof]
                ]
            ]
        ]
      \end{forest}
      \\
      \toprule
      \tsc{acc} relative pronoun \tit{w-e-n}
      \\
      \cmidrule{1-1}
          \begin{forest} boom
          [\tsc{rel}P
              [\tsc{rel}P
                  [\phantom{x}\tit{w}\phantom{x}, roof]
              ]
              [\tsc{med}P
                  [\tsc{med}P
                      [\phantom{x}\tit{e}\phantom{x}, roof]
                  ]
                  [\tsc{acc}P,
                  tikz={
                  \node[label=below:{\tit{n}},
                  draw,circle,
                  scale=0.8,
                  fit to=tree]{};
                  \node[draw,circle,
                  dashed,
                  scale=0.85,
                  fit to=tree]{};
                  }
                      [\tsc{f}2]
                      [\tsc{nom}P
                          [\ac{f}1]
                          [\tsc{ind}P
                              [\phantom{xxx}, roof]
                          ]
                      ]
                  ]
              ]
          ]
        \end{forest}
        \\
      \bottomrule
  \end{tabular}
  \end{adjustbox}
  \caption {Modern German \tsc{ext}\scsub{acc} vs. \tsc{int}\scsub{acc} → \tit{wen}}
  \label{fig:mg-int=ext}
\end{figure}

The relative pronoun consists of three morphemes: \tit{w}, \tit{e} and \tit{n}.
The extra light head consists of a single morpheme: \tit{n}.
As usual, I circle the part of the structure that corresponds to a particular lexical entry, and I place the corresponding phonology under it.
I draw a dashed circle around each constituent that is a constituent in both the extra light head and the relative pronoun.

The extra light head consists of a single constituent: the \tsc{acc}P.
This \tsc{acc}P is also a constituent within the relative pronoun. Therefore, the relative pronoun can delete the extra light head. I signal the deletion of the extra light head by marking the content of its circle gray.

Consider the example in \ref{ex:mg-acc-dat-rep}, in which the internal dative case competes against the external accusative case. The relative clause is marked in bold.
The internal case is dative, as the predicate \tit{vertrauen} `to trust' takes dative objects. The relative pronoun \tit{wem} `\ac{rel}.\ac{an}.\ac{dat}' appears in the dative case. This is the element that surfaces.
The external case is accusative, as the predicate \tit{einladen} `to invite' takes accusative objects. The extra light head \tit{n} `\ac{elh}.\ac{an}.\ac{acc}' appears in the accusative case. It is placed between square brackets because it does not surface.

\exg. Ich {lade ein} [n], \tbf{wem} \tbf{auch} \tbf{Maria} \tbf{vertraut}.\\
1\ac{sg}.\ac{nom} invite.\ac{pres}.1\ac{sg}\scsub{[acc]} \tsc{elh}.\ac{an}.\ac{dat} \tsc{rp}.\ac{an}.\ac{dat} also Maria.\ac{nom} trust.\ac{pres}.3\ac{sg}\scsub{[dat]}\\
`I invite whoever Maria also trusts.' \flushfill{Modern German, adapted from \pgcitealt{vogel2001}{344}}\label{ex:mg-acc-dat-rep}

In Figure \ref{fig:mg-int-wins}, I give the syntactic structure of the extra light head at the top and the syntactic structure of the relative pronoun at the bottom.

\begin{figure}[htbp]
  \center
  \begin{adjustbox}{max height=0.9\textheight}
  \begin{tabular}[b]{c}
      \toprule
      \tsc{acc} extra light head \tit{n}
      \\
      \cmidrule{1-1}
      \begin{forest} boom
        [\tsc{acc}P,
        tikz={
        \node[label=below:{\tit{n}},
        draw,circle,
        scale=0.8,
        fit to=tree]{};
        \node[draw,circle,
        dashed,
        scale=0.85,
        fill=DG,fill opacity=0.2,
        fit to=tree]{};
        }
            [\tsc{f}2]
            [\tsc{nom}P
                [\ac{f}1]
                [\tsc{ind}P
                    [\phantom{xxx}, roof]
                ]
            ]
        ]
      \end{forest}
      \\
      \toprule
      \tsc{dat} relative pronoun \tit{w-e-m}
      \\
      \cmidrule{1-1}
          \begin{forest} boom
            [\tsc{rel}P
                [\tsc{rel}P
                    [\phantom{x}\tit{w}\phantom{x}, roof]
                ]
                [\tsc{med}P
                    [\tsc{med}P
                        [\phantom{x}\tit{e}\phantom{x}, roof]
                    ]
                    [\tsc{dat}P,
                    tikz={
                    \node[label=below:{\tit{m}},
                    draw,circle,
                    scale=0.85,
                    fit to=tree]{};
                    }
                        [\tsc{f}3]
                        [\tsc{acc}P,
                        tikz={
                        \node[draw,circle,
                        dashed,
                        scale=0.8,
                        fit to=tree]{};
                        }
                            [\tsc{f}2]
                            [\tsc{nom}P
                                [\ac{f}1]
                                [\tsc{ind}P
                                    [\phantom{xxx}, roof]
                                ]
                            ]
                        ]
                    ]
                ]
            ]
        \end{forest}
        \\
      \bottomrule
  \end{tabular}
  \end{adjustbox}
   \caption {Modern German \tsc{ext}\scsub{acc} vs. \tsc{int}\scsub{dat} → \tit{wem}}
  \label{fig:mg-int-wins}
\end{figure}

The relative pronoun consists of three morphemes: \tit{w}, \tit{e} and \tit{m}.
The extra light head consists of a single morpheme: \tit{n}.
As usual, I circle the part of the structure that corresponds to a particular lexical entry, and I place the corresponding phonology under it.
I draw a dashed circle around each constituent that is a constituent in both the extra light head and the relative pronoun.

The extra light head consists of a single constituent: the \tsc{acc}P.
This \tsc{acc}P is also a constituent within the relative pronoun. Therefore, the relative pronoun can delete the extra light head. I signal the deletion of the extra light head by marking the content of its circle gray.

Consider the examples in \ref{ex:mg-dat-acc-rep}, in which the internal accusative case competes against the external dative case. The relative clauses are marked in bold. It is not possible to make a grammatical headless relative in this situation.

The internal case is accusative, as the predicate \tit{mögen} `to like' takes accusative objects. The relative pronoun \tit{wen} `\ac{rel}.\ac{an}.\ac{acc}' appears in the accusative case.
The external case is dative, as the predicate \tit{vertrauen} `to trust' takes dative objects. The extra light head \tit{m} `\ac{elh}.\ac{an}.\ac{dat}' appears in the dative case.
\ref{ex:mg-dat-acc-rep-rp} is the variant of the sentence in which the extra light head is absent (indicated by the square brackets) and the relative pronoun surfaces, and it is ungrammatical.
\ref{ex:mg-dat-acc-rep-lh} is the variant of the sentence in which the relative pronoun is absent (indicated by the square brackets) and the extra light head surfaces, and it is ungrammatical too.

\ex.\label{ex:mg-dat-acc-rep}
\ag. *Ich vertraue [m], \tbf{wen} \tbf{auch} \tbf{Maria} \tbf{mag}.\\
1\ac{sg}.\ac{nom} trust.\ac{pres}.1\ac{sg}\scsub{[dat]} \tsc{elh}.\ac{an}.\ac{dat} \tsc{rp}.\ac{an}.\ac{acc} also Maria.\ac{nom} like.\ac{pres}.3\ac{sg}\scsub{[acc]}\\
`I trust whoever Maria also likes.' \flushfill{Modern German, adapted from \pgcitealt{vogel2001}{345}}\label{ex:mg-dat-acc-rep-rp}
\bg. *Ich vertraue m, [\tbf{wen}] \tbf{auch} \tbf{Maria} \tbf{mag}.\\
1\ac{sg}.\ac{nom} trust.\ac{pres}.1\ac{sg}\scsub{[dat]} \tsc{elh}.\ac{an}.\ac{dat} \tsc{rp}.\ac{an}.\ac{acc} also Maria.\ac{nom} like.\ac{pres}.3\ac{sg}\scsub{[acc]}\\
`I trust whoever Maria also likes.' \flushfill{Modern German, adapted from \pgcitealt{vogel2001}{345}}\label{ex:mg-dat-acc-rep-lh}

In Figure \ref{fig:mg-ext-wins}, I give the syntactic structure of the extra light head at the top and the syntactic structure of the relative pronoun at the bottom.

\begin{figure}[htbp]
  \center
  \begin{adjustbox}{max height=0.9\textheight}
  \begin{tabular}[b]{c}
      \toprule
      \tsc{dat} extra light head \tit{m}
      \\
      \cmidrule{1-1}
      \begin{forest} boom
        [\tsc{dat}P,
        tikz={
        \node[label=below:{\tit{m}},
        draw,circle,
        scale=0.85,
        fit to=tree]{};
        }
            [\tsc{f}3]
            [\tsc{acc}P,
            tikz={
            \node[draw,circle,
            dashed,
            scale=0.8,
            fit to=tree]{};
            }
                [\tsc{f}2]
                [\tsc{nom}P
                    [\ac{f}1]
                    [\tsc{ind}P
                        [\phantom{xxx}, roof]
                    ]
                ]
            ]
        ]
      \end{forest}
      \\
      \toprule
      \tsc{dat} relative pronoun \tit{w-e-n}
      \\
      \cmidrule{1-1}
          \begin{forest} boom
            [\tsc{rel}P
                [\tsc{rel}P
                    [\phantom{x}\tit{w}\phantom{x}, roof]
                ]
                [\tsc{med}P
                    [\tsc{med}P
                        [\phantom{x}\tit{e}\phantom{x}, roof]
                    ]
                    [\tsc{acc}P,
                    tikz={
                    \node[label=below:{\tit{n}},
                    draw,circle,
                    scale=0.8,
                    fit to=tree]{};
                    \node[draw,circle,
                    dashed,
                    scale=0.85,
                    fit to=tree]{};
                    }
                        [\tsc{f}2]
                        [\tsc{nom}P
                            [\ac{f}1]
                            [\tsc{ind}P
                                [\phantom{xxx}, roof]
                            ]
                        ]
                    ]
                ]
            ]
        \end{forest}
        \\
      \bottomrule
  \end{tabular}
  \end{adjustbox}
   \caption {Modern German \tsc{ext}\scsub{dat} vs. \tsc{int}\scsub{acc} ↛ \tit{m}/\tit{wen}}
  \label{fig:mg-ext-wins}
\end{figure}


The relative pronoun consists of three morphemes: \tit{w}, \tit{e} and \tit{n}.
The extra light head consists of a single morpheme: \tit{m}.
As usual, I circle the part of the structure that corresponds to a particular lexical entry, and I place the corresponding phonology under it.
I draw a dashed circle around each constituent that is a constituent in both the extra light head and the relative pronoun.

The extra light head consists of a single constituent: the \tsc{dat}P.
In this case, the relative pronoun does not contain this constituent. The relative pronoun only contains the \tsc{acc}P, and it lacks the \tsc{f}3 that makes a \tsc{dat}P. Since the weaker feature containment requirement is not met, the stronger constituent requirement cannot be met either.
The extra light head also does not contain all constituents or features that the relative pronoun contains, because it lacks the complete constituents \tsc{med}P and \tsc{rel}P.
Therefore, the relative pronoun cannot delete the extra light head, and the extra light head can also not delete the relative pronoun.


\section{Summary}
