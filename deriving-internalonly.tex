% !TEX root = thesis.tex

\chapter{Deriving the internal-only type}\label{ch:deriving-onlyinternal}

In Chapter \ref{ch:the-basic-idea}, I suggested that languages of the internal-only type have two lexical entries that spell out light heads and relative pronouns in the language: a portmanteau for phi and case features and a separate lexical entry that spells out the feature \tsc{rel}. This means that the internal syntax of light heads and relative pronouns looks as shown in Figure \ref{fig:rel-lh-intonly-simple}.

\begin{figure}[htbp]
  \center
  \begin{tabular}[b]{ccc}
      \toprule
      light head & & relative pronoun \\
      \cmidrule(lr){1-1} \cmidrule(lr){3-3}
      \begin{forest} boom
      [\tsc{k}P,
      tikz={
      \node[draw,circle,
      scale=0.85,
      fit to=tree]{};
      }
          [\tsc{k}]
          [ϕP
              [\phantom{xxx}, roof, baseline]
          ]
      ]
      \end{forest}
      & \phantom{x} &
    \begin{forest} boom
      [\tsc{rel}P, s sep=20mm
          [\tsc{rel}P,
          tikz={
          \node[draw,circle,
          scale=0.85,
          fit to=tree]{};
          }
              [\phantom{xxx}, roof, baseline]
          ]
          [\tsc{k}P,
          tikz={
          \node[draw,circle,
          scale=0.85,
          fit to=tree]{};
          }
              [\tsc{k}]
              [ϕP
                  [\phantom{xxx}, roof, baseline]
              ]
          ]
      ]
    \end{forest}\\
      \bottomrule
  \end{tabular}
   \caption {\tsc{lh} and \tsc{rp} in the internal-only type}
  \label{fig:rel-lh-intonly-simple}
\end{figure}

These lexical entries lead to the grammaticality pattern shown in Table \ref{tbl:overview-intonly}.

\begin{table}[htbp]
  \center
  \caption{Grammaticality in the internal-only type (repeated)}
    \begin{adjustbox}{max width=\textwidth}
      \begin{tabular}{cccccc}
        \toprule
      situation           & \multicolumn{2}{c}{lexical entries}       & containment         & deleted             & surfacing           \\
      \cmidrule(lr){1-1}    \cmidrule(lr){2-3}                          \cmidrule(lr){4-4}    \cmidrule(lr){5-5}    \cmidrule(lr){6-6}
                          & \tsc{lh}            & \tsc{rp}            &                     &                     &                     \\
                            \cmidrule(lr){2-2}    \cmidrule(lr){3-3}
      \tsc{k}\scsub{int} = \tsc{k}\scsub{ext}               &
      [\tsc{k}\scsub{1}[ϕ]]                                 &
      [\tsc{rel}], [\tsc{k}\scsub{1}[ϕ]]                    &
      structure & \tsc{lh} & \tsc{rp}\scsub{int}            \\
      \tsc{k}\scsub{int} > \tsc{k}\scsub{ext}               &
      [\tsc{k}\scsub{1}[ϕ]]                                 &
      [\tsc{rel}], [\tsc{k}\scsub{2}[\tsc{k}\scsub{1}[ϕ]]]  &
      structure & \tsc{lh} & \tsc{rp}\scsub{int}            \\
      \tsc{k}\scsub{int} < \tsc{k}\scsub{ext}               &
      [\tsc{k}\scsub{2}[\tsc{k}\scsub{1}[ϕ]]]               &
      [\tsc{rel}], [\tsc{k}\scsub{1}[ϕ]]                    &
      no & none & *                                         \\
      \bottomrule
      \end{tabular}
    \end{adjustbox}
    \label{tbl:overview-intonly}
\end{table}

Consider the first situation in which the internal and the external case match. The light head consists of a phi and case feature portmanteau. The relative pronoun consists of the same morpheme plus an additional morpheme that spells out the feature \tsc{rel}. The lexical entries create a syntactic structure such that the light head is structurally contained in the relative pronoun. Therefore, the light head can be deleted, and the relative pronoun surfaces, bearing the internal case.

Consider now the situation in which the internal case wins the case competition. The light head consists of a phi and case feature portmanteau. The relative pronoun consists of a phi and case feature portmanteau that contains at least one more case feature than the light head (\tsc{k}\scsub{2} in Table \ref{tbl:overview-intonly}) plus an additional morpheme that spells out the feature \tsc{rel}. The lexical entries create a syntactic structure such that the light head is structurally contained in the relative pronoun. Therefore, the light head can be deleted, and the relative pronoun surfaces, bearing the internal case.

Finally, consider the situation in which the external case would win the case competition. The relative pronoun consists of a phi and case feature portmanteau and an additional morpheme that spells out the feature \tsc{rel}. Compared to the relative pronoun, the light head lacks the morpheme that spells out \tsc{rel}, and it contains at least one more case feature (\tsc{k}\scsub{2} in Figure \ref{tbl:overview-intonly}). The lexical entries create a syntactic structure such that neither the light head nor the relative pronoun is structurally contained in the other element. Therefore, none of the elements can be deleted, and there is no headless relative construction possible.

In Chapter \ref{ch:typology}, I showed that Modern German is a language of the internal-only type. In this chapter, I show that Modern German light heads and relative pronouns have the type of internal syntax described in Figure \ref{fig:rel-lh-intonly-simple}. I give a compact version of the internal syntax of Modern German light heads and relative pronouns in Figure \ref{fig:rel-lh-mg}.

\begin{figure}[htbp]
  \center
  \begin{tabular}[b]{ccc}
      \toprule
      light head & & relative pronoun \\
      \cmidrule(lr){1-1} \cmidrule(lr){3-3}
      \begin{forest} boom
        [\tsc{k}P,
        tikz={
        \node[label=below:\tit{r/n/m},
        draw,circle,
        scale=0.75,
        fit to=tree]{};
        }
            [\tsc{k}]
            [ϕP
                [\phantom{xxx}, roof, baseline]
            ]
        ]
      \end{forest}
      & \phantom{x} &
      \begin{forest} boom
        [\tsc{rel}P, s sep=15mm
            [\tsc{rel}P,
            tikz={
            \node[label=below:\tit{we},
            draw,circle,
            scale=0.75,
            fit to=tree]{};
            }
                [\phantom{xxx}, roof]
            ]
            [\tsc{k}P,
            tikz={
            \node[label=below:\tit{r/n/m},
            draw,circle,
            scale=0.75,
            fit to=tree]{};
            }
                [\tsc{k}]
                [ϕP
                    [\phantom{xxx}, roof, baseline]
                ]
            ]
        ]
      \end{forest}\\
      \bottomrule
  \end{tabular}
   \caption {\tsc{lh} and \tsc{rp} in Modern German}
  \label{fig:rel-lh-mg}
\end{figure}

Consider the light head in Figure \ref{fig:rel-lh-mg}.
Light heads (i.e. phi and case features) in Modern German are spelled out by a single morpheme, indicated by the circle around the structure. They are spelled out as \tit{r}, \tit{n} or \tit{m}, depending on which case they realize.
Consider the relative pronoun in Figure \ref{fig:rel-lh-mg}.
Relative pronouns in Modern German consist of two morphemes: the constituent that forms the light head (i.e. phi and case features) and the \tsc{rel}P, again indicated by the circles. The constituent that forms the light head has the same spellout as in the light head (\tit{n} or \tit{m}), and the \tsc{rel}P is spelled out as \tit{we}.\footnote{
Besides \tit{wer} `who', Modern German also has other relative pronouns, such as \tit{der} `who', \tit{welcher} `which'. I only discuss \tit{wer} here, since it is the relative pronoun used in the headless relatives I discuss.
}\footnote{
The inanimate relative pronoun is \tit{was}. Later on in this section I discuss how the vowel in \tit{we} alternates in different genders.
}
Throughout this chapter, I discuss the exact feature content of light heads and relative pronouns, I give lexical entries for them, and I show how these lexical entries lead to the internal syntax shown in Figure \ref{fig:rel-lh-mg}.

The chapter is structured as follows.
First, I discuss the relative pronoun. I start by decomposing it into the two morphemes I showed in Figure \ref{fig:rel-lh-mg}. Then I show which features each of the morphemes corresponds to. I illustrate how different morphemes are combined into the internal syntax in Figure \ref{fig:rel-lh-mg}.
Then I discuss the light head. I argue that Modern German headless relatives are derived from a type of light-headed relative clause that does not surface in the language. I show that the light head corresponds to one of the morphemes of the relative pronoun (the \tsc{k}P in Figure \ref{fig:rel-lh-mg}).
Finally, I compare the internal syntax of the light head and the relative pronoun. I show that the light head can be deleted when the internal case matches the external case or when the internal case is more complex than the external case. When the external case is more complex, I show that none of the elements can be deleted.


\section{The Modern German relative pronoun}\label{sec:mg-rel}

In the introduction of this chapter, I suggested that the internal syntax of relative pronouns in Modern German looks as shown in \ref{ex:simple-intonly-rp}.

\ex.\label{ex:simple-intonly-rp}
\begin{forest} boom
  [\tsc{rel}P, s sep=15mm
      [\tsc{rel}P,
      tikz={
      \node[label=below:\tit{we},
      draw,circle,
      scale=0.75,
      fit to=tree]{};
      }
          [\phantom{xxx}, roof]
      ]
      [\tsc{k}P,
      tikz={
      \node[label=below:\tit{r/n/m},
      draw,circle,
      scale=0.75,
      fit to=tree]{};
      }
          [\tsc{k}]
          [ϕP
              [\phantom{xxx}, roof]
          ]
      ]
  ]
\end{forest}

In Chapter \ref{ch:the-basic-idea}, I suggested that relative pronouns consist of at least three features: \tsc{rel}, ϕ and \tsc{k}.
In this section, I show that the relative pronoun consists of more features than that.
Still, the crucial claim I made in Chapter \ref{ch:the-basic-idea} remains unchanged: internal-only languages (of which Modern German is an example) have a portmanteau for the features that correspond to phi and case features and a morpheme that spells out the features the light head does not contain.
I show the complete structure that I work towards in this section in \ref{ex:mg-rp}.\footnote{
The \tsc{k}P in this functional sequence is a placeholder for multiple case projections.
When the relative pronoun is the nominative, the \tsc{k}P consists of the feature \tsc{k}1, and it forms the \tsc{nom}P.
When the relative pronoun is the accusative, the \tsc{k}P consists of the features \tsc{k}1 and \tsc{k}2, and they form the \tsc{acc}P.
When the relative pronoun is the dative, the \tsc{k}P consists of the features \tsc{k}1, \tsc{k}2 and \tsc{k}3, and they form the \tsc{dat}P.
}

\ex.\label{ex:mg-rp}
\begin{adjustbox}{max width=0.9\textwidth}
\begin{forest} boom
  [\tsc{rel}P, s sep=40mm
      [\tsc{rel}P,
      tikz={
      \node[label=below:\tit{we},
      draw,circle,
      scale=0.95,
      fit to=tree]{};
      }
          [\tsc{rel}]
          [\tsc{wh}P
              [\tsc{wh}]
              [\#P
                  [\#]
                  [\tsc{an}]
              ]
          ]
      ]
      [\tsc{k}P,
      tikz={
      \node[label=below:\tit{r/n/m},
      draw,circle,
      scale=0.95,
      fit to=tree]{};
      }
          [\tsc{k}]
          [\#P
              [\#]
              [\tsc{an}P
                  [\tsc{an}]
                  [\tsc{cl}P
                      [\tsc{cl}]
                      [\tsc{ref}]
                  ]
              ]
          ]
      ]
  ]
\end{forest}
\end{adjustbox}

I discuss two relative pronouns: the animate accusative and the animate dative. These are the two forms that I compare the internal syntax of in Section \ref{sec:comparing-mg}.\footnote{
For reasons of space, I do not discuss the animate nominative \tit{wer} `\tsc{rp}.\tsc{an}.\tsc{nom}'. I assume its analysis is identical to the one I propose for \tit{wen} and \tit{wem}, except that \tit{wer} spells out fewer case features. I work out the proposal for \tit{wen} and \tit{wem} to be able to do a comparison between Modern German and Polish in which the relative pronouns spell out exactly the same feature content.
} I show them in \ref{ex:mg-rels}.

\ex.\label{ex:mg-rels}
\a. we-n `\tsc{rp}.\tsc{an}.\tsc{acc}'
\b. we-m `\tsc{rp}.\tsc{an}.\tsc{dat}'

I decompose the relative pronouns into two morphemes: the \tit{we} and the final consonant (\tit{n} or \tit{m}). For each morpheme, I discuss which features they spell out, and I give their lexical entries. In the next section, I show how I construct the relative pronouns by combining the separate morphemes.

I start with the final consonants: \tit{n} and \tit{m}.
These two morphemes correspond to what I called the phi and case feature portmanteau in Chapter \ref{ch:the-basic-idea} and the introduction to this chapter.
I argue that the phi features actually correspond to gender (or animateness) features, number features and pronominal features. Adding this all up, I claim that the final consonants correspond to number features, gender features, pronominal features and case features. Consider Table \ref{tbl:mg-paradigm-wh-rels}.

\begin{table}[htbp]
\center
\caption {Modern German relative pronouns in headless relatives \pgcitep{durrell2016}{5.3.3}}
\begin{tabular}{ccc}
\toprule
            & \ac{an} & \tsc{inan}\\
  \cmidrule{2-3}
  \ac{nom}  & we-r    & wa-s     \\
  \ac{acc}  & we-n    & wa-s     \\
  \ac{dat}  & we-m    & -        \\
\bottomrule
\end{tabular}
\label{tbl:mg-paradigm-wh-rels}
\end{table}

The final consonants change depending on animacy and case.\footnote{
The vowel also differs between animacy. I return to this point when I discuss the feature content of the \tit{we}.
}
The differing final consonant can be observed in several contexts besides relative pronouns. Table \ref{tbl:mg-dieser} gives an overview of the demonstrative \tit{dieser} `this' in Modern German in two numbers, three genders and three cases.\footnote{
Notice that the animate forms in Table \ref{tbl:mg-paradigm-wh-rels} are the masculine forms in Table \ref{tbl:mg-dieser} and that the inanimate forms in Table \ref{tbl:mg-paradigm-wh-rels} are the neuter forms in Table \ref{tbl:mg-dieser}. This is a pattern that appears more often.
}

\begin{table}[htbp]
\center
\caption {Modern German \tit{dieser} demonstratives \pgcitep{durrell2016}{5.1.2}}
 \begin{tabular}{ccccc}
 \toprule
             & \tsc{m}.\tsc{sg} & \tsc{n}.\tsc{sg} & \ac{f}.\tsc{sg} & \tsc{pl} \\
   \cmidrule{2-5}
   \ac{nom}  & diese-r        & diese-s         & diese          & diese   \\
   \ac{acc}  & diese-n        & diese-s         & diese          & diese   \\
   \ac{dat}  & diese-m        & diese-m         & diese-r        & diese-n \\
 \bottomrule
 \end{tabular}
 \label{tbl:mg-dieser}
\end{table}

Table \ref{tbl:mg-dieser} shows that the final consonant differs depending on gender, number and case. There is no vowel that differs between the different forms. I conclude from this that the consonant realizes features having to do with gender, number and case. In other words, the final consonant is a portmanteau that realizes gender, number and case features.

For number and gender, I adopt the features that are distinguished by \citet{harley2002} for pronouns in a crosslinguistic study with over 100 languages. The feature \tsc{cl} corresponds to a gender feature, which is inanimate or neuter if it is not combined with any other features. Combining \tsc{cl} with the feature \tsc{an} gives the animate or masculine gender.\footnote{
If the features \tsc{cl} and \tsc{an} are combined with the feature \tsc{fem}, it becomes the feminine gender.
}
The feature \ac{\#} corresponds to number, which is singular if it is not combined with any other features.

For case, I adopt the features of \citet{caha2009}, already introduced in Chapter \ref{ch:decomposition}. The feature \tsc{k}1 and \tsc{k}2 corresponds to the accusative, and the features \tsc{k}1, \tsc{k}2 and \tsc{k}3 correspond to the dative.

Having discussed the number, gender and case features, only the pronominal features remain. Another context in which the final consonants appear (besides their use in relative pronouns and demonstrative pronouns) is as pronouns in colloquial speech.\footnote{
The singular feminine dative \tit{r} and the plural dative \tit{n} cannot be easily used as pronouns.
} In \ref{ex:mg-weak}, I give examples of the masculine accusative singular and masculine dative singular.

\ex.\label{ex:mg-weak}
\ag. Ich wollt'n gestern schon anrufen.\\
 I {wanted 3\tsc{sg}.\tsc{m}.\tsc{acc}} yesterday already call\\
 `I already wanted to call him yesterday.'
\bg. Ich helf'm sein Fahrrad zu reparieren.\\
 I {help 3\tsc{sg}.\tsc{m}.\tsc{dat}} his bike to repare\\
 `I help him reparing his bike.'

This means that the forms also correspond to pronominal features.
I follow \citet{harley2002} who claim that all pronouns contain the feature \tsc{ref}, because they are referential expressions.\footnote{
To be more precise, the final consonants correspond to the weak pronoun in Modern German. I elaborate on this in Section \ref{sec:aside-mg-rp}.
}

I give the lexical entries for \tit{n} and \tit{m} in \ref{ex:mg-entries-nm}.
The \tit{n} is the accusative animate singular, so it spells out the features \tsc{ref}, \tsc{cl}, \tsc{an}, \#, \tsc{k}1 and \tsc{k}2. The \tit{m} is the dative animate singular, so it spells out the features that the \tit{n} spells out plus \tsc{k}3.

\ex.\label{ex:mg-entries-nm}
\a.\label{ex:mg-entry-n}
\begin{forest} boom
  [\tsc{acc}P
      [\tsc{k}2]
      [\tsc{nom}P
          [\tsc{k}1]
          [\#P
              [\#]
              [\tsc{an}P
                  [\tsc{an}]
                  [\tsc{cl}P
                      [\tsc{cl}]
                      [\tsc{ref}]
                  ]
              ]
          ]
      ]
  ]
  {\draw (.east) node[right]{⇔ \tit{n}}; }
\end{forest}
\b.\label{ex:mg-entry-m}
\begin{forest} boom
  [\tsc{dat}P
      [\tsc{k}3]
      [\tsc{acc}P
          [\tsc{k}2]
          [\tsc{nom}P
              [\tsc{k}1]
              [\#P
                  [\#]
                  [\tsc{an}P
                      [\tsc{an}]
                      [\tsc{cl}P
                          [\tsc{cl}]
                          [\tsc{ref}]
                      ]
                  ]
              ]
          ]
      ]
  ]
  {\draw (.east) node[right]{⇔ \tit{m}}; }
\end{forest}

Note that the ordering of the features here is not random. I motivate the ordering in Section \ref{sec:combining}.

I continue with the morpheme \tit{we}. This morpheme corresponds to what I called the \tsc{rel}-feature in Chapter \ref{ch:the-basic-idea} and in the introduction to this chapter. I argue that this morpheme actually spells out the operator features \tsc{wh} and \tsc{rel} and number and gender features.

Note here that number and gender features are also spelled out by the final consonants. I assume that they are spelled out twice within the relative pronoun. This does not mean that they are semantically present twice. Their double presence is purely due to spellout reasons. I return to this point in Section \ref{sec:combining}.

Consider Table \ref{tbl:mg-paradigm-dem} and Table \ref{tbl:mg-paradigm-wh-rels-rep}, repeated from Table \ref{tbl:mg-paradigm-wh-rels}.

\begin{table}[H]
\center
\caption {Modern German \tit{der} demonstratives \pgcitep{durrell2016}{5.4.1}}
 \begin{tabular}{ccccc}
 \toprule
             & \ac{m}  & \tsc{n} & \ac{f} & \ac{pl} \\
   \cmidrule{2-5}
   \ac{nom}  & de-r   & da-s   & die   & die    \\
   \ac{acc}  & de-n   & da-s   & die   & die    \\
   \ac{dat}  & de-m   & de-m   & de-r  & de-n   \\
 \bottomrule
 \end{tabular}
 \label{tbl:mg-paradigm-dem}
\end{table}

\begin{table}[htbp]
\center
\caption {Modern German relative pronouns (headless relatives) \pgcitep{durrell2016}{5.3.3} (repeated)}
\begin{tabular}{ccc}
\toprule
            & \ac{an} & \tsc{inan}\\
  \cmidrule{2-3}
  \ac{nom}  & we-r    & wa-s     \\
  \ac{acc}  & we-n    & wa-s     \\
  \ac{dat}  & we-m    & -        \\
\bottomrule
\end{tabular}
\label{tbl:mg-paradigm-wh-rels-rep}
\end{table}

The morpheme \tit{we} combines with the same endings as the morpheme \tit{de} does in demonstrative pronouns (or relative pronouns in headed relatives).\footnote{
Note that the \tsc{wh}-pronouns in Table \ref{tbl:mg-paradigm-wh-rels-rep}, unlike the demonstratives, do not have feminine and plural forms. As far as I know, this holds for all relative pronouns in languages of the internal-only type (cf. also for Finnish, even though it makes a lot of morphological distinctions) and of the matching type. Relative pronouns in languages of the unrestricted type do inflect for feminine and plural, as well as always-external languages.
It is not clear to me how the observation about the morphology is connected to the two language types.
}
This identifies the \tit{de} and, more importantly for the discussion here, the \tit{we} as a separate morpheme.\footnote{
It is also possible to analyze \tit{we} as two separate morphemes: \tit{w} and \tit{e}. This further decomposition would not make a difference for the analysis I propose here. What is crucial is that phi and case features correspond to a single morpheme and the other part has its own morpheme or morphemes.
}\footnote{
I actually think that \tit{we} also spells out deixis features. I elaborate on this in Section \ref{sec:deixis-rp}.
}

I start with discussing the operator features \tsc{wh} and \tsc{rel}.
\tsc{wh} is a feature that \tsc{wh}-pronouns, such as \tsc{wh}-relative pronouns and interrogatives, share. The feature triggers the construction of a set of alternatives in the sense of \citet{rooth1985,rooth1992}.
The feature \tsc{rel} is present to establish a relation.
I assume that a relation is established with the light head. The semantics of the headless relative a whole is then an individual that has been picked from a set of alternatives \citep[cf.]{caponigro2003}.

I continue with the last two features that are spelled out by \tit{we}, namely the number feature \# and the gender feature \tsc{an}. Consider again Table \ref{tbl:mg-paradigm-wh-rels-rep}. In the different genders, not only the final consonants differ, but also the vowel. This suggests that \tit{we} also realizes gender features.\footnote{
In Section \ref{sec:mg-w-er-cv} I discuss an alternative segmentation of \tit{wer}.
}

I end with discussing the number feature \#. I derive its presence from the fact that \tsc{wh}-pronouns in Modern German can only show singular verbal agreement and no plural agreement. Consider the examples in \ref{ex:mg-wh}.

\ex.\label{ex:mg-wh}
\ag. Wer mach-t das?\\
who do-3\tsc{sg} that\\
`Who is/are doing that?'\label{ex:mg-wh-sg}
\b. *Wer mach-en das?\\
who do-3\tsc{pl} that\\
intended: `Who are doing that?'\label{ex:mg-wh-pl}

In \ref{ex:mg-wh-sg}, the verb \tit{macht} appears in third person singular. It agrees with the \tsc{wh}-pronoun \tit{wer} `who'. This question can be interpreted as referring to a single referent or multiple, as indicated by the translation. The sentence in \ref{ex:mg-wh-pl}, in which the verb \tit{machen} has third person plural agreement, is ungrammatical.

In sum, the morpheme \tit{we} corresponds to the features \tsc{wh}, \tsc{rel}, \# and \tsc{an} as shown in \ref{ex:mg-entry-we}.

\ex. \begin{forest} boom
  [\tsc{rel}P
      [\tsc{rel}]
      [\tsc{wh}P
          [\tsc{wh}]
          [\#P
              [\#]
              [\tsc{an}]
          ]
      ]
  ]
  {\draw (.east) node[right]{⇔ \tit{we}}; }
\end{forest}\label{ex:mg-entry-we}

At this point, I gave lexical entries for each of the morphemes that the relative pronoun consists of (in \ref{ex:mg-entry-n}, \ref{ex:mg-entry-m} and \ref{ex:mg-entry-we}), and I showed what the relative pronoun as a whole looks like. I repeat it from \ref{ex:mg-rp} in \ref{ex:mg-rp-rep}.

\ex.\label{ex:mg-rp-rep}
\begin{adjustbox}{max width=0.9\textwidth}
\begin{forest} boom
  [\tsc{rel}P, s sep=40mm
      [\tsc{rel}P,
      tikz={
      \node[label=below:\tit{we},
      draw,circle,
      scale=0.95,
      fit to=tree]{};
      }
          [\tsc{rel}]
          [\tsc{wh}P
              [\tsc{wh}]
              [\#P
                  [\#]
                  [\tsc{an}]
              ]
          ]
      ]
      [\tsc{k}P,
      tikz={
      \node[label=below:\tit{r/n/m},
      draw,circle,
      scale=0.95,
      fit to=tree]{};
      }
          [\tsc{k}]
          [\#P
              [\#]
              [\tsc{an}P
                  [\tsc{an}]
                  [\tsc{cl}P
                      [\tsc{cl}]
                      [\tsc{ref}]
                  ]
              ]
          ]
      ]
  ]
\end{forest}
\end{adjustbox}

What is still needed, is a theory for combining the morphemes into relative pronouns. This theory should determine which morphemes should be combined with each other in which order. Ultimately, the result needs to be the internal syntax in \ref{ex:mg-rp-rep}.
Ideally, theory that derives this is not language-specific, but the same for all languages. In Section \ref{sec:combining} I show how this is accomplished in Nanosyntax. Readers who are not interested in the precise mechanics can proceed directly to Section \ref{sec:light-mg}.


\section{Aside: Notes on Modern German relative pronouns}\label{sec:aside-mg-rp}

This section takes a small detour to discuss three aspects of Modern German relative pronouns. It does not belong to the core of the story, and it can be skipped over without losing track of the reasoning. Moreover, I do not incorporate what I discuss here in the lexical entries in this dissertation. I start by discussing that the final consonant is a weak pronoun, then I elaborate on the deixis features that the relative pronoun spells out, and I end with an alternative segmentation of the relative pronoun.

\subsection{Modern German weak pronouns}\label{sec:mg-weak-pronoun}

In Section \ref{sec:mg-rel} I noted that I assume that the final consonant of the relative pronoun spells out pronominal features. In this section I show that it should be classified as a weak pronoun.

\citet{cardinaletti1994} split pronouns in three classes: strong pronouns, weak pronouns and clitics. Following the tests in \citet{cardinaletti1994} that distinguish the types from each other, the pronouns in \ref{ex:mg-weak} are neither strong pronouns nor clitics, and therefore, should be classified as weak pronouns.

First, \tit{n} and \tit{m} are not strong pronouns because of how they behave under coordination and under focus.
Strong pronouns can be coordinated. \tit{n} and \tit{m} cannot be coordinated, as shown in \ref{ex:wk-pron-coord}.

\ex.\label{ex:wk-pron-coord}
\ag. *Ich wollte Jan und n gestern schon anrufen.\\
 I wanted Jan and 3\tsc{sg}.\tsc{m}.\tsc{acc} yesterday already call\\
 `I already wanted to call Jan and him yesterday.'
\bg. *Ich helfe Jan und m sein Fahrrad zu reparieren.\\
 I help Jan and 3\tsc{sg}.\tsc{m}.\tsc{acc} his bike to repare\\
 `I help Jan and him repairing his bike.'

Strong pronouns can be focused, whereas \tit{n} and \tit{m} cannot be focused.

Second, the consonants are not clitics because clitics cannot combine with prepositions, but \tit{n} and \tit{m} can, as shown in \ref{ex:wk-pron-prep}.\footnote{
It seems that these examples are not grammatical for all speakers of Modern German. For these speakers, the pronouns are possibly clitics.
}

\ex.\label{ex:wk-pron-prep}
\ag. Jan hat morgen Geburtstag. Ich habe schon ein Geschenk für'n gekauft.\\
 Jan has tomorrow birthday I have already a gift {for 3\tsc{sg}.\tsc{m}.\tsc{acc}} bought\\
 `It's Jan's birthday tomorrow. I already bought him a gift.'
\bg. Ich habe mich gestern mit Jan getroffen. Ich war mit'm im Wald wandern.\\
 I have me yesterday with Jan met I was {with 3\tsc{sg}.\tsc{m}.\tsc{dat}} in forest hiking\\
 `I met with Jan yesterday. I was hiking with him in the woods.'

Clitics can either follow a dative object or precede it. Strong and weak pronouns can only follow it. \tit{n} and \tit{m} can only follow a dative object.

Since \tit{n} and \tit{m} are not strong pronouns and not clitics, they are weak pronouns.
Therefore, I propose that actually two pronominal features are present: \tsc{ref} and Σ. The feature Σ is present because the consonants are weak pronouns \citep{cardinaletti1994}.
I assume that clitics lack the features \tsc{ref} (which corresponds to the LP in \pgcitealt{cardinaletti1994}{61}) and the feature Σ. Strong pronouns have, in addition to \tsc{ref} and Σ, another feature (C in terms of \pgcitealt{cardinaletti1994}{61}).

I leave the distinction between different classes of pronouns of the main discussion and the lexical entries because they are not relevant for the analysis.


\subsection{Deixis in relative pronouns}\label{sec:deixis-rp}

In Section \ref{sec:mg-rel} I mentioned that I assume that relative pronouns also spell out deixis features. In this section I elaborate on this.

In relative pronouns it does not express spatial deixis, but discourse deixis: it establishes a relation with an antecedent.
Generally, three types of deixis are distinguished: proximal, medial and distal. I argue that \tit{e} in the relative pronoun corresponds to the medial. Generally speaking, \tsc{wh}-pronouns combine with the medial or the distal. English has morphological evidence for this claim. Demonstratives in English can combine with either the proximal or this medial/distal, as shown in \ref{ex:english-dem}.

\ex.\label{ex:english-dem}
 \ag. this\\
 \tsc{dem}.\tsc{prox}\\
 \bg. that\\
 \tsc{dem}.\tsc{med/dist}\\

\tsc{wh}-pronouns combine with the medial/distal and are ungrammatical when combined with the proximal, shown in \ref{ex:english-wh}.

\ex.\label{ex:english-wh}
 \ag. *whis\\
 \tsc{wh}.\tsc{prox}\\
 \bg. what\\
 \tsc{wh}.\tsc{med/dist}\\

The use of the medial in \tsc{wh}-pronouns can be understood conceptually if one connects spatial deixis to discourse deixis \citep[cf.][]{colasanti2019}. The proximal is spatially near the speaker, and it refers to knowledge that the speaker possesses. The medial is spatially near the hearer, and it refers to knowledge that the hearer possesses. The distal is spatially away from the speaker and the hearer, and refers to knowledge that neither of them possess. In \tsc{wh}-pronouns, the speaker is not aware of the knowledge, so the use of the proximal is excluded. Since I do not have explicit evidence for the presence of the distal, I assume that it is the medial that combines with the \tsc{wh}-pronoun.

I adopt the features for deixis distinguished by \citet{lander2018}. The feature \tsc{dx}\scsub{1} corresponds to the proximal, the features \tsc{dx}\scsub{1} and \tsc{dx}\scsub{2} correspond to the medial, and the features \tsc{dx}\scsub{1}, \tsc{dx}\scsub{2} and \tsc{dx}\scsub{3} correspond to the distal.
The difference between the proximal, the medial and the distal cannot be observed in Modern German, because it is syncretic all of them \pgcitep{lander2018}{387}, see Table \ref{tbl:mg-paradigm-dem}.

I leave the deixis features out of the main discussion and the lexical entries because they are not relevant for the analysis.

\subsection{Alternative segmatation of \tit{wer}}\label{sec:mg-w-er-cv}

In Section \ref{sec:mg-rel} I analyzed the relative pronoun \tit{wer} (and \tit{wen} and \tit{wem}) as consisting of two morphemes: \tit{we} and \tit{r}. In this section I present an alternative to this. This alternative is to let \tit{wer} correspond two lexical entries of which the phonological part look as in \ref{ex:wer-cv}.

\ex.\label{ex:wer-cv}
\a.\label{ex:w-cv} /\tit{w}/ + CV
\b.\label{ex:er-c} /\tit{er}/ + C

Under this analysis, the final consonant has the vowel \tit{e} in its lexical entry (as shown in \ref{ex:er-c}), but it does not have a phonological slot for a vowel (i.e. no C). When the lexical entry is present without the lexical entry in \ref{ex:w-cv}, the vowel \tit{e} does not surface, because there is only a slot for a consonant. Only when the lexical entry combines with a lexical entry that does have a slot for a vowel (such as \ref{ex:wer-cv}), the vowel \tit{e} gets to surface.

A theoretical advantage of this analysis is that there is no need to specify a \tit{da} and a \tit{de} and a \tit{wa} and a \tit{we} for the different genders in the lexicon. The vowel is part of the lexical entry that belongs to the final consonant and it gets to surface because of the vowel slot that the \tit{w} or \tit{d} introduces.

An empirical advantage of this analysis concerns the vowel \tit{e}. The dative forms in all gender and numbers have the \tit{e}, which I assigned to masculine gender. This holds for the genitive forms too, which I have not given here. If \tit{we} is not specified for gender (but maybe still for number) and the vowel belongs to the final consonant, it can be inserted for non-masculines too.

The strong masculine singular pronoun in nominative in Modern German is \tit{er}. It seems it can be spelled out by the lexical entry in \ref{ex:er-c} and another lexical entry that just introduces a slot for a vowel. This is not the case for the same pronoun in accusative and dative case: then the additional lexical entry seems to be a slot for a vowel that has already been filled with in /i/ (for \tit{ihn} and \tit{ihm}). For the nominative and accusative neuter singular pronoun, the slot is filled with an /ɛ/ (for \tit{es}). I leave it for future research to investigate how this difference should be modeled. An observation that might be relevant in doing that is that in the paradigm of the possessives (\tit{mein}) there are three cells that do not take an ending: the masculine singular nominative and the neuter singular nominative and accusative.

Notice also that the feminine singular and the plurals do not have a weak pronoun and they do not have a marker in forms like \tit{diese} `this' (see Table \ref{tbl:mg-dieser}). This could be because their lexical entries also contain only a slot for a consonant, and their phonology only consists of vowels, so the content of the lexical entry only appears when it is combined with a morpheme that introduces a slot for a vowel.

As this matter is not relevant for the core of my analysis, I put it aside for now. For ease of exposition I simply assign a phonological exponent to each lexical entry and I do not make further distinctions in C and V slots.



\section{Combining morphemes in Nanosyntax}\label{sec:combining}

The way Nanosyntax combines different morphemes is not by glueing them together directly from the lexicon. Instead, features are merged one by one using two components that drive the derivation. These two components are (1) a functional sequence, in which the features that need to be merged are specified including the order they are merged in, and (2) the Spellout Algorithm, which describes the spellout procedure. The lexical entries that are available within a language interact with the derivation in such a way that the morphemes get combined in the right way. Note that the functional sequence and the Spellout Algorithm are stable across languages. The only difference between languages lies in their lexical entries.

\ref{ex:fseq-wh-rel} shows the functional sequence for relative pronouns. It gives all features it contains and their hierarchical ordering.

\ex.\label{ex:fseq-wh-rel}
\begin{adjustbox}{max width=0.9\textwidth}
\begin{forest} boom
   [\tsc{k}P
       [\tsc{k}]
       [\tsc{rel}P
           [\tsc{rel}]
           [\tsc{wh}P
               [\tsc{wh}]
               [\#P
                   [\#]
                   [\tsc{an}P
                       [\tsc{an}]
                       [\tsc{cl}P
                           [\tsc{cl}]
                           [\tsc{ref}]
                       ]
                   ]
               ]
           ]
       ]
   ]
\end{forest}
\end{adjustbox}

Starting from the bottom, these are pronominal feature \tsc{ref}, gender features \tsc{cl} and \tsc{an}, a number feature \#, operator features \tsc{wh} and \tsc{rel} and case features \tsc{k}.

This order is motivated as follows. Pronominal features (\tsc{ref}) are the nominal part of the structure and therefore the bottom-most feature.
Both \citet{picallo2008} and \citet{kramer2016} argue that number (\#) is hierarchically higher than gender (\tsc{cl} and \tsc{an}). Case (\tsc{k}) is agreed to be higher than number (\#) \citep[cf.][]{bittner1996}.

For the position of the operator features (\tsc{wh} and \tsc{rel}) consider \ref{ex:position-operator}.

\ex.\label{ex:position-operator}
\a.\label{ex:of-the-kids} of the children
\b.\label{ex:of-which-kids} of which children

The linear order in \ref{ex:of-the-kids} reflects the hierarchical ordering of \tsc{k} > \tsc{d} > \tsc{n}. \tit{Of} is an instance of \tsc{k}, \tit{the} is an instance of \tsc{d}, and \tit{child} is an instance of \tsc{n}.
\ref{ex:of-which-kids} shows that the order is the same if the definite is substituted by the \tsc{wh}-word \tit{which}, suggesting that the operator features are also positioned between \tsc{k} and \tsc{n}.
Notice also that the plural morpheme \tit{-ren} appears more to the right, hence lower in the structure, than the operator features.
Finally, I assume that the feature \tsc{rel} is hierarchy higher than \tsc{wh} \citep[cf.]{baunaz2018a}.

Before I construct the relative pronouns, I explain how the spellout procedure in Nanosyntax works. Features (Fs) are merged one by one according to the functional sequence, starting from the bottom. After each instance of merge, the constructed phrase must be spelled out, as stated in \ref{ex:cyclic-phrasal-spellout}.

\ex. \tbf{Cyclic phrasal spellout} \citep{caha2021}\\
Spellout must successfully apply to the output of every Merge F operation. After successful spellout, the derivation may terminate, or proceed to another round of Merge F.\label{ex:cyclic-phrasal-spellout}

Spellout is successful when the phrase that contains the newly merged feature forms a constituent in a lexical tree that is part of the language's lexicon.
When the new feature is merged, it forms a phrase with all features merged so far.
If this created phrase cannot be spelled out successfully (i.e. when it does not form a constituent in a lexical tree), there are two movement operations possible that modify the syntactic structure in such a way that the newly merged feature becomes part of a different syntactic structure.
These movements are triggered because spellout needs to successully apply. Therefore, they are called spellout-driven movements.
A Spellout Algorithm specifies which movement operations apply and in which order this happens. I give it in \ref{ex:spellout-algorithm}.

\ex. \tbf{Spellout Algorithm} (as in \citealt{caha2021}, based on \citealt{starke2018})\label{ex:spellout-algorithm}
 \a. Merge F and spell out.\label{ex:spellout-algorithm-phrasal}
 \b. If (a) fails, move the Spec of the complement and spell out.\label{ex:spellout-algorithm-spec}
 \b. If (b) fails, move the complement of F and spell out.\label{ex:spellout-algorithm-comp}

I informally reformulate what is in \ref{ex:spellout-algorithm}, starting with the first line in \ref{ex:spellout-algorithm-phrasal}. This says that a feature F is merged, and we try to spell out the newly created phrase FP.
When the spellout in \ref{ex:spellout-algorithm-phrasal} fails (i.e. when there is no match in the lexicon), we continue to the next two lines, \ref{ex:spellout-algorithm-spec} and \ref{ex:spellout-algorithm-comp}, which describe the two types of rescue movements that can take place then.
In the discussion about Modern German, only the first line leads to successful spellout. In the next chapter in which I discuss Polish derivations, the second and third line also lead to successful spellouts. I give the full algorithm here to give the complete picture from the start.

If these two movement operations still do not lead to a successful spellout, there are two more derivational options possible: Backtracking and Spec Formation. I return to these options later in this section, when they are relevant in the derivation of Modern German relative pronouns.

With this background in place, I start constructing the accusative relative pronoun. I repeat the available lexical entries in \ref{ex:mg-entries-all-rp}.

\ex.\label{ex:mg-entries-all-rp}
\a.\label{ex:mg-entry-n-rep}
\begin{forest} boom
  [\tsc{acc}P
      [\tsc{k}2]
      [\tsc{nom}P
          [\tsc{k}1]
          [\#P
              [\#]
              [\tsc{an}P
                  [\tsc{an}]
                  [\tsc{cl}P
                      [\tsc{cl}]
                      [\tsc{ref}]
                  ]
              ]
          ]
      ]
  ]
  {\draw (.east) node[right]{⇔ \tit{n}}; }
\end{forest}
\b.\label{ex:mg-entry-m-rep}
\begin{forest} boom
  [\tsc{dat}P
      [\tsc{k}3]
      [\tsc{acc}P
          [\tsc{k}2]
          [\tsc{nom}P
              [\tsc{k}1]
              [\#P
                  [\#]
                  [\tsc{an}P
                      [\tsc{an}]
                      [\tsc{cl}P
                          [\tsc{cl}]
                          [\tsc{ref}]
                      ]
                  ]
              ]
          ]
      ]
  ]
  {\draw (.east) node[right]{⇔ \tit{m}}; }
\end{forest}
\b.\label{ex:mg-entry-we-rep}
\begin{forest} boom
  [\tsc{rel}P
      [\tsc{rel}]
      [\tsc{wh}P
          [\tsc{wh}]
          [\#P
              [\#]
              [\tsc{an}]
          ]
      ]
  ]
  {\draw (.east) node[right]{⇔ \tit{we}}; }
\end{forest}

Starting from the bottom of the functional sequence, the first two features that are merged are \tsc{ref} and \tsc{cl}, creating a \tsc{cl}P.

\ex.
\begin{forest} boom
  [\tsc{cl}P
       [\tsc{cl}]
       [\tsc{ref}]
  ]
\end{forest}

The syntactic structure forms a constituent in the lexical tree in \ref{ex:mg-entry-n-rep}. Therefore, the \tsc{cl}P is spelled out as \tit{n}, as shown in \ref{ex:mg-spellout-n-ref-cl}.

\ex.\label{ex:mg-spellout-n-ref-cl}
\begin{forest} boom
  [\tsc{cl}P,
  tikz={
  \node[label=below:\tit{n},
  draw,circle,
  scale=0.9,
  fit to=tree]{};
  }
       [\tsc{cl}]
       [\tsc{ref}]
  ]
\end{forest}

As usual, I mark this by circling the part of the structure that corresponds to the lexical entry, and placing the corresponding phonology below it.
This spellout option corresponds to \ref{ex:spellout-algorithm-phrasal} in the Spellout Algorithm.

There are more features in the functional sequence, so the next feature is merged.
This next feature is the feature \tsc{an}, and a \tsc{an}P is created.
The syntactic structure forms a constituent in the lexical tree in \ref{ex:mg-entry-n-rep}.
Therefore, the \tsc{an}P is spelled out as \tit{n}, shown in \ref{ex:mg-spellout-n-cl-an}.

\ex.\label{ex:mg-spellout-n-cl-an}
\begin{forest} boom
  [\tsc{an}P,
  tikz={
  \node[label=below:\tit{n},
  draw,circle,
  scale=0.9,
  fit to=tree]{};
  }
      [\tsc{an}]
      [\tsc{cl}P
           [\tsc{cl}]
           [\tsc{ref}]
      ]
  ]
\end{forest}

The next feature is the feature \#, and a \#P is created.
The syntactic structure forms a constituent in the lexical tree in \ref{ex:mg-entry-n-rep}.
Therefore, the \#P is spelled out as \tit{n}, shown in \ref{ex:mg-spellout-n-ind}.

\ex.\label{ex:mg-spellout-n-ind}
\begin{forest} boom
  [\#P,
  tikz={
  \node[label=below:\tit{n},
  draw,circle,
  scale=0.9,
  fit to=tree]{};
  }
      [\#]
      [\tsc{an}P
          [\tsc{an}]
          [\tsc{cl}P
              [\tsc{cl}]
               [\tsc{ref}]
          ]
      ]
  ]
\end{forest}

The next feature in the functional sequence in \ref{ex:fseq-wh-rel} is the feature \tsc{wh}. This feature cannot be spelled out as the other ones before, which I show in what follows.
The feature \tsc{wh} is merged, and a \tsc{wh}P is created, as shown in \ref{ex:mg-spellout-wh-phrasal}.

\ex.\label{ex:mg-spellout-wh-phrasal}
\begin{forest} boom
  [\tsc{wh}P
      [\tsc{wh}]
      [\#P
          [\#]
          [\tsc{an}P
              [\tsc{an}]
              [\tsc{cl}P
                  [\tsc{cl}]
                  [\tsc{ref}]
              ]
          ]
      ]
  ]
\end{forest}

This syntactic structure does not form a constituent in the lexical tree in \ref{ex:mg-entry-n-rep}. It contains the feature \tsc{wh}, which \ref{ex:mg-entry-n} does not contain.
There is also no other lexical tree that contains the structure in \ref{ex:mg-spellout-wh-phrasal} as a constituent. Therefore, there is no successful spellout for the syntactic structure in the derivational step in which the structure is spelled out as a single phrase (\ref{ex:spellout-algorithm-phrasal} in the Spellout Algorithm).

The first movement option in the Spellout Algorithm is moving the specifier, as described in \ref{ex:spellout-algorithm-spec}. As there is no specifier in this structure, the first movement option is irrelevant.
The second movement option in the Spellout Algorithm is moving the complement, as described in \ref{ex:spellout-algorithm-comp}. In this case, the complement of \tsc{wh}, the \#P, is moved to the specifier of \tsc{wh}P. As this movement option does not lead to a successful match, I do not show it here. I come back to it in Chapter \ref{ch:deriving-matching}, in which it does lead to a successful match.

As I mentioned earlier, there are two more derivational options possible: Backtracking and Spec Formation. Derivationally, Backtracking comes first. However, since this does not lead to a successful spellout here I first introduce Spec Formation and I return to Backtracking later. Spec Formation is a last resort operation, when the feature cannot be spelled out by any of the preceding options. It is formalized as in \ref{ex:merge-spec}.

\ex.\label{ex:merge-spec}
\tbf{Spec Formation} \citep{starke2018}:\\
If Merge F has failed to spell out (even after Backtracking), try to spawn a new derivation providing F and merge that with the current derivation, projecting F to the top node.

I reformulate this informally: if none of the preceding spellout options lead to a successful spellout, a last resort operation applies. The feature that has not been spelled out yet, is merged with some other features (to which I shortly come back) in a separate workspace. Crucially, the phrase that is created is contained in a lexical tree in the language's lexicon. Finally, the feature is spelled out successfully. The newly created phrase (the spec) is merged as a whole with the already existing structure.

Now I come back to the `other' features that the feature is merged with to create a phrase that can be spelled out. This cannot be just any feature. What is crucial here again is the functional sequence. The newly merged feature is merged with features that precede it in this sequence.\footnote{
There are three different proposals on Spec Formation.
\citet{caha2019} argue that there can only be a single feature overlap between the two phrases.
\citet{de2018} argue that there cannot be any overlap at all. The features that used in the second workspace are removed from the structure in the main workspace.
In this dissertation, I work with the proposal in \citet{starke2018}, in which the overlap between the phrase on the left and the phrase on the right can also be more than a single feature. This is the only proposal of the three that allows me to derive all the forms I encounter.
}
This can be a single feature or multiples ones. I illustrate this with the Modern German relative pronouns.

For the feature \tsc{wh} it means that it is merged with the feature \#. Then, the lexicon is checked for a lexical tree that contains the phrase \tsc{wh}P that contains \tsc{wh} and \#, as shown in \ref{ex:mg-spellout-wh-ind}.

\ex.\label{ex:mg-spellout-wh-ind}
\begin{forest} boom
  [\tsc{wh}P
      [\tsc{wh}]
      [\#]
  ]
\end{forest}

This syntactic structure does not form a constituent in any of the lexical trees in the language's lexicon.
Therefore, the feature \tsc{wh} combines not only with the feature merged before it, but with a phrase that consists of the two features merged before it: \# and \tsc{an}. I give the phrase this gives in \ref{ex:mg-structure-wh-ind-an}.

\ex.\label{ex:mg-structure-wh-ind-an}
\begin{forest} boom
  [\tsc{wh}P
      [\tsc{wh}]
      [\#P
          [\#]
          [\tsc{an}]
      ]
  ]
\end{forest}

This syntactic structure forms a constituent in the lexical tree in \ref{ex:mg-entry-we-rep}. Therefore, the \tsc{wh}P is spelled out as \tit{we}, as shown in \ref{ex:mg-spellout-whp-ind}.

\ex.\label{ex:mg-spellout-whp-ind}
\begin{forest} boom
  [\tsc{wh}P,
   tikz={
   \node[label=below:\tit{we},
   draw,circle,
   scale=0.9,
   fit to=tree]{};
   }
      [\tsc{wh}]
      [\#P
          [\#]
          [\tsc{an}]
      ]
  ]
\end{forest}

The newly created phrase is merged as a whole with the already existing structure. As specified in \ref{ex:merge-spec}, the feature \tsc{wh} projects to the top node. I show the results in \ref{ex:mg-spellout-whp}.

\ex.\label{ex:mg-spellout-whp}
\begin{adjustbox}{max width=0.9\textwidth}
\begin{forest} boom
  [\tsc{wh}P, s sep = 30mm
      [\tsc{wh}P,
       tikz={
       \node[label=below:\tit{we},
       draw,circle,
       scale=0.9,
       fit to=tree]{};
       }
          [\tsc{wh}]
          [\#P
              [\#]
              [\tsc{an}]
          ]
      ]
      [\#P,
      tikz={
      \node[label=below:\tit{n},
      draw,circle,
      scale=0.95,
      fit to=tree]{};
      }
          [\#]
          [\tsc{an}P
              [\tsc{an}]
              [\tsc{cl}P
                  [\tsc{cl}]
                  [\tsc{ref}]
              ]
          ]
      ]
  ]
\end{forest}
\end{adjustbox}

Notice here that there is an overlap of multiple features between the phrase on the right and the phrase on the left.

The next feature in the functional sequence is the feature \tsc{rel}. As always, it is merged to the existing syntactic structure, which is now the \tsc{wh}P. The result is the \tsc{rel}P shown in \ref{ex:mg-spellout-rel-phrasal}.

\ex.\label{ex:mg-spellout-rel-phrasal}
\begin{adjustbox}{max width=0.9\textwidth}
\begin{forest} boom
  [\tsc{rel}P
      [\tsc{rel}]
      [\tsc{wh}P, s sep = 30mm
          [\tsc{wh}P,
           tikz={
           \node[label=below:\tit{we},
           draw,circle,
           scale=0.9,
           fit to=tree]{};
           }
              [\tsc{wh}]
              [\#P
                  [\#]
                  [\tsc{an}]
              ]
          ]
          [\#P,
          tikz={
          \node[label=below:\tit{n},
          draw,circle,
          scale=0.95,
          fit to=tree]{};
          }
              [\#]
              [\tsc{an}P
                  [\tsc{an}]
                  [\tsc{cl}P
                      [\tsc{cl}]
                      [\tsc{ref}]
                  ]
              ]
          ]
      ]
  ]
\end{forest}
\end{adjustbox}

This whole structure does not form a constituent in any of the lexical trees in the language's lexicon. Neither of the spellout driven movement operations leads to a successful spellout. This means that, once again, the derivation reaches a point at which one of the two other possible derivational options come into play. As I mentioned before, Backtracking comes first, and this is the operation that leads to a successful spellout here.

Consider the syntactic structure in \ref{ex:mg-spellout-rel-phrasal} again. The feature \tsc{rel} is merged with the highest \tsc{wh}P. In this position it cannot be spelled out.
Consider now the lexical entry in \ref{ex:mg-entry-we-rep}. This is a lexical tree that contains \tsc{rel}. This means that the feature \tsc{rel} somehow needs to end up in the Spec that has just been merged.
I follow \citet{caha2019} who proposes that this happens via Backtracking. He argues that the main idea of Backtracking is that a feature is merged with a different tree than the one it was merged with before, as stated in \ref{ex:backtracking}.\footnote{
In this dissertation I do not discuss the effect that Backtracking `normally' has, namely to try a different spellout option at the previous cycle. That does not mean that I assume it is not part of the derivation: I actually assume it a step that is attempted. I refrain from mentioning it, because this does not lead to a successful spellout in any of the derivations I discuss.
}

\ex. \tbf{The logic of backtracking} \pgcitep{caha2019}{198}\\\label{ex:backtracking}
When spellout of F fails, go back to the previous cycle, and provide a different configuration for Merge F.

Imagine a situation in which the previous feature was spelled out with a complex specifier and the next feature reaches the derivational option Backtracking. This is exactly the situation that arises after \tsc{rel} is merged. Providing a different configuration means splitting up the two phrases, and then merging the feature again. Specifically, I adopt the proposal in which the feature is merged in both workspaces, as stated in \ref{ex:multiple-merge}.

\ex. \tbf{Multiple Merge} \pgcitep{caha2019}{227}\\\label{ex:multiple-merge}
When backtracking reopens multiple workspaces, merge F in each such workspace.

For the example under discussion, the situation looks as in \ref{ex:mg-rel-2x}.

\ex.\label{ex:mg-rel-2x}
\a.\label{ex:mg-whp-rel}
\begin{forest} boom
  [\tsc{rel}P
      [\tsc{rel}]
      [\tsc{wh}P
          [\tsc{wh}]
          [\#P
              [\#]
              [\tsc{an}]
          ]
      ]
  ]
\end{forest}
\b.\label{ex:mg-indp-rel}
\begin{forest} boom
  [\tsc{rel}P
      [\tsc{rel}]
      [\#P
          [\#]
          [\tsc{an}P
              [\tsc{an}]
              [\tsc{cl}P
                  [\tsc{cl}]
                   [\tsc{ref}]
              ]
          ]
      ]
  ]
\end{forest}

The feature \tsc{rel} is merged in both workspaces, so it combines with the \tsc{wh}P in \ref{ex:mg-whp-rel} and with the \#P in \ref{ex:mg-indp-rel}. From here on, the derivation proceeds, as usual, according to the Spellout Algorithm, with the only difference that it happens in two workspaces simultaneously. Spellout has to be successful in at least one of the two workspaces.

In the case of \ref{ex:mg-rel-2x}, the spellout of \tsc{rel} is successful in the syntactic structure in \ref{ex:mg-whp-rel}.
This syntactic structure forms a constituent in the lexical tree in \ref{ex:mg-entry-we-rep}, which corresponds to the \tit{we}.
There is no successful spellout for \ref{ex:mg-indp-rel}, so the \tsc{rel} is removed from this structure.
As spellout has succeeded at least once, the workspaces can be merged back together. The result is shown in \ref{ex:mg-spellout-relp-inspec}.

\ex.\label{ex:mg-spellout-relp-inspec}
\begin{adjustbox}{max width=0.9\textwidth}
\begin{forest} boom
  [\tsc{rel}P, s sep = 40mm
      [\tsc{rel}P,
       tikz={
       \node[label=below:\tit{we},
       draw,circle,
       scale=0.95,
       fit to=tree]{};
       }
          [\tsc{rel}]
          [\tsc{wh}P
              [\tsc{wh}]
              [\#P
                  [\#]
                  [\tsc{an}]
              ]
          ]
      ]
      [\#P,
      tikz={
      \node[label=below:\tit{n},
      draw,circle,
      scale=0.95,
      fit to=tree]{};
      }
          [\#]
          [\tsc{an}P
              [\tsc{an}]
              [\tsc{cl}P
                  [\tsc{cl}]
                  [\tsc{ref}]
              ]
          ]
      ]
  ]
\end{forest}
\end{adjustbox}

The next feature in the functional sequence is \tsc{k}1. This feature should somehow end up merging with \#P, because it forms a constituent in the lexical tree in \ref{ex:mg-entry-n-rep}, which corresponds to \tit{n}. This can again be achieved via Backtracking in which phrases are split up. I go through the derivation step by step.

The feature \tsc{k}1 is merged with the existing syntactic structure, creating a \tsc{nom}P.
This structure does not form a constituent in any of the lexical trees in the language's lexicon, and neither of the spellout driven movements leads to a successful spellout.
Backtracking leads to splitting up the \tsc{rel}P from the \#P.
The feature \tsc{k}1 is merged in both workspaces, so with the \tsc{rel}P and with the \#P. The spellout of \tsc{k}1 is successful when it is combined with the \#P.
It forms a constituent in the lexical tree in \ref{ex:mg-entry-n-rep}, which corresponds to the \tit{n}.
The \tsc{nom}P is spelled out as \tit{n}, and all constituents are merged back into the existing syntactic structure, as shown in \ref{ex:mg-spellout-rel-nom}.

\ex.\label{ex:mg-spellout-rel-nom}
\begin{adjustbox}{max width=0.9\textwidth}
\begin{forest} boom
      [\tsc{rel}P, s sep=40mm
          [\tsc{rel}P,
          tikz={
          \node[label=below:\tit{we},
          draw,circle,
          scale=0.95,
          fit to=tree]{};
          }
              [\tsc{rel}]
              [\tsc{wh}P
                  [\tsc{wh}]
                  [\#P
                      [\#]
                      [\tsc{an}]
                  ]
              ]
          ]
          [\tsc{nom}P,
          tikz={
          \node[label=below:\tit{n},
          draw,circle,
          scale=0.95,
          fit to=tree]{};
          }
              [\tsc{k}1]
              [\#P
                  [\#]
                  [\tsc{an}P
                      [\tsc{an}]
                      [\tsc{cl}P
                          [\tsc{cl}]
                          [\tsc{ref}]
                      ]
                  ]
              ]
          ]
      ]
  ]
\end{forest}
\end{adjustbox}

For the accusative relative pronoun, the last feature is merged: the \tsc{k}2. The derivation for \tsc{k}2 resembles the derivation of \tsc{k}1. The feature is merged with the existing syntactic structure, creating a \tsc{acc}P.
This structure does not form a constituent in any of the lexical trees in the language's lexicon, and neither of the spellout driven movements leads to a successful spellout.
Backtracking leads to splitting up the \tsc{rel}P from the \tsc{nom}P.
The feature \tsc{k}2 is merged in both workspaces, so with the \tsc{rel}P and with the \tsc{nom}P. The spellout of \tsc{k}2 is successful when it is combined with the \tsc{nom}P.
It forms a constituent in the lexical tree in \ref{ex:mg-entry-n-rep}, which corresponds to the \tit{n}. The \tsc{acc}P is spelled out as \tit{n}, and all constituents are merged back into the existing syntactic structure, as shown in \ref{ex:mg-spellout-rel-acc}.

\ex.\label{ex:mg-spellout-rel-acc}
\begin{adjustbox}{max width=0.9\textwidth}
\begin{forest} boom
      [\tsc{rel}P, s sep=45mm
          [\tsc{rel}P,
          tikz={
          \node[label=below:\tit{we},
          draw,circle,
          scale=1,
          fit to=tree]{};
          }
              [\tsc{rel}]
              [\tsc{wh}P
                  [\tsc{wh}]
                  [\#P
                      [\#]
                      [\tsc{an}]
                  ]
              ]
          ]
          [\tsc{acc}P,
          tikz={
          \node[label=below:\tit{n},
          draw,circle,
          scale=0.95,
          fit to=tree]{};
          }
              [\tsc{k}2]
              [\tsc{nom}P
                  [\tsc{k}1]
                  [\#P
                      [\#]
                      [\tsc{an}P
                          [\tsc{an}]
                          [\tsc{cl}P
                              [\tsc{cl}]
                              [\tsc{ref}]
                          ]
                      ]
                  ]
              ]
          ]
      ]
  ]
\end{forest}
\end{adjustbox}

For the dative relative pronoun, one more feature is merged: the \tsc{k}3. The derivation for \tsc{k}3 resembles the derivation of \tsc{k}1 and \tsc{k}2. The feature is merged with the existing syntactic structure, creating a \tsc{dat}P.
This structure does not form a constituent in any of the lexical trees in the language's lexicon, and neither of the spellout driven movements leads to a successful spellout.
Backtracking leads to splitting up the \tsc{rel}P from the \tsc{acc}P.
The feature \tsc{k}3 is merged in both workspaces, so with the \tsc{rel}P and with the \tsc{acc}P. The spellout of \tsc{k}3 is successful when it is combined with the \tsc{acc}P.
It forms a constituent in the lexical tree in \ref{ex:mg-entry-m-rep}.
The \tsc{dat}P is spelled out as \tit{m}, and all constituents are merged back into the existing syntactic structure, as shown in \ref{ex:mg-spellout-rel-dat}.

\ex.\label{ex:mg-spellout-rel-dat}
\begin{adjustbox}{max width=0.9\textwidth}
\begin{forest} boom
      [\tsc{rel}P, s sep=45mm
          [\tsc{rel}P,
          tikz={
          \node[label=below:\tit{we},
          draw,circle,
          scale=1,
          fit to=tree]{};
          }
              [\tsc{rel}]
              [\tsc{wh}P
                  [\tsc{wh}]
                  [\#P
                      [\#]
                      [\tsc{an}]
                  ]
              ]
          ]
          [\tsc{dat}P,
          tikz={
          \node[label=below:\tit{m},
          draw,circle,
          scale=0.95,
          fit to=tree]{};
          }
              [\tsc{k}3]
              [\tsc{acc}P
                  [\tsc{k}2]
                  [\tsc{nom}P
                      [\tsc{k}1]
                      [\#P
                          [\#]
                          [\tsc{an}P
                              [\tsc{an}]
                              [\tsc{cl}P
                                  [\tsc{cl}]
                                  [\tsc{ref}]
                              ]
                          ]
                      ]
                  ]
              ]
          ]
      ]
  ]
\end{forest}
\end{adjustbox}

To summarize, I decomposed the relative pronoun into the two morphemes: \tit{we} and the final consonant (\tit{n} and \tit{m}). I showed which features each of the morphemes spells out and what the internal syntax looks like that they are combined into. It is this internal syntax that determines whether the light head can be deleted or not.

\section{The Modern German (extra) light head}\label{sec:light-mg}

I have suggested that headless relatives are derived from light-headed relatives. The light head or the relative pronoun can be deleted when either of them is contained in the other one.
In Chapter \ref{ch:the-basic-idea}, I mentioned that languages have two possible light heads. I also noted that headless relatives in Modern German can only be derived from light-headed relatives that are headed by one of these heads. In this section I give arguments that exclude the second light head as a possible light head.

In this section I discuss both possible light heads. I start by discussing a light-headed relative that is attested in Modern German. If the headless relative is derived from this light-headed relative, the deletion would have to be optional. I consider this scenario, and I give two arguments against it.
Then I take the light head from the existing light-headed relative as a point of departure, and I modify it in such a way that it is appropriate as a light head for a headless relative in Modern German. I argue that this light head is the head of the light-headed relative that Modern German headless relative are derived from. This light-headed relative does not exist in Modern German, so the deletion of the light head is obligatory.

In the introduction of this chapter, I claimed that the internal syntax of light heads in Modern German looks as shown in \ref{ex:mg-lh-complex}.

\ex.\label{ex:mg-lh-complex}
\begin{forest} boom
  [\tsc{k}P,
  tikz={
  \node[label=below:\tit{r/n/m},
  draw,circle,
  scale=0.75,
  fit to=tree]{};
  }
      [\tsc{k}]
      [ϕP
          [\phantom{xxx}, roof]
      ]
  ]
\end{forest}

In this section, I determine the exact feature content of the light head.
I end up claiming that the phi and case feature portmanteau of the relative pronoun is the light head in headless relatives. I show the complete structure that I work towards in this section in \ref{ex:mg-lh}.

\ex.\label{ex:mg-lh}
\begin{forest} boom
    [\tsc{k}P,
    tikz={
    \node[label=below:\tit{r/n/m},
    draw,circle,
    scale=0.95,
    fit to=tree]{};
    }
        [\tsc{k}]
        [\#P
            [\#]
            [\tsc{an}P
                [\tsc{an}]
                [\tsc{cl}P
                    [\tsc{cl}]
                    [\tsc{ref}]
                ]
            ]
        ]
    ]
\end{forest}

I give an example of an existing Modern German light-headed relative in \ref{ex:mg-den-wen}.\footnote{
Not every speaker of Modern German accepts the combination of the light head \tit{den} and the \tsc{wh}-relative \tit{wen} as shown in the example in \ref{ex:mg-den-wen}.
Most prefer another light-headed relative, in which the relative pronoun is the \tsc{d}-pronoun. I give an example in \ref{ex:mg-den-den}.

\exg. Ich umarme den, den ich mag.\\
I hug \tsc{d}.\tsc{m}.\tsc{sg}.\tsc{acc} \tsc{rp}.\tsc{m}.\tsc{sg}.\tsc{acc} I like\\
`I hug him that I like.'\label{ex:mg-den-den}

This relative pronoun generally appears in relative clauses headed by a full NP, shown in \ref{ex:mg-den-headed}.

\exg. Ich umarme den Mann, den ich mag.\\
I hug \tsc{d}.\tsc{m}.\tsc{sg}.\tsc{acc} man \tsc{rp}.\tsc{m}.\tsc{sg}.\tsc{acc} I like\\
`I hug the man that I like.'\label{ex:mg-den-headed}

A combination between a full NP and a \tsc{wh}-pronoun is ungrammatical, as shown in \ref{ex:mg-wen-headed}.

\exg. *Ich umarme den Mann, wen ich mag.\\
I hug \tsc{d}.\tsc{m}.\tsc{sg}.\tsc{acc} man \tsc{rp}.\tsc{m}.\tsc{sg}.\tsc{acc} I like\\
`I hug the man that I like.'\label{ex:mg-wen-headed}

Even though not every speaker of Modern German likes the combination of \tit{den} and \tit{wen} in \ref{ex:mg-den-wen}, all speakers I consulted judged it better than the combination of a full NP and a \tsc{wh}-pronoun as in \ref{ex:mg-wen-headed}.
}\footnote{
It seems light-headed relatives with inanimates, as in \ref{ex:mg-das-was}, are judged better than examples with animates \citep[cf.][ftn. 29]{hanink2018}

\exg. Ich esse das, was ich mag.\\
I hug \tsc{dem}.\tsc{n}.\tsc{sg}.\tsc{acc} \tsc{rp}.\tsc{n}.\tsc{sg}.\tsc{acc} I like\\
`I eat that what I like.'\label{ex:mg-das-was}

In turn, examples with different cases, as in are judged worse than examples with matching cases, although they are still attested \citep[ftn. 6]{fuss2014}.

\exg. Wer das sagt, dem sind die mitleidigen Blicke gewiss.\\
\tsc{rp}.\tsc{m}.\tsc{sg}.\tsc{nom} that says \tsc{dem}.\tsc{m}.\tsc{sg}.\tsc{dat} are the pitying looks certainly\\
`He, who says that, certainly gets pitying looks.'\label{ex:mg-wer-dem}

\phantom{x}
}

\exg. Ich umarme den, wen ich mag.\\
I hug \tsc{dem}.\tsc{m}.\tsc{sg}.\tsc{acc} \tsc{rp}.\tsc{an}.\tsc{acc} I like\\
`I hug the man that I like.'\label{ex:mg-den-wen}

In \ref{ex:mg-den-wen}, the relative pronoun is the \tsc{wh}-pronoun \tit{wen} `\tsc{rp}.\tsc{an}.\tsc{acc}', and the light head is the \tsc{d}-pronoun \tit{den} `\tsc{dem}.\tsc{m}.\tsc{sg}.\tsc{acc}'. For easy reference, I call this light-headed relative the \tit{den}-\tit{wen} relative.

One hypothesis is that the demonstrative \tit{den} `\tsc{dem}.\tsc{m}.\tsc{sg}.\tsc{acc}' is deleted from the light-headed relative in \ref{ex:mg-den-wen} and that the headless relative in \ref{ex:mg-wen} remains.\footnote{
This is exactly what \citet{hanink2018} argues for. She claims that the feature content of the demonstrative \tit{den} matches the feature content of the relative pronoun \tit{wen}. Therefore, the light head is by default deleted. Only if the light head carries an extra focus feature it surfaces.
}
For easy reference, I call this headless relative the \tit{wen} relative.

\exg. Ich umarme, wen ich mag.\\
I hug \tsc{rp}.\tsc{an}.\tsc{acc} I like\\
`I hugs who I like.'\label{ex:mg-wen}

The demonstrative in \ref{ex:mg-den-wen} is the second possible light head that I introduced in Chapter \ref{ch:the-basic-idea}. In this section I give two arguments against the hypothesis that the light-headed relative headed by the demonstrative is the source of the headless relative in Modern German. Both arguments have to do with interpretation. In Chapter \ref{ch:deriving-unrestricted} I discuss another argument, possibly even stronger, which concerns phonology.
The first argument is that in headless relatives the phrase \tit{auch immer} `ever' can appear, as shown in \ref{ex:mg-wh-for-headless-ever}.

\exg. Ich umarme, wen {auch immer} ich mag.\\
I hug \tsc{rp}.\tsc{an}.\tsc{acc} ever I like\\
`I hug whoever I like.'\label{ex:mg-wh-for-headless-ever}

Light-headed relatives do not allow for this phrase to be inserted, illustrated in \ref{ex:mg-wh-for-headed-ever}.

\exg. *Ich umarme den, wen {auch immer} ich mag.\\
I hug \tsc{dem}.\tsc{m}.\tsc{sg}.\tsc{acc} \tsc{rp}.\tsc{an}.\tsc{acc} ever I like\\
`I hug him whoever I like.'\label{ex:mg-wh-for-headed-ever}

The second argument against the \tit{den}-\tit{wen} relative being the source of the \tit{wen} relative comes from the differences between the interpretation of the two constructions. Broadly speaking, the \tit{wen} relative has two interpretations (see \citealt{s̆imík2020} for a recent elaborate overview on the semantics of free relatives). The \tit{den}-\tit{wen} has only one of them. I show this schematically in Table \ref{tbl:mg-interpretations}.

\begin{table}[htbp]
  \center
  \caption{Interpretations of \tit{wen} and \tit{den}-\tit{wen} relatives}
\begin{tabular}{ccc}
  \toprule
                & \tit{wen} & \tit{den}-\tit{wen} \\
                \cmidrule{2-3}
definite-like   & ✔         & ✔                   \\
universal-like  & ✔         & *                   \\
\bottomrule
\end{tabular}
\label{tbl:mg-interpretations}
\end{table}

The first interpretation of the \tit{wen} relative is a definite-like one. This interpretation corresponds to a definite description. Consider the context which facilitates a definite-interpretation and the repeated \tit{den}-\tit{wen} and \tit{wen} relative in \ref{ex:mg-context-def}.

\ex.
\a. Context: Yesterday I met with two friends. I like one of them. The other one I do not like so much.\label{ex:mg-context-def}
\bg. Ich umarme den, wen ich mag.\\
I hug \tsc{dem}.\tsc{m}.\tsc{sg}.\tsc{acc} \tsc{rp}.\tsc{an}.\tsc{acc} I like\\
`I hug who I like.'
\bg. Ich umarme, wen ich mag.\\
I hug \tsc{rp}.\tsc{an}.\tsc{acc} I like\\
`I hugs who I like.'

A definite-like interpretation is one in which I hug the person that I like.
The interpretation is available for the \tit{wen} relative and for the \tit{den}-\tit{wen} relative.

The second interpretation of the \tit{wen} relative is a universal-like one. This interpretation corresponds to a universal quantifier. Consider the context which facilitates a universal-interpretation and the repeated \tit{den}-\tit{wen} and \tit{wen} relative in \ref{ex:mg-context-univ}.

\ex.
\a. I have a general habit of hugging everybody that I like.\label{ex:mg-context-univ}
\bg. \#Ich umarme den, wen ich mag.\\
I hug \tsc{dem}.\tsc{m}.\tsc{sg}.\tsc{acc} \tsc{rp}.\tsc{an}.\tsc{acc} I like\\
`I hug who I like.'
\bg. Ich umarme, wen ich mag.\\
I hug \tsc{rp}.\tsc{an}.\tsc{acc} I like\\
`I hug who I like.'

A universal-like interpretation is one in which I hug everybody that I like.
This interpretation is available for the \tit{wen} relative, but not for the \tit{den}-\tit{wen} relative.

There are some indications that the universal-like interpretation of headless relatives is the main interpretation that should be accounted for.
First, informants have reported to me that headless relatives with case mismatches become more acceptable in the universal-like interpretation compared to the definite-like interpretation.
Second, \pgcitet{s̆imík2020}{4} notes that some languages do not easily allow for the definite-like interpretation of headless relatives with an \tit{ever}-morpheme. There is no language documented that does not allow for the universal-like interpretation, but does allow the definite-like interpretation.

In sum, there are two arguments against the \tit{den}-\tit{wen} relative being the source of the \tit{wen} relative. In what follows, I show how the presence of \tit{den} leads to having only the definite-like interpretation. I suggest that the problem lies in the feature content of the demonstrative. I point out how the feature content should be modified such that it is a suitable light head for a headless relative.

The light head in the \tit{den}-\tit{wen} relative is a demonstrative. A demonstrative refers back to a linguistic or extra-linguistic antecedent. Consider the context in \ref{ex:mg-context-def} again. The demonstrative \tit{den} in the \tit{den}-\tit{wen} relative refers back to the friend of Jan that he likes, and the construction is grammatical. Now consider the context in \ref{ex:mg-context-univ} again. In this case, there is no antecedent for the demonstrative \tit{den} to refer back to, and the structure is infelicitous.

I decompose demonstrative \tit{den} into different morphemes to investigate what it is about the demonstrative that forces the definite-like interpretation. The demonstrative consists (at least) of the two morphemes \tit{de} and \tit{n}. One of these morphemes is identical to the \tsc{wh}-relative pronoun: the \tit{n}, which spells out pronominal, number, gender and case features. The other morpheme differs: the \tit{de}, which establishes a definite reference.\footnote{
In Chapter \ref{ch:deriving-unrestricted} I describe in more detail what features (the Old High German counterpart of) \tit{de} corresponds to.
}

So far, I established that the \tit{den}-\tit{wen} relative cannot be the source from which the headless relative is derived. Still, since I assume that headless relatives are derived from light-headed relatives, there must be some light-headed relative that is the source. I propose that the light head in the light-headed relative is even lighter than the head in the \tit{den}-\tit{wen} relative: it is an extra light head.

I propose that the extra light head is the element that is left once the morpheme \tit{de} is absent. This is the morpheme that is the final consonant of the relative pronoun. I give the extra light-headed relative that the \tit{wen}-relative is derived from in \ref{ex:mg-real-base}. The brackets around the light head indicate that it is obligatorily deleted.

\exg. Ich umarmt [n], wen ich mag.\\
I hug \tsc{elh}.\tsc{an}.\tsc{acc} \tsc{rp}.\tsc{an}.\tsc{acc} I like\\
`I hug who I like.'\label{ex:mg-real-base}

In the remainder of this section, I discuss the two extra light heads that I compare the internal syntax of in Section \ref{sec:comparing-mg}. These are the accusative animate and the dative animate, shown in \ref{ex:mg-lhs}.\footnote{
Again, for reasons of space, I do not discuss the nominative form. I assume its analysis is identical to the one I propose for the accusative and the dative.
}

\ex.\label{ex:mg-lhs}
\a. n `\tsc{elh}.\tsc{an}.\tsc{acc}'
\b. m `\tsc{elh}.\tsc{an}.\tsc{dat}'

As I noted before, these forms do not surface as light heads in a light-headed relative. They do surface as pronouns in colloquial speech in the language.

In Chapter \ref{ch:the-basic-idea}, I suggested that the relative pronoun contains at least one feature more than the extra light head. In my proposal, it is actually two features, namely \tsc{wh} and \tsc{rel}. This leaves the functional sequence for the extra light head as shown in \ref{ex:fseq-elh}.

\ex.\label{ex:fseq-elh}
\begin{forest} boom
  [\tsc{k}P
      [\tsc{k}]
      [\#P
          [\#]
          [\tsc{an}P
              [\tsc{an}]
              [\tsc{cl}P
                  [\tsc{cl}]
                  [\tsc{ref}]
              ]
          ]
      ]
  ]
\end{forest}

It contains the pronominal feature \tsc{ref}, the gender features \tsc{cl} and \tsc{an}, the number feature \# and case features \tsc{k}.

I introduced the lexical entries that are required to spell out these features in Section \ref{sec:mg-rel}. I repeat them in \ref{ex:mg-entries-all-lh}.

\ex.\label{ex:mg-entries-all-lh}
\a.\label{ex:mg-entry-n-rep1}
 \begin{forest} boom
   [\tsc{acc}P
       [\tsc{k}2]
       [\tsc{nom}P
           [\tsc{k}1]
           [\#P
               [\#]
               [\tsc{an}P
                   [\tsc{an}]
                   [\tsc{cl}P
                       [\tsc{cl}]
                       [\tsc{ref}]
                   ]
               ]
           ]
       ]
   ]
   {\draw (.east) node[right]{⇔ \tit{n}}; }
 \end{forest}
\b.\label{ex:mg-entry-m-rep1}
 \begin{forest} boom
   [\tsc{dat}P
       [\tsc{k}3]
       [\tsc{acc}P
           [\tsc{k}2]
           [\tsc{nom}P
               [\tsc{k}1]
               [\#P
                   [\#]
                   [\tsc{an}P
                       [\tsc{an}]
                       [\tsc{cl}P
                           [\tsc{cl}]
                           [\tsc{ref}]
                       ]
                   ]
               ]
           ]
       ]
   ]
   {\draw (.east) node[right]{⇔ \tit{m}}; }
 \end{forest}

The derivations of the extra light heads are straight-forward ones. The features are merged one by one, and after each new phrase is created, it is spelled out as a whole. I still go through them step by step.

First, the features \tsc{ref} and \tsc{cl} are merged, and the \tsc{cl}P is created.
The syntactic structure forms a constituent in the lexical tree in \ref{ex:mg-entry-n-rep1}.
Therefore, the \tsc{cl}P is spelled out as \tit{n}.
Exactly the same happens for the features \tsc{an}, \# and \tsc{k}1.
They are merged, they form a constituent in the lexical tree in \ref{ex:mg-entry-n-rep1}, and they are spelled out as \tit{n}.

The last feature that is merged for the accusative extra light head is the \tsc{k}2.
It is merged, and the \tsc{acc}P is created.
The syntactic structure forms a constituent in the lexical tree in \ref{ex:mg-entry-n-rep1}.
Therefore, the \tsc{acc}P is spelled out as \tit{n}, as shown in \ref{ex:mg-elh-acc}.

\ex. \begin{forest} boom
    [\tsc{acc}P,
    tikz={
    \node[label=below:\tit{n},
    draw,circle,
    scale=0.95,
    fit to=tree]{};
    }
        [\tsc{k}2]
        [\tsc{nom}P
            [\tsc{k}1]
            [\#P
                [\#]
                [\tsc{an}P
                    [\tsc{an}]
                    [\tsc{cl}P
                        [\tsc{cl}]
                        [\tsc{ref}]
                    ]
                ]
            ]
        ]
    ]
\end{forest}
\label{ex:mg-elh-acc}

For the dative extra light head another feature is merged: the \tsc{k}3.
The feature \tsc{k}3 is merged, and the \tsc{dat}P is created.
The syntactic structure forms a constituent in the lexical tree in \ref{ex:mg-entry-m-rep1}.
Therefore, the \tsc{dat}P is spelled out as \tit{m}, as shown in \ref{ex:mg-elh-dat}.

\ex. \label{ex:mg-elh-dat}
\begin{forest} boom
[\tsc{dat}P,
tikz={
\node[label=below:\tit{m},
draw,circle,
scale=1,
fit to=tree]{};
}
    [\tsc{k}3]
    [\tsc{acc}P
        [\tsc{k}2]
        [\tsc{nom}P
            [\tsc{k}1]
            [\#P
                [\#]
                [\tsc{an}P
                    [\tsc{an}]
                    [\tsc{cl}P
                        [\tsc{cl}]
                        [\tsc{ref}]
                    ]
                ]
            ]
        ]
    ]
]
\end{forest}

In sum, Modern German headless relatives are derived from a light-headed relative with an extra light head. This extra light head is spelled out by a single phi and case feature portmanteau. The lexical entries used to spell this light head out are also used to spell out part of the internal syntax of the relative pronoun.

% At first sight it seems like \citet{fuss2014} discuss a exception to this claim, namely headless relatives with \tsc{d}-pronouns. However, they claim that these headless relatives are actually light-headed relatives in which one of two syncretic elements is deleted by haplology.


\section{Comparing light heads and relative pronouns}\label{sec:comparing-mg}

In this section, I compare the internal syntax of extra light heads to the internal syntax of relative pronouns in Modern German. This is the worked out version of the comparisons in Section \ref{sec:basic-internal}. What is different here is that I show the comparison for Modern German specifically, and that the content of the internal syntax that is being compared is motivated earlier in this chapter.

I give three examples, in which the internal and external case vary.
I start with an example with matching cases, in which the internal and the external case are both accusative.
Then I give an example in which the internal dative case is more complex than the external accusative case.
I end with an example in which the external dative case is more complex than the internal accusative case.
I show that the first two examples are grammatical and the last one is not. I derive this by showing that only in the first two situations the light head is structurally contained in the relative pronoun, and that it can therefore then be deleted.
In the third example, neither the light head nor the relative pronoun is structurally contained in the other element.
I do not discuss formal containment in this chapter, because it never leads to a successful deletion when structural containment does not.

I start with the situation in which the cases match.
Consider the example in \ref{ex:mg-acc-acc-rep}, in which the internal accusative case competes against the external accusative case. The relative clause is marked in bold.
The internal case is accusative, as the predicate \tit{mögen} `to like' takes accusative objects. The relative pronoun \tit{wen} `\ac{rel}.\ac{an}.\ac{acc}' appears in the accusative case. This is the element that surfaces.
The external case is accusative as well, as the predicate \tit{einladen} `to invite' also takes accusative objects. The extra light head \tit{n} `\ac{elh}.\ac{an}.\ac{acc}' appears in the accusative case. It is placed between square brackets because it does not surface.

\exg. Ich lade [n] ein, \tbf{wen} \tbf{auch} \tbf{Maria} \tbf{mag}.\\
 1\ac{sg}.\ac{nom} invite.\ac{pres}.1\ac{sg}\scsub{[acc]} \tsc{elh}.\ac{an}.\ac{acc} {} \tsc{rp}.\ac{an}.\ac{acc} Maria.\ac{nom} like.\ac{pres}.3\ac{sg}\scsub{[acc]}\\
 `I invite who Maria also likes.' \flushfill{Modern German, adapted from \pgcitealt{vogel2001}{344}}\label{ex:mg-acc-acc-rep}

In Figure \ref{fig:mg-int=ext}, I give the syntactic structure of the extra light head at the top and the syntactic structure of the relative pronoun at the bottom.

\begin{figure}[htbp]
  \center
  \begin{adjustbox}{max height=0.9\textheight}
  \begin{tabular}[b]{c}
        \toprule
        \tsc{acc} extra light head \tit{n}\\
        \cmidrule{1-1}
      \begin{forest} boom
        [\tsc{acc}P,
        tikz={
        \node[label=below:{\tit{n}},
        draw,circle,
        scale=0.8,
        fit to=tree]{};
        \node[draw,circle,
        dashed,
        scale=0.85,
        fill=DG,fill opacity=0.2,
        fit to=tree]{};
        }
            [\tsc{k}2]
            [\tsc{nom}P
                [\tsc{k}1]
                [\#P
                    [\phantom{xxx}, roof]
                ]
            ]
        ]
      \end{forest}
      \\
      \toprule
      \tsc{acc} relative pronoun \tit{we-n}
      \\
      \cmidrule{1-1}
          \begin{forest} boom
          [\tsc{rel}P
              [\tsc{rel}P
                  [\phantom{x}\tit{we}\phantom{x}, roof]
              ]
              [\tsc{acc}P,
              tikz={
              \node[label=below:{\tit{n}},
              draw,circle,
              scale=0.8,
              fit to=tree]{};
              \node[draw,circle,
              dashed,
              scale=0.85,
              fit to=tree]{};
              }
                  [\tsc{k}2]
                  [\tsc{nom}P
                      [\tsc{k}1]
                      [\#P
                          [\phantom{xxx}, roof]
                      ]
                  ]
              ]
          ]
        \end{forest}
        \\
      \bottomrule
  \end{tabular}
  \end{adjustbox}
  \caption {Modern German \tsc{ext}\scsub{acc} vs. \tsc{int}\scsub{acc} → \tit{wen}}
  \label{fig:mg-int=ext}
\end{figure}

The extra light head consists of a single morpheme: \tit{n}.
The relative pronoun consists of two morphemes: \tit{we} and \tit{n}.
As usual, I circle the part of the structure that corresponds to a particular lexical entry, or I reduce the structure to a triangle, and I place the corresponding phonology below it.
I draw a dashed circle around the \tsc{acc}P, as it is the biggest possible element that is structurally contained in both the extra light head and the relative pronoun.

The extra light head consists of a single morpheme: the \tsc{acc}P.
This \tsc{acc}P is structurally contained in the relative pronoun. Therefore, the extra light head can be deleted. I signal the deletion of the extra light head by marking the content of its circle gray.
The surface element is the relative pronoun that bears the internal case: \tit{wen}.

For reasons of space I do not show the comparisons of the other matching situations. These are situations in which both the internal and external case are nominative or both the internal and external case are dative. The same logic as I showed in Figure \ref{fig:mg-int=ext} works for these situations too.

I continue with the situation in which the internal case is the more complex one.
Consider the example in \ref{ex:mg-acc-dat-rep}, in which the internal dative case competes against the external accusative case. The relative clause is marked in bold.
The internal case is dative, as the predicate \tit{vertrauen} `to trust' takes dative objects. The relative pronoun \tit{wem} `\ac{rel}.\ac{an}.\ac{dat}' appears in the dative case. This is the element that surfaces.
The external case is accusative, as the predicate \tit{einladen} `to invite' takes accusative objects. The extra light head \tit{n} `\ac{elh}.\ac{an}.\ac{acc}' appears in the accusative case. It is placed between square brackets because it does not surface.

\exg. Ich lade [n] ein, \tbf{wem} \tbf{auch} \tbf{Maria} \tbf{vertraut}.\\
1\ac{sg}.\ac{nom} invite.\ac{pres}.1\ac{sg}\scsub{[acc]} \tsc{elh}.\ac{an}.\ac{dat} {} \tsc{rp}.\ac{an}.\ac{dat} also Maria.\ac{nom} trust.\ac{pres}.3\ac{sg}\scsub{[dat]}\\
`I invite whoever Maria also trusts.' \flushfill{Modern German, adapted from \pgcitealt{vogel2001}{344}}\label{ex:mg-acc-dat-rep}

In Figure \ref{fig:mg-int-wins}, I give the syntactic structure of the extra light head at the top and the syntactic structure of the relative pronoun at the bottom.

\begin{figure}[htbp]
  \center
  \begin{adjustbox}{max height=0.9\textheight}
  \begin{tabular}[b]{c}
      \toprule
      \tsc{acc} extra light head \tit{n}
      \\
      \cmidrule{1-1}
      \begin{forest} boom
        [\tsc{acc}P,
        tikz={
        \node[label=below:{\tit{n}},
        draw,circle,
        scale=0.8,
        fit to=tree]{};
        \node[draw,circle,
        dashed,
        scale=0.85,
        fill=DG,fill opacity=0.2,
        fit to=tree]{};
        }
            [\tsc{k}2]
            [\tsc{nom}P
                [\tsc{k}1]
                [\#P
                    [\phantom{xxx}, roof]
                ]
            ]
        ]
      \end{forest}
      \\
      \toprule
      \tsc{dat} relative pronoun \tit{we-m}
      \\
      \cmidrule{1-1}
          \begin{forest} boom
            [\tsc{rel}P
                [\tsc{rel}P
                    [\phantom{x}\tit{we}\phantom{x}, roof]
                ]
                [\tsc{dat}P,
                tikz={
                \node[label=below:{\tit{m}},
                draw,circle,
                scale=0.85,
                fit to=tree]{};
                }
                    [\tsc{k}3]
                    [\tsc{acc}P,
                    tikz={
                    \node[draw,circle,
                    dashed,
                    scale=0.8,
                    fit to=tree]{};
                    }
                        [\tsc{k}2]
                        [\tsc{nom}P
                            [\tsc{k}1]
                            [\#P
                                [\phantom{xxx}, roof]
                            ]
                        ]
                    ]
                ]
            ]
        \end{forest}
        \\
      \bottomrule
  \end{tabular}
  \end{adjustbox}
   \caption {Modern German \tsc{ext}\scsub{acc} vs. \tsc{int}\scsub{dat} → \tit{wem}}
  \label{fig:mg-int-wins}
\end{figure}

The extra light head consists of a single morpheme: \tit{n}.
The relative pronoun consists of two morphemes: \tit{we} and \tit{m}.
I draw a dashed circle around the \tsc{acc}P, as it is the biggest possible element that is structurally a constituent in both the extra light head and the relative pronoun.

The extra light head consists of a single morpheme: the \tsc{acc}P.
This \tsc{acc}P is structurally contained in the relative pronoun. Therefore, the extra light can be deleted. I signal the deletion of the extra light head by marking the content of its circle gray.
The surface element is the relative pronoun that bears the internal case: \tit{wem}.

For reasons of space I do not show the comparisons of the other situations in which the internal case is more complex. These are situations in which the internal case is dative and the external case is nominative and in which the internal case is accusative and the external case is nominative. The same logic as I showed in Figure \ref{fig:mg-int-wins} works for these situations too.

I end with the situation in which the external case is the more complex one.
Consider the examples in \ref{ex:mg-dat-acc-rep}, in which the internal accusative case competes against the external dative case. The relative clauses are marked in bold. It is not possible to make a grammatical headless relative in this situation.
The internal case is accusative, as the predicate \tit{mögen} `to like' takes accusative objects. The relative pronoun \tit{wen} `\ac{rel}.\ac{an}.\ac{acc}' appears in the accusative case.
The external case is dative, as the predicate \tit{vertrauen} `to trust' takes dative objects. The extra light head \tit{m} `\ac{elh}.\ac{an}.\ac{dat}' appears in the dative case.
\ref{ex:mg-dat-acc-rep-rp} is the variant of the sentence in which the extra light head is absent (indicated by the square brackets) and the relative pronoun surfaces, which is ungrammatical.
\ref{ex:mg-dat-acc-rep-lh} is the variant of the sentence in which the relative pronoun is absent (indicated by the square brackets) and the extra light head surfaces, which is ungrammatical too.

\ex.\label{ex:mg-dat-acc-rep}
\ag. *Ich vertraue [m], \tbf{wen} \tbf{auch} \tbf{Maria} \tbf{mag}.\\
1\ac{sg}.\ac{nom} trust.\ac{pres}.1\ac{sg}\scsub{[dat]} \tsc{elh}.\ac{an}.\ac{dat} \tsc{rp}.\ac{an}.\ac{acc} also Maria.\ac{nom} like.\ac{pres}.3\ac{sg}\scsub{[acc]}\\
`I trust whoever Maria also likes.' \flushfill{Modern German, adapted from \pgcitealt{vogel2001}{345}}\label{ex:mg-dat-acc-rep-rp}
\bg. *Ich vertraue m, [\tbf{wen}] \tbf{auch} \tbf{Maria} \tbf{mag}.\\
1\ac{sg}.\ac{nom} trust.\ac{pres}.1\ac{sg}\scsub{[dat]} \tsc{elh}.\ac{an}.\ac{dat} \tsc{rp}.\ac{an}.\ac{acc} also Maria.\ac{nom} like.\ac{pres}.3\ac{sg}\scsub{[acc]}\\
`I trust whoever Maria also likes.' \flushfill{Modern German, adapted from \pgcitealt{vogel2001}{345}}\label{ex:mg-dat-acc-rep-lh}

In Figure \ref{fig:mg-ext-wins}, I give the syntactic structure of the extra light head at the top and the syntactic structure of the relative pronoun at the bottom.

\begin{figure}[htbp]
  \center
  \begin{adjustbox}{max height=0.9\textheight}
  \begin{tabular}[b]{c}
      \toprule
      \tsc{dat} extra light head \tit{m}
      \\
      \cmidrule{1-1}
      \begin{forest} boom
        [\tsc{dat}P,
        tikz={
        \node[label=below:{\tit{m}},
        draw,circle,
        scale=0.85,
        fit to=tree]{};
        }
            [\tsc{k}3]
            [\tsc{acc}P,
            tikz={
            \node[draw,circle,
            dashed,
            scale=0.8,
            fit to=tree]{};
            }
                [\tsc{k}2]
                [\tsc{nom}P
                    [\tsc{k}1]
                    [\#P
                        [\phantom{xxx}, roof]
                    ]
                ]
            ]
        ]
      \end{forest}
      \\
      \toprule
      \tsc{dat} relative pronoun \tit{we-n}
      \\
      \cmidrule{1-1}
          \begin{forest} boom
            [\tsc{rel}P
                [\tsc{rel}P
                    [\phantom{x}\tit{we}\phantom{x}, roof]
                ]
                [\tsc{acc}P,
                tikz={
                \node[label=below:{\tit{n}},
                draw,circle,
                scale=0.8,
                fit to=tree]{};
                \node[draw,circle,
                dashed,
                scale=0.85,
                fit to=tree]{};
                }
                    [\tsc{k}2]
                    [\tsc{nom}P
                        [\tsc{k}1]
                        [\#P
                            [\phantom{xxx}, roof]
                        ]
                    ]
                ]
            ]
        \end{forest}
        \\
      \bottomrule
  \end{tabular}
  \end{adjustbox}
   \caption {Modern German \tsc{ext}\scsub{dat} vs. \tsc{int}\scsub{acc} ↛ \tit{m}/\tit{wen}}
  \label{fig:mg-ext-wins}
\end{figure}

The extra light head consists of a single morpheme: \tit{m}.
The relative pronoun consists of two morphemes: \tit{we} and \tit{n}.
I draw a dashed circle around the \tsc{acc}P, as it is the biggest possible element that is structurally a constituent in both the extra light head and the relative pronoun.

In this case, the light head is not structurally contained in the relative pronoun.
The extra light head consists of a single morpheme: the \tsc{dat}P.
The relative pronoun only contains the \tsc{acc}P, and it lacks the \tsc{k}3 that makes a \tsc{dat}P. Since the weaker feature containment requirement is not met, the stronger constituent containment requirement cannot be met either.\footnote{
The relative pronoun contains the \tsc{acc}P, so in principle the \tsc{acc}P could be deleted. Then a new spellout has to be found for the \tsc{dat}P that only contains \tsc{f}3. As this lexical entry does not exist, the structure is rules out.
}

The relative pronoun is not structurally contained in the light head. It lacks the complete constituent and \tsc{rel}P.
Therefore, the extra light cannot be deleted, and the relative pronoun cannot be deleted either.
As a result, there is no grammatical headless relative possible.

For reasons of space I do not show the comparisons of the other situations in which the external case is more complex. These are situations in which the internal case is nominative and the external case is accusative and in which the internal case is nominative and the external case is dative. The same logic as I showed in Figure \ref{fig:mg-ext-wins} works for these situations too.

What I sketched in this section is true for the headless relatives with animates that I gave examples of in Chapter \ref{ch:typology}. It does not hold for headless relatives with inanimates, which I already briefly mentioned in Chapter \ref{ch:the-basic-idea}. I repeat the relevant example in \ref{ex:mg-nom-acc-syn}, including the extra light head which I assume to be there.

\exg. Ich erzähle [s], \tbf{was} \tbf{immer} \tbf{mir} \tbf{gefällt}.\\
 1\ac{sg}.\ac{nom} tell.\ac{pres}.1\ac{sg}\scsub{[acc]} \tsc{elh}.\ac{inan}.\tsc{acc} \tsc{rp}.\ac{inan}.\tsc{nom} ever 1\tsc{sg}.\tsc{dat} pleases.\ac{pres}.3\ac{sg}\scsub{[nom]}\\
 `I tell whatever pleases me.' \flushfill{Modern German, adapted from \pgcitealt{vogel2001}{344}}\label{ex:mg-nom-acc-syn}

In \ref{ex:mg-nom-acc-syn}, the internal nominative case competes against the external accusative case. The relative clause is marked in bold.
The internal case is nominative, as the predicate \tit{gefallen} `to please' takes nominative objects. The relative pronoun \tit{was} `\ac{rel}.\ac{inan}.\ac{nom}' appears in the nominative case. This is the element that surfaces
The external case is accusative, as the predicate \tit{erzählen} `to tell' takes accusative objects. The extra light head \tit{s} `\ac{elh}.\ac{inan}.\ac{acc}' appears in the accusative case. It is placed between square brackets because it does not surface.
For inanimates, there is a syncretism between nominative and accusative. In these cases, the extra light head can be deleted via formal containment. In what follows, I briefly describe the comparison.

The inanimate accusative extra light head consists of a single morpheme (\tit{s}). The inanimate nominative relative pronoun consists of two morphemes (\tit{wa} and \tit{s}) (see Section \ref{sec:mg-w-er-cv}).
The extra light head (the \tsc{acc}P realized by \tit{s}) is formally contained in the relative pronoun (the \tsc{rel}P realized by \tit{wa-s}).
Therefore, the extra light head can be deleted, and the surface element is the relative pronoun that bears the external case: \tit{was}.


\section{Summary and discussion}

Modern German is an example of an internal-only type of language. This means that headless relatives are grammatical in the language, as long as the internal and external case match or the internal case is the more complex one.

I derive this from the internal syntax of light heads and relative pronouns in Modern German. The features of the light head are spelled out by a single lexical entry, which spells out phi and case features. The features of the relative pronoun are spelled out by the same lexical entry plus one which amongst other spells out a relative feature. The internal syntax of the Modern German light head and relative pronoun are shown in Figure \ref{fig:rel-lh-mg-sum}.

\begin{figure}[htbp]
  \center
  \begin{tabular}[b]{ccc}
      \toprule
      light head & & relative pronoun \\
      \cmidrule(lr){1-1} \cmidrule(lr){3-3}
      \begin{forest} boom
        [\tsc{k}P,
        tikz={
        \node[label=below:\tit{r/n/m},
        draw,circle,
        scale=0.75,
        fit to=tree]{};
        }
            [\tsc{k}]
            [ϕP
                [\phantom{xxx}, roof, baseline]
            ]
        ]
      \end{forest}
      & \phantom{x} &
      \begin{forest} boom
        [\tsc{rel}P, s sep=15mm
            [\tsc{rel}P,
            tikz={
            \node[label=below:\tit{we},
            draw,circle,
            scale=0.75,
            fit to=tree]{};
            }
                [\phantom{xxx}, roof]
            ]
            [\tsc{k}P,
            tikz={
            \node[label=below:\tit{r/n/m},
            draw,circle,
            scale=0.75,
            fit to=tree]{};
            }
                [\tsc{k}]
                [ϕP
                    [\phantom{xxx}, roof, baseline]
                ]
            ]
        ]
      \end{forest}\\
      \bottomrule
  \end{tabular}
   \caption {\tsc{elh} and \tsc{rp} in Modern German (repeated)}
  \label{fig:rel-lh-mg-sum}
\end{figure}

A crucial characteristic of internal-only languages such as Modern German is that they have a portmanteau for phi and case features. Therefore, the light head is structurally contained in the relative pronoun when the internal and the external case match and when the internal case is the more complex one. As a result, the light head can be deleted, and the relative pronoun can surface, bearing the internal case.

When the internal case is the more complex one, neither the light head nor the relative pronoun is structurally contained in the other element. None of the elements can be deleted, and there is no grammatical headless relative possible.

It remains an open question why the extra light head in \ref{ex:mg-real-base} cannot surface and needs to be deleted. A possible answer comes from drawing a parallel between the two possible light heads and two other forms that are morphologically identical to them. These are the strong and weak definite in \citet{schwarz2009}. \posscitet{schwarz2009} strong definite is anaphoric in nature, and the weak definite encodes uniqueness. I give an example of a strong definite in \ref{ex:mg-florian-strong}. The strong definite is \tit{dem}, and \tit{dem Freund} `the friend' refers back to the linguistic antecedent \tit{einen Freund} `a friend'.

\exg. Hans hat heute einen Freund zum Essen mit nach Hause gebracht. Er hat uns vorher ein Foto von dem Freund gezeigt.\\
Hans has today a friend {to the} dinner with to home brought he has us beforehand a photo of the\scsub{strong} friend shown\\
`Hans brought a friend home for dinner today. He had shown us a photo of the friend beforehand.'\label{ex:mg-florian-strong}

Weak definites are used when situational uniqueness is involved. This uniqueness can be global or within a restricted domain. I give two examples in \ref{ex:mg-florian-weak}. In \ref{ex:mg-florian-weak-hund}, the dog is unique in this specific situation of the break-in. In \ref{ex:mg-florian-weak-mond}, the moon is unique for us people on the planet. As such, the weak definites \tit{m} in \tit{vom Hund} `by the dog' and in \tit{zum Mond} `to the moon' are used.

\ex.\label{ex:mg-florian-weak}
\ag. Der Einbrecher ist {zum Glück} vom Hund verjagt worden.\\
the burglar is luckily {by the\scsub{weak}} dog {chased away} been\\
`Luckily, the burglar was chased away by the dog.'\label{ex:mg-florian-weak-hund}
\bg. Armstrong flog als erster zum Mond.\\
Armstrong flew as {first one} {to the\scsub{weak}} moon\\
`Armstrong was the first one to fly to the moon.' \flushfill{Modern German, \pgcitealt{schwarz2009}{40}}\label{ex:mg-florian-weak-mond}

\tit{Dem} in \ref{ex:mg-florian-strong} is morphologically identical to the demonstrative, and the two instances of \tit{m} in \ref{ex:mg-florian-weak} are identical to the extra light head.

The meaning of \posscitet{schwarz2009} strong definite seems similar to the meaning of the demonstrative in the \tit{den}-\tit{wen} relative.
I do not see right away how the extra light head in headless relatives could encode uniqueness. One possibility is that the feature content of his and my form differs slightly after all. Another possibility is that the fact that his form combines with a preposition and an overt nouns leads to a change in interpretation.

This brings me back to why the extra light head never surfaces as a head of the relative clause. Consider the sentence in \ref{ex:mg-florian-fritz-m}.

\exg. *Fritz ist jetzt im Haus, das er sich letztes Jahr gebaut hat.\\
Fritz is now {in the} house that the \tsc{refl} last year built has\\
`Fritz is now in the house that he built last year.' \flushfill{Modern German, \pgcitealt{schwarz2009}{22} after \pgcitealt{hartmann1978}{77}}\label{ex:mg-florian-fritz-m}

Just as the extra light head, the weak definite cannot be the head of a relative clause.

Now consider \ref{ex:mg-florian-fritz-dem}.

\exg. Fritz ist jetzt in dem Haus, das er sich letztes Jahr gebaut hat.\\
Fritz is now in the house that the \tsc{refl} last year built has\\
`Fritz is now in the house that he built last year.' \flushfill{Modern German, \pgcitealt{schwarz2009}{22} after \pgcitealt{hartmann1978}{77}}\label{ex:mg-florian-fritz-dem}

If the weak definite is replaced with the strong definite, the sentence becomes grammatical.
Whatever causes the ungrammaticality of \ref{ex:mg-florian-fritz-m} also rules out a light-headed relative headed by the extra light head.
