% !TEX root = thesis.tex

\chapter{A recurring pattern}\label{ch:recurring}

This chapter introduces the pattern that forms the focus of the first part of the dissertation. In Section \ref{sec:pattern-rels} I show that case competition in headless relatives adheres to the case scale in \ref{ex:case-scale-intro}.

\ex. \ac{nom} < \ac{acc} < \ac{dat}\label{ex:case-scale-intro}

Then I show that this pattern is not unique to headless relatives. It appears in more syntactic and morphological phenomena. Section \ref{sec:impl-hier} discusses two implicational hierarchies that show the same case ordering. The hierarchies concern agreement and relativization across languages. Section \ref{sec:case-morphology} shows that the case scale also shows up in morphological patterns. It can be observed in patterns of syncretism and in morphological containment.


\section{In headless relatives}\label{sec:pattern-rels}

As the name suggests, headless relatives are relative clauses that lack an (overt) head. The internal case, the case from the relative clause, and the external case, the case from the main clause, compete to surface on the relative pronoun. It has been argued in the literature that the two competing cases always adhere a to particular case scale \citep[cf.][]{harbert1978,pittner1995,vogel2001,grosu2003,caha2019,bergsma2019}. This is the scale I gave in the introduction, repeated here in \ref{ex:case-scale}. Elements more to the right on this scale win over elements more to the left on this scale.\footnote{
In the literature about headless relatives, the genitive is often discussed together with the nominative, accusative and dative \citep[cf.][]{harbert1978,pittner1995}. In this dissertation I do not discuss the genitive. The reason is that I restrict myself to cases that appear in all possible case competition combinations. As the genitive does not fulfill that requirement, it is therefore excluded. In Chapter \ref{ch:conclusion} I briefly return to the issue.
}

\ex. \ac{nom} < \ac{acc} < \ac{dat}\label{ex:case-scale}

This can be reformulated as follows. In a competition, dative wins over accusative, and dative wins over nominative. Additionally, accusative wins over nominative. In this section I illustrate this scale with examples. When two cases compete, the relative pronoun always appears in the case more to the right on the case scale. It does not matter whether it is the internal or the external case. I illustrate this with examples from headless relatives in Gothic \citep{harbert1978}.

I start with the competition between dative and accusative. Following the case scale in \ref{ex:case-scale}, the relative pronoun appears in dative case and never in accusative.

Consider the example in \ref{ex:gothic-acc-dat-rep}, repeated from the introduction. In this example, the internal case is dative and the external case is accusative.
The relative clause and the relative pronoun are marked in bold.
The internal case is dative. The preposition \tit{ana} `on' takes dative complements.
The external case is accusative. The predicate \tit{ushafjands} `picking up' takes accusative objects.
The relative pronoun \tit{þamm(a)} `who.\ac{dat}' appears in the internal case: the dative.
Examples, in which the relative pronoun appears in accusative case, the internal case is dative and the external case is accusative, are unattested.

\exg. ushafjands \tbf{ana} \tbf{þamm} \tbf{-ei} \tbf{lag}\\
 {picking up}\scsub{[acc]} on\scsub{[dat]} what.\ac{dat} \-\ac{comp} lay\\
 `picking up (that) on which he lay' \flushfill{Gothic, \ac{luke} 5:25, adapted from \pgcitealt{harbert1978}{343}}\label{ex:gothic-acc-dat-rep}

Consider the example in \ref{ex:gothic-dat-acc-rep}, repeated from the introduction. In this example, the internal case is accusative and the external case is dative.
The relative clause is marked in bold, the relative pronoun is not.
The internal case is accusative. The predicate \tit{qiþiþ} `say' takes accusative objects.
The external case is dative. The predicate \tit{taujau} `do' takes dative indirect objects.
The relative pronoun \tit{þamm} `who.\ac{dat}' appears in the external case: the dative.
Examples, in which the relative pronoun appears in accusative case, the internal case is accusative and the external case is dative, are unattested.

\exg. hva nu wileiþ ei taujau þamm \tbf{-ei} \tbf{qiþiþ} \tbf{þiudan} \tbf{Iudaie}?\\
 what now want that do\scsub{[dat]} who.\ac{dat} -\ac{comp} say\scsub{[acc]} king {of Jews}\\
 `what now do you wish that I do to (him) whom you call King of the Jews?' \flushfill{Gothic, \ac{mark} 15:12, adapted from \pgcitealt{harbert1978}{339}}\label{ex:gothic-dat-acc-rep}

I continue with the competition between dative and nominative. Following the case scale in \ref{ex:case-scale}, the relative pronoun appears in dative case and never in nominative.

Consider the example in \ref{ex:gothic-dat-nom}, in which the internal case is dative and the external case is nominative.
The relative clause and the relative pronoun are marked in bold.
The internal case is dative. The predicate \tit{fraletada} `is forgiven' takes dative objects.
The external case is nominative. The predicate \tit{frijod} `loves' takes nominative subjects.
The relative pronoun \tit{þamm(a)} `who.\ac{dat}' appears in the internal case: the dative.
Examples, in which the relative pronoun appears in nominative case, the internal case is dative and the external case is nominative, are unattested.

\exg. iþ \tbf{þamm} \tbf{-ei} \tbf{leitil} \tbf{fraletada} leitil frijod\\
 but who.\ac{dat} -\ac{comp} little {is forgiven}\scsub{[dat]} little loves\scsub{[nom]}\\
 `but the one whom little is forgiven loves little' \flushfill{Gothic, \ac{luke} 7:47, adapted from \pgcitealt{harbert1978}{342}}\label{ex:gothic-dat-nom}

 Consider the example in \ref{ex:gothic-nom-dat}, in which the internal case is nominative and the external case is dative.
 The relative clause is marked in bold, the relative pronoun is not.
 The internal case is nominative. The predicate \tit{sind fraþjaiþ} `are above' takes a nominative subject.
 The external case is dative. The predicate \tit{fraþjaiþ} `think on' takes dative indirect objects.
 The relative pronoun \tit{þaim} `what.\ac{dat}' appears in the external case: the dative.
 Examples, in which the relative pronoun appears in nominative case, the internal case is nominative and the external case is dative, are unattested.

 \exg. þaim \tbf{-ei} \tbf{iupa} \tbf{sind} fraþjaiþ \\
  what.\ac{dat} -\ac{comp} above are\scsub{[nom]} {think on}\scsub{[dat]}\\
  `set your mind on those which are above' \flushfill{Gothic, \ac{col} 3:2, adapted from \pgcitealt{harbert1978}{339}}\label{ex:gothic-nom-dat}

I finish with the competition between accusative and nominative. Following the case scale in \ref{ex:case-scale}, the relative pronoun appears in accusative case and never in nominative.

Consider the example in \ref{ex:gothic-acc-nom}, in which the internal case is accusative and the external case is nominative.
The relative clause and the relative pronoun are marked in bold.
The internal case is accusative. The predicate \tit{frijos} `love' takes accusative objects.
The external case is nominative. The predicate \tit{siuks ist} `is sick' takes nominative subjects.
The relative pronoun \tit{þan} `who.\ac{acc}' appears in the internal case: the accusative.
Examples, in which the relative pronoun appears in nominative case, the internal case is accusative and the external case is nominative, are unattested.

\exg. \tbf{þan} \tbf{-ei} \tbf{frijos} siuks ist\\
 who.\ac{acc} -\ac{comp} love\scsub{[acc]} sick is\scsub{[nom]}\\
 `the one whom you love is sick' \flushfill{Gothic, \ac{john} 11:3, adapted from \pgcitealt{harbert1978}{342}}\label{ex:gothic-acc-nom}

Consider the example in \ref{ex:gothic-nom-acc}, in which the internal case is nominative and the external case is accusative.
The relative clause is marked in bold, the relative pronoun is not.
The internal case is nominative. The predicate \tit{ist us Laudeikaion} `is from Laodicea' takes nominative subjects.
The external case is accusative. The predicate \tit{ussiggwaid} `read' takes accusative objects.
The relative pronoun \tit{þo} `what.\ac{acc}' appears in the external case: the accusative.
Examples, in which the relative pronoun appears in nominative case, the internal case is nominative and the external case is accusative, are unattested.

\exg. jah þo \tbf{-ei} \tbf{ist} \tbf{us} \tbf{Laudeikaion} jus ussiggwaid\\
 and what.\ac{acc} -\ac{comp} is\scsub{[nom]} from Laodicea you read\scsub{[acc]}\\
 `and read that which is from Laodicea' \flushfill{Gothic, \ac{col} 4:16, adapted from \pgcitealt{harbert1978}{357}}\label{ex:gothic-nom-acc}

A summary of the Gothic data as a whole is given in Table \ref{tbl:summary-gothic}. The left column shows the internal case between square brackets. The upper row shows the external case between square brackets. The other cells indicate the case of the relative pronoun. The diagonal is left blank, because these are instances in which the internal and external case match.
The remaining six cells show instances where the internal and external case differ. Within the cells, two cases are given. The case in the lower left corner stands for the relative pronoun in the internal case. The case in the upper right corner stands for the relative pronoun in the external case. The grammatical examples are marked in gray. The unattested examples are marked with an asterix and are unmarked.\footnote{
Throughout this dissertation * stands for 'not found in natural language'. For extinct languages this means that there are no attested examples. For modern languages it means that the examples are ungrammatical.
}

\begin{table}[ht]
  \center
  \caption {Case competition in Gothic headless relatives}
    % !TEX root = ../thesis.tex

\begin{tabular}{c|c|c|c}
  \toprule
    \diagbox[linecolor=white]{\ac{int}}{\ac{ext}}
        & [\ac{nom}]
        & [\ac{acc}]
        & [\ac{dat}]
        \\ \cmidrule{1-4}
    [\ac{nom}]
        & 
        & \diagbox[linecolor=white]{*\ac{nom}}{\colorbox{LG}{\ac{acc}}}
        & \diagbox[linecolor=white]{*\ac{nom}}{\colorbox{LG}{\ac{dat}}}
        \\ \cmidrule{1-4}
    [\ac{acc}]
        & \diagbox[linecolor=white]{\colorbox{LG}{\ac{acc}}}{*\ac{nom}}
        &
        & \diagbox[linecolor=white]{*\ac{acc}}{\colorbox{LG}{\ac{dat}}}
        \\ \cmidrule{1-4}
    [\ac{dat}]
        & \diagbox[linecolor=white]{\colorbox{LG}{\ac{dat}}}{*\ac{nom}}
        & \diagbox[linecolor=white]{\colorbox{LG}{\ac{dat}}}{*\ac{acc}}
        &
        \\
  \bottomrule
\end{tabular}

    \label{tbl:summary-gothic}
\end{table}

The three instances in the lower left corner correspond to the examples \ref{ex:gothic-acc-nom}, \ref{ex:gothic-dat-nom} and \ref{ex:gothic-dat-acc-rep}. In the attested examples, the relative pronoun appears in the internal case.
The three instances in the upper right corner correspond to the examples in \ref{ex:gothic-nom-acc}, \ref{ex:gothic-nom-dat} and \ref{ex:gothic-acc-dat-rep}. In the attested examples, the relative pronoun appears in the external case.

\begin{table}[ht]
  \center
  \caption {Summary of Gothic matching headless relative data}
    \begin{tabular}{c|c|c|c}
      \toprule
            & [\ac{nom}]
            & [\ac{acc}]
            & [\ac{dat}]
            \\ \cmidrule{1-4}
        [\ac{nom}]
            &
            & \ac{acc}
            & \ac{dat}
            \\ \cmidrule{1-4}
        [\ac{acc}]
            & \ac{acc}
            &
            & \ac{dat}
            \\ \cmidrule{1-4}
        [\ac{dat}]
            & \ac{dat}
            & \ac{dat}
            &
            \\
      \bottomrule
    \end{tabular}
    \label{tbl:summary-gothic-1}
\end{table}

To sum up, case competition in headless relative is subject to the case scale, repeated in \ref{ex:case-scale-rep}.

\ex. \ac{nom} < \ac{acc} < \ac{dat}\label{ex:case-scale-rep}

If two cases compete, dative wins over accusative and nominative, and accusative wins over nominative. In this section I gave examples from Gothic that illustrate this. As I mentioned in the introduction of this section, this case scale is not specific for Gothic, but it holds across languages (cf. see \citealt{pittner1995} for Modern, Middle High and Old High German, \citealt{grosu2003} for Ancient Greek and \citealt{daskalaki2011} for Modern Greek).\footnote{
These languages differ from Gothic in that they are subject to an additional constraint. That is, they only allow either the internal or the external case to win case competitions. If the other case is more to the right on the case scale \ref{ex:case-scale-rep}, the result is ungrammatical.
Old High German is an example of a language that only allows the external case to win the case competition. If the internal case is more to the right on the case scale, the headless relative is ungrammatical. Modern German is an example of a language that only allows the internal case to win the case competition. If the external case is more to the right on the case scale, the headless relative is ungrammatical.
This topic is the main focus of Part \ref{part:complexity} of this dissertation.}

In the remainder of this chapter I show that headless relatives are not the only place where the case scale shows up. Instead, it appears with more syntactic phenomena. Moreover, exactly this scale is also reflected in morphology.


\section{In syntax}\label{sec:impl-hier}

In this section I discuss two additional syntactic phenomena that reflect the \ac{nom} < \ac{acc} < \ac{dat} scale. The first one is an implicational hierarchy that concerns agreement. The second one is an implicational hierarchy about relativization.


\subsection{Agreement}

Agreement can be seen as ``a systematic covariance between a semantic or formal property of one element and a formal property of another'' \citep{steel1978}. Put differently, the shape of one element changes according to some properties of an element it relates to. In this section I discuss the agreement between a predicate and its arguments.

It differs per language with how many of its arguments a predicate agrees. However, it is not random with which agreement takes place. Instead, there is an implicational hierarchy that is identical to the one observed for headless relatives: \ac{nom} < \ac{acc} < \ac{dat}.

\citet{moravcsik1978} formulated the implicational hierarchy in terms of grammatical functions subject, direct object and indirect object.\footnote{
\citet{moravcsik1978} also included adverbs on the lowest end of the hierarchy. I leave them out here, because they are not relevant for the discussion.
}
The hierarchy is schematically represented in Figure \ref{fig:agr-sub-do-io}. It should be read as follows: if a language allows the predicate to agree with the argument in a particular circle, it also allows the predicate to agree with the argument in the circle around it.

\begin{figure}[ht]
  \centering
  \begin{tikzpicture}
    \draw (0,1) circle (2.25);
    \draw [fill opacity=0.4, fill=LG] (0,0.5) circle (1.75);
    \draw [fill opacity=0.4, fill=DG] (0,0) circle (1.25);

    \node[] at (0,2.75) {subject};
    \node[] at (0,1.5) {direct object};
    \node[align=center] at (0,0) {indirect\\ object};
  \end{tikzpicture}
  \caption{\posscitealt{moravcsik1978} schema}
  \label{fig:agr-sub-do-io}
\end{figure}

Then, there are four types of languages possible: first, a language that does not show any agreement; second, a language that shows agreement only with the subject and not with the direct and indirect object; third, a language that shows agreement with the subject and direct object but not with the indirect object; and fourth, a language that shows agreement with the subject, the direct object and the indirect object.

The implicational hierarchy holds for languages, not for sentences. That is, it is not the case that in a language of a particular type all instances of the grammatical function show agreement. To be more precise, in a language of the second type, that only shows agreement with the subject, not all subjects have to show agreement. Particular types of subject, such as experiencer subjects often do not show any agreement.

Japanese is an example of a language that does not show any agreement on the predicate. An example is given in \ref{ex:japanese-agr}. The predicate \tit{okutta} `sent' does not agree with the subject \tit{Tarooga} `Taro', with the direct object \tit{nimotuo} `package' or with the indirect object \tit{Hanakoni} `Hanako'.

\exg. Taroo-ga Hanako-ni nimotu-o okutta.\\
 Taro-\ac{nom} Hanako-\ac{dat} package-\ac{acc} sent\\
 `Taro sent Hanako a package.' \flushfill{Japanese, \pgcitealt{miyagawa2004}{5}}\label{ex:japanese-agr}

German is an example of a language that shows agreement with the subject of the clause. An example is given in \ref{ex:german-agr}. The predicate \tit{gibst} `give' contains the morpheme \tit{-st}, marked in bold. This morpheme is the agreement morpheme for second person singular subjects (in the present tense). The predicate \tit{gibst} `give' agrees in person and number with the subject \tit{du} `you'. There is no agreement with the direct object \tit{das Buch} `the book' or the indirect object \tit{mir} `me'.

\exg. Du gib \tbf{-st} mir das Buch.\\
 you give -2\ac{sg}.\ac{pres} me the book\\
 `You give me the book.' \flushfill{German}\label{ex:german-agr}

Hungarian is an example of a language that shows agreement with the subject and the direct object of a clause. An example is given in \ref{ex:hungarian-agr}. The predicate \tit{adom} `give' contains the morpheme \tit{-om}, marked in bold. This is a portmonteau morpheme for a first person singular subject and a third person object agreement. The predicate \tit{adom} `give' agrees with the subject \tit{én} `I' and the direct object \tit{a könyvet} `the book'. There is no agreement with the indirect object \tit{neked} `you'. Agreement with the the first person singular subject \tit{én} `I' and second person singular indirect object \tit{neked} `you.\ac{dat}.\ac{sg}' is ungrammatical, as indicated by the ungrammaticality of \tit{-lak}.

\exg. (Én) neked ad \tbf{-om}/ *-lak a könyv -et\\
 I you.\ac{dat}.\ac{sg} give -1\ac{sg}.\ac{subj}>3.\ac{obj} -1\ac{sg}.\ac{subj}>2.\ac{obj} the book -\ac{acc}\\
 `I give you the book.' \flushfill{Hungarian, András Bárány p.c.}\label{ex:hungarian-agr}

Basque is an example of a language that shows agreement with the subject, the direct object and the indirect object. Basque is an ergative-absolutive language, so in transitive clauses subjects are marked as ergative and objects are marked as absolutive. An example from the Bizkaian dialect is given in \ref{ex:basque-agr}. The stem of the auxiliary \tit{aus} combines with the morphemes \tit{d-}, \tit{-ta} and \tit{-zu}, marked in bold. The morpheme \tit{d-} is the agreement morpheme for third person singular as direct objects, which is here \tit{liburua} `the book'. The morpheme \tit{-ta} is the agreement morpheme for first person singular indirect objects, which is here \tit{niri} `me'. The morpheme \tit{-zu} is the agreement morpheme for second person singular ergative subjects, which is here \tit{zuk} `you'.

\exg. Zu-k ni-ri liburu-a emon \tbf{d} -aus \tbf{-ta} \tbf{-zu}.\\
 you-\ac{erg} me-\ac{dat} book-\ac{def}.\ac{abs} given \ac{abs}.3\ac{sg} -\ac{aux} -\ac{dat}.1\ac{sg} -\ac{erg}.2\ac{sg}\\
 `You gave me the book.' \flushfill{Bizkaian Basque, adapted from \pgcitealt{arregi2004}{45}}\label{ex:basque-agr}

Putting the languages in \posscitet{moravcsik1978} figure gives the following result.

 \begin{figure}[ht]
   \centering
   \begin{tikzpicture}
     \draw (0,1) circle (2.25);
     \draw [fill opacity=0.4, fill=LG] (0,0.5) circle (1.75);
     \draw [fill opacity=0.4, fill=DG] (0,0) circle (1.25);

     \node[] at (0,2.75) {subject};
     \node[] at (0,1.5) {direct object};
     \node[align=center] at (0,0) {indirect\\ object};

     \node[] at (2.5,3) {\footnotesize{● Japanese}};
     \node[] at (2.25,2) {\footnotesize{● German}};
     \node[] at (2,1) {\footnotesize{● Hungarian}};
     \node[] at (1.375,0) {\footnotesize{● Basque}};
   \end{tikzpicture}
   \caption{\posscitealt{moravcsik1978} schema with languages}
   \label{fig:agr-sub-do-io-lang}
 \end{figure}

\citet{gilligan1987} performed a typological study among 100 genetically and areally diverse languages, which confirms the picture. The results are shown in Table \ref{tbl:agr-typo}. There are 23 languages that do not show any agreement, like Japanese. There are 31 languages that show agreement only with the subject and not with the direct and indirect object, like German. There are 25 languages that show agreement with the subject and direct object but not with the indirect object, like Hungarian. There are 23 languages that show agreement with the subject, the direct object and the indirect object, like Basque.

 \begin{table}[ht]
   \center
   \caption {Agreement accessibility}
     \begin{tabular}[t]{ccccc}
       \toprule
           \multicolumn{3}{c}{agreement with} &              &         \\
       \cmidrule{1-3}
                    & direct & indirect       & number       &         \\
           subject  & object & object         & of languages & example \\
       \cmidrule{1-3} \cmidrule{4-4} \cmidrule{5-5}
           *    & * & * & 23  & Japanese    \\
           ✔    & * & * & 31  & German     \\
           ✔    & ✔ & * & 25  & Hungarian  \\
           ✔    & ✔ & ✔ & 23  & Basque     \\
           ✔    & * & ✔ & (1) & -          \\
           {*}  & ✔ & ✔ & 0   & -          \\
           {*}  & x & * & 0   & -          \\
           {*}  & * & ✔ & 0   & -          \\
       \bottomrule
     \end{tabular}
     \label{tbl:agr-typo}
 \end{table}

It is often the case that subjects appear in nominative case, and that direct objects appear in accusative. However, this is not always the case. Subjects can be non-nominative and direct objects can be non-accusative. \citet{bobaljik2006} argues that the implicational hierarchy is more accurate if it is stated in terms of case rather than grammatical function. He argues for the picture shown in \ref{fig:agr-nom-acc-dat}.\footnote{
Actually, \citet{bobaljik2006} also includes ergative-absolutive languages, and argues for the picture in Figure \ref{fig:agr-def-dep-dat}. Default case can be nominative or absolutive case (in transitive clauses), and dependent case can be accusative and ergative case.

\begin{figure}[H]
  \centering
  \footnotesize{
  \begin{tikzpicture}
    \draw (0,1) circle (2.25);
    \draw [fill opacity=0.4, fill=LG] (0,0.5) circle (1.75);
    \draw [fill opacity=0.4, fill=DG] (0,0) circle (1.25);

    \node[] at (0,2.75) {default case};
    \node[] at (0,1.5) {dependent case};
    \node[] at (0,0) {dative};
  \end{tikzpicture}
  }
  \caption{\footnotesize{\posscitealt{bobaljik2006} actual schema}}
  \label{fig:agr-def-dep-dat}
\end{figure}

In the languages I discuss in this dissertation, I focus on languages that have nominative as default case and accusative as dependent case, so Figure \ref{fig:agr-nom-acc-dat} suffices.}

\begin{figure}[ht]
  \centering
  \begin{tikzpicture}
    \draw (0,1) circle (2.25);
    \draw [fill opacity=0.4, fill=LG] (0,0.5) circle (1.75);
    \draw [fill opacity=0.4, fill=DG] (0,0) circle (1.25);

    \node[] at (0,2.75) {\ac{nom}};
    \node[] at (0,1.75) {\ac{acc}};
    \node[] at (0,0) {\ac{dat}};
  \end{tikzpicture}
  \label{fig:agr-nom-acc-dat}
  \caption{\posscitealt{bobaljik2006} simplified schema}
\end{figure}

\citeauthor{bobaljik2006} gives examples of situations in which grammatical function and morphological case do not match. In these situations, case seem to capture the facts for the implicational hierarchy, and grammatical function does not. I give two examples from Icelandic that illustrate this point.

Icelandic is a language that has dative subjects. If agreement takes place with the grammatical subject, it is expected that the dative subject agrees with the predicate. This is not what happens, as illustrated in \ref{ex:icelandic-dat-subj}. The dative subject \tit{morgum studentum} `many students' is plural. The sentence is ungrammatical with the predicate \tit{líka} `like' inflecting for plural as well. So, the dative subject does not agree in number with the predicate. In other words, it is not the grammatical subject that shows agreement.

\exg. *Morgum studentum líka verkið.\\
 many students.\ac{dat} like.\ac{pl} job.\ac{nom} \\
`Many students like the job.' \flushfill{\pgcitealt{harley1995}{208}}\label{ex:icelandic-dat-subj}

Instead, it is the nominative object that agrees with the verb. This is illustrated in \ref{ex:icelandic-nom-obj}. The dative subject \tit{konunginum} `the king' is singular. The nominative object \tit{ambáttir} `slaves' is plural. The predicate \tit{voru} `were' is inflected for plural, agreeing with the nominative object. This is expected if morphological case determines agreement: it is the nominative that shows agreement. The grammatical role, the fact that this nominative is an object, does not influence agreement.

\exg. Um veturinn voru konunginum gefnar ambáttir\\
In the.winter were.\ac{pl} the.king.\ac{dat} given slaves.\ac{nom}\\
`In the winter, the king was given (female) slaves.' \flushfill{\pgcitealt{zaenen1985}{112}}\label{ex:icelandic-nom-obj}

In conclusion, agreement follows the same implicational hierarchy as the case scale in headless relatives: \ac{nom} < \ac{acc} < \ac{dat}.


\subsection{Relativization}

Relativization refers to the process in which a relative clause is derived from a non-relative clause. An example of the non-relative clause is given in \ref{ex:rel-non}. The relative clause derived from that is shown in \ref{ex:rel-realati}. The head of the relative clause is \tit{woman} and precedes the clause. The relative pronoun follows the head. The head of the head does not appear in the relative clause anymore.

\ex.
\a. You like the woman. \label{ex:rel-non}
\b. \tbf{the} \tbf{woman}, who you like \label{ex:rel-realati}

In \ref{ex:rel-realati}, it is the object of the clause that is relativized. It differs per language which elements can be relativized with a particular strategy. Just like the distribution was not random for agreement, it is not random which elements can be relativized. Instead, there is an implicational hierarchy that is identical to the one observed for the case scale: \ac{nom} < \ac{acc} < \ac{dat}.

\citet{keenan1977} formulated the implicational hierarchy in terms of the grammatical functions subject, direct object and indirect object.\footnote{
\citet{keenan1977} also included obliques, possessives and objects of comparison on the lowest end of the hierarchy. I leave them out here, because they are not relevant for the discussion.
}
The implicational hierarchy is schematically represented in Figure \ref{fig:relativization}. It should be read as follows: if a language allows a particular relativization strategy of the grammatical function in a particular circle, it also allows this relativization strategy of the grammatical function of the circle around it. The languages in the figure give examples of the circles they are in.

\begin{figure}[ht]
  \centering
  \begin{tikzpicture}
    \draw (0,1) circle (2.25);
    \draw [fill opacity=0.4, fill=LG] (0,0.5) circle (1.75);
    \draw [fill opacity=0.4, fill=DG] (0,0) circle (1.25);

    \node[] at (0,2.75) {subject};
    \node[] at (0,1.5) {direct object};
    \node[align=center] at (0,0) {indirect\\ object};

    \node[] at (2.25,2) {\footnotesize{● Malagasy}};
    \node[] at (2,1) {\footnotesize{● Malay}};
    \node[] at (1.375,0) {\footnotesize{● Basque}};
  \end{tikzpicture}
  \caption{Schema for relativization}
  \label{fig:relativization}
\end{figure}

There are four types of languages possible: first, a language that allows only the subject to be relativized with a particular strategy and not the direct and indirect object; second, a language that allows the subject and direct object to be relativized with a particular strategy but not the indirect object; and third, a language that allows the subject, the direct object and the indirect object to be relativized with a particular strategy.

Malagasy is an example of a language that allows subjects to be relativized using a particular strategy, but not direct and indirect objects. \ref{ex:malagasy-decl} is an example of a declarative sentence in Malagasy. It is a transitive sentence that contains the subject \tit{ny mpianatra} `the student' and the direct object \tit{ny vehivavy} `the woman'.

\exg. Nahita ny vehivavy ny mpianatra.\\
 saw the woman the student\\
 `The student saw the woman.' \flushfill{Malagasy, \pgcitealt{keenan1977}{70}}\label{ex:malagasy-decl}

In \ref{ex:malagasy-subj}, the subject from the declarative sentence, marked in bold, is relativized. The subject \tit{ny mpianatra} `the student' appears in the first position of the clause. It is followed by the invariable relativizer \tit{izay} `that'. After that, the rest of the relative clause follows, in this case \tit{nahita ny vehivavy} `saw the woman'.

\exg. \tbf{ny} \tbf{mpianatra} izay nahita ny vehivavy\\
 the student that saw the woman\\
 `the student that saw the woman' \flushfill{Malagasy, \pgcitealt{keenan1977}{70}, my boldfacing}\label{ex:malagasy-subj}

The object of \ref{ex:malagasy-decl} cannot be relativized in the same way, as shown in \ref{ex:malagasy-no-do}. Here the object \tit{ny vehivavy} `the woman', marked in bold, appears in the first position of the clause. It is again followed by the relativizer \tit{izay} `that' and the rest of the relative clause, which is here \tit{nahita ny mpianatra} `saw the student'. This example is ungrammatical.

\exg. *\tbf{ny} \tbf{vehivavy} izay nahita ny mpianatra\\
 the woman that saw the student\\
 `the woman that the student saw' \flushfill{Malagasy, \pgcitealt{keenan1977}{70}, my boldfacing}\label{ex:malagasy-no-do}

Later in this section I draw the parallel between subject and nominative, direct object and accusative and indirect object and dative \citep[after][]{caha2009}. As Malagasy does not have any overt morphological system, it does not hold that the subject corresponds to the nominative in this case.
German is another example of a language that allows subjects to be relativized using a particular strategy, but not direct and indirect object. This strategy is the participle construction \citep{keenan1977}. This strategy is a secondary strategy that exist besides the main strategy that can be used to relativize direct and indirect objects. \ref{ex:german-rel-decl} is an example of a declarative sentence in German. It is a transitive sentence that contains the subject \tit{die Frau} `the woman' and the object \tit{der Mann} `the man'.

\exg. Die Frau küsst den Mann.\\
the woman kisses the man\\
`The woman is kissing the man.' \flushfill{German}\label{ex:german-rel-decl}

The subject from the declarative in \ref{ex:german-rel-decl}, sentence \tit{die Frau} `the woman', is relativized in \ref{ex:german-rel-subj}. The predicate from the declarative clause \tit{küsst} `kisses' is turned in into the participle \tit{küssende} `kissing'. The participle appears at the end of the reduced relative clause \tit{den Mann küssende} `the man kissing'. The reduced relative clause directly precedes the noun of the subject, creating distance between the determiner \tit{die} `the' and \tit{Frau} `woman', which are both marked in bold.

\exg. \tbf{die} den Mann küssende \tbf{Frau}\\
 the the man kissing woman\\
 `the woman who is kissing the man' \flushfill{German}\label{ex:german-rel-subj}

The object from the declarative sentence in \ref{ex:german-rel-decl}, \tit{den Mann} `the man', cannot be relativized like the subject, as shown in \ref{ex:german-rel-no-do}. Again, the predicate from the declarative clause \tit{küsst} `kisses' is turned in into the participle \tit{küssende} `kissing'. The participle appears at the end of the relative clause \tit{die Frau küssende} `the woman kissing'. The reduced relative clause directly precedes the noun of the object, creating distance between the determiner \tit{der} `the' and \tit{Mann} `man', which are both marked in bold. This example is ungrammatical.

\exg. *\tbf{den} die Frau küssende \tbf{Mann}\\
 the the woman kissing man\\
 intended: `the man that the woman is kissing' \flushfill{German}\label{ex:german-rel-no-do}

Malay is an example of a language that has a relativization strategy for subjects and direct objects, but not for indirect objects. \ref{ex:malay-do} shows an example in which the object is relativized. The object here is \tit{ayam} `chicken', marked in bold. It is followed by the relativizer \tit{yang} `that'. After that, the rest of the relative clause \tit{Aminah sedang memakan} `Aminah is eating' follows. The same strategy works to relativize subjects, which is not illustrated with an example.

\exg. Ali bunoh \tbf{ayam} yang Aminah sedang memakan.\\
 Ali kill chicken that Aminah \ac{prog} eat\\
 `Ali killed the chicken that Aminah is eating.' \flushfill{Malay, \pgcitealt{keenan1977}{71}, my boldfacing}\label{ex:malay-do}

Indirect objects cannot be relativized using the same strategy. \ref{ex:malay-decl} is an example of a ditransitive sentence in Malay. The indirect object \tit{kapada perempuan itu} `to the woman' cannot be relativized using \tit{yang}.

\exg. Ali beri {ubi kentang} itu kapada perempuan itu.\\
 Ali give potato the to woman the\\
 `Ali gave the potato to the woman.'\label{ex:malay-decl} \flushfill{Malay, \pgcitealt{keenan1977}{71}}

This is illustrated by the examples in \ref{ex:malay-no-io}. In \ref{ex:malay-no-io1}, the direct object \tit{perempuan kapada} `to the woman', marked in bold, appears in the first position of the clause. It is followed by the relativizer \tit{yang} `that' and the rest of the relative clause \tit{Ali beri ubi kentang itu kapada} `Ali gave the potato to'. This example in ungrammatical.
The example in \ref{ex:malay-no-io2} differs from \ref{ex:malay-no-io1} in that the preposition \tit{kapada} `to' has been moved such that it precedes the relativizer \tit{yang} `that'. This example is ungrammatical as well, indicating this was not the reason for the ungrammaticality.

\ex.\label{ex:malay-no-io}
\ag. *\tbf{perempuan} yang Ali beri {ubi kentang} itu kapada\\
 woman that Ali give potato the to\\\label{ex:malay-no-io1}
\bg. *\tbf{perempuan} \tbf{kapada} yang Ali beri {ubi kentang} itu\\
 woman to who Ali give potato that\\\flushfill{Malay, \pgcitealt{keenan1977}{71}, my boldfacing}\label{ex:malay-no-io2}

Basque is an example of a language that has a particular relativization strategy for subjects, direct objects and indirect objects. \ref{ex:basque-decl} is an example of a declarative ditransitive sentence in Basque. The sentence contains the subject \tit{gizonak} `the man', the direct object \tit{liburua} `the book' and the indirect object \tit{emakumeari} `the woman'.

\exg. Gizon-a-k emakume-a-ri liburu-a eman dio.\\
 man-\ac{def}-\ac{erg} woman-\ac{def}-\ac{dat} book-\ac{def}.\ac{abs} give has\\
 `The man has given the book to the woman.' \flushfill{Basque, \pgcitealt{keenan1977}{72}}\label{ex:basque-decl}

A relative clause in Basque appears in the prenominal position and it is marked by the invariable marker \tit{-n}.\footnote{
Additionally, the relativized positions do not appear in verbal agreement anymore, but this not visible in the example, because they are all phonologically zero.
}
\ref{ex:basque-sub} shows the three relativizations that are derived from \ref{ex:basque-decl}.
In \ref{ex:basque-sub}, the ergative subject \tit{gizonak} `the man' from \ref{ex:basque-decl} is relativized. The head \tit{gizona} `the man', marked in bold, has lost its ergative marker \tit{-k}, and follows the relative clause \tit{makumeari liburua eman dio} `who has given the book to the woman'. The suffix \tit{-n} is attached to the relative clause.
In \ref{ex:basque-do}, the absolutive direct object \tit{liburua} `the book' from \ref{ex:basque-decl} is relativized. The head \tit{liburua} `the book', marked in bold, follows the relative clause \tit{gizonak emakumeari eman dion} `that the man has given to the woman'.\footnote{
The absolutive direct object \tit{liburua} `the book' does not have an additional overt absolutive marker, so this difference cannot be observed when it is relativized.
}
The suffix \tit{-n} is attached to the relative clause.
In \ref{ex:basque-io}, the dative indirect object \tit{emakumeari} `the woman' from \ref{ex:basque-decl} is relativized. The head \tit{emakumea} `the man', marked in bold, has lost its dative marker \tit{-ri}, and follows the relative clause \tit{gizonak liburua eman dion} `that the man has given the book to'. The suffix \tit{-n} is attached to the relative clause.

\ex.\label{ex:basque-rel}
\ag. emakume-a-ri liburu-a eman dio-n \tbf{gizon-a}\\
 woman-\ac{def}-\ac{dat} book-\ac{def}.\ac{abs} give has-\ac{rel} man-\ac{def}\\
 `the man who has given the book to the woman'\label{ex:basque-sub}
\bg. gizon-a-k emakume-a-ri eman dio-n \tbf{liburu-a}\\
 man-\ac{def}-\ac{erg} woman-\ac{def}-\ac{dat} give has-\ac{rel} book-\ac{def}\\
 `the book that the man has given to the woman'\label{ex:basque-do}
\bg. gizon-a-k liburu-a eman dio-n \tbf{emakume-a}\\
 man-\ac{def}-\ac{erg} book-\ac{def}.\ac{abs} give has-\ac{rel} woman-\ac{def}\\
 `the woman that the man has given the book to' \flushfill{Basque, \pgcitealt{keenan1977}{72}, my boldfacing}\label{ex:basque-io}

\citet{caha2009} restates the implicational hierarchy in terms of case. Subject corresponds to nominative, direct object corresponds to accusative, and indirect object corresponds to dative. Again, the case scale \ac{nom} < \ac{acc} < \ac{dat} can be observed.


\section{In morphology}\label{sec:case-morphology}

In the two previous sections I showed that the case scale \ac{nom} < \ac{acc} < \ac{dat} can be observed in three syntactic phenomena. First, it shows up in case competition in headless relatives. Second, the case scale forms the basis for the implicational hierarchy observed in agreement across languages. Third, the identical implicational holds for relativization strategies cross-linguistically.

In this section, I show that this same case scale also shows up in morphology. First, syncretism only targets continuous regions on the case scale. Second, several languages show formal containment that mirrors the case scale.


\subsection{Syncretism}

Syncretism refers to the phenomenon whereby two or more different functions are fulfilled by a single form \citep[cf.][]{baerman2002}. In this section I discuss literature that shows that syncretism patterns among nominative, accusative and dative are not random. Instead, they pattern along the case scale \ac{nom} < \ac{acc} < \ac{dat}.

It has widely been observed that syncretism is restricted by the linear sequence \ac{nom} --- \ac{acc} --- \ac{dat} \citep{baerman2005,caha2009,zompi2017} (and see \citealt{mcfadden2018,smith2019} for similar claims concerning root suppletion). That is, if one orders cases in this linear sequence, only contiguous regions in the sequence turn out to be syncretic.
Following that, four possible patterns are attested crosslinguistically. First, all three cases are syncretic. Second, nominative and accusative are syncretic and the dative is not. Third, the accusative and the dative are syncretic and the nominative is not. Fourth, all cases are non-syncretic.

There is one pattern that is not attested crosslinguistically. This pattern does not target continuous regions, but non-contiguous ones: nominative and dative are syncretic and accusative is not. In other words, there is no ABA pattern (in which a form B intervenes between the two identically formed As) \citep{bobaljik2012}.

Table \ref{tbl:syncretisms} shows examples for each of these possible patterns. I give an example from a syncretism between nominative, accusative and dative from Dutch. The second person plural pronoun is \tit{jullie} `you.\ac{pl}' is syncretic between nominative, accusative and dative.
I give an example from a syncretism between nominative and accusative but not dative from German. The third person singular feminine \tit{sie} `she/her' is syncretic between nominative and accusative. The dative has a separate form: \tit{ihr} `her'.
I give an example from a syncretism between accusative and dative but not nominative from Icelandic. The first person singular plural is \tit{okkur} `us' is syncretic between accusative and dative. The nominative has a separate form: \tit{við} `we' \pgcitep{einarsson1949}{68}.
I give an example from three distinct forms from Faroese. The second person singular is \tit{tú} `you' for nominative, \tit{teg} `you' for accusative and \tit{tær} `you' for dative \pgcitep{lockwood1977}{70}.
Crucially, to the best of my knowledge, there is no language in which the nominative and the dative are syncretic but the accusative is not.

\begin{table}[ht]
  \center
  \caption {Syncretism in \ac{nom} --- \ac{acc} --- \ac{dat}}
    \begin{tabular}{cccccccc}
      \toprule
          \multicolumn{3}{c}{pattern}
            & \ac{nom}
            & \ac{acc}
            & \ac{dat}
            & translation
            & language \\
      \cmidrule(lr){1-3} \cmidrule(lr){4-6} \cmidrule(lr){7-7} \cmidrule(lr){8-8}
          A & A & A
            & \cellcolor{LG}jullie
            & \cellcolor{LG}jullie
            & \cellcolor{LG}jullie
            & 2\ac{pl}
            & Dutch \\
          A & A & B
            & \cellcolor{LG}sie
            & \cellcolor{LG}sie
            & ihr
            & 3\ac{sg}.\ac{f}
            & German \\
          A & B & B
            & við
            & \cellcolor{LG}okkur
            & \cellcolor{LG}okkur
            & 1\ac{pl}
            & Icelandic \\
          A & B & C
            & tú
            & teg
            & tær
            & 2\ac{sg}
            & Faroese \\
          A & B & A
            & \cellcolor{LG}
            &
            & \cellcolor{LG}
            &
            & not attested \\
      \bottomrule
    \end{tabular}\label{tbl:syncretisms}
\end{table}

In sum, case syncretism follows the ordering of the case scale in headless relatives: \ac{nom} < \ac{acc} < \ac{dat}.


\subsection{Formal containment}

This section shows a second way in which \ac{nom} < \ac{acc} < \ac{dat} is reflected in morphology: formal containment \citep[cf.][]{smith2019,zompi2017,caha2010}. In some languages, the form that is used for the accusative literally contains the form that is used for the nominative. In turn, the forms for the dative contains the form for the accusative. I illustrate this phenomenon with examples from Khanty.

Khanty (or Ostyak) shows formal containment in some of its pronouns (\pgcitealt{nikolaeva1999}{16} after \citealt{smith2019}). Three examples are given in Table \ref{tbl:cont-khanty}.

The nominative form for the first person singular is \tit{ma} `I.\ac{nom}'. The form for the accusative is \tit{ma:ne:m} `me'. This is the form for the nominative \tit{ma} plus the accusative marker \tit{-ne:m}. The form for the dative is \tit{ma:ne:mna} `me'. This is the form for the accusative \tit{ma:ne:m} plus the dative marker \tit{-na}. So, dative formally contains the accusative, and the accusative formally contains the nominative.

The third person singular and first person plural show the same pattern. The accusative forms \tit{luwe:l} `him/her' and \tit{muŋe:w} `us' contain the nominative forms \tit{luw} and the \tit{muŋ} plus the accusative marker \tit{-e:l} or \tit{-e:w}. The dative forms \tit{luwe:lna} `him/her' and \tit{muŋe:wna} `us' contain the accusative forms \tit{luwe:l} and \tit{muŋe:w} plus the dative marker \tit{-na}. Again, the dative formally contains the accusative, which in turn contains the nominative.

\begin{table}[ht]
  \center
	\caption {Case containment in Khanty}
		\begin{tabular}{clll}
		\toprule
              & \ac{1}\ac{sg}
              & \ac{3}\ac{sg}
              & \ac{1}\ac{pl}                           \\
		          \cmidrule{2-4}
    \ac{nom}  & ma
              & luw
              & muŋ                                     \\
    \ac{acc}  & ma:\tbf{-ne:m}
              & luw\tbf{-e:l}
              & muŋ\tbf{-e:w}                           \\
    \ac{dat}  & ma:\tbf{-ne:m}\tcol{DG}{\tbf{-na}}
              & luw\tbf{-e:l}\tcol{DG}{\tbf{-na}}
              & muŋ\tbf{-e:w}\tcol{DG}{\tbf{-na}}       \\
		\bottomrule
  \end{tabular}\label{tbl:cont-khanty}
\end{table}

Other languages that show this phenomenon are West Tocharian \citep{gippert1987} and Vlakh and Kalderaš Romani (respectively \citealt{friedman1991} and \citealt{boretzky1994}).

In sum, some languages morphologically look like \ac{nom}-\ac{acc}-\ac{dat}. This exactly reflects the case scale \ac{nom} < \ac{acc} < \ac{dat}.

\section{Summary}

Case competition in headless relatives adheres to the case scale in \ref{ex:case-scale-sum}. If the internal and external case differ, cases more on the right of the scale win over cases more to the left on the case.

\ex. \ac{nom} < \ac{acc} < \ac{dat}\label{ex:case-scale-sum}

This case scale is not only found in case competition in headless relatives. Implicational hierarchies regarding two syntactic phenomena appear across languages. The first one concerns agreement. If a language shows agreement with datives, it also shows agreement with accusatives and nominatives. If a language shows agreement with accusatives, it also shows agreement with nominatives.
The second implicational hierarchy concerns relativization. If a dative in a language can be relativized with a particular strategy, an accusative and a nominative can be too using the same strategy. If an accusative can be relativized with a particular strategy, so can a nominative with this strategy.

The case scale also shows up in morphological patterns. First, if the cases are ordered according to the case scale, syncretism only target continuous forms, no ABA pattern appears. Second, some languages show how the dative formally contains accusative, and how the accusative formally contains the nominative.

These phenomena show that the pattern observed in headless relatives is not something that stands on itself. The scale is a pattern that recurs across languages and across different phenomena. Therefore, it should not be treated as an special process with its own stipulated rule. Instead, it is something general that should also follow from general processes in languages.

The next chapter shows how features of the nominative, accusative and dative are organized. All facts presented in this chapter can be derived from the organization of these features.
