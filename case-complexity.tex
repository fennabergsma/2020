% !TEX root = thesis.tex

\chapter{Case complexity}

\section{The pattern}

\ex. \tsc{int:nom}, \tsc{ext:acc}
\a. \tsc{nom} not attested
\bg. jah þ-o-ei ist us Laudeikaion jus ussiggwaid\\
 and D-\tsc{f.sg.acc}-\tsc{comp} is\scsub{[nom]} from Laodicea you read\scsub{[acc]}\\
 `and read that which is from Laodicea' \hfill (Colossians 4:16, gloss and translation by \citealt[357]{harbert1978})

\ex. \tsc{int:nom}, \tsc{ext:dat}
\a. \tsc{nom} not attested
\bg. þ-aim-ei iupa sind fraþjaiþ\\
 D-\tsc{pl.dat}-\tsc{comp} above are\scsub{[nom]} {think on}\scsub{[dat]}\\
 `set your mind on those which are above' \hfill (Colossians 3:2, gloss and translation by \citealt[339]{harbert1978})

\ex. \tsc{int:nom}, \tsc{ext:gen}
\a. \tsc{nom} not attested
\b. \tsc{gen} not attested

\ex. \tsc{int:acc}, \tsc{ext:nom}
\ag. þ-an-ei frijos siuks ist\\
 D-\tsc{m.sg.acc}-\tsc{comp} love\scsub{[acc]} sick is\scsub{[nom]}\\
 `the one whom you love is sick' \hfill (John 11:3, gloss and translation by \citealt[342]{harbert1978})
\b. \tsc{nom} not attested

\ex. \tsc{int:acc}, \tsc{ext:dat}
\a. \tsc{acc} not attested
\bg. hva nu wileiþ ei taujau þ-amm-ei qiþiþ þiudan Iudaie?\\
 what now want that do\scsub{[dat]} D-\tsc{m.sg.dat}-\tsc{comp} say\scsub{[acc]} king {of Jews}\\
 `what now do you wish that I do to him whom you call King of the Jews?' \hfill (Mark 15:12, gloss and translation by \citealt[339]{harbert1978})

\ex. \tsc{int:acc}, \tsc{ext:gen}, partitive?
\a. \tsc{acc} not attested
\bg. ni waiht þ-iz-ei gasehvun\\
 not thing\scsub{[gen]} D-\tsc{n.sg.gen}-\tsc{comp} saw\scsub{[acc]}\\
 `not any of that which they saw' \hfill (Luke 9:36, gloss and translation by \citealt[340]{harbert1978})

\ex. \tsc{int:dat}, \tsc{ext:nom}
\ag. iþ þ-amm-ei leitil fraletada leitil frijod\\
 but D-\tsc{m.sg.dat}-\tsc{comp} little {is forgiven\scsub{[dat]}} little loves\scsub{[nom]}\\
 `but the one whom little is forgiven loves little' \hfill (Luke 7:47, gloss and translation by \citealt[342]{harbert1978})
\b. \tsc{nom} not attested

\ex. \tsc{int:dat}, \tsc{ext:acc}, is with a preposition
\ag. ushafjands ana þ-amm-ei lag\\
 {picking up}\scsub{[acc]}\scsub{[dat]} on D-\tsc{m.sg.dat}-\tsc{comp} lay\\
 `picking up that on which he lay' \hfill (Luke 5:25, gloss and translation by \citealt[343]{harbert1978})
\b. \tsc{acc} not attested

\ex. \tsc{int:dat}, \tsc{ext:gen}
\a. \tsc{dat} not attested
\b. \tsc{gen} not attested

\ex.  \tsc{int:gen}, \tsc{ext:nom}
\a. \tsc{gen} not attested
\b. \tsc{nom} not attested

\ex. \tsc{int:gen}, \tsc{ext:acc}, partitive?
\ag. bugei þ-iz-ei þaurbeima\\
 buy\scsub{[acc]} D-\tsc{n.sg.gen}-\tsc{comp} need\scsub{[gen]}\\
 `buy that which we need' \hfill (John 13:29, gloss and translation by \citealt[343]{harbert1978})
\b. \tsc{acc} not attested

\ex. \tsc{int:gen}, \tsc{ext:dat}
\a. \tsc{gen} not attested
\b. \tsc{dat} not attested

\begin{table}[h]
  \center
  \caption {Case attraction in headless relatives in Gothic}
    \begin{minipage}{\linewidth}
      \begin{tabularx}{\textwidth}{c|Y|Y|Y|Y}
        \toprule
          \diagbox[linecolor=white]{\tsc{int}}{\tsc{ext}}
              & \tsc{[nom]}
              & \tsc{[acc]}
              & \tsc{[dat]}
              & \tsc{[gen]}
              \\ \cmidrule(lr){1-5}
          \tsc{[nom]}
              & \colorbox{LG}{\tsc{nom}}
              & \diagbox[linecolor=white]{?\tsc{nom}}{\colorbox{DG}{\tsc{acc}}}
              & \diagbox[linecolor=white]{?\tsc{nom}}{\colorbox{DG}{\tsc{dat}}}
              & \diagbox[linecolor=white]{?\tsc{nom}}{?\tsc{gen}}
              \\ \cmidrule(lr){1-5}
          \tsc{[acc]}
              & \diagbox[linecolor=white]{\colorbox{DG}{\tsc{acc}}}{?\tsc{nom}}
              &	\colorbox{LG}{\tsc{acc}}
              &	\diagbox[linecolor=white]{?\tsc{acc}}{\colorbox{DG}{\tsc{dat}}}
              &	\diagbox[linecolor=white]{?\tsc{acc}}{\colorbox{DG}{\tsc{gen}}}
              \\ \cmidrule(lr){1-5}
          \tsc{[dat]}
              & \diagbox[linecolor=white]{\colorbox{DG}{\tsc{dat}}}{?\tsc{nom}}
              &	\diagbox[linecolor=white]{\colorbox{DG}{\tsc{dat}}}{?\tsc{acc}}
              & \colorbox{LG}{\tsc{dat}}
              & \diagbox[linecolor=white]{?\tsc{dat}}{?\tsc{gen}}
              \\ \cmidrule(lr){1-5}
          \tsc{[gen]}
              & \diagbox[linecolor=white]{?\tsc{gen}}{?\tsc{nom}}
              & \diagbox[linecolor=white]{\colorbox{DG}{\tsc{gen}}}{?\tsc{acc}}
              & \diagbox[linecolor=white]{?\tsc{gen}}{?\tsc{dat}}
              & \colorbox{LG}{\tsc{gen}}
              \\
        \bottomrule
      \end{tabularx}
    \end{minipage}
\end{table}

\ex. \tsc{nom} < \tsc{acc} < \tsc{dat}

reasons to leave out genitive

\begin{itemize}
  \item Gothic: it is not verbs that select for genitive, but
\end{itemize}


\section{Another instance of case competition}

numerals


\section{Case containment}

\subsection{Morphological containment}

\begin{table}[h]
  \center
	\caption {Transparent case containment in Khanty \citep[16]{nikolaeva1999}
	\begin{minipage}{0.4\linewidth}
		\begin{tabularx}{\textwidth}{c *{3}{Y}}
		\toprule
        & \tsc{nom} & \tsc{acc}       & \tsc{dat}         \\
		\midrule
    1sg & ma        & ma\tbf{:-ne:m}  & ma\tbf{:-ne:m}-na \\
    3sg & luw       & luw\tbf{-e:l}   & luw\tbf{-e:l}-na  \\
    1pl & muŋ       & muŋ\tbf{-e:w}   & muŋ\tbf{-e:w}-na  \\
		\bottomrule
		\end{tabularx}
	\end{minipage}
\end{table}

\begin{table}[h]
  \center
	\caption {Transparent case containment in Kalderaš Romani \citep[31-46]{boretzky1994}}
	\begin{minipage}{0.6\linewidth}
		\begin{tabularx}{\textwidth}{c *{3}{Y}}
		\toprule
    \tsc{nom} & \tsc{acc}       & \tsc{dat}           &             \\
		\midrule
    phral     & phral-\tbf{és}  & phral-\tbf{és}-kə   & `brother'   \\
    phral-(á) & phral-\tbf{én}  & phral-\tbf{én}-gə   & `brothers'  \\
    rakl-í    & rakl-\tbf{já}   & rakl-\tbf{já}-kə    & `girl'      \\
    rakl-já   & rakl-já-\tbf{n} & rakl-\tbf{já}-n-gə  & `girls'     \\
		\bottomrule
		\end{tabularx}
	\end{minipage}
\end{table}


\begin{table}[h]
  \center
	\caption {Transparent case containment in West Tocharian \citep[23-24]{gippert1987}}
	\begin{minipage}{0.4\linewidth}
		\begin{tabularx}{\textwidth}{c *{3}{Y}}
		\toprule
    \tsc{nom} & \tsc{acc}      & \tsc{dat}          &          \\
		\midrule
    yakwi     & yakwe-\tbf{ṃ}  & yäkwe-\tbf{ṃ}-ts   & `horses' \\
    eṅkwi     & eṅkwe-\tbf{ṃ}  & eṅkwe-\tbf{ṃ}-ts   & `men'    \\
		\bottomrule
		\end{tabularx}
	\end{minipage}
\end{table}




\subsection{Suppletion}

\subsubsection{ABB}


cognates widespread in Indo-European
Icelandic ég mig mér

cognates across Slavic
Russian 1pl my nas nam

cognates across Slavic
Serbian 3sg(m) on nje-ga nje-mu
Serbian 3sg(f) ona nju njoj
Serbian 3pl(m) oni njih njima


\subsubsection{ABC}

Indo-European:
Albanian 3sg.f ajo (a)të asaj 95
Nakh-Dagestanian:
Khinalugh 1sg zɨ jä as(ɨr)


\subsubsection{AAB}

Algic:
Yurok 3sg yoɂ, woɂ, yoɂo·t, woɂo·t yoɂo·t, woɂo·t weyaɂik

Australian:
Wardaman 3sg narnaj narnaj-(j)i gunga
Wardaman 3pl narnaj-bulu narnaj-bulu-yi wurrugu

\subsection{Syncretism}




\section{Nanosyntax}

\subsection{Basics}

\subsection{Spellout}

\ex. The Superset Principle \citet{starke2009}: \\
A lexically stored tree matches a syntactic node iff the lexically stored tree contains the syntactic node.

\ex. The Elsewhere Condition (\citealt{kiparsky1973}, formulated as in \citealt{caha2020}):\\
When two entries can spell out a given node, the more specific entry wins. Under the Superset Principle governed insertion, the more specific entry is the one which has fewer unused features.

\ex. Merge F and \label{ex:spellout}
 \a. Spell out FP.
 \b. If (a) fails, attempt movement of the spec of the complement of \tsc{f}, and retry (a).
 \b. If (b) fails, move the complement of \tsc{f}, and retry (a).

When a new match is found, it overrides previous spellouts.

\ex. Cyclic Override \citep{starke2018}:\\
Lexicalisation at a node XP overrides any previous match at a phrase contained in XP.

If the spellout procedure in \ref{ex:spellout} fails, backtracking takes place.

\ex. Backtracking \citep{starke2018}:\\
When spellout fails, go back to the previous cycle, and try the next option for that cycle.\label{ex:backtracking}

If backtracking also does not help, a specifier is constructed.

\ex. Spec Formation \citep{starke2018}:\\
If Merge F has failed to spell out (even after backtracking), try to spawn a new derivation providing the feature F and merge that with the current derivation, projecting the feature F at the top node.\label{ex:specformation}



\section{Analysis}

No syntax of relative clauses yet, just "when one contains the other, the contained one can be deleted"




\section{Bigger picture}

Case is complex
