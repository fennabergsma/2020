% !TEX root = thesis.tex

\chapter{Case complexity}

\section{The pattern}


\ex. \tsc{int:nom}, \tsc{ext:acc}
\a. \tsc{nom} not attested
\bg. jah þ-o-ei ist us Laudeikaion jus ussiggwaid\\
 and D-\tsc{f.sg.acc}-\tsc{comp} is\scsub{[nom]} from Laodicea you read\scsub{[acc]}\\
 `and read that which is from Laodicea' \hfill (Colossians 4:16, gloss and translation by \citealt[357]{harbert1978})

\ex. \tsc{int:nom}, \tsc{ext:dat}
\a. \tsc{nom} not attested
\bg. þ-aim-ei iupa sind fraþjaiþ\\
 D-\tsc{pl.dat}-\tsc{comp} above are\scsub{[nom]} {think on}\scsub{[dat]}\\
 `set your mind on those which are above' \hfill (Colossians 3:2, gloss and translation by \citealt[339]{harbert1978})

\ex. \tsc{int:acc}, \tsc{ext:nom}
\ag. þ-an-ei frijos siuks ist\\
 D-\tsc{m.sg.acc}-\tsc{comp} love\scsub{[acc]} sick is\scsub{[nom]}\\
 `the one whom you love is sick' \hfill (John 11:3, gloss and translation by \citealt[342]{harbert1978})
\b. \tsc{nom} not attested

\ex. \tsc{int:acc}, \tsc{ext:dat}
\a. \tsc{acc} not attested
\bg. hva nu wileiþ ei taujau þ-amm-ei qiþiþ þiudan Iudaie?\\
 what now want that do\scsub{[dat]} D-\tsc{m.sg.dat}-\tsc{comp} say\scsub{[acc]} king {of Jews}\\
 `what now do you wish that I do to him whom you call King of the Jews?' \hfill (Mark 15:12, gloss and translation by \citealt[339]{harbert1978})

\ex. \tsc{int:dat}, \tsc{ext:nom}
\ag. iþ þ-amm-ei leitil fraletada leitil frijod\\
 but D-\tsc{m.sg.dat}-\tsc{comp} little {is forgiven\scsub{[dat]}} little loves\scsub{[nom]}\\
 `but the one whom little is forgiven loves little' \hfill (Luke 7:47, gloss and translation by \citealt[342]{harbert1978})
\b. \tsc{nom} not attested

\ex. \tsc{int:dat}, \tsc{ext:acc}, is with a preposition
\ag. ushafjands ana þ-amm-ei lag\\
 {picking up}\scsub{[acc]}\scsub{[dat]} on D-\tsc{m.sg.dat}-\tsc{comp} lay\\
 `picking up that on which he lay' \hfill (Luke 5:25, gloss and translation by \citealt[343]{harbert1978})
\b. \tsc{acc} not attested

\begin{table}[h]
  \center
  \caption {Case attraction in headless relatives in Gothic}
    \begin{minipage}{0.8\linewidth}
      \begin{tabularx}{\textwidth}{c|Y|Y|Y}
        \toprule
          \diagbox[linecolor=white]{\tsc{int}}{\tsc{ext}}
              & \tsc{[nom]}
              & \tsc{[acc]}
              & \tsc{[dat]}
              \\ \cmidrule(lr){1-4}
          \tsc{[nom]}
              & \colorbox{LG}{\tsc{nom}}
              & \diagbox[linecolor=white]{?\tsc{nom}}{\colorbox{DG}{\tsc{acc}}}
              & \diagbox[linecolor=white]{?\tsc{nom}}{\colorbox{DG}{\tsc{dat}}}
              \\ \cmidrule(lr){1-4}
          \tsc{[acc]}
              & \diagbox[linecolor=white]{\colorbox{DG}{\tsc{acc}}}{?\tsc{nom}}
              &	\colorbox{LG}{\tsc{acc}}
              &	\diagbox[linecolor=white]{?\tsc{acc}}{\colorbox{DG}{\tsc{dat}}}
              \\ \cmidrule(lr){1-4}
          \tsc{[dat]}
              & \diagbox[linecolor=white]{\colorbox{DG}{\tsc{dat}}}{?\tsc{nom}}
              &	\diagbox[linecolor=white]{\colorbox{DG}{\tsc{dat}}}{?\tsc{acc}}
              & \colorbox{LG}{\tsc{dat}}
              \\
        \bottomrule
      \end{tabularx}
    \end{minipage}
\end{table}

\ex. \tsc{nom} < \tsc{acc} < \tsc{dat}

\phantom{x}



\section{Accessibility hierarchy}

\ex. \tsc{nom} < \tsc{acc} < \tsc{dat}

\phantom{x}




\section{Morphology}

\subsection{Morphological containment}

\begin{table}[h]
  \center
	\caption {Transparent case containment in Khanty \citep[16]{nikolaeva1999}}
	\begin{minipage}{0.7\linewidth}
		\begin{tabularx}{\textwidth}{c *{3}{Z}}
		\toprule
              & \tsc{1sg}
              & \tsc{3sg}
              & \tsc{1pl}                               \\
		\midrule
    \tsc{nom} & ma
              & luw
              & muŋ                                     \\
    \tsc{acc} & ma\tbf{:-ne:m}
              & luw\tbf{-e:l}
              & muŋ\tbf{-e:w}                           \\
    \tsc{dat} & ma\tbf{:-ne:m}-\textcolor{DG}{\tbf{na}}
              & luw\tbf{-e:l}-\textcolor{DG}{\tbf{na}}
              & muŋ\tbf{-e:w}-\textcolor{DG}{\tbf{na}}  \\
		\bottomrule
		\end{tabularx}
	\end{minipage}
\end{table}



\begin{table}[h]
  \center
	\caption {Transparent case containment in Kalderaš Romani \citep[31-46]{boretzky1994}}
	\begin{minipage}{0.9\linewidth}
		\begin{tabularx}{\textwidth}{c *{4}{Z}}
		\toprule
              & `brother'
              & `brothers'
              & `girl'
              & `girls'                                   \\
		\midrule
    \tsc{nom} & phral
              & phral-(á)
              & rakl-í
              & rakl-já                                   \\
    \tsc{acc} & phral-\tbf{és}
              & phral-\tbf{én}
              & rakl-\tbf{já}
              & rakl-já-\tbf{n}                           \\
    \tsc{dat} & phral-\tbf{és}-\textcolor{DG}{\tbf{kə}}
              & phral-\tbf{én}-\textcolor{DG}{\tbf{gə}}
              & rakl-\tbf{já}-\textcolor{DG}{\tbf{kə}}
              & rakl-já-\tbf{n}-\textcolor{DG}{\tbf{gə}}  \\
		\bottomrule
		\end{tabularx}
	\end{minipage}
\end{table}


\begin{table}[h]
  \center
	\caption {Transparent case containment in West Tocharian \citep[23-24]{gippert1987}}
	\begin{minipage}{0.6\linewidth}
		\begin{tabularx}{\textwidth}{c *{4}{Z}}
		\toprule
              & `horses'
              & `men'                                  \\
		\midrule
    \tsc{nom} & yakwi
              & eṅkwi                                  \\
    \tsc{acc} & yakwe-\tbf{ṃ}
              & eṅkwe-\tbf{ṃ}                          \\
    \tsc{dat} & yäkwe-\tbf{ṃ}-\textcolor{DG}{\tbf{ts}}
              & eṅkwe-\tbf{ṃ}-\textcolor{DG}{\tbf{ts}} \\
		\bottomrule
		\end{tabularx}
	\end{minipage}
\end{table}

\ex. \tsc{nom} < \tsc{acc} < \tsc{dat}

\phantom{x}

\subsection{Suppletion}

\ex. \tsc{nom} < \tsc{acc} < \tsc{dat}

\phantom{x}

\subsubsection{ABB}


cognates widespread in Indo-European - Icelandic\\
cognates across Slavic - Russian\\
cognates across Slavic - Serbian

% \begin{table}[h]
%   \center
% 	\caption {ABB patterns in suppletion}
% 	\begin{minipage}{0.8\linewidth}
% 		\begin{tabularx}{\textwidth}{c *{5}{Y}}
% 		\toprule
%               & Icelandic & Russian   & \multicolumn{3}{l}{Serbian}               \\
% 		\midrule
%               & \tsc{1sg} & \tsc{1pl} & \tsc{3sg.f} & \tsc{3sg.m}  & \tsc{3sg.n}  \\
%     \midrule
%     \tsc{nom} & ég        & my        &  ona        & oni          & on           \\
%     \tsc{acc} & \tbf{m}ig & \tbf{n}as & \tbf{nj}u   & \tbf{nji}h   & \tbf{nje}-ga \\
%     \tsc{dat} & \tbf{m}ér & \tbf{n}am & \tbf{nj}oj  & \tbf{nji}ma  & \tbf{nje}-mu \\
%     \bottomrule
% 		\end{tabularx}
% 	\end{minipage}
% \end{table}



\subsubsection{ABC}

\begin{table}[h]
  \center
	\caption {ABB patterns in suppletion}
	\begin{minipage}{0.4\linewidth}
		\begin{tabularx}{\textwidth}{c *{2}{Y}}
		\toprule
              & Albanian    & Khinalugh   \\
		\midrule
              & \tsc{3sg.f} & \tsc{1sg}   \\
    \midrule
    \tsc{nom} & ajo          & zɨ         \\
    \tsc{acc} & (a)të        & jä         \\
    \tsc{dat} & asaj         & as(ɨr)     \\
    \bottomrule
		\end{tabularx}
	\end{minipage}
\end{table}



\subsubsection{AAB}


% \begin{table}[h]
%   \center
% 	\caption {ABB patterns in suppletion}
% 	\begin{minipage}{0.8\linewidth}
% 		\begin{tabularx}{\textwidth}{c *{3}{Y}}
% 		\toprule
%               & Yurok               & \multicolumn{2}{c}{Wardaman} \\
% 	  \midrule
%               & \tsc{3sg}           & \tsc{3sg}   & \tsc{3pl}      \\
%     \midrule
%     \tsc{nom} & yoɂ(o·t), woɂ(o·t)  & narnaj      & narnaj-bulu    \\
%     \tsc{acc} & yoɂo·t, woɂo·t      & narnaj-(j)i & narnaj-bulu-yi \\
%     \tsc{dat} & weyaɂik             & gunga       & wurrugu        \\
%     \bottomrule
% 		\end{tabularx}
% 	\end{minipage}
% \end{table}
%




\subsection{Syncretism}

\subsubsection{ABB}
\subsubsection{ABC}
\subsubsection{AAB}




\ex. \tsc{nom} < \tsc{acc} < \tsc{dat}



\section{Case complexity in syntax}

\ex.
\begin{forest} boom
  [\tsc{datP}
      [\tsc{dat}]
      [\tsc{accP}
          [\tsc{acc}]
          [\tsc{nomP}
              [\tsc{nom}]
              [..,roof]
          ]
      ]
  ]
\end{forest}

Explain how all phenomena relate to this.

\phantom{hi}
