% !TEX root = thesis.tex

\chapter{OHG}

\section{Isidor}

\subsection{question mark}

after : Huuer ist , der eo . . . gahorti

ist za . . . der ni galaubit , daz imo



\subsection{examples}

internal: dat, external: nom?

\exg. Ibu christus auur got ni uuari, \tbf{dhemu} \tbf{in} \tbf{psalmom} \tbf{chiquhedan} \tbf{uuard}\\
wenn,falls ChristusSG.NOM aber,jedoch GottSG.NOM nicht seinSUBJ.PAST.SG.3 derMASC.SG.DAT in Psalm,LobgesangPL.DAT sprechen,singen werdenIND.PAST.SG.3\\
`' \flushfill{\ac{ohg}, \ac{isid}}


internal: nom, external, nom

\exg. Dher euuuih hrinit, hrinit sines augin sehun .\\
 der.MASC.SG.NOM ihr.PL.ACC.2 berührenIND.PRES.SG.3 berührenIND.PRES.SG.3 seinNEUT.SG.GEN.ST AugeSG.GEN PupilleSG.ACC\\
`' \flushfill{\ac{ohg}, \ac{isid}}


internal: nom, external:nom, light-headed!

\exg. Innan dhiu dher quhimit, dher chisendit uuirdhit\\
bis dass der.MASC.SG.NOM kommen.IND.PRES.SG.3  der.MASC.SG.NOM senden,schicken werden.IND.PRES.SG.3\\
`' \flushfill{\ac{ohg}, \ac{isid}}


\exg. ``Ih bibringu fona iacobes samin endi fona iuda dhen mina berga chisitzit.''\\
ich.SG.NOM.1 hervorbringen.IND.PRES.SG.1 von Jakob.SG.GEN Samuel.SG.DAT und von Judas.SG.ABL \tsc{rel.}MASC.SG.ACC mein.MASC.PL.ACC.ST Berg.PL.ACC besitzen.IND.PRES.SG.3\\
`' \flushfill{\ac{ohg}, \ac{isid} 34:3}\label{ex:ohg-acc-nom}



\phantom{x}

\section{Otfrid}

\exg. Bis tú nu {zi wáre} furira Ábrahame ? ouh thén man hiar nu zálta\\
 sein du nun,jetzt wahrlich erhabener,wertvoller Abraham ? auch,und der,die,das man hier nun,jetzt erzählen,verkünden\\
 `xxx' \flushfill{\ac{ohg}, \ac{otfrid} III 18:33}\label{ex:ohg-dat-acc}


\phantom{x}
