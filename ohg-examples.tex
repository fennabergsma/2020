% !TEX root = thesis.tex

\chapter{OHG}

\section{Isidor}

\subsection{question mark}

after : Huuer ist , der eo . . . gahorti

ist za . . . der ni galaubit , daz imo



\subsection{examples}



internal: dat, external: nom?

\exg. Ibu christus auur got ni uuari, \tbf{dhemu} \tbf{in} \tbf{psalmom} \tbf{chiquhedan} \tbf{uuard}\\
wenn,falls ChristusSG.NOM aber,jedoch GottSG.NOM nicht seinSUBJ.PAST.SG.3 derMASC.SG.DAT in Psalm,LobgesangPL.DAT sprechen,singen werdenIND.PAST.SG.3\\
`' \flushfill{\ac{ohg}, \ac{isid}}


internal: nom, external, nom

\exg. Dher euuuih hrinit, hrinit sines augin sehun .\\
 der.MASC.SG.NOM ihr.PL.ACC.2 berührenIND.PRES.SG.3 berührenIND.PRES.SG.3 seinNEUT.SG.GEN.ST AugeSG.GEN PupilleSG.ACC\\
`' \flushfill{\ac{ohg}, \ac{isid}}


internal: nom, external:nom, light-headed!

\exg. Innan dhiu dher quhimit, dher chisendit uuirdhit\\
bis dass der.MASC.SG.NOM kommen.IND.PRES.SG.3  der.MASC.SG.NOM senden,schicken werden.IND.PRES.SG.3\\
`' \flushfill{\ac{ohg}, \ac{isid}}


\exg. ``Ih bibringu fona iacobes samin endi fona iuda dhen mina berga chisitzit.''\\
ich.SG.NOM.1 hervorbringen.IND.PRES.SG.1 von Jakob.SG.GEN Samuel.SG.DAT und von Judas.SG.ABL \tsc{rel.}MASC.SG.ACC mein.MASC.PL.ACC.ST Berg.PL.ACC besitzen.IND.PRES.SG.3\\
`' \flushfill{\ac{ohg}, \ac{isid} 34:3}\label{ex:ohg-acc-nom}



\phantom{x}

\section{Otfrid}

\exg. Bis tú nu {zi wáre} furira Ábrahame ? ouh thén man hiar nu zálta\\
 sein du nun,jetzt wahrlich erhabener,wertvoller Abraham ? auch,und der,die,das man hier nun,jetzt erzählen,verkünden\\
 `xxx' \flushfill{\ac{ohg}, \ac{otfrid} III 18:33}\label{ex:ohg-dat-acc}

COUNTEREXAMPLE: int: nom, ext: acc

\exg. tház	si	uns	béran	scolti	thér	unsih	gihéilti\\
dass	sieFEM.SG.NOM.3	wirPL.DAT.1	(hervor)bringen,gebären	sollen,werdenSUBJ.PAST.SG.3	derMASC.SG.NOM	wirPL.ACC.1	rettenSUBJ.PAST.SG.3\\
 `xxx' \flushfill{\ac{ohg}, \ac{otfrid}}

 \exg. gébe themo	ni	éigi\\
 gebenSUBJ.PRES.SG.3	derMASC.SG.DAT	nicht	haben,besitzenSUBJ.PRES.SG.3\\
 `' \flushfill{\ac{ohg}, \ac{otfrid}}

  internal: nom, external: dat?

  \exg. thia	láz	ih	themo	iz	lísit	thar\\
  dieFEM.SG.ACC	lassenIND.PRES.SG.1	ichSG.NOM.1	derMASC.SG.DAT	esNEUT.SG.ACC.3	lesenIND.PRES.SG.3	da, dort\\
  `I leave her to him who reads it' \flushfill{\ac{ohg}, \ac{otfrid}}


?

Al	io	súlicha	giwúrt	so	duat	thes	géistes	giburt	thén	zi	thiu	gigángent
then = dative plural
dative subject in embedded clause?

internal: dat? external: nom?

\exg. nist	themo	sér	bizeinit	noh	léides	wiht	giméinit\\
nicht.seinIND_PRES_SG_3	derMASC_SG_DAT	Schmerz,Leid,Kummer,Unglück;SG_NOM Böses	bezeichnen,bestimmenIND_PRES_SG_3 und.nicht Leid,Schmerz,UnglückSG_GEN	Ding,etwasSG_NOM	sagen,meinen,bestimmenIND_PRES_SG_3\\
  `' \flushfill{\ac{ohg}, \ac{otfrid}}



internal: acc, external: nom

\exg. thíz	ist	then	sie	zéllent	joh	then	sie	sláhan	wollent	!\\
diesesNEUT_SG_NOM	seinIND_PRES_SG_3	derMASC_SG_ACC	MASC_PL_NOM_3	erzählenIND_PRES_PL_3		und	derMASC_SG_ACC	MASC_PL_NOM_3	töten	wollenIND_PRES_PL_3\\

hetzelfde als deze
internal: acc, external: nom

\exg. Thiz ist [then sie zellent].\\
 this is whom(ACC) they talk\\
 `This is the one whom they talk about.' \flushfill{\ac{ohg}, ?}



\phantom{x}
