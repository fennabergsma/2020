% !TEX root = thesis.tex

\chapter{Technical implementation}


\section{Background}



\ex. \tbf{Spellout Algorithm:}\\
Merge F and \label{ex:spellout}
 \a. Spell out FP.
 \b. If (a) fails, attempt movement of the spec of the complement of \tsc{f}, and retry (a).
 \b. If (b) fails, move the complement of \tsc{f}, and retry (a).

When a new match is found, it overrides previous spellouts.

\ex. \tbf{Cyclic Override} \citep{starke2018}:\\
Lexicalisation at a node XP overrides any previous match at a phrase contained in XP.

If the spellout procedure in \ref{ex:spellout} fails, backtracking takes place.

\ex. \tbf{Backtracking} \citep{starke2018}:\\
When spellout fails, go back to the previous cycle, and try the next option for that cycle.\label{ex:backtracking}

If backtracking also does not help, a specifier is constructed.

\ex. \tbf{Spec Formation} \citep{starke2018}:\\
If Merge F has failed to spell out (even after backtracking), try to spawn a new derivation providing the feature F and merge that with the current derivation, projecting the feature F at the top node.\label{ex:specformation}

\phantom{hi}



\section{Derivations}
