% !TEX root = thesis.tex

\chapter{The derivations of the patterns}\label{ch:relativization}

In the previous chapter I showed that languages with case competition come in two variants. First, there are languages that allow the internal and external case to surface when they win the competition, such as Gothic and Old High German. Second, there are languages that only allow the internal case to surface when it wins the competition, such as Modern German. Crucially, there is no language that only allows the external case to surface when it wins the competition.

The aim of this chapter is twofold. First, I discuss how the non-existence of the external-only pattern can be explained. Second, I discuss how the difference between internal-and-external languages and internal-only languages can be derived. I introduce both matters a bit more in this introduction.

I start with the non-existence of the external-only pattern. blablabla

Let me now turn to the matter of crosslinguistic differences. Every speaker of a language needs to learn what the pattern for its language is. Headless relatives are infrequent, is what can be said about at least Modern German. Even though not everybody likes the construction to begin with (they prefer (light-)headed relatives), people seem to have the clear intuition that \tsc{int}>\tsc{ext} is much better than the other way around. It seems implausible that learners of German learn this pattern from the few examples they got (there are just too few to make a generalization). Still, the intuition exist. And it is very particular: more complex case wins over less complex case, but only if the internal case is more complex than the external case. This already sounds hard to learn from the input as a generalization. People have also been describing it like this: formulation from Cinque in his book. If it does not come from the input, where does it come from? I claim that it comes from other properties of the language. In Grosu's terminology: is it derived or basic? Ideally, we would want it to be derived.

A similar avenue was pursued by \citealt{himmelreich2017}. She specific languages for having different types of agree (up, down) and different types of probes (active, non-active). Doing that, she successfully derived free relatives and parasitic gaps in different languages. Grosu 1994 linked richness of inflection to liberality. He actually talked about the richness of pro.

The crucial difference with I'm doing is that I'm not relying on an arbitrary value I assigned to a language (say null head is active probe, probing only happens upwards). Like I briefly mentioned in Chapter \ref{ch:decomposition}, Nanosyntax models crosslinguistic variation as differences in the lexicon, how the features are packaged together differently. That means that I look for patterns within the languages themselves, and let the facts of the headless relatives follow from those. Specifically, I derive the different behaviors from relative pronouns and the external head that I introduce in this chapter.

In Section \ref{sec:internal-wins} I discuss the situation in which the internal case wins the competition and the relative pronoun surfaces in the internal case. This situation is attested in all languages with case competition: the internal-only ones, such as Modern German, and the internal-and-external ones, such as Gothic and Old High German. I start with showing that relative pronouns are always part of the relative clause. That is why the relative pronoun also takes the internal case, the case assigned in the relative clause. Then, I introduce an element that appears external to the relative clause that carries the external case. This is needed to allow for the comparison between the internal and external case, and to allow for languages like Polish, that do not allow for any case mismatches. The existence of this element is independently motivated by languages that overtly show it. I show that this head contains a subset of the features that the internal head contains. If the case features on the external element are then also a subset of the case features on the internal element, the internal element element can delete the external one. For instance, \tit{wem} deletes \tit{en}.

In Section \ref{sec:external-wins} I discuss the situation in which the external case wins the case competition and the relative pronoun surfaces in the external case. This situation is not attested in all of the case competition languages. This situation arises if the external head contains all case features of the internal head. In addition, and this is what distinguishes the internal-and-external languages from the internal-only languages, the features of the relative pronoun are a subset of the features of the external head. So, \tit{em} cannot delete \tit{wen}, but \tit{dem} can delete \tit{den}.

From these two proposals follows that it is impossible to have third pattern, which is indeed also not attested. When the external case wins over the internal one, there is a situation in which the external case could delete the internal one. So, it is impossible to have the second option but not the first one.


\section{The internal wins pattern}\label{sec:internal-wins}

This section discusses the situation in which the external case wins the case competition and the relative pronoun surfaces in the internal case.

Examples of this situation in the three languages under discussion are these:

\exg. Uns besucht \tbf{wen} \tbf{Maria} \tbf{mag}.\\
 we.\ac{acc} visit.3\ac{sg}\scsub{[nom]} \tsc{rel}.\ac{acc}.\tsc{an} Maria.\ac{nom} like.3\ac{sg}\scsub{[acc]}\\
 `Who visits us, Maria likes.' \flushfill{adapted from \pgcitealt{vogel2001}{343}}\label{ex:mg-int}
\bg. thíz ist \tbf{then} \tbf{sie} \tbf{zéllent}\\
 \ac{dem}.\ac{sg}.\ac{n}.\ac{nom} be.\ac{pres}.3\ac{sg}\scsub{[nom]} \ac{rel}.\ac{sg}.\ac{m}.\ac{acc} 3\ac{pl}.\ac{m}.\ac{nom} tell.\ac{pres}.3\ac{pl}\scsub{[acc]}\\
 `this is the one whom they talk about' \flushfill{Old High German, \ac{otfrid} III 16:50}\label{ex:ohg-int}
\bg. \tbf{þan} \tbf{-ei} \tbf{frijos} siuks ist\\
 \ac{rel}.\ac{sg}.\ac{m}.\ac{acc} -\ac{comp} love.\ac{pres}.2\ac{sg}.\scsub{[acc]} sick be.\ac{pres}.3\ac{sg}\scsub{[nom]}\\
 `the one whom you love is sick' \flushfill{Gothic, \ac{john} 11:3, adapted from \pgcitealt{harbert1978}{342}}\label{ex:gothic-int}

I start by looking at Modern German, so the example in \ref{ex:mg-int}. The section is set up like this:

At the end, I show that this analysis can also be applied to Old High German and Gothic.


\subsection{Relative pronouns in the relative clause}

In this section I show that the relative pronoun in Modern German headless relatives is part of the relative clause. The evidence comes from extraposition. In Modern German, it is possible to extrapose a CP (a clause), but not a DP (a noun phrase). In this section I first show that Modern German CPs can be extraposed and DPs cannot. Then I illustrate how relative clauses including the relative pronoun in headless relatives pattern with CPs: they can be extraposed as well. This indicates that the relative pronoun is part of the relative clause.

The sentences in \ref{ex:mg-extrapose-cp} show that it is possible to extrapose a CP. In \ref{ex:mg-extrapose-cp-base}, the clausal object \tit{wie es dir geht} `how you are doing', marked here in bold, appears in its base position. It can be extraposed to the right edge of the clause, shown in \ref{ex:mg-extrapose-cp-moved}.

\ex.\label{ex:mg-extrapose-cp}
\ag. Mir ist \tbf{wie} \tbf{es} \tbf{dir} \tbf{geht} egal.\\
 1\tsc{sg}.\tsc{dat} is how it 2\tsc{sg}.\tsc{dat} goes {the same}\\
 `I don't care how you are doing.' \flushfill{Modern German}\label{ex:mg-extrapose-cp-base}
\bg. Mir is egal \tbf{wie} \tbf{es} \tbf{dir} \tbf{geht}.\\
 1\tsc{sg}.\tsc{dat} is {the same} how it 2\tsc{sg}.\tsc{dat} goes\\
 `I don't care how you are doing.' \flushfill{Modern German}\label{ex:mg-extrapose-cp-moved}

\ref{ex:mg-extrapose-dp} illustrates that it is impossible to extrapose a DP. The clausal object of \ref{ex:mg-extrapose-cp} is replaced by the simplex noun phrase \tit{die Sache} `that matter'.
In \ref{ex:mg-extrapose-dp-base} the object, marked in bold, appears in its base position. In \ref{ex:mg-extrapose-dp-moved} it is extraposed, and the sentence is no longer grammatical.

\ex.\label{ex:mg-extrapose-dp}
\ag. Mir ist \tbf{die} \tbf{Sache} egal.\\
 1\tsc{sg}.\tsc{dat} is that matter {the same}\\
 `I don't care about that matter.' \flushfill{Modern German}\label{ex:mg-extrapose-dp-base}
\bg. *Mir ist egal \tbf{die} \tbf{Sache}.\\
 1\tsc{sg}.\tsc{dat} is {the same} that matter\\
 `I don't care about that matter.' \flushfill{Modern German}\label{ex:mg-extrapose-dp-moved}

The same asymmetry between CPs and DPs can be observed with relative clauses. A relative clause is a CP, and the head of a relative clause is a DP. The sentences in \ref{ex:extra-headed} contain the relative clause \tit{was er gestohlen hat} `what he has stolen'. This is marked in bold in the examples. The (light) head of the relative clause is \tit{das}.
In \ref{ex:extra-headed-base}, the relative clause and its head appear in base position. In \ref{ex:extra-headed-only-clause}, the relative clause is extraposed. This is grammatical, because it is possible to extrapose CPs in Modern German. In \ref{ex:extra-headed-head-clause}, the relative clause and the head are extraposed. This is ungrammatical, because it is possible to extrapose DPs.

\ex. \label{ex:extra-headed} adapted from Groos/v Riemsdijk
\ag. Jan hat das, \tbf{was} \tbf{er} \tbf{gestohlen} \tbf{hat}, zurückgegeben.\\
Jan has the money which he stolen has returned\\
\glt `Jan has returned the money that he has stolen.'\label{ex:extra-headed-base}
\bg. Jan hat das zurückgegeben, \tbf{was} \tbf{er} \tbf{gestohlen} \tbf{hat}.\\
the Hans has the money returned which he stolen has\\
\glt `Jan has returned the money that he has stolen.'\label{ex:extra-headed-only-clause}
\cg. *Jan hat zurückgegeben, das, \tbf{was} \tbf{er} \tbf{gestohlen} \tbf{hat}.\\
Jan has returned the money which he stolen has\\
\glt `Jan has returned the money that he has stolen.'\label{ex:extra-headed-head-clause}

The same can be observed in relative clauses without a head. \ref{ex:extra-headless} is the same sentence as in \ref{ex:extra-headed} only without the overt head. The relative clause is marked in bold again.
In \ref{ex:extra-headless-base}, the relative clause appears in base position. In \ref{ex:extra-headless-clause}, the relative clause is extraposed. This is grammatical, because it is possible to extrapose CPs in Modern German. In \ref{ex:extra-headless-no-rel}, the relative clause is extraposed without the relative pronouns. This is ungrammatical, because the relative pronoun is part of the CP.
This shows that the relative pronoun in headless relatives in Modern German are necessarily part of a CP, which is here a relative clause.

\ex.\label{ex:extra-headless}
\ag. Jan hat \tbf{was} \tbf{er} \tbf{gestohlen} \tbf{hat} zurückgegeben.\\
Jan has what he stolen has returned\\
`Hans has returned what he has stolen.' \citet[185]{groos1981}\label{ex:extra-headless-base}
\bg. Jan hat zurückgegeben \tbf{was} \tbf{er} \tbf{gestohlen} \tbf{hat}.\\
Jan has returned what he stolen has\\
`Hans has returned what he has stolen.' \citet[185]{groos1981}\label{ex:extra-headless-clause}
\bg. *Jan hat \tbf{was} zurückgegeben \tbf{er} \tbf{gestohlen} \tbf{hat}.\\
Jan has what returned he stolen has\\
`Hans has returned what he has stolen.' \citet[185]{groos1981}\label{ex:extra-headless-no-rel}

In conclusion, extraposition facts show, independently of the case facts, that the relative pronoun in Modern German is syntactically part of the relative clause.


\subsection{Two open issues}

The previous section showed that the relative pronoun in a Modern German headless relative is syntactically part of the relative clause. The analysis is as follows: the relative pronoun (part of the relative clause) takes the case of the predicate in the relative clause. The sentence is grammatical if the external case is contained in the relative clause case.

\ex.
\begin{forest} boom
[
   [CP, s sep=20mm
      [\ac{acc}P,
      tikz={
      \node[label=below:\tit{wen},
      draw,circle,
      scale=0.85,
      fit to=tree]{};
      }
          [\tsc{f2}]
          [\tsc{nomP}
              [\tsc{f1}]
              [XP
                  [\phantom{xxx}, roof]
              ]
          ]
      ]
      [VP
          [\tit{Maria mag}, roof]
      ]
   ]
]
\end{forest}

Now there are two problems, two subjects that need more said about them.

(1) An important requirement here is that the external case is less complex than the internal one! Is this part of the derivation as well? Because it is clear that somehow reference needs to be made to it. So far we have this picture: no external case yet. %First, normally there is a single DP connected to a predicate. Here there is a single DP, the relative pronoun, that is connected to both of the predicates. Or, it is first connected to the clause it is in, but then there is this `secondary' relationship.

(2) why do not all languages behave like Modern German? Where is space for differences between languages? For Gothic and Old High German we are still ok, because we also allow for this type of case attraction there. But how about a language like Polish, that does not allow for it? How can this type of case attraction be excluded?

%(2) if we follow this logic, to show this type of case competition? (OHG and Gothic also allow this one, so that is fine, they just allow something else in addition, and I'll get back to that later) But there are languages like Polish, which is so-called `strictly matching'. It always gives an ungrammatical result when there is case competition (and not syncretism or two identical cases).

In this section I first lay out these two problems in more detail. The next section introduces a head, external to the relative clause, that solves both these problems.


\subsubsection{How to get access to external case}

The first open issue with the analysis in X is how the relative pronoun gets access to the external case.

The relative pronoun in Modern German headless relatives is sensitive to both the internal and the external case. Consider the examples in \ref{ex:mg-internal}. In both sentences, the internal case is accusative, because the predicate in the relative clause \tit{mögen} `to like' takes accusative objects. The external case differs between the two sentences. In \ref{ex:mg-dat-acc-wen} the external case is dative, because the predicate \tit{vertrauen} `to trust' takes dative objects.  In \ref{ex:mg-int}, the external case is nominative, because \tit{besuchen} `to visit' takes nominative subjects.

\ex.\label{ex:mg-internal}
\ag. *Ich vertraue wen \tbf{auch} \tbf{Maria} \tbf{mag}. \\
I.\ac{nom} trust.1\ac{sg}\scsub{[dat]} \tsc{rel}.\ac{acc}.\tsc{an} also Maria.\ac{nom} like.3\ac{sg}\scsub{[acc]}.\\
`I trust whoever Maria also likes.' \flushfill{adapted from \pgcitealt{vogel2001}{345}}\label{ex:mg-dat-acc-wen}
\bg. Uns besucht \tbf{wen} \tbf{Maria} \tbf{mag}.\\
 we.\ac{acc} visit.3\ac{sg}\scsub{[nom]} \tsc{rel}.\ac{acc}.\tsc{an} Maria.\ac{nom} like.3\ac{sg}\scsub{[acc]}\\
 `Who visits us, Maria likes.' \flushfill{adapted from \pgcitealt{vogel2001}{343}}\label{ex:mg-int-rep}

The sentence in \ref{ex:mg-dat-acc-wen} is ungrammatical, and the one in \ref{ex:mg-int} is not. The internal case cannot be the source of ungrammaticality, because the relative clauses are identical regarding case, i.e. they both take accusative. The external case differs, however. In Chapter X I showed that headless relatives in Modern German are (just like e.g. Gothic) sensitive to the case scale: \tsc{nom} < \tsc{acc} < \tsc{dat}.

\ref{ex:mg-dat-acc-wen} is grammatical, because the internal accusative case wins over the external nominative. \ref{ex:mg-int} is ungrammatical, because the internal accusative case cannot win the case competition over the external dative. It can be concluded that the relative pronoun in Modern German headless relatives cares about both the internal and the external case.

In sum, even though the relative pronoun in Modern German headless relatives is always part of the relative clause, the relative pronoun also takes the external case into account. That means that the relative pronoun needs to have access to the main clause case. I propose that this can be achieved by introducing an external head to the relative clause. In Section X I show how this solves the issue.


\subsubsection{How to disallow the pattern}

The second issue with the analysis in X is how to disallow the Modern German pattern for other languages.

In Chapter X I showed that there are also languages that do not show case competition. Polish is an example of a language that only allows for headless relatives when the internal and the external case match (or when the relative pronoun is syncretic for these cases). The language is so-called `strictly matching'.

\ex.
\ag. *Jan ufa kogokolkwiek wpuścil do domu.\\
Jan trust.\tsc{3sg}\scsub{dat} \tsc{rel}.\tsc{acc}.\tsc{m}.\tsc{sg} let.\tsc{3sg}\scsub{acc} to home\\
`Jan trusts whoever he let into the house.'
\bg. *Jan ufa komukolkwiek wpuścil do domu.\\
Jan trust.\tsc{3sg}\scsub{dat} \tsc{rel}.\tsc{dat}.\tsc{m}.\tsc{sg} let.\tsc{3sg}\scsub{acc} to home\\
`Jan trusts whoever he let into the house.'

\ex.
\ag. *Jan lubi kogokolkwiek dokucza.\\
Jan like.\tsc{3sg}\scsub{acc} \tsc{rel}.\tsc{acc}.\tsc{m}.\tsc{sg} tease.\tsc{3sg}\scsub{dat}\\
`Jan likes whoever he teases.'
\bg. *Jan lubi komukolkwiek dokucza.\\
Jan like.\tsc{3sg}\scsub{acc} \tsc{rel}.\tsc{dat}.\tsc{m}.\tsc{sg} tease.\tsc{3sg}\scsub{dat}\\
`Jan likes whoever he teases.'

Consider the Polish and German example below.

\ex.
\ag. *Jan lubi \tbf{komu} \tbf{(kolwiek)} \tbf{dokucza}.\\
Jan like.3\tsc{sg}\scsub{[acc]} \tsc{rel}.\tsc{dat}.\tsc{sg}.\tsc{an} ever tease.3\tsc{sg}\scsub{[dat]}\\
`Jan likes whoever he teases.'\flushfill{Polish}\label{ex:polish-acc-dat}
\bg. Ich {lade ein} \tbf{wem} \tbf{auch} \tbf{Maria} \tbf{vertraut}. \\
 I.\ac{nom} invite.1\tsc{sg}\scsub{[acc]} \tsc{rel}.\ac{dat}.\tsc{an} also Maria.\ac{nom} trust.3\tsc{sg}\scsub{[dat]}.\\
 `I invite whoever Maria also trusts.' \flushfill{Modern German, adapted from \pgcitealt{vogel2001}{344}}\label{ex:mg-acc-dat-disallow}

Why is the German sentence grammatical and the Polish one not? I do not want to say that one language allows for case competition and the other one does not. Instead, the difference follows from something within the language.

On the surface forms of these headless relatives, there is nothing that differs between them. However, a closely related construction, the light-headed relative, does show a difference between the languages.

I propose that headless relatives are derived from a construction that contains an external head. This allows for making a distinction between languages that show case competition (like Modern German) and languages that do not (like Polish). In Section X I show how this solves the issue.



\subsection{Solution: an external head}

I propose a single answer to both issues raised in Section X: relative clauses in headless relatives have an external heads. So even though we say these relatives are headless relatives, at some point in the derivation there actually is a head of the relative clause.

This is the head of the relative clause, and it is deleted in Modern German (as long as it contains all cases of the relative pronoun). This external head in Polish is not deleted, because it does not add any extra definiteness or indefiniteness. Before I show how this works exactly, I show some independent evidence for assuming that there is this head.


\subsubsection{Languages with two heads}

There is also independent evidence for this head, namely from languages that actually let the head surface.

Here there are two identical copies of the head, one inside the relative clause, one outside of the relative clause.

\exg. [\tbf{doü} adiyano-no] \tbf{doü} deyalukhe\\
 sago give.3\tsc{pl}.\tsc{nonfut}-\tsc{conn} sago finished.\tsc{ajd}\\
 `The sago that they gave is finished.' \flushfill{Kombai, Dryer 2005}

I give an example of a language in which the external head follows the relative clause. There are also languages in which the head precedes the relative clause, e.g. xx

The external head is not always an exact copy of the head inside of the relative clause. An example from xx here shows that the head outside of the relative clause can also be a subset of what the element inside of the relative clause is. In this case, there is an \tit{old man} and a \tit{person}.

\exg. [\tbf{yare} gamo khereja bogi-n-o] \tbf{rumu} na-momof-a\\
 {old man} join.\tsc{ss} work \tsc{dur}.do.\tsc{3sg}.\tsc{nf}-\tsc{tr}-\tsc{conn} person my-uncle-\tsc{pred}\\
 `The old man who is joining the work is my uncle.'

So, we have the head. Translating this to relative pronouns, there is the relative pronoun, and something identical or smaller than a relative pronoun outside of the relative clause. In Chapter X I show what the feature content of the head exactly is.

Let me now show how this solves the external case problems and how it helps exclude some languages.


\subsubsection{Getting access to external case}

--give here a table with on one side the tree repeated and on the other side the head, with XP at the bottom?--
split the relative pronoun up in the w part and the other part, because this is already the subset relation

Now this accusative here can license an external nominative case. The idea is: it is inside this accusative already. We understand why it could not license a dative: \tsc{f3} is not contained in the accusative.


\subsubsection{Disallowing the pattern}

\ex. Polish light-headed relative

\ex. German light-headed relative
\a. das was
\b. das das

In German a sense of definiteness is added, because of the \tsc{d} in \tit{das}. In Polish that is not the case, because \tit{to} in Polish does not necessarily have definiteness. Evidence for that for Czech comes from Radek. I take that to exist for Polish as well.

So, Polish can involve a second head (the light head) without changing the meaning of the construction. German cannot. Now it is important to note the timing of this `repair' strategy. This has to namely be in the course of the derivation, it cannot be something that is inserted `afterwards'. What I mean with this is that there needs to be an element (which is going to become the light head in Polish but not in German) that is available during the derivation. Depending on what this element looks like, Polish shows the light head.

\ex. Polish
\a. rel clause with \tsc{acc}
\b. other element: \tsc{nom}
\b. t, c

\ex. German
\a. rel clause with \tsc{acc}
\b. other element: \tsc{nom}
\b. w (because d would change the meaning)

So, Polish has an `out', which German does not have.

What is this external element? This is the external head, that actually shows up in some languages: double-headed relative clauses.


\subsection{Featural content of the head}

indefinite noun, as cinque and the content of the external head visible in some languages

\subsection{Syntactic position of external head}

Where is this head in the syntactic structure?

  \begin{itemize}
    \item Somewhere where the relative pronoun can delete it: where it is c-commanded by the relative pronoun
    \item Somewhere where it can receive case from the main clause
    \item Where it normally is in SOV languages (does the thing in Polish move because it is a svo language?)
  \end{itemize}


So this works.

\ex.
\begin{forest} boom
[, s sep=20mm
    [CP, s sep=20mm
        [\ac{acc}P,
        tikz={
        \node[label=below:\tit{wen},
        draw,circle,
        scale=0.85,
        fit to=tree]{};
        }
            [\tsc{f2}]
            [\tsc{nomP},
            tikz={
            \node[draw,circle,transparent,
            fill=DG,fill opacity=0.2,
            scale=0.8,
            fit to=tree]{};
            }
                [\tsc{f1}]
                [XP
                    [\phantom{xxx}, roof]
                ]
            ]
        ]
        [VP
            [\tit{Maria mag}, roof]
        ]
    ]
    [\textcolor{LG}{\tsc{nomP}},
    tikz={
    \node[draw,circle,
    scale=0.8,
    fit to=tree]{};
    }
        [\textcolor{LG}{\ac{nom}},edge=LG]
        [\textcolor{LG}{XP},baseline,edge=LG
            [\textcolor{LG}{\phantom{xxx}},
            roof, baseline, edge=LG
            ]
        ]
    ]
]
\end{forest}

But here it does not.

\ex.
\begin{forest} boom
[, s sep=20mm
    [CP, s sep=20mm
        [\ac{nom}P,
        tikz={
        \node[label=below:\tit{wer},
        draw,circle,
        fill=DG,fill opacity=0.2,
        scale=0.85,
        fit to=tree]{};
        }
            [\tsc{f1}]
            [XP
                [\phantom{xxx}, roof]
            ]
        ]
        [VP
            [\tit{mir sympatisch ist}, roof]
        ]
    ]
    [\ac{acc}P
        [\tsc{f2}]
        [\textcolor{LG}{\tsc{nomP}},
        tikz={
        \node[draw,circle,
        scale=0.8,
        fit to=tree]{};
        }
            [\textcolor{LG}{\ac{nom}},edge=LG]
            [\textcolor{LG}{XP},baseline,edge=LG
                [\textcolor{LG}{\phantom{xxx}},
                roof, baseline, edge=LG
                ]
            ]
        ]
    ]
]
\end{forest}



\tit{as} is deleted because it cannot surface on its own? why not?

what is this deletion to begin with?

Finnish is like German because its light-headed relatives also add some definiteness.




\section{The external wins pattern}\label{sec:external-wins}

main clause more complex: full double-headed structure, deletion of element in speccp

Why does this not happen in Modern German? no idea..

\subsection{Old High German}

has attraction, so it could be derived from deletion under c-command under identity

\subsection{Gothic}

does not have attraction



\section{Excluding the third pattern}





\section{Alternative analyses}

\subsection{Himmelreich}



\subsection{Grafting story}

For this pattern a single element analysis seems intuitive, if you assume that case is complex and that syntax works bottom-up. First you built the relative clause, with the big case in there. Then you build the main clause and you let the more complex case in the embedded clause license the main clause predicate.

Consider the example in \ref{ex:mg-nom-acc-grafting}. Here the internal case is accusative and the external one nominative.

\exg. Uns besucht \tbf{wen} \tbf{Maria} \tbf{mag}.\\
 we.\ac{acc} visit.3\ac{sg}\scsub{[nom]} \tsc{rel}.\ac{acc}.\tsc{an} Maria.\ac{nom} like.3\ac{sg}\scsub{[acc]}\\
 `Who visits us, Maria likes.' \flushfill{adapted from \pgcitealt{vogel2001}{343}}\label{ex:mg-nom-acc-grafting}

The relative clause is built, including the accusative relative pronoun. Now the main clause predicate can merge with the nominative that is contained within the accusative.

 \ex.
 \begin{forest} boom
	 [,name=src, s sep=15mm
			[VP
			 		[\tit{besucht}, roof]
			]
		 	[,no edge, s sep=20mm
	       [\ac{acc}P,
				 tikz={
				 \node[label=below:\tit{wen},
				 draw,circle,
				 scale=0.85,
				 fit to=tree]{};
				 }
	           [\tsc{f2}]
	           [\tsc{nomP},name=tgt
	               [\tsc{f1}]
	               [XP
	                   [\phantom{xxx}, roof]
	               ]
	           ]
	       ]
				 [VP
				 		 [\tit{Maria mag}, roof]
				 ]
			]
	 ]
	 \draw (src) to[out=south east,in=north east] (tgt);
 \end{forest}\label{ex:acc-nom-grafting}

The other way around does not work. Consider \ref{ex:mg-acc-nom-grafting}. This is an example with nominative as internal case and accusative as external case.

\exg. *Ich {lade ein}, wen \tbf{mir} \tbf{sympathisch} \tbf{ist}.\\
I.\ac{nom} invite.1\ac{sg}\scsub{[acc]} \tsc{rel}.\ac{acc}.\tsc{an} I.\ac{dat} nice be.3\ac{sg}\scsub{[nom]}\\
`I invite who I like.' \flushfill{adapted from \pgcitealt{vogel2001}{344}}\label{ex:mg-acc-nom-grafting}

Now the relative clause is built first again, this time only including the nominative case. There is no accusative node to merge with for the external predicate. Instead, the relative pronoun would need to grow to accusative somehow and then the merge could take place. This is the desired result, because the sentence is ungrammatical.

\ex.
\begin{forest} boom
  [,name=src, s sep=15mm
     [VP
         [\tit{lade ein}, roof]
     ]
         [,no edge
    			[\tsc{nomP},
    			tikz={
    			\node[label=below:\tit{wer},
    			draw,circle,
    			scale=0.85,
    			fit to=tree]{};
    			}
    					[\tsc{f1}]
    					[XP
    							[\phantom{xxx}, roof]
    					]
    			]
    			[VP
    					[\tit{mir sympatisch ist}, roof]
    			]
    	 ]
    ]
\end{forest}\label{ex:nom-acc-grafting}

So, this seems to work fine. The assumptions you have to do in order to make this are the following. First, case is complex. Second, you can remerge an embedded node (grafting). For the first one I have argued in Chapter \ref{ch:decomposition}. The second one could use some additional argumentation. It is a mix between internal remerge (move) and external merge, namely external remerge. Other literature on multidominance and grafting, other phenomena. Problems: linearization, .. But even if fix all these theoretical problems, there is an empirical one.

That is, I want to connect this behavior of Modern German headless relatives to the shape of its relative pronouns. These pronouns are \tsc{wh}-elements. The OHG and Gothic ones are not \tsc{wh}, they are \tsc{d}. Their relative pronouns look different, and so their headless relatives can also behave differently.

\section{Summary}

here
