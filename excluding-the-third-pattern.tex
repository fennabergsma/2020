% !TEX root = thesis.tex

\chapter{Excluding the third pattern}\label{ch:relativization}

As I have shown in the previous chapter, there are two types of languages with case competition possible:
(1) a language like Gothic or Old High German, in which the case competition always takes place
(2) a language like Modern German, in which the case competition only takes place when the internal case can win. If not, there is no grammatical result

Crucially, there exist no language in which case competition only takes place when the external case can win. This would be a language which is the opposite of Modern German.

Schematically this can be shown like this:

\begin{table}[H]
 \center
 \caption {Variation in case competition languages}
  \begin{tabular}{ccc}
  \toprule
        & \ac{int}>\ac{ext}  & \ac{ext}>\ac{int} \\
        \cmidrule{2-3}
  Gothic, \ac{ohg} & ✔          & ✔         \\
  \ac{mg}  & ✔           & *         \\
  n.a.     & *          & ✔         \\
  \bottomrule
  \end{tabular}
\end{table}

Proposal of wh in a nutshell: the wh-relative is part of the relative clause. that's why it takes the internal case. I argue for an external head, that carries the external case. This head is also necessary to distinguish Modern German from languages like Polish. The relative pronoun deletes the external head.

Proposal of d in a nutshell: the analysis here is partly identical to the one for Modern German wh-relatives. d-relative is also part of the relative clause, and it can take the internal case, and it deletes the external head. Additionally, there can be a d, d-combination, and the relative pronoun is deleted under c-command under containment.

From these two proposals follows that it is impossible to have third pattern, which is indeed also not attested. When the external case wins over the internal one, there is a situation in which the external case could delete the internal one. So, it is impossible to have the second option but not the first one.

% Some intro about head of the relative clause and relative pronoun.

% So, we have seen that relative pronouns are sensitive to the internal case and to the external case. In this chapter we go back to what this means for the `head' in headless relatives. In headless relatives this head is namely missing. Like I said in the introduction, there are several ways of interpreting that.

% In this section I discuss the matter whether it is actually the relative pronoun or the head of the relative clause that surfaces in headless relatives. I argue that for Modern German it is always the relative pronoun in the relative clause that surfaces. This follows from the fact that Modern German has \tsc{wh}-pronouns as relative pronouns in its headless relatives. In Gothic and Old High German it is sometimes the relative pronoun and sometimes the (light) head of the relative clause that surfaces as relative pronoun. This is a consequence of the \tsc{d}-morphology of the relative pronoun in Old High German and Gothic.

% I connect this behavior of Modern German headless relatives to the shape of its relative pronouns. Modern German has \tsc{wh}-pronouns as relative pronouns in headless relative constructions. In this section I show that it is a consequence of having \tsc{wh}-pronouns as relative pronouns that the internal case can only surface as winner in the case competition. The different behavior of Gothic and Old High German headless relatives follows from the \tsc{d}-pronoun these languages use as their relative pronouns.

% In both types of languages, I analyze it as if there was a head, but it is phonologically empty/deleted. I show that this allows me to make a distinction between German and languages like Polish, that do not allow for the case competition as German does. OHG and Gothic also independently show that they can be derived from a double element structure.


\section{\tsc{wh}-relatives}

My claim: the behavior in Modern German headless relatives is a consequence of its \tsc{wh}-morphology.

In this section I discuss headless relatives in Modern German. Modern German only allows the relative pronoun only to surface in the internal case, not in the external case. The relative pronoun used in headless relatives in this language is a \tsc{wh}-pronoun. I show in this section that syntactically it is part of the relative clause but that the relative pronoun is sensitive to both the internal and external case.

Modern German differs crucially from Gothic and Old High German in that its headless relative are only grammatical when the internal case wins the case competition over the external case. In other words, the relative pronoun can only surface in the internal case, i.e. the case of the relative clause. This seems to indicate that the relative pronoun is part of the relative clause.

This can be modelled as follows:

\exg. Uns besucht \tbf{wen} \tbf{Maria} \tbf{mag}.\\
 we.\ac{acc} visit.3\ac{sg}\scsub{[nom]} \tsc{rel}.\ac{acc}.\tsc{an} Maria.\ac{nom} like.3\ac{sg}\scsub{[acc]}\\
 `Who visits us, Maria likes.' \flushfill{adapted from \pgcitealt{vogel2001}{343}}\label{ex:mg-nom-acc-wen}

 \ex.
 \begin{forest} boom
[
	 	[CP, s sep=20mm
       [\ac{acc}P,
			 tikz={
			 \node[label=below:\tit{wen},
			 draw,circle,
			 scale=0.85,
			 fit to=tree]{};
			 }
           [\tsc{f2}]
           [\tsc{nomP}
               [\tsc{f1}]
               [XP
                   [\phantom{xxx}, roof]
               ]
           ]
       ]
			 [VP
			 		 [\tit{Maria mag}, roof]
			 ]
		]
]
 \end{forest}

The structure shows an accusative in the relative clause. An \tsc{acc}P by definition contains a \tsc{nom}P, so the nominative is there.

We also understand that the other way around does not work: a nominative in the relative clause does not contain an accusative.


\subsection{The relative pronoun in the relative clause}

In this section I show that the relative pronoun in Modern German headless relatives is part of the relative clause. The evidence comes from extraposition. In Modern German, it is possible to extrapose a clause (a CP), but not a noun phrase (DP). The relative clause including the relative pronoun in a headless relative can be extraposed, which indicates that the relative pronoun is part of the relative clause. In what follows I show that CPs can be extraposed and DPs cannot. Then I illustrate how relative clauses including the relative pronoun in headless relatives pattern with CPs.

The sentences in \ref{ex:mg-extrapose-cp} show that it is possible to extrapose CP. In \ref{ex:mg-extrapose-cp-base}, the clausal object \tit{wie es dir geht} `how you are doing', marked here in bold, appears in its base position. This objects can be extraposed to the right edge of the clause, shown in \ref{ex:mg-extrapose-cp-moved}.

\ex.\label{ex:mg-extrapose-cp}
\ag. Mir ist \tbf{wie} \tbf{es} \tbf{dir} \tbf{geht} egal.\\
 1\tsc{sg}.\tsc{dat} is how it 2\tsc{sg}.\tsc{dat} goes {the same}\\
 `I don't care how you are doing.' \flushfill{Modern German}\label{ex:mg-extrapose-cp-base}
\bg. Mir is egal \tbf{wie} \tbf{es} \tbf{dir} \tbf{geht}.\\
 1\tsc{sg}.\tsc{dat} is {the same} how it 2\tsc{sg}.\tsc{dat} goes\\
 `I don't care how you are doing.' \flushfill{Modern German}\label{ex:mg-extrapose-cp-moved}

\ref{ex:mg-extrapose-dp} illustrates that it is impossible to extrapose a DP. The clausal object of \ref{ex:mg-extrapose-cp} is replaced by a simplex noun phrase \tit{die Sache} `that matter'.
In \ref{ex:mg-extrapose-dp-base} the object, marked in bold, appears in its base position. In \ref{ex:mg-extrapose-dp-moved} it is extraposed, and the sentence is no longer grammatical.

\ex.\label{ex:mg-extrapose-dp}
\ag. Mir ist \tbf{die} \tbf{Sache} egal.\\
 1\tsc{sg}.\tsc{dat} is that matter {the same}\\
 `I don't care about that matter.' \flushfill{Modern German}\label{ex:mg-extrapose-dp-base}
\bg. *Mir ist egal \tbf{die} \tbf{Sache}.\\
 1\tsc{sg}.\tsc{dat} is {the same} that matter\\
 `I don't care about that matter.' \flushfill{Modern German}\label{ex:mg-extrapose-dp-moved}

The same CP-DP asymmetry can be observed with relative clauses. A relative clause is a CP, and the head of a relative clause is a DP. The sentences in \ref{ex:extra-headed} contain the relative clause \tit{was er gestohlen hat} `what he has stolen'. This is marked in bold in the examples. The (light) head of the relative clause is \tit{das}.
In \ref{ex:extra-headed-base}, the relative clause and its head appear in base position. In \ref{ex:extra-headed-only-clause}, the relative clause is extraposed. This is grammatical, because it is possible to extrapose CPs in Modern German. In \ref{ex:extra-headed-head-clause}, the relative clause and the head are extraposed. This is ungrammatical, because it is possible to extrapose DPs.

\ex. \label{ex:extra-headed} adapted from Groos/v Riemsdijk
\ag. Jan hat das, \tbf{was} \tbf{er} \tbf{gestohlen} \tbf{hat}, zurückgegeben.\\
Jan has the money which he stolen has returned\\
\glt `Jan has returned the money that he has stolen.'\label{ex:extra-headed-base}
\bg. Jan hat das zurückgegeben, \tbf{was} \tbf{er} \tbf{gestohlen} \tbf{hat}.\\
the Hans has the money returned which he stolen has\\
\glt `Jan has returned the money that he has stolen.'\label{ex:extra-headed-only-clause}
\cg. *Jan hat zurückgegeben, das, \tbf{was} \tbf{er} \tbf{gestohlen} \tbf{hat}.\\
Jan has returned the money which he stolen has\\
\glt `Jan has returned the money that he has stolen.'\label{ex:extra-headed-head-clause}

The same can be observed in relative clauses without a head. \ref{ex:extra-headless} is the same sentence as in \ref{ex:extra-headed} only without the overt head. The relative clause is marked in bold again.
In \ref{ex:extra-headless-base}, the relative clause and its head appear in base position. In \ref{ex:extra-headless-clause}, the relative clause is extraposed. This is grammatical, because it is possible to extrapose CPs in Modern German. In \ref{ex:extra-headless-no-rel}, the relative clause and the head are extraposed. This is ungrammatical, because it is possible to extrapose DPs.
This shows that the relative pronoun in headless relatives in Modern German are necessarily part of the relative clause (a CP).

\ex.\label{ex:extra-headless}
\ag. Jan hat \tbf{was} \tbf{er} \tbf{gestohlen} \tbf{hat} zurückgegeben.\\
Jan has what he stolen has returned\\
`Hans has returned what he has stolen.' \citet[185]{groos1981}\label{ex:extra-headless-base}
\bg. Jan hat zurückgegeben \tbf{was} \tbf{er} \tbf{gestohlen} \tbf{hat}.\\
Jan has returned what he stolen has\\
`Hans has returned what he has stolen.' \citet[185]{groos1981}\label{ex:extra-headless-clause}
\bg. *Jan hat \tbf{was} zurückgegeben \tbf{er} \tbf{gestohlen} \tbf{hat}.\\
Jan has what returned he stolen has\\
`Hans has returned what he has stolen.' \citet[185]{groos1981}\label{ex:extra-headless-no-rel}

In conclusion, extraposition facts show independently of the case facts that the relative pronoun in Modern German is syntactically part of the relative clause.


\subsection{Two problems}

The previous section showed that the relative pronoun in a Modern German headless relative is syntactically part of the relative clause. The analysis is as follows: the relative pronoun (part of the relative clause) takes the case of the predicate in the relative clause. The sentence is grammatical if the external case is contained in the relative clause case.

The opposite does not work: if the external case is not contained in the relative clause, that case requirement cannot be satisfied. The external case cannot fly into the relative clause.

Now there are two problems, to subjects that need more said about them.

(1) we keep talking about this external case. Is this part of the derivation as well? Because it is clear that somehow reference needs to be made to it. So far we have this picture: no external case yet. %First, normally there is a single DP connected to a predicate. Here there is a single DP, the relative pronoun, that is connected to both of the predicates. Or, it is first connected to the clause it is in, but then there is this `secondary' relationship.

(2) why do not all languages behave like Modern German? Where is place for differences between languages? For Gothic and Old High German we are still ok, because we also allow for this type of case attraction there. But how about a language like Polish, that does not allow for it? How can this type of case attraction be excluded?

To solve both of these problems, I introduce an external head. %(2) if we follow this logic, to show this type of case competition? (OHG and Gothic also allow this one, so that is fine, they just allow something else in addition, and I'll get back to that later) But there are languages like Polish, which is so-called `strictly matching'. It always gives an ungrammatical result when there is case competition (and not syncretism or two identical cases).


\subsubsection{Care about external case}

%%relative pronoun cares about internal and external case
Just like in Gothic and Old High German, the Modern German relative pronoun is sensitive to the internal and external case.
The internal case only surfaces when it is more complex than the external case.

Consider the examples in \ref{ex:mg-internal}. In both sentences, the internal case is accusative, because the predicate in the relative clause \tit{mögen} `to like' takes accusative objects. The external case differs between the two sentences. In \ref{ex:mg-dat-acc-wen} the external case is dative, because the predicate \tit{vertrauen} `to trust' takes dative objects.  In \ref{ex:mg-nom-acc-wen}, the external case is nominative, because \tit{besuchen} `to visit' takes nominative subjects.

\ex.\label{ex:mg-internal}
\ag. *Ich vertraue wen \tbf{auch} \tbf{Maria} \tbf{mag}. \\
I.\ac{nom} trust.1\ac{sg}\scsub{[dat]} \tsc{rel}.\ac{acc}.\tsc{an} also Maria.\ac{nom} like.3\ac{sg}\scsub{[acc]}.\\
`I trust whoever Maria also likes.' \flushfill{adapted from \pgcitealt{vogel2001}{345}}\label{ex:mg-dat-acc-wen}
\bg. Uns besucht \tbf{wen} \tbf{Maria} \tbf{mag}.\\
 we.\ac{acc} visit.3\ac{sg}\scsub{[nom]} \tsc{rel}.\ac{acc}.\tsc{an} Maria.\ac{nom} like.3\ac{sg}\scsub{[acc]}\\
 `Who visits us, Maria likes.' \flushfill{adapted from \pgcitealt{vogel2001}{343}}\label{ex:mg-nom-acc-wen}

The sentence in \ref{ex:mg-dat-acc-wen} is ungrammatical, and the one in \ref{ex:mg-nom-acc-wen} is not. The relative clauses in both sentences are identical case-wise, so the difference in ungrammaticality cannot lie there. The only observable differences are found in the main clause. The former sentence is ungrammatical, because the internal accusative case cannot win the case competition over the external dative. The latter sentence is grammatical, because the internal accusative case wins over the external nominative. The conclusion that can be drawn from this is that the relative pronoun in Modern German headless relatives cares about both the internal and the external case.

In sum, even though the relative pronoun in Modern German headless relatives is always part of the relative clause, the relative pronoun also takes the external case into account. That means that the relative pronoun somehow needs to have access to the main clause case. In the next section I argue that this `somehow' can be done with an external head.


\subsubsection{How to exclude this}

Now I do not want to say that Polish and German differ in that Polish is `strictly matching' and Modern German is `more relaxed', that this is something construction- and language-specific. Instead, I let the distinction in headless relatives follow from something within the language. This something is their light-headed relatives.

Then there are languages like Polish. In Polish, case competition does not take place at all. All combinations are ungrammatical. Consider x.

\ex.
\ag. *Jan lubi \tbf{komu} \tbf{(kolwiek)} \tbf{dokucza}.\\
Jan like.3\tsc{sg}\scsub{[acc]} \tsc{rel}.\tsc{dat}.\tsc{sg}.\tsc{an} ever tease.3\tsc{sg}\scsub{[dat]}\\
`Jan likes whoever he teases.'
\bg. *Jan ufa \tbf{komu} \tbf{(kolwiek)} \tbf{wpuścil} \tbf{do} \tbf{domu}.\\
 Jan trust.3\tsc{sg}\scsub{[dat]} \tsc{rel}.\tsc{dat}.\tsc{sg}.\tsc{an} ever let.3\tsc{sg}\scsub{[acc]} to home\\
 `Jan trusts whoever he let into the house.'

Where does the difference between Modern German and Polish lie? Where can we model this difference? Polish and Modern German seem to be pretty similar here. Both have a \tsc{wh}-pronoun functioning as relative pronoun for instance. In Section X I show an external head can solve this problem too.



\subsection{Solution: an external head}

I propose a single answer to both of these questions: there is an external head. So even though we say these relatives are headless relatives, at some point in the derivation there actually is a head of the relative clause.


This is the head of the relative clause, and it is deleted in Modern German (as long as it contains all cases of the relative pronoun). This external head in Polish is not deleted, because it does not add any extra definiteness or indefiniteness. Before I show how this works exactly, I show some independent evidence for assuming that there is this head.


\subsubsection{Languages with two heads}

There is also independent evidence for this head, namely from languages that actually let the head surface.

Here there are two identical copies of the head, one inside the relative clause, one outside of the relative clause.

\exg. [\tbf{doü} adiyano-no] \tbf{doü} deyalukhe\\
 sago give.3\tsc{pl}.\tsc{nonfut}-\tsc{conn} sago finished.\tsc{ajd}\\
 `The sago that they gave is finished.' \flushfill{Kombai, Dryer 2005}

I give an example of a language in which the external head follows the relative clause. There are also languages in which the head precedes the relative clause, e.g. xx

The external head is not always an exact copy of the head inside of the relative clause. An example from xx here shows that the head outside of the relative clause can also be a subset of what the element inside of the relative clause is. In this case, there is an \tit{old man} and a \tit{person}.

\exg. [\tbf{yare} gamo khereja bogi-n-o] \tbf{rumu} na-momof-a\\
 {old man} join.\tsc{ss} work \tsc{dur}.do.\tsc{3sg}.\tsc{nf}-\tsc{tr}-\tsc{conn} person my-uncle-\tsc{pred}\\
 `The old man who is joining the work is my uncle.'

So, we have the head. Translating this to relative pronouns, there is the relative pronoun, and something identical or smaller than a relative pronoun outside of the relative clause. In Chapter X I show what the feature content of the head exactly is.

Let me now show how this solves the external case problems and how it helps exclude some languages.


\subsubsection{Solving external case care}

--give here a table with on one side the tree repeated and on the other side the head, with XP at the bottom?--
split the relative pronoun up in the w part and the other part, because this is already the subset relation

Now this accusative here can license an external nominative case. The idea is: it is inside this accusative already. We understand why it could not license a dative: \tsc{f3} is not contained in the accusative.


\subsubsection{Excluding some languages}

\ex. Polish light-headed relative

\ex. German light-headed relative
\a. das was
\b. das das

In German a sense of definiteness is added, because of the \tsc{d} in \tit{das}. In Polish that is not the case, because \tit{to} in Polish does not necessarily have definiteness. Evidence for that for Czech comes from Radek. I take that to exist for Polish as well.

So, Polish can involve a second head (the light head) without changing the meaning of the construction. German cannot. Now it is important to note the timing of this `repair' strategy. This has to namely be in the course of the derivation, it cannot be something that is inserted `afterwards'. What I mean with this is that there needs to be an element (which is going to become the light head in Polish but not in German) that is available during the derivation. Depending on what this element looks like, Polish shows the light head.

\ex. Polish
\a. rel clause with \tsc{acc}
\b. other element: \tsc{nom}
\b. t, c

\ex. German
\a. rel clause with \tsc{acc}
\b. other element: \tsc{nom}
\b. w (because d would change the meaning)

So, Polish has an `out', which German does not have.

What is this external element? This is the external head, that actually shows up in some languages: double-headed relative clauses.



\subsection{Syntactic position of external head}

Where is this head in the syntactic structure?

  \begin{itemize}
    \item Somewhere where the relative pronoun can delete it: where it is c-commanded by the relative pronoun
    \item Somewhere where it can receive case from the main clause
    \item Where it normally is in SOV languages (does the thing in Polish move because it is a svo language?)
  \end{itemize}


So this works.

\ex.
\begin{forest} boom
[, s sep=20mm
    [CP, s sep=20mm
        [\ac{acc}P,
        tikz={
        \node[label=below:\tit{wen},
        draw,circle,
        scale=0.85,
        fit to=tree]{};
        }
            [\tsc{f2}]
            [\tsc{nomP},
            tikz={
            \node[draw,circle,transparent,
            fill=DG,fill opacity=0.2,
            scale=0.8,
            fit to=tree]{};
            }
                [\tsc{f1}]
                [XP
                    [\phantom{xxx}, roof]
                ]
            ]
        ]
        [VP
            [\tit{Maria mag}, roof]
        ]
    ]
    [\textcolor{LG}{\tsc{nomP}},
    tikz={
    \node[draw,circle,
    scale=0.8,
    fit to=tree]{};
    }
        [\textcolor{LG}{\ac{nom}},edge=LG]
        [\textcolor{LG}{XP},baseline,edge=LG
            [\textcolor{LG}{\phantom{xxx}},
            roof, baseline, edge=LG
            ]
        ]
    ]
]
\end{forest}

But here it does not.

\ex.
\begin{forest} boom
[, s sep=20mm
    [CP, s sep=20mm
        [\ac{nom}P,
        tikz={
        \node[label=below:\tit{wer},
        draw,circle,
        fill=DG,fill opacity=0.2,
        scale=0.85,
        fit to=tree]{};
        }
            [\tsc{f1}]
            [XP
                [\phantom{xxx}, roof]
            ]
        ]
        [VP
            [\tit{mir sympatisch ist}, roof]
        ]
    ]
    [\ac{acc}P
        [\tsc{f2}]
        [\textcolor{LG}{\tsc{nomP}},
        tikz={
        \node[draw,circle,
        scale=0.8,
        fit to=tree]{};
        }
            [\textcolor{LG}{\ac{nom}},edge=LG]
            [\textcolor{LG}{XP},baseline,edge=LG
                [\textcolor{LG}{\phantom{xxx}},
                roof, baseline, edge=LG
                ]
            ]
        ]
    ]
]
\end{forest}



\tit{as} is deleted because it cannot surface on its own? why not?

what is this deletion to begin with?

Finnish is like German because its light-headed relatives also add some definiteness.




\section{\tsc{d}-relatives}

My claim: d-morphology allows for both directions

\subsection{Like Modern German}

d-relative pronoun is also in embedded clause
same story

\subsection{Derived from light-headed relatives}

main clause more complex: full double-headed structure, deletion of element in speccp

Why does this not happen in Modern German? no idea..

\subsubsection{Old High German}

has attraction, so it could be derived from deletion under c-command under identity

\subsubsection{Gothic}

does not have attraction








\section{Alternative analyses}

\subsection{Himmelreich}



\subsection{Grafting story}

For this pattern a single element analysis seems intuitive, if you assume that case is complex and that syntax works bottom-up. First you built the relative clause, with the big case in there. Then you build the main clause and you let the more complex case in the embedded clause license the main clause predicate.

Consider the example in \ref{ex:mg-nom-acc-grafting}. Here the internal case is accusative and the external one nominative.

\exg. Uns besucht \tbf{wen} \tbf{Maria} \tbf{mag}.\\
 we.\ac{acc} visit.3\ac{sg}\scsub{[nom]} \tsc{rel}.\ac{acc}.\tsc{an} Maria.\ac{nom} like.3\ac{sg}\scsub{[acc]}\\
 `Who visits us, Maria likes.' \flushfill{adapted from \pgcitealt{vogel2001}{343}}\label{ex:mg-nom-acc-grafting}

The relative clause is built, including the accusative relative pronoun. Now the main clause predicate can merge with the nominative that is contained within the accusative.

 \ex.
 \begin{forest} boom
	 [,name=src, s sep=15mm
			[VP
			 		[\tit{besucht}, roof]
			]
		 	[,no edge, s sep=20mm
	       [\ac{acc}P,
				 tikz={
				 \node[label=below:\tit{wen},
				 draw,circle,
				 scale=0.85,
				 fit to=tree]{};
				 }
	           [\tsc{f2}]
	           [\tsc{nomP},name=tgt
	               [\tsc{f1}]
	               [XP
	                   [\phantom{xxx}, roof]
	               ]
	           ]
	       ]
				 [VP
				 		 [\tit{Maria mag}, roof]
				 ]
			]
	 ]
	 \draw (src) to[out=south east,in=north east] (tgt);
 \end{forest}\label{ex:acc-nom-grafting}

The other way around does not work. Consider \ref{ex:mg-acc-nom-grafting}. This is an example with nominative as internal case and accusative as external case.

\exg. *Ich {lade ein}, wen \tbf{mir} \tbf{sympathisch} \tbf{ist}.\\
I.\ac{nom} invite.1\ac{sg}\scsub{[acc]} \tsc{rel}.\ac{acc}.\tsc{an} I.\ac{dat} nice be.3\ac{sg}\scsub{[nom]}\\
`I invite who I like.' \flushfill{adapted from \pgcitealt{vogel2001}{344}}\label{ex:mg-acc-nom-grafting}

Now the relative clause is built first again, this time only including the nominative case. There is no accusative node to merge with for the external predicate. Instead, the relative pronoun would need to grow to accusative somehow and then the merge could take place. This is the desired result, because the sentence is ungrammatical.

\ex.
\begin{forest} boom
  [,name=src, s sep=15mm
     [VP
         [\tit{lade ein}, roof]
     ]
         [,no edge
    			[\tsc{nomP},
    			tikz={
    			\node[label=below:\tit{wer},
    			draw,circle,
    			scale=0.85,
    			fit to=tree]{};
    			}
    					[\tsc{f1}]
    					[XP
    							[\phantom{xxx}, roof]
    					]
    			]
    			[VP
    					[\tit{mir sympatisch ist}, roof]
    			]
    	 ]
    ]
\end{forest}\label{ex:nom-acc-grafting}

So, this seems to work fine. The assumptions you have to do in order to make this are the following. First, case is complex. Second, you can remerge an embedded node (grafting). For the first one I have argued in Chapter \ref{ch:decomposition}. The second one could use some additional argumentation. It is a mix between internal remerge (move) and external merge, namely external remerge. Other literature on multidominance and grafting, other phenomena. Problems: linearization, .. But even if fix all these theoretical problems, there is an empirical one.

That is, I want to connect this behavior of Modern German headless relatives to the shape of its relative pronouns. These pronouns are \tsc{wh}-elements. The OHG and Gothic ones are not \tsc{wh}, they are \tsc{d}. Their relative pronouns look different, and so their headless relatives can also behave differently.
