% !TEX root = thesis.tex

\chapter{Deriving the matching type}\label{ch:deriving-matching}

In Chapter \ref{ch:the-basic-idea}, I suggested that languages of the matching type have a lexical entry that spells out phi features and another one that spells out case features. This is the crucial difference with internal-only languages such as Modern German, that have a portmanteau for phi and case features. It means that the internal syntax of light heads and relative pronouns looks as shown in Figure \ref{fig:rel-lh-matching-simple}.

\begin{figure}[htbp]
  \center
  \begin{tabular}[b]{ccc}
      \toprule
      light head & & relative pronoun \\
      \cmidrule(lr){1-1} \cmidrule(lr){3-3}
      \begin{forest} boom
      [\tsc{k}P, s sep = 17.5 mm
          [ϕP,
          tikz={
          \node[draw,circle,
          scale=0.85,
          fit to=tree]{};
          }
              [\phantom{xxx}, roof]
          ]
          [\tsc{k}P,
          tikz={
          \node[draw,circle,
          scale=0.85,
          fit to=tree]{};
          }
              [\tsc{k}, baseline]
          ]
      ]
      \end{forest}
      & \phantom{x} &
    \begin{forest} boom
      [\tsc{rel}P, s sep = 12.5 mm
          [\tsc{rel}P,
          tikz={
          \node[draw,circle,
          scale=0.85,
          fit to=tree]{};
          }
              [\phantom{xxx}, roof, baseline]
          ]
          [\tsc{k}P, s sep = 17.5 mm
              [ϕP,
              tikz={
              \node[draw,circle,
              scale=0.85,
              fit to=tree]{};
              }
                  [\phantom{xxx}, roof]
              ]
              [\tsc{k}P,
              tikz={
              \node[draw,circle,
              scale=0.85,
              fit to=tree]{};
              }
                  [\tsc{k}, baseline]
              ]
          ]
      ]
    \end{forest}\\
      \bottomrule
  \end{tabular}
   \caption {\tsc{lh} and \tsc{rp} in the matching type}
  \label{fig:rel-lh-matching-simple}
\end{figure}

These lexical entries lead to the grammaticality pattern shown in Table \ref{tbl:overview-matching}.

\begin{table}[htbp]
  \center
  \caption{Grammaticality in the matching type (repeated)}
  \begin{adjustbox}{max width=\textwidth}
  \begin{tabular}{cccccc}
    \toprule
    situation           & \multicolumn{2}{c}{lexical entries}       & containment         & deleted             & surfacing           \\
    \cmidrule(lr){1-1}    \cmidrule(lr){2-3}                          \cmidrule(lr){4-4}    \cmidrule(lr){5-5}    \cmidrule(lr){6-6}
                        & \tsc{lh}            & \tsc{rp}            &                     &                     &                     \\
                          \cmidrule(lr){2-2}    \cmidrule(lr){3-3}
  \tsc{k}\scsub{int} = \tsc{k}\scsub{ext}               &
  [\tsc{k}\scsub{1}], [ϕ]                               &
  [\tsc{rel}], [\tsc{k}\scsub{1}], [ϕ]                  &
  structure & \tsc{lh} & \tsc{rp}\scsub{int}            \\
  \tsc{k}\scsub{int} > \tsc{k}\scsub{ext}               &
  [\tsc{k}\scsub{1}], [ϕ]                               &
  [\tsc{rel}], [\tsc{k}\scsub{2}[\tsc{k}\scsub{1}]], [ϕ]&
  no & none & *                                         \\
  \tsc{k}\scsub{int} < \tsc{k}\scsub{ext}               &
  [\tsc{k}\scsub{2}[\tsc{k}\scsub{1}]], [ϕ]             &
  [\tsc{rel}], [\tsc{k}\scsub{1}], [ϕ]                  &
  no & none & *                                         \\
  \bottomrule
  \end{tabular}
  \end{adjustbox}
\label{tbl:overview-matching}
\end{table}

Consider first the situation in which the internal and the external case match. The light head consists of a phi feature morpheme and a case feature morpheme. The relative pronoun consists of the same two morphemes plus an additional morpheme that spells out the feature \tsc{rel}. The lexical entries create a syntactic structure such that the light head is structurally contained in the relative pronoun. Therefore, the light head can be deleted, and the relative pronoun surfaces, bearing the internal case.
In this situation, whether there is a phi and case feature portmanteau (as in internal-only languages) or two separate morphemes for the features (as in matching languages) does not make a difference for whether or not the light head can be deleted. It can in both cases.

Consider now the situation in which the internal case would win the case competition. The light head consists of a phi feature morpheme and a case feature morpheme. The relative pronoun consists of the same phi feature morpheme, a case morpheme that that contains at least one more case feature than the light head (\tsc{k}\scsub{2} in Figure \ref{tbl:overview-matching}) plus an additional morpheme that spells out the feature \tsc{rel}. The lexical entries create a syntactic structure such that neither the light head nor the relative pronoun is structurally contained in the other element. Therefore, none of the elements can be deleted, and there is no headless relative construction possible.
In this situation, whether there is a phi and case feature portmanteau (as in internal-only languages) or two separate morphemes for the features (as in matching languages) makes crucial difference for whether or not the light head can be deleted. It can when there is a phi and case feature portmanteau and it cannot when there are two separate morphemes for the features.

Finally, consider the situation in which the external case would win the case competition. The relative pronoun consists of a phi feature morpheme, a case feature morpheme and an additional morpheme that spells out the feature \tsc{rel}. Compared to the relative pronoun, the light head lacks the morpheme that spells out \tsc{rel}, and it contains at least one more case feature (\tsc{k}\scsub{2} in Figure \ref{tbl:overview-matching}). The lexical entries create a syntactic structure such that neither the light head nor the relative pronoun is structurally contained in the other element. Therefore, none of the elements can be deleted, and there is no headless relative construction possible.
In this situation, whether there is a phi and case feature portmanteau (as in internal-only languages) or two separate morphemes for the features (as in matching languages) does not make a difference for whether or not the light head or the relative pronoun can be deleted. It cannot in both cases.

In Chapter \ref{ch:typology}, I showed that Polish is a language of the matching type. In this chapter, I show that Polish light heads and relative pronouns have the type of internal syntax described in Figure \ref{fig:rel-lh-matching-simple}. I give a compact version of the internal syntax of Polish light heads and relative pronouns in Figure \ref{fig:rel-lh-pol}.

\begin{figure}[htbp]
  \center
  \begin{adjustbox}{max width=\textwidth}
    \begin{tabular}[b]{ccc}
        \toprule
        light head & & relative pronoun \\
        \cmidrule(lr){1-1} \cmidrule(lr){3-3}
        \begin{forest} boom
        [\tsc{k}P, s sep = 17.5 mm
            [ϕP,
            tikz={
            \node[label=below:\tit{o},
            draw,circle,
            scale=0.85,
            fit to=tree]{};
            }
                [\phantom{xxx}, roof]
            ]
            [\tsc{k}P,
            tikz={
            \node[label=below:\tit{go/mu},
            draw,circle,
            scale=0.85,
            fit to=tree]{};
            }
                [\tsc{k}, baseline]
            ]
        ]
        \end{forest}
        & \phantom{x} &
      \begin{forest} boom
        [\tsc{rel}P, s sep = 12.5 mm
            [\tsc{rel}P,
            tikz={
            \node[label=below:\tit{k},
            draw,circle,
            scale=0.85,
            fit to=tree]{};
            }
                [\phantom{xxx}, roof, baseline]
            ]
            [\tsc{k}P, s sep = 17.5 mm
                [ϕP,
                tikz={
                \node[label=below:\tit{o},
                draw,circle,
                scale=0.85,
                fit to=tree]{};
                }
                    [\phantom{xxx}, roof]
                ]
                [\tsc{k}P,
                tikz={
                \node[label=below:\tit{go/mu},
                draw,circle,
                scale=0.85,
                fit to=tree]{};
                }
                    [\tsc{k}, baseline]
                ]
            ]
        ]
      \end{forest}\\
        \bottomrule
    \end{tabular}
  \end{adjustbox}
   \caption {\tsc{lh} and \tsc{rp} in Polish}
  \label{fig:rel-lh-pol}
\end{figure}

Consider the light head in Figure \ref{fig:rel-lh-pol}.
Light heads (i.e. the phi and case features) in Polish are spelled out by two morphemes, which are both circled. The phi features are spelled out as \tit{o} and the case features are spelled out as \tit{go} or \tit{mu}, depending on which case they realize.
Consider the relative pronoun in Figure \ref{fig:rel-lh-pol}.
Relative pronouns in Polish consist of three morphemes: the constituent that forms the light head (i.e. phi and case feature morphemes) and the \tsc{rel}P, again indicated by the circles. The constituent that forms the light head has the same spellout as in the light head (\tit{o} and \tit{go} or \tit{mu}), and the \tsc{rel}P is spelled out as \tit{k}.
Throughout this chapter, I discuss the exact feature content of relative pronouns and light heads, I give lexical entries for them, and I show how these lexical entries lead to the internal syntax shown in Figure \ref{fig:rel-lh-pol}.

The chapter is structured as follows.
First, I discuss the relative pronoun. I decompose it into the three morphemes I showed in Figure \ref{fig:rel-lh-pol}. Then I show which features each of the morphemes corresponds to.
Then I discuss the light head. I argue that Polish headless relatives are, just as Modern German headless relatives, derived from a type of light-headed relative clause that does not surface in the language. I show that the light head corresponds to one of the morphemes of the relative pronoun (the \tsc{k}P in Figure \ref{fig:rel-lh-pol}).
Importantly, the features that form the Polish light head and relative pronoun are the same ones that form the Modern German light head and relative pronoun. The only difference between the two languages is how the features are spelled out.
Finally, I compare the internal syntax of the light head and the relative pronoun. I show that the light head can only be deleted when the internal case matches the external case. When the internal and external case differ, none of the elements can be deleted.


\section{The Polish relative pronoun}\label{sec:pol-rel}

In the introduction of this chapter, I suggested that the internal syntax of relative pronouns in Polish looks as shown in \ref{ex:simple-matching-rp}.

\ex.\label{ex:simple-matching-rp}
\begin{forest} boom
  [\tsc{rel}P, s sep = 12.5 mm
      [\tsc{rel}P,
      tikz={
      \node[label=below:\tit{k},
      draw,circle,
      scale=0.85,
      fit to=tree]{};
      }
          [\phantom{xxx}, roof]
      ]
      [\tsc{k}P, s sep = 17.5 mm
          [ϕP,
          tikz={
          \node[label=below:\tit{o},
          draw,circle,
          scale=0.85,
          fit to=tree]{};
          }
              [\phantom{xxx}, roof]
          ]
          [\tsc{k}P,
          tikz={
          \node[label=below:\tit{go/mu},
          draw,circle,
          scale=0.85,
          fit to=tree]{};
          }
              [\tsc{k}]
          ]
      ]
  ]
\end{forest}

In Chapter \ref{ch:the-basic-idea}, I suggested that relative pronouns consist of at least three features: \tsc{rel}, ϕ and \tsc{k}.
I showed that Modern German relative pronouns contain more features than that in Chapter \ref{ch:deriving-onlyinternal}.
In this section, I show that Polish relative pronouns consist of the same features.
Still, the crucial claim I made in Chapter \ref{ch:the-basic-idea} remains unchanged: matching languages (of which Polish is an example) have a separate morpheme for phi features, one for case features and one for the features the light head does not contain. Actually, the morpheme for case features contains a number feature and the phi feature morpheme does not contain one, but this does not influence the point here.
I show the complete structure that I work towards in this section in \ref{ex:pol-rp}.

\ex.\label{ex:pol-rp}
\begin{adjustbox}{max width=0.9\textwidth}
\begin{forest} boom
  [\tsc{rel}P, s sep=35mm
      [\tsc{rel}P,
      tikz={
      \node[label=below:\tit{k},
      draw,circle,
      scale=0.95,
      fit to=tree]{};
      }
          [\tsc{rel}]
          [\tsc{wh}P
              [\tsc{wh}]
              [\#P
                  [\#]
                  [\tsc{an}P
                      [\tsc{an}]
                      [\tsc{cl}]
                  ]
              ]
          ]
      ]
      [\tsc{acc}P, s sep=25mm
      [\tsc{an}P,
          tikz={
          \node[label=below:\tit{o},
          draw,circle,
          scale=0.9,
          fit to=tree]{};
          }
          [\tsc{an}P]
          [\tsc{cl}P
              [\tsc{cl}]
              [\tsc{ref}]
          ]
      ]
          [\tsc{k}P,
          tikz={
          \node[label=below:\tit{go/mu},
          draw,circle,
          scale=0.9,
          fit to=tree]{};
          }
              [\tsc{k}]
              [\#P
                  [\#]
              ]
          ]
      ]
  ]
  \end{forest}
  \end{adjustbox}

I discuss two relative pronouns: the animate accusative and the animate dative. These are the two forms that I compare the internal syntax of in Section \ref{sec:comparing-polish}. I show them in \ref{ex:pol-rels}.

\ex.\label{ex:pol-rels}
\a. k-o-go `\tsc{rp}.\tsc{an}.\tsc{acc}'
\b. k-o-mu `\tsc{rp}.\tsc{an}.\tsc{dat}'

I decompose the relative pronouns into three morphemes: \tit{k}, \tit{o} and the suffix (\tit{go} or \tit{mu}). For each morpheme, I discuss which features they spell out, I give their lexical entries, and I show how I construct the relative pronouns by combining the separate morphemes.

I start with the suffixes \tit{go} and \tit{mu}.
These two morphemes correspond to what I called the case feature morpheme in Chapter \ref{ch:the-basic-idea} and the introduction to this chapter. In addition, the morphemes spell out a number feature.

To determine their exact feature content, I first focus on \tit{mu}. Then, I extend the analysis to \tit{go}. The morpheme \tit{mu} (and also the \tit{go}) do not only appear in relative pronouns. They also show up as nominal endings, adjectival endings and in other pronominal forms. Interestingly, forms containing \tit{mu} are often syncretic between masculine dative and neuter dative. Consider Table \ref{tbl:pol-datives}.

\begin{table}[htbp]
  \center
  \caption{Syncretism in dative neuter pronouns in Polish \pgcitep{swan2002}{156,171}}
  \begin{tabular}[b]{ccc}
    \toprule
                      & \tsc{m}.\tsc{dat}   & \tsc{n}.\tsc{dat}  \\
    \cmidrule{2-3}
    \tit{je}-pronoun  & je-mu    & je-mu   \\
    \tit{n}-pronoun   & nie-mu   & nie-mu  \\
    \tsc{dem}         & te-mu    & te-mu   \\
    \bottomrule
  \end{tabular}
  \label{tbl:pol-datives}
\end{table}

The table shows three forms: the \tit{je}-pronoun is the long version of the third person singular pronoun (not the clitic), the \tit{n}-pronoun is the third person singular pronoun used after prepositions \pgcitep{swan2002}{156-157}, and the \tsc{dem} is the demonstrative.
In all three forms there is a syncretism between the neuter and the masculine in the dative case. The complete pronouns are syncretic. I set up a system that can derive the syncretism between the two genders. Doing this allows me to establish which features the morpheme \tit{mu} spells out.

I discussed in Chapter \ref{ch:decomposition} that syncretisms can be derived in Nanosyntax via the Superset Principle. The lexicon contains a lexical entry that is specified for the form that corresponds to most features. To illustrate this, I repeat the lexical entry for the Dutch \tit{jullie} `you' in \ref{ex:dutch-jullie-lexicon-rep}.

\ex.
\begin{forest} boom
  [\ac{dat}P
      [\tsc{k}3]
      [\ac{acc}P
          [\tsc{k}2]
          [\ac{nom}P
              [\ac{k}1]
              [2\ac{pl}P
                  [\phantom{xxx}, roof]
              ]
          ]
      ]
  ]
  {\draw (.east) node[right]{⇔ \tit{jullie}}; }
\end{forest}
\label{ex:dutch-jullie-lexicon-rep}

\tit{Jullie} is syncretic between nominative, accusative and dative. It is specified for dative in the lexicon, it is the most complex case of the three. The nominative, the accusative and the dative second person plural in Dutch are spelled out as \tit{jullie}, because the \tsc{dat}P, the \tsc{acc}P and the \tsc{nom}P are all contained in the lexical tree in \ref{ex:dutch-jullie-lexicon-rep} (Superset Principle), and there is no more specific lexical entry available in Dutch (Elsewhere Condition).
Importantly, the potentially unused features (so the \tsc{k}3 or \tsc{k}3 and the \tsc{k}2) are the top-most features of the lexical tree in \ref{ex:dutch-jullie-lexicon-rep}, so that the constituent that needs to be spelled out is still contained in the lexical tree.

In what follows, I show how I can derive the syncretisms for the forms in Table \ref{tbl:pol-datives}. I propose that \tit{jemu}, \tit{niemu} and \tit{temu} spell out the syntactic structure in \ref{ex:pol-mu-dat-fseq}.

\ex.\label{ex:pol-mu-dat-fseq}
\begin{forest} boom
  [\tsc{dat}P
      [\tsc{k}3]
      [\tsc{acc}P
          [\tsc{k}2]
          [\tsc{nom}P
            [\tsc{k}1]
            [\#P
                [\#]
                [\tsc{an}\ac{P}
                    [\tsc{an}]
                    [\tsc{cl}P
                        [\tsc{cl}]
                        [XP
                            [\phantom{xxx}, roof]
                        ]
                    ]
                ]
            ]
          ]
      ]
  ]
\end{forest}

I do not discuss the feature content that distinguishes \tit{jemu}, \tit{niemu} and \tit{temu}, but I call them XP.
Following the functional sequence I suggested in Chapter \ref{ch:deriving-onlyinternal}, all forms contain the feature \tsc{cl} for inanimate/neuter gender, \tsc{an} for animate/masculine gender and \# for singular number and case features up to the dative.

The forms \tit{jemu}, \tit{niemu} and \tit{temu} are syncretic between the masculine and the neuter. This can be captured if the highest feature in the lexical tree is the feature that distinguishes masculine and neuter gender.
This distinguishing feature is the feature \tsc{an} \citep{harley2002}, which is not the highest feature in \ref{ex:pol-mu-dat-fseq}. Fortunately, different from \tit{jullie}, \tit{jemu}, \tit{niemu} and \tit{temu} are (at least) bimorphemic: they contain morphemes \tit{je}, \tit{nie} or \tit{te} and the morpheme \tit{mu}. The highest feature of one of the two morphemes needs to be the feature \tsc{an}.
I suggest that this is the case for \tit{je}, \tit{nie} and \tit{te}, as shown in \ref{ex:pol-entry-nie/je/te}.

\ex. \label{ex:pol-entry-nie/je/te}
\begin{forest} boom
  [\tsc{an}P
      [\tsc{an}]
      [\tsc{cl}P
          [\tsc{cl}]
          [XP
              [\phantom{xxx}, roof]
          ]
      ]
  ]
  {\draw (.east) node[right]{⇔ \tit{je/nie/te}}; }
\end{forest}

This means that \tit{je}, \tit{nie} and \tit{te} spell out gender features and other features, which differ per form and I refer to here as XP.

Since the feature \tsc{an} is the topmost feature in the lexical tree, it can be the forms can easily spell both the structure with or without the feature \tsc{an}. This means that \tit{je}, \tit{nie} and \tit{te} correspond to the masculine or to the neuter form.

In \ref{ex:pol-spellout-nie/je/te-an}, I give the syntactic structure of a masculine form.

\ex.\label{ex:pol-spellout-nie/je/te-an}
\begin{forest} boom
  [\tsc{an}P,
  tikz={
  \node[label=below:\tit{je/nie/te},
  draw,circle,
  scale=0.8,
  fit to=tree]{};
  }
      [\tsc{an}]
      [\tsc{cl}P
          [\tsc{cl}]
          [XP
              [\phantom{xxx}, roof]
          ]
      ]
  ]
\end{forest}

The syntactic structure forms a constituent within the lexical tree in \ref{ex:pol-entry-nie/je/te}, and the structure can be spelled out as \tit{je/nie/te}.

In \ref{ex:pol-spellout-nie/je/te-cl}, I give the syntactic structure of a neuter form.

\ex.\label{ex:pol-spellout-nie/je/te-cl}
\begin{forest} boom
  [\tsc{cl}P,
  tikz={
  \node[label=below:\tit{je/nie/te},
  draw,circle,
  scale=0.8,
  fit to=tree]{};
  }
      [\tsc{cl}]
      [XP
          [\phantom{xxx}, roof]
      ]
  ]
\end{forest}

Again, the syntactic structure forms a constituent within the lexical tree in \ref{ex:pol-entry-nie/je/te}, and the structure can be spelled out as \tit{je/nie/te}.

This means that the lexical tree for the suffix \tit{mu} should contain all features in the functional sequence in \ref{ex:pol-mu-dat-fseq} that are not spelled out by \tit{je/nie/te} so far. These are the number feature and all case features up to the dative.

I give the lexical entry for \tit{mu} in \ref{ex:pol-entry-mu}.

\ex. \label{ex:pol-entry-mu}
\begin{forest} boom
  [\tsc{dat}P
      [\tsc{k}3]
      [\tsc{acc}P
          [\tsc{k}2]
          [\tsc{nom}P
              [\tsc{k}1]
              [\#P
                  [\#]
              ]
          ]
      ]
  ]
  {\draw (.east) node[right]{⇔ \tit{mu}}; }
\end{forest}

Notice here that \tit{mu} has a unary bottom, meaning that it has a single feature at the bottom of its structure (just as the morpheme \tit{e:l} in Khanty, see Chapter \ref{ch:decomposition} and Chapter \ref{ch:the-basic-idea}).
Therefore, it is inserted as the result of movement. It also means that the inserted phonological form follows the already present phonology and is spelled out as a suffix. This is how the correct order of \tit{je/nie/te} and \tit{mu} comes about. Later on in this section, I illustrate the movement operation and how it is a step of the Spellout Algorithm.

The morpheme \tit{go} is identical to the morpheme \tit{mu}, except for that it expresses accusative case instead of dative case. Therefore, the morpheme \tit{go} differs from \tit{mu} in that it lacks the feature \tsc{k}3 in its lexical entry, as shown in \ref{ex:pol-entry-go}.

\ex. \label{ex:pol-entry-go}
\begin{forest} boom
  [\tsc{acc}P
      [\tsc{k}2]
      [\tsc{nom}P
          [\tsc{k}1]
          [\#P
              [\#]
          ]
      ]
  ]
  {\draw (.east) node[right]{⇔ \tit{go}}; }
\end{forest}

In sum, the morphemes \tit{go} and \tit{mu} spell out case features and a number feature.

This leaves the two morphemes \tit{k} and \tit{o}. First I discuss the \tit{o}.
This morpheme corresponds to what I called the phi feature morpheme in Chapter \ref{ch:the-basic-idea} and the introduction to this chapter.
I show that this morpheme corresponds to pronominal features and gender features.

First I show that the \tit{o} does not only appear in animate relative pronouns, but also in the inanimate relative pronoun and even in other pronouns. I go through this rather long reasoning to show that \tit{o} is present in pronominal environments, even though it does not surface as an \tit{o}.
Consider the relative pronouns in Table \ref{tbl:pol-rps}.

\begin{table}[htbp]
  \center
  \caption{Polish (in)animate relative pronouns \pgcitep{swan2002}{160}}
  \begin{tabular}[b]{ccc}
    \toprule
              & \tsc{an}  & \tsc{inan} \\
    \cmidrule{2-3}
    \tsc{nom} & kto       & c-o        \\
    \tsc{acc} & k-o-go    & c-o        \\
    \tsc{gen} & k-o-go    & cz-e-go    \\
    \tsc{dat} & k-o-mu    & cz-e-mu    \\
    \tsc{ins} & k-i-m     & cz-y-m     \\
    \bottomrule
  \end{tabular}
  \label{tbl:pol-rps}
\end{table}

I ignore the nominative and accusative for now and come back to them later in this section.
In the genitive, dative and instrumental, the final suffixes in the animate and the inanimate paradigm are the same.\footnote{
I include genitive and the instrumental in the paradigms to show that the patterns observed in the dative are not standing on themselves. Instead, they are more generally attested in Polish, and they deserve an explanation.
In Polish, the genitive comes between the accusative and the dative, i.e. it is more complex than the accusative and less complex than the dative. However, I do not incorporate them in the syntactic structures.
This does not change anything about the main point about case I want to make: the dative is more complex than the accusative.
}
The forms differ in their initial consonant and the vowel. The animates have a \tit{k} and an \tit{o} or \tit{i}, and the inanimates have a \tit{cz} and an \tit{e} or \tit{y}.

There are several ways to analyze this.
The first possibility is to not decompose the portion before the suffix. Under this analysis, Polish has the morphemes \tit{ko}, \tit{ki}, \tit{cze} and \tit{czy}. The point that is missed is that the animates always have a \tit{k} and inanimates always have a \tit{cz}.

A second possibility that captures this observation is an analysis in which Polish has the morphemes \tit{k}, \tit{o}, \tit{i} and \tit{cz}, \tit{e} and \tit{y}.\footnote{
This is more or less what \citet{wiland2019} proposes.
}
What is not captured now is that numerous \tsc{wh}-elements in Polish start with a \tit{k}. I give some examples in \ref{ex:pol-wh-k}.\footnote{
The \tit{k} in \ref{ex:pol-gdzie} gets voiced into \tit{g} because it is followed by \tit{d}.
}

\ex.\label{ex:pol-wh-k}
\ag. k-tóry\\
 which\\
\bg. k-iedy\\
 when\\
\bg. g-dzie\\
 where\\\label{ex:pol-gdzie}
 \flushfill{Polish, \pgcitealt{swan2002}{180,183,184}}

Moreover, according to \pgcitet{swan2002}{23} \tit{cz} is not a primary consonant in Polish but a derived one, and the consonants \tit{cz} and \tit{c} are derived from \tit{k}. In other words, the \tit{k} goes to \tit{c} in some morphophonological environments, and the \tit{k} goes to \tit{cz} in other morphophonological environments.

I propose that one of the environments in which \tit{k} goes to \tit{c} and \tit{cz} is the inanimate relative pronouns.
I suggest that the morphophonological environment that turns \tit{k} into \tit{c} and \tit{cz} is the presence of an \tit{i} in the inanimate relative pronoun paradigm.\footnote{
Phonologically, the \tit{k} corresponds to /k/, the \tit{c} corresponds to /ts/ and the \tit{cz} corresponds to the /tʂ/.
} The presence of \tit{i} also causes other phonological changes to take place.
In the animate relative pronoun these phonological changes do not take place, since there the morpheme \tit{i} is not present.\footnote{
As first sight, there seems to be a contradiction here: the inanimate is featurally speaking less complex than the animate  \citep[cf.][]{harley2002}, but morphologically the inanimate is more complex than the animate: it contains the additional morpheme \tit{i}. I return to this point in footnote \ref{ftn:inam-pointer} of this chapter to show how this apparent contradiction can be resolved.
}

In Table \ref{tbl:pol-rp-underl}, I show the animate relative pronouns on the left and what I propose to be the underlying forms of the inanimate relative pronouns on the right.\footnote{
I do not decompose the animate nominative relative pronoun (unlike \citealt{wiland2019}, who identifies the \tit{t} as the demonstrative stem). My reason for not decomposing \tit{kto} is that it does not contain a suffix that can be observed elsewhere in the language (unlike the other cases which can be found in for instance adjectival inflection, see \citealt{swan2002}{126}). Therefore, I assume \tit{kto} to be a fixed expression. I do not give a detailed analysis of the pronoun, as it is not a form I discuss in my derivations.
}

\begin{table}[htbp]
  \center
  \caption{Underlying forms of Polish (in)animate relative pronouns}
  \begin{tabular}[b]{ccc}
    \toprule
              & \tsc{an}  & \tsc{inam}  \\
    \cmidrule{2-3}
    \tsc{nom} & kto       & k-i-o       \\
    \tsc{acc} & k-o-go    & k-i-o       \\
    \tsc{gen} & k-o-go    & k-i-o-go    \\
    \tsc{dat} & k-o-mu    & k-i-o-mu    \\
    \tsc{ins} & k-i-m     & k-i-i-m     \\
    \bottomrule
  \end{tabular}
  \label{tbl:pol-rp-underl}
\end{table}

Under this analysis, Polish only has the morphemes \tit{k}, \tit{o} and \tit{i} that can be observed in the animate plus an \tit{i} that is present throughout the whole paradigm in the inanimate.
I put the underlying forms and the surface form of the inanminate relative pronoun side by side in Table \ref{tbl:pol-rps-underl-real}.

\begin{table}[htbp]
  \center
  \caption{Underlying and surface forms of Polish inanimate relative pronouns}
  \begin{tabular}[b]{ccc}
    \toprule
              & underlying  & surface    \\
    \cmidrule{2-3}
    \tsc{nom} & k-i-o       &  c-o      \\
    \tsc{acc} & k-i-o       &  c-o      \\
    \tsc{gen} & k-i-o-go    &  cze-go  \\
    \tsc{dat} & k-i-o-mu    &  cze-mu  \\
    \tsc{ins} & k-i-i-m     &  czy-m   \\
    \bottomrule
  \end{tabular}
  \label{tbl:pol-rps-underl-real}
\end{table}

The sequence \tit{k-i-i} becomes \tit{czy} in the instrumental, and the sequence \tit{k-i-o} becomes \tit{cze} in the genitive and dative.\footnote{
Under this analysis, the \tsc{wh}-element \tit{czyj} `whose' is underlyingly \tit{k-i-i-j}.
}
To get from the underlying form to the surface form, several phonological processes are taking place, which are all independently observed within Polish. First I discuss the instrumental, in which \tit{k-i-i} becomes \tit{czy}.

I start with the combination of /k/ and /i/ becoming /i/, as shown in \ref{ex:pol-phon-k+i}.

\ex.\label{ex:pol-phon-k+i}
/k/ + /i/ → /c/

Consider the paradigm for the singular of \tit{lampa} `light' and the singular of \tit{córka} `daugther' in Table \ref{tbl:pol-jk-to-c}.

\begin{table}[htbp]
  \center
  \caption{Polish feminine nouns with hard p/k stem \pgcitep{swan2002}{47,49}}
  \begin{tabular}[b]{ccc}
    \toprule
          & light.\tsc{sg} & daughter.\tsc{sg} \\
            \cmidrule{2-3}
\tsc{nom} & lamp-a         & córk-a            \\
\tsc{acc} & lamp-ę         & córk-ę            \\
\tsc{gen} & lamp-y         & córk-i            \\
\tsc{dat} & lamp-i-e       & córc-e            \\
\tsc{ins} & lamp-ą         & córk-ą            \\
  \bottomrule
  \end{tabular}
\label{tbl:pol-jk-to-c}
\end{table}

The stem and the suffixes are identical in both paradigms, except for in the dative.\footnote{
There is also the change from /i/ to /y/ in the genitive of \tit{lampa}.
}
There, the stem of \tit{córka} does no longer end with a \tit{k}, but with a \tit{c}. Also, part of the suffix, the \tit{i}, has disappeared. Analyzing \tit{córce} as \tit{córk-i-e} brings back regularity in the paradigm. Assuming that this change also takes place in the inanimate relative pronoun, the result of this change is \tit{c-i-m}.

I continue with the combination of \tit{c} and \tit{i} becoming \tit{czy}, as shown in \ref{ex:pol-phon-c+i}.

\ex.\label{ex:pol-phon-c+i}
\a. /c/ + /i/ → /czy/

This change can be independently observed in \ref{ex:pol-waltz}.

\exg. walc-ik: walczyk\\
waltz-\tsc{dim} waltz.\tsc{dim}\\
\flushfill{\pgcitealt{swan2002}{26}}\label{ex:pol-waltz}

The noun \tit{walc} `waltz' combines with the diminutive marker \tit{ik}. The sequence \tit{c-ik} changes to \tit{czyk}. Assuming that this change also takes place in the inanimate relative pronoun, the result of this change is \tit{czy-m}.

I repeat the table with the underlying and surface forms in Table \ref{tbl:pol-rps-underl-real-rep}. Compare the nominative and the accusative to the genitive and the dative.

\begin{table}[htbp]
  \center
  \caption{Underlying and surface forms of Polish inanimate relative pronouns}
  \begin{tabular}[b]{ccc}
    \toprule
              & underlying  & surface    \\
    \cmidrule{2-3}
    \tsc{nom} & k-i-o       &  c-o      \\
    \tsc{acc} & k-i-o       &  c-o      \\
    \tsc{gen} & k-i-o-go    &  cze-go  \\
    \tsc{dat} & k-i-o-mu    &  cze-mu  \\
    \tsc{ins} & k-i-i-m     &  czy-m   \\
    \bottomrule
  \end{tabular}
  \label{tbl:pol-rps-underl-real-rep}
\end{table}

The sequence \tit{k-i-o} changes into \tit{co} in the nominative and accusative. The same \tit{k-i-o} sequence turns into \tit{cze} in the genitive and dative. This raises the question of how two identical sequences can lead to two different outcomes.

I start by looking at the nominative and the accusative. Here only one phonological change seems to take place, which is the combination of \tit{k} and \tit{i} becoming \tit{c}, which I also showed for the instrumental. I repeat it in \ref{ex:pol-phon-k+i-rep}.

\ex.\label{ex:pol-phon-k+i-rep}
/k/ + /i/ → /c/

Assuming that this change also takes place in the inanimate relative pronoun, the result of this change is \tit{co}.

Now consider the genitive and the dative. If here only the \tit{k} plus \tit{i} becomes \tit{c} change takes place, the result would \tit{c-o-go}/\tit{c-o-mu}, which is incorrect. The sequence \tit{c-o} should somehow still change into \tit{cze}. Now assume that in the genitive and the dative the \tit{i} would have `double effect', meaning that it also has an influence on the vowel \tit{o}. Before I discuss how the \tit{i} can sometimes have a `double effect' and sometimes `single effect', I show that the required phonological processes are attested in Polish.

I start with the combination of \tit{i} and \tit{o} becoming \tit{e}, as shown in \ref{ex:pol-phon-i+o}.

\ex.\label{ex:pol-phon-i+o}
/i/ + /o/ → /e/

Consider the paradigm for the singular of \tit{biurko} `desk' and the singular of \tit{słońce} `sun' in Table \ref{tbl:pol-io-to-e}.

\begin{table}[htbp]
  \center
  \caption{Polish neuter nouns with hard k/c stem \pgcitep{swan2002}{116,117}}
  \begin{tabular}[b]{ccc}
    \toprule
          & desk.\tsc{sg} & sun.\tsc{sg} \\
            \cmidrule{2-3}
\tsc{nom} & biurk-o       & słońc-e      \\
\tsc{acc} & biurk-o       & słońc-e      \\
\tsc{gen} & biurk-a       & słońc-a      \\
\tsc{dat} & biurk-u       & słońc-u      \\
  \bottomrule
  \end{tabular}
\label{tbl:pol-io-to-e}
\end{table}

The stem and the suffixes are identical in both paradigms, except for in the nominative and the accusative.
There, the suffix is \tit{o} on the stem \tit{biurk}, and the suffix is \tit{e} on the stem \tit{słońc}. Remember that the \tit{c} is not a primary consonant in Polish \pgcitep{swan2002}{23} and that it is possibly derived from a combination of \tit{k} and \tit{i} (see \ref{ex:pol-phon-k+i-rep}).
Analyzing \tit{słońc-e} as \tit{słońk-i-o} brings back regularity in the paradigm.\footnote{
Under this analysis, \tit{słońc-a} and \tit{słońc-u} are underlyingly \tit{słońk-i-a} and \tit{słońk-i-u}. In these two forms, the \tit{i} would need to change the \tit{k} into a \tit{c} but not affect the vowel.
}
Assuming that this change also takes place in the inanimate relative pronoun, the result of this change is \tit{c-e-go}/\tit{c-e-mu}.

I continue with the change in which the combination of \tit{c} and \tit{e} become \tit{cze}, as shown in \ref{ex:pol-phon-ce}.

\ex.\label{ex:pol-phon-ce}
/c/ + /e/ → /cze/

I give the example in which the combination of \tit{c} and \tit{e} results in \tit{cze} in \ref{ex:pol-father}.

\exg. ojc-e: ojcze\\
 father-\tsc{voc} father.\tsc{voc}\\
\flushfill{\pgcitealt{swan2002}{26}}\label{ex:pol-father}

The noun \tit{ojc} `father' combines with the vocative marker \tit{e}. The sequence \tit{c-e} changes to \tit{cze}. Assuming that this change also takes place in the inanimate relative pronoun, the result of this change is \tit{cze-go}/\tit{cze-mu}.

It is crucial for the analysis that the \tit{i} in \tit{co} has a single effect, and it only affects the initial consonant. In the genitive and the dative there is a double effect, and both the initial consonant and the vowel are affected. I propose that this difference is not due to a difference in the \tit{i}, but a difference in the \tit{o}. Morphologically, the \tit{o} in the genitive and dative spells out different features than the \tit{o} in the nominative and the accusative, such as nominative and accusative case, which in the case of the genitive and dative are realized by \tit{go} and \tit{mu}. As the two forms are morphologically different, they have separate lexical entries, and they can correspond to different phonology. The phonological representation of the \tit{o} in the nominative and accusative should be such that it resists the effect of the \tit{i}, and the \tit{o} in the genitive and dative should welcome the effect of the \tit{i}. I do not work out a phonological proposal for how this works.\footnote{
Maybe it is possible to analyze the \tit{o} in the nominative and accusative as an \tit{o} without a slot for a vowel and the \tit{o} in the genitive and dative as an \tit{o} plus a slot for a vowel.
}

The conclusion I draw from this rather long reasoning is that the morpheme \tit{o} in \tit{kogo} and \tit{komu} is not specific to animate relative pronouns, but it also appears elsewhere, for instance in inanimate relative pronouns, demonstratives and in the \tit{n}- and \tit{je}-pronouns, as shown in Table \ref{tbl:pol-o-everywhere}.\footnote{
In this dissertation I do not discuss the exact feature content that corresponds to \tit{i}. I assume it spells out features that have to do with being strong pronouns. See footnote \ref{ftn:inam-pointer} of this chapter for why the morpheme is not inserted in animate relative pronouns.
}

\begin{table}[htbp]
  \center
  \caption{Underlying and surface forms of Polish dative pronouns}
  \begin{tabular}[b]{ccc}
    \toprule
                      & underlying  & surface  \\
    \cmidrule{2-3}
    \tsc{dem}         & t-i-o-mu    &  te-mu   \\
    \tit{nie}-pronoun & n-i-o-mu    &  nie-mu \\
    \tit{je}-pronoun  & i-o-mu      &  je-mu   \\
    \bottomrule
  \end{tabular}
  \label{tbl:pol-o-everywhere}
\end{table}

Under my analysis, the \tit{o} is present in all these pronouns, although it does not appear on the surface. The \tit{o} combines with other morphemes like the \tit{i} (in all other pronouns), the \tit{t} in demonstratives and the \tit{n} in pronouns that combine with prepositions.

What these elements (i.e. relative pronoun, demonstratives and third singular pronouns) all have in common is that they are pronouns. Moreover, they can all appear in both animate and inanimate gender. Therefore, I assume that the \tit{o} spells out pronominal features and gender features. I give the lexical entry in \ref{ex:pol-entry-o}.

\ex. \label{ex:pol-entry-o}
\begin{forest} boom
  [\tsc{an}P
      [\tsc{an}]
      [\tsc{cl}P
          [\tsc{cl}]
          [\tsc{ref}]
      ]
  ]
  {\draw (.east) node[right]{⇔ \tit{o}}; }
\end{forest}

Finally, I discuss the morpheme \tit{k}. This morpheme corresponds to what I called the \tsc{rel}-feature in Chapter \ref{ch:the-basic-idea} and in the introduction to this chapter. I argue that this morpheme actually spells out the operator features \tsc{wh} and \tsc{rel} and number and gender features.

Just as in Modern German, number and gender features are already spelled out once, this time by the suffixes \tit{go} and \tit{mu}. Again I assume that they are spelled out twice within the relative pronoun. They are semantically present twice, but their double presence is purely due to spellout reasons.

I start with the operator features \tsc{wh} and \tsc{rel}. The Polish relative pronouns are \tsc{wh}-pronouns, and they are also used as interrogatives. Therefore, just as the Modern German \tit{we}, the Polish \tit{k} spells out the features \tsc{wh} and \tsc{rel}.

Finally, since the relative pronouns do not have a morphological plural, I assume that \tit{k} contains the feature \#.
Lastly, I assume that \tit{k} also contains the features \tsc{an} and \tsc{cl}. For this I do not have any independent support. \footnote{\label{ftn:inam-pointer}
I make this assumption to make room for the \tit{i} to be inserted in the inanimate.
To be able to derive the inanimate relative pronoun, I also assume that there is a pointer in the lexical entry for \tit{k}, as shown in \ref{ex:pol-entry-k-pointer} (see \citealt{wyngaerd2018} which illustrates the use of pointers).

\ex.\label{ex:pol-entry-k-pointer}
\begin{forest} boom
  [\tsc{rel}P
      [\tsc{rel}]
      [\tsc{wh}P
          [\tsc{wh}]
          [\#P, edge=->
              [\#]
              [\tsc{an}P
                  [\tsc{an}]
                  [\tsc{cl}]
              ]
          ]
      ]
  ]
  {\draw (.east) node[right]{⇔ \tit{k}}; }
\end{forest}

The pointer is situated above the \#P. That means that if there is no animate feature in the structure, the \# can also not be spelled out with \tit{k}. Then there is another morpheme necessary that contributes the feature \#. I propose that this is \tit{i}, which causes the phonological processes described in this section.
}

In sum, the morpheme \tit{k} realizes the features \tsc{wh}, \tsc{rel}, \#, \tsc{an} and \tsc{cl}.

\ex.\label{ex:pol-entry-k}
\begin{forest} boom
  [\tsc{rel}P
      [\tsc{rel}]
      [\tsc{wh}P
          [\tsc{wh}]
          [\#P
              [\#]
              [\tsc{an}P
                  [\tsc{an}]
                  [\tsc{cl}]
              ]
          ]
      ]
  ]
  {\draw (.east) node[right]{⇔ \tit{k}}; }
\end{forest}

In what follows, I show how the Polish relative pronouns are constructed. I follow the same functional sequence as I did for Modern German. Also, of course, the spellout procedure is identical. The outcome is different because of the different lexical entries Polish has. I repeat the available lexical entries in \ref{ex:pol-entries-all-rp}.

\ex.\label{ex:pol-entries-all-rp}
\a.\label{ex:pol-entry-o-rep-rp}
\begin{forest} boom
  [\tsc{an}P
      [\tsc{an}]
      [\tsc{cl}P
          [\tsc{cl}]
          [\tsc{ref}]
      ]
  ]
  {\draw (.east) node[right]{⇔ \tit{o}}; }
\end{forest}
\b. \label{ex:pol-entry-go-rep-rp}
\begin{forest} boom
  [\tsc{acc}P
      [\tsc{k}2]
      [\tsc{nom}P
          [\tsc{k}1]
          [\#P
              [\#]
          ]
      ]
  ]
  {\draw (.east) node[right]{⇔ \tit{go}}; }
\end{forest}
\b. \label{ex:pol-entry-mu-rep-rp}
\begin{forest} boom
  [\tsc{dat}P
      [\tsc{k}3]
      [\tsc{acc}P
          [\tsc{k}2]
          [\tsc{nom}P
              [\tsc{k}1]
              [\#P
                  [\#]
              ]
          ]
      ]
  ]
  {\draw (.east) node[right]{⇔ \tit{mu}}; }
\end{forest}
\b. \label{ex:pol-entry-k-rep-rp}
\begin{forest} boom
  [\tsc{rel}P
      [\tsc{rel}]
      [\tsc{wh}P
          [\tsc{wh}]
          [\#P
              [\#]
              [\tsc{an}P
                  [\tsc{an}]
                  [\tsc{cl}]
              ]
          ]
      ]
  ]
  {\draw (.east) node[right]{⇔ \tit{k}}; }
\end{forest}

Starting the derivation from the bottom, the first two features that are merged are \tsc{ref} and \tsc{cl}, creating a \tsc{cl}P.
The syntactic structure forms a constituent in the lexical tree in \ref{ex:pol-entry-o-rep-rp}, which corresponds to \tit{o}.
Therefore, the \tsc{cl}P is spelled out as \tit{o}, which I do not show here.
Then, the feature \tsc{an} is merged, and a \tsc{an}P is created.
The syntactic structure forms a constituent in the lexical tree in \ref{ex:pol-entry-o-rep-rp}.
Therefore, the \tsc{an}P is spelled out as \tit{o}, shown in \ref{ex:pol-spellout-o-rel}.

\ex.\label{ex:pol-spellout-o-rel}
\begin{forest} boom
  [\tsc{an}P,
  tikz={
  \node[label=below:\tit{o},
  draw,circle,
  scale=0.9,
  fit to=tree]{};
  }
      [\tsc{an}]
      [\tsc{cl}P
          [\tsc{cl}]
          [\tsc{ref}]
      ]
  ]
\end{forest}

The next feature in the functional sequence is the feature \#. It is merged, and an \#P is created. This syntactic structure does not form a constituent in the lexical tree in \ref{ex:pol-entry-o-rep-rp}. There is no other lexical tree that contains the syntactic structure as a constituent.
Therefore, there is no successfull spellout for the syntactic structure in the derivational step in which the structure is spelled out as a single phrase (\ref{ex:spellout-algorithm-phrasal-rep} in the Spellout Algorithm, repeated from Chapter \ref{ch:deriving-onlyinternal}).

\ex. \tbf{Spellout Algorithm} (as in \citealt{caha2021}, based on \citealt{starke2018})\label{ex:spellout-algorithm-rep}
 \a. Merge F and spell out.\label{ex:spellout-algorithm-phrasal-rep}
 \b. If (a) fails, move the Spec of the complement and spell out.\label{ex:spellout-algorithm-spec-rep}
 \b. If (b) fails, move the complement of F and spell out.\label{ex:spellout-algorithm-comp-rep}

The first movement option in the Spellout Algorithm is moving the specifier, as described in \ref{ex:spellout-algorithm-spec-rep}. As there is no specifier in this structure, the first movement option is irrelevant.
The second movement option in the Spellout Algorithm is moving the complement, as described in \ref{ex:spellout-algorithm-comp-rep}. In this case, the complement of \#, the \tsc{an}P, is moved to the specifier of \#P. I show this movement in \ref{ex:pol-movement}.\footnote{
In its landing position the internal structure of the \tsc{an}P is no longer shown (to save some space), and its phonological form is placed under the triangle. The strikethrough of the lower \tsc{an}P indicates that the complement of \# disappears.
}

\ex.\label{ex:pol-movement}
\begin{forest} boom
  [\#P
      [\tsc{an}P,name=tgt
          [\phantom{x}\tit{o}\phantom{x}, roof]
      ]
      [\#P
          [\#]
          [\sout{\tsc{an}P}, name=src,
          tikz={
          \node[label=below:\tit{o},
          draw,circle,
          scale=0.9,
          fit to=tree]{};
          }
              [\tsc{an}]
              [\tsc{cl}P
                  [\tsc{cl}]
                  [\tsc{ref}]
              ]
          ]
      ]
  ]
\draw[->,dashed] (src) to[out=south west,in=east] (tgt);
\end{forest}

The \#P has a different internal syntax now. It still contains the feature \#, but the \tsc{an}P is no longer a sister of \#. Now the \tsc{an}P is moved away, the \#P forms a constituent in the lexical tree of \ref{ex:pol-entry-go-rep-rp}.
Therefore, the \#P is spelled out as \tit{go}, as shown in \ref{ex:pol-spellout-o-ind}.

\ex.\label{ex:pol-spellout-o-ind}
\begin{forest} boom
  [\#P
  [\tsc{an}P
      [\phantom{x}\tit{o}\phantom{x}, roof]
  ]
      [\#P,
      tikz={
      \node[label=below:\tit{go},
      draw,circle,
      scale=0.95,
      fit to=tree]{};
      }
          [\#]
      ]
  ]
\end{forest}

Next, the feature \tsc{wh} is merged.
The derivation for this feature resembles the derivation of \tsc{wh} in Modern German.
The feature is merged with the existing syntactic structure, creating a \tsc{wh}P.
This structure does not form a constituent in any of the lexical trees in the language's lexicon, and neither of the spellout driven movements leads to a successful spellout.
Therefore, in a second workspace, the feature \tsc{wh} is merged with the feature \# (the previous syntactic feature on the functional sequence) into a \tsc{wh}P. This syntactic structure does not form a constituent in any of the lexical trees in the language's lexicon.
Therefore, the feature \tsc{wh} combines not only with the feature merged before it, but with a phrase that consists of the two features merged before it: \# and \tsc{an}. Also this syntactic structure does not form a constituent in any of the lexical trees in the language's lexicon.
Therefore, the feature \tsc{wh} combines with a phrase that consists of the three features merged before it: \#, \tsc{an} and \tsc{cl}. This syntactic structure forms a constituent in the lexical tree in \ref{ex:pol-entry-k-rep-rp}, which corresponds to the \tit{k}.
Therefore, the \tsc{wh}P is spelled out as \tit{k}. The newly created phrase is merged as a whole with the already existing structure, and projects to the top node, as shown in \ref{ex:pol-spellout-whp}.

\ex.\label{ex:pol-spellout-whp}
\begin{adjustbox}{max width=0.9\textwidth}
\begin{forest} boom
  [\tsc{wh}P, s sep=25mm
      [\tsc{wh}P,
      tikz={
      \node[label=below:\tit{k},
      draw,circle,
      scale=0.95,
      fit to=tree]{};
      }
          [\tsc{wh}]
          [\#P
              [\#]
              [\tsc{an}P
                  [\tsc{an}]
                  [\tsc{cl}]
              ]
          ]
      ]
      [\#P, s sep=25mm
      [\tsc{an}P,
          tikz={
          \node[label=below:\tit{o},
          draw,circle,
          scale=0.95,
          fit to=tree]{};
          }
          [\tsc{an}P]
          [\tsc{cl}P
              [\tsc{cl}]
              [\tsc{ref}]
          ]
      ]
          [\#P,
          tikz={
          \node[label=below:\tit{go},
          draw,circle,
          scale=0.9,
          fit to=tree]{};
          }
              [\#]
          ]
      ]
  ]
\end{forest}
\end{adjustbox}

The next feature in the functional sequence is the feature \tsc{rel}. The derivation for this feature resembles the derivation of \tsc{rel} in Modern German.
The feature is merged with the existing syntactic structure, creating a \tsc{rel}P.
This structure does not form a constituent in any of the lexical trees in the language's lexicon, and neither of the spellout driven movements leads to a successful spellout.
Backtracking leads to splitting up the \tsc{wh}P from the \#P.
The feature \tsc{rel} is merged in both workspaces, so with \tsc{wh}P and and with \#P. The spellout of \tsc{rel} is successful when it is combined with the \tsc{wh}P.
It forms a constituent in the lexical tree in \ref{ex:pol-entry-k-rep-rp}, which corresponds to the \tit{k}.
The \tsc{rel}P is spelled out as \tit{k}, and it is merged back to the existing syntactic structure.

The next feature on the functional sequence is \tsc{k}1. This feature should somehow end up merging with \#P, because it forms a constituent in the lexical tree in \ref{ex:pol-entry-go-rep-rp}, which corresponds to the \tit{go}.
This is achieved via Backtracking in which phrases are split up and going through the Spellout Algorithm. I go through the derivation step by step.
The feature \tsc{k}1 is merged with the existing syntactic structure, creating a \tsc{nom}P.
This structure does not form a constituent in any of the lexical trees in the language's lexicon, and neither of the spellout driven movements leads to a successful spellout.
Backtracking leads to splitting up the \tsc{rel}P from the \#P.
The feature \tsc{k}1 is merged in both workspaces, so with the \tsc{rel}P and and with the \#P. None of these phrases form a constituent in any of the lexical trees in the language's lexicon.
The first movement option in the Spellout Algorithm is moving the specifier. In the \tsc{rel}P there is no specifier, so this movement option is irrelevant. In the \#P, however, there is a specifier, which is moved to the specifier of \tsc{nom}P.
This syntactic structure forms a constituent in the lexical tree in \ref{ex:pol-entry-go-rep-rp}, which corresponds to the \tit{go}.
The \tsc{nom}P is spelled out as \tit{go}, and the \tsc{nom}P is merged back to the existing syntactic structure.

For the accusative relative pronoun, the last feature on the functional sequence is the feature \tsc{k}2. Its derivation proceeds the same as the one for the feature \tsc{k}1.
The feature \tsc{k}2 is merged with the existing syntactic structure, creating a \tsc{acc}P.
This structure does not form a constituent in any of the lexical trees in the language's lexicon, and neither of the spellout driven movements leads to a successful spellout.
Backtracking leads to splitting up the \tsc{rel}P from the \tsc{nom}P.
The feature \tsc{k}2 is merged in both workspaces, so with the \tsc{rel}P and and with the \tsc{nom}P. None of these phrases form a constituent in any of the lexical trees in the language's lexicon.
The first movement option in the Spellout Algorithm is moving the specifier. In the \tsc{rel}P there is no specifier, so this movement option is irrelevant. In the \tsc{nom}P, however, there is a specifier, which is moved to the specifier of \tsc{acc}P.
This syntactic structure forms a constituent in the lexical tree in \ref{ex:pol-entry-go-rep-rp}, which corresponds to the \tit{go}.
The \tsc{acc}P is spelled out as \tit{go}, and the \tsc{acc}P is merged back to the existing syntactic structure, as shown in \ref{ex:pol-spellout-rel-acc}.

\ex.\label{ex:pol-spellout-rel-acc}
\begin{adjustbox}{max width=0.9\textwidth}
\begin{forest} boom
  [\tsc{rel}P, s sep=35mm
      [\tsc{rel}P,
      tikz={
      \node[label=below:\tit{k},
      draw,circle,
      scale=0.95,
      fit to=tree]{};
      }
          [\tsc{rel}]
          [\tsc{wh}P
              [\tsc{wh}]
              [\#P
                  [\#]
                  [\tsc{an}P
                      [\tsc{an}]
                      [\tsc{cl}]
                  ]
              ]
          ]
      ]
      [\tsc{acc}P, s sep=30mm
      [\tsc{an}P,
          tikz={
          \node[label=below:\tit{o},
          draw,circle,
          scale=0.95,
          fit to=tree]{};
          }
          [\tsc{an}P]
          [\tsc{cl}P
              [\tsc{cl}]
              [\tsc{ref}]
          ]
      ]
          [\tsc{acc}P,
          tikz={
          \node[label=below:\tit{go},
          draw,circle,
          scale=0.95,
          fit to=tree]{};
          }
              [\tsc{k}2]
              [\tsc{nom}P
                  [\tsc{k}1]
                  [\#P
                      [\#]
                  ]
              ]
          ]
      ]
  ]
\end{forest}
\end{adjustbox}

For the dative relative pronoun, the last feature on the functional sequence is the feature \tsc{k}3. Its derivation proceeds the same as the one for the feature \tsc{k}2.
The feature \tsc{k}3 is merged with the existing syntactic structure, creating a \tsc{dat}P.
This structure does not form a constituent in any of the lexical trees in the language's lexicon, and neither of the spellout driven movements leads to a successful spellout.
Backtracking leads to splitting up the \tsc{rel}P from the \tsc{acc}P.
The feature \tsc{k}3 is merged in both workspaces, so with the \tsc{rel}P and and with the \tsc{acc}P. None of these phrases form a constituent in any of the lexical trees in the language's lexicon.
The first movement option in the Spellout Algorithm is moving the specifier. In the \tsc{rel}P there is no specifier, so this movement option is irrelevant. In the \tsc{acc}P, however, there is a specifier, which is moved to the specifier of \tsc{dat}P.
This syntactic structure forms a constituent in the lexical tree in \ref{ex:pol-entry-mu-rep-rp}, which corresponds to the \tit{mu}.
The \tsc{dat}P is spelled out as \tit{mu}, and the \tsc{dat}P is merged back to the existing syntactic structure, as shown in \ref{ex:pol-spellout-rel-dat}.

\ex.\label{ex:pol-spellout-rel-dat}
\begin{adjustbox}{max width=0.9\textwidth}
\begin{forest} boom
  [\tsc{rel}P, s sep=35mm
      [\tsc{rel}P,
      tikz={
      \node[label=below:\tit{k},
      draw,circle,
      scale=0.95,
      fit to=tree]{};
      }
          [\tsc{rel}]
          [\tsc{wh}P
              [\tsc{wh}]
              [\#P
                  [\#]
                  [\tsc{an}P
                      [\tsc{an}]
                      [\tsc{cl}]
                  ]
              ]
          ]
      ]
      [\tsc{dat}P, s sep=30mm
      [\tsc{an}P,
          tikz={
          \node[label=below:\tit{o},
          draw,circle,
          scale=0.95,
          fit to=tree]{};
          }
          [\tsc{an}P]
          [\tsc{cl}P
              [\tsc{cl}]
              [\tsc{ref}]
          ]
      ]
          [\tsc{dat}P,
          tikz={
          \node[label=below:\tit{mu},
          draw,circle,
          scale=0.95,
          fit to=tree]{};
          }
              [\tsc{k}3]
              [\tsc{acc}P
                  [\tsc{k}2]
                  [\tsc{nom}P
                      [\tsc{k}1]
                      [\#P
                          [\#]
                      ]
                  ]
              ]
          ]
      ]
  ]
\end{forest}
\end{adjustbox}

To summarize, I decomposed the relative pronoun into three morphemes: \tit{k}, \tit{o} and the suffix (\tit{go} and \tit{mu}). I showed which features each of the morphemes spells out and what the internal syntax looks like that they are combined into. It is this internal syntax that determines whether the light head can be deleted or not.


\section{The Polish extra light head}\label{sec:pol-elh}

I have suggested that headless relatives are derived from light-headed relatives. The light head or the relative pronoun can be deleted when either of them is contained in the other one.
In Chapter \ref{ch:the-basic-idea}, I mentioned that languages have two possible light heads. I also noted that headless relatives in Polish (just like the ones in Modern German) can only be derived from light-headed relatives that are headed by one of these heads. In this section I give arguments that exclude the second light head as a possible light head.

For Modern German, I discussed both possible light heads. I started by discussing a light-headed relative that is attested in Modern German. If the headless relative was derived from this light-headed relative, the deletion would have to be optional. I considered this scenario, and I gave two arguments against it.
Then I took the light head from the existing light-headed relative as a point of departure, and I modified it in such a way that it is appropriate as a light head for a headless relative in Modern German. I argued that this light head is the head of the light-headed relative that Modern German headless relative are derived from. This light-headed relative does not exist in Modern German, so the deletion of the light head is obligatory.
I do the same investigation for Polish, and I reach the same conclusion as I did for Modern German.

In the introduction of this chapter, I claimed that the internal syntax of light heads in Polish looks as shown in \ref{ex:pol-lh-complex}.

\ex.\label{ex:pol-lh-complex}
\begin{forest} boom
  [\tsc{k}P, s sep = 17.5 mm
      [ϕP,
      tikz={
      \node[draw,circle,
      scale=0.85,
      fit to=tree]{};
      }
          [\phantom{xxx}, roof]
      ]
      [\tsc{k}P,
      tikz={
      \node[draw,circle,
      scale=0.85,
      fit to=tree]{};
      }
          [\tsc{k}]
      ]
  ]
\end{forest}

In this section, I determine the exact feature content of the light head.
As I suggested in Chapter \ref{ch:deriving-onlyinternal} for Modern German, I end up claiming that the phi and case features morpheme of the relative pronoun is the light head in headless relatives. I show the complete structure that I work towards in this section in \ref{ex:pol-elh}.

% Crucially, the constituent structure in \ref{ex:pol-elh} is the same as in \ref{ex:simple-matching}. Recall from Chapter \ref{ch:deriving-onlyinternal} that in Modern German the extra light head spells out as a single constituent. The Polish extra light head consists of two constituents, as shown in \ref{ex:pol-elh}. This is the crucial difference between the two languages that leads them to be of different types in headless relatives.

\ex.\label{ex:pol-elh}
\begin{adjustbox}{max width=0.9\textwidth}
\begin{forest} boom
  [\tsc{k}P, s sep=25mm
      [\tsc{an}P,
      tikz={
      \node[label=below:\tit{o},
      draw,circle,
      scale=0.95,
      fit to=tree]{};
      }
          [\tsc{an}]
          [\tsc{cl}P
              [\tsc{cl}]
              [\tsc{ref}]
          ]
      ]
      [\tsc{k}P,
      tikz={
      \node[label=below:\tit{go/mu},
      draw,circle,
      scale=0.9,
      fit to=tree]{};
      }
          [\tsc{k}]
          [\#P
              [\#]
          ]
      ]
  ]
\end{forest}
\end{adjustbox}

Consider the existing Polish light-headed relative in \ref{ex:pol-light-headed}.

\exg. Jan śpiewa to, co Maria śpiewa.\\
Jan sings \tsc{dem}.\tsc{m}.\tsc{sg}.\tsc{acc} \tsc{rp}.\tsc{an}.\tsc{acc} Maria sings\\
`John sings what Mary sings.' \flushfill{Polish, \pgcitealt{citko2004}{103}}\label{ex:pol-light-headed}

This light-headed relative, headed by the demonstrative, could potentially be the source of headless relatives.

For Modern German, I gave two arguments for not taking this existing light-headed relative as source of the headless relative. In what follows, I show that these arguments hold for Polish in the same way as they did for Modern German.

First, in headless relatives the morpheme \tit{kolwiek} `ever' can appear, as shown in \ref{ex:pol-headless-ever}.

\exg. Jan śpiewa co -kolwiek Maria śpiewa.\\
Jan sings \tsc{rp}.\tsc{an}.\tsc{acc} ever Maria sings\\
`Jan sings everything Maria sings.' \flushfill{Polish, \pgcitealt{citko2004}{116}}\label{ex:pol-headless-ever}

Light-headed relatives do not allow this morpheme to be inserted, illustrated in \ref{ex:pol-headed-ever}.\footnote{
\citet{citko2004} takes the complementary distribution of \tit{kolwiek} `ever' and the demonstrative to mean that they share the same syntactic position. I have nothing to say about the exact syntactic position of \tit{ever}, but in my account it cannot be the head of the relative clause, as this position is reserved for the extra light head. My reason for the incompatibility of \tit{ever} and the demonstrative is that they are semantically incompatible.
}

\exg. *Jan śpiewa to, co -kolwiek Maria śpiewa.\\
Jan sings \tsc{dem}.\tsc{m}.\tsc{sg}.\tsc{acc} \tsc{rp}.\tsc{an}.\tsc{acc} ever Maria sings\\
`John sings what Mary sings.' \flushfill{Polish, \pgcitealt{citko2004}{116}}\label{ex:pol-headed-ever}

The second argument against the existing light-headed relatives being the source of headless relatives comes from their interpretation. Headless relatives have two possible interpretations, and light-headed relatives have only one of these.
just as in Modern German, Polish headless relatives can be analyzed as either universal or definite \pgcitep{citko2004}{103}.
Light-headed relatives, such as the one in \ref{ex:pol-light-headed}, only have the definite interpretation.

In the remainder of this section, I discuss the two extra light heads that I compare the internal syntax of in Section \ref{sec:comparing-polish}. These are the accusative animate and the dative animate, shown in \ref{ex:pol-elhs}.
% \footnote{
% It is also possible that the strong pronoun is syncretic with the extra light head and that the extra light head is actually also spelled out as \tit{jego}/\tit{jemu}. This would mean that the strong extra light head consists of even more than two morphemes. For my proposal, it is important to show that the extra light head consists of at least two morphemes, one of which spells out case features. It works equally well when the non-case part of the structure actually consists of more one morpheme. I continue working out a proposal in which the extra light head is bimorphemic.
% }

\ex.\label{ex:pol-elhs}
\a. o-go `\tsc{elh}.\tsc{an}.\tsc{acc}'
\b. o-mu `\tsc{elh}.\tsc{an}.\tsc{dat}'

As I noted before, these forms do not surface as light heads in a light-headed relative. They do also not surface anywhere else in the language.

In Chapter \ref{ch:deriving-onlyinternal}, I showed that the relative pronoun contains two features more than the extra light head, namely \tsc{wh} and \tsc{rel}. This means that the functional sequence for the extra light head is as shown in \ref{ex:fseq-elh-rep}.

\ex.\label{ex:fseq-elh-rep}
\begin{forest} boom
  [\tsc{k}P
      [\tsc{k}]
      [\#P
          [\#]
          [\tsc{an}P
              [\tsc{an}]
              [\tsc{cl}P
                  [\tsc{cl}]
                  [\tsc{ref}]
              ]
          ]
      ]
  ]
\end{forest}

The functional sequence contains the pronominal feature \tsc{ref}, the gender features \tsc{cl} and \tsc{an}, the number feature \# and case features \tsc{k}.

I introduced the lexical entries that are required to spell out these features in Section \ref{sec:pol-rel}. I repeat them in \ref{ex:pol-entries-rep-lh}.

\ex.\label{ex:pol-entries-rep-lh}
\a.\label{ex:pol-entry-o-rep-lh}
\begin{forest} boom
  [\tsc{an}P
      [\tsc{an}]
      [\tsc{cl}P
          [\tsc{cl}]
          [\tsc{ref}]
      ]
  ]
  {\draw (.east) node[right]{⇔ \tit{o}}; }
\end{forest}
\b. \label{ex:pol-entry-go-rep-lh}
\begin{forest} boom
  [\tsc{acc}P
      [\tsc{k}2]
      [\tsc{nom}P
          [\tsc{k}1]
          [\#P
              [\#]
          ]
      ]
  ]
  {\draw (.east) node[right]{⇔ \tit{go}}; }
\end{forest}
\b. \label{ex:pol-entry-mu-rep-lh}
\begin{forest} boom
  [\tsc{dat}P
      [\tsc{k}3]
      [\tsc{acc}P
          [\tsc{k}2]
          [\tsc{nom}P
              [\tsc{k}1]
              [\#P
                  [\#]
              ]
          ]
      ]
  ]
  {\draw (.east) node[right]{⇔ \tit{mu}}; }
\end{forest}

In what follows, I construct the Polish extra light heads. Until the feature \#, the derivation is identical to the one of the relative pronoun. I give the syntactic structure at that point in \ref{ex:pol-spellout-o-ind-rep}.

\ex.\label{ex:pol-spellout-o-ind-rep}
\begin{forest} boom
  [\#P, s sep=25mm
      [\tsc{an}P,
      tikz={
      \node[label=below:\tit{o},
      draw,circle,
      scale=0.9,
      fit to=tree]{};
      }
          [\tsc{an}]
          [\tsc{cl}P
              [\tsc{cl}]
              [\tsc{ref}]
          ]
      ]
      [\#P,
      tikz={
      \node[label=below:\tit{go},
      draw,circle,
      scale=0.95,
      fit to=tree]{};
      }
          [\#]
      ]
  ]
\end{forest}

Then, the feature \tsc{k}1 is merged. The feature \tsc{k}1 is merged with the \#P, forming an \tsc{nom}P. This phrase is not contained in any of the Polish lexical entries. The first movement is tried, and the specifier of the \#P, the \tsc{an}P, is moved to the specifier of \tsc{nom}P. This phrase is contained in the lexical tree in \ref{ex:pol-entry-go-rep-lh}, so it is spelled out as \tit{go}, as shown in \ref{ex:pol-elh-nom}.

\ex.\label{ex:pol-elh-nom}
\begin{adjustbox}{max width=0.9\textwidth}
\begin{forest} boom
  [\tsc{nom}P, s sep=25mm
      [\tsc{an}P,
      tikz={
      \node[label=below:\tit{o},
      draw,circle,
      scale=0.9,
      fit to=tree]{};
      }
          [\tsc{an}]
          [\tsc{cl}P
              [\tsc{cl}]
              [\tsc{ref}]
          ]
      ]
      [\tsc{nom}P,
      tikz={
      \node[label=below:\tit{go},
      draw,circle,
      scale=0.9,
      fit to=tree]{};
      }
          [\tsc{k}1]
          [\#P
              [\#]
          ]
      ]
  ]
\end{forest}
\end{adjustbox}

For the accusative extra light head, the last feature is merged: the \tsc{k}2.
The feature is merged with the \tsc{nom}P, forming an \tsc{acc}P. This phrase is not contained in any of the lexical entries. The first movement is tried, and the specifier of the \tsc{nom}P, the \tsc{an}P, is moved to the specifier of \tsc{acc}P. This phrase is contained in the lexical tree in \ref{ex:pol-entry-go-rep-lh}, so it is spelled out as \tit{go}, as shown in \ref{ex:pol-elh-acc}.

\ex.\label{ex:pol-elh-acc}
\begin{adjustbox}{max width=0.9\textwidth}
\begin{forest} boom
  [\tsc{acc}P, s sep=25mm
      [\tsc{an}P,
      tikz={
      \node[label=below:\tit{o},
      draw,circle,
      scale=0.9,
      fit to=tree]{};
      }
          [\tsc{an}]
          [\tsc{cl}P
              [\tsc{cl}]
              [\tsc{ref}]
          ]
      ]
      [\tsc{acc}P,
      tikz={
      \node[label=below:\tit{go},
      draw,circle,
      scale=0.9,
      fit to=tree]{};
      }
          [\tsc{k}2]
          [\tsc{nom}P
              [\tsc{k}1]
              [\#P
                  [\#]
              ]
          ]
      ]
  ]
\end{forest}
\end{adjustbox}

For the dative relative pronoun, one more feature is merged: the \tsc{k}3.
The feature  is merged with the \tsc{acc}P, forming an \tsc{dat}P. This phrase is not contained in any of the lexical entries. The first movement is tried, and the specifier of the \tsc{acc}P, the \tsc{an}P, is moved to the specifier of \tsc{dat}P.
This phrase is contained in the lexical tree in \ref{ex:pol-entry-mu-rep-lh}, so it is spelled out as \tit{mu}, as shown in \ref{ex:pol-elh-dat}.

\ex.\label{ex:pol-elh-dat}
\begin{adjustbox}{max width=0.9\textwidth}
\begin{forest} boom
  [\tsc{dat}P, s sep=30mm
      [\tsc{an}P,
      tikz={
      \node[label=below:\tit{o},
      draw,circle,
      scale=0.95,
      fit to=tree]{};
      }
          [\tsc{an}]
          [\tsc{cl}P
              [\tsc{cl}]
              [\tsc{ref}]
          ]
      ]
      [\tsc{dat}P,
      tikz={
      \node[label=below:\tit{mu},
      draw,circle,
      scale=0.95,
      fit to=tree]{};
      }
          [\tsc{k}3]
          [\tsc{acc}P
              [\tsc{k}2]
              [\tsc{nom}P
                  [\tsc{k}1]
                  [\#P
                      [\#]
                  ]
              ]
          ]
      ]
  ]
\end{forest}
\end{adjustbox}

In sum, Polish headless relatives are derived from a light-headed relative with an extra light head, just as they are in Modern German. The extra light head is spelled out a lexical entry that spells out phi features and another one that spells out case features. The lexical entries used to spell this light head out are also used to spell out part of the internal syntax of the relative pronoun.


\section{Comparing light heads and relative pronouns}\label{sec:comparing-polish}

In this section, I compare the internal syntax of extra light heads to the internal syntax of relative pronouns in Polish. This is the worked out version of the comparisons in Section \ref{sec:basic-matching}. What is different here is that I show the comparison for Polish specifically, and that the content of the internal syntax that is being compared is motivated earlier in this chapter.

I give three examples, in which the internal and external case vary.
I start with an example with matching cases, in which the internal and the external case are both accusative.
Then I give an example in which the internal dative case is more complex than the external accusative case.
I end with an example in which the external dative case is more complex than the internal accusative case.
I show that the first example is grammatical and that the last two are not. I derive this by showing that only in the first situation the light head is structurally contained in the relative pronoun, and that it can therefore then be deleted.
In the other two examples, neither the light head nor the relative pronoun is structurally contained in the other element.
I do not discuss formal containment in this chapter, because it never leads to a successful deletion when structural containment does not.

I start with the matching cases.
Consider the example in \ref{ex:polish-acc-acc-rep}, in which the internal accusative case competes against the external accusative case. The relative clause is marked in bold.
The internal case is accusative, as the predicate \tit{lubić} `to like' takes accusative objects. The relative pronoun \tit{kogo} `\ac{rp}.\ac{an}.\ac{acc}' appears in the accusative case. This is the element that surfaces.
The external case is accusative as well, as the predicate \tit{lubić} `to like' also takes accusative objects. The extra light head \tit{ogo} `\ac{elh}.\ac{an}.\ac{acc}' appears in the accusative case. It is placed between square brackets because it does not surface.

\exg. Jan lubi [ogo] \tbf{kogo} \tbf{-kolkwiek} \tbf{Maria} \tbf{lubi}.\\
 Jan like.\tsc{3sg}\scsub{[acc]} \tsc{elh}.\tsc{acc}.\tsc{an}.\tsc{sg}  \tsc{rp}.\tsc{acc}.\tsc{an} ever Maria like.\tsc{3sg}\scsub{[acc]}\\
 `Jan likes whoever Maria likes.' \flushfill{Polish, adapted from \citealt{citko2013} after \pgcitealt{himmelreich2017}{17}}\label{ex:polish-acc-acc-rep}

In Figure \ref{fig:polish-int=ext}, I give the syntactic structure of the extra light head at the top and the syntactic structure of the relative pronoun at the bottom.

\begin{figure}[htbp]
  \center
  \begin{adjustbox}{max height=0.9\textheight}
  \begin{tabular}[b]{c}
        \toprule
        \tsc{acc} extra light head \tit{o-go} \\
        \cmidrule{1-1}
        \begin{forest} boom
          [\tsc{acc}P, s sep=20mm,
          tikz={
          \node[
          draw, circle,
          fill=DG,fill opacity=0.2,
          scale=0.95,
          yshift=-0.5cm,
          dashed,
          fit to=tree]{};
          }
              [\tsc{an}P
                  [\phantom{x}o\phantom{x}, roof]
              ]
              [\tsc{acc}P,
              tikz={
              \node[label=below:\tit{go},
              draw,circle,
              scale=0.9,
              fit to=tree]{};
              }
                  [\tsc{k}2]
                  [\tsc{nom}P
                      [\tsc{k}1]
                      [\#P
                          [\#]
                      ]
                  ]
              ]
          ]
        \end{forest}
        \vspace{0.3cm}
      \\
      \toprule
      \tsc{acc} relative pronoun \tit{k-o-go}
      \\
      \cmidrule{1-1}
      \begin{forest} boom
        [\tsc{rel}P, s sep=15mm
            [\tsc{rel}P
                [\phantom{x}k\phantom{x}, roof]
            ]
            [\tsc{acc}P, s sep=20mm,
            tikz={
            \node[
            draw, circle,
            scale=0.95,
            yshift=-0.5cm,
            dashed,
            fit to=tree]{};
            }
                [\tsc{an}P
                    [\phantom{x}o\phantom{x}, roof]
                ]
                [\tsc{acc}P,
                tikz={
                \node[label=below:\tit{go},
                draw,circle,
                scale=0.9,
                fit to=tree]{};
                }
                    [\tsc{k}2]
                    [\tsc{nom}P
                        [\tsc{k}1]
                        [\#P
                            [\#]
                        ]
                    ]
                ]
            ]
        ]
      \end{forest}
      \vspace{0.3cm}
      \\
      \bottomrule
  \end{tabular}
  \end{adjustbox}
   \caption {Polish \tsc{ext}\scsub{acc} vs. \tsc{int}\scsub{acc} → \tit{kogo}}
  \label{fig:polish-int=ext}
\end{figure}

The extra light head consists of two morphemes: \tit{o} and \tit{go}.
The relative pronoun consists of three morphemes: \tit{k}, \tit{o} and \tit{go}.
As usual, I circle the part of the structure that corresponds to a particular lexical entry, or I reduce the structure to a triangle, and I place the corresponding phonology below it.
I draw a dashed circle around the \tsc{acc}P, as it is the biggest possible element that is structurally a constituent in both the extra light head and the relative pronoun.

The extra light head consists of two constituents: the \tsc{an}P and the (lower) \tsc{acc}P. Together they form the (higher) \tsc{acc}P.
This (higher) \tsc{acc}P is structurally contained in the relative pronoun. Therefore, the extra light head can be deleted. I signal the deletion of the extra light head by marking the content of its circle gray.
The surface element is the relative pronoun that bears the internal case: \tit{kogo}.

I continue with the example in which the internal case is more complex than the external case.
Consider the examples in \ref{ex:polish-acc-dat-rep}, in which the internal dative case competes against the external accusative case. The relative clauses are marked in bold. It is not possible to make a grammatical headless relative in this situation.
The internal case is dative, as the predicate \tit{dokuczać} `to tease' takes dative objects. The relative pronoun \tit{komu} `\ac{rp}.\ac{an}.\ac{dat}' appears in the dative case.
The external case is accusative, as the predicate \tit{lubić} `to like' takes accusative objects. The extra light head \tit{ogo} `\ac{elh}.\ac{an}.\ac{acc}' appears in the accusative case.
\ref{ex:polish-acc-dat-rel} is the variant of the sentence in which the extra light head is absent (indicated by the square brackets) and the relative pronoun surfaces, which is ungrammatical.
\ref{ex:polish-acc-dat-lh} is the variant of the sentence in which the relative pronoun is absent (indicated by the square brackets) and the extra light head surfaces, which is ungrammatical too.

\ex.\label{ex:polish-acc-dat-rep}
\ag. *Jan lubi [ogo] \tbf{komu} \tbf{-kolkwiek} \tbf{dokucza}.\\
Jan like.\tsc{3sg}\scsub{[acc]} \tsc{elh}.\tsc{acc}.\tsc{an} \tsc{rp}.\tsc{dat}.\tsc{an}.\tsc{sg} ever tease.\tsc{3sg}\scsub{[dat]}\\
`Jan likes whoever he teases.' \flushfill{Polish, adapted from \citealt{citko2013} after \pgcitealt{himmelreich2017}{17}}\label{ex:polish-acc-dat-rel}
\bg. *Jan lubi ogo [\tbf{komu}] \tbf{-kolkwiek} \tbf{dokucza}.\\
Jan like.\tsc{3sg}\scsub{[acc]} \tsc{elh}.\tsc{acc}.\tsc{an} \tsc{rp}.\tsc{dat}.\tsc{an}.\tsc{sg} ever tease.\tsc{3sg}\scsub{[dat]}\\
`Jan likes whoever he teases.' \flushfill{Polish, adapted from \citealt{citko2013} after \pgcitealt{himmelreich2017}{17}}\label{ex:polish-acc-dat-lh}

In Figure \ref{fig:polish-int-wins}, I give the syntactic structure of the extra light head at the top and the syntactic structure of the relative pronoun at the bottom.

\begin{figure}[htbp]
  \center
  \begin{adjustbox}{max height=0.9\textheight}
  \begin{tabular}[b]{c}
        \toprule
        \tsc{acc} extra light head \tit{o-go} \\
        \cmidrule{1-1}
        \begin{forest} boom
          [\tsc{acc}P, s sep=20mm
              [\tsc{an}P,
              tikz={
              \node[
              draw,circle,
              scale=0.85,
              dashed,
              fit to=tree]{};
              }
                  [\phantom{x}o\phantom{x}, roof]
              ]
              [\tsc{acc}P,
              tikz={
              \node[label=below:\tit{go},
              draw,circle,
              scale=0.9,
              fit to=tree]{};
              \node[
              draw,circle,
              scale=0.95,
              dashed,
              fit to=tree]{};
              }
                  [\tsc{k}2]
                  [\tsc{nom}P
                      [\tsc{k}1]
                      [\#P
                          [\#]
                      ]
                  ]
              ]
          ]
        \end{forest}
        \vspace{0.3cm}
      \\
      \toprule
      \tsc{acc} relative pronoun \tit{k-o-mu}
      \\
      \cmidrule{1-1}
      \begin{forest} boom
        [\tsc{rel}P, s sep=15mm
            [\tsc{rel}P
                [\phantom{x}k\phantom{x}, roof]
            ]
            [\tsc{dat}P, s sep=22mm
                [\tsc{an}P,
                tikz={
                \node[
                draw,circle,
                scale=0.85,
                dashed,
                fit to=tree]{};
                }
                    [\phantom{x}o\phantom{x}, roof]
                ]
                [\tsc{dat}P,
                tikz={
                \node[label=below:\tit{mu},
                draw,circle,
                scale=0.95,
                fit to=tree]{};
                }
                    [\tsc{k}3]
                    [\tsc{acc}P, tikz={
                    \node[
                    draw,circle,
                    scale=0.9,
                    dashed,
                    fit to=tree]{};
                    }
                        [\tsc{k}2]
                        [\tsc{nom}P
                            [\tsc{k}1]
                            [\#P
                                [\#]
                            ]
                        ]
                    ]
                ]
            ]
        ]
      \end{forest}
      \\
      \bottomrule
  \end{tabular}
  \end{adjustbox}
   \caption {Polish \tsc{ext}\scsub{acc} vs. \tsc{int}\scsub{dat} ↛ \tit{ogo}/\tit{komu}}
  \label{fig:polish-int-wins}
\end{figure}

The light head consists of two morphemes: \tit{o} and \tit{go}.
The relative pronoun consists of three morphemes: \tit{k}, \tit{o} and \tit{mu}.
I draw a dashed circle around the \tsc{an}P and the \tsc{acc}P, as they are the biggest possible elements that are structurally constituents in both the extra light head and the relative pronoun.

In this case, the light head is not a constituent that is structurally contained in the relative pronoun.
The extra light head consists of two constituents: the \tsc{an}P and the (lower) \tsc{acc}P. Together they form the (higher) \tsc{acc}P.
Both of these constituents are also constituents that are structurally contained in the relative pronoun. However, the (higher) \tsc{acc}P is not a constituent that is structurally contained in the relative pronoun. The constituent in which the \tsc{acc}P is contained also contains the feature \tsc{k}3 that makes it a \tsc{dat}P.
In other words, each feature and even each subconstituent of the extra light head is contained in the relative pronoun. However, they do not form a single constituent that is structurally contained in the relative pronoun. Therefore, the extra light head cannot be deleted.

Recall from Section \ref{sec:comparing-mg} that this is the crucial example in which Modern German and Polish differ. The contrast lies in that the extra light head in Modern German corresponds to a single lexical entry and in Polish it corresponds to two lexical entries.
In Modern German, extra light heads in a less complex case form a constituent that is structurally contained in the relative pronoun.
In Polish, they do not.
Relative pronouns in a complex case still contain all features of the extra light head in a less complex case, but the extra light head does not form a single constituent that is structurally contained in the relative pronoun. That is, the weaker feature containment requirement is met, but the stronger constituent containment requirement is not. This shows the necessity of formulating the proposal in terms of containment as a single constituent.\footnote{
A single constituent that the relative pronoun contains is the \tsc{an}P. This could be deleted, leaving the \tsc{acc}P, spelled out as \tit{go}. This structure would then be ruled out, because \tit{go} cannot be the head of a relative clause. The same holds for only deleting \tit{go} and leaving \tit{o} behind.
}

The relative pronoun is not a constituent that is structurally or formally contained in the light head. It lacks the complete constituent and \tsc{rel}P.
Therefore, the extra light cannot be deleted, and the relative pronoun cannot be deleted either.
As a result, there is no grammatical headless relative possible.

I end with the example in which the external case is more complex than the internal case.
Consider the examples in \ref{ex:polish-dat-acc-rep}, in which the internal dative case competes against the external accusative case. The relative clauses are marked in bold. It is not possible to make a grammatical headless relative in this situation.
The internal case is accusative, as the predicate \tit{wpuścić} `to let' takes accusative objects. The relative pronoun \tit{kogo} `\ac{rp}.\ac{an}.\ac{acc}' appears in the accusative case.
The external case is dative, as the predicate \tit{ufać} `to trust' takes dative objects. The extra light head \tit{omu} `\ac{elh}.\ac{an}.\ac{dat}' appears in the dative case.
\ref{ex:polish-dat-acc-rel} is the variant of the sentence in which the extra light head is absent (indicated by the square brackets) and the relative pronoun surfaces, which is ungrammatical.
\ref{ex:polish-dat-acc-lh} is the variant of the sentence in which the relative pronoun is absent (indicated by the square brackets) and the extra light head surfaces, which is ungrammatical too.

\ex.\label{ex:polish-dat-acc-rep}
\ag. *Jan ufa [omu] \tbf{kogo} \tbf{-kolkwiek} \tbf{wpuścil} \tbf{do} \tbf{domu}.\\
Jan trust.\tsc{3sg}\scsub{[dat]} \tsc{elh}.\tsc{dat}.\tsc{an} \tsc{rp}.\tsc{acc}.\tsc{an} ever let.\tsc{3sg}\scsub{[acc]} to home\\
`Jan trusts whoever he let into the house.' \flushfill{Polish, adapted from \citealt{citko2013} after \pgcitealt{himmelreich2017}{17}}\label{ex:polish-dat-acc-rel}
\bg. *Jan ufa omu [\tbf{kogo}] \tbf{-kolkwiek} \tbf{wpuścil} \tbf{do} \tbf{domu}.\\
Jan trust.\tsc{3sg}\scsub{[dat]} \tsc{elh}.\tsc{dat}.\tsc{an} \tsc{rp}.\tsc{acc}.\tsc{an} ever let.\tsc{3sg}\scsub{[acc]} to home\\
`Jan trusts whoever he let into the house.' \flushfill{Polish, adapted from \citealt{citko2013} after \pgcitealt{himmelreich2017}{17}}\label{ex:polish-dat-acc-lh}

In Figure \ref{fig:polish-ext-wins}, I give the syntactic structure of the extra light head at the top and the syntactic structure of the relative pronoun at the bottom.

\begin{figure}[htbp]
  \center
  \begin{adjustbox}{max height=0.9\textheight}
  \begin{tabular}[b]{c}
        \toprule
        \tsc{dat} extra light head \tit{o-mu} \\
        \cmidrule{1-1}
        \begin{forest} boom
          [\tsc{dat}P, s sep=25mm
              [\tsc{an}P,
              tikz={
              \node[
              draw,circle,
              scale=0.85,
              dashed,
              fit to=tree]{};
              }
                  [\phantom{x}o\phantom{x}, roof]
              ]
              [\tsc{dat}P,
              tikz={
              \node[label=below:\tit{mu},
              draw,circle,
              scale=0.95,
              fit to=tree]{};
              }
                  [\tsc{k}3]
                  [\tsc{acc}P,
                  tikz={
                  \node[
                  draw,circle,
                  scale=0.9,
                  dashed,
                  fit to=tree]{};
                  }
                      [\tsc{k}2]
                      [\tsc{nom}P
                          [\tsc{k}1]
                          [\#P
                              [\#]
                          ]
                      ]
                  ]
              ]
          ]
        \end{forest}
        \vspace{0.3cm}
      \\
      \toprule
      \tsc{acc} relative pronoun \tit{k-o-go}
      \\
      \cmidrule{1-1}
      \begin{forest} boom
        [\tsc{rel}P, s sep=15mm
            [\tsc{rel}P
                [\phantom{x}k\phantom{x}, roof]
            ]
            [\tsc{acc}P, s sep=20mm
                [\tsc{an}P,
                tikz={
                \node[
                draw,circle,
                scale=0.85,
                dashed,
                fit to=tree]{};
                }
                    [\phantom{x}o\phantom{x}, roof]
                ]
                [\tsc{acc}P,
                tikz={
                \node[label=below:\tit{go},
                draw,circle,
                scale=0.9,
                fit to=tree]{};
                \node[
                draw,circle,
                scale=0.95,
                dashed,
                fit to=tree]{};
                }
                    [\tsc{k}2]
                    [\tsc{nom}P
                        [\tsc{k}1]
                        [\#P
                            [\#]
                        ]
                    ]
                ]
            ]
        ]
      \end{forest}
      \\
      \bottomrule
  \end{tabular}
  \end{adjustbox}
   \caption {Polish \tsc{ext}\scsub{dat} vs. \tsc{int}\scsub{acc} ↛ \tit{omu}/\tit{kogo}}
  \label{fig:polish-ext-wins}
\end{figure}

The light head consists of two morphemes: \tit{o} and \tit{mu}.
The relative pronoun consists of three morphemes: \tit{k}, \tit{o} and \tit{go}.
I draw a dashed circle around the \tsc{an}P and the \tsc{acc}P, as they are the biggest possible elements that are structurally constituents in both the extra light head and the relative pronoun.

In this case, the light head is not a constituent that is structurally contained in the relative pronoun.
The light head consists of two morphemes: \tit{o} and \tit{mu}.
The relative pronoun only contains the \tsc{acc}P, and it lacks the \tsc{k}3 that makes a \tsc{dat}P. Since the weaker feature containment requirement is not met, the stronger constituent containment requirement cannot be met either.

The relative pronoun is not a constituent that is structurally contained in the light head. It lacks the complete constituent \tsc{rel}P.
Therefore, the extra light cannot be deleted, and the relative pronoun cannot be deleted either.
As a result, there is no grammatical headless relative possible.

\section{Summary and discussion}

Polish is an example of a matching type of language. This means that headless relatives are grammatical in the language only when the internal and external case match.

I derive this from the internal syntax of light heads and relative pronouns in Polish. The features of the light head are spelled out by two lexical entries, which respectively spell out phi and case features. The features of the relative pronoun are spelled out by the same lexical entries plus one that amongst other spells out the relative feature. The internal syntax of the Polish light head and relative pronoun are shown in Figure \ref{fig:rel-lh-pol-sum}.

\begin{figure}[htbp]
  \center
  \begin{adjustbox}{max width=\textwidth}
    \begin{tabular}[b]{ccc}
        \toprule
        light head & & relative pronoun \\
        \cmidrule(lr){1-1} \cmidrule(lr){3-3}
        \begin{forest} boom
        [\tsc{k}P, s sep = 20 mm
            [ϕP,
            tikz={
            \node[label=below:\tit{o},
            draw,circle,
            scale=0.85,
            fit to=tree]{};
            }
                [\phantom{xxx}, roof]
            ]
            [\tsc{k}P,
            tikz={
            \node[label=below:\tit{go/mu},
            draw,circle,
            scale=0.85,
            fit to=tree]{};
            }
                [\tsc{k}, baseline]
            ]
        ]
        \end{forest}
        & \phantom{x} &
      \begin{forest} boom
        [\tsc{rel}P, s sep = 15 mm
            [\tsc{rel}P,
            tikz={
            \node[label=below:\tit{k},
            draw,circle,
            scale=0.85,
            fit to=tree]{};
            }
                [\phantom{xxx}, roof, baseline]
            ]
            [\tsc{k}P, s sep = 20 mm
                [ϕP,
                tikz={
                \node[label=below:\tit{o},
                draw,circle,
                scale=0.85,
                fit to=tree]{};
                }
                    [\phantom{xxx}, roof]
                ]
                [\tsc{k}P,
                tikz={
                \node[label=below:\tit{go/mu},
                draw,circle,
                scale=0.85,
                fit to=tree]{};
                }
                    [\tsc{k}, baseline]
                ]
            ]
        ]
      \end{forest}\\
        \bottomrule
    \end{tabular}
  \end{adjustbox}
   \caption {\tsc{elh} and \tsc{rp} in Polish (repeated)}
  \label{fig:rel-lh-pol-sum}
\end{figure}

A crucial characteristic of matching languages such as Polish is that they have separate morphemes for phi and case features. Therefore, the light head is structurally contained in the relative pronoun when the internal and the external case match. As a result, the light head can be deleted, and the relative pronoun can surface, bearing the internal case.
When the internal and external case differ, neither the light head nor the relative pronoun is structurally contained in the other element. None of the elements can be deleted, and there is no grammatical headless relative possible.

In other words, the crucial difference between Modern German and Polish that leads them to be of different language types is that I analyze Modern German extra light heads as monomorphemic (e.g. \tit{n}) and Polish extra light heads as bimorphemic (e.g. \tit{o-go}).
This raises the question of whether Polish extra light heads could not be monomorphemic too. A possible hypothesis would be that the morpheme \tit{go} is the extra light head and not \tit{o-go}.

A parallel between the Polish \tit{go} (and \tit{mu}) and the Modern German \tit{n} (and \tit{m}) is that they both surface as pronouns in their respective languages. However, there is a difference between the Polish \tit{go} and Modern German \tit{n}. The Polish \tit{go} is analyzed as a clitic \citep{swan2002}, whereas the Modern German \tit{n} is analyzed as a weak pronoun (see Section \ref{sec:mg-weak-pronoun}).
According to \citet{cardinaletti1994}, clitics and weak pronouns differ from each other in that clitics cannot follow prepositions whereas weak pronouns can, and weak pronouns cannot follow dative objects whereas clitics can. Polish \tit{go} and \tit{mu} cannot follow prepositions \pgcitep{swan2002}{157}, whereas Modern German \tit{n} and \tit{m} can (see Section \ref{sec:mg-weak-pronoun}).\footnote{
As I mentioned in Section \ref{sec:mg-weak-pronoun} not all speakers of Modern German allow the \tit{n} and \tit{m} to follow prepositions. This might suggests that these pronouns are clitics in their grammar, just as in Polish. This raises the question of whether these speakers are also as Polish speakers with the respect to headless relatives, in that they only accept matching cases (which is the version of Modern German described in \citealt{groos1981}).
I do not investigate this question any further in this thesis.
I also do not work out an analysis for the variety of Modern German in which \tit{n} and \tit{m} are clitics.
}
Theoretically, clitics lack certain pronominal features (\tsc{ref} and Σ, see Section \ref{sec:mg-weak-pronoun}), which are indeed not spelled out by the clitic in my analysis of Polish, but by \tit{o}.\footnote{
In my analysis, Polish clitics also lack gender features.
}
I assume that the weak pronoun in Polish is \tit{o-go} (which combines with the \tit{i} I introduced in Section \ref{sec:pol-rel} to become a strong pronoun). A question that remains unanswered is why \tit{o-go} does not surface as a weak pronoun.
