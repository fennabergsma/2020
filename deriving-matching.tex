% !TEX root = thesis.tex

\chapter{Deriving the matching type}\label{ch:deriving-matching}

In Chapter \ref{ch:the-basic-idea}, I suggested that languages of the matching type have a morpheme that spells out phi-features and another one that spells out case-features. This is the crucial difference with internal-only languages such as Modern German, that have a portmanteau for phi- and case-features. This means that the features of the relative pronoun and the light head are spelled out in such a way that they form the constituents shown in Figure \ref{fig:rel-lh-matching-simple}.

\begin{figure}[htbp]
  \center
  \begin{tabular}[b]{ccc}
      \toprule
      light head & & relative pronoun \\
      \cmidrule(lr){1-1} \cmidrule(lr){3-3}
      \begin{forest} boom
      [\tsc{k}P, s sep = 20 mm
          [ϕP,
          tikz={
          \node[draw,circle,
          scale=0.85,
          fit to=tree]{};
          }
              [\phantom{xxx}, roof]
          ]
          [\tsc{k}P,
          tikz={
          \node[draw,circle,
          scale=0.85,
          fit to=tree]{};
          }
              [\tsc{k}, baseline]
          ]
      ]
      \end{forest}
      & \phantom{x} &
    \begin{forest} boom
      [\tsc{rel}P, s sep = 15 mm
          [\tsc{rel}P,
          tikz={
          \node[draw,circle,
          scale=0.85,
          fit to=tree]{};
          }
              [\phantom{xxx}, roof, baseline]
          ]
          [\tsc{k}P, s sep = 20 mm
              [ϕP,
              tikz={
              \node[draw,circle,
              scale=0.85,
              fit to=tree]{};
              }
                  [\phantom{xxx}, roof]
              ]
              [\tsc{k}P,
              tikz={
              \node[draw,circle,
              scale=0.85,
              fit to=tree]{};
              }
                  [\tsc{k}, baseline]
              ]
          ]
      ]
    \end{forest}\\
      \bottomrule
  \end{tabular}
   \caption {\tsc{lh} and \tsc{rp} in the matching type}
  \label{fig:rel-lh-matching-simple}
\end{figure}

These lexical entries lead to a grammaticality pattern as shown in Table \ref{tbl:overview-matching}.

\begin{table}[htbp]
  \center
  \caption{Grammaticality in the matching type}
  \begin{adjustbox}{max width=\textwidth}
  \begin{tabular}{cccccc}
    \toprule
    situation           & \multicolumn{2}{c}{lexical entries}       & containment         & deleted             & surfacing           \\
    \cmidrule(lr){1-1}    \cmidrule(lr){2-3}                          \cmidrule(lr){4-4}    \cmidrule(lr){5-5}    \cmidrule(lr){6-6}
                        & \tsc{lh}            & \tsc{rp}            &                     &                     &                     \\
                          \cmidrule(lr){2-2}    \cmidrule(lr){3-3}
  \tsc{k}\scsub{int} = \tsc{k}\scsub{ext}               &
  [\tsc{k}\scsub{1}], [ϕ]                               &
  [\tsc{rel}], [\tsc{k}\scsub{1}], [ϕ]                  &
  structure & \tsc{lh} & \tsc{rp}\scsub{int}            \\
  \tsc{k}\scsub{int} > \tsc{k}\scsub{ext}               &
  [\tsc{k}\scsub{1}], [ϕ]                               &
  [\tsc{rel}], [\tsc{k}\scsub{2}[\tsc{k}\scsub{1}]], [ϕ] &
  no & none & *                                         \\
  \tsc{k}\scsub{int} < \tsc{k}\scsub{ext}               &
  [\tsc{k}\scsub{2}[\tsc{k}\scsub{1}]], [ϕ]             &
  [\tsc{rel}], [\tsc{k}\scsub{1}], [ϕ]                  &
  no & none & *                                         \\
  \bottomrule
  \end{tabular}
  \end{adjustbox}
\label{tbl:overview-matching}
\end{table}

First consider the situation in which the internal and the external case match. The light head consists of a phi-feature morpheme and a case-feature morpheme. The relative pronoun consists of the same two morphemes plus an additional morpheme that spells out the feature \tsc{rel}. These lexical entries create such a structure that  the light head structurally forms a constituent within the relative pronoun. Therefore, the light head can be deleted, and the relative pronoun that bears the internal case surfaces.
In this situation, the fact that there is a phi- and case-feature portmanteau (as in internal-only languages) or there are two separate morphemes for the features (as in matching languages) does not make a difference for the containment and for the grammaticality.

Consider now the situation in the internal case wins the case competition. The light head consists of a phi-feature morpheme and a case-feature morpheme. The relative pronoun consists of the same phi-feature morpheme, a case-morpheme that that contains at least one more case feature than the light head (\tsc{k}\scsub{2} in Figure \ref{tbl:overview-matching}) plus an additional morpheme that spells out the feature \tsc{rel}. These lexical entries create such syntactic structures that neither the light head nor the relative pronoun structurally forms a constituent within the other element. Therefore, none of the elements can be deleted, and there is no headless relative construction possible.
In this situation, the fact that there is a phi- and case-feature portmanteau (as in internal-only languages) or there are two separate morphemes for the features (as in matching languages) does make a difference for the containment and for the grammaticality: this situation is grammatical in internal-only languages and it is not in matching languages.

Finally, consider the situation in which the external case wins the case competition. The relative pronoun consists of a phi-feature morpheme, a case-feature morpheme and an additional morpheme that spells out the feature \tsc{rel}. Compared to the relative pronoun, the light head lacks the morpheme that spells out \tsc{rel}, and it contains at least one more case feature (\tsc{k}\scsub{2} in Figure \ref{tbl:overview-intonly}). These lexical entries create such syntactic structures that neither the light head nor the relative pronoun structurally forms a constituent within the other element. Therefore, none of the elements can be deleted, and there is no headless relative construction possible.
In this situation, the fact that there is a phi- and case-feature portmanteau (as in internal-only languages) or there are two separate morphemes for the features (as in matching languages) does not make a difference for the containment and for the grammaticality.

In Chapter \ref{ch:typology}, I showed that Polish is a language of the matching type. In this chapter, I show that Polish light heads and relative pronouns have this type of structure described in Figure \ref{fig:rel-lh-matching-simple}. I give a compact version of the structures in Figure \ref{fig:rel-lh-pol}.

\begin{figure}[htbp]
  \center
  \begin{adjustbox}{max width=\textwidth}
  \begin{tabular}[b]{ccc}
      \toprule
      light head & & relative pronoun \\
      \cmidrule(lr){1-1} \cmidrule(lr){3-3}
      \begin{forest} boom
        [\tsc{k}P, s sep = 15mm
            [\tsc{an}P,
            tikz={
            \node[
            draw,circle,
            scale=0.75,
            fit to=tree]{};
            }
                [\phantom{xxx}, roof, baseline]
            ]
            [\tsc{k}P,
            tikz={
            \node[
            draw,circle,
            scale=0.85,
            fit to=tree]{};
            }
                [\tsc{k}]
                [\tsc{ind}P
                    [\tsc{ind}]
                ]
            ]
        ]
      \end{forest}
      & \phantom{x} &
      \begin{forest} boom
        [\tsc{rel}P
            [\tsc{rel}P,
            tikz={
            \node[
            draw,circle,
            scale=0.75,
            fit to=tree]{};
            }
                [\phantom{xxx}, roof]
            ]
            [\tsc{k}P, s sep = 15mm
                [\tsc{an}P,
                tikz={
                \node[
                draw,circle,
                scale=0.75,
                fit to=tree]{};
                }
                    [\phantom{xxx}, roof, baseline]
                ]
                [\tsc{k}P,
                tikz={
                \node[
                draw,circle,
                scale=0.85,
                fit to=tree]{};
                }
                    [\tsc{k}]
                    [\tsc{ind}P
                        [\tsc{ind}]
                    ]
                ]
            ]
        ]
      \end{forest}\\
      \bottomrule
  \end{tabular}
  \end{adjustbox}
   \caption {\tsc{lh} and \tsc{rp} in Polish}
  \label{fig:rel-lh-pol}
\end{figure}

Consider the light head in Figure \ref{fig:rel-lh-pol}.
Light heads (i.e. the phi- and case-features) in Polish are spelled out by two morphemes, which are both circled. The phi-features are spelled out as \tit{o} and the case features are spelled out as \tit{go} or \tit{mu}, depending on which case they realize.
Consider the relative pronoun in Figure \ref{fig:rel-lh-pol}.
Relative pronouns in Polish contain one more morpheme than light heads: the constituent that forms the light head (i.e. phi- and case-features) and the \tsc{rel}P, again circled. The \tsc{rel}P is spelled out as \tit{k}.
Throughout this chapter, I discuss the exact feature content of relative pronouns and light heads, I give lexical entries for them, and I show how these lexical entries form the constituents shown in Figure \ref{fig:rel-lh-pol}.

The chapter is structured as follows.
First, I discuss the relative pronoun. I decompose the relative pronoun into the three morphemes I showed in Figure \ref{fig:rel-lh-pol}, and I show which features correspond to which morphemes.
Then, I discuss the light head. I show that the Polish light head consist of a subset of the features of the relative pronoun. I also show that Polish headless relatives are derived from a type of light-headed relative clause that does not surface in the language, just like their Modern German counterparts.
Importantly, the features that form the Polish light head and relative pronoun are the same ones that form the Modern German ones. The only difference between the two languages is how the features are spelled out.
Finally, I compare the constituents of the light head and the relative pronoun. I show that the light head can only be deleted when the internal case matches the external case. When the internal and external case differ, I show that none of the elements can be deleted.




\section{The Polish relative pronoun}\label{sec:pol-rel}

In the introduction of this chapter, I suggested that in Polish features of the relative pronoun are spelled out in such a way that they form the constituents shown in \ref{ex:simple-matching-rp}.

\ex.\label{ex:simple-matching-rp}
\begin{forest} boom
  [\tsc{rel}P
      [\tsc{rel}P
          [\phantom{xxx}, roof]
      ]
      [\tsc{k}P
          [ϕP
              [\phantom{xxx}, roof]
          ]
          [\tsc{k}P
              [\tsc{k}]
          ]
      ]
  ]
\end{forest}

In Figure \ref{ex:simple-matching-rp} I only show three features: \tsc{rel}, ϕ and \tsc{k}. In Chapter \ref{ch:deriving-onlyinternal} I showed that Modern German relative pronouns contain more features than that. In this section, I show that Polish relative pronouns consist of the same features.
Crucially, the main claim I made in Chapter \ref{ch:the-basic-idea} remains unchanged: Polish has a separate morpheme for phi-features, one for case-features and one for the features the light head does not contain. Actually, the morpheme for case-features contains a number feature and the phi-feature morpheme does not contain one, but this does not influence the point here.
I show the complete structure that I work towards in this section in \ref{ex:pol-rp}.

\ex.\label{ex:pol-rp}
\begin{adjustbox}{max width=0.9\textwidth}
\begin{forest} boom
  [\tsc{rel}P, s sep=40mm
      [\tsc{rel}P,
      tikz={
      \node[label=below:\tit{k},
      draw,circle,
      scale=0.95,
      fit to=tree]{};
      }
          [\tsc{rel}]
          [\tsc{wh}P
              [\tsc{wh}]
              [\tsc{ind}P
                  [\tsc{ind}]
                  [\tsc{an}P
                      [\tsc{an}]
                      [\tsc{cl}]
                  ]
              ]
          ]
      ]
      [\tsc{acc}P, s sep=30mm
      [\tsc{an}P,
          tikz={
          \node[label=below:\tit{o},
          draw,circle,
          scale=0.9,
          fit to=tree]{};
          }
          [\tsc{an}P]
          [\tsc{cl}P
              [\tsc{cl}]
              [\tsc{ref}]
          ]
      ]
          [\tsc{k}P,
          tikz={
          \node[label=below:\tit{go/mu},
          draw,circle,
          scale=0.9,
          fit to=tree]{};
          }
              [\tsc{k}]
              [\tsc{ind}P
                  [\tsc{ind}]
              ]
          ]
      ]
  ]
  \end{forest}
  \end{adjustbox}

I discuss two relative pronouns: the animate accusative and the animate dative. These are the two forms that I compare the constituents of in Section \ref{sec:comparing-pol}. I show them in \ref{ex:pol-rels}.

\ex.\label{ex:pol-rels}
\ag. k-o-go\\
 \tsc{rp}.\tsc{an}.\tsc{acc}\\
\bg. k-o-mu\\
 \tsc{rp}.\tsc{an}.\tsc{dat}\\

I decompose the relative pronouns in three morphemes: \tit{k}, \tit{o} and the suffix (\tit{go} or \tit{mu}). For each morpheme, I discuss which features they spell out, I give their lexical entries, and I show how I construct the relative pronouns by combining the separate morphemes.

I start with the suffixes \tit{go} and \tit{mu}.
These two morphemes correspond to what I called the case-feature morpheme in Chapter \ref{ch:the-basic-idea} and the introduction to this chapter. In addition, the morphemes spell out a number feature.

First I focus on \tit{mu} and I extend the analysis to \tit{go}. Consider Table \ref{tbl:pol-datives}.

\begin{table}[htbp]
  \center
  \caption{Syncretic \tsc{n}/\tsc{m} dative forms \citep{swan2002}}
  \begin{tabular}[b]{ccc}
    \toprule
                      & \tsc{m}   & \tsc{n}  \\
    \cmidrule{2-3}
    \tit{je}-pronoun  & je-mu    & je-mu   \\
    \tit{ni}-pronoun  & nie-mu   & nie-mu  \\
    \tsc{dem}         & te-mu    & te-mu   \\
    \bottomrule
  \end{tabular}
  \label{tbl:pol-datives}
\end{table}

The table shows three forms in which there is a syncretism between the neuter and the masculine in the dative case. The complete pronouns are syncretic. I set up a system that can derive the syncretism between the two genders. Doing this allows me to establish which features the morpheme \tit{mu} spells out.

I discussed in Chapter \ref{ch:decomposition} that syncretisms can be derived in Nanosyntax via the Superset Principle. The lexicon contains a lexical entry that is specified for the form that corresponds to the most features. To illustrate this, I repeat the lexical entry for the Dutch \tit{jullie} `you' in \ref{ex:dutch-jullie-lexicon-rep}.

\ex.
\begin{forest} boom
  [\ac{dat}P
      [\ac{f}3]
      [\ac{acc}P
          [\ac{f}2]
          [\ac{nom}P
              [\ac{f}1]
              [2\ac{pl}P
                  [\phantom{xxx}, roof]
              ]
          ]
      ]
  ]
  {\draw (.east) node[right]{⇔ \tit{jullie}}; }
\end{forest}
\label{ex:dutch-jullie-lexicon-rep}

\tit{Jullie} is syncretic between nominative, accusative and dative. It is specified for dative in the lexicon, because the dative contains the accusative and the nominative. The nominative and accusative second person plural in Dutch are spelled out as \tit{jullie} as well, because the \tsc{dat}P contains the \tsc{acc}P which contains \tsc{nom}P (Superset Principle), and there is no more specific lexical entry available in Dutch (Elsewhere Condition). It is important that the potentially unused features (so the \tsc{f}3 or \tsc{f}3 and the \tsc{f2}) are at the top, so that the constituent that needs to be spelled out is still contained in the lexical tree.

In what follows, I show how I can derive the syncretisms for the forms in Table \ref{tbl:pol-datives}. I propose that \tit{jego}, \tit{niego} and \tit{tego} spell out the syntactic structure in \ref{ex:pol-mu-dat-fseq}.

\ex.\label{ex:pol-mu-dat-fseq}
\begin{forest} boom
  [\tsc{dat}P
      [\tsc{f}3]
      [\tsc{acc}P
          [\tsc{f}2]
          [\tsc{nom}P
            [\tsc{f}1]
            [\tsc{ind}P
                [\tsc{ind}]
                [\tsc{an}P
                    [\tsc{an}]
                    [\tsc{cl}P
                        [\tsc{cl}]
                        [XP
                            [\phantom{xxx}, roof]
                        ]
                    ]
                ]
            ]
          ]
      ]
  ]
\end{forest}

I do not discuss the feature content that distinguishes \tit{jego}, \tit{niego} and \tit{tego}, but I call them XP.
Following the functional sequence I suggested in Chapter \ref{ch:deriving-onlyinternal}, all forms contain the feature \tsc{cl} for inanimate/neuter gender, \tsc{an} for animate/masculine gender and \tsc{ind} for singular number and case features up to the dative.

The forms \tit{jego}, \tit{niego} and \tit{tego} are syncretic between the masculine and the neuter. This can be captured if the highest feature in the lexical tree is the feature that distinguishes masculine and neuter gender.
This distinguishing feature is the feature \tsc{an} \citep{harley2002}, which is not the the highest feature in \ref{ex:pol-mu-dat-fseq}. Fortunately, different from \tit{jullie}, the forms are bimorphemic: they contain morphemes \tit{je}, \tit{nie} or \tit{te} and the morpheme \tit{mu}. The highest feature of one of the two morphemes needs to be the feature \tsc{an}.
I show that the highest feature of the lexical entry for \tit{je}, \tit{nie} and \tit{te} is \tsc{an}, as shown in \ref{ex:pol-entry-nie/je/te}.

\ex. \label{ex:pol-entry-nie/je/te}
\begin{forest} boom
  [\tsc{an}P
      [\tsc{an}]
      [\tsc{cl}P
          [\tsc{cl}]
          [XP
              [\phantom{xxx}, roof]
          ]
      ]
  ]
  {\draw (.east) node[right]{⇔ \tit{je/nie/te}}; }
\end{forest}

Since the feature \tsc{an} can be either present or absent, the forms can easily spell both the structure with or without the feature \tsc{an}.

In \ref{ex:pol-spellout-nie/je/te-an}, I give the syntactic structure of a masculine form.

\ex.\label{ex:pol-spellout-nie/je/te-an}
\begin{forest} boom
  [\tsc{an}P,
  tikz={
  \node[label=below:\tit{je/nie/te},
  draw,circle,
  scale=0.8,
  fit to=tree]{};
  }
      [\tsc{an}]
      [\tsc{cl}P
          [\tsc{cl}]
          [XP
              [\phantom{xxx}, roof]
          ]
      ]
  ]
\end{forest}

The syntactic structure forms a constituent within the lexical tree in \ref{ex:pol-entry-nie/je/te}, and the structure can be spelled out as \tit{je/nie/te}.

In \ref{ex:pol-spellout-nie/je/te-cl}, I give the syntactic structure of a masculine form.

\ex.\label{ex:pol-spellout-nie/je/te-cl}
\begin{forest} boom
  [\tsc{cl}P,
  tikz={
  \node[label=below:\tit{o},
  draw,circle,
  scale=0.8,
  fit to=tree]{};
  }
      [\tsc{cl}]
      [XP
          [\phantom{xxx}, roof]
      ]
  ]
\end{forest}

The syntactic structure forms a constituent within the lexical tree in \ref{ex:pol-entry-nie/je/te}, and the structure can be spelled out as \tit{je/nie/te}.

This means that the lexical tree for the suffix \tit{mu} should contain all features in \ref{ex:pol-elh-dat} that are not spelled out by \tit{je/nie/te} so far. These are the feature \tsc{ind} and all case features up to the dative. I give the lexical entry for \tit{mu} in \ref{ex:pol-entry-mu}.

\ex. \label{ex:pol-entry-mu}
\begin{forest} boom
  [\tsc{dat}P
      [\tsc{f}3]
      [\tsc{acc}P
          [\tsc{f}2]
          [\tsc{nom}P
              [\tsc{f}1]
              [\tsc{ind}P
                  [\tsc{ind}]
              ]
          ]
      ]
  ]
  {\draw (.east) node[right]{⇔ \tit{mu}}; }
\end{forest}

Notice here that \tit{mu} has a unary bottom. Therefore, it can be inserted as the result of movement. That means that the lexical entry follows the existing structure and is spelled out as a suffix. This is how the correct order of \tit{je/nie/te} and \tit{mu} comes about. I show how this works when I construct the extra light head later on in this section.

The morpheme \tit{go} differs from \tit{mu} in that it lacks the feature \tit{f}3. I give its lexical entry in \ref{ex:pol-entry-go}.

\ex. \label{ex:pol-entry-go}
\begin{forest} boom
  [\tsc{acc}P
      [\tsc{f}2]
      [\tsc{nom}P
          [\tsc{f}1]
          [\tsc{ind}P
              [\tsc{ind}]
          ]
      ]
  ]
  {\draw (.east) node[right]{⇔ \tit{go}}; }
\end{forest}

In sum, the morphemes \tit{go} and \tit{mu} spell out case features and a number feature.

This leaves the two morphemes \tit{k} and \tit{o}. First I discuss the \tit{o}.
This morpheme corresponds to what I called the phi-feature morpheme in Chapter \ref{ch:the-basic-idea} and the introduction to this chapter.
I show that this morpheme corresponds to pronominal features and gender features.

First I will show that the \tit{o} does not only appear in relative pronouns, but it also appears in other pronouns.
Consider the relative pronouns in Table \ref{tbl:pol-rps}.

\begin{table}[htbp]
  \center
  \caption{Polish (in)animate relative pronouns \pgcitep{swan2002}{160}}
  \begin{tabular}[b]{ccc}
    \toprule
              & \tsc{an}  & \tsc{inan} \\
    \cmidrule{2-3}
    \tsc{nom} & kto       & c-o        \\
    \tsc{acc} & k-o-go    & c-o        \\
    \tsc{gen} & k-o-go    & cz-e-go    \\
    \tsc{dat} & k-o-mu    & cz-e-mu    \\
    \tsc{ins} & k-i-m     & cz-y-m     \\
    \bottomrule
  \end{tabular}
  \label{tbl:pol-rps}
\end{table}

I leave the nominative and accusative aside from now and come back to them later.
From the genitive on, the final suffixes in the animate and the inanimate paradigm are the same.\footnote{
I include genitive and the instrumental in the paradigms to show that the patterns observed in the dative are not standing on themselves. Instead, they are more generally attested in Polish, and they deserve an explanation.
In Polish, the genitive comes between the accusative and the dative, i.e. it is more complex than the accusative and less complex than the dative. However, I do not incorporate them in the syntactic structures.
This does not change anything about the main point about case I want to make: the dative is more complex than the accusative.
}
The forms differ in their initial consonant and the vowel. The animates have a \tit{k} and an \tit{o} or \tit{i}, and the inanimates have a \tit{cz} and a \tit{e} or \tit{y}.

There are several ways to analyze this.
The first possibility is to not decompose the portion before the suffix. Under this analysis, Polish has the morphemes \tit{ko}, \tit{ki}, \tit{cze} and \tit{czy}. The point that is missed then is that the animates always have a \tit{k} and inanimates always have a \tit{cz}.

A second possibility that captures this observation is an analysis in which Polish has the morphemes \tit{k}, \tit{o}, \tit{i} and \tit{cz}, \tit{e} and \tit{y}.\footnote{
This is more or less what \citet{wiland2019} proposes.
}
What is not captured now is that numerous \tsc{wh}-elements in Polish start with a \tit{k}. I give some examples in \ref{ex:pol-wh-k}.\footnote{
The \tit{k} in \ref{ex:pol-gdzie} gets voiced into \tit{g} because it is followed by \tit{d}.
}

\ex.\label{ex:pol-wh-k}
\ag. k-tóry\\
 which\\
\bg. k-iedy\\
 when\\
\bg. g-dzie\\
 where\\\label{ex:pol-gdzie}
 \flushfill{Polish, \pgcitealt{swan2002}{180,183-184}}

Moreover, \tit{cz} is not a primary consonant in Polish but a derived one \pgcitep{swan2002}{23}. The consonants \tit{cz} and \tit{c} are derived from \tit{k}.

I propose that the \tit{k} is present in the inanimate relative pronouns. They appear as a consequence of being combined with an \tit{j}. I show the proposed decomposition in Table \ref{tbl:pol-rp-underl}.

\begin{table}[htbp]
  \center
  \caption{Polish (in)animate relative pronouns (underlying forms) \pgcitep{swan2002}{160}}
  \begin{tabular}[b]{ccc}
    \toprule
              & \tsc{an}  & \tsc{inam}  \\
    \cmidrule{2-3}
    \tsc{nom} & kto       & k-j-o       \\
    \tsc{acc} & k-o-go    & k-j-o       \\
    \tsc{gen} & k-o-go    & k-j-o-go    \\
    \tsc{dat} & k-o-mu    & k-j-o-mu    \\
    \tsc{ins} & k-i-m     & k-j-i-m     \\
    \bottomrule
  \end{tabular}
  \label{tbl:pol-rp-underl}
\end{table}

Under this analysis, Polish only has the morphemes \tit{k}, \tit{o} and \tit{i} that can be observed in the animate plus an \tit{j} that is present throughout the whole paradigm in the inanimate.\footnote{
There seems to be a contradition here: the inanimate is featurally speaking less complex than the animate  \citep[cf.]{harley2002}, but morphologically the inanimate is more complex than the animate: it contains the additional morpheme \tit{i}. I return to this seemingly contradition later in this section.
}
This hypothesizes that Polish relative pronouns have the underlying forms shown in Table \ref{tbl:pol-rps-underl-real}.

\begin{table}[htbp]
  \center
  \caption{Polish inanimate relative pronouns (underlying + surface forms) \pgcitep{swan2002}{160}}
  \begin{tabular}[b]{ccc}
    \toprule
              & underlying  & surface    \\
    \cmidrule{2-3}
    \tsc{nom} & k-i-o       &  c-o      \\
    \tsc{acc} & k-i-o       &  c-o      \\
    \tsc{gen} & k-i-o-go    &  cz-e-go  \\
    \tsc{dat} & k-i-o-mu    &  cz-e-mu  \\
    \tsc{ins} & k-i-i-m     &  cz-y-m   \\
    \bottomrule
  \end{tabular}
  \label{tbl:pol-rps-underl-real}
\end{table}

The sequence \tit{k-i-i} becomes \tit{czy} in the instrumental, and the sequence \tit{k-i-o} becomes \tit{cze} in the genitive and dative.
To get from the underlying form to the surface form, three phonological processes are taking place, which are all independently observed within Polish.


\footnote{
Under this analysis, \tit{czyj} `whose' is underlyingly \tit{k-i-i-j}.
}

\ex.\label{ex:pol-phon-j-vowel}
\a. /o/ → /e/ / /i/\underline{\hspace{0.6cm}}\label{ex:pol-phon-j-o}
\b. /i/ → /y/ / /i/\underline{\hspace{0.6cm}}\label{ex:pol-phon-j-i}

I continue with the change shown in \ref{ex:pol-phon-j-o}, in which the combination of /i/ and /i/ results in /y/.

\exg. walc-ik: walczyk\\
waltz-\tsc{dim}\\
(this is two in one) \flushfill{\pgcitealt{swan2002}{26}}

I continue with the change shown in \ref{ex:pol-phon-j-o}, in which the combination of /o/ and /i/ results in /e/.

\ex. moje-story


\footnote{
In the nominative and accusative only the change in \ref{ex:pol-phon-j-k} is happening. I assume that the \tit{o} in the nominative and accusative resists the process in \ref{ex:pol-phon-j-vowel}. This is because the \tit{o} in the nominative and accusative is a `different' \tit{o} than all the other ones. This \tit{o} namely spells out different features than the \tit{o}.
}

\begin{table}[htbp]
  \center
  \caption{Polish inanimate relative pronouns (underlying + surface forms) \pgcitep{swan2002}{160}}
  \begin{tabular}[b]{cc}
    \toprule
              & after first step  \\
    \cmidrule{2-2}
    \tsc{nom} & k-i-o           \\
    \tsc{acc} & k-i-o             \\
    \tsc{gen} & k-i-e-go      \\
    \tsc{dat} & k-i-e-mu      \\
    \tsc{ins} & k-i-y-m        \\
    \bottomrule
  \end{tabular}
  \label{tbl:pol-rps-step-1}
\end{table}

\ex.\label{ex:pol-phon-j-k}
\a. /k/ /i/ → /c/

I start with the change shown in \ref{ex:pol-phon-j-k}, in which the combination of /k/ and /i/ results in /c/. Consider the paradigm for the singular of \tit{lampa} `light' and the singular of \tit{córka} `daugther' in Table \ref{tbl:pol-jk-to-c}.

\begin{table}[htbp]
  \center
  \caption{Polish nouns \pgcitep{swan2002}{47,57}}
  \begin{tabular}[b]{ccc}
    \toprule
          & light.\tsc{sg} & daughter.\tsc{sg} \\
            \cmidrule{2-3}
\tsc{nom} & lamp-a         & córk-a            \\
\tsc{acc} & lamp-ę         & córk-ę            \\
\tsc{gen} & lamp-y         & córk-i            \\
\tsc{dat} & lamp-i-e       & córc-e            \\
\tsc{ins} & lamp-ą         & córk-ą            \\
  \bottomrule
  \end{tabular}
\label{tbl:pol-jk-to-c}
\end{table}

The stem and the suffixes are identical in both paradigms, except for in the dative.\footnote{
Notice also the change from /i/ to /y/ in the genitive of \tit{lampa}. Possibly, the underlying form is here \tit{lamp-i-i}, and this is an instance of the change given in \ref{ex:pol-phon-j-i}.
}
There, the stem of \tit{córka} does no longer end with a \tit{k}, but with a \tit{c}. Also, part of the suffix, namely the \tit{i} has disappeared. Analyzing \tit{córc-e} as \tit{córk-i-e} brings back regularity in the paradigm.

\begin{table}[htbp]
  \center
  \caption{Polish inanimate relative pronouns (underlying + surface forms) \pgcitep{swan2002}{160}}
  \begin{tabular}[b]{cc}
    \toprule
              & after second step  \\
    \cmidrule{2-2}
    \tsc{nom} & c-o           \\
    \tsc{acc} & c-o             \\
    \tsc{gen} & c-e-go      \\
    \tsc{dat} & c-e-mu      \\
    \tsc{ins} & c-y-m        \\
    \bottomrule
  \end{tabular}
  \label{tbl:pol-rps-step-2}
\end{table}

\ex.
\b. /c/ → /cz/ / \underline{\hspace{0.6cm}}/e/
\b. /c/ → /cz/ / \underline{\hspace{0.6cm}}/y/



Finally, there is the change of \tit{c} before \tit{e/y} to \tit{cz}.

\exg. ojc-e: ojcze\\
 father.\tsc{voc}\\
\flushfill{\pgcitealt{swan2002}{26}}


\exg. walc-ik: walczyk\\
 waltz-\tsc{dim}\\
 (this is two in one) \flushfill{\pgcitealt{swan2002}{26}}

In sum, the \tit{o} in relative pronoun can appear as an \tit{e} when it is palatalized.

The conclusion I draw from this is that the morpheme \tit{o} is not specific to animate relative pronouns, but it also appears elsewhere, for instance in inanimate relative pronouns, demonstratives and the \tit{ni}- and \tit{je}-pronouns. What these elements all have in common is that they are pronouns. Therefore, I assume that the \tit{o} spells out pronominal features. In addition, it also spells out gender features.

\ex. \label{ex:pol-entry-o}
\begin{forest} boom
  [\tsc{an}P
      [\tsc{an}]
      [\tsc{cl}P
          [\tsc{cl}]
          [\tsc{ref}]
      ]
  ]
  {\draw (.east) node[right]{⇔ \tit{o}}; }
\end{forest}

The question is which features \tit{k} spells out and which \tit{o}. We know now that \tit{e} is underlyingly \tit{j} + \tit{o}. How do \tit{tego} and \tit{kogo} differ? They are demonstrative vs. \tsc{wh}-element. \tit{k} does question, \tit{t} does \tsc{dem}. I actually assume that \tit{k} does \tsc{dem} too, but I leave that out here. This means that the \tit{o} needs to do `other stuff'. It is the base, so what I have referring to as phi-features.\footnote{
\tit{jego} is actually \tit{j} + \tit{ogo}
}



I discuss the features that I assume the \tit{k} spells out one by one.

I start with the operator features \tsc{wh} and \tsc{rel}. The relative pronouns are \tsc{wh}-pronouns, which are also used as interrogatives in Polish. Therefore, just like the Modern German \tit{w}, the Polish \tit{k} spells out the features \tsc{wh} and \tsc{rel}.

Finally, since the relative pronouns do not have a morphological plural, I assume that \tit{k} contains the feature \tsc{ind}.
Lastly, \tsc{k} also contains \tsc{an} and \tsc{cl}.

In sum, the morpheme \tit{k} realizes the features \tsc{wh}, \tsc{rel}, \tsc{ind}, \tsc{an} and \tsc{cl}.\footnote{
Actually, to be able to derive the inanimate relative pronoun, I assume that there is a pointer in the lexical entry for \tit{k}, as shown in \ref{ex:pol-entry-k-pointer}.

\ex.\label{ex:pol-entry-k-pointer}
\begin{forest} boom
  [\tsc{rel}P
      [\tsc{rel}]
      [\tsc{wh}P
          [\tsc{wh}]
          [\tsc{ind}P, edge=->
              [\tsc{ind}]
              [\tsc{an}P
                  [\tsc{an}]
                  [\tsc{cl}]
              ]
          ]
      ]
  ]
  {\draw (.east) node[right]{⇔ \tit{k}}; }
\end{forest}

The pointer is situated above the \tsc{ind}P. That means that if there is no animate feature in the structure, the \tsc{ind} can also not be spelled out with \tit{k}. What follows is that there is another morpheme necessary that contributes the feature \tsc{ind}. I propose that this is \tit{j} which forms strong pronouns. This is what causes the phonological processes described in Section \ref{sec:pol-elh}.
}

\ex.\label{ex:pol-entry-k}
\begin{forest} boom
  [\tsc{rel}P
      [\tsc{rel}]
      [\tsc{wh}P
          [\tsc{wh}]
          [\tsc{ind}P
              [\tsc{ind}]
              [\tsc{an}P
                  [\tsc{an}]
                  [\tsc{cl}]
              ]
          ]
      ]
  ]
  {\draw (.east) node[right]{⇔ \tit{k}}; }
\end{forest}

In what follows, I show how the Polish relative pronouns are constructed. I follow the same functional sequence as I did for Modern German. Also, of course, the spellout procedure is identical. The outcome is different because of the different lexical entries Polish has.

Starting from the bottom, the first two features that are merged \tsc{ref} and \tsc{cl}, creating a \tsc{cl}P.
The syntactic structure forms a constituent in the lexical tree in \ref{ex:pol-entry-o}, which corresponds to the \tit{o}.
Therefore, the \tsc{cl}P is spelled out as \tit{o}, which I do not show here.
Then, the feature \tsc{an} is merged, and a \tsc{an}P is created.
The syntactic structure forms a constituent in the lexical tree in \ref{ex:pol-entry-o}.
Therefore, the \tsc{an}P is spelled out as \tit{o}, shown in \ref{ex:pol-spellout-o-rel}.

\ex.\label{ex:pol-spellout-o-rel}
\begin{forest} boom
  [\tsc{an}P,
  tikz={
  \node[label=below:\tit{o},
  draw,circle,
  scale=0.9,
  fit to=tree]{};
  }
      [\tsc{an}]
      [\tsc{cl}P
          [\tsc{cl}]
          [\tsc{ref}]
      ]
  ]
\end{forest}

The next feature in the functional sequence is the feature \tsc{ind}. This feature cannot be spelled out as the other ones before. The feature \tsc{ind} is merged, and a \tsc{ind}P is created. This syntactic structure does not form a constituent in the lexical tree in \ref{ex:pol-entry-o}. There is also no other lexical tree that contains the syntactic structure as a constituent.
Therefore, there is no succesfull spellout for the syntactic structure in the derivational step in which the structure is spelled out as a single phrase (\ref{ex:spellout-algorithm-phrasal-rep} in the Spellout Algorithm, repeated from Chapter \ref{ch:deriving-onlyinternal}).

\ex. \tbf{Spellout Algorithm} (as in \citealt{caha2020a}, based on \citealt{starke2018})\label{ex:spellout-algorithm-rep}
 \a. Merge F and spell out.\label{ex:spellout-algorithm-phrasal-rep}
 \b. If (a) fails, move the Spec of the complement and spell out.\label{ex:spellout-algorithm-spec-rep}
 \b. If (b) fails, move the complement of F and spell out.\label{ex:spellout-algorithm-comp-rep}

The first movement option in the Spellout Algorithm is moving the specifier, as described in \ref{ex:spellout-algorithm-spec-rep}. As there is no specifier in this structure, so the first movement option irrelevant.
The second movement option in the Spellout Algorithm is moving the complement, as described in \ref{ex:spellout-algorithm-comp}. In this case, the complement of \tsc{ind}, the \tsc{an}P, is moved to the specifier of \tsc{ind}P. I show this movement in \ref{ex:pol-movement}.

\ex.\label{ex:pol-movement}
\begin{forest} boom
  [\tsc{ind}P
      [\tsc{an}P,name=tgt
          [\phantom{x}\tit{o}\phantom{x}, roof]
      ]
      [\tsc{ind}P
          [\tsc{ind}]
          [\sout{\tsc{an}P}, name=src,
          tikz={
          \node[label=below:\tit{o},
          draw,circle,
          scale=0.9,
          fit to=tree]{};
          }
              [\tsc{an}]
              [\tsc{cl}P
                  [\tsc{cl}]
                  [\tsc{ref}]
              ]
          ]
      ]
  ]
\draw[->,dashed] (src) to[out=south west,in=east] (tgt);
\end{forest}

The \tsc{ind}P is a different constituent now. It still contains the feature \tsc{ind}, but it no longer contains the \tsc{an}P. The syntactic structure forms a constituent in the lexical tree of \ref{ex:pol-entry-go}.
Therefore, the \tsc{ind}P is spelled out as \tit{go}, as shown in \ref{ex:pol-spellout-o-ind}.

\ex.\label{ex:pol-spellout-o-ind}
\begin{forest} boom
  [\tsc{ind}P
  [\tsc{an}P
      [\phantom{x}\tit{o}\phantom{x}, roof]
  ]
      [\tsc{ind}P,
      tikz={
      \node[label=below:\tit{go},
      draw,circle,
      scale=0.95,
      fit to=tree]{};
      }
          [\tsc{ind}]
      ]
  ]
\end{forest}

Next, the feature \tsc{wh} is merged.
The derivation for this feature resembles the derivation of \tsc{wh} in Modern German.
The feature is merged with the existing syntactic structure, creating a \tsc{wh}P.
This structure does not form a constituent in any of the lexical trees in the language's lexicon, and neither of the spellout driven movements leads to a successful spellout.
Therefore, in a second workspace, the feature \tsc{wh} is merged with the feature \tsc{ind} (the previous syntactic feature on the functional sequence) into a \tsc{wh}P. This syntactic structure does not form a constituent in any of the lexical trees in the language's lexicon.
Therefore, the feature \tsc{wh} combines not only with the feature merged before it, but with a phrase that consists of the two features merged before it: \tsc{ind} and \tsc{an}. Also this syntactic structure does not form a constituent in any of the lexical trees in the language's lexicon.
Therefore, the feature \tsc{wh} combines with a phrase that consists of the three features merged before it: \tsc{ind}, \tsc{an} and \tsc{cl}. This syntactic structure forms a constituent in the lexical tree in \ref{ex:pol-entry-k}, which corresponds to the \tit{k}.
Therefore, the \tsc{wh}P is spelled out as \tit{k}. The newly created phrase is merged as a whole with the already existing structure, and projects to the top node, as shown in \ref{ex:pol-spellout-whp}.

\ex.\label{ex:pol-spellout-whp}
\begin{adjustbox}{max width=0.9\textwidth}
\begin{forest} boom
  [\tsc{wh}P, s sep=30mm
      [\tsc{wh}P,
      tikz={
      \node[label=below:\tit{k},
      draw,circle,
      scale=0.95,
      fit to=tree]{};
      }
          [\tsc{wh}]
          [\tsc{ind}P
              [\tsc{ind}]
              [\tsc{an}P
                  [\tsc{an}]
                  [\tsc{cl}]
              ]
          ]
      ]
      [\tsc{ind}P, s sep=30mm
      [\tsc{an}P,
          tikz={
          \node[label=below:\tit{o},
          draw,circle,
          scale=0.95,
          fit to=tree]{};
          }
          [\tsc{an}P]
          [\tsc{cl}P
              [\tsc{cl}]
              [\tsc{ref}]
          ]
      ]
          [\tsc{ind}P,
          tikz={
          \node[label=below:\tit{go},
          draw,circle,
          scale=0.9,
          fit to=tree]{};
          }
              [\tsc{ind}]
          ]
      ]
  ]
\end{forest}
\end{adjustbox}

The next feature in the functional sequence is the feature \tsc{rel}. The derivation for this feature resembles the derivation of \tsc{rel} in Modern German.
The feature is merged with the existing syntactic structure, creating a \tsc{rel}P.
This structure does not form a constituent in any of the lexical trees in the language's lexicon, and neither of the spellout driven movements leads to a successful spellout.
Backtracking leads splitting up the \tsc{wh}P from the \tsc{ind}P.
The feature \tsc{rel} is merged in both workspaces, so with \tsc{wh}P and and with \tsc{ind}P. The spellout of \tsc{rel} is successful when it is combined with the \tsc{wh}P.
It namely forms a constituent in the lexical tree in \ref{ex:pol-entry-k}, which corresponds to the \tit{k}.
The \tsc{rel}P is spelled out as \tit{k}, and the \tsc{rel}P is merged back to the existing syntactic structure.

The next feature on the functional sequence is \tsc{f}1. This feature should somehow end up merging with \tsc{ind}P, because it forms a constituent in the lexical tree in \ref{ex:pol-entry-go-rep}, repeated from \ref{ex:pol-entry-go}, which corresponds to the \tit{go}.

\ex. \label{ex:pol-entry-go-rep}
\begin{forest} boom
  [\tsc{acc}P
      [\tsc{f}2]
      [\tsc{nom}P
          [\tsc{f}1]
          [\tsc{ind}P
              [\tsc{ind}]
          ]
      ]
  ]
  {\draw (.east) node[right]{⇔ \tit{go}}; }
\end{forest}

This is achieved via Backtracking in which phrases are split up and going through the Spellout Algorithm. I go through the derivation step by step.
The feature \tsc{f}1 is merged with the existing syntactic structure, creating a \tsc{nom}P.
This structure does not form a constituent in any of the lexical trees in the language's lexicon, and neither of the spellout driven movements leads to a successful spellout.
Backtracking leads splitting up the \tsc{rel}P from the \tsc{ind}P.
The feature \tsc{f}1 is merged in both workspaces, so with the \tsc{rel}P and and with the \tsc{ind}P. None of these phrases form a constituent in any of the lexical trees in the language's lexicon.
The first movement option in the Spellout Algorithm is moving the specifier. In the \tsc{rel}P there is no specifier, so this movement option is irrelevant. In the \tsc{ind}P, however, there is a specifier, which is moved to the specifier of \tsc{nom}P.
This syntactic structure forms a constituent in the lexical tree in \ref{ex:pol-entry-go-rep}, which corresponds to the \tit{go}.
The \tsc{nom}P is spelled out as \tit{go}, and the \tsc{nom}P is merged back to the existing syntactic structure.

For the accusative relative pronoun, the last feature on the functional sequence is the feature \tsc{f}2. Its derivation proceeds the same as the one for the feature \tsc{f}1.
The feature \tsc{f}2 is merged with the existing syntactic structure, creating a \tsc{acc}P.
This structure does not form a constituent in any of the lexical trees in the language's lexicon, and neither of the spellout driven movements leads to a successful spellout.
Backtracking leads splitting up the \tsc{rel}P from the \tsc{nom}P.
The feature \tsc{f}2 is merged in both workspaces, so with the \tsc{rel}P and and with the \tsc{nom}P. None of these phrases form a constituent in any of the lexical trees in the language's lexicon.
The first movement option in the Spellout Algorithm is moving the specifier. In the \tsc{rel}P there is no specifier, so this movement option is irrelevant. In the \tsc{nom}P, however, there is a specifier, which is moved to the specifier of \tsc{acc}P.
This syntactic structure forms a constituent in the lexical tree in \ref{ex:pol-entry-go-rep}, which corresponds to the \tit{go}.
The \tsc{acc}P is spelled out as \tit{go}, and the \tsc{acc}P is merged back to the existing syntactic structure, as shown in \ref{ex:pol-spellout-rel-acc}.

\ex.\label{ex:pol-spellout-rel-acc}
\begin{adjustbox}{max width=0.9\textwidth}
\begin{forest} boom
  [\tsc{rel}P, s sep=40mm
      [\tsc{rel}P,
      tikz={
      \node[label=below:\tit{k},
      draw,circle,
      scale=0.95,
      fit to=tree]{};
      }
          [\tsc{rel}]
          [\tsc{wh}P
              [\tsc{wh}]
              [\tsc{ind}P
                  [\tsc{ind}]
                  [\tsc{an}P
                      [\tsc{an}]
                      [\tsc{cl}]
                  ]
              ]
          ]
      ]
      [\tsc{acc}P, s sep=30mm
      [\tsc{an}P,
          tikz={
          \node[label=below:\tit{o},
          draw,circle,
          scale=0.95,
          fit to=tree]{};
          }
          [\tsc{an}P]
          [\tsc{cl}P
              [\tsc{cl}]
              [\tsc{ref}]
          ]
      ]
          [\tsc{acc}P,
          tikz={
          \node[label=below:\tit{go},
          draw,circle,
          scale=0.95,
          fit to=tree]{};
          }
              [\tsc{f}2]
              [\tsc{nom}P
                  [\tsc{f}1]
                  [\tsc{ind}P
                      [\tsc{ind}]
                  ]
              ]
          ]
      ]
  ]
\end{forest}
\end{adjustbox}

For the accusative relative pronoun, the last feature on the functional sequence is the feature \tsc{f}3. Its derivation proceeds the same as the one for the feature \tsc{f}2.
The feature \tsc{f}3 is merged with the existing syntactic structure, creating a \tsc{dat}P.
This structure does not form a constituent in any of the lexical trees in the language's lexicon, and neither of the spellout driven movements leads to a successful spellout.
Backtracking leads splitting up the \tsc{rel}P from the \tsc{acc}P.
The feature \tsc{f}3 is merged in both workspaces, so with the \tsc{rel}P and and with the \tsc{acc}P. None of these phrases form a constituent in any of the lexical trees in the language's lexicon.
The first movement option in the Spellout Algorithm is moving the specifier. In the \tsc{rel}P there is no specifier, so this movement option is irrelevant. In the \tsc{acc}P, however, there is a specifier, which is moved to the specifier of \tsc{dat}P.
This syntactic structure forms a constituent in the lexical tree in \ref{ex:pol-entry-mu-rep}, which corresponds to the \tit{mu}.

\ex. \label{ex:pol-entry-mu-rep}
\begin{forest} boom
  [\tsc{dat}P
      [\tsc{f}3]
      [\tsc{acc}P
          [\tsc{f}2]
          [\tsc{nom}P
              [\tsc{f}1]
              [\tsc{ind}P
                  [\tsc{ind}]
              ]
          ]
      ]
  ]
  {\draw (.east) node[right]{⇔ \tit{mu}}; }
\end{forest}

The \tsc{dat}P is spelled out as \tit{mu}, and the \tsc{dat}P is merged back to the existing syntactic structure, as shown in \ref{ex:pol-spellout-rel-dat}.

\ex.\label{ex:pol-spellout-rel-dat}
\begin{adjustbox}{max width=0.9\textwidth}
\begin{forest} boom
  [\tsc{rel}P, s sep=40mm
      [\tsc{rel}P,
      tikz={
      \node[label=below:\tit{k},
      draw,circle,
      scale=0.95,
      fit to=tree]{};
      }
          [\tsc{rel}]
          [\tsc{wh}P
              [\tsc{wh}]
              [\tsc{ind}P
                  [\tsc{ind}]
                  [\tsc{an}P
                      [\tsc{an}]
                      [\tsc{cl}]
                  ]
              ]
          ]
      ]
      [\tsc{dat}P, s sep=35mm
      [\tsc{an}P,
          tikz={
          \node[label=below:\tit{o},
          draw,circle,
          scale=0.95,
          fit to=tree]{};
          }
          [\tsc{an}P]
          [\tsc{cl}P
              [\tsc{cl}]
              [\tsc{ref}]
          ]
      ]
          [\tsc{dat}P,
          tikz={
          \node[label=below:\tit{mu},
          draw,circle,
          scale=0.95,
          fit to=tree]{};
          }
              [\tsc{f}3]
              [\tsc{acc}P
                  [\tsc{f}2]
                  [\tsc{nom}P
                      [\tsc{f}1]
                      [\tsc{ind}P
                          [\tsc{ind}]
                      ]
                  ]
              ]
          ]
      ]
  ]
\end{forest}
\end{adjustbox}

To summarize, I decomposed the relative pronoun into the three morphemes \tit{k}, \tit{o} and the suffix (\tit{go} and \tit{mu}). I showed which features each of the morphemes spells out, and in which constituents the features are combined. It is these constituents that determine whether the light head can be deleted or not.





\section{The Polish extra light head}\label{sec:pol-elh}

I have suggested that headless relatives are derived from light-headed relatives. The light head or the relative pronoun can be deleted when either of them forms a constituent within other. In the introduction of this chapter, I claimed that in Polish features of the light head are spelled out in such a way that they form the constituency shown in \ref{ex:pol-lh-complex}.

\ex.\label{ex:pol-lh-complex}
\begin{forest} boom
  [\tsc{k}P, s sep = 15mm
      [\tsc{an}P,
      tikz={
      \node[label=below:\tit{o},
      draw,circle,
      scale=0.75,
      fit to=tree]{};
      }
          [\phantom{xxx}, roof, baseline]
      ]
      [\tsc{k}P,
      tikz={
      \node[label=below:\tit{go/mu},
      draw,circle,
      scale=0.85,
      fit to=tree]{};
      }
          [\tsc{k}]
          [\tsc{ind}P
              [\tsc{ind}]
          ]
      ]
  ]
\end{forest}

\ref{ex:pol-lh-complex} shows that light heads consist of at least two features: ϕ and \tsc{k}, as I suggested in Chapter \ref{ch:the-basic-idea}. In this section, I specify the feature content of the light head in more detail.
Just like I suggested in Chapter \ref{ch:deriving-onlyinternal}, I end up claiming that the phi- and case-feature morphemes of the relative pronoun are the light head in headless relatives. I show the complete structure that I work towards in this section in \ref{ex:pol-elh}.

\ex.\label{ex:pol-elh}
\begin{adjustbox}{max width=0.9\textwidth}
\begin{forest} boom
  [\tsc{k}P, s sep=40mm
      [\tsc{an}P,
      tikz={
      \node[label=below:\tit{o},
      draw,circle,
      scale=0.95,
      fit to=tree]{};
      }
          [\tsc{an}]
          [\tsc{cl}P
              [\tsc{cl}]
              [\tsc{ref}]
          ]
      ]
      [\tsc{k}P,
      tikz={
      \node[label=below:\tit{go/mu},
      draw,circle,
      scale=0.9,
      fit to=tree]{};
      }
          [\tsc{k}]
          [\tsc{ind}P
              [\tsc{ind}]
          ]
      ]
  ]
\end{forest}
\end{adjustbox}

%
% In this section I argue that the extra light head in Polish is consists of the \tit{o} and a suffix. This means that I propose the light-headed relative that headless are derived from is of the type I show in \ref{ex:pol-elh-rp}
%
% \exg. Jan lubi [ogo] \tbf{kogo} \tbf{-kolkwiek} \tbf{Maria} \tbf{lubi}.\\
% Jan like.\tsc{3sg}\scsub{[acc]} \tsc{elh}.\tsc{acc}.\tsc{an} \tsc{rp}.\tsc{acc}.\tsc{an} ever Maria like.\tsc{3sg}\scsub{[acc]}\\
% `Jan likes whoever Maria likes.' \flushfill{Polish, adapted from \citealt{citko2013} after \pgcitealt{himmelreich2017}{17}}\label{ex:pol-elh-rp}

For Modern German, I considered two kinds of light-headed relatives as the source of the headless relative.
First, the light-headed relative is derived from an existing light-headed relative, and the deletion of the light head is optional. Second, the light-headed relative is derived from a light-headed relative that does not surfaces in Modern German, and the deletion of the light head is obligatory.
For Modern German I concluded it was the second, and I proposed which features this extra light head should consist of. This set of features in Polish corresponds to the extra light head \tit{ogo} or \tit{omu}, which is not attested as a light head in an existing light-headed relative in Polish.

In the rest of this section I consider the existing Polish light-headed relative that could potentially be the source for headless relatives. This is the light-headed relative that in which the demonstrative is the light head, as shown in \ref{ex:pol-light-headed}.

\exg. Jan śpiewa to, co Maria śpiewa.\\
Jan sings \tsc{dem}.\tsc{m}.\tsc{sg}.\tsc{acc} \tsc{rp}.\tsc{an}.\tsc{acc} Maria sings\\
`John sings what Mary sings.' \flushfill{Polish, \pgcitealt{citko2004}{103}}\label{ex:pol-light-headed}

For Modern German, I gave two arguments for not taking this existing light-headed relative as source of the headless relative. In what follows, I show that these arguments hold for Polish in the same way do for Modern German.

First, in headless relatives the morpheme \tit{kolwiek} `ever' can appear, as shown in \ref{ex:pol-headless-ever}.

\exg. Jan śpiewa co -kolwiek Maria śpiewa.\\
Jan sings \tsc{rp}.\tsc{an}.\tsc{acc} ever Maria sings\\
`Jan sings everything Maria sings.' \flushfill{Polish, \pgcitealt{citko2004}{116}}\label{ex:pol-headless-ever}

Light-headed relatives do not allow this morpheme to be inserted, illustrated in \ref{ex:pol-headed-ever}.

\exg. *Jan śpiewa to, co -kolwiek Maria śpiewa.\\
Jan sings \tsc{dem}.\tsc{m}.\tsc{sg}.\tsc{acc} \tsc{rp}.\tsc{an}.\tsc{acc} ever Maria sings\\
`John sings what Mary sings.' \flushfill{Polish, \pgcitealt{citko2004}{116}}\label{ex:pol-headed-ever}

Just like for Modern German, I assume that the headless relative is not derived from an ungrammatical structure.\footnote{
\citet{citko2004} takes the complementary distribution of \tit{kolwiek} `ever' and the light head to mean that they share the same syntactic position. I have nothing to say about the exact syntactic position of \tit{ever}, but in my account it cannot be the head of the relative clause, as this position is reserved for the extra light head. My reason for the incompatibility of \tit{ever} and the light head is that they are semantically incompatible.

For concreteness, I assume \tit{ever} to be situated within the relative clause. Placing it in the main clause generates a different meaning, illustrated by the contrast in meaning between \ref{ex:cz-wh-ever} and \ref{ex:cz-ever-wh} in Czech.

\ex.
\ag. Sním, co -koliv mi uvaříš.\\
 eat.\tsc{1}sg what ever I.\tsc{dat} cook.2\tsc{sg}\\
 `I will eat whatever you will cook for me.'\label{ex:cz-wh-ever}
\bg. Sním co -koliv, co mi uvaříš.\\
 eat.\tsc{1}sg what ever what I.\tsc{dat} cook.2\tsc{sg}\\
 `I will eat anything that you will cook for me.' \flushfill{Czech, \pgcitealt{simik2016}{115}}\label{ex:cz-ever-wh}

\phantom{x}
}

The second argument against the existing light-headed relatives being the source of headless relatives comes from their interpretation. Headless relatives have two possible interpretations, and light-headed relatives have only one of these.
Just like in Modern German, Polish headless relatives can be analyzed as either universal or definite \pgcitep{citko2004}{103}.
Light-headed relatives, such as the one in \ref{ex:pol-light-headed}, only have the definite interpretation.






The two morphemes that spell out the features in \ref{ex:fseq-wh-lh-pol} are the vowel and the suffix, as shown in \ref{ex:pol-elhs}.

\ex.\label{ex:pol-elhs}
\ag. o-go\\
 \tsc{elh}.\tsc{an}.\tsc{acc}\\
\bg. o-mu\\
 \tsc{elh}.\tsc{an}.\tsc{dat}\\

Specifically, I argue that Polish light heads have the structure shown in \ref{ex:pol-elh}.

\ex.\label{ex:pol-elh}
\begin{adjustbox}{max width=0.9\textwidth}
\begin{forest} boom
  [\tsc{dat}P, s sep=40mm
      [\tsc{an}P,
      tikz={
      \node[label=below:\tit{o},
      draw,circle,
      scale=0.95,
      fit to=tree]{};
      }
          [\tsc{an}]
          [\tsc{cl}P
              [\tsc{cl}]
              [\tsc{ref}]
          ]
      ]
      [\tsc{k}P,
      tikz={
      \node[label=below:\tit{go/mu},
      draw,circle,
      scale=0.9,
      fit to=tree]{};
      }
          [\tsc{k}]
          [\tsc{ind}P
              [\tsc{ind}]
          ]
      ]
  ]
\end{forest}
\end{adjustbox}

Crucially, the constituent structure in \ref{ex:pol-elh} is the same as in \ref{ex:simple-matching}. Recall from Chapter \ref{ch:deriving-onlyinternal} that in Modern German the extra light head spells out as a single constituent. The Polish extra light head consists of two constituents, as shown in \ref{ex:pol-elh}. This is the crucial difference between the two languages that leads them to be of different types in headless relatives.

Before I discuss the feature content of the morphemes of the extra light head, I show that the two morphemes spell out the features in \ref{ex:fseq-wh-lh-pol}. Therefore, I need to discuss the \tit{o} of the relative pronoun. The \tit{o} can namely become an \tit{e} under particular phonological circumstances. This allows me to show that \tit{ogo} and \tit{omu} are the extra light heads in Polish.


I repeat the functional sequence of the extra light head in \ref{ex:fseq-wh-lh-pol-rep}.

\ex. \begin{forest} boom
  [\tsc{k}P
      [\tsc{k}]
      [\tsc{ind}P
          [\tsc{ind}]
          [\tsc{an}P
              [\tsc{an}]
              [\tsc{cl}P
                  [\tsc{cl}]
                  [\tsc{ref}]
              ]
          ]
      ]
  ]
\end{forest}
\label{ex:fseq-wh-lh-pol-rep}

The strong pronoun in Table \ref{tbl:pol-prons} differs from the extra light head in one feature: \tsc{c}. As each morpheme needs to spell out at least one feature, I assume that \tit{j} spells out this \tsc{c}.\footnote{
Actually, I assume it spells out \tsc{c} and \tsc{ind} as it is a Complex Spec. As this is not crucial to the point I am making, I do not discuss it any further for now.
}\footnote{
It is also possible that the strong pronoun is syncretic with the extra light head and that the extra light head is actually also spelled out as \tit{jego}/\tit{jemu}. This would mean that the strong extra light head consists of even more than two morphemes. For my proposal, it is important to show that the extra light head consists of at least two morphemes, one of which spells out case features. It works equally well when the non-case part of the structure actually consists of more one morpheme. I continue working out a proposal in which the extra light head is bimorphemic.
}
This leaves the functional sequence in \ref{ex:fseq-wh-lh-pol-rep} to be spelled out by the morphemes \tit{o} and \tit{go}/\tit{mu}.

In what follows, I discuss which features they \tit{o} and the suffixes (\tit{go} and \tit{mu}) spell out, and I give their lexical entries. At the end of the section, I show how the extra light heads are constructed.


In what follows, I construct the Polish extra light heads. Until the feature \tsc{ind}, the derivation is identical to the one of the relative pronoun.

%the situation is:

Therefore, I start at point at which the feature \tsc{f}1 is merged. The feature \tsc{f}1 is merged with the \tsc{ind}P, forming an \tsc{nom}P. This phrase is not contained in any of the lexical entries Polish. The first movement is tried, and the specifier of the \tsc{ind}P, the \tsc{an}P, is moved to the specifier of \tsc{nom}P. This phrase is contained in the lexical tree in \ref{ex:pol-entry-go}, so it is spelled out as \tit{go}.

For the accusative extra light head, the last feature is merged: the \tsc{f}2.
The feature is merged with the \tsc{nom}P, forming an \tsc{acc}P. This phrase is not contained in any of the lexical entries. The first movement is tried, and the specifier of the \tsc{nom}P, the \tsc{an}P, is moved to the specifier of \tsc{acc}P. This phrase is contained in the lexical tree in \ref{ex:pol-entry-go}, so it is spelled out as \tit{go}, as shown in \ref{ex:pol-elh-acc}.

\ex.\label{ex:pol-elh-acc}
\begin{adjustbox}{max width=0.9\textwidth}
\begin{forest} boom
  [\tsc{dat}P, s sep=40mm
      [\tsc{an}P,
      tikz={
      \node[label=below:\tit{o},
      draw,circle,
      scale=0.9,
      fit to=tree]{};
      }
          [\tsc{an}]
          [\tsc{cl}P
              [\tsc{cl}]
              [\tsc{ref}]
          ]
      ]
      [\tsc{acc}P,
      tikz={
      \node[label=below:\tit{go},
      draw,circle,
      scale=0.9,
      fit to=tree]{};
      }
          [\tsc{f}2]
          [\tsc{nom}P
              [\tsc{f}1]
              [\tsc{ind}P
                  [\tsc{ind}]
              ]
          ]
      ]
  ]
\end{forest}
\end{adjustbox}

For the dative relative pronoun, one more feature is merged: the \tsc{f}3.
The feature  is merged with the \tsc{acc}P, forming an \tsc{dat}P. This phrase is not contained in any of the lexical entries. The first movement is tried, and the specifier of the \tsc{acc}P, the \tsc{an}P, is moved to the specifier of \tsc{dat}P.
This phrase is contained in the lexical tree in \ref{ex:pol-entry-mu}, so it is spelled out as \tit{mu}, as shown in \ref{ex:pol-elh-dat}.

\ex.\label{ex:pol-elh-dat}
\begin{adjustbox}{max width=0.9\textwidth}
\begin{forest} boom
  [\tsc{dat}P, s sep=45mm
      [\tsc{an}P,
      tikz={
      \node[label=below:\tit{o},
      draw,circle,
      scale=0.95,
      fit to=tree]{};
      }
          [\tsc{an}]
          [\tsc{cl}P
              [\tsc{cl}]
              [\tsc{ref}]
          ]
      ]
      [\tsc{dat}P,
      tikz={
      \node[label=below:\tit{mu},
      draw,circle,
      scale=0.95,
      fit to=tree]{};
      }
          [\tsc{f}3]
          [\tsc{acc}P
              [\tsc{f}2]
              [\tsc{nom}P
                  [\tsc{f}1]
                  [\tsc{ind}P
                      [\tsc{ind}]
                  ]
              ]
          ]
      ]
  ]
\end{forest}
\end{adjustbox}

In sum, just like Modern German, Polish headless relatives do not seem to be derived from light-headed relatives in which the light head is a demonstrative. A difference between Polish and Modern German demonstratives is that Polish ones do not spell out definite features. The fact that Polish demonstratives are also not the light head of a headless relative confirm that deixis features have to be absent from the extra light head.



\section{Comparing Polish constituents}\label{sec:comparing-polish}

In this section, I compare the constituents of extra light heads to those of relative pronouns in Polish. This is the worked out version of the comparisons in Section \ref{sec:basic-matching}. What is different here is that I show the comparison for Polish specifically, and that I motivated the content of the constituents that are being compared.

I give three examples, in which the internal and external case vary.
I start with an example with matching cases, in which the internal and the external case are both accusative.
Then I give an example in which the internal dative case is more complex than the external accusative case.
I end with an example in which the external dative case is more complex than the internal accusative case.
I show that the first examples is grammatical and that the last two are not. I derive this by showing that only in the first situation the light head forms a constituent within the relative pronoun, and that it can therefore then be deleted.

I start with the matching cases.
Consider the example in \ref{ex:polish-acc-acc-rep}, in which the internal accusative case competes against the external accusative case. The relative clause is marked in bold.
The internal case is accusative, as the predicate \tit{lubić} `to like' takes accusative objects. The relative pronoun \tit{kogo} `\ac{rel}.\ac{an}.\ac{acc}' appears in the accusative case. This is the element that surfaces.
The external case is accusative as well, as the predicate \tit{lubić} `to like' also takes accusative objects. The extra light head \tit{ogo} `\ac{elh}.\ac{an}.\ac{acc}' appears in the accusative case. It is placed between square brackets because it does not surface.

\exg. Jan lubi [ogo] \tbf{kogo} \tbf{-kolkwiek} \tbf{Maria} \tbf{lubi}.\\
 Jan like.\tsc{3sg}\scsub{[acc]} \tsc{dem}.\tsc{acc}.\tsc{an}.\tsc{sg}  \tsc{rp}.\tsc{acc}.\tsc{an} ever Maria like.\tsc{3sg}\scsub{[acc]}\\
 `Jan likes whoever Maria likes.' \flushfill{Polish, adapted from \citealt{citko2013} after \pgcitealt{himmelreich2017}{17}}\label{ex:polish-acc-acc-rep}

In Figure \ref{fig:polish-int=ext}, I give the syntactic structure of the extra light head at the top and the syntactic structure of the relative pronoun at the bottom.

\begin{figure}[htbp]
  \center
  \begin{adjustbox}{max height=0.9\textheight}
  \begin{tabular}[b]{c}
        \toprule
        \tsc{acc} extra light head \tit{} \\
        \cmidrule{1-1}
        \begin{forest} boom
          [\tsc{acc}P, s sep=20mm,
          tikz={
          \node[
          draw, circle,
          fill=DG,fill opacity=0.2,
          scale=0.95,
          yshift=-0.5cm,
          dashed,
          fit to=tree]{};
          }
              [\tsc{an}P
                  [\phantom{x}o\phantom{x}, roof]
              ]
              [\tsc{acc}P,
              tikz={
              \node[label=below:\tit{go},
              draw,circle,
              scale=0.9,
              fit to=tree]{};
              }
                  [\tsc{f}2]
                  [\tsc{nom}P
                      [\tsc{f}1]
                      [\tsc{ind}P
                          [\tsc{ind}]
                      ]
                  ]
              ]
          ]
        \end{forest}
        \vspace{0.3cm}
      \\
      \toprule
      \tsc{acc} relative pronoun \tit{k-o-go}
      \\
      \cmidrule{1-1}
      \begin{forest} boom
        [\tsc{rel}P, s sep=15mm
            [\tsc{rel}P
                [\phantom{x}k\phantom{x}, roof]
            ]
            [\tsc{acc}P, s sep=20mm,
            tikz={
            \node[
            draw, circle,
            scale=0.95,
            yshift=-0.5cm,
            dashed,
            fit to=tree]{};
            }
                [\tsc{an}P
                    [\phantom{x}o\phantom{x}, roof]
                ]
                [\tsc{acc}P,
                tikz={
                \node[label=below:\tit{go},
                draw,circle,
                scale=0.9,
                fit to=tree]{};
                }
                    [\tsc{f}2]
                    [\tsc{nom}P
                        [\tsc{f}1]
                        [\tsc{ind}P
                            [\tsc{ind}]
                        ]
                    ]
                ]
            ]
        ]
      \end{forest}
      \vspace{0.3cm}
      \\
      \bottomrule
  \end{tabular}
  \end{adjustbox}
   \caption {Polish \tsc{ext}\scsub{acc} vs. \tsc{int}\scsub{acc} → \tit{kogo}}
  \label{fig:polish-int=ext}
\end{figure}

The relative pronoun consists of three morphemes: \tit{k}, \tit{o} and \tit{go}.
The extra light head consists of two morphemes: \tit{o} and \tit{go}.
As usual, I circle the part of the structure that corresponds to a particular lexical entry, and I place the corresponding phonology under it, or I reduce the structure to a triangle, and I place the corresponding phonology under it.
I draw a dashed circle around each constituent that is a constituent in both the extra light head and the relative pronoun.

The extra light head consists of two constituents: the \tsc{an}P and the (lower) \tsc{acc}P. Together they form the (higher) \tsc{acc}P.
This \tsc{acc}P is also a constituent within the relative pronoun. Therefore, the extra light head can be deleted. I signal the deletion of the extra light head by marking the content of its circle gray.

I continue with the example in which the internal case is more complex than the external case.
Consider the examples in \ref{ex:polish-acc-dat-rep}, in which the internal dative case competes against the external accusative case. The relative clauses are marked in bold. It is not possible to make a grammatical headless relative in this situation.
The internal case is dative, as the predicate \tit{dokuczać} `to tease' takes dative objects. The relative pronoun \tit{komu} `\ac{rel}.\ac{an}.\ac{dat}' appears in the dative case.
The external case is accusative, as the predicate \tit{lubić} `to like' takes accusative objects. The extra light head \tit{ogo} `\ac{elh}.\ac{an}.\ac{acc}' appears in the accusative case.
\ref{ex:polish-acc-dat-rel} is the variant of the sentence in which the extra light head is absent (indicated by the square brackets) and the relative pronoun surfaces, and it is ungrammatical.
\ref{ex:polish-acc-dat-lh} is the variant of the sentence in which the relative pronoun is absent (indicated by the square brackets) and the extra light head surfaces, and it is ungrammatical too.

\ex.\label{ex:polish-acc-dat-rep}
\ag. *Jan lubi [ogo] \tbf{komu} \tbf{-kolkwiek} \tbf{dokucza}.\\
Jan like.\tsc{3sg}\scsub{[acc]} \tsc{elh}.\tsc{acc}.\tsc{an} \tsc{rp}.\tsc{dat}.\tsc{an}.\tsc{sg} ever tease.\tsc{3sg}\scsub{[dat]}\\
`Jan likes whoever he teases.' \flushfill{Polish, adapted from \citealt{citko2013} after \pgcitealt{himmelreich2017}{17}}\label{ex:polish-acc-dat-rel}
\bg. *Jan lubi ogo [\tbf{komu}] \tbf{-kolkwiek} \tbf{dokucza}.\\
Jan like.\tsc{3sg}\scsub{[acc]} \tsc{elh}.\tsc{acc}.\tsc{an} \tsc{rp}.\tsc{dat}.\tsc{an}.\tsc{sg} ever tease.\tsc{3sg}\scsub{[dat]}\\
`Jan likes whoever he teases.' \flushfill{Polish, adapted from \citealt{citko2013} after \pgcitealt{himmelreich2017}{17}}\label{ex:polish-acc-dat-lh}

In Figure \ref{fig:polish-int-wins}, I give the syntactic structure of the extra light head at the top and the syntactic structure of the relative pronoun at the bottom.

\begin{figure}[htbp]
  \center
  \begin{adjustbox}{max height=0.9\textheight}
  \begin{tabular}[b]{c}
        \toprule
        \tsc{acc} extra light head \tit{o-go} \\
        \cmidrule{1-1}
        \begin{forest} boom
          [\tsc{acc}P, s sep=20mm
              [\tsc{an}P,
              tikz={
              \node[
              draw,circle,
              scale=0.85,
              dashed,
              fit to=tree]{};
              }
                  [\phantom{x}o\phantom{x}, roof]
              ]
              [\tsc{acc}P,
              tikz={
              \node[label=below:\tit{go},
              draw,circle,
              scale=0.9,
              fit to=tree]{};
              \node[
              draw,circle,
              scale=0.95,
              dashed,
              fit to=tree]{};
              }
                  [\tsc{f}2]
                  [\tsc{nom}P
                      [\tsc{f}1]
                      [\tsc{ind}P
                          [\tsc{ind}]
                      ]
                  ]
              ]
          ]
        \end{forest}
        \vspace{0.3cm}
      \\
      \toprule
      \tsc{acc} relative pronoun \tit{k-o-mu}
      \\
      \cmidrule{1-1}
      \begin{forest} boom
        [\tsc{rel}P, s sep=15mm
            [\tsc{rel}P
                [\phantom{x}k\phantom{x}, roof]
            ]
            [\tsc{dat}P, s sep=22mm
                [\tsc{an}P,
                tikz={
                \node[
                draw,circle,
                scale=0.85,
                dashed,
                fit to=tree]{};
                }
                    [\phantom{x}o\phantom{x}, roof]
                ]
                [\tsc{dat}P,
                tikz={
                \node[label=below:\tit{mu},
                draw,circle,
                scale=0.95,
                fit to=tree]{};
                }
                    [\tsc{f}3]
                    [\tsc{acc}P, tikz={
                    \node[
                    draw,circle,
                    scale=0.9,
                    dashed,
                    fit to=tree]{};
                    }
                        [\tsc{f}2]
                        [\tsc{nom}P
                            [\tsc{f}1]
                            [\tsc{ind}P
                                [\tsc{ind}]
                            ]
                        ]
                    ]
                ]
            ]
        ]
      \end{forest}
      \\
      \bottomrule
  \end{tabular}
  \end{adjustbox}
   \caption {Polish \tsc{ext}\scsub{acc} vs. \tsc{int}\scsub{dat} ↛ \tit{ogo}/\tit{komu}}
  \label{fig:polish-int-wins}
\end{figure}

The relative pronoun consists of three morphemes: \tit{k}, \tit{o} and \tit{mu}.
The light head consists of two morphemes: \tit{o} and \tit{go}.
Again, I circle the part of the structure that corresponds to a particular lexical entry, and I place the corresponding phonology under it, or I reduce the structure to a triangle, and I place the corresponding phonology under it.
I draw a dashed circle around each constituent that is a constituent in both the extra light head and the relative pronoun.

The extra light head consists of two constituents: the \tsc{an}P and the (lower) \tsc{acc}P. Together they form the (higher) \tsc{acc}P.
Both of these constituents are also constituents within the relative pronoun. However, the (higher) \tsc{acc}P is not a constituent within the relative pronoun. The constituent in which the \tsc{acc}P is contained namely also contains the feature \tsc{f}3 that makes it a \tsc{dat}P.
In other words, each feature and even each constituent of the extra light head is contained in the relative pronoun. However, they are not contained in the relative pronoun as a single constituent. Therefore, the extra light head cannot be deleted.

Recall from Section \ref{sec:comparing-mg} that this is the crucial example in which Modern German and Polish differ. The contrast lies in that the extra light head in Modern German forms a single constituent and in Polish it forms two constituents. In Modern German, relative pronouns in a more complex case contain the extra light head in a less complex case as a single constituent. In Polish, they do not. Relative pronouns in a complex case still contain all features of an extra light head in a less complex case, but the extra light head is not a single constituent within the relative pronoun. That is, the weaker feature containment requirement is met, but the stronger constituent containment requirement is not. This shows the necessity of formulating the proposal in terms of containment as a single constituent.

I continue with the example in which the external case is more complex than the internal case.
Consider the examples in \ref{ex:polish-dat-acc-rep}, in which the internal dative case competes against the external accusative case. The relative clauses are marked in bold. It is not possible to make a grammatical headless relative in this situation.
The internal case is accusative, as the predicate \tit{wpuścić} `to let' takes accusative objects. The relative pronoun \tit{kogo} `\ac{rel}.\ac{an}.\ac{acc}' appears in the accusative case.
The external case is dative, as the predicate \tit{ufać} `to trust' takes dative objects. The extra light head \tit{omu} `\ac{elh}.\ac{an}.\ac{dat}' appears in the dative case.
\ref{ex:polish-dat-acc-rel} is the variant of the sentence in which the extra light head is absent (indicated by the square brackets) and the relative pronoun surfaces, and it is ungrammatical.
\ref{ex:polish-dat-acc-lh} is the variant of the sentence in which the relative pronoun is absent (indicated by the square brackets) and the extra light head surfaces, and it is ungrammatical too.

\ex.\label{ex:polish-dat-acc-rep}
\ag. *Jan ufa [omu] \tbf{kogo} \tbf{-kolkwiek} \tbf{wpuścil} \tbf{do} \tbf{domu}.\\
Jan trust.\tsc{3sg}\scsub{[dat]} \tsc{elh}.\tsc{dat}.\tsc{an} \tsc{rp}.\tsc{acc}.\tsc{an} ever let.\tsc{3sg}\scsub{[acc]} to home\\
`Jan trusts whoever he let into the house.' \flushfill{Polish, adapted from \citealt{citko2013} after \pgcitealt{himmelreich2017}{17}}\label{ex:polish-dat-acc-rel}
\bg. Jan ufa omu [\tbf{kogo}] \tbf{-kolkwiek} \tbf{wpuścil} \tbf{do} \tbf{domu}.\\
Jan trust.\tsc{3sg}\scsub{[dat]} \tsc{elh}.\tsc{dat}.\tsc{an} \tsc{rp}.\tsc{acc}.\tsc{an} ever let.\tsc{3sg}\scsub{[acc]} to home\\
`Jan trusts whoever he let into the house.' \flushfill{Polish, adapted from \citealt{citko2013} after \pgcitealt{himmelreich2017}{17}}\label{ex:polish-dat-acc-lh}

In Figure \ref{fig:polish-ext-wins}, I give the syntactic structure of the extra light head at the top and the syntactic structure of the relative pronoun at the bottom.

\begin{figure}[htbp]
  \center
  \begin{adjustbox}{max height=0.9\textheight}
  \begin{tabular}[b]{c}
        \toprule
        \tsc{dat} extra light head \tit{o-mu} \\
        \cmidrule{1-1}
        \begin{forest} boom
          [\tsc{dat}P, s sep=25mm
              [\tsc{an}P,
              tikz={
              \node[
              draw,circle,
              scale=0.85,
              dashed,
              fit to=tree]{};
              }
                  [\phantom{x}o\phantom{x}, roof]
              ]
              [\tsc{dat}P,
              tikz={
              \node[label=below:\tit{mu},
              draw,circle,
              scale=0.95,
              fit to=tree]{};
              }
                  [\tsc{f}3]
                  [\tsc{acc}P,
                  tikz={
                  \node[
                  draw,circle,
                  scale=0.9,
                  dashed,
                  fit to=tree]{};
                  }
                      [\tsc{f}2]
                      [\tsc{nom}P
                          [\tsc{f}1]
                          [\tsc{ind}P
                              [\tsc{ind}]
                          ]
                      ]
                  ]
              ]
          ]
        \end{forest}
        \vspace{0.3cm}
      \\
      \toprule
      \tsc{acc} relative pronoun \tit{k-o-go}
      \\
      \cmidrule{1-1}
      \begin{forest} boom
        [\tsc{rel}P, s sep=15mm
            [\tsc{rel}P
                [\phantom{x}k\phantom{x}, roof]
            ]
            [\tsc{acc}P, s sep=20mm
                [\tsc{an}P,
                tikz={
                \node[
                draw,circle,
                scale=0.85,
                dashed,
                fit to=tree]{};
                }
                    [\phantom{x}o\phantom{x}, roof]
                ]
                [\tsc{acc}P,
                tikz={
                \node[label=below:\tit{go},
                draw,circle,
                scale=0.9,
                fit to=tree]{};
                \node[
                draw,circle,
                scale=0.95,
                dashed,
                fit to=tree]{};
                }
                    [\tsc{f}2]
                    [\tsc{nom}P
                        [\tsc{f}1]
                        [\tsc{ind}P
                            [\tsc{ind}]
                        ]
                    ]
                ]
            ]
        ]
      \end{forest}
      \\
      \bottomrule
  \end{tabular}
  \end{adjustbox}
   \caption {Polish \tsc{ext}\scsub{dat} vs. \tsc{int}\scsub{acc} ↛ \tit{omu}/\tit{kogo}}
  \label{fig:polish-ext-wins}
\end{figure}

The relative pronoun consists of three morphemes: \tit{k}, \tit{o} and \tit{go}.
The light head consists of two morphemes: \tit{o} and \tit{mu}.
Again, I circle the part of the structure that corresponds to a particular lexical entry, and I place the corresponding phonology under it, or I reduce the structure to a triangle, and I place the corresponding phonology under it.
I draw a dashed circle around each constituent that is a constituent in both the extra light head and the relative pronoun.

The extra light head consists of two constituents: the \tsc{an}P and the (lower) \tsc{dat}P.
In this case, the relative pronoun does not contain both these constituents. The relative pronoun only contains the \tsc{acc}P, and it lacks the \tsc{f}3 that makes a \tsc{dat}P. Since the weaker feature containment requirement is not met, the stronger constituent containment requirement cannot be met either.
The extra light head also does not contain all constituents or features that the relative pronoun contains, because it lacks the complete \tsc{rel}P.
Therefore, the extra light head cannot be deleted, and the relative pronoun cannot be deleted either.

\section{Summary}
