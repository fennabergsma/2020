% !TEX root = thesis.tex

\chapter{Deriving the matching type}\label{ch:deriving-matching}

Languages of the matching type can be summarizes as in Table \ref{tbl:rel-lh-pol}

\begin{table}[htbp]
  \center
  \caption{Grammaticality in the internal-only type}
\begin{tabular}{cc}
  \toprule
                                        & surface pronoun         \\
  \cmidrule(lr){2-2}
\tsc{k}\scsub{int} = \tsc{k}\scsub{ext} & \tsc{rp}\scsub{int/ext} \\
\tsc{k}\scsub{int} > \tsc{k}\scsub{ext} & *     \\
\tsc{k}\scsub{int} < \tsc{k}\scsub{ext} & *                       \\
\bottomrule
\end{tabular}
\label{tbl:rel-lh-pol}
\end{table}

When the internal and the external case match, and there is a tie, the relative pronoun surfaces in the this particular case (just like in all other language types).
When the internal case wins the case competition, this type of language does not allow the internal case to surface. This means that the relative pronoun with a more complex internal case cannot be the surface pronoun.
When the external case wins the case competition, this type of language does not allow the external case to surface. This means that the light head with a more complex external case cannot be the surface pronoun.

The situation in which the matching type of language differs from the internal-only type is the one in which the internal case wins the case competition. This is grammatical in the internal-only type of language but it is ungrammatical in the matching type of language.
In Chapter \ref{ch:constituent-containment}, I suggested that this difference can be derived from a difference in spellout between the two languages types. A different spellout namely leads to a different constituency within relative pronouns and light heads. In the internal-only type of language, the ϕP appears lower in the structure than the case projections, as I showed is Chapter \ref{ch:deriving-onlyinternal}. In the matching type of language, the ϕP appears higher than the case projections shown in Figure \ref{fig:rel-lh-intonly-derive}.

\begin{figure}[htbp]
  \center
  \begin{tabular}[b]{ccc}
    \toprule
    light head & & relative pronoun \\
    \cmidrule(lr){1-1} \cmidrule(lr){3-3}
    \begin{forest} boom
    [ϕP
        [ϕ]
        [\tsc{k}P
            [\tsc{k}, baseline]
        ]
    ]
    \end{forest}
    & \phantom{x} &
  \begin{forest} boom
    [\tsc{rel}P
        [\tsc{rel}]
        [ϕP
            [ϕ]
            [\tsc{k}P
                [\tsc{k}, baseline]
            ]
        ]
    ]
  \end{forest}\\
    \bottomrule
  \end{tabular}
   \caption {\tsc{lh} and \tsc{rp} in the matching type}
  \label{fig:rel-lh-matching-derive}
\end{figure}

When the internal and the external case match, the relative pronoun can delete the light head, because the light head forms a single constituent within the relative pronoun.
When the internal case is more complex than the external case, the light head is not a single constituent within the relative pronoun anymore. The relative pronoun contains all features of the light head, but they are spread over separate constituents. As a result, there is no grammatical form to surface when the internal case is more complex.
When the external case is more complex than the internal case, the relative pronoun is not a single constituent within the light head. The relative pronoun contains features that are not part of the light head. As a result, there is no grammatical form to surface when the internal case is more complex.

In Chapter \ref{ch:constituent-containment}, I suggested that the difference in structures between the internal-only type and the matching type is a consequence of spellout. The change in constituency is a result of the fact that case projections correspond to their own morpheme in matching languages and they are spelled out together with phi-features in internal-only languages. In Chapter \ref{ch:deriving-matching}, I showed that Modern German indeed has a portmanteau for case and phi-features. In this Chapter, I show that Polish has two morphemes that corresponds to case and phi-features.

I give a compact version of the Polish light heads and relative pronouns in Figure \ref{fig:rel-lh-pol}.

\begin{figure}[htbp]
  \center
  \begin{adjustbox}{max width=\textwidth}
  \begin{tabular}[b]{ccc}
      \toprule
      light head & & relative pronoun \\
      \cmidrule(lr){1-1} \cmidrule(lr){3-3}
      \begin{forest} boom
        [\tsc{k}P, s sep = 15mm
            [\tsc{an}P,
            tikz={
            \node[
            draw,circle,
            scale=0.75,
            fit to=tree]{};
            }
                [\phantom{xxx}, roof, baseline]
            ]
            [\tsc{k}P,
            tikz={
            \node[
            draw,circle,
            scale=0.85,
            fit to=tree]{};
            }
                [\tsc{k}]
                [\tsc{ind}P
                    [\tsc{ind}]
                ]
            ]
        ]
      \end{forest}
      & \phantom{x} &
      \begin{forest} boom
        [\tsc{rel}P
            [\tsc{rel}P,
            tikz={
            \node[
            draw,circle,
            scale=0.75,
            fit to=tree]{};
            }
                [\phantom{xxx}, roof]
            ]
            [\tsc{k}P, s sep = 15mm
                [\tsc{an}P,
                tikz={
                \node[
                draw,circle,
                scale=0.75,
                fit to=tree]{};
                }
                    [\phantom{xxx}, roof, baseline]
                ]
                [\tsc{k}P,
                tikz={
                \node[
                draw,circle,
                scale=0.85,
                fit to=tree]{};
                }
                    [\tsc{k}]
                    [\tsc{ind}P
                        [\tsc{ind}]
                    ]
                ]
            ]
        ]
      \end{forest}\\
      \bottomrule
  \end{tabular}
  \end{adjustbox}
   \caption {\tsc{lh} and \tsc{rp} in Polish}
  \label{fig:rel-lh-pol}
\end{figure}

I compare the Polish light head and the relative pronoun in Figure \ref{fig:rel-lh-pol} to the structures of the light head and the relative pronoun I gave in Figure \ref{fig:rel-lh-matching}.

Consider the light head in Figure \ref{fig:rel-lh-pol}.
Light heads in Polish are spelled out by two morphemes, which are both circled. The morpheme on the right does not only correspond to case features, but also to number feature (\tsc{ind}). The remainder of what corresponds to the ϕP is the morpheme on the left (\tsc{an}P).
Therefore, just like the structure of the light head in Figure \ref{fig:rel-lh-matching}, the structure of the light head in Figure \ref{fig:rel-lh-pol} consists of two morphemes.

Consider the relative pronoun in Figure \ref{fig:rel-lh-pol}.
Relative pronouns in Polish contain one more morpheme than light heads: the \tsc{rel}P. As already became clear in Chapter \ref{ch:deriving-onlyinternal}, the relative pronoun contains more a single feature more than the light head, which is in Polish all contained in \tsc{rel}P. Besides that, the structure of the relative pronoun in Figure \ref{fig:rel-lh-pol} is identical to the one in Figure \ref{fig:rel-lh-matching}.

Crucially, the constituency in Figure \ref{fig:rel-lh-mg} is the same as it is in Figure \ref{fig:rel-lh-intonly}. Therefore, the deletion possibilities I described for Figure \ref{fig:rel-lh-matching} take place.

The chapter is structured as follows.
First, I discuss the light head. I decompose the light heads into the two morphemes I showed in Figure \ref{fig:rel-lh-pol}, and I show which features each of the morphemes corresponds to. I show that Polish headless relatives are derived from a type of light-headed relative clause that does not surface in the language, just like their Modern German counterparts.
Then, I discuss the relative pronoun. I show that the Polish relative pronouns consist of one more morpheme than Polish light heads.
Importantly, the features that form the Polish light head and relative pronoun are the same ones that form the Modern German ones. The only difference between the two languages is how the features are spelled out.
Finally, I compare the constituents of the light head and the relative pronoun. I show that the relative pronoun can only delete the light head when the internal case matches the external case. When the internal and external case differ, I show that none of the elements can delete the other one.


\section{The Polish (extra) light head}

In Chapter \ref{ch:constituent-containment}, I argued that in the matching type, the features of the light head are spelled out in such a way that they form the constituency shown in Figure \ref{ex:simple-matching}.

\ex.\label{ex:simple-matching}
\begin{forest} boom
[ϕP
    [ϕ]
    [\tsc{k}P
        [\tsc{k}]
    ]
]
\end{forest}

In Chapter \ref{ch:deriving-onlyinternal}, I argued for Modern German that the functional sequence for the extra light head contains the features in \ref{ex:fseq-wh-lh-pol}.

\ex. \begin{forest} boom
  [\tsc{k}P
      [\tsc{k}]
      [\tsc{ind}P
          [\tsc{ind}]
          [\tsc{an}P
              [\tsc{an}]
              [\tsc{cl}P
                  [\tsc{cl}]
                  [ΣP
                      [Σ]
                      [\tsc{ref}]
                  ]
              ]
          ]
      ]
  ]
\end{forest}
\label{ex:fseq-wh-lh-pol}

In this section, I argue that Polish light heads have the structure shown in \ref{ex:pol-elh}.

\ex.\label{ex:pol-elh}
\begin{adjustbox}{max width=0.9\textwidth}
\begin{forest} boom
  [\tsc{dat}P, s sep=40mm
      [\tsc{an}P,
      tikz={
      \node[label=below:\tit{o},
      draw,circle,
      scale=0.95,
      fit to=tree]{};
      }
          [\tsc{an}]
          [\tsc{cl}P
              [\tsc{cl}]
              [ΣP
                  [Σ]
                  [\tsc{ref}]
              ]
          ]
      ]
      [\tsc{k}P,
      tikz={
      \node[label=below:\tit{go/mu},
      draw,circle,
      scale=0.9,
      fit to=tree]{};
      }
          [\tsc{k}]
          [\tsc{ind}P
              [\tsc{ind}]
          ]
      ]
  ]
\end{forest}
\end{adjustbox}

Recall from Chapter \ref{ch:deriving-onlyinternal} that Modern German the extra light head spells out as a single constituent. In this chapter I argue that the Polish extra light head consists of two constituents, as shown in \ref{ex:pol-elh}. This is the crucial difference between the two languages that leads them to be of different types in headless relatives.

I discuss two extra light heads: the animate accusative and the animate dative. These are the two forms that I compare the constituents of in Section \ref{sec:comparing-polish}. I show them in \ref{ex:pol-elhs}.

\ex.\label{ex:pol-elhs}
\ag. o-go\\
 `\tsc{elh}.\tsc{an}.\tsc{acc}'\\
\bg. o-mu\\
 `\tsc{elh}.\tsc{an}.\tsc{dat}'\\

I decompose the extra light heads in two morphemes: the \tit{o} and the final suffix (\tit{go} and \tit{mu}). For each morpheme, I discuss which features they spell out, and I give their lexical entries. In the end, I show how the extra light heads are constructed.

Even though I start with the extra light heads, I start with the relative pronouns. consider the table.

\begin{table}[htbp]
  \center
  \caption{Polish (in)animate relative pronouns \pgcitep{swan2002}{160}}
  \begin{tabular}[b]{ccc}
    \toprule
              & \tsc{an}  & \tsc{inam} \\
    \cmidrule{2-3}
    \tsc{nom} & kto       & c-o        \\
    \tsc{acc} & k-o-go    & c-o        \\
    \tsc{gen} & k-o-go    & cz-e-go    \\
    \tsc{dat} & k-o-mu    & cz-e-mu    \\
    \bottomrule
  \end{tabular}
  \label{tbl:pol-rels-1}
\end{table}

The inanimates start with a cz. The cz is not a primary vowel but a derived one. it appears when it is combines with a xx.

give example of that

The \tit{e} is also a product of the \tit{j} coming together with the xx.

give example of that

So, actually, what we are looking at is this:

\begin{table}[htbp]
  \center
  \caption{Polish (in)animate relative pronouns \pgcitep{swan2002}{160}}
  \begin{tabular}[b]{ccc}
    \toprule
              & \tsc{an}  & \tsc{inam} \\
    \cmidrule{2-3}
    \tsc{nom} & kto       & k-j-o        \\
    \tsc{acc} & k-o-go    & k-j-o        \\
    \tsc{gen} & k-o-go    & k-j-o-go    \\
    \tsc{dat} & k-o-mu    & k-j-o-mu    \\
    \bottomrule
  \end{tabular}
  \label{tbl:pol-rels}
\end{table}

Now, analyze the personal pronouns like this too:

I start from \tit{je}-pronouns.

this je differs in one feature from the extra light head: namely the \tsc{c}. I assume that the strong pronoun spells out \tit{j}. and the rest of the structure is light head. now the question is: which portion does \tit{o} spell out and which portion does the suffix?


I start with the morphemes \tit{go} and \tit{mu}. Consider the masculine and neuter personal pronouns in Table \ref{tbl:pol-prons}.\footnote{
Polish has three types of third person pronouns \pgcitep{swan2002}{156-157}. Not all types of pronouns exist for all numbers and genders, but they do for the masculine accusative and dative singular.
\tit{Je}-pronouns are used in clause-initial position or when emphasis or a contrast is expressed.
\tit{Ni}-pronouns are used after prepositions.
Clitics are used in non-stressed contexts. The only third person clitics that exist are \tit{go} and \tit{mu}. According to \pgcitet{franks2004}{22}, these clitics are not `real' clitics, since they syntactically behave like phrases. The deficiency is only just prosodic.
}

\begin{table}[htbp]
  \center
  \caption{3\tsc{sg} personal pronouns \pgcitep{swan2002}{156}}
  \begin{tabular}[b]{ccc}
    \toprule
              & \tsc{m}.\tsc{sg}  & \tsc{n}.\tsc{sg}  \\
    \cmidrule{2-3}
    \tsc{acc} & je-go             & je                \\
    \tsc{gen} & je-go             & je-go             \\
    \tsc{dat} & je-mu             & je-mu             \\
    \bottomrule
  \end{tabular}
  \label{tbl:pol-prons}
\end{table}

Notice that the morpheme \tit{mu} does not only appear as the dative suffix in the masculine, but also in the neuter. The morpheme \tit{go} appears as the accusative and genitive suffix in the masculine and as the genitive suffix in the neuter.\footnote{
I include genitive in the paradigms to show that the patterns observed in the dative are not standing on themselves. Instead, they are more generally attested in Polish, and they deserve an explanation. However, I do not incorporate them in the syntactic structures.
The reason for that is that the genitive in Polish is comes between the accusative and the dative, i.e. it is more complex than the accusative and less complex than the dative.
This does not change anything about the main point about case I want to make: the dative is more complex than the accusative.
}
Moreover, the complete pronouns are syncretic: in all cases, the suffix combines with the morpheme \tit{je}. I set up a system that can derive the syncretism between the two genders. Doing this allows me to establish which features the morphemes \tit{go} and \tit{mu} spell out.

I discussed in Chapter \ref{ch:decomposition} that syncretisms can be derived in Nanosyntax via the Superset Principle. The lexicon contains a lexical entry that is specified for the form that corresponds to the most features. To illustrate this, I repeat the lexical entry for the Dutch \tit{jullie} `you' in \ref{ex:dutch-jullie-lexicon-rep}.

\ex.
\begin{forest} boom
  [\ac{dat}P
      [\ac{f}3]
      [\ac{acc}P
          [\ac{f}2]
          [\ac{nom}P
              [\ac{f}1]
              [2\ac{pl}P
                  [\phantom{xxx}, roof]
              ]
          ]
      ]
  ]
  {\draw (.east) node[right]{⇔ \tit{jullie}}; }
\end{forest}
\label{ex:dutch-jullie-lexicon-rep}

\tit{Jullie} is syncretic between nominative, accusative and dative. It is specified for dative in the lexicon, because the dative contains the accusative and the nominative. The nominative and accusative second person plural in Dutch are spelled out as \tit{jullie} as well, because the \tsc{dat}P contains the \tsc{acc}P which contains \tsc{nom}P (Superset Principle), and there is no more specific lexical entry available in Dutch (Elsewhere Condition). It is important that the potentially unused features (so the \tsc{f}3 or \tsc{f}3 and the \tsc{f2}) are at the top, so that the constituent that needs to be spelled out is still contained in the lexical tree.

I show how I get this syncretism for \tit{jemu}. Different from \tit{jullie}, \tit{jemu} consists of two morphemes: \tit{je} and \tit{mu}. I give the functional sequence that I assume \tit{jemu} spells out. These are case features up to the dative, the feature \tsc{ind} for singular number, the gender features \tsc{cl} and \tsc{an} and some XP for whatever other features \tit{jemu} contains.

\ex. \label{ex:fseq-jemu}
\begin{forest} boom
  [\ac{dat}P
      [\ac{f}3]
      [\ac{acc}P
          [\ac{f}2]
          [\ac{nom}P
              [\ac{f}1]
              [\tsc{ind}P
                  [\tsc{ind}]
                  [\tsc{an}P
                      [\tsc{an}]
                      [\tsc{cl}P
                          [\tsc{cl}]
                          [XP
                              [\phantom{xxx}, roof]
                          ]
                      ]
                  ]
              ]
          ]
      ]
  ]
\end{forest}

The morpheme \tit{je} is syncretic between the masculine and neuter. That means that the highest feature in the lexical tree needs to be the feature \tsc{an}. I give the lexical entry in \ref{ex:pol-entry-je}.

\ex. \label{ex:pol-entry-je}
\begin{forest} boom
  [\tsc{an}P
      [\tsc{an}]
      [\tsc{cl}P
          [\tsc{cl}]
          [XP
              [\phantom{xxx}, roof]
          ]
      ]
  ]
  {\draw (.east) node[right]{⇔ \tit{je}}; }
\end{forest}

When the lexical entry for \tit{je} is as in \ref{ex:pol-entry-je}, it can be inserted if there is a animate or an inanimate syntactic structure. In \ref{ex:pol-spellout-je-an} I give a syntactic structure of an animate. The syntactic structure forms a constituent within the lexical tree in \ref{ex:pol-entry-je}, and the structure can be spelled out as \tit{je}.

\ex.\label{ex:pol-spellout-je-an}
\begin{forest} boom
  [\tsc{an}P,
  tikz={
  \node[label=below:\tit{je},
  draw,circle,
  scale=0.85,
  fit to=tree]{};
  }
      [\tsc{an}]
      [\tsc{cl}P
          [\tsc{cl}]
          [XP
              [\phantom{xxx}, roof]
          ]
      ]
  ]
\end{forest}

In \ref{ex:pol-spellout-je-cl} I give a syntactic structure of an animate. The syntactic structure also forms a constituent within the lexical tree in \ref{ex:pol-entry-je}, and the structure can be spelled out as \tit{je}.

\ex.\label{ex:pol-spellout-je-cl}
\begin{forest} boom
  [\tsc{cl}P,
  tikz={
  \node[label=below:\tit{je},
  draw,circle,
  scale=0.85,
  fit to=tree]{};
  }
      [\tsc{cl}]
      [XP
          [\phantom{xxx}, roof]
      ]
  ]
\end{forest}

What follows from this is that the lexical trees for the suffix \tit{mu} should contain all features in \ref{ex:fseq-jemu} that are not spelled out by \tit{je} so far. These are the feature \tsc{ind} and all case features up to the dative. I give the lexical entry for \tit{mu} in \ref{ex:pol-entry-mu}.

\ex. \label{ex:pol-entry-mu}
\begin{forest} boom
  [\tsc{dat}P
      [\tsc{f}3]
      [\tsc{acc}P
          [\tsc{f}2]
          [\tsc{nom}P
              [\tsc{f}1]
              [\tsc{ind}P
                  [\tsc{ind}]
              ]
          ]
      ]
  ]
  {\draw (.east) node[right]{⇔ \tit{mu}}; }
\end{forest}

Notice here that \tit{mu} has a unary bottom. Therefore, it can be inserted as the result of movement. That means that the lexical entry follows the existing structure and is spelled out as a suffix. This is how the correct order of \tit{je} and \tit{mu} comes about. I show how this works when I construct the relative pronouns.

The morpheme \tit{go} is not used in the accusative neuter, but it is in the genitive. What I take away from this is that the morpheme \tit{go} needs to have \tsc{ind} as the lowest feature too, so that it can combine with the feature \tsc{an} if that is present and with the feature \tsc{cl} if \tsc{an} is absent. I give the lexical entry for \tit{go} in \ref{ex:pol-entry-go}.

\ex. \label{ex:pol-entry-go}
\begin{forest} boom
  [\tsc{acc}P
      [\tsc{f}2]
      [\tsc{nom}P
          [\tsc{f}1]
          [\tsc{ind}P
              [\tsc{ind}]
          ]
      ]
  ]
  {\draw (.east) node[right]{⇔ \tit{go}}; }
\end{forest}




I continue with the morpheme \tit{o}. I propose that \tit{o} spells out the features \tsc{thing}, \tsc{person}, Σ, \tsc{cl} and \tsc{an}. Compare the animate and inanimate relative pronouns in Table \ref{tbl:pol-rels}.

\begin{table}[htbp]
  \center
  \caption{Polish (in)animate relative pronouns \pgcitep{swan2002}{160}}
  \begin{tabular}[b]{ccc}
    \toprule
              & \tsc{an}  & \tsc{inam} \\
    \cmidrule{2-3}
    \tsc{nom} & kto       & c-o        \\
    \tsc{acc} & k-o-go    & c-o        \\
    \tsc{gen} & k-o-go    & cz-e-go    \\
    \tsc{dat} & k-o-mu    & cz-e-mu    \\
    \bottomrule
  \end{tabular}
  \label{tbl:pol-rels}
\end{table}

%there is also kim and czym
%k automatically softens to k' before y, which apparently goes even further in the wh-words? p.15
%k is a primary consonant, c and cz are not. they are derived. k is primary with respect to c and cz p.23

The animate and the inanimate combine with the same suffix in the genitive and the dative. They differ, however, in the initial consonant and the following vowel. The animate form starts with a \tit{k} and has the \tit{o} as a vowel, and the inanimate form starts with a \tit{cz} and has the \tit{e} as a vowel. In Polish, the consonant \tit{k} is replaced by a \tit{cz} when it is followed by an \tit{e} \pgcitep{swan2002}{24}.\footnote{
I assume that the change from \tsc{c} to \tit{cz} is palatalization as a consequence of the combination of /t͡s/ and /ɛ/ \pgcitep{swan2002}{26}.
}\footnote{
In demonstratives, there is no alternation between the masculines and the neuters, even though they combine with the same suffixes as \tsc{wh}-pronouns do. I attribute this difference to how \tsc{wh}-pronouns and demonstratives differ regarding gender. Demonstratives get their gender from the (possibly phonologically empty) head noun, and the gender is syntactic (i.e. it depends on the grammatical gender of the head noun). \tsc{wh}-pronouns do not combine with a noun, so they get their gender from themselves. I assume that this difference translates into that the demonstratives always have only \tsc{thing} as their lowest feature and \tsc{wh}-pronouns can have \tsc{thing} or \tsc{thing} plus \tsc{person}.
} Therefore, I assume that the change from \tit{k} to \tit{cz} is phonology. The lexical entry for the \tit{o} differs from the one for the \tit{e} in that is spells out the feature \tsc{thing} plus the feature \tsc{person} (and not only the feature \tsc{thing}), and the feature \tsc{animate} plus the feature \tsc{class} (and not only the feature \tsc{class}).

Finally, as it is a pronominal element which is not a clitic, I assume that the \tit{o} spells out the feature Σ.

% why is c-o also a c and not a k? is there a way to say that if was e for quite a while before it got overridden as a o?
% Note that there is an \tit{o} /ɔ/ in the nominative and accusative. This is a different /ɔ/ than the one in the animate relative pronouns. It namely also spells out case features.

I give the lexical entry for \tit{o} in \ref{ex:pol-entry-e}.

\ex.\label{ex:pol-entry-e}
\begin{forest} boom
  [\tsc{an}P
      [\tsc{an}]
      [\tsc{cl}P
          [\tsc{cl}]
          [ΣP
              [Σ]
              [\tsc{ref}]
          ]
      ]
  ]
  {\draw (.east) node[right]{⇔ \tit{o}}; }
\end{forest}






In what follows, I construct the Polish relative pronouns. I follow the same functional sequence as I did for Modern German. Also, of course, the spellout procedure is identical. The outcome is different because of the different lexical entries Polish has.

Starting from the bottom, the first two features that are merged at \tsc{thing} and \tsc{person}, creating a \tsc{person}P.
The syntactic structure forms a constituent in the lexical tree in \ref{ex:pol-entry-e}, repeated from \ref{ex:pol-entry-e}, which corresponds to the \tit{o}.
Therefore, the \tsc{person}P is spelled out as \tit{o}, which I do not show here.

The features Σ, \tsc{cl} and \tsc{an} are merged and spelled out in the same way.
First, the feature Σ is merged, and a ΣP is created.
The syntactic structure forms a constituent in the lexical tree in \ref{ex:pol-entry-e}.
Therefore, the ΣP is spelled out as \tit{o}.
Then, the feature \tsc{cl} is merged, and a \tsc{cl}P is created.
The syntactic structure forms a constituent in the lexical tree in \ref{ex:pol-entry-e}.
Therefore, the \tsc{cl}P is spelled out as \tit{o}.
Finally, the feature \tsc{an} is merged, and a \tsc{an}P is created.
The syntactic structure forms a constituent in the lexical tree in \ref{ex:pol-entry-e}.
Therefore, the \tsc{cl}P is spelled out as \tit{o}, shown in \ref{ex:pol-spellout-e-rel}.

\ex.\label{ex:pol-spellout-e-rel}
\begin{forest} boom
  [\tsc{an}P,
  tikz={
  \node[label=below:\tit{o},
  draw,circle,
  scale=0.9,
  fit to=tree]{};
  }
      [\tsc{an}]
      [\tsc{cl}P
          [\tsc{cl}]
          [ΣP
              [Σ]
              [\tsc{ref}]
          ]
      ]
  ]
\end{forest}

The next feature in the functional sequence is the feature \tsc{ind}. This feature cannot be spelled out as the other ones before. The feature \tsc{ind} is merged, and a \tsc{ind}P is created. This syntactic structure does not form a constituent in the lexical tree in \ref{ex:pol-entry-e}. There is also no other lexical tree that contains the syntactic structure as a constituent.
Therefore, there is no succesfull spellout for the syntactic structure in the derivational step in which the structure is spelled out as a single phrase (\ref{ex:spellout-algorithm-phrasal-rep} in the Spellout Algorithm, repeated from Chapter \ref{ch:deriving-onlyinternal}).

\ex. \tbf{Spellout Algorithm} (as in \citealt{caha2020a}, based on \citealt{starke2018})\label{ex:spellout-algorithm-rep}
 \a. Merge F and spell out.\label{ex:spellout-algorithm-phrasal-rep}
 \b. If (a) fails, move the Spec of the complement and spell out.\label{ex:spellout-algorithm-spec-rep}
 \b. If (b) fails, move the complement of F and spell out.\label{ex:spellout-algorithm-comp-rep}

The first movement option in the Spellout Algorithm is moving the specifier, as described in \ref{ex:spellout-algorithm-spec-rep}. As there is no specifier in this structure, so the first movement option irrelevant.
The second movement option in the Spellout Algorithm is moving the complement, as described in \ref{ex:spellout-algorithm-comp}. In this case, the complement of \tsc{ind}, the \tsc{an}P, is moved to the specifier of \tsc{ind}P.
The \tsc{ind}P is a different constituent now. It still contains the feature \tsc{ind}, but it no longer contains the \tsc{an}P. The syntactic structure forms a constituent in the lexical tree in \ref{ex:pol-entry-go-rep1}.

\ex.\label{ex:pol-entry-go-rep1}
\begin{forest} boom
  [\tsc{acc}P
      [\tsc{f}2]
      [\tsc{nom}P
          [\tsc{f}1]
          [\tsc{ind}P
              [\tsc{ind}]
          ]
      ]
  ]
{\draw (.east) node[right]{⇔ \tit{go}}; }
\end{forest}

Therefore, the \tsc{ind}P is spelled out as \tit{go}, as shown in \ref{ex:pol-spellout-e-ind}.

\ex.\label{ex:pol-spellout-e-ind}
\begin{forest} boom
  [\tsc{ind}P
  [\tsc{an}P
      [\phantom{x}\tit{o}\phantom{x}, roof]
  ]
      [\tsc{ind}P,
      tikz={
      \node[label=below:\tit{go},
      draw,circle,
      scale=0.95,
      fit to=tree]{};
      }
          [\tsc{ind}]
      ]
  ]
\end{forest}

The feature \tsc{f}1 is merged with the \tsc{ind}P, forming an \tsc{nom}P. This phrase is not contained in any of the lexical entries in \ref{ex:pol-entries-elh}. The first movement is tried: the specifier of the \tsc{ind}P, the \tsc{an}P, is moved to the specifier of \tsc{nom}P. This phrase is contained in the lexical tree in \ref{ex:pol-entry-go-rep}, so it is spelled out as \tit{go}.
The feature \tsc{f}2 is merged with the \tsc{nom}P, forming an \tsc{acc}P. This phrase is not contained in any of the lexical entries in \ref{ex:pol-entries-elh}. The first movement is tried: the specifier of the \tsc{nom}P, the \tsc{an}P, is moved to the specifier of \tsc{acc}P. This phrase is contained in the lexical tree in \ref{ex:pol-entry-go-rep}, so it is spelled out as \tit{go}.

The accusative animate extra light head is shown in \ref{ex:pol-elh-acc}.

\ex.\label{ex:pol-elh-acc}
\begin{adjustbox}{max width=0.9\textwidth}
\begin{forest} boom
  [\tsc{dat}P, s sep=40mm
      [\tsc{an}P,
      tikz={
      \node[label=below:\tit{o},
      draw,circle,
      scale=0.9,
      fit to=tree]{};
      }
          [\tsc{an}]
          [\tsc{cl}P
              [\tsc{cl}]
              [ΣP
                  [Σ]
                  [\tsc{ref}]
              ]
          ]
      ]
      [\tsc{acc}P,
      tikz={
      \node[label=below:\tit{go},
      draw,circle,
      scale=0.9,
      fit to=tree]{};
      }
          [\tsc{f}2]
          [\tsc{nom}P
              [\tsc{f}1]
              [\tsc{ind}P
                  [\tsc{ind}]
              ]
          ]
      ]
  ]
\end{forest}
\end{adjustbox}

The dative animate extra light head is constructed as its accusative counterpart, except for that the feature \tsc{f}3 is added to create a dative.

The feature \tsc{f}3 is merged with the \tsc{acc}P, forming an \tsc{dat}P. This phrase is not contained in any of the lexical entries in \ref{ex:pol-entries-elh}. The first movement is tried: the specifier of the \tsc{acc}P, the \tsc{an}P, is moved to the specifier of \tsc{dat}P. This phrase is contained in the lexical tree in \ref{ex:pol-entry-mu-rep}, so it is spelled out as \tit{mu}.

The dative animate extra light head is shown in \ref{ex:pol-elh-dat}.

\ex.\label{ex:pol-elh-dat}
\begin{adjustbox}{max width=0.9\textwidth}
\begin{forest} boom
  [\tsc{dat}P, s sep=45mm
      [\tsc{an}P,
      tikz={
      \node[label=below:\tit{o},
      draw,circle,
      scale=0.95,
      fit to=tree]{};
      }
          [\tsc{an}]
          [\tsc{cl}P
              [\tsc{cl}]
              [ΣP
                  [Σ]
                  [\tsc{ref}]
              ]
          ]
      ]
      [\tsc{dat}P,
      tikz={
      \node[label=below:\tit{mu},
      draw,circle,
      scale=0.95,
      fit to=tree]{};
      }
          [\tsc{f}3]
          [\tsc{acc}P
              [\tsc{f}2]
              [\tsc{nom}P
                  [\tsc{f}1]
                  [\tsc{ind}P
                      [\tsc{ind}]
                  ]
              ]
          ]
      ]
  ]
\end{forest}
\end{adjustbox}

So, the light-headed relative that headless relatives are derived from is:

\exg. Jan lubi [ogo] \tbf{kogo} \tbf{-kolkwiek} \tbf{Maria} \tbf{lubi}.\\
Jan like.\tsc{3sg}\scsub{[acc]} \tsc{elh}.\tsc{acc}.\tsc{an} \tsc{rp}.\tsc{acc}.\tsc{an} ever Maria like.\tsc{3sg}\scsub{[acc]}\\
`Jan likes whoever Maria likes.' \flushfill{Polish, adapted from \citealt{citko2013} after \pgcitealt{himmelreich2017}{17}}\label{ex:pol-elh-rp}

For Modern German, I considered two kinds of light-headed relatives as the source of the headless relative.
First, the light-headed relative is derived from an existing light-headed relative, and the deletion of the light head is optional. Second, the light-headed relative is derived from a light-headed relative that does not surfaces in Modern German, and the deletion of the light head is obligatory.
For Modern German I concluded it was the second, and I proposed which features this extra light head should consist of. This set of features in Polish corresponds to the extra light head \tit{ogo} or \tit{omu}, which is not attested as a light head in an existing light-headed relative in Polish.

In the rest of this section I consider the existing Polish light-headed relative that could potentially be the source for headless relatives. This is the light-headed relative that in which the demonstrative is the light head, as shown in \ref{ex:pol-light-headed}.

\exg. Jan śpiewa to, co Maria śpiewa.\\
Jan sings \tsc{dem}.\tsc{m}.\tsc{sg}.\tsc{acc} \tsc{rp}.\tsc{an}.\tsc{acc} Maria sings\\
`John sings what Mary sings.' \flushfill{Polish, \pgcitealt{citko2004}{103}}\label{ex:pol-light-headed}

For Modern German, I gave two arguments for not taking this existing light-headed relative as source of the headless relative. In what follows, I show that these arguments hold for Polish in the same way do for Modern German.

First, in headless relatives the morpheme \tit{kolwiek} `ever' can appear, as shown in \ref{ex:pol-headless-ever}.

\exg. Jan śpiewa co -kolwiek Maria śpiewa.\\
Jan sings \tsc{rp}.\tsc{an}.\tsc{acc} ever Maria sings\\
`Jan sings everything Maria sings.' \flushfill{Polish, \pgcitealt{citko2004}{116}}\label{ex:pol-headless-ever}

Light-headed relatives do not allow this morpheme to be inserted, illustrated in \ref{ex:pol-headed-ever}.

\exg. *Jan śpiewa to, co -kolwiek Maria śpiewa.\\
Jan sings \tsc{dem}.\tsc{m}.\tsc{sg}.\tsc{acc} \tsc{rp}.\tsc{an}.\tsc{acc} ever Maria sings\\
`John sings what Mary sings.' \flushfill{Polish, \pgcitealt{citko2004}{116}}\label{ex:pol-headed-ever}

Just like for Modern German, I assume that the headless relative is not derived from an ungrammatical structure.\footnote{
\citet{citko2004} takes the complementary distribution of \tit{kolwiek} `ever' and the light head to mean that they share the same syntactic position. I have nothing to say about the exact syntactic position of \tit{ever}, but in my account it cannot be the head of the relative clause, as this position is reserved for the extra light head. My reason for the incompatibility of \tit{ever} and the light head is that they are semantically incompatible.

For concreteness, I assume \tit{ever} to be situated within the relative clause. Placing it in the main clause generates a different meaning, illustrated by the contrast in meaning between \ref{ex:cz-wh-ever} and \ref{ex:cz-ever-wh} in Czech.

\ex.
\ag. Sním, co -koliv mi uvaříš.\\
 eat.\tsc{1}sg what ever I.\tsc{dat} cook.2\tsc{sg}\\
 `I will eat whatever you will cook for me.'\label{ex:cz-wh-ever}
\bg. Sním co -koliv, co mi uvaříš.\\
 eat.\tsc{1}sg what ever what I.\tsc{dat} cook.2\tsc{sg}\\
 `I will eat anything that you will cook for me.' \flushfill{Czech, \pgcitealt{simik2016}{115}}\label{ex:cz-ever-wh}

\phantom{x}
}

The second argument against the existing light-headed relatives being the source of headless relatives comes from their interpretation. Headless relatives have two possible interpretations, and light-headed relatives have only one of these.
Just like in Modern German, Polish headless relatives can be analyzed as either universal or definite \pgcitep{citko2004}{103}.
Light-headed relatives, such as the one in \ref{ex:pol-light-headed}, only have the definite interpretation.

In sum, just like Modern German, Polish headless relatives do not seem to be derived from light-headed relatives in which the light head is a demonstrative. A difference between Polish and Modern German demonstratives is that Polish ones do not spell out definite features. The fact that Polish demonstratives are also not the light head of a headless relative confirm that deixis features have to be absent from the extra light head.


\section{The Polish relative pronoun}\label{sec:pol-rel}

In Chapter \ref{ch:constituent-containment}, I argued that in the matching type, the features of the light head are spelled out in such a way that they form the constituency shown in Figure \ref{ex:simple-matching-rel}.

\ex.\label{ex:simple-matching-rel}
\begin{forest} boom
  [\tsc{rel}P
      [\tsc{rel}]
      [ϕP
          [ϕ]
          [\tsc{k}P
              [\tsc{k}]
          ]
      ]
  ]
\end{forest}

In Chapter \ref{ch:deriving-onlyinternal}, I argued for Modern German that the functional sequence for the extra light head contains the features in \ref{ex:pol-fseq-rel}.

\ex.\label{ex:pol-fseq-rel}
\begin{adjustbox}{max width=0.9\textwidth}
  \begin{forest} boom
   [\tsc{k}P
       [\tsc{k}]
       [\tsc{rel}P
           [\tsc{rel}]
           [\tsc{wh}P
               [\tsc{wh}]
               [\tsc{med}P
                   [\tsc{dx}\scsub{2}]
                   [\tsc{prox}P
                       [\tsc{dx}\scsub{1}]
                       [\tsc{c}P
                           [\tsc{c}]
                           [\tsc{ind}P
                               [\tsc{ind}]
                               [\tsc{an}P
                                   [\tsc{an}]
                                   [\tsc{cl}P
                                       [\tsc{cl}]
                                       [ΣP
                                            [Σ]
                                            [\tsc{ref}]
                                       ]
                                   ]
                               ]
                           ]
                       ]
                   ]
               ]
           ]
       ]
   ]
  \end{forest}
\end{adjustbox}

In this section, I argue that Polish relative pronouns have the structure shown in \ref{ex:pol-rel}.

\ex.\label{ex:pol-rel}
\begin{adjustbox}{max width=0.9\textwidth}
\begin{forest} boom
  [\tsc{rel}P, s sep=75mm
      [\tsc{rel}P,
      tikz={
      \node[label=below:\tit{k},
      draw,circle,
      scale=1,
      fit to=tree]{};
      }
          [\tsc{rel}]
          [\tsc{wh}P
              [\tsc{wh}]
              [\tsc{med}P
                  [\tsc{dx}\scsub{2}]
                  [\tsc{prox}P
                      [\tsc{dx}\scsub{1}]
                      [\tsc{c}P
                          [\tsc{c}]
                          [\tsc{ind}P
                              [\tsc{ind}]
                              [\tsc{an}P
                                  [\tsc{an}]
                                  [\tsc{cl}]
                              ]
                          ]
                      ]
                  ]
              ]
          ]
      ]
      [\tsc{acc}P, s sep=35mm
      [\tsc{an}P,
          tikz={
          \node[label=below:\tit{o},
          draw,circle,
          scale=0.95,
          fit to=tree]{};
          }
          [\tsc{an}P]
          [\tsc{cl}P
              [\tsc{cl}]
              [ΣP
                  [Σ]
                  [\tsc{ref}]
              ]
          ]
      ]
          [\tsc{k}P,
          tikz={
          \node[label=below:\tit{go/mu},
          draw,circle,
          scale=0.9,
          fit to=tree]{};
          }
              [\tsc{k}]
              [\tsc{ind}P
                  [\tsc{ind}]
              ]
          ]
      ]
  ]
  \end{forest}
  \end{adjustbox}

For the \tit{o} and the final suffix I already gave the lexical entries in the previous section. I discuss which features the \tsc{k} spells out, and I give their lexical entries. \tit{k} spells out all the extra features. In the end, I show how the relative pronouns are constructed.

I argue that \tit{k} spells out five features: \tsc{wh}, \tsc{rel}, \tsc{dx}\scsub{1}, \tsc{dx}\scsub{2} \tsc{ind}. I discuss them one by one.

I start with the operator features \tsc{wh} and \tsc{rel}. The relative pronouns are \tsc{wh}-pronouns, which are also used as interrogatives in Polish. Therefore, just like the Modern German \tit{w}, the Polish \tit{k} spells out the features \tsc{wh} and \tsc{rel}.

I continue with the deixis features. Consider Table \ref{tbl:pol-rel-dem} again.

\begin{table}[htbp]
  \center
  \caption{Polish relative pronouns and demonstratives \pgcitep{swan2002}{160,171}}
  \begin{tabular}[b]{ccc}
    \toprule
              & \tsc{rp}.\tsc{an} & \tsc{dem}.\tsc{m} \\
    \cmidrule{2-3}
    \tsc{nom} & kto               & t-en               \\
    \tsc{acc} & k-ogo             & t-ego              \\
    \tsc{gen} & k-ogo             & t-ego              \\
    \tsc{dat} & k-omu             & t-emu              \\
    \bottomrule
  \end{tabular}
  \label{tbl:pol-rel-dem}
\end{table}

So, the \tit{k} spells out the feature that are contained in the \tit{t} and in the \tit{j}, which are \tsc{c}, \tsc{deix}\scsub{1} and \tsc{deix}\scsub{2}.\footnote{
Unlike Modern German, Polish demonstratives are not marked for definiteness.
}

The demonstratives I gave in Table \ref{tbl:pol-rel-dem} are used as proximal and medial. I give an example in \ref{ex:pol-prox-med}. There is a separate marker for the distal, as shown in \ref{ex:pol-dist}.

\ex.
\ag. to auto\\
 \tsc{dem}.\tsc{prox}/\tsc{med} car.\tsc{n}.\tsc{nom}\\\label{ex:pol-prox-med}
\bg. tam-to auto\\
 \tsc{dem}.\tsc{dist} car.\tsc{n}.\tsc{nom}\\\label{ex:pol-dist}
 \flushfill{Polish, \pgcitealt{wiland2019}{93}}

Finally, since the relative pronouns do not have a morphological plural, I assume that \tit{k} contains the feature \tsc{ind}. Lastly, \tsc{k} also contains \tsc{an} and \tsc{cl}.

In sum, the morpheme \tit{k} realizes the features \tsc{wh}, \tsc{rel}, \tsc{dx}\scsub{1}, \tsc{dx}\scsub{2} and \tsc{ind}.

\ex.\label{ex:pol-entry-k}
\begin{forest} boom
  [\tsc{rel}P
      [\tsc{rel}]
      [\tsc{wh}P
          [\tsc{wh}]
          [\tsc{med}P
              [\tsc{dx}\scsub{2}]
              [\tsc{prox}P
                  [\tsc{dx}\scsub{1}]
                  [\tsc{c}P
                      [\tsc{c}]
                      [\tsc{ind}P
                          [\tsc{ind}]
                          [\tsc{an}P
                              [\tsc{an}]
                              [\tsc{cl}]
                          ]
                      ]
                  ]
              ]
          ]
      ]
  ]
  {\draw (.east) node[right]{⇔ \tit{k}}; }
\end{forest}

how I derive the czego








The next feature in the functional sequence is the feature \tsc{dx}\scsub{1}. The derivation for this feature resembles the derivation of \tsc{dx}\scsub{1} in Modern German.
The feature is merged with the existing syntactic structure, creating a \tsc{prox}P.
This structure does not form a constituent in any of the lexical trees in the language's lexicon, and neither of the spellout driven movements leads to a successful spellout.
Therefore, in a second workspace, the feature \tsc{dx}\scsub{1} is merged with the feature \tsc{ind} (the previous syntactic feature on the functional sequence) into a \tsc{prox}P. This syntactic structure forms a constituent in the lexical tree in \ref{ex:pol-entry-k-rep}, repeated from \ref{ex:pol-entry-k}, which corresponds to \tit{k}.

\ex.\label{ex:pol-entry-k-rep}
\begin{forest} boom
  [\tsc{rel}P
      [\tsc{rel}]
      [\tsc{wh}P
          [\tsc{wh}]
          [\tsc{med}P
              [\tsc{dx}\scsub{2}]
              [\tsc{prox}P
                  [\tsc{dx}\scsub{1}]
                  [\tsc{c}P
                      [\tsc{c}]
                      [\tsc{ind}P
                          [\tsc{ind}]
                          [\tsc{an}P
                              [\tsc{an}]
                              [\tsc{cl}]
                          ]
                      ]
                  ]
              ]
          ]
      ]
  ]
  {\draw (.east) node[right]{⇔ \tit{k}}; }
\end{forest}

Therefore, the \tsc{prox}P is spelled out as \tit{k}. The newly created phrase is merged as a whole with the already existing structure, and projects to the top node, as shown in \ref{ex:pol-spellout-proxp}.

\ex.\label{ex:pol-spellout-proxp}
\begin{adjustbox}{max width=0.9\textwidth}
\begin{forest} boom
  [\tsc{c}P, s sep=30mm
      [\tsc{c}P,
      tikz={
      \node[label=below:\tit{k},
      draw,circle,
      scale=0.95,
      fit to=tree]{};
      }
          [\tsc{c}]
          [\tsc{ind}P
              [\tsc{ind}]
              [\tsc{an}P
                  [\tsc{an}]
                  [\tsc{cl}]
              ]
          ]
      ]
      [\tsc{ind}P, s sep=35mm
      [\tsc{an}P,
          tikz={
          \node[label=below:\tit{o},
          draw,circle,
          scale=0.95,
          fit to=tree]{};
          }
          [\tsc{an}P]
          [\tsc{cl}P
              [\tsc{cl}]
              [ΣP
                  [Σ]
                  [\tsc{ref}]
              ]
          ]
      ]
          [\tsc{ind}P,
          tikz={
          \node[label=below:\tit{go},
          draw,circle,
          scale=0.9,
          fit to=tree]{};
          }
              [\tsc{ind}]
          ]
      ]
  ]
\end{forest}
\end{adjustbox}

The next feature in the functional sequence is the feature \tsc{dx}\scsub{2}. The derivation for this feature resembles the derivation of \tsc{dx}\scsub{2} in Modern German.
The feature is merged with the existing syntactic structure, creating a \tsc{med}P.
This structure does not form a constituent in any of the lexical trees in the language's lexicon, and neither of the spellout driven movements leads to a successful spellout.
Backtracking leads splitting up the \tsc{prox}P from the \tsc{ind}P.
The feature \tsc{dx}\scsub{2} is merged in both workspaces, so with \tsc{prox}P and and with \tsc{ind}P. The spellout of \tsc{dx}\scsub{2} is successful when it is combined with the \tsc{prox}P.
It namely forms a constituent in the lexical tree in \ref{ex:pol-entry-k-rep}, which corresponds to the \tit{k}.
The \tsc{med}P is spelled out as \tit{k}, and the \tsc{med}P is merged back to the existing syntactic structure.

The derivations for the features \tsc{wh} and \tsc{rel} happen the same way.
The feature \tsc{wh} is merged with the existing syntactic structure, creating a \tsc{wh}P.
This structure does not form a constituent in any of the lexical trees in the language's lexicon, and neither of the spellout driven movements leads to a successful spellout.
Backtracking leads splitting up the \tsc{ind}P from the \tsc{ind}P.
The feature \tsc{wh} is merged in both workspaces, so with \tsc{med}P and and with \tsc{ind}P. The spellout of \tsc{wh} is successful when it is combined with the \tsc{med}P.
It namely forms a constituent in the lexical tree in \ref{ex:pol-entry-k-rep}, which corresponds to the \tit{k}.
The \tsc{wh}P is spelled out as \tit{k}, and the \tsc{wh}P is merged back to the existing syntactic structure.

Similarly, the feature \tsc{rel} is merged with the existing syntactic structure, creating a \tsc{rel}P.
This structure does not form a constituent in any of the lexical trees in the language's lexicon, and neither of the spellout driven movements leads to a successful spellout.
Backtracking leads splitting up the \tsc{wh}P from the \tsc{ind}P.
The feature \tsc{rel} is merged in both workspaces, so with \tsc{wh}P and and with \tsc{ind}P. The spellout of \tsc{rel} is successful when it is combined with the \tsc{med}P.
It namely forms a constituent in the lexical tree in \ref{ex:pol-entry-k-rep}, which corresponds to the \tit{k}.
The \tsc{rel}P is spelled out as \tit{k}, and the \tsc{rel}P is merged back to the existing syntactic structure.

The next feature on the functional sequence is \tsc{f}1. This feature should somehow end up merging with \tsc{ind}P, because it forms a constituent in the lexical tree in \ref{ex:pol-entry-go-rep1}, which corresponds to the \tit{go}. This is achieved via Backtracking in which phrases are split up and going through the Spellout Algorithm. I go through the derivation step by step.
The feature \tsc{f}1 is merged with the existing syntactic structure, creating a \tsc{nom}P.
This structure does not form a constituent in any of the lexical trees in the language's lexicon, and neither of the spellout driven movements leads to a successful spellout.
Backtracking leads splitting up the \tsc{rel}P from the \tsc{ind}P.
The feature \tsc{f}1 is merged in both workspaces, so with the \tsc{rel}P and and with the \tsc{ind}P. None of these phrases form a constituent in any of the lexical trees in the language's lexicon.
The first movement option in the Spellout Algorithm is moving the specifier. In the \tsc{rel}P there is no specifier, so this movement option is irrelevant. In the \tsc{ind}P, however, there is a specifier, which is moved to the specifier of \tsc{nom}P.
This syntactic structure forms a constituent in the lexical tree in \ref{ex:pol-entry-go-rep1}, which corresponds to the \tit{go}.
The \tsc{nom}P is spelled out as \tit{go}, and the \tsc{nom}P is merged back to the existing syntactic structure.

For the accusative relative pronoun, the last feature on the functional sequence is the feature \tsc{f}2. Its derivation preceeds the same as the one for the feature \tsc{f}1.
The feature \tsc{f}2 is merged with the existing syntactic structure, creating a \tsc{acc}P.
This structure does not form a constituent in any of the lexical trees in the language's lexicon, and neither of the spellout driven movements leads to a successful spellout.
Backtracking leads splitting up the \tsc{rel}P from the \tsc{nom}P.
The feature \tsc{f}2 is merged in both workspaces, so with the \tsc{rel}P and and with the \tsc{nom}P. None of these phrases form a constituent in any of the lexical trees in the language's lexicon.
The first movement option in the Spellout Algorithm is moving the specifier. In the \tsc{rel}P there is no specifier, so this movement option is irrelevant. In the \tsc{nom}P, however, there is a specifier, which is moved to the specifier of \tsc{acc}P.
This syntactic structure forms a constituent in the lexical tree in \ref{ex:pol-entry-go-rep1}, which corresponds to the \tit{go}.
The \tsc{acc}P is spelled out as \tit{go}, and the \tsc{acc}P is merged back to the existing syntactic structure.

\ex.
\begin{adjustbox}{max width=0.9\textwidth}
\begin{forest} boom
  [\tsc{rel}P, s sep=75mm
      [\tsc{rel}P,
      tikz={
      \node[label=below:\tit{k},
      draw,circle,
      scale=0.95,
      fit to=tree]{};
      }
          [\tsc{rel}]
          [\tsc{wh}P
              [\tsc{wh}]
              [\tsc{med}P
                  [\tsc{dx}\scsub{2}]
                  [\tsc{prox}P
                      [\tsc{dx}\scsub{1}]
                      [\tsc{c}P
                          [\tsc{c}]
                          [\tsc{ind}P
                              [\tsc{ind}]
                              [\tsc{an}P
                                  [\tsc{an}]
                                  [\tsc{cl}]
                              ]
                          ]
                      ]
                  ]
              ]
          ]
      ]
      [\tsc{acc}P, s sep=40mm
      [\tsc{an}P,
          tikz={
          \node[label=below:\tit{o},
          draw,circle,
          scale=0.95,
          fit to=tree]{};
          }
          [\tsc{an}P]
          [\tsc{cl}P
              [\tsc{cl}]
              [ΣP
                  [Σ]
                  [\tsc{ref}]
              ]
          ]
      ]
          [\tsc{acc}P,
          tikz={
          \node[label=below:\tit{go},
          draw,circle,
          scale=0.9,
          fit to=tree]{};
          }
              [\tsc{f}2]
              [\tsc{nom}P
                  [\tsc{f}1]
                  [\tsc{ind}P
                      [\tsc{ind}]
                  ]
              ]
          ]
      ]
  ]
\end{forest}
\end{adjustbox}

For the accusative relative pronoun, the last feature on the functional sequence is the feature \tsc{f}3. Its derivation preceeds the same as the one for the feature \tsc{f}2.
The feature \tsc{f}3 is merged with the existing syntactic structure, creating a \tsc{dat}P.
This structure does not form a constituent in any of the lexical trees in the language's lexicon, and neither of the spellout driven movements leads to a successful spellout.
Backtracking leads splitting up the \tsc{rel}P from the \tsc{acc}P.
The feature \tsc{f}3 is merged in both workspaces, so with the \tsc{rel}P and and with the \tsc{acc}P. None of these phrases form a constituent in any of the lexical trees in the language's lexicon.
The first movement option in the Spellout Algorithm is moving the specifier. In the \tsc{rel}P there is no specifier, so this movement option is irrelevant. In the \tsc{acc}P, however, there is a specifier, which is moved to the specifier of \tsc{dat}P.
This syntactic structure forms a constituent in the lexical tree in \ref{ex:pol-entry-mu-rep1}, which corresponds to the \tit{mu}.

\ex.\label{ex:pol-entry-mu-rep1}
\begin{forest} boom
  [\tsc{dat}P
      [\tsc{f}3]
      [\tsc{acc}P
          [\tsc{f}2]
          [\tsc{nom}P
              [\tsc{f}1]
              [\tsc{ind}P
                  [\tsc{ind}]
              ]
          ]
      ]
  ]
{\draw (.east) node[right]{⇔ \tit{mu}}; }
\end{forest}

The \tsc{dat}P is spelled out as \tit{mu}, and the \tsc{dat}P is merged back to the existing syntactic structure.

\ex. \label{ex:pol-spellout-rel-dat}
\begin{adjustbox}{max width=0.9\textwidth}
\begin{forest} boom
  [\tsc{rel}P, s sep=75mm
      [\tsc{rel}P,
      tikz={
      \node[label=below:\tit{k},
      draw,circle,
      scale=0.95,
      fit to=tree]{};
      }
          [\tsc{rel}]
          [\tsc{wh}P
              [\tsc{wh}]
              [\tsc{med}P
                  [\tsc{dx}\scsub{2}]
                  [\tsc{prox}P
                      [\tsc{dx}\scsub{1}]
                      [\tsc{c}P
                          [\tsc{c}]
                          [\tsc{ind}P
                              [\tsc{ind}]
                              [\tsc{an}P
                                  [\tsc{an}]
                                  [\tsc{cl}]
                              ]
                          ]
                      ]
                  ]
              ]
          ]
      ]
      [\tsc{dat}P, s sep=40mm
      [\tsc{an}P,
          tikz={
          \node[label=below:\tit{o},
          draw,circle,
          scale=0.95,
          fit to=tree]{};
          }
          [\tsc{an}P]
          [\tsc{cl}P
              [\tsc{cl}]
              [ΣP
                  [Σ]
                  [\tsc{ref}]
              ]
          ]
      ]
          [\tsc{dat}P,
          tikz={
          \node[label=below:\tit{mu},
          draw,circle,
          scale=0.9,
          fit to=tree]{};
          }
              [\tsc{f}3]
              [\tsc{acc}P
                  [\tsc{f}2]
                  [\tsc{nom}P
                      [\tsc{f}1]
                      [\tsc{ind}P
                          [\tsc{ind}]
                      ]
                  ]
              ]
          ]
      ]
  ]
\end{forest}
\end{adjustbox}

To summarize, I decomposed the relative pronoun into the three morphemes \tit{k}, \tit{o} and the suffix (\tit{go} and \tit{mu}). I showed which features each of the morphemes spells out, and in which constituents the features are combined. It is these constituents that determine whether the relative pronoun can delete the light head or not.



\section{Comparing Polish constituents}\label{sec:comparing-polish}

In this section, I compare the constituents of extra light heads to those of relative pronouns in Polish. I give three examples, in which the internal and external case vary.
I start with an example with matching cases: the internal and the external case are both accusative.
Then I give an example in which the internal case is more complex than the external case: the internal case is the dative and the external case is the accusative.
I end with an example in which the external case is more complex than the internal case: the internal case is the accusative and the external case is the dative.
In Polish, a matching language, only the first example is grammatical. I derive this by showing that only in this situation the relative pronoun can delete the light head. When the cases match, the light head forms namely a constituent that is contained in the structure of the relative pronoun.

I start with the matching cases.
Consider the example in \ref{ex:polish-acc-acc-rep}, in which the internal accusative case competes against the external accusative case. The relative clause is marked in bold.
The internal case is accusative, as the predicate \tit{lubić} `to like' takes accusative objects. The relative pronoun \tit{kogo} `\ac{rel}.\ac{an}.\ac{acc}' appears in the accusative case. This is the element that surfaces.
The external case is accusative as well, as the predicate \tit{lubić} `to like' also takes accusative objects. The extra light head \tit{ogo} `\ac{elh}.\ac{an}.\ac{acc}' appears in the accusative case. It is placed between square brackets because it does not surface.

\exg. Jan lubi [ogo] \tbf{kogo} \tbf{-kolkwiek} \tbf{Maria} \tbf{lubi}.\\
 Jan like.\tsc{3sg}\scsub{[acc]} \tsc{dem}.\tsc{acc}.\tsc{an}.\tsc{sg}  \tsc{rp}.\tsc{acc}.\tsc{an} ever Maria like.\tsc{3sg}\scsub{[acc]}\\
 `Jan likes whoever Maria likes.' \flushfill{Polish, adapted from \citealt{citko2013} after \pgcitealt{himmelreich2017}{17}}\label{ex:polish-acc-acc-rep}

In Figure \ref{fig:polish-int=ext}, I give the syntactic structure of the extra light head at the top and the syntactic structure of the relative pronoun at the bottom.

\begin{figure}[htbp]
  \center
  \begin{adjustbox}{max height=0.9\textheight}
  \begin{tabular}[b]{c}
        \toprule
        \tsc{acc} extra light head \tit{} \\
        \cmidrule{1-1}
        \begin{forest} boom
          [\tsc{acc}P, s sep=40mm,
          tikz={
          \node[
          draw, constituent-deletion,
          yshift=-0.4cm,
          fill=DG,fill opacity=0.2,
          scale=1.25,
          dashed,
          fit to=tree]{};
          }
              [\tsc{an}P,
              tikz={
              \node[label=below:\tit{o},
              draw,circle,
              scale=0.95,
              fit to=tree]{};
              }
                  [\tsc{an}]
                  [\tsc{cl}P
                      [\tsc{cl}]
                      [ΣP
                          [Σ]
                          [\tsc{ref}]
                      ]
                  ]
              ]
              [\tsc{acc}P,
              tikz={
              \node[label=below:\tit{go},
              draw,circle,
              scale=0.9,
              fit to=tree]{};
              }
                  [\tsc{f}2]
                  [\tsc{nom}P
                      [\tsc{f}1]
                      [\tsc{ind}P
                          [\tsc{ind}]
                      ]
                  ]
              ]
          ]
        \end{forest}
        \vspace{0.3cm}
      \\
      \toprule
      \tsc{acc} relative pronoun \tit{k-o-go}
      \\
      \cmidrule{1-1}
      \begin{forest} boom
        [\tsc{rel}P, s sep=20mm
            [\tsc{rel}P
                [\phantom{x}k\phantom{x}, roof]
            ]
            [\tsc{acc}P, s sep=40mm, tikz={
            \node[
            draw, constituent-deletion, yshift=-0.4cm,
            scale=1.25,
            dashed,
            fit to=tree]{};
            }
                [\tsc{an}P,
                tikz={
                \node[label=below:\tit{o},
                draw,circle,
                scale=0.95,
                fit to=tree]{};
                }
                    [\tsc{an}]
                    [\tsc{cl}P
                        [\tsc{cl}]
                        [ΣP
                            [Σ]
                            [\tsc{ref}]
                        ]
                    ]
                ]
                [\tsc{acc}P,
                tikz={
                \node[label=below:\tit{go},
                draw,circle,
                scale=0.9,
                fit to=tree]{};
                }
                    [\tsc{f}2]
                    [\tsc{nom}P
                        [\tsc{f}1]
                        [\tsc{ind}P
                            [\tsc{ind}]
                        ]
                    ]
                ]
            ]
        ]
      \end{forest}
      \vspace{0.3cm}
      \\
      \bottomrule
  \end{tabular}
  \end{adjustbox}
   \caption {Polish \tsc{ext}\scsub{acc} vs. \tsc{int}\scsub{acc} → \tit{kogo}}
  \label{fig:polish-int=ext}
\end{figure}

The relative pronoun consists of three morphemes: \tit{k}, \tit{o} and \tit{go}.
The extra light head consists of two morphemes: \tit{o} and \tit{go}.
As usual, I circle the part of the structure that corresponds to a particular lexical entry, and I place the corresponding phonology under it.
I draw a dashed circle around each constituent that is a constituent in both the extra light head and the relative pronoun.

The extra light head consists of two constituents: the \tsc{an}P and the (lower) \tsc{acc}P. Together they form the (higher) \tsc{acc}P.
This \tsc{acc}P is also a constituent within the relative pronoun. Therefore, the relative pronoun can delete the extra light head. I signal the deletion of the extra light head by marking the content of its circle gray.

I continue with the example in which the internal case is more complex than the external case.
Consider the examples in \ref{ex:polish-acc-dat-rep}, in which the internal dative case competes against the external accusative case. The relative clauses are marked in bold. It is not possible to make a grammatical headless relative in this situation.
The internal case is dative, as the predicate \tit{dokuczać} `to tease' takes dative objects. The relative pronoun \tit{komu} `\ac{rel}.\ac{an}.\ac{dat}' appears in the dative case.
The external case is accusative, as the predicate \tit{lubić} `to like' takes accusative objects. The extra light head \tit{ogo} `\ac{elh}.\ac{an}.\ac{acc}' appears in the accusative case.
\ref{ex:polish-acc-dat-rel} is the variant of the sentence in which the extra light head is absent (indicated by the square brackets) and the relative pronoun surfaces, and it is ungrammatical.
\ref{ex:polish-acc-dat-lh} is the variant of the sentence in which the relative pronoun is absent (indicated by the square brackets) and the extra light head surfaces, and it is ungrammatical too.

\ex.\label{ex:polish-acc-dat-rep}
\ag. *Jan lubi [ogo] \tbf{komu} \tbf{-kolkwiek} \tbf{dokucza}.\\
Jan like.\tsc{3sg}\scsub{[acc]} \tsc{elh}.\tsc{acc}.\tsc{an} \tsc{rp}.\tsc{dat}.\tsc{an}.\tsc{sg} ever tease.\tsc{3sg}\scsub{[dat]}\\
`Jan likes whoever he teases.' \flushfill{Polish, adapted from \citealt{citko2013} after \pgcitealt{himmelreich2017}{17}}\label{ex:polish-acc-dat-rel}
\bg. *Jan lubi ogo [\tbf{komu}] \tbf{-kolkwiek} \tbf{dokucza}.\\
Jan like.\tsc{3sg}\scsub{[acc]} \tsc{elh}.\tsc{acc}.\tsc{an} \tsc{rp}.\tsc{dat}.\tsc{an}.\tsc{sg} ever tease.\tsc{3sg}\scsub{[dat]}\\
`Jan likes whoever he teases.' \flushfill{Polish, adapted from \citealt{citko2013} after \pgcitealt{himmelreich2017}{17}}\label{ex:polish-acc-dat-lh}

In Figure \ref{fig:polish-int-wins}, I give the syntactic structure of the extra light head at the top and the syntactic structure of the relative pronoun at the bottom.

\begin{figure}[htbp]
  \center
  \begin{adjustbox}{max height=0.9\textheight}
  \begin{tabular}[b]{c}
        \toprule
        \tsc{acc} extra light head \tit{o-go} \\
        \cmidrule{1-1}
        \begin{forest} boom
          [\tsc{acc}P, s sep=42mm
              [\tsc{an}P,
              tikz={
              \node[label=below:\tit{o},
              draw,circle,
              scale=0.95,
              fit to=tree]{};
              \node[
              draw,circle,
              scale=1,
              dashed,
              fit to=tree]{};
              }
                  [\tsc{an}]
                  [\tsc{cl}P
                      [\tsc{cl}]
                      [ΣP
                          [Σ]
                          [\tsc{ref}]
                      ]
                  ]
              ]
              [\tsc{acc}P,
              tikz={
              \node[label=below:\tit{go},
              draw,circle,
              scale=0.9,
              fit to=tree]{};
              \node[
              draw,circle,
              scale=0.95,
              dashed,
              fit to=tree]{};
              }
                  [\tsc{f}2]
                  [\tsc{nom}P
                      [\tsc{f}1]
                      [\tsc{ind}P
                          [\tsc{ind}]
                      ]
                  ]
              ]
          ]
        \end{forest}
        \vspace{0.3cm}
      \\
      \toprule
      \tsc{acc} relative pronoun \tit{k-o-mu}
      \\
      \cmidrule{1-1}
      \begin{forest} boom
        [\tsc{rel}P, s sep=15mm
            [\tsc{rel}P
                [\phantom{x}k\phantom{x}, roof]
            ]
            [\tsc{dat}P, s sep=45mm
                [\tsc{an}P,
                tikz={
                \node[label=below:\tit{o},
                draw,circle,
                scale=0.95,
                fit to=tree]{};
                \node[
                draw,circle,
                scale=1,
                dashed,
                fit to=tree]{};
                }
                    [\tsc{an}]
                    [\tsc{cl}P
                        [\tsc{cl}]
                        [ΣP
                            [Σ]
                            [\tsc{ref}]
                        ]
                    ]
                ]
                [\tsc{dat}P,
                tikz={
                \node[label=below:\tit{mu},
                draw,circle,
                scale=0.95,
                fit to=tree]{};
                }
                    [\tsc{f}3]
                    [\tsc{acc}P, tikz={
                    \node[
                    draw,circle,
                    scale=0.9,
                    dashed,
                    fit to=tree]{};
                    }
                        [\tsc{f}2]
                        [\tsc{nom}P
                            [\tsc{f}1]
                            [\tsc{ind}P
                                [\tsc{ind}]
                            ]
                        ]
                    ]
                ]
            ]
        ]
      \end{forest}
      \\
      \bottomrule
  \end{tabular}
  \end{adjustbox}
   \caption {Polish \tsc{ext}\scsub{acc} vs. \tsc{int}\scsub{dat} ↛ \tit{ogo}/\tit{komu}}
  \label{fig:polish-int-wins}
\end{figure}

The relative pronoun consists of three morphemes: \tit{k}, \tit{o} and \tit{mu}.
The light head consists of two morphemes: \tit{o} and \tit{go}.
Again, I circle the part of the structure that corresponds to a particular lexical entry, and I place the corresponding phonology under it.
I draw a dashed circle around each constituent that is a constituent in both the extra light head and the relative pronoun.

The extra light head consists of two constituents: the \tsc{an}P and the (lower) \tsc{acc}P. Together they form the (higher) \tsc{acc}P.
Both of these constituents are also constituents within the relative pronoun. However, the (higher) \tsc{acc}P is not a constituent within the relative pronoun. The constituent in which the \tsc{acc}P is contained namely also contains the feature \tsc{f}3 that makes it a \tsc{dat}P.
In other words, each feature and even each constituent of the extra light head is contained in the relative pronoun. However, they are not contained in the relative pronoun as a single constituent. Therefore, the relative pronoun cannot delete the extra light head.

Recall from Section \ref{sec:comparing-mg} that this is the crucial example in which Modern German and Polish differ. The contrast lies in that the extra light head in Modern German forms a single constituent and in Polish it forms two constituents. In Modern German, relative pronouns in a more complex case contain the extra light head in a less complex case as a single constituent. In Polish, they do not. Relative pronouns in a complex case still contain all features of an extra light head in a less complex case, but the extra light head is not a single constituent within the relative pronoun. That is, the weaker feature containment requirement is met, but the stronger constituent containment requirement is not. This shows the necessity of formulating the proposal in terms of containment as a single constituent.

I continue with the example in which the external case is more complex than the internal case.
Consider the examples in \ref{ex:polish-dat-acc-rep}, in which the internal dative case competes against the external accusative case. The relative clauses are marked in bold. It is not possible to make a grammatical headless relative in this situation.
The internal case is accusative, as the predicate \tit{wpuścić} `to let' takes accusative objects. The relative pronoun \tit{kogo} `\ac{rel}.\ac{an}.\ac{acc}' appears in the accusative case.
The external case is dative, as the predicate \tit{ufać} `to trust' takes dative objects. The extra light head \tit{omu} `\ac{elh}.\ac{an}.\ac{dat}' appears in the dative case.
\ref{ex:polish-dat-acc-rel} is the variant of the sentence in which the extra light head is absent (indicated by the square brackets) and the relative pronoun surfaces, and it is ungrammatical.
\ref{ex:polish-dat-acc-lh} is the variant of the sentence in which the relative pronoun is absent (indicated by the square brackets) and the extra light head surfaces, and it is ungrammatical too.

\ex.\label{ex:polish-dat-acc-rep}
\ag. *Jan ufa [omu] \tbf{kogo} \tbf{-kolkwiek} \tbf{wpuścil} \tbf{do} \tbf{domu}.\\
Jan trust.\tsc{3sg}\scsub{[dat]} \tsc{elh}.\tsc{dat}.\tsc{an} \tsc{rp}.\tsc{acc}.\tsc{an} ever let.\tsc{3sg}\scsub{[acc]} to home\\
`Jan trusts whoever he let into the house.' \flushfill{Polish, adapted from \citealt{citko2013} after \pgcitealt{himmelreich2017}{17}}\label{ex:polish-dat-acc-rel}
\bg. Jan ufa omu [\tbf{kogo}] \tbf{-kolkwiek} \tbf{wpuścil} \tbf{do} \tbf{domu}.\\
Jan trust.\tsc{3sg}\scsub{[dat]} \tsc{elh}.\tsc{dat}.\tsc{an} \tsc{rp}.\tsc{acc}.\tsc{an} ever let.\tsc{3sg}\scsub{[acc]} to home\\
`Jan trusts whoever he let into the house.' \flushfill{Polish, adapted from \citealt{citko2013} after \pgcitealt{himmelreich2017}{17}}\label{ex:polish-dat-acc-lh}

In Figure \ref{fig:polish-ext-wins}, I give the syntactic structure of the extra light head at the top and the syntactic structure of the relative pronoun at the bottom.

\begin{figure}[htbp]
  \center
  \begin{adjustbox}{max height=0.9\textheight}
  \begin{tabular}[b]{c}
        \toprule
        \tsc{dat} extra light head \tit{o-mu} \\
        \cmidrule{1-1}
        \begin{forest} boom
          [\tsc{dat}P, s sep=45mm
              [\tsc{an}P,
              tikz={
              \node[label=below:\tit{o},
              draw,circle,
              scale=0.95,
              fit to=tree]{};
              \node[
              draw,circle,
              scale=1,
              dashed,
              fit to=tree]{};
              }
                  [\tsc{an}]
                  [\tsc{cl}P
                      [\tsc{cl}]
                      [ΣP
                          [Σ]
                          [\tsc{ref}]
                      ]
                  ]
              ]
              [\tsc{dat}P,
              tikz={
              \node[label=below:\tit{mu},
              draw,circle,
              scale=0.95,
              fit to=tree]{};
              }
                  [\tsc{f}3]
                  [\tsc{acc}P,
                  tikz={
                  \node[
                  draw,circle,
                  scale=0.9,
                  dashed,
                  fit to=tree]{};
                  }
                      [\tsc{f}2]
                      [\tsc{nom}P
                          [\tsc{f}1]
                          [\tsc{ind}P
                              [\tsc{ind}]
                          ]
                      ]
                  ]
              ]
          ]
        \end{forest}
        \vspace{0.3cm}
      \\
      \toprule
      \tsc{acc} relative pronoun \tit{k-o-go}
      \\
      \cmidrule{1-1}
      \begin{forest} boom
        [\tsc{rel}P, s sep=15mm
            [\tsc{rel}P
                [\phantom{x}k\phantom{x}, roof]
            ]
            [\tsc{acc}P, s sep=42mm
                [\tsc{an}P,
                tikz={
                \node[label=below:\tit{o},
                draw,circle,
                scale=0.95,
                fit to=tree]{};
                \node[
                draw,circle,
                scale=1,
                dashed,
                fit to=tree]{};
                }
                    [\tsc{an}]
                    [\tsc{cl}P
                        [\tsc{cl}]
                        [ΣP
                            [Σ]
                            [\tsc{ref}]
                        ]
                    ]
                ]
                [\tsc{acc}P,
                tikz={
                \node[label=below:\tit{go},
                draw,circle,
                scale=0.9,
                fit to=tree]{};
                \node[
                draw,circle,
                scale=0.95,
                dashed,
                fit to=tree]{};
                }
                    [\tsc{f}2]
                    [\tsc{nom}P
                        [\tsc{f}1]
                        [\tsc{ind}P
                            [\tsc{ind}]
                        ]
                    ]
                ]
            ]
        ]
      \end{forest}
      \\
      \bottomrule
  \end{tabular}
  \end{adjustbox}
   \caption {Polish \tsc{ext}\scsub{dat} vs. \tsc{int}\scsub{acc} ↛ \tit{omu}/\tit{kogo}}
  \label{fig:polish-ext-wins}
\end{figure}

The relative pronoun consists of three morphemes: \tit{k}, \tit{o} and \tit{go}.
The light head consists of two morphemes: \tit{o} and \tit{mu}.
Again, I circle the part of the structure that corresponds to a particular lexical entry, and I place the corresponding phonology under it.
I draw a dashed circle around each constituent that is a constituent in both the extra light head and the relative pronoun.

The extra light head consists of two constituents: the \tsc{an}P and the (lower) \tsc{dat}P.
In this case, the relative pronoun does not contain both these constituents. The relative pronoun only contains the \tsc{acc}P, and it lacks the \tsc{f}3 that makes a \tsc{dat}P. Since the weaker feature containment requirement is not met, the stronger constituent requirement cannot be met either.
The extra light head also does not contain all constituents or features that the relative pronoun contains, because it lacks the complete \tsc{rel}P.
Therefore, the relative pronoun cannot delete the extra light head, and the extra light head can also not delete the relative pronoun.

\section{Summary}
