% !TEX root = thesis.tex

% all subconstituents need to contain other constituents

\section{option 1}

fseq: \tsc{ref} --- (\tsc{wh}) --- (\tsc{ana}) --- \tsc{class} --- \tsc{masc} --- \tsc{ind} --- (\tsc{d}) --- \tsc{f1} --- \tsc{f2} --- \tsc{f3}

\begin{itemize}
  \item I need to have the \tsc{wh} right after the \tsc{ref}, because otherwise I don't know how to get \tit{k} to overwrite \tit{t}. This means that \tsc{d} and \tsc{wh} are at quite different places. I don't have any idea about how acceptable my fseq is like this.
  \item I need a feature for the difference between \tsc{dem} and \tsc{wh}s in Polish: \tsc{ana}. I did not want to put it right below \tsc{k}, because then I don't have space anymore to let the \tit{mu} backtrack into the vowel. However, it needs to be there if I want to let the \tit{o} contain the \tit{e}. I don't see how I can have them both. Now I put it under \tsc{class}, which already causes \tit{o} not to contain \tit{e}. That means that I also exclude the situations in which there are matching cases. Another reason for not letting \tit{mu} backtrack into the vowel is that it is syncretic between neuter and masculine, which can be captured by letting \tsc{masc} be the highest features (that's how I have it for \tsc{dat}) below now.
  \item I don't see of a way of make the Polish non-matching ones ungrammatical but making the matching ones grammatical. I can exclude them all, and find a different reason for why the matching ones are ok.
  \item For Modern German, I now let \tsc{wh} form a complex spec. When \tsc{ana} is merged, backtracking opens the two structures up, and \tsc{ana} merges with \tsc{wh}. It's not necessary important here, but I do not see how I could ever get \tsc{wa} for for instance \tit{was}. How do I let gender play a role here? I need them all in my structure on the right.
  \item Old High German differs from the two other languages in that it has a more complex light head, and that it has \tsc{d} instead of \tsc{wh}. You could see \tsc{d} and \tsc{wh} as two instances of \tsc{rel}.
\end{itemize}


\begin{table}[H]
\begin{tabular}{lll}
                & light head
                & relative pronoun                                                                    \\
Polish          & \tsc{ref} \tsc{class} \tsc{masc} \tsc{ind} \tsc{f}s
                & \tsc{ref} \tsc{\tbf{wh}} \tsc{\tbf{ana}} \tsc{class} \tsc{masc} \tsc{ind} \tsc{f}s  \\
Modern German   & \tsc{ref} \tsc{class} \tsc{masc} \tsc{ind} \tsc{f}s
                & \tsc{ref} \tsc{\tbf{wh}} \tsc{\tbf{ana}} \tsc{class} \tsc{masc} \tsc{ind} \tsc{f}s  \\
Old High German & \tsc{ref} \tsc{\tbf{ana}} \tsc{class} \tsc{masc} \tsc{ind} \tsc{\tbf{d}} \tsc{f}s
                & \tsc{ref} \tsc{\tbf{ana}} \tsc{class} \tsc{masc} \tsc{ind} \tsc{\tbf{d}} \tsc{f}s   \\
\end{tabular}
\end{table}

\clearpage

\ex. Polish: \tsc{ext} \tsc{dat}\\
\begin{forest} boom
  [\tsc{ref}P, s sep=30mm
      [\tsc{ref}P, s sep=20mm
          [\tsc{ref}P,
          tikz={
          \node[label=below:\tit{t},
          draw,circle,
          scale=0.85,
          fit to=tree]{};
          }
              [\tsc{ref}, roof]
          ]
          [\tsc{masc}P,
          tikz={
          \node[label=below:\tit{e},
          draw,circle,
          scale=0.85,
          fit to=tree]{};
          }
              [\tsc{masc}]
              [\tsc{class}P
                  [\tsc{class}]
              ]
          ]
      ]
      [\tsc{dat}P,
      tikz={
      \node[label=below:\tit{mu},
      draw,circle,
      scale=0.85,
      fit to=tree]{};
      }
          [\tsc{f}4]
          [\tsc{gen}P
              [\tsc{f}3]
              [\tsc{acc}P
                  [\tsc{f}2]
                  [\tsc{nom}P
                      [\tsc{f}1]
                      [\tsc{ind}P
                          [\tsc{ind}]
                      ]
                  ]
              ]
          ]
      ]
  ]
\end{forest}

\ex. Polish: \tsc{ext} \tsc{acc}\\
\begin{forest} boom
  [\tsc{ref}P, s sep=40mm
      [\tsc{ref}P, s sep=20mm
          [\tsc{ref}P,
          tikz={
          \node[label=below:\tit{t},
          draw,circle,
          scale=0.85,
          fit to=tree]{};
          }
              [\tsc{ref}, roof]
          ]
          [\tsc{ind}P,
          tikz={
          \node[label=below:\tit{e},
          draw,circle,
          scale=0.85,
          fit to=tree]{};
          }
              [\tsc{ind}]
              [\tsc{masc}P
                  [\tsc{masc}]
                  [\tsc{class}P
                      [\tsc{class}]
                  ]
              ]
          ]
      ]
      [\tsc{acc}P,
      tikz={
      \node[label=below:\tit{go},
      draw,circle,
      scale=0.85,
      fit to=tree]{};
      }
          [\tsc{f}2]
          [\tsc{nom}P
              [\tsc{f}1]
          ]
      ]
  ]
\end{forest}



\ex. Polish: \tsc{int} \tsc{dat}\\
\begin{forest} boom
  [\tsc{wh}P, s sep=40mm
      [\tsc{wh}P, s sep=20mm
          [\tsc{wh}P,
          tikz={
          \node[label=below:\tit{k},
          draw,circle,
          scale=0.85,
          fit to=tree]{};
          }
              [\tsc{wh}]
              [\tsc{ref}P,
                  [\tsc{ref}, roof]
              ]
          ]
          [\tsc{masc}P,
          tikz={
          \node[label=below:\tit{o},
          draw,circle,
          scale=0.85,
          fit to=tree]{};
          }
              [\tsc{masc}]
              [\tsc{class}P
                  [\tsc{class}]
                  [\tsc{ana}P
                      [\tsc{ana}]
                  ]
              ]
          ]
      ]
      [\tsc{dat}P,
      tikz={
      \node[label=below:\tit{mu},
      draw,circle,
      scale=0.85,
      fit to=tree]{};
      }
          [\tsc{f}4]
          [\tsc{gen}P
              [\tsc{f}3]
              [\tsc{acc}P
                  [\tsc{f}2]
                  [\tsc{nom}P
                      [\tsc{f}1]
                      [\tsc{ind}P
                          [\tsc{ind}]
                      ]
                  ]
              ]
          ]
      ]
  ]
\end{forest}

\ex. Polish: \tsc{int} \tsc{acc}\\
\begin{forest} boom
  [\tsc{wh}P, s sep=40mm
      [\tsc{wh}P, s sep=20mm
          [\tsc{wh}P,
          tikz={
          \node[label=below:\tit{k},
          draw,circle,
          scale=0.85,
          fit to=tree]{};
          }
              [\tsc{wh}]
              [\tsc{ref}P,
                  [\tsc{ref}, roof]
              ]
          ]
          [\tsc{ind}P,
          tikz={
          \node[label=below:\tit{o},
          draw,circle,
          scale=0.85,
          fit to=tree]{};
          }
              [\tsc{ind}]
              [\tsc{masc}P
                  [\tsc{masc}]
                  [\tsc{class}P
                      [\tsc{class}]
                      [\tsc{ana}P
                          [\tsc{ana}]
                      ]
                  ]
              ]
          ]
      ]
      [\tsc{acc}P,
      tikz={
      \node[label=below:\tit{go},
      draw,circle,
      scale=0.85,
      fit to=tree]{};
      }
          [\tsc{f}2]
          [\tsc{nom}P
              [\tsc{f}1]
          ]
      ]
  ]
\end{forest}


\ex. Modern German: \tsc{ext} \tsc{acc}\\
\begin{forest} boom
  [\tsc{acc}P,
  tikz={
  \node[label=below:\tit{n},
  draw,circle,
  scale=0.85,
  fit to=tree]{};
  }
      [\tsc{f}2]
      [\tsc{nom}P
          [\tsc{f}1]
          [\tsc{ind}P
              [\tsc{ind}]
              [\tsc{masc}P
                  [\tsc{masc}]
                  [\tsc{class}P
                      [\tsc{class}]
                      [\tsc{ref}P,
                          [\tsc{ref}, roof]
                      ]
                  ]
              ]
          ]
      ]
  ]
\end{forest}

\ex. Modern German: \tsc{ext} \tsc{nom}\\
\begin{forest} boom
  [\tsc{nom}P,
  tikz={
  \node[label=below:\tit{r},
  draw,circle,
  scale=0.85,
  fit to=tree]{};
  }
      [\tsc{f}1]
      [\tsc{ind}P
          [\tsc{ind}]
          [\tsc{masc}P
              [\tsc{masc}]
              [\tsc{class}P
                  [\tsc{class}]
                  [\tsc{ref}P,
                      [\tsc{ref}, roof]
                  ]
              ]
          ]
      ]
  ]
\end{forest}

\ex. Modern German: \tsc{int} \tsc{acc}\\
\begin{forest} boom
  [\tsc{wh}P
      [\tsc{wh}P,
      tikz={
      \node[label=below:\tit{we},
      draw,circle,
      scale=0.85,
      fit to=tree]{};
      }
          [\tsc{ana}]
          [\tsc{wh}]
      ]
      [\tsc{acc}P,
      tikz={
      \node[label=below:\tit{n},
      draw,circle,
      scale=0.85,
      fit to=tree]{};
      }
          [\tsc{f}2]
          [\tsc{nom}P
              [\tsc{f}1]
              [\tsc{ind}P
                  [\tsc{ind}]
                  [\tsc{masc}P
                      [\tsc{masc}]
                      [\tsc{class}P
                          [\tsc{class}]
                          [\tsc{ref}P,
                              [\tsc{ref}, roof]
                          ]
                      ]
                  ]
              ]
          ]
      ]
  ]
\end{forest}

\ex. Modern German: \tsc{int} \tsc{nom}\\
\begin{forest} boom
  [\tsc{wh}P
      [\tsc{wh}P,
      tikz={
      \node[label=below:\tit{we},
      draw,circle,
      scale=0.85,
      fit to=tree]{};
      }
          [\tsc{ana}]
          [\tsc{wh}]
      ]
      [\tsc{nom}P,
      tikz={
      \node[label=below:\tit{r},
      draw,circle,
      scale=0.85,
      fit to=tree]{};
      }
          [\tsc{f}1]
          [\tsc{ind}P
              [\tsc{ind}]
              [\tsc{masc}P
                  [\tsc{masc}]
                  [\tsc{class}P
                      [\tsc{class}]
                      [\tsc{ref}P,
                          [\tsc{ref}, roof]
                      ]
                  ]
              ]
          ]
      ]
  ]
\end{forest}




\ex. Old High German: \tsc{int}/\tsc{ext} \tsc{acc}\\
\begin{forest} boom
  [\tsc{d}P
      [\tsc{d}P,
      tikz={
      \node[label=below:\tit{d},
      draw,circle,
      scale=0.85,
      fit to=tree]{};
      }
          [\tsc{d}, roof]
      ]
      [\tsc{ana}P
          [\tsc{ana}P,
          tikz={
          \node[label=below:\tit{e},
          draw,circle,
          scale=0.85,
          fit to=tree]{};
          }
              [\tsc{ana}, roof]
          ]
          [\tsc{acc}P,
          tikz={
          \node[label=below:\tit{n},
          draw,circle,
          scale=0.85,
          fit to=tree]{};
          }
              [\tsc{f}2]
              [\tsc{nom}P
                  [\tsc{f}1]
                  [\tsc{ind}P
                      [\tsc{ind}]
                      [\tsc{masc}P
                          [\tsc{masc}]
                          [\tsc{class}P
                              [\tsc{class}]
                              [\tsc{ref}P,
                                  [\tsc{ref}, roof]
                              ]
                          ]
                      ]
                  ]
              ]
          ]
      ]
  ]
\end{forest}


\ex. Old High German: \tsc{int}/\tsc{ext} \tsc{nom}\\
\begin{forest} boom
  [\tsc{d}P
      [\tsc{d}P,
      tikz={
      \node[label=below:\tit{d},
      draw,circle,
      scale=0.85,
      fit to=tree]{};
      }
          [\tsc{d}, roof]
      ]
      [\tsc{ana}P
          [\tsc{ana}P,
          tikz={
          \node[label=below:\tit{e},
          draw,circle,
          scale=0.85,
          fit to=tree]{};
          }
              [\tsc{ana}, roof]
          ]
          [\tsc{nom}P,
          tikz={
          \node[label=below:\tit{r},
          draw,circle,
          scale=0.85,
          fit to=tree]{};
          }
              [\tsc{f}1]
              [\tsc{ind}P
                  [\tsc{ind}]
                  [\tsc{masc}P
                      [\tsc{masc}]
                      [\tsc{class}P
                          [\tsc{class}]
                          [\tsc{ref}P,
                              [\tsc{ref}, roof]
                          ]
                      ]
                  ]
              ]
          ]
      ]
  ]
\end{forest}

\clearpage

\section{option 2}

fseq: \tsc{ref} --- (\tsc{ana}) --- \tsc{class} --- \tsc{masc} --- \tsc{ind} --- \tsc{x} --- (\tsc{wh}) --- (\tsc{d}) --- \tsc{f1} --- \tsc{f2} --- \tsc{f3}

\begin{itemize}
  \item I introduced an \tsc{x} that corresponds to Polish \tit{t}. That way, \tit{t} corresponds to features that form a complex spec, and I can merge it with features later (like \tsc{wh}/\tsc{d}). \tsc{ana} is still low, because it needs to change the vowel in Polish, but it can't be at the top, because the shrinking of \tit{e}/\tit{o} needs to take place there.
\end{itemize}

\begin{table}[H]
\begin{tabular}{lll}
                & light head
                & relative pronoun                                                                    \\
Polish          & \tsc{ref} \tsc{class} \tsc{masc} \tsc{ind} \tsc{x} \tsc{f}s
                & \tsc{ref} \tsc{\tbf{ana}} \tsc{class} \tsc{masc} \tsc{ind} \tsc{x} \tsc{\tbf{wh}} \tsc{f}s  \\
Modern German   & \tsc{ref} \tsc{class} \tsc{masc} \tsc{ind} \tsc{x} \tsc{f}s
                & \tsc{ref} \tsc{\tbf{ana}} \tsc{class} \tsc{masc} \tsc{ind} \tsc{x} \tsc{\tbf{wh}} \tsc{f}s  \\
Old High German & \tsc{ref} \tsc{\tbf{ana}} \tsc{class} \tsc{masc} \tsc{ind} \tsc{x} \tsc{\tbf{d}} \tsc{f}s
                & \tsc{ref} \tsc{\tbf{ana}} \tsc{class} \tsc{masc} \tsc{ind} \tsc{x} \tsc{\tbf{d}} \tsc{f}s  \\
\end{tabular}
\end{table}

\begin{itemize}
  \item Polish: \tit{t} is whatever features \tsc{x} corresponds to, some kind of indefinite demonstrative. When \tsc{x} is merged, it forms a complex spec. Features merged after \tsc{x} get into the right tree via backtracking and \tsc{x} and the rest up. \tsc{ana} is merged low, so the right vowel is already chosen quite early. The switch between \tit{mu} and \tit{go} is an interesting one: it happens when some case features are already merged, and then backtracking takes place all the way back until before the complex spec was merged.
  \item

\end{itemize}

\ex. Polish: \tsc{ext} \tsc{dat}\\
\begin{forest} boom
  [\tsc{x}P, s sep=20mm
      [\tsc{x}P,
      tikz={
      \node[label=below:\tit{t},
      draw,circle,
      scale=0.85,
      fit to=tree]{};
      }
          [\tsc{x}, roof]
      ]
      [\tsc{dat}P, s sep=40mm
          [\tsc{masc}P,
          tikz={
          \node[label=below:\tit{e},
          draw,circle,
          scale=0.85,
          fit to=tree]{};
          }
              [\tsc{masc}]
              [\tsc{class}P
                  [\tsc{class}]
                  [\tsc{ref}P,
                      [\tsc{ref}, roof]
                  ]
              ]
          ]
          [\tsc{dat}P,
          tikz={
          \node[label=below:\tit{mu},
          draw,circle,
          scale=0.85,
          fit to=tree]{};
          }
              [\tsc{f}4]
              [\tsc{gen}P
                  [\tsc{f}3]
                  [\tsc{acc}P
                      [\tsc{f}2]
                      [\tsc{nom}P
                          [\tsc{f}1]
                          [\tsc{ind}P
                              [\tsc{ind}]
                          ]
                      ]
                  ]
              ]
          ]
      ]
  ]
\end{forest}

\ex. Polish: \tsc{ext} \tsc{acc}\\
\begin{forest} boom
  [\tsc{x}P, s sep=20mm
      [\tsc{x}P,
      tikz={
      \node[label=below:\tit{t},
      draw,circle,
      scale=0.85,
      fit to=tree]{};
      }
          [\tsc{x}, roof]
      ]
      [\tsc{acc}P, s sep=40mm
          [\tsc{ind}P,
          tikz={
          \node[label=below:\tit{e},
          draw,circle,
          scale=0.85,
          fit to=tree]{};
          }
              [\tsc{ind}]
              [\tsc{mascP}
                  [\tsc{masc}]
                  [\tsc{class}P
                      [\tsc{class}]
                      [\tsc{ref}P,
                          [\tsc{ref}, roof]
                      ]
                  ]
              ]
          ]
          [\tsc{acc}P,
          tikz={
          \node[label=below:\tit{go},
          draw,circle,
          scale=0.85,
          fit to=tree]{};
          }
              [\tsc{f}2]
              [\tsc{nom}P
                  [\tsc{f}1]
              ]
          ]
      ]
  ]
\end{forest}

\ex. Polish: \tsc{int} \tsc{dat}\\
\begin{forest} boom
  [\tsc{x}P, s sep=20mm
      [\tsc{x}P,
      tikz={
      \node[label=below:\tit{k},
      draw,circle,
      scale=0.85,
      fit to=tree]{};
      }
          [\tsc{wh}]
          [\tsc{x}P,
              [\tsc{x}, roof]
          ]
      ]
      [\tsc{dat}P, s sep=40mm
          [\tsc{masc}P,
          tikz={
          \node[label=below:\tit{o},
          draw,circle,
          scale=0.85,
          fit to=tree]{};
          }
              [\tsc{masc}]
              [\tsc{class}P
                  [\tsc{class}]
                  [\tsc{ana}P
                      [\tsc{ana}]
                      [\tsc{ref}P,
                          [\tsc{ref}, roof]
                      ]
                  ]
              ]
          ]
          [\tsc{dat}P,
          tikz={
          \node[label=below:\tit{mu},
          draw,circle,
          scale=0.85,
          fit to=tree]{};
          }
              [\tsc{f}4]
              [\tsc{gen}P
                  [\tsc{f}3]
                  [\tsc{acc}P
                      [\tsc{f}2]
                      [\tsc{nom}P
                          [\tsc{f}1]
                          [\tsc{ind}P
                              [\tsc{ind}]
                          ]
                      ]
                  ]
              ]
          ]
      ]
  ]
\end{forest}

\ex. Polish: \tsc{int} \tsc{acc}\\
\begin{forest} boom
  [\tsc{x}P, s sep=20mm
      [\tsc{x}P,
      tikz={
      \node[label=below:\tit{k},
      draw,circle,
      scale=0.85,
      fit to=tree]{};
      }
          [\tsc{wh}]
          [\tsc{x}P,
              [\tsc{x}, roof]
          ]
      ]
      [\tsc{acc}P, s sep=40mm
          [\tsc{masc}P,
          tikz={
          \node[label=below:\tit{o},
          draw,circle,
          scale=0.85,
          fit to=tree]{};
          }
              [\tsc{masc}]
              [\tsc{class}P
                  [\tsc{class}]
                  [\tsc{ana}P
                      [\tsc{ana}]
                      [\tsc{ref}P,
                          [\tsc{ref}, roof]
                      ]
                  ]
              ]
          ]
          [\tsc{acc}P,
          tikz={
          \node[label=below:\tit{go},
          draw,circle,
          scale=0.85,
          fit to=tree]{};
          }
              [\tsc{f}2]
              [\tsc{nom}P
                  [\tsc{f}1]
              ]
          ]
      ]
  ]
\end{forest}



\ex. Modern German: \tsc{ext} \tsc{acc}\\
\begin{forest} boom
  [\tsc{acc}P,
  tikz={
  \node[label=below:\tit{n},
  draw,circle,
  scale=0.85,
  fit to=tree]{};
  }
      [\tsc{f}2]
      [\tsc{nom}P
          [\tsc{f}1]
          [\tsc{x}
              [\tsc{xp}]
              [\tsc{ind}P
                  [\tsc{ind}]
                  [\tsc{masc}P
                      [\tsc{masc}]
                      [\tsc{class}P
                          [\tsc{class}]
                          [\tsc{ref}P,
                              [\tsc{ref}, roof]
                          ]
                      ]
                  ]
              ]
          ]
      ]
  ]
\end{forest}

\ex. Modern German: \tsc{ext} \tsc{nom}\\
\begin{forest} boom
  [\tsc{nom}P,
  tikz={
  \node[label=below:\tit{r},
  draw,circle,
  scale=0.85,
  fit to=tree]{};
  }
      [\tsc{f}1]
      [\tsc{x}
          [\tsc{xp}]
          [\tsc{ind}P
              [\tsc{ind}]
              [\tsc{masc}P
                  [\tsc{masc}]
                  [\tsc{class}P
                      [\tsc{class}]
                      [\tsc{ref}P,
                          [\tsc{ref}, roof]
                      ]
                  ]
              ]
          ]
      ]
  ]
\end{forest}

\ex. Modern German: \tsc{int} \tsc{acc}\\
\begin{forest} boom
  [\tsc{wh}P
      [\tsc{wh}P,
      tikz={
      \node[label=below:\tit{w},
      draw,circle,
      scale=0.85,
      fit to=tree]{};
      }
          [\tsc{wh}, roof]
      ]
      [\tsc{ana}P
          [\tsc{ana}P,
          tikz={
          \node[label=below:\tit{e},
          draw,circle,
          scale=0.85,
          fit to=tree]{};
          }
              [\tsc{ana}, roof]
          ]
          [\tsc{acc}P,
          tikz={
          \node[label=below:\tit{n},
          draw,circle,
          scale=0.85,
          fit to=tree]{};
          }
              [\tsc{f}2]
              [\tsc{nom}P
                  [\tsc{f}1]
                  [\tsc{x}P
                      [\tsc{x}]
                      [\tsc{ind}P
                          [\tsc{ind}]
                          [\tsc{masc}P
                              [\tsc{masc}]
                              [\tsc{class}P
                                  [\tsc{class}]
                                  [\tsc{ref}P,
                                      [\tsc{ref}, roof]
                                  ]
                              ]
                          ]
                      ]
                  ]
              ]
          ]
      ]
  ]
\end{forest}

\ex. Modern German: \tsc{int} \tsc{nom}\\
\begin{forest} boom
  [\tsc{wh}P
      [\tsc{wh}P,
      tikz={
      \node[label=below:\tit{w},
      draw,circle,
      scale=0.85,
      fit to=tree]{};
      }
          [\tsc{wh}, roof]
      ]
      [\tsc{ana}P
          [\tsc{ana}P,
          tikz={
          \node[label=below:\tit{e},
          draw,circle,
          scale=0.85,
          fit to=tree]{};
          }
              [\tsc{ana}, roof]
          ]
          [\tsc{nom}P,
          tikz={
          \node[label=below:\tit{r},
          draw,circle,
          scale=0.85,
          fit to=tree]{};
          }
              [\tsc{f}1]
              [\tsc{x}
                  [\tsc{xp}]
                  [\tsc{ind}P
                      [\tsc{ind}]
                      [\tsc{masc}P
                          [\tsc{masc}]
                          [\tsc{class}P
                              [\tsc{class}]
                              [\tsc{ref}P,
                                  [\tsc{ref}, roof]
                              ]
                          ]
                      ]
                  ]
              ]
          ]
      ]
  ]
\end{forest}


\ex. Old High German: \tsc{int}/\tsc{ext} \tsc{acc}\\
\begin{forest} boom
  [\tsc{d}P
      [\tsc{d}P,
      tikz={
      \node[label=below:\tit{d},
      draw,circle,
      scale=0.85,
      fit to=tree]{};
      }
          [\tsc{d}, roof]
      ]
      [\tsc{ana}P
          [\tsc{ana}P,
          tikz={
          \node[label=below:\tit{e},
          draw,circle,
          scale=0.85,
          fit to=tree]{};
          }
              [\tsc{ana}, roof]
          ]
          [\tsc{acc}P,
          tikz={
          \node[label=below:\tit{n},
          draw,circle,
          scale=0.85,
          fit to=tree]{};
          }
              [\tsc{f}2]
              [\tsc{nom}P
                  [\tsc{f}1]
                  [\tsc{x}P
                      [\tsc{x}]
                      [\tsc{ind}P
                          [\tsc{ind}]
                          [\tsc{masc}P
                              [\tsc{masc}]
                              [\tsc{class}P
                                  [\tsc{class}]
                                  [\tsc{ref}P,
                                      [\tsc{ref}, roof]
                                  ]
                              ]
                          ]
                      ]
                  ]
              ]
          ]
      ]
  ]
\end{forest}


\ex. Old High German: \tsc{int}/\tsc{ext} \tsc{nom}\\
\begin{forest} boom
  [\tsc{d}P
      [\tsc{d}P,
      tikz={
      \node[label=below:\tit{d},
      draw,circle,
      scale=0.85,
      fit to=tree]{};
      }
          [\tsc{d}, roof]
      ]
      [\tsc{ana}P
          [\tsc{ana}P,
          tikz={
          \node[label=below:\tit{e},
          draw,circle,
          scale=0.85,
          fit to=tree]{};
          }
              [\tsc{ana}, roof]
          ]
          [\tsc{nom}P,
          tikz={
          \node[label=below:\tit{r},
          draw,circle,
          scale=0.85,
          fit to=tree]{};
          }
              [\tsc{f}1]
              [\tsc{x}P
                  [\tsc{x}]
                  [\tsc{ind}P
                      [\tsc{ind}]
                      [\tsc{masc}P
                          [\tsc{masc}]
                          [\tsc{class}P
                              [\tsc{class}]
                              [\tsc{ref}P,
                                  [\tsc{ref}, roof]
                              ]
                          ]
                      ]
                  ]
              ]
          ]
      ]
  ]
\end{forest}

\chapter{Constituent containment}\label{ch:relativization}

In Chapter \ref{ch:case-competition-typology} I introduced two descriptive parameters that generate the attested languages, as shown in Figure \ref{fig:two-parameters}.
The first parameter concerns whether the external case is allowed to surface when it wins the case competition (allow \tsc{ext}?). This parameter distinguishes between non-matching languages (e.g. Old High German) on the one hand and internal-only languages (e.g. Modern German) and matching languages (e.g. Polish) on the other hand.
The second parameter concerns whether the internal case is allowed to surface when it wins the case competition (allow \tsc{int?}). This parameter distinguishes between internal-only languages (e.g. as Modern German) on the one hand and non-matching languages (e.g. Polish) on the other hand.

\begin{figure}[H]
  \centering
    \footnotesize{
    \begin{tikzpicture}[node distance=1.5cm]
      \node (question2) [question]
      {allow \tsc{ext}?};
          \node (outcome2) [outcome, below of=question2, xshift=-1.5cm]
          {non-matching};
              \node (example2) [example, below of=outcome2, yshift=0.25cm]
              {\scriptsize{e.g. Gothic, Old High German, Classical Greek}};
          \node (question3) [question, below of=question2, xshift=2cm, yshift=-0.5cm]
          {allow \tsc{int}?};
              \node (outcome3) [outcome, below of=question3, xshift=-1.5cm]
              {internal-only};
                  \node (example3) [example, below of=outcome3, yshift=0.25cm]
                  {\scriptsize{e.g. Modern German\\\phantom{x}}};
              \node (outcome4) [outcome, below of=question3, xshift=1.5cm]
              {matching};
                  \node (example4) [example, below of=outcome4, yshift=0.25cm]
                  {\scriptsize{e.g. Polish\\\phantom{x}}};

    \draw [arrow] (question2) -- node[anchor=east] {yes} (outcome2);
    \draw [arrow] (question2) -- node[anchor=west] {no} (question3);
    \draw [arrow] (question3) -- node[anchor=east] {yes} (outcome3);
    \draw [arrow] (question3) -- node[anchor=west] {no} (outcome4);
    \end{tikzpicture}
    }
    \caption{Two descriptive parameters generate three language types}
    \label{fig:two-parameters}
\end{figure}

``A natural question at this point is whether this typology needs to be fully stipulative, or is to some extent derivable from independent properties of individual languages'' \citet{grosu1994}{147}

The goal of this chapter is to give the theoretical counterparts of these descriptive parameters. Goal: something that can be observed independently.

This chapter is structured as follows.


\section{The basic idea}\label{sec:basic-idea}

The goal of this chapter is to give a theoretical account for the crosslinguistic variation found in headless relatives. It should derive how languages differ in whether they allow the internal case and the external case to surface when either of them wins the case competition. In other words, the chapter gives the theoretical counterparts of the descriptive parameters given in Figure \ref{fig:two-parameters}. This section gives the basic idea behind my proposal. Throughout the rest of the chapter I motivate my proposal, and I illustrate it with examples. Before I can describe the theoretical counterparts, I need to introduce some concepts: the internal element and external element and the base of these elements.

I start with the internal and external element. In the thesis so far, I stated that the relative pronoun appears in the case that wins the case competition. I have not been explicit about where the case competition takes place. In order to avoid introducing theoretical machinery just for case competition situations, I assume it takes place in syntax. I propose that at some point in the derivation headless relatives have an internal and an external element.\footnote{
I am far from the only one that assumes this. Himmelreich, Hanink, but also Bresnan/Grimshaw, Groos/Riemsdijk, Harbert..
} The internal element bears the internal case, and the external element bears the external case. At the end of the derivation, the element bearing the more complex case surfaces as the relative pronoun, at least if the element is allowed to surface.

Now I turn to the so-called base of the internal and external element. The internal and the external element do not only consist of case features. They also contain other features. These other features have to do with referentiality, number, gender, definiteness, etc. I call the part of the internal and external element that corresponds to these features the base part. I refer to the part that corresponds to the case features as the case part.

Table \ref{tbl:component-elements} summarizes what I just laid out. At some point in the derivation, headless relatives contain an internal and an external element. The internal element consists of an internal base part and an internal case part. The external element consists of an external base part and an external case part.
There are three options for the relative pronoun. First, it can appear as the internal element: as the internal base plus the internal case. Second, it can appear as the external element: as the external base plus the external case. Third, there is no grammatical case for the relative pronoun.

\begin{table}[H]
  \center
  \caption{Components of the internal and external element}
\begin{tabular}{cccccc}
  \toprule
\multicolumn{2}{c}{\tsc{int} element} & \multicolumn{2}{c}{\tsc{ext} element} & \multicolumn{2}{c}{\tsc{rel} pronoun} \\
\cmidrule(lr){1-2}                      \cmidrule(lr){3-4}                    \cmidrule(lr){5-6}
base\scsub{int} & case\scsub{int}     & base\scsub{ext} & case\scsub{ext}     & base\scsub{int} & case\scsub{int}     \\
                &                     &                 &                     & base\scsub{ext} & case\scsub{ext}     \\
                &                     &                 &                     & \multicolumn{2}{c}{*}                 \\
\bottomrule
\end{tabular}
\label{tbl:component-elements}
\end{table}

To make this concrete, consider the example in \ref{ex:ohg-nom-nom-rep}, repeated from Chapter \ref{ch:case-competition-typology}. In this example, the internal nominative case competes against the external nominative case. The relative pronoun surfaces in the nominative case.

\exg. quham dher chisendit scolda uuerdhan\\
 come.\ac{pst}.3\ac{sg}\scsub{[nom]} \ac{rel}.\ac{sg}.\ac{m}.\ac{nom} send.\ac{pst}.\ac{ptcp}\scsub{[nom]} should.\ac{pst}.3\ac{sg} become.\ac{inf}\\
 `the one, who should have been sent, came' \flushfill{Old High German, \ac{isid} 35:5}\label{ex:ohg-nom-nom-rep}

The situation is schematically shown in Table \ref{tbl:component-dhen}.
In my proposal, the internal element for this sentence is \tit{dher} and the external element is \tit{dher} too. In Section \ref{sec:deriving-nonmatching} I motivate the featural content and phonological form of the internal and external element.
The base part of \tit{dher} is the morpheme \tit{dh}, and the case part of \tit{dher} is the morpheme \tit{r}.\footnote{
This is a simplification of the reality. The morpheme \tit{er} realizes besides case features also non-case features, such as gender and number. This simplification can be made, because the non-case features that are present in the internal element are also present in the external element.
\label{fn:base}}
The relative pronoun in this sentence is \tit{dher}. It is impossible to see whether this is the internal or the external element, because they are identical.

\begin{table}[H]
  \center
  \caption{The internal and external element of \ref{ex:ohg-nom-nom-rep}}
\begin{tabular}{cccccc}
  \toprule
\multicolumn{2}{c}{\tsc{int} element}   & \multicolumn{2}{c}{\tsc{ext} element}   & \multicolumn{2}{c}{\tsc{ext} element} \\
\cmidrule(lr){1-2}                        \cmidrule(lr){3-4}                        \cmidrule(lr){5-6}
base\scsub{int} & case\scsub{int}       & base\scsub{ext} & case\scsub{ext}       & base\scsub{rel} & case\scsub{rel}     \\
\cmidrule(lr){1-1}  \cmidrule(lr){2-2}    \cmidrule(lr){3-3}  \cmidrule(lr){4-4}    \cmidrule(lr){5-5}  \cmidrule(lr){6-6}
dh & er                                 & dh & er                                 & dh & er                               \\
\bottomrule
\end{tabular}
\label{tbl:component-dhen}
\end{table}

Now I have introduced the concepts internal and external element and their bases, I turn to the basic idea behind my proposal. The goal is to give a theoretical account that derives whether a language allows the internal or external case to surface when it wins the case competition or whether it does not. I propose that this follows from the comparison between the internal and external base within a language.
In the comparison, I rely on containment, just as I did in Chapter \ref{ch:decomposition} when comparing cases. I went with the following reasoning. A more complex case wins over a less complex case because the former contains all features that the latter contains. Concretely, the dative wins over the accusative because the dative contains all features that the accusative contains, the dative wins over the nominative because the dative contains all features that the nominative contains, and the accusative wins over the nominative because the accusative contains all features that the nominative contains.
I apply the same reasoning in comparing the internal and external base. When the internal base contains the features of the external base, the internal element is allowed to surface. When the external base contains the features of the internal base, the external element is allowed to surface.
%at first sight this seems very much related to what Hanink proposes for Modern German. Something is non-pronounced if it contains the features. A crucial difference here is that she formulates it in terms of context sensitive rules, but she does not motivate where these rules come from. I do not have language-specific rules.

I illustrate this proposal by showing how this plays out in three different languages. Table \ref{tbl:overview} gives an overview.
In Old High German, the internal base contains the external base, and the external base contains the internal base. Therefore, Old High German allows for matching cases, it allows the internal case to win, and it allows the external case to win.
In Modern German, the internal base contains the external base, but the external base does not contain the internal base. Therefore, Modern German allows for matching cases, and it allows the internal case to win.
In Polish, the internal base does not contain the external base, and the external base does not contain the internal base. Still, Polish only allows for matching cases.\footnote{
Later I explain how this is possible.
}%why is this possible? If I follow my own system, I'm saying the one base needs to contain the other in order for the example to be grammatical. That is not the situation here: the internal base is not contained in the external one, but still the internal element surfaces! What is different, however, is that there is no competition taking place. There is a tie. So then this chapter should be only about the cases in which there are non-matches. There should be an independent mechanism that makes sure matching cases are always grammatical.. which is the same mechanism that lets non-matching but syncretic cases be grammatical, which I do not talk about in depth

\begin{table}[H]
  \center
  \caption{Overview of languages}
\begin{tabular}{cccc}
  \toprule
                & \tsc{int} = \tsc{ext} & \tsc{int} > \tsc{ext} & \tsc{int} < \tsc{ext} \\
                      \cmidrule{1-4}
Old High German & ✔                     & ✔                     & ✔                   \\
Modern German   & ✔                     & ✔                     & *                   \\
Polish          & ✔                     & *                     & *                   \\
\bottomrule
\end{tabular}
\label{tbl:overview}
\end{table}

In what follows I discuss these three situations in the three languages.
In the first situation (\tsc{int} = \tsc{ext}), the internal and external case match.
In the second situation (\tsc{int} > \tsc{ext}), the internal case is more complex than the external case.
In the third situation (\tsc{int} < \tsc{ext}), the internal case is more complex than the external case.

Table \ref{tbl:basic-ohg} shows the three situation for Old High German. In Old High German, the internal base consists of features that correspond to the morpheme \tit{dh}, and so does the external base. The nominative case consists of features that correspond to the morpheme \tit{er}, and the accusative case consists of features that correspond to the morpheme \tit{en}. I motivate this in Section \ref{sec:deriving-only-internal}.

The first row shows the situation in which the internal nominative case matches the external nominative case (\tsc{int} = \tsc{ext}). The internal case part, which corresponds to the nominative \tit{er} morpheme, contains the external case part, which corresponds to the nominative \tit{er} morpheme. The internal base part, which corresponds to the morpheme \tit{dh}, contains the external base part, which corresponds to the morpheme \tit{dh} too. The relative pronoun that surfaces is the internal element with its internal base and its internal case: \tit{dher}.

The second row shows the situation in which internal accusative case is more complex than the external nominative case (\tsc{int} > \tsc{ext}). The internal case part, which corresponds to the accusative \tit{en} morpheme, contains the external case part, which corresponds to the nominative \tit{er} morpheme. The internal base part, which corresponds to the morpheme \tit{dh}, contains the external base part, which corresponds to the morpheme \tit{dh} too. The relative pronoun that surfaces is the internal element with its internal base and its internal case: \tit{dher}.

The third row shows the situation in which external accusative case is more complex than the internal nominative case (\tsc{int} < \tsc{ext}). The external case part, which corresponds to the accusative \tit{en} morpheme, contains the internal case part, which corresponds to the nominative \tit{er} morpheme. The external base part, which corresponds to the morpheme \tit{dh}, contains the internal base part, which corresponds to the morpheme \tit{dh} too. The relative pronoun that surfaces is the external element with its external base and its external case: \tit{dhen}.

\begin{table}[H]
  \center
  \caption{Base comparison in Old High German}
\begin{tabular}{ccccccc}
  \toprule
                      & \multicolumn{2}{c}{\tsc{int} element}  & \multicolumn{2}{c}{\tsc{ext} element}  & \multicolumn{2}{c}{\tsc{rel} pronoun} \\
                        \cmidrule(lr){2-3}                        \cmidrule(lr){4-5}                      \cmidrule(lr){6-7}
                      & base\scsub{int} & case\scsub{int}       & base\scsub{ext} & case\scsub{ext}     & base\scsub{rel} & case\scsub{rel} \\
                        \cmidrule(lr){2-2}    \cmidrule(lr){3-3}  \cmidrule(lr){4-4} \cmidrule(lr){5-5}   \cmidrule(lr){6-6} \cmidrule(lr){7-7}
\tsc{int} = \tsc{ext} & dhe & r                                 & dhe & r                               & dhe & r                           \\
\tsc{int} > \tsc{ext} & dhe & n                                 & dhe & r                               & dhe & n                           \\
\tsc{int} < \tsc{ext} & dhe & r                                 & dhe & n                               & dhe & n                           \\
\bottomrule
\end{tabular}
\label{tbl:basic-ohg}
\end{table}

Table \ref{tbl:basic-mg} shows the three situation for Modern German. In Modern German, the internal base consists of features that correspond to the morpheme \tit{we}. The external base does not consist of any features, so there is no morpheme that corresponds to it.\footnote{
Actually, not entirely, see footnote \ref{fn:base}.
} The nominative case consists of features that correspond to the morpheme \tit{r}, and the accusative case consists of features that correspond to the morpheme \tit{n}.
I motivate this in Section \ref{sec:deriving-only-internal}.

The first row shows the situation in which the internal nominative case matches the external nominative case (\tsc{int} = \tsc{ext}). The internal case part, which corresponds to the nominative \tit{r} morpheme, contains the external case part, which corresponds to the nominative \tit{r} morpheme. The internal base part, which corresponds to the morpheme \tit{we}, contains the external base part, because there is no morpheme. The relative pronoun that surfaces is the internal element with its internal base and its internal case: \tit{wer}.

The second row shows the situation in which internal accusative case is more complex than the external nominative case (\tsc{int} > \tsc{ext}). The internal case part, which corresponds to the accusative \tit{n} morpheme, contains the external case part, which corresponds to the nominative \tit{r} morpheme. The internal base part, which corresponds to the morpheme \tit{we}, contains the external base part, because there is no morpheme. The relative pronoun that surfaces is the internal element with its internal base and its internal case: \tit{wer}.

The third row shows the situation in which external accusative case is more complex than the internal nominative case (\tsc{int} < \tsc{ext}). The external case part, which corresponds to the accusative \tit{n} morpheme, contains the internal case part, which corresponds to the nominative \tit{r} morpheme. The external base part, which is empty, does not contain the internal base part, which corresponds to the morpheme \tit{we}. The winner of the case competition is the external case, but the external base does not contain the internal base, so there is no grammatical relative pronoun.

\begin{table}[H]
  \center
  \caption{Base comparison in Modern German}
\begin{tabular}{ccccccc}
  \toprule
                      & \multicolumn{2}{c}{\tsc{int} element}  & \multicolumn{2}{c}{\tsc{ext} element}  & \multicolumn{2}{c}{\tsc{rel} pronoun} \\
                        \cmidrule(lr){2-3}                        \cmidrule(lr){4-5}                      \cmidrule(lr){6-7}
                      & base\scsub{int} & case\scsub{int}       & base\scsub{ext} & case\scsub{ext}     & base\scsub{rel} & case\scsub{rel} \\
                        \cmidrule(lr){2-2}    \cmidrule(lr){3-3}  \cmidrule(lr){4-4} \cmidrule(lr){5-5}   \cmidrule(lr){6-6} \cmidrule(lr){7-7}
\tsc{int} = \tsc{ext} & we & r                                  &  & r                                  & we & r                           \\
\tsc{int} > \tsc{ext} & we & n                                  &  & r                                  & we & n                           \\
\tsc{int} < \tsc{ext} & we & r                                  &  & n                                  & \multicolumn{2}{c}{\tsc{*}}      \\
\bottomrule
\end{tabular}
\label{tbl:basic-mg}
\end{table}

Table \ref{tbl:basic-polish} shows the three situation for Polish. In Polish, the internal base consists of features that correspond to the morpheme \tit{ko}. The external base consists of features that correspond to the morpheme \tit{te}.
The accusative case consists of features that correspond to the morpheme \tit{go}, and the dative case consists of features that correspond to the morpheme \tit{mu}.
I motivate this in Section \ref{sec:deriving-matching}.

The first row shows the situation in which the internal nominative case matches the external nominative case (\tsc{int} = \tsc{ext}). The internal case part, which corresponds to the accusative \tit{go} morpheme, contains the external case part, which corresponds to the accusative \tit{go} morpheme. The internal base part, which corresponds to the morpheme \tit{ko}, does not contain the internal base part, which corresponds to the morpheme \tit{te}. Still, the relative pronoun that surfaces is the internal element with its internal base and its internal case: \tit{kogo}.\footnote{
This is unexpected, and I talk about that later.
}

The second row shows the situation in which internal dative case is more complex than the external accusative case (\tsc{int} > \tsc{ext}). The internal case part, which corresponds to the dative \tit{mu} morpheme, contains the external case part, which corresponds to the accusative \tit{go} morpheme. The internal base part, which corresponds to the morpheme \tit{ko}, does not contain the internal base part, which corresponds to the morpheme \tit{te}. The winner of the case competition is the internal case, but the internal base does not contain the external base, so there is no grammatical relative pronoun.

The third row shows the situation in which external dative case is more complex than the internal accusative case (\tsc{int} > \tsc{ext}). The external case part, which corresponds to the dative \tit{mu} morpheme, contains the internal case part, which corresponds to the accusative \tit{go} morpheme. The external base part, which corresponds to the morpheme \tit{te}, does not contain the internal base part, which corresponds to the morpheme \tit{ko}. The winner of the case competition is the external case, but the external base does not contain the internal base, so there is no grammatical relative pronoun.

\begin{table}[H]
  \center
  \caption{Base comparison in Polish}
\begin{tabular}{ccccccc}
  \toprule
                      & \multicolumn{2}{c}{\tsc{int} element}  & \multicolumn{2}{c}{\tsc{ext} element}  & \multicolumn{2}{c}{\tsc{rel} pronoun} \\
                        \cmidrule(lr){2-3}                        \cmidrule(lr){4-5}                      \cmidrule(lr){6-7}
                      & base\scsub{int} & case\scsub{int}       & base\scsub{ext} & case\scsub{ext}     & base\scsub{rel} & case\scsub{rel} \\
                        \cmidrule(lr){2-2}    \cmidrule(lr){3-3}  \cmidrule(lr){4-4} \cmidrule(lr){5-5}   \cmidrule(lr){6-6} \cmidrule(lr){7-7}
\tsc{int} = \tsc{ext} & ko & go                                  & te & go                              & ko & go                           \\
\tsc{int} > \tsc{ext} & ko & mu                                  & te & go                              & \multicolumn{2}{c}{\tsc{*}}       \\
\tsc{int} < \tsc{ext} & ko & go                                  & te & mu                              & \multicolumn{2}{c}{\tsc{*}}       \\
\bottomrule
\end{tabular}
\label{tbl:basic-polish}
\end{table}

Taking this all together, there are two competitions taking place in headless relatives: case competition and base competition. Case competition determines which case wins and base competition determines whether this case is allowed to surface.
I put this in the metaphor with the committee that I introduced in Section \ref{sec:possible-patterns}. The committee learns who wins the case competition, and it can either approve this case or not approve it. The information that the committee uses for its decision is the comparison between the internal and the external base. The committee approves the winning case if the base associated with it contains the base associated with the losing case.

In sum, a relative pronoun can surface in the internal case if the internal case and base contains the external case and base.
It works the same the other way around: a relative pronoun can surface in the external case if the external case and base contain the internal case and base.
Notice that there is a crucial difference between comparing the cases and comparing the bases. The internal and external case can differ from sentence to sentence. The internal and external base remain stable throughout the language.



\section{Deriving the non-matching type}\label{sec:deriving-nonmatching}

The non-matching type of language allows for matching cases, it allows the internal case to win, and it allows the external case to win. I have been describing Old High German as an example of this type. In this section, I show what it is about Old High German that causes the language to be of the non-matching type. I propose that the crucial factor is that Old High German has a syncretic internal and external base. Since they are syncretic, the features in the internal base contain the features in the external base, and the features in the external base contain just as well the features in the internal base. The internal base containing the external base causes the internal case to be allowed to surface when it wins the case competition. The external base containing the internal base causes the external case to be allowed to surface when it wins the case competition.

This section is structured as follows. First, I argue that Old High German headless relatives are derived from relative clauses headed by a light head, i.e. light-headed relatives. In this analysis, the internal element is what can descriptively be called the relative pronoun, and the external element is what can descriptively be called the light head. The internal element surfaces as the relative pronoun when the internal case is more complex, and the external element surfaces as the relative pronoun when the external case is more complex. In this section, I decompose the internal and external element, and I show which morpheme corresponds to which features. Both elements consist of two morphemes: a base part and a case part. I go through the examples in Table \ref{tbl:forms-ohg}, showing per situation how the base and case parts syntactically contain the other base and case parts. This containment is crucial. When the internal base contains the external base, the internal case is allowed to surface when it is more complex, and when the external base contains the internal base, the external case is allowed to surface when it is more complex.

\begin{table}[H]
  \center
  \caption{Base comparison in Old High German}
\begin{tabular}{ccccccc}
  \toprule
                      & \multicolumn{2}{c}{\tsc{int} element}  & \multicolumn{2}{c}{\tsc{ext} element}  & \multicolumn{2}{c}{\tsc{rel} pronoun} \\
                        \cmidrule(lr){2-3}                        \cmidrule(lr){4-5}                      \cmidrule(lr){6-7}
                      & base\scsub{int} & case\scsub{int}       & base\scsub{ext} & case\scsub{ext}     & base\scsub{rel} & case\scsub{rel} \\
                        \cmidrule(lr){2-2}    \cmidrule(lr){3-3}  \cmidrule(lr){4-4} \cmidrule(lr){5-5}   \cmidrule(lr){6-6} \cmidrule(lr){7-7}
\tsc{int} = \tsc{ext} & dhe & r                                 & dhe & r                               & dhe & r                           \\
\tsc{int} > \tsc{ext} & dhe & n                                 & dhe & r                               & dhe & n                           \\
\tsc{int} < \tsc{ext} & dhe & r                                 & dhe & n                               & dhe & n                           \\
\bottomrule
\end{tabular}
\label{tbl:forms-ohg}
\end{table}

I propose headless relatives are derived from light-headed relatives (\citealt{fuss2014,hanink2018} argue the same but for Modern German\footnote{
A difference with Modern German is that one of the elements can only be absent when the cases match. In Section \ref{ch:discussion} I return to the point why Modern German does not have non-matching headless relatives that look like Old High German, although it still has syncretic relative pronouns and light heads.
}).
In a light-headed relative, the head of a relative is not a full noun phrase, but it is a bit `lighter': it only consists of a demonstrative. Consider the light-headed relative in \ref{ex:ohg-double}. \tit{Thér} `\tsc{dem}.\tsc{sg}.\tsc{m}.\tsc{nom}' is the head of the relative clause, which is the external element. \tit{Then} `\tsc{rel}.\tsc{sg}.\tsc{m}.\tsc{acc}' is the relative pronoun of the relative clause, which is the internal element.

\exg. eno nist thiz thér then ir suochet zi arslahanne?\\
 now {not be.3\ac{sg}} \tsc{dem}.\tsc{sg}.\tsc{n}.\tsc{nom} \tsc{dem}.\tsc{sg}.\tsc{m}.\tsc{nom}
 \tsc{rel}.\tsc{sg}.\tsc{m}.\tsc{acc} 2\ac{pl}.\tsc{nom} seek.2\tsc{pl} to kill.\tsc{inf}.\ac{sg}.\tsc{dat}\\
 `Isn't this now the one, who you seek to kill?'\label{ex:ohg-double}

The difference between a light-headed relative and a headless relative is that in headless relatives, either the internal or the external is absent. The absent element is the one that has the least complex case. This shows the presence of two elements in Old High German is optional.\footnote{
This sharply contrasts with headless relatives in Modern German, which are always ungrammatical when both the internal and external elements surface. I come back to this in Section \ref{sec:deriving-only-internal}.
}
In Old High German, there are three possible constructions: the internal and external element can both surface, only the internal element can surface and only the external element can surface. If only one of the two elements surfaces, this is the element that bears the most complex case, which is either the internal or the external one, as I have shown in Chapter \ref{ch:case-competition-typology}. I assume that whether both or only one of the elements surfaces is determined by information structure. In \ref{ex:ohg-double}, the external element \tit{thér} `\tsc{dem}.\tsc{sg}.\tsc{m}.\tsc{nom}' is the candidate to be absent. However, it seems plausible that this is emphasized in this sentence and that it, therefore, cannot be absent.

Support for the idea that Old High German headless relatives are derived from light-headed ones comes from their interpretation. Headless relatives in which the relative pronoun starts with a \tit{d}, such as in Old High German, seem to be linked to individuating or definite readings and not to generalizing or indefinite readings \citep[cf.][]{fuss2017}. I illustrate this with the two examples I repeat from Chapter  \ref{ch:case-competition-typology}.

Consider the example in \ref{ex:ohg-nom-acc-interpretation}, repeated from Chapter \ref{ch:case-competition-typology}.
In this example, the author refers to the specific person which was talked about, and not to any or every person that was talked about.

\exg. Thíz ist \tbf{then} \tbf{sie} \tbf{zéllent}\\
\ac{dem}.\ac{sg}.\ac{n}.\ac{nom} be.\ac{pres}.3\ac{sg}\scsub{[nom]} \ac{rel}.\ac{sg}.\ac{m}.\ac{acc}
3\ac{pl}.\ac{m}.\ac{nom} tell.\ac{pres}.3\ac{pl}\scsub{[acc]}\\
`this is the one whom they talk about'\\
not: `this is whoever they talk about' \flushfill{Old High German, \ac{otfrid} III 16:50}\label{ex:ohg-nom-acc-interpretation}

Consider also the example in \ref{ex:ohg-nom-acc-interpretation}, repeated from Chapter \ref{ch:case-competition-typology}.
In this example, the author refers to the specific person who spoke to someone, and not to any or every person who spoke to someone.

\exg. enti aer {ant uurta} demo \tbf{zaimo} \tbf{sprah}\\
and 3\ac{sg}.\ac{m}.\ac{nom} reply.\ac{pst}.3\ac{sg}\scsub{[dat]} \ac{rel}.\ac{sg}.\ac{m}.\ac{dat} {to 3\ac{sg}.\ac{m}.\ac{dat}} speak.\ac{pst}.3\ac{sg}\scsub{[nom]}\\
`and he replied to the one who spoke to him'\\
not: `and he replied to whoever spoke to him'
 \flushfill{Old High German, \ac{mons} 7:24, adapted from \pgcitealt{pittner1995}{199}}\label{ex:ohg-dat-nom-rep}

I conclude that the internal element in Old High German is the descriptive relative pronoun, and the external element in Old High German is the descriptive light head. In what follows I closely examine the internal structure of the internal and external element. I illustrate how the internal base and the external base are identical, so they contain each other.

The light head in a light-headed relative is a demonstrative pronoun. Relative and demonstrative pronouns are syncretic in Old High German \pgcitep{braune2018}{338}. Table \ref{tbl:rel-dem-ohg} gives an overview of the forms in singular and plural, neuter, masculine and feminine and nominative, accusative and dative. The pronouns consist of two morphemes: a \tit{d} and suffix that differs per number, gender and case.\footnote{
\tit{d} can also be written as \tit{dh} and \tit{th}, \tit{ë} and \tit{ē} can also be \tit{e} and \tit{é} \pgcitep{braune2018}{339}.
}\footnote{
The suffix could also be further divided into a vowel and a suffix. As this is not relevant for the discussion here, I refrain from doing that.
}

\begin{table}[H]
 \center
 \caption {Relative/demonstrative pronouns in Old High German \pgcitep{braune2018}{339}}
  \begin{tabular}{cccc}
  \toprule
            & \ac{n}.\ac{sg}  & \ac{m}.\ac{sg}      & \ac{f}.\ac{sg}    \\
        \cmidrule{2-4}
  \ac{nom}  & d-aȥ            & d-ër                & d-iu               \\
  \ac{acc}  & d-aȥ            & d-ën                & d-ea/d-ia         \\
  \ac{dat}  & d-ëmu/d-ëmo     & d-ëmu/d-ëmo         & d-ëru/d-ëro       \\
  \bottomrule
            & \ac{n}.\ac{pl}  & \ac{m}.\ac{pl}      &  \ac{f}.\ac{pl}  \\
        \cmidrule{2-4}
  \ac{nom}  & d-iu            &  d-ē/d-ea/d-ia/d-ie & d-eo/-io         \\
  \ac{acc}  & d-iu            &  d-ē/d-ea/d-ia/d-ie & d-eo/-io         \\
  \ac{dat}  & d-ēm/d-ēn       &  d-ēm/d-ēn          & d-ēm/d-ēn        \\
    \bottomrule
  \end{tabular}
  \label{tbl:rel-dem-ohg}
\end{table}

The suffixes that combine with the \tit{d} in demonstrative and relative pronouns also appear on adjectives. This is illustrated in Table \ref{tbl:adj-ohg}.

\begin{table}[H]
 \center
 \caption {Adjectives on \tit{-a-/-ō-} in Old High German \pgcitealt{braune2018}{300}}
  \begin{tabular}{cccc}
  \toprule
            & \ac{n}.\ac{sg}    & \ac{m}.\ac{sg}      & \ac{f}.\ac{sg}    \\
    \cmidrule{2-4}
  \ac{nom}  & jung, jung-aȥ     & jung, jung-ēr       & jung, jung-iu     \\
  \ac{acc}  & jung, jung-aȥ     & jung-an             & jung-a            \\
  \ac{dat}  & jung-emu/jung-emo & jung-emu/jung-emo   & jung-eru/jung-ero \\
  \bottomrule
            & \ac{n}.\ac{pl}    & \ac{m}.\ac{pl}      &  \ac{f}.\ac{pl}   \\
      \cmidrule{2-4}
  \ac{nom}  & jung-iu           &  jung-e             & jung-o            \\
  \ac{acc}  & jung-iu           &  jung-e             & jung-o            \\
  \ac{dat}  & jung-ēm/jung-ēn   &  jung-ēm/jung-ēn    & jung-ēm/jung-ēn   \\
    \bottomrule
  \end{tabular}
  \label{tbl:adj-ohg}
\end{table}

I conclude from this that the suffix expresses features that are specific to being nominal, like number, gender and case. Not part of the suffix are features that are specific to being a demonstrative or relative pronoun, like anaphoricity and definiteness. I assume that these are expressed by the morpheme \tit{d}.

In this section, I only discuss two forms: the nominative and accusative masculine singular relative and demonstrative pronoun. The nominative is \tit{dër} and the accusative is \tit{dën}. In what follows, I discuss the featural content of the morphemes \tit{d}, \tit{ër} and \tit{ën}. I start with the features that are expressed by the suffixes \tit{ër} and \tit{ën}.

For the suffixes, I use pronominal features that are distinguished by \citet{harley2002}: \tsc{ref}, \tsc{class}, \tsc{masc} and \tsc{ind}. \tsc{ref} refers to a referring expression, which all pronouns contain. The feature \tsc{class} refers to gender features, which is neuter if it is not combined with any other features. Combining \tsc{class} with the feature \tsc{masc} gives a masculine gender. \tsc{ind} refer to number, which is singular if it is not combined with any other features.
I addition, I use the case features introduced by \citet{caha2009}, which I already discussed in Chapter \ref{ch:decomposition}. \tsc{f}1 refers to a nominative, and \tsc{f}1 and \tsc{f}2 refers to an accusative.

This allows me to propose the following lexical entries for the two suffixes.

\ex.
\begin{forest} boom
  [\tsc{nom}P
      [\tsc{f}1]
      [\tsc{ind}P
          [\tsc{ind}]
          [\tsc{masc}P
              [\tsc{masc}]
              [\tsc{class}P
                  [\tsc{class}]
                  [\tsc{ref}]
              ]
          ]
      ]
  ]
  {\draw (.east) node[right]{⇔ \tit{en}}; }
\end{forest}
\label{ex:ohg-er-lexicon}

\ex.
\begin{forest} boom
  [\tsc{acc}P
      [\tsc{f}2]
      [\tsc{nom}P
          [\tsc{f}1]
          [\tsc{ind}P
              [\tsc{ind}]
              [\tsc{masc}P
                  [\tsc{masc}]
                  [\tsc{class}P
                      [\tsc{class}]
                      [\tsc{ref}]
                  ]
              ]
          ]
      ]
  ]
  {\draw (.east) node[right]{⇔ \tit{en}}; }
\end{forest}
\label{ex:ohg-en-lexicon}

The \tit{d} morpheme corresponds to definiteness and anaphoricity. Anaphoricity establishes a relation with another element in the (linguistic) discourse. Definiteness encodes that the referent is specific.
%what can I say about the difference between REL and DEM? Should I include some features REL that causes to movement to the left periphery?

\ex.
\begin{forest} boom
  [\tsc{d}P
      [\tsc{d}]
      [\tsc{ana}]
  ]
  {\draw (.east) node[right]{⇔ \tit{d}}; }
\end{forest}
\label{ex:ohg-d-lexicon}

So, the two relative pronouns look like this.\footnote{A question that arises here is how the case features can form a constituent to the exclusion of definiteness and anaphoricity. I come back to this issue in Chapter \ref{ch:discussion}.}

\ex.
\a.
\begin{forest} boom
  [\tsc{d}P
      [\tsc{d}P,
      tikz={
      \node[label=below:\tit{d},
      draw,circle,
      scale=0.80,
      fit to=tree]{};
      }
          [\tsc{d}]
          [\tsc{ana}]
      ]
      [\tsc{acc}P,
      tikz={
      \node[label=below:\tit{en},
      draw,circle,
      scale=0.85,
      fit to=tree]{};
      }
          [\tsc{f}2]
          [\tsc{nom}P
              [\tsc{f1}]
              [\tsc{ind}P
                  [\tsc{ind}]
                  [\tsc{masc}P
                      [\tsc{masc}]
                      [\tsc{class}P
                          [\tsc{class}]
                          [\tsc{ref}]
                      ]
                  ]
              ]
          ]
      ]
  ]
\end{forest}
\b.
\begin{forest} boom
  [\tsc{d}P
      [\tsc{d}P,
      tikz={
      \node[label=below:\tit{d},
      draw,circle,
      scale=0.80,
      fit to=tree]{};
      }
          [\tsc{d}]
          [\tsc{ana}]
      ]
      [\tsc{nom}P,
      tikz={
      \node[label=below:\tit{er},
      draw,circle,
      scale=0.85,
      fit to=tree]{};
      }
          [\tsc{f1}]
          [\tsc{ind}P
              [\tsc{ind}]
              [\tsc{masc}P
                  [\tsc{masc}]
                  [\tsc{class}P
                      [\tsc{class}]
                      [\tsc{ref}]
                  ]
              ]
          ]
      ]
  ]
\end{forest}

So, there is a base part and a case part. Actually, the case part also has some case in it, but that does not matter because they are identical in both case parts.

I start with the first example from Table \ref{tbl:forms-ohg}, in which the internal and external case match (\tsc{int} = \tsc{ext}). The example that corresponds to this example is given in \ref{ex:ohg-nom-nom-rep}.

\exg. quham dher chisendit scolda uuerdhan\\
 come.\ac{pst}.3\ac{sg}\scsub{[nom]} \ac{rel}.\ac{sg}.\ac{m}.\ac{nom} send.\ac{pst}.\ac{ptcp}\scsub{[nom]} should.\ac{pst}.3\ac{sg} become.\ac{inf}\\
 `the one, who should have been sent, came' \flushfill{Old High German, \ac{isid} 35:5}\label{ex:ohg-nom-nom-rep-workout}

 \begin{table}[H]
   \center
  \caption {Old High German: \ac{int} = \ac{ext}}
   \begin{tabular}[b]{cc}
       \toprule
       \ac{int}  &   \ac{ext} \\ \cmidrule{1-2}
       \footnotesize{
       \begin{forest} boom
         [DP
             [DP,
               tikz={
               \node[draw,circle,
               fill=DG,fill opacity=0.2,
               scale=0.75,
               DG,dashed,
               fit to=tree]{};
               }
                 [\tit{d}, roof]
             ]
             [\tsc{d}P, s sep=20mm
                 [\tsc{d}P,
                 tikz={
                 \node[draw,circle,
                 fill=DG,fill opacity=0.2,
                 scale=0.75,
                 DG,dashed,
                 fit to=tree]{};
                 }
                     [\tit{e}, roof]
                 ]
                 [\tsc{nom}P,
                 tikz={
                 \node[label=below:\tit{er},
                 draw,circle,
                 scale=0.85,
                 fit to=tree]{};
                 \node[draw,circle,
                 fill=DG,fill opacity=0.2,
                 DG,dashed,
                 scale=0.8,
                 fit to=tree]{};
                 }
                     [\tsc{f1}]
                     [\tsc{ind}P
                         [\phantom{xxx},
                         roof, baseline
                         ]
                     ]
                 ]
             ]
         ]
       \end{forest}
       }
       &
       \footnotesize{
       \begin{forest} boom
         [\textcolor{DG}{DP}
             [\textcolor{DG}{DP},edge=DG,
             tikz={
             \node[draw,circle,
             scale=0.75,
             DG,dashed,
             fit to=tree]{};
             }
                 [\textcolor{DG}{\tit{d}}, roof, edge=DG]
             ]
             [\textcolor{DG}{\tsc{ana}P},edge=DG, s sep=20mm
                 [\textcolor{DG}{\tsc{ana}P},edge=DG,
                 tikz={
                 \node[draw,circle,
                 scale=0.75,
                 DG,dashed,
                 fit to=tree]{};
                 }
                     [\textcolor{DG}{\tit{e}}, roof, edge=DG]
                 ]
                 [\textcolor{DG}{\ac{nom}},edge=DG,
                 tikz={
                 \node[label=below:\textcolor{DG}{\tit{r}},
                 draw,circle,
                 scale=0.8,
                 DG,
                 fit to=tree]{};
                 \node[
                 draw,circle,
                 scale=0.75,
                 dashed,DG,
                 fit to=tree]{};
                 }
                     [\textcolor{DG}{\tsc{f1}},baseline,edge=DG]
                     [\textcolor{DG}{\tsc{ind}P},edge=DG
                         [\phantom{xxx},
                         roof, baseline, edge=DG
                         ]
                     ]
                 ]
             ]
         ]
       \end{forest}
       }\\
       \bottomrule
   \end{tabular}
   \label{tbl:ohg-match}
 \end{table}

% There are two independent containment relations. Problem?

\exg. Thíz ist \tbf{then} \tbf{sie} \tbf{zéllent}\\
\ac{dem}.\ac{sg}.\ac{n}.\ac{nom} be.\ac{pres}.3\ac{sg}\scsub{[nom]} \ac{rel}.\ac{sg}.\ac{m}.\ac{acc} 3\ac{pl}.\ac{m}.\ac{nom} tell.\ac{pres}.3\ac{pl}\scsub{[acc]}\\
`this is the one whom they talk about' \flushfill{Old High German, \ac{otfrid} III 16:50}\label{ex:ohg-nom-acc-rep}

\begin{table}[H]
  \center
 \caption {Old High German: \ac{int} > \ac{ext}}
  \begin{tabular}[b]{cc}
      \toprule
      \ac{int}  &   \ac{ext} \\ \cmidrule{1-2}
      \footnotesize{
      \begin{forest} boom
        [DP
            [DP,
              tikz={
              \node[draw,circle,
              fill=DG,fill opacity=0.2,
              scale=0.75,
              DG,dashed,
              fit to=tree]{};
              }
                [\tit{d}, roof]
            ]
            [\tsc{ana}P, s sep=20mm
                [\tsc{ana}P,
                tikz={
                \node[draw,circle,
                fill=DG,fill opacity=0.2,
                scale=0.75,
                DG,dashed,
                fit to=tree]{};
                }
                    [\tit{e}, roof]
                ]
                [\tsc{acc}P,
                tikz={
                \node[label=below:\tit{n},
                draw,circle,
                scale=0.85,
                fit to=tree]{};
                }
                    [\tsc{f}2]
                    [\tsc{nom}P,
                      tikz={
                      \node[draw,circle,
                      fill=DG,fill opacity=0.2,
                      DG,dashed,
                      scale=0.8,
                      fit to=tree]{};
                      }
                        [\tsc{f1}]
                        [\tsc{ind}P
                            [\phantom{xxx},
                            roof, baseline
                            ]
                        ]
                    ]
                ]
            ]
        ]
      \end{forest}
      }
      &
      \footnotesize{
      \begin{forest} boom
        [\textcolor{DG}{DP}
            [\textcolor{DG}{DP},edge=DG,
            tikz={
            \node[draw,circle,
            scale=0.75,
            DG,dashed,
            fit to=tree]{};
            }
                [\textcolor{DG}{\tit{d}}, roof, edge=DG]
            ]
            [\textcolor{DG}{\tsc{ana}P},edge=DG, s sep=20mm
                [\textcolor{DG}{\tsc{ana}P},edge=DG,
                tikz={
                \node[draw,circle,
                scale=0.75,
                DG,dashed,
                fit to=tree]{};
                }
                    [\textcolor{DG}{\tit{e}}, roof, edge=DG]
                ]
                [\textcolor{DG}{\ac{nom}},edge=DG,
                tikz={
                \node[label=below:\textcolor{DG}{\tit{r}},
                draw,circle,
                scale=0.75,
                DG,
                fit to=tree]{};
                \node[
                draw,circle,
                scale=0.8,
                dashed,DG,
                fit to=tree]{};
                }
                    [\textcolor{DG}{\tsc{f1}},baseline,edge=DG]
                    [\textcolor{DG}{\tsc{ind}P},edge=DG
                        [\phantom{xxx},
                        roof, baseline, edge=DG
                        ]
                    ]
                ]
            ]
        ]
      \end{forest}
      }\\
      \bottomrule
  \end{tabular}
  \label{tbl:ohg-int-wins}
\end{table}

\exg. ih bibringu fona iacobes samin endi fona iuda dhen \tbf{mina} \tbf{berga} \tbf{chisitzit}\\
1\ac{sg}.\ac{nom} {create}.\ac{pres}.1\ac{sg}\scsub{[acc]} of Jakob.\ac{gen} seed.\ac{sg}.\ac{dat} and of Judah.\ac{dat} \ac{rel}.\ac{sg}.\ac{m}.\ac{acc} my.\ac{acc}.\ac{m}.\ac{pl} mountain.\ac{acc}.\ac{pl} possess.\ac{pres}.3\ac{sg}\scsub{[nom]}\\
`I create of the seed of Jacob and of Judah the one, who possess my mountains' \flushfill{Old High German, \ac{isid} 34:3}\label{ex:ohg-acc-nom-rep}

\begin{table}[H]
  \center
 \caption {Old High German: \ac{int} < \ac{ext}}
  \begin{tabular}[b]{cc}
      \toprule
      \ac{int}  &   \ac{ext} \\ \cmidrule{1-2}
      \footnotesize{
      \begin{forest} boom
        [\textcolor{DG}{DP}
            [\textcolor{DG}{DP},edge=DG,
            tikz={
            \node[draw,circle,
            scale=0.75,
            DG,dashed,
            fit to=tree]{};
            }
                [\textcolor{DG}{\tit{d}}, roof, edge=DG]
            ]
            [\textcolor{DG}{\tsc{ana}P},edge=DG
                [\textcolor{DG}{\tsc{ana}P},edge=DG,
                tikz={
                \node[draw,circle,
                scale=0.75,
                DG,dashed,
                fit to=tree]{};
                }
                    [\textcolor{DG}{\tit{e}}, roof, edge=DG]
                ]
                [\textcolor{DG}{\ac{nom}},edge=DG,
                tikz={
                \node[label=below:\textcolor{DG}{\tit{r}},
                draw,circle,
                scale=0.75,
                DG,
                fit to=tree]{};
                \node[
                draw,circle,
                scale=0.8,
                dashed,DG,
                fit to=tree]{};
                }
                    [\textcolor{DG}{\tsc{f1}},baseline,edge=DG]
                    [\textcolor{DG}{\tsc{ind}P},edge=DG
                        [\phantom{xxx},
                        roof, baseline, edge=DG
                        ]
                    ]
                ]
            ]
        ]
      \end{forest}
      }
      &
      \footnotesize{
      \begin{forest} boom
        [DP
            [DP,
              tikz={
              \node[draw,circle,
              fill=DG,fill opacity=0.2,
              scale=0.75,
              DG,dashed,
              fit to=tree]{};
              }
                [\tit{d}, roof]
            ]
            [\tsc{ana}P
                [\tsc{ana}P,
                tikz={
                \node[draw,circle,
                fill=DG,fill opacity=0.2,
                scale=0.75,
                DG,dashed,
                fit to=tree]{};
                }
                    [\tit{e}, roof]
                ]
                [\tsc{acc}P,
                tikz={
                \node[label=below:\tit{n},
                draw,circle,
                scale=0.85,
                fit to=tree]{};
                }
                    [\tsc{f}2]
                    [\tsc{nom}P,
                      tikz={
                      \node[draw,circle,
                      fill=DG,fill opacity=0.2,
                      DG,dashed,
                      scale=0.8,
                      fit to=tree]{};
                      }
                        [\tsc{f1}]
                        [\tsc{ind}P
                            [\phantom{xxx},
                            roof, baseline
                            ]
                        ]
                    ]
                ]
            ]
        ]
      \end{forest} }\\
      \bottomrule
  \end{tabular}
  \label{tbl:ohg-ext-wins}
\end{table}

To sum up, Old High German allows the internal and the external case to surface when either of them wins the case competition. This is due to the fact that the bases of the internal and the external element are syncretic. Because of that, the internal base contains the external base, which allows the internal case to surface, and the external base contains the internal base, which allows the external case to surface.


\section{Deriving the internal-only type}\label{sec:deriving-only-internal}

Only internal wins, external cannot. I illustrate this with nominative and accusative.

\ex.
\ag. Uns besucht, \tbf{wen} \tbf{Maria} \tbf{mag}.\\
 2\ac{pl}.\ac{acc} visit.\ac{pres}.3\ac{sg}\scsub{[nom]} \ac{rel}.\ac{an}.\ac{acc} Maria.\ac{nom} like.\ac{pres}.3\ac{sg}\scsub{[acc]}\\
 `Who visits us, Maria likes.' \flushfill{Modern German, adapted from \pgcitealt{vogel2001}{343}}\label{ex:mg-nom-acc-rep-intro}
\bg. *Ich {lade ein}, wen \tbf{mir} \tbf{sympathisch} \tbf{ist}.\\
 1\ac{sg}.\ac{nom} invite.\ac{pres}.1\ac{sg}\scsub{[acc]} \ac{rel}.\ac{an}.\ac{acc} 1\ac{sg}.\ac{dat} nice be.\ac{pres}.3\ac{sg}\scsub{[nom]}\\
 `I invite who I like.' \flushfill{Modern German, adapted from \pgcitealt{vogel2001}{344}}\label{ex:mg-acc-nom-rep-intro}

 In headless relative constructions, there is a single element that surfaces: the relative pronoun. In this section, I show that the relative pronoun is syntactically part of the relative clause. The evidence comes from extraposition data in Modern German. In Modern German, it is possible to extrapose a CP (a clause), but not a DP (a noun phrase). In this section I first show that Modern German CPs can be extraposed and DPs cannot. Then I illustrate how relative clauses including the relative pronoun in headless relatives pattern with CPs: they can be extraposed as well. I conclude that the relative pronoun is the internal element in the headless relative.

 The sentences in \ref{ex:mg-extrapose-cp} show that it is possible to extrapose a CP. In \ref{ex:mg-extrapose-cp-base}, the clausal object \tit{wie es dir geht} `how you are doing', marked here in bold, appears in its base position. It can be extraposed to the right edge of the clause, shown in \ref{ex:mg-extrapose-cp-moved}.

 \ex.\label{ex:mg-extrapose-cp}
 \ag. Mir ist \tbf{wie} \tbf{es} \tbf{dir} \tbf{geht} egal.\\
  1\tsc{sg}.\tsc{dat} is how it 2\tsc{sg}.\tsc{dat} goes {the same}\\
  `I don't care how you are doing.'\label{ex:mg-extrapose-cp-base}
 \bg. Mir is egal \tbf{wie} \tbf{es} \tbf{dir} \tbf{geht}.\\
  1\tsc{sg}.\tsc{dat} is {the same} how it 2\tsc{sg}.\tsc{dat} goes\\
  `I don't care how you are doing.' \label{ex:mg-extrapose-cp-moved}\flushfill{Modern German}

 \ref{ex:mg-extrapose-dp} illustrates that it is impossible to extrapose a DP. The clausal object of \ref{ex:mg-extrapose-cp} is replaced by the simplex noun phrase \tit{die Sache} `that matter'.
 In \ref{ex:mg-extrapose-dp-base} the object, marked in bold, appears in its base position. In \ref{ex:mg-extrapose-dp-moved} it is extraposed, and the sentence is no longer grammatical.

 \ex.\label{ex:mg-extrapose-dp}
 \ag. Mir ist \tbf{die} \tbf{Sache} egal.\\
  1\tsc{sg}.\tsc{dat} is that matter {the same}\\
  `I don't care about that matter.'\label{ex:mg-extrapose-dp-base}
 \bg. *Mir ist egal \tbf{die} \tbf{Sache}.\\
  1\tsc{sg}.\tsc{dat} is {the same} that matter\\
  `I don't care about that matter.' \label{ex:mg-extrapose-dp-moved}\flushfill{Modern German}

 The same asymmetry between CPs and DPs can be observed with relative clauses. A relative clause is a CP, and the head of a relative clause is a DP. The sentences in \ref{ex:extra-headed} contain the relative clause \tit{was er gekocht hat} `what he has stolen'. This is marked in bold in the examples. The (light) head of the relative clause is \tit{das}.
 In \ref{ex:extra-headed-base}, the relative clause and its head appear in base position. In \ref{ex:extra-headed-only-clause}, the relative clause is extraposed. This is grammatical, because it is possible to extrapose CPs in Modern German. In \ref{ex:extra-headed-head-clause}, the relative clause and the head are extraposed. This is ungrammatical, because it is possible to extrapose DPs.

 \ex.\label{ex:extra-headed}
 \ag. Jan hat das, \tbf{was} \tbf{er} \tbf{gekocht} \tbf{hat}, aufgegessen.\\
  Jan has that what he cooked has eaten\\
 `Jan has eaten what he cooked.'\label{ex:extra-headed-base}
 \bg. Jan hat das aufgegessen, \tbf{was} \tbf{er} \tbf{gekocht} \tbf{hat}.\\
  Jan has that eaten what he cooked has\\
 `Jan has eaten what he cooked.'\label{ex:extra-headed-only-clause}
 \cg. *Jan hat aufgegessen, das, \tbf{was} \tbf{er} \tbf{gekocht} \tbf{hat}.\\
  Jan has eaten that what he cooked has\\
 `Jan has eaten what he cooked.'\label{ex:extra-headed-head-clause} \flushfill{Modern German}

 The same can be observed in relative clauses without a head. \ref{ex:extra-headless} is the same sentence as in \ref{ex:extra-headed} only without the overt head. The relative clause is marked in bold again.
 In \ref{ex:extra-headless-base}, the relative clause appears in base position. In \ref{ex:extra-headless-clause}, the relative clause is extraposed. This is grammatical, because it is possible to extrapose CPs in Modern German. In \ref{ex:extra-headless-no-rel}, the relative clause is extraposed without the relative pronouns. This is ungrammatical, because the relative pronoun is part of the CP.
 This shows that the relative pronoun in headless relatives in Modern German are necessarily part of a CP, which is here a relative clause.

 \ex.\label{ex:extra-headless}
 \ag. Jan hat \tbf{was} \tbf{er} \tbf{gekocht} \tbf{hat} aufgegessen.\\
 Jan has what he cooked has eaten\\
 `Jan has eaten what he cooked.'\label{ex:extra-headless-base}
 \bg. Jan hat aufgegessen \tbf{was} \tbf{er} \tbf{gekocht} \tbf{hat}.\\
 Jan has eaten what he cooked has\\
 `Jan has eaten what he cooked.'\label{ex:extra-headless-clause}
 \bg. *Jan hat \tbf{was} aufgegessen \tbf{er} \tbf{gekocht} \tbf{hat}.\\
 Jan has what eaten he cooked has\\
 `Jan has eaten what he cooked.'\label{ex:extra-headless-no-rel}\flushfill{Modern German}

 In conclusion, extraposition facts show that the relative pronoun in Modern German is syntactically part of the relative clause. Therefore, the relative pronoun is the internal element in headless relative construction.


 The deletion in Modern German is not optional, but obligatory. The reason for that is that the weak demonstrative is phonologically(?) not heavy enough to be the head of a relative clause. Maybe not only phonologically, because \tit{vom} also does not work..

 are free relatives restrictive or non-restrictive? > restrictive, and restrictive and weak are incompatible :)  >> this is why we have deletion!

 \ex. Sie ist vom Mann, mit dem sie gestern ausgegangen ist, versetzt worden.




 In the previous section I introduced the relative pronoun as the internal element. This means that the other element is the external element. This section starts with the observation that there actually are languages in which two elements surface in so-called double-headed relative clauses. In these languages, the external head is a subset of the internal head, and that some features like \tsc{d} and case are necessarily excluded in the external head. I adopt this insight, and I apply it to the headless relative situation. I propose that the external head in headless relatives is a copy of a specific part of the relative pronoun.

 As I said earlier, I need two elements to do case competition with. In headless relatives, I only see a single one surfacing. However, some languages actually show two elements surfacing. Here there are two copies of the element, one inside the relative clause, one outside of the relative clause.

 \exg. [\tbf{doü} adiyan-o-no] \tbf{doü} deyalukhe\\
  sago give.3\tsc{pl}.\tsc{nonfut}-{tr}-\tsc{conn} sago finished.\tsc{adj}\\
  `The sago that they gave is finished.' \flushfill{Kombai, \pgcitealt{vries1993}{78}}

 The external element is not always an exact copy of the element inside of the relative clause. An example from Kombai shows that the element outside of the relative clause can also be a subset of what the element inside of the relative clause is. Here I give two examples, there is an \tit{old man} and a \tit{person}, and there is \tit{pig} and a \tit{thing}.

 \ex.
 \ag. [\tbf{yare} gamo khereja bogi-n-o] \tbf{rumu} na-momof-a\\
  {old man} join.\ac{ss} work do.\ac{dur}.3\ac{sg}.\ac{nf}-\ac{tr}-\ac{conn} person my-uncle-\ac{pred}\\
  `The old man, who is joining the work, is my uncle.' 77
 \bg. [\tbf{ai} fali-khano] \tbf{ro} nagu-n-ay-a.\\
  pig carry-go.3\tsc{pl}.\tsc{nf} thing our-\tsc{tr}-pig-\ac{pred}\\
  `The pig they took away, is ours.' \flushfill{Kombai, \pgcitealt{vries1993}{77}}

 Let me now apply what we have seen so far to headless relatives. Headless relatives do not have an overt NP, so this cannot be copied. However, there is the relative pronoun which is specified for number, gender, case, etc. Are all of these features copied onto the external element? The copy is the portion of the nominal extended projection c-commanded by the relative clause. A headless relative is a restrictive relative clause. Therefore, there is no \tsc{d} and no case.

 Is it possible to add features onto the external head after it has been copied? Yes, for example D, as the example shows, but also case.

 \exg. Junya-wa [Ayaka-ga \tbf{ringo}-o mui-ta] sono \tbf{ringo}-o tabe-ta.\\
 Junya-\ac{top} Ayaka-\ac{nom} apple-\ac{acc} peel-\ac{pst} that apple-\ac{acc} eat-\ac{pst}\\
 ‘Junya ate the apples that Ayaka peeled.’ \flushfill{Japanese, \pgcitealt{erlewine2016}{2}}

 In sum, the external element is a copy of a subset of the features of the relative pronoun. Definiteness and case are not copied. New features can be merged onto the external element.



\begin{table}[H]
 \center
 \caption {Relative pronouns in headless relatives in Modern German}
  \begin{tabular}{cc}
  \toprule
              & \ac{an} \\
    \cmidrule{2-2}
    \ac{nom}  & w-e-r  \\
    \ac{acc}  & w-e-n  \\
    \ac{dat}  & w-e-m  \\
  \bottomrule
  \end{tabular}
\end{table}

three morphemes: \tsc{wh}, \tsc{ana}, number+gender+case

accusative relative pronoun

\ex.
\begin{forest} boom
  [\tsc{wh}P
      [\tsc{wh}P
          [\tit{w}, roof]
      ]
      [\tsc{ana}P
          [\tsc{ana}P
              [\tit{e}, roof]
          ]
          [\tsc{acc}P,
          tikz={
          \node[label=below:\tit{n},
          draw,circle,
          scale=0.85,
          fit to=tree]{};
          }
              [\tsc{f}2]
              [\tsc{nom}P
                  [\tsc{f1}]
                  [\tsc{ind}P
                      [\phantom{xxx},
                      roof
                      ]
                  ]
              ]
          ]
      ]
  ]
\end{forest}

nominative relative pronoun

\ex.
\begin{forest} boom
  [\tsc{wh}P
      [\tsc{wh}P
          [\tit{w}, roof]
      ]
      [\tsc{ana}P
          [\tsc{ana}P
              [\tit{e}, roof]
          ]
          [\tsc{nom}P,
          tikz={
          \node[label=below:\tit{r},
          draw,circle,
          scale=0.85,
          fit to=tree]{};
          }
              [\tsc{f}1]
              [\tsc{ind}P
                  [\phantom{xxx},
                  roof
                  ]
              ]
          ]
      ]
  ]
\end{forest}


I copy the \tsc{ind} and I only merge the cases.

Modern German has two types of demonstratives: the strong one and the weak one.

The strong article is used when there is an anaphoric relation. Often there is a linguistic antecedent that is referred back to.

\exg. Hans hat heute \tbf{einen} \tbf{Freund} zum Essen mit nach Hause gebracht. Er hat uns vorher ein Foto \tbf{vom}/ \tbf{von} \tbf{dem} \tbf{Freund} gezeigt.\\
Hans has today a friend {to the} dinner with to home brought he has us beforehand a photo {of the\scsub{weak}} of the\scsub{strong} friend shown\\
`Hans brought a friend home for dinner today. He had shown us a photo of the friend beforehand.'

Weak articles are used when situational uniqueness is involved. Uniqueness can be global or within a restricted domain. The discourse participants mutually shared knowledge that uniqueness holds.

\ex.
\ag. Der Einbrecher ist {zum Glück} vom /von dem Hund verjagt worden.\\
the burglar is luckily {by the\scsub{weak}} by the\scsub{strong} dog {chased away} been\\
`Luckily, the burglar was chased away by the dog.'
\bg. Armstrong flog als erster zum Mond.\\
Armstrong flew as {first one} {to the\scsub{weak}} moon\\
`Armstrong was the first one to fly to the moon.' \flushfill{Modern German, \pgcitealt{schwarz2009}{40}}

In the headless relatives, there is uniqueness. Show?

The strong article cannot be used because it does not go together with the free choice interpretation of \tsc{wh}-relatives (say something about Hanink).

The weak article is used. accusative:

\ex.
\begin{forest} boom
[\tsc{acc}P,
tikz={
\node[label=below:\tit{n},
draw,circle,
scale=0.85,
fit to=tree]{};
}
    [\tsc{f}2]
    [\tsc{nom}P
        [\tsc{f1}]
        [\tsc{ind}P
            [\phantom{xxx},
            roof
            ]
        ]
    ]
]
\end{forest}

nominative:

\ex.
\begin{forest} boom
[\tsc{nom}P,
tikz={
\node[label=below:\tit{r},
draw,circle,
scale=0.85,
fit to=tree]{};
}
    [\tsc{f1}]
    [\tsc{ind}P
        [\phantom{xxx},
        roof
        ]
    ]
]
\end{forest}

\exg. Uns besucht \tbf{wen} \tbf{Maria} \tbf{mag}.\\
 we.\ac{acc} visit.3\ac{sg}\scsub{[nom]} \tsc{rel}.\ac{acc}.\tsc{an} Maria.\ac{nom} like.3\ac{sg}\scsub{[acc]}\\
 `Who visits us, Maria likes.' \flushfill{adapted from \pgcitealt{vogel2001}{343}}

the internal case is more complex than the external case, and the internal base part is more complex than the external non-cas part

\begin{table}[H]
  \center
 \caption {Modern German: \ac{int} > \ac{ext}}
  \begin{tabular}[b]{cc}
      \toprule
      \ac{int}  &   \ac{ext} \\ \cmidrule{1-2}
      \begin{forest} boom
        [\tsc{wh}P
            [\tsc{wh}P
                [\tit{w}, roof]
            ]
            [\tsc{ana}P
                [\tsc{ana}P
                    [\tit{e}, roof]
                ]
                [\tsc{acc}P,
                tikz={
                \node[label=below:\tit{n},
                draw,circle,
                scale=0.85,
                fit to=tree]{};
                }
                    [\tsc{f}2]
                    [\tsc{nom}P,
                    tikz={
                    \node[draw,circle,
                    fill=DG,fill opacity=0.2,
                    DG,dashed,
                    scale=0.8,
                    fit to=tree]{};
                    }
                        [\tsc{f1}]
                        [\tsc{ind}P
                            [\phantom{xxx},
                            roof, baseline
                            ]
                        ]
                    ]
                ]
            ]
        ]
      \end{forest}
      &
      \begin{forest} boom
        [\textcolor{DG}{\ac{nom}},edge=DG,
        tikz={
        \node[label=below:\textcolor{DG}{\tit{r}},
        draw,circle,
        scale=0.75,
        DG,
        fit to=tree]{};
        \node[
        draw,circle,
        scale=0.8,
        dashed,DG,
        fit to=tree]{};
        }
            [\textcolor{DG}{\tsc{f1}},edge=DG]
            [\textcolor{DG}{\tsc{ind}P},edge=DG
                [\phantom{xxx},
                roof, baseline,edge=DG
                ]
            ]
        ]
      \end{forest}\\
      \bottomrule
  \end{tabular}
  \label{tbl:mg-int-wins}
\end{table}

\exg. *Ich {lade ein}, wen \tbf{mir} \tbf{sympathisch} \tbf{ist}.\\
1\ac{sg}.\ac{nom} invite.\ac{pres}.1\ac{sg}\scsub{[acc]} \ac{rel}.\ac{an}.\ac{acc} 1\ac{sg}.\ac{dat} nice be.\ac{pres}.3\ac{sg}\scsub{[nom]}\\
`I invite who I like.' \flushfill{Modern German, adapted from \pgcitealt{vogel2001}{344}}\label{ex:mg-acc-nom-rep}

the external case is more complex than the internal case, but the external base part is not more complex than the internal base part

\begin{table}[H]
  \center
 \caption {Modern German: \ac{int} > \ac{ext}}
  \begin{tabular}[b]{cc}
      \toprule
      \ac{int}  &   \ac{ext} \\ \cmidrule{1-2}
      \begin{forest} boom
        [\tsc{wh}P
            [\tsc{wh}P
                [\tit{w}, roof]
            ]
            [\tsc{ana}P
                [\tsc{ana}P
                    [\tit{e}, roof]
                ]
                [\textcolor{DG}{\ac{nom}},edge=DG,
                tikz={
                \node[label=below:\textcolor{DG}{\tit{r}},
                draw,circle,
                scale=0.75,
                DG,
                fit to=tree]{};
                \node[
                draw,circle,
                scale=0.8,
                dashed,DG,
                fit to=tree]{};
                }
                    [\textcolor{DG}{\tsc{f1}},edge=DG]
                    [\textcolor{DG}{\tsc{ind}P},edge=DG
                        [\phantom{xxx},
                        roof, baseline,edge=DG
                        ]
                    ]
                ]
            ]
        ]
      \end{forest}
      &
      \begin{forest} boom
      [\tsc{acc}P,
      tikz={
      \node[label=below:\tit{n},
      draw,circle,
      scale=0.85,
      fit to=tree]{};
      }
          [\tsc{f}2]
          [\tsc{nom}P,
          tikz={
          \node[draw,circle,
          fill=DG,fill opacity=0.2,
          DG,dashed,
          scale=0.8,
          fit to=tree]{};
          }
              [\tsc{f1}]
              [\tsc{ind}P
                  [\phantom{xxx},
                  roof, baseline
                  ]
              ]
          ]
      ]
      \end{forest}\\
      \bottomrule
  \end{tabular}
  \label{tbl:mg-ext-wins}
\end{table}



\section{Deriving the matching type}\label{sec:deriving-matching}


Polish only allows the deletion of the light head in the matching situation. It is not obligatory there, you can just as well have a light-headed relative. The deletion is possible, because you have two elements that are pretty similar?

\exg. Jan czyta to, co Maria czyta.\\
 Jan read this what Maria reads\\
 `Jan reads what Maria reads.' \flushfill{Polish, \pgcitealt{citko2004}{96}}


Radek: Czech distinguishes between accidental uniqueness and inherent uniqueness. Accidental uniqueness: with \tsc{dem}, inherent uniqueness: without \tsc{dem}.

Radek's situation:

Two student assistants A and B are at their shared workdesk, which they share with other student assistants and where there’s a computer and a couple of other things, including a book (it doesn’t really matter to whom the book belongs). A is looking for a pencil, B says

\exg. Nějaká tužka je vedle {počítače /\#toho počítače}.\\
some pencil is {next to} computer \tsc{dem} computer\\
`There’s a pencil next to the computer.'

All situations like the topic situation – A and B’s shared office (desk)– have exactly one computer in it.

\exg. Nějaká tužka je vedle {té knížky /\#knížky}\\
some pencil is {next to} \tsc{dem} book book\\
`There’s a pencil next to the book.'

There is exactly one book in the topic situation – A and B’s shared office (desk) – and it does not hold that all situation like the topic situation have exactly one book in it

Florian showed that this is different for Modern German:

\begin{table}[H]
\begin{tabular}{c|ccc}
\toprule
       & anaphoric                & situational uniqueness              & inherent uniqueness                 \\
       \cmidrule{2-4}
Polish & \tsc{dem}  & \cellcolor{DG}\tsc{dem}             & ∅                                   \\
German & \tsc{dem}\scsub{strong}  & \cellcolor{LG}\tsc{dem}\scsub{weak} & \cellcolor{LG}\tsc{dem}\scsub{weak} \\
\bottomrule
\end{tabular}
\end{table}

\tit{to} is incompatible with \tit{ever}, because \tit{to} makes it accidentally uniqueness and \tit{ever} requires inherent uniqueness


\section{Excluding the external-only type}




\section{Summary}

The linguistic counterpart of `allow \tsc{ext}?' is whether the internal base and the external base are syncretic (base\scsub{int} = base\scsub{ext}?).
The linguistic counterpart of `allow \tsc{int}?' is whether the external base is a clitic (base\scsub{ext} = clitic?).

\begin{figure}[H]
  \centering
    \footnotesize{
    \begin{tikzpicture}[node distance=1.5cm]
      \node (question2) [question]
      {base\scsub{int} = base\scsub{ext}?};
          \node (outcome2) [outcome, below of=question2, xshift=-1.5cm]
          {complex};
              \node (example2) [example, below of=outcome2, yshift=0.25cm]
              {\scriptsize{e.g. Gothic, Old High German, Classical Greek}};
          \node (question3) [question, below of=question2, xshift=2cm, yshift=-0.5cm]
          {base\scsub{ext} = clitic?};
              \node (outcome3) [outcome, below of=question3, xshift=-1.5cm]
              {\ac{int} + complex};
                  \node (example3) [example, below of=outcome3, yshift=0.25cm]
                  {\scriptsize{e.g. Modern German\\\phantom{x}}};
              \node (outcome4) [outcome, below of=question3, xshift=1.5cm]
              {matching};
                  \node (example4) [example, below of=outcome4, yshift=0.25cm]
                  {\scriptsize{e.g. Polish\\\phantom{x}}};

    \draw [arrow] (question2) -- node[anchor=east] {yes} (outcome2);
    \draw [arrow] (question2) -- node[anchor=west] {no} (question3);
    \draw [arrow] (question3) -- node[anchor=east] {no} (outcome3);
    \draw [arrow] (question3) -- node[anchor=west] {yes} (outcome4);
    \end{tikzpicture}
    }
    \caption{Two theoretical parameters generate three language types}
    \label{fig:formal-parameters}
\end{figure}

\section{Aside: a larger syntactic context}

If you talk about different patterns, there can be different locations to put your parameters. Himmelreich put her parameters in the structure. I put my parameters in the elements themselves. I show what an analysis like Himmelreich looks like, and I show then that it is difficult to reduce that then to differences in the lexicon (because it has to do with agree?).

So what I do is keep the parameters that she was differing stable. I change the things that she kept constant, the internal and external element. Does her structure then work with what I want? Not entirely, because I have to do a c-command that is going in the wrong direction.
Then I show a syntactic structure that could be compatible with mine, and I show why a grafting one is not.



In this dissertation I focus on when languages allow the internal and external case to win the case competition. In my proposal, this depends on the comparison between the internal and external base. The larger syntactic context in which this takes place should be kept stable. For concreteness, I show a possible implementation in Cinque's double-headed analysis of relative clause. I do by no means claim that claim this is the only or even correct implementation.

According to Cinque, every type of relative clause in every language is underlyingly double-headed. Evidence for this claim comes from languages that show this morphologically. An example from Kombai is given in \ref{ex:kombai}. The head of the relative clause is \tit{doü} `sago', and it appears inside the relative clause and outside.

\exg. [\tbf{doü} adiyan-o-no] \tbf{doü} deyalukhe\\
 sago give.3\tsc{pl}.\tsc{nonfut}-{tr}-\tsc{conn} sago finished.\tsc{adj}\\
 `The sago that they gave is finished.' \flushfill{Kombai, \pgcitealt{vries1993}{78}}\label{ex:kombai}

The internal and external instances of \tit{doü} correspond to the internal and external element I assume to be there in the headless relatives.

\ref{ex:double-syntax} shows the syntactic structure of the sentence in \ref{ex:kombai}.

\ex.
\begin{forest} boom
[CP
   [FP
      [CP
          [\tsc{int}
             [\tit{doü}, roof]
          ]
          [CP
              [\tit{adiyan-o-no}, roof]
          ]
      ]
      [\tsc{ext}
         [\tit{doü}, roof]
      ]
   ]
   [VP
      [\tit{deyalukhe}, roof]
   ]
]
\end{forest}\label{ex:double-syntax}

In most languages one of the two heads is deleted throughout the derivation.

According to \citealt{cinqueforthcoming}, the internal element can delete the external element, because the internal element c-commands the external element. This is c-command according to Kayne's definition of it: the internal element is in the specifier of the specifier of the FP.

\ex.
\begin{forest} boom
[
   [CP
       [\tsc{int}
          [\phantom{xxx}, roof]
       ]
       [CP
           [\phantom{xxx}, roof]
       ]
   ]
   [\tsc{ext}
      [\phantom{xxx}, roof]
   ]
]
\end{forest}\label{ex:cinque-int-wins}

In order for the internal element to be able to delete the external element, a movement needs to take place. The external element moves over the relative clause.\footnote{
What remains unclear is what the trigger is for the movement of the external element over relative clause is.
}
From this position, the external element can delete the internal one, because the external element c-commands the internal one.

\ex.
\begin{forest} boom
[
    [\tsc{ext}
       [\phantom{xxx}, roof]
    ]
    [FP
       [CP
           [\tsc{int}
              [\phantom{xxx}, roof]
           ]
           [CP
               [\phantom{xxx}, roof]
           ]
       ]
       [\tit{t\scsub{ext}}]
    ]
]
\end{forest}

Also talk about \tsc{d} here, and that maybe Old High German deletes a thing without a \tsc{d} when the internal thing wins. does that also have a not so definite interpretation?


What does not work:

For this pattern a single element analysis seems intuitive, if you assume that case is complex and that syntax works bottom-up. First you built the relative clause, with the big case in there. Then you build the main clause and you let the more complex case in the embedded clause license the main clause predicate.

Consider the example in \ref{ex:mg-nom-acc-grafting}. Here the internal case is accusative and the external one nominative.

\exg. Uns besucht \tbf{wen} \tbf{Maria} \tbf{mag}.\\
 we.\ac{acc} visit.3\ac{sg}\scsub{[nom]} \tsc{rel}.\ac{acc}.\tsc{an} Maria.\ac{nom} like.3\ac{sg}\scsub{[acc]}\\
 `Who visits us, Maria likes.' \flushfill{adapted from \pgcitealt{vogel2001}{343}}\label{ex:mg-nom-acc-grafting}

The relative clause is built, including the accusative relative pronoun. Now the main clause predicate can merge with the nominative that is contained within the accusative.

 \ex.
 \begin{forest} boom
  [,name=src, s sep=15mm
   [VP
      [\tit{besucht}, roof]
   ]
    [,no edge, s sep=20mm
        [\ac{acc}P,
     tikz={
     \node[label=below:\tit{wen},
     draw,circle,
     scale=0.85,
     fit to=tree]{};
     }
            [\tsc{f2}]
            [\tsc{nomP},name=tgt
                [\tsc{f1}]
                [XP
                    [\phantom{xxx}, roof]
                ]
            ]
        ]
     [VP
        [\tit{Maria mag}, roof]
     ]
   ]
  ]
  \draw (src) to[out=south east,in=north east] (tgt);
 \end{forest}\label{ex:acc-nom-grafting}

The other way around does not work. Consider \ref{ex:mg-acc-nom-grafting}. This is an example with nominative as internal case and accusative as external case.

\exg. *Ich {lade ein}, wen \tbf{mir} \tbf{sympathisch} \tbf{ist}.\\
I.\ac{nom} invite.1\ac{sg}\scsub{[acc]} \tsc{rel}.\ac{acc}.\tsc{an} I.\ac{dat} nice be.3\ac{sg}\scsub{[nom]}\\
`I invite who I like.' \flushfill{adapted from \pgcitealt{vogel2001}{344}}\label{ex:mg-acc-nom-grafting}

Now the relative clause is built first again, this time only including the nominative case. There is no accusative node to merge with for the external predicate. Instead, the relative pronoun would need to grow to accusative somehow and then the merge could take place. This is the desired result, because the sentence is ungrammatical.

\ex.
\begin{forest} boom
  [,name=src, s sep=15mm
     [VP
         [\tit{lade ein}, roof]
     ]
         [,no edge
       [\tsc{nomP},
       tikz={
       \node[label=below:\tit{wer},
       draw,circle,
       scale=0.85,
       fit to=tree]{};
       }
         [\tsc{f1}]
         [XP
           [\phantom{xxx}, roof]
         ]
       ]
       [VP
         [\tit{mir sympatisch ist}, roof]
       ]
      ]
    ]
\end{forest}\label{ex:nom-acc-grafting}

So, this seems to work fine. The assumptions you have to do in order to make this are the following. First, case is complex. Second, you can remerge an embedded node (grafting). For the first one I have argued in Chapter \ref{ch:decomposition}. The second one could use some additional argumentation. It is a mix between internal remerge (move) and external merge, namely external remerge. Other literature on multidominance and grafting, other phenomena. Problems: linearization, .. But even if fix all these theoretical problems, there is an empirical one.

That is, I want to connect this behavior of Modern German headless relatives to the shape of its relative pronouns. These pronouns are \tsc{wh}-elements. The OHG and Gothic ones are not \tsc{wh}, they are \tsc{d}. Their relative pronouns look different, and so their headless relatives can also behave differently.




Himmelreich

there are agree relations between
- V\scsub{ext} and \tsc{ext}
- V\scsub{int} and \tsc{int}
- \tsc{int} and \tsc{ext}

three parameters:
1 relation between V\scsub{ext} and \tsc{ext} + V\scsub{int} and \tsc{int} are symmetric or asymmetric
2 relation between \tsc{ext} and \tsc{int} are symmetric or asymmetric
3 if \tsc{ext} --- \tsc{int} is asymmetric, \tsc{ext} or \tsc{int} probes

I keep the parameters she has stable, the bigger syntactic context is the same everywhere. I vary the content of \tsc{ext}

\section{Aside: D between ϕ and K}

How can cases and number/gender be packaged together to the exclusion of D? Because D is actually merged below K. Let me illustrate this.

\ex. \tbf{Spellout Algorithm:}\\
Merge F and \label{ex:spellout}
 \a. Spell out FP.
 \b. If (a) fails, attempt movement of the spec of the complement of \tsc{f}, and retry (a).
 \b. If (b) fails, move the complement of \tsc{f}, and retry (a).

\ex.
\begin{forest} boom
  [\tsc{class}P,
  tikz={
  \node[label=below:\tit{er},
  draw,circle,
  scale=0.80,
  fit to=tree]{};
  }
      [\tsc{class}]
      [\tsc{ref}]
  ]
\end{forest}

When a new match is found, it overrides previous spellouts.

\ex. \tbf{Cyclic Override} \citep{starke2018}:\\
Lexicalisation at a node XP overrides any previous match at a phrase contained in XP.

\ex.
\begin{forest} boom
  [\tsc{masc}P,
  tikz={
  \node[label=below:\tit{er},
  draw,circle,
  scale=0.85,
  fit to=tree]{};
  }
      [\tsc{masc}]
      [\tsc{class}P
          [\tsc{class}]
          [\tsc{ref}]
      ]
  ]
\end{forest}

\ex.
\begin{forest} boom
  [\tsc{ind}P,
  tikz={
  \node[label=below:\tit{er},
  draw,circle,
  scale=0.85,
  fit to=tree]{};
  }
      [\tsc{ind}]
      [\tsc{masc}P
          [\tsc{masc}]
          [\tsc{class}P
              [\tsc{class}]
              [\tsc{ref}]
          ]
      ]
  ]
\end{forest}

\tsc{ana} cannot be spelled out together with what has been spelled out so far. Try the movements, which is not helping. I need to make a left branch.

\ex.
\begin{forest} boom
  [\tsc{ana}P
      [\tsc{ana}]
      [\tsc{ind}P,
      tikz={
      \node[label=below:\tit{er},
      draw,circle,
      scale=0.85,
      fit to=tree]{};
      }
          [\tsc{ind}]
          [\tsc{masc}P
              [\tsc{masc}]
              [\tsc{class}P
                  [\tsc{class}]
                  [\tsc{ref}]
              ]
          ]
      ]
  ]
\end{forest}

A specifier is constructed.

\ex. \tbf{Spec Formation} \citep{starke2018}:\\
If Merge F has failed to spell out (even after backtracking), try to spawn a new derivation providing the feature F and merge that with the current derivation, projecting the feature F at the top node.\label{ex:specformation}

\ex.
\begin{forest} boom
  [\tsc{d}P
      [\tsc{ana}P,
      tikz={
      \node[label=below:\tit{d},
      draw,circle,
      scale=0.80,
      fit to=tree]{};
      }
          [\tsc{d}]
          [\tsc{ana}]
      ]
      [\tsc{ind}P,
      tikz={
      \node[label=below:\tit{er},
      draw,circle,
      scale=0.85,
      fit to=tree]{};
      }
          [\tsc{ind}]
          [\tsc{masc}P
              [\tsc{masc}]
              [\tsc{class}P
                  [\tsc{class}]
                  [\tsc{ref}]
              ]
          ]
      ]
  ]
\end{forest}


If the spellout procedure in \ref{ex:spellout} fails, backtracking takes place.

\ex. \tbf{Backtracking} \citep{starke2018}:\\
When spellout fails, go back to the previous cycle, and try the next option for that cycle.\label{ex:backtracking}

\ex.
\a.
\begin{forest} boom
  [\tsc{nom}P,
  tikz={
  \node[label=below:\tit{er},
  draw,circle,
  scale=0.85,
  fit to=tree]{};
  }
      [\tsc{f}1]
      [\tsc{ind}P
          [\tsc{ind}]
          [\tsc{masc}P
              [\tsc{masc}]
              [\tsc{class}P
                  [\tsc{class}]
                  [\tsc{ref}]
              ]
          ]
      ]
  ]
\end{forest}
\b.
\begin{forest} boom
[\tsc{d}P,
tikz={
\node[label=below:\tit{d},
draw,circle,
scale=0.80,
fit to=tree]{};
}
    [\tsc{d}]
    [\tsc{ana}]
]
\end{forest}

\ex.
\begin{forest} boom
  [\tsc{acc}P,
  tikz={
  \node[label=below:\tit{er},
  draw,circle,
  scale=0.85,
  fit to=tree]{};
  }
      [\tsc{f}2]
      [\tsc{nom}P
          [\tsc{f}1]
          [\tsc{ind}P
              [\tsc{ind}]
              [\tsc{masc}P
                  [\tsc{masc}]
                  [\tsc{class}P
                      [\tsc{class}]
                      [\tsc{ref}]
                  ]
              ]
          ]
      ]
  ]
\end{forest}



\begin{forest} boom
  [\tsc{d}P
      [\tsc{d}P,
      tikz={
      \node[label=below:\tit{d},
      draw,circle,
      scale=0.80,
      fit to=tree]{};
      }
          [\tsc{d}]
          [\tsc{ana}]
      ]
      [\tsc{nom}P,
      tikz={
      \node[label=below:\tit{er},
      draw,circle,
      scale=0.80,
      fit to=tree]{};
      }
          [\tsc{f1}]
          [\tsc{ind}P
              [\phantom{xxx},
              roof, baseline
              ]
          ]
      ]
  ]
\end{forest}




\phantom{x}
