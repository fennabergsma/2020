feature% !TEX root = thesis.tex

\chapter{Constituent containment}\label{ch:relativization}

In Chapter \ref{ch:typology} I introduced two descriptive parameters that generate the attested languages, as shown in Figure \ref{fig:two-parameters}.
The first parameter concerns whether the external case is allowed to surface when it wins the case competition (allow \tsc{ext}?). This parameter distinguishes between non-matching languages (e.g. Old High German) on the one hand and internal-only languages (e.g. Modern German) and matching languages (e.g. Polish) on the other hand.
The second parameter concerns whether the internal case is allowed to surface when it wins the case competition (allow \tsc{int?}). This parameter distinguishes between internal-only languages (e.g. as Modern German) on the one hand and non-matching languages (e.g. Polish) on the other hand.

\begin{figure}[H]
  \centering
    \footnotesize{
    \begin{tikzpicture}[node distance=1.5cm]
      \node (question2) [question]
      {allow \tsc{int}?};
          \node (outcome2) [outcome, below of=question2, xshift=-1.5cm]
          {matching};
              \node (example2) [example, below of=outcome2, yshift=0.25cm]
              {\scriptsize{e.g. Polish\\\phantom{x}}};
          \node (question3) [question, below of=question2, xshift=2cm, yshift=-0.5cm]
          {allow \tsc{ext}?};
              \node (outcome3) [outcome, below of=question3, xshift=-1.5cm]
              {internal-only};
                  \node (example3) [example, below of=outcome3, yshift=0.25cm]
                  {\scriptsize{e.g. Modern German\\\phantom{x}}};
              \node (outcome4) [outcome, below of=question3, xshift=1.5cm]
              {unrestricted};
                  \node (example4) [example, below of=outcome4, yshift=0.25cm]
                  {\scriptsize{e.g. Gothic, Old High German, Classical Greek}};

    \draw [arrow] (question2) -- node[anchor=east] {no} (outcome2);
    \draw [arrow] (question2) -- node[anchor=west] {yes} (question3);
    \draw [arrow] (question3) -- node[anchor=east] {no} (outcome3);
    \draw [arrow] (question3) -- node[anchor=west] {yes} (outcome4);
    \end{tikzpicture}
    }
    \caption{Two descriptive parameters generate three language types}
    \label{fig:two-parameters}
\end{figure}

``A natural question at this point is whether this typology needs to be fully stipulative, or is to some extent derivable from independent properties of individual languages'' \citet{grosu1994}{147}

The goal of this chapter is to give the theoretical counterparts of these descriptive parameters. Goal: something that can be observed independently.

This chapter is structured as follows.


\section{The basic idea}\label{sec:basic-idea}

This section gives the basic idea behind my proposal. Throughout the rest of the chapter I motivate my proposal, and I illustrate it with examples.
First I introduce the assumption that headless relatives are derived from light-headed relatives. The light head bears the external case, and the relative pronoun bears the internal case, illustrated in \ref{ex:light+rel}.

\ex. light head\scsub{ext} [relative pronoun\scsub{int} ... ]\label{ex:light+rel}

In a headless relative, either the light head or the relative pronoun is absent.
This happens under the following condition: an element (i.e. light head or relative pronoun) is absent when each of its constituents is contained in a constituent of the other element (i.e. light head or relative pronoun).

Consider the light-headed relative in \ref{ex:ohg-light-headed}.
\tit{Thér} `\tsc{dem}.\tsc{sg}.\tsc{m}.\tsc{nom}' is the light head of the relative clause. This is the element that appears in the external case, the case that reflects the grammatical role in the main clause.
\tit{Then} `\tsc{rel}.\tsc{sg}.\tsc{m}.\tsc{acc}' is the relative pronoun of the relative clause. This is the element that appears in the internal case, the case that reflects the grammatical role within the relative clause.

\exg. eno nist thiz thér \tbf{then} \tbf{ir} \tbf{suochet} \tbf{zi} \tbf{arslahanne}?\\
 now {not be.3\ac{sg}} \tsc{dem}.\tsc{sg}.\tsc{n}.\tsc{nom} \tsc{dem}.\tsc{sg}.\tsc{m}.\tsc{nom}
 \tsc{rel}.\tsc{sg}.\tsc{m}.\tsc{acc} 2\ac{pl}.\tsc{nom} seek.2\tsc{pl} to kill.\tsc{inf}.\ac{sg}.\tsc{dat}\\
 `Isn't this now the one, who you seek to kill?'\label{ex:ohg-light-headed}

In my proposal, the difference between a light-headed relative and a headless relative is that in a headless relative either the light head or the relative pronoun does not surface.\footnote{
Others say this too.
} The surfacing element is the one that bears the winning case, and the absent element is the one that bears the losing case. This means that what I have so far been glossing as and calling the relative pronoun is sometimes actually the light head and sometimes the relative pronoun. To reflect that, I call the surfacing element from now on the surface pronoun.

Table \ref{tbl:options-surface-pronoun} lists the two possibilities that I just laid out plus an additional one.
First, the relative pronoun, which bears the internal case, can appear as the surface pronoun. Second, the light head, which bears the external case, can appear as the surface pronoun. The third option is that there is no grammatical form for the surface pronoun.

\begin{table}[H]
  \center
  \caption{Options for the surface pronoun}
\begin{tabular}{ccc}
  \toprule
surface pronoun             \\
\cmidrule(lr){1-1}
light head\scsub{ext}       \\
relative pronoun\scsub{int} \\
{*}                         \\
\bottomrule
\end{tabular}
\label{tbl:options-surface-pronoun}
\end{table}

Table \ref{tbl:overview-rel-light} gives an overview per language type of which of the three options in Table \ref{tbl:options-surface-pronoun} is chosen when the internal and external case differ.

\begin{table}[H]
  \center
  \caption{Light head and relative pronoun per language}
\begin{tabular}{cccc}
  \toprule
                & \tsc{int} > \tsc{ext}        & \tsc{int} < \tsc{ext} &                  \\
                & relative pronoun\scsub{int}  & light head\scsub{ext} &                  \\
                \cmidrule{2-3}
non-matching    & ✔                            & ✔                     & Old High German  \\
internal-only   & ✔                            & *                     & Modern German    \\
matching        & *                            & *                     & Polish           \\
external-only   & *                            & ✔                     & not attested     \\
\bottomrule
\end{tabular}
\label{tbl:overview-rel-light}
\end{table}

The first column list the types of languages.
The second column shows the situation in which the internal case is the most complex. The potential surface pronoun is the relative pronoun that bears the internal case.
The third column shows the situation in which the external case is the most complex. The potential surface pronoun is the light head that bears the external case.
The checkmark (✔) and asterix (*) indicate whether the potential surface pronoun appears or whether there is no grammatical form for the surface pronoun.
The fourth column gives an example of the language type that I discuss in this chapter.
A language of the non-matching type, like Old High German, allows both the internal case and the external case to surface when either of them wins the case competition. The surface pronoun can be either the light head or the relative pronoun.
A language of the internal-only type, like Modern German, allows only the internal case to surface when it wins the case competition, and it does not allow the external case to do so. The surface pronoun can only be the relative pronoun and not the light head.
The matching type of language, as Polish, allows neither the internal nor the external case to surface when either of them wins the case competition. The surface pronoun can neither be the relative pronoun nor the light head.\footnote{
That is, in the non-matching cases. Later on I show that the relative pronoun surfaces in matching contexts.
}
The language type that is not attested is the external-only type. That means that there is no language that allows only the external case to surface when it wins the case competition, and it does not allow the internal case to do so. In other words, there exist no language, in which the surface pronoun can only be the light head and not the relative pronoun.

What I have done in this section so far is reformulate the two descriptive parameters from Figure \ref{fig:two-parameters} into two other descriptive parameters.
Whether the the internal case is allowed to surface corresponds to whether the relative pronoun surfaces. That implicates that the light head is absent, so it has been deleted.
Similarly, whether the the external case is allowed to surface corresponds to whether the light head surfaces. That implicates that the relative pronoun is absent, so it has been deleted.
I show this in Figure \ref{fig:two-parameters-different}.

\begin{figure}[H]
  \centering
    \footnotesize{
    \begin{tikzpicture}[node distance=1.5cm]
      \node (question2) [question]
      {delete \tsc{rel}?};
          \node (outcome2) [outcome, below of=question2, xshift=-1.5cm]
          {matching};
              \node (example2) [example, below of=outcome2, yshift=0.25cm]
              {\scriptsize{e.g. Polish\\\phantom{x}}};
          \node (question3) [question, below of=question2, xshift=2cm, yshift=-0.5cm]
          {delete \tsc{lh}?};
              \node (outcome3) [outcome, below of=question3, xshift=-1.5cm]
              {internal-only};
                  \node (example3) [example, below of=outcome3, yshift=0.25cm]
                  {\scriptsize{e.g. Modern German\\\phantom{x}}};
              \node (outcome4) [outcome, below of=question3, xshift=1.5cm]
              {unrestricted};
                  \node (example4) [example, below of=outcome4, yshift=0.25cm]
                  {\scriptsize{e.g. Gothic, Old High German, Classical Greek}};

    \draw [arrow] (question2) -- node[anchor=east] {no} (outcome2);
    \draw [arrow] (question2) -- node[anchor=west] {yes} (question3);
    \draw [arrow] (question3) -- node[anchor=east] {no} (outcome3);
    \draw [arrow] (question3) -- node[anchor=west] {yes} (outcome4);
    \end{tikzpicture}
    }
    \caption{Allow \tsc{rel}/\tsc{lh} as parameters}
    \label{fig:two-parameters-different}
\end{figure}

Reformulating these parameters is not just restating the generalization in different terms. With this new formulation, I am able to identify the elements (i.e. the light head and the relative pronoun) that bear the cases and indicate whether it is possible to delete them in a particular language type. In my analysis, it is the relative pronoun that is sometimes able to delete the light head or the light head that is able to delete the relative pronoun.

I propose that whether or not a light head or relative pronoun) can delete the other follows from the comparison between the light head and relative pronoun. Light heads and relative pronouns namely do not only correspond to case features, but also to other features (having to do with number, gender, etc.). In this chapter I show that light heads and relative pronouns differ between the languages I describe. I illustrate how these differences in light heads and relative pronouns lead to the different language types in Figure \ref{fig:two-parameters-different}.

In the comparison between light heads and relative pronouns, I rely on containment, just as I did in Chapter \ref{ch:decomposition} when comparing cases. For case competition, I reasoned as follows. A more complex case wins over a less complex case because the former contains all features that the latter contains. Concretely, the dative wins over the accusative because the dative contains all features that the accusative contains, the dative wins over the nominative because the dative contains all features that the nominative contains, and the accusative wins over the nominative because the accusative contains all features that the nominative contains.

Figure \ref{fig:acc-nom-structure} illustrates this for the accusative and the nominative. The XP here can be any type of case marked element.
I draw a dashed circle around the features that are features in both the accusative and the nominative.
As each feature of the accusative is also a feature within the nominative, the accusative wins the case competition. I illustrate this by marking the content of the dashed circle for the nominative gray.

\begin{figure}[H]
  \center
  \begin{tabular}[b]{ccc}
      \toprule
      \tsc{nom} & & \tsc{acc} \\
      \cmidrule(lr){1-1} \cmidrule(lr){3-3}
      \begin{forest} boom
        [\tsc{nom}P,
        tikz={
        \node[draw,circle,
        dashed,
        scale=0.8,
        fill=DG,fill opacity=0.2,
        fit to=tree]{};
        }
            [\tsc{f}1]
            [XP
                [\phantom{xxx}, roof, baseline]
            ]
        ]
      \end{forest}
      & \phantom{x} &
      \begin{forest} boom
        [\tsc{acc}P
            [\tsc{f}2]
            [\tsc{nom}P,
            tikz={
            \node[draw,circle,
            dashed,
            scale=0.8,
            fit to=tree]{};
            }
                [\tsc{f}1]
                [XP
                    [\phantom{xxx}, roof, baseline]
                ]
            ]
        ]
      \end{forest}\\
      \bottomrule
  \end{tabular}
   \caption {\tsc{nom} vs. \tsc{acc} = \tsc{acc}}
  \label{fig:acc-nom-structure}
\end{figure}

I formulated case containment in terms of feature containment. The accusative wins over the nominative because the accusative contains all features that the nominative contains.
The image in Figure \ref{fig:acc-nom-structure} is also compatible with a stronger requirement than feature containment: constituent containment. The accusative wins over the nominative because it contains the \tsc{nom}P.

In Figure \ref{fig:acc-nom-structure-moved-out} I show the same picture as in \ref{fig:acc-nom-structure} except for that the XP has moved out of the \tsc{acc}P.
I draw a dashed circle around the constituents that are constituents in both the accusative and the nominative.
There is still feature containment: the nominative contains \tsc{f}1 and XP and so does the accusative. However, there is no longer constituent containment: the \tsc{nom}P constituent containing \tsc{f}1 and XP is no longer a constituent in the \tsc{acc}P.

\begin{figure}[H]
  \center
  \begin{tabular}[b]{ccc}
      \toprule
      \tsc{nom} & & \tsc{acc} \\
      \cmidrule(lr){1-1} \cmidrule(lr){3-3}
      \begin{forest} boom
        [\tsc{nom}P
            [\tsc{f}1]
            [XP,
            tikz={
            \node[draw,circle,
            dashed,
            scale=0.8,
            fit to=tree]{};
            }
                [\phantom{xxx}, roof, baseline]
            ]
        ]
      \end{forest}
      & \phantom{x} &
      \begin{forest} boom
        [
            [XP,
            tikz={
            \node[draw,circle,
            dashed,
            scale=0.8,
            fit to=tree]{};
            }
                [\phantom{xxx}, roof, baseline]
            ]
            [\tsc{acc}P
                [\tsc{f}2]
                [\tsc{nom}P
                    [\tsc{f}1]
                ]
            ]
        ]
      \end{forest}\\
      \bottomrule
  \end{tabular}
   \caption {\tsc{nom} vs. modified \tsc{acc} ≠ \tsc{acc}}
  \label{fig:acc-nom-structure-moved-out}
\end{figure}

I use constituent containment to explain why deletion of the light head or relative pronoun is sometimes possible and sometimes not. In Section \ref{sec:deriving-matching} I show that only this stronger constituent containment requirement (and not the weaker feature containment requirement) is able to distinguish the internal-only from the matching type.

I apply the constituent containment reasoning to comparing the light head and relative pronoun. This time the most complex one is not the winner of the case competition. Instead, the most complex one is the element that can delete the other element. The relative pronoun can delete the light head, when the relative pronoun contains all constituents of the light head. The light head can delete the relative pronoun, when the light head contains all constituents of the relative pronoun.

In order to be able to compare the light head and the relative pronoun, I zoom in on their syntactic structure. In Section \ref{sec:deriving-only-internal} to \ref{sec:deriving-nonmatching} I give arguments to support the structures I am assuming here. Figure \ref{fig:rel-lh-structure} gives a simplified representation of them.\footnote{
The structures in Figure \ref{fig:rel-lh-structure} are not base structures but derived ones. I assume the base structure to be as in Figure \ref{fig:rel-lh-structure-base}.

\begin{figure}[H]
  \footnotesize{
  \center
  \begin{tabular}[b]{ccc}
      \toprule
      light head & & relative pronoun  \\
      \cmidrule(lr){1-1} \cmidrule(lr){3-3}
      \begin{forest} boom
        [\tsc{k}P,
            [\tsc{k}]
            [ϕP
                [\phantom{x}ϕ\phantom{x}, roof, baseline]
            ]
        ]
      \end{forest}\\
      & \phantom{x} &
      \begin{forest} boom
        [\tsc{rel}P
            [\tsc{rel}]
            [\tsc{k}P
                [\tsc{k}]
                [ϕP
                    [\phantom{x}ϕ\phantom{x}, roof, baseline]
                ]
            ]
        ]
      \end{forest}\\
      \bottomrule
  \end{tabular}
}
\caption{\footnotesize{Light head and relative pronoun (base structure)}}
\label{fig:rel-lh-structure-base}
\end{figure}

In Section \ref{sec:deriving-only-internal} I show how I reach the derived structure. I work with the derived structure in the main text because this is the configuration in which the containment relations under discussion hold.
}
The light head and the relative pronoun partly contain the same syntactic features. The features they have in common are case (\tsc{k}) and what I here call phi-features (ϕ). The light head and the relative pronoun differ from each other in that the relative pronoun in addition has a relative feature (\tsc{rel}).

\begin{figure}[H]
  \center
  \begin{tabular}[b]{ccc}
      \toprule
      light head & & relative pronoun \\
      \cmidrule(lr){1-1} \cmidrule(lr){3-3}
      \begin{forest} boom
      [\tsc{k}P
          [ϕP
              [\phantom{x}ϕ\phantom{x}, roof, baseline]
          ]
          [\tsc{k}P,
              [\tsc{k}]
          ]
      ]
      \end{forest}
      & \phantom{x} &
    \begin{forest} boom
      [\tsc{rel}P
          [\tsc{rel}]
          [\tsc{k}P
              [ϕP
                  [\phantom{x}ϕ\phantom{x}, roof, baseline]
              ]
              [\tsc{k}P,
                  [\tsc{k}]
              ]
          ]
      ]
    \end{forest}\\
      \bottomrule
  \end{tabular}
   \caption {Light head and relative pronoun}
  \label{fig:rel-lh-structure}
\end{figure}

I compare the light head and the relative pronoun in terms of constituent containment. The relative pronoun can delete the light head because the relative pronoun contains all constituents the light head contains.
I illustrate this in Figure \ref{fig:rel-lh-structure-containment}. I draw a dashed circle around each constituent that is a constituent in both the light head and the relative pronoun.
The \tsc{rel}P contains the \tsc{k}P, so the relative pronoun can delete the light head. I illustrate this by marking the content of the dashed circle for the \tsc{k}P gray.

\begin{figure}[H]
  \center
  \begin{tabular}[b]{ccc}
      \toprule
      light head & & relative pronoun \\
      \cmidrule(lr){1-1} \cmidrule(lr){3-3}
      \begin{forest} boom
        [\tsc{k}P,
        tikz={
        \node[draw,circle,
        dashed,
        scale=0.9,
        fit to=tree]{};
        }
            [ϕP
                [\phantom{x}ϕ\phantom{x}, roof, baseline]
            ]
            [\tsc{k}P,
                [\tsc{k}]
            ]
        ]
      \end{forest}
      & \phantom{x} &
      \begin{forest} boom
        [\tsc{rel}P
            [\tsc{rel}]
            [\tsc{k}P,
            tikz={
            \node[draw,circle,
            dashed,
            scale=0.9,
            fit to=tree]{};
            }
                [ϕP
                    [\phantom{x}ϕ\phantom{x}, roof, baseline]
                ]
                [\tsc{k}P
                    [\tsc{k}]
                ]
            ]
        ]
      \end{forest}\\
      \bottomrule
  \end{tabular}
   \caption {Light head and relative pronoun}
  \label{fig:rel-lh-structure-containment}
\end{figure}

Not all constituents of the relative pronoun are contained in the light head, so the light head cannot delete the relative pronoun.
The language type that I describe here is the internal-only type, the one that Modern German shows. However, not all language are of the internal-only type.
I assume that the structures in Figure \ref{fig:rel-lh-structure} hold for all languages types, but that they differ in how they are spelled out, which causes the languages to be of different types. Before I come back to how the other language types deviate, I show how the internal-only type fares with different internal and external cases.

I start with an example with matching cases in Figure \ref{fig:rel-lh-nom-structure}. The relative pronoun appears in the nominative, and the light head does too.
I draw a dashed circle around each constituent that is a constituent in both the light head and the relative pronoun.
Consider the light head. The constituent \tsc{nom}P is also a constituent in the relative pronoun, contained in the \tsc{rel}P.
As the constituent of the light head is also a constituent within the relative pronoun, the light head can be absent. I illustrate this by marking the content of the dashed circles for the light head gray.

\begin{figure}[H]
  \center
  \begin{tabular}[b]{ccc}
      \toprule
      light head & & relative pronoun \\
      \cmidrule(lr){1-1} \cmidrule(lr){3-3}
      \begin{forest} boom
        [\tsc{nom}P,
        tikz={
        \node[draw,circle,
        dashed,
        scale=0.9,
        fill=DG,fill opacity=0.2,
        fit to=tree]{};
        }
            [ϕP
                [\phantom{x}ϕ\phantom{x}, roof, baseline]
            ]
            [\tsc{nom}P,
                [\tsc{f}1]
            ]
        ]
      \end{forest}
      & \phantom{x} &
      \begin{forest} boom
        [\tsc{rel}P
            [\tsc{rel}]
            [\tsc{nom}P,
            tikz={
            \node[draw,circle,
            dashed,
            scale=0.9,
            fit to=tree]{};
            }
                [ϕP
                    [\phantom{x}ϕ\phantom{x}, roof, baseline]
                ]
                [\tsc{nom}P,
                    [\tsc{f}1]
                ]
            ]
        ]
      \end{forest}\\
      \bottomrule
  \end{tabular}
   \caption {\tsc{nom} relative pronoun and \tsc{nom} light head}
  \label{fig:rel-lh-nom-structure}
\end{figure}

I continue with the example in Figure \ref{fig:rel-acc-lh-nom-structure}, in which the relative pronoun bears a more complex case than the light head.
I draw a dashed circle around each constituent that is a constituent in both the light head and the relative pronoun. Different from the example in Figure \ref{fig:rel-lh-nom-structure}, there are now two separate constituents.
I start with the right-most constituent of the light head: \tsc{nom}P. This constituent is also a constituent in the relative pronoun, contained in the lower \tsc{acc}P.
I continue with the left-most constituent of the light head: the ϕP. This constituent is also a constituent in the relative pronoun, contained in the higher \tsc{acc}P.
As each constituent of the light head is also a constituent within the relative pronoun, the light head can be absent. I illustrate this by marking the content of the dashed circles for the light head gray.

\begin{figure}[H]
  \center
  \begin{tabular}[b]{ccc}
      \toprule
      light head & & relative pronoun \\
      \cmidrule(lr){1-1} \cmidrule(lr){3-3}
      \begin{forest} boom
        [\tsc{nom}P, s sep=15mm
            [ϕP,
            tikz={
            \node[draw,circle,
            dashed,
            scale=0.8,
            fill=DG,fill opacity=0.2,
            fit to=tree]{};
            }
                [\phantom{x}ϕ\phantom{x}, roof, baseline]
            ]
            [\tsc{nom}P,
            tikz={
            \node[draw,circle,
            dashed,
            scale=0.8,
            fill=DG,fill opacity=0.2,
            fit to=tree]{};
            }
                [\tsc{f}1]
            ]
        ]
      \end{forest}
      & \phantom{x} &
      \begin{forest} boom
        [\tsc{rel}P
            [\tsc{rel}]
            [\tsc{acc}P
                [ϕP,
                tikz={
                \node[draw,circle,
                dashed,
                scale=0.8,
                fit to=tree]{};
                }
                    [\phantom{x}ϕ\phantom{x}, roof, baseline]
                ]
                [\tsc{acc}P
                    [\tsc{f}2]
                    [\tsc{nom}P,
                    tikz={
                    \node[draw,circle,
                    dashed,
                    scale=0.8,
                    fit to=tree]{};
                    }
                        [\tsc{f}1]
                    ]
                ]
            ]
        ]
      \end{forest}\\
      \bottomrule
  \end{tabular}
   \caption {\tsc{nom} light head and \tsc{acc} relative pronoun}
  \label{fig:rel-acc-lh-nom-structure}
\end{figure}

I end with the example in Figure \ref{fig:rel-nom-lh-acc-structure}, in which the light head bears a more complex case than the relative pronoun.
I draw a dashed circle around each constituent that is a constituent in both the light head and the relative pronoun. Different from the previous two examples in Figure \ref{fig:rel-lh-nom-structure} and Figure \ref{fig:rel-acc-lh-nom-structure}, neither of the elements contains all constituents of the other element. The relative pronoun does not contain all constituents that the light head contains, and the light head does not contain all constituents that the relative pronoun contains. As a result, none of the elements can be absent.

I start by showing that the light head cannot be absent.
Consider the right-most constituent of the light head: \tsc{acc}P. This constituent is not a constituent in the relative pronoun: the relative pronoun has a constituent \tsc{nom}P, but it does not contain \tsc{f}2 to make it an \tsc{acc}P.
The light head has a constituent that is not a constituent in the relative pronoun, so the light head cannot be absent.

The relative pronoun can also not be absent.
Consider the left-most constituent of the relative pronoun: \tsc{rel}P. This constituent is not a constituent in the light head: the light head lacks the features the \tsc{rel}P.
The relative pronoun has a constituent that is not a constituent in the light head, so the relative pronoun cannot be absent.
In sum, neither of the elements contains all constituents that the other element contains, so none of the elements can be absent, and none of them is marked gray.

\begin{figure}[H]
  \center
  \begin{tabular}[b]{ccc}
      \toprule
      light head & & relative pronoun \\
      \cmidrule(lr){1-1} \cmidrule(lr){3-3}
      \begin{forest} boom
        [\tsc{acc}P
            [ϕP,
            tikz={
            \node[draw,circle,
            dashed,
            scale=0.8,
            fit to=tree]{};
            }
                [\phantom{x}ϕ\phantom{x}, roof, baseline]
            ]
            [\tsc{acc}P
                [\tsc{f}2]
                [\tsc{nom}P,
                tikz={
                \node[draw,circle,
                dashed,
                scale=0.8,
                fit to=tree]{};
                }
                    [\tsc{f}1]
                ]
            ]
        ]
      \end{forest}
      & \phantom{x} &
      \begin{forest} boom
        [\tsc{rel}P
            [\tsc{rel}]
            [\tsc{nom}P, s sep=15mm
                [ϕP,
                tikz={
                \node[draw,circle,
                dashed,
                scale=0.8,
                fit to=tree]{};
                }
                    [\phantom{x}ϕ\phantom{x}, roof, baseline]
                ]
                [\tsc{nom}P,
                tikz={
                \node[draw,circle,
                dashed,
                scale=0.8,
                fit to=tree]{};
                }
                    [\tsc{f}1]
                ]
            ]
        ]
      \end{forest}\\
      \bottomrule
  \end{tabular}
   \caption { \tsc{nom} relative pronoun and \tsc{acc} light head}
  \label{fig:rel-nom-lh-acc-structure}
\end{figure}

I return to the point of crosslinguistic differences.
I do not derive the difference between the languages from changing the feature content of the light head and relative pronoun per language.\footnote{
The feature content of the non-matching languages differs slightly from that of the internal-only and matching languages. This is due to the fact that this language type uses a different type of relative pronoun. The basic idea of the relative pronoun having at least one more feature than the light head remains the same.
}
Instead, the difference comes from how the light heads and relative pronouns are spelled out. In Sections \ref{sec:deriving-only-internal} to Section \ref{sec:deriving-nonmatching}, I show how I implement this idea. For matching languages like Polish the intuition is that they package their features together differently in such a way that the constituents of the relative pronoun do not contain the constituents of the light head.

Non-matching languages like Old High German have a syncretism between light heads and relative pronouns. It seems that syncretism can fix that. I draw a parallel with case syncretism in headless relatives, but also syncretism outside of case and headless relatives.

The structures in Figures \ref{fig:rel-lh-nom-structure} to \ref{fig:rel-nom-lh-acc-structure} exclude the external-only pattern. There is no way that each constituent of the light head contains a constituent of the relative pronoun but not the other way around.

I return to the metaphor with the committee that I introduced in Section \ref{sec:possible-patterns}. I wrote that first case competition takes place, in which a more complex case wins over a less complex case. This case competition can now be reformulated into a more general mechanism, namely constituent comparison. A more complex case corresponds to a constituent that contains the constituent that corresponds to a less complex case.
Subsequently, I noted that there is a committee it can either approve this case or not approve it. The approval happens based on where the winning case comes from: from inside to the relative clause (internal) or from outside to the relative clause (external). The information that the committee uses for the approval of the case relies on the same mechanism as case competition, namely constituent comparison. A relative pronoun corresponds to a constituent that contains the constituent that corresponds to a light head.
In other words, the grammaticality of a headless relative depends on several instances of constituent comparison. The constituents that are compared are those of the light head and the relative pronoun, which both elements bear their own case. Case is special in that it can differ from sentence to sentence within a language. Therefore, its effect can be observed within a particular language. The part of the light head and relative pronoun that does not involve case features is stable within a language.

In this dissertation I describe different language types in case competition in headless relatives. In my account, the different language types are a result of a comparison of the light head and the relative pronoun in the language.
The larger syntactic context in which this takes place should be kept stable. The operation that deletes the light head or the relative pronoun is the same for all language types. In this work, I do not specify on which larger syntactic structure and which deletion operation should be used. In Section \ref{sec:larger-syntax} I discuss existing proposals on these topics and to what extend they are compatible with my account.

To summarize, in this section I introduced the assumptions that headless relatives are derived from light-headed relatives and that relative pronouns contain at least one more feature than light heads. A headless relative is grammatical when either the light head or the relative pronoun contains all features of the other element. This set of assumptions derives that only a more complex case can surface and that there is no language of the external-only type.

% (\citealt{fuss2014,hanink2018} argue the same but for Modern German).\footnote{
% A difference with Modern German is that there can only be a single element when the cases match. In Section \ref{ch:discussion} I return to the point why Modern German does not have non-matching headless relatives that look like Old High German, although it still has syncretic light heads and relative pronouns.
% }).
%
% \footnote{
% I am far from the only one that assumes this. Himmelreich, Hanink, but also Bresnan/Grimshaw, Groos/Riemsdijk, Harbert..
% }

%at first sight this seems very much related to what Hanink proposes for Modern German. Something is non-pronounced if it contains the features. A crucial difference here is that she formulates it in terms of context sensitive rules, but she does not motivate where these rules come from. I do not have language-specific rules.


\section{Deriving the internal-only type}\label{sec:deriving-only-internal}

First I discuss the relative pronoun, then I discuss the light head, and then I compare the two. I show that this is the pattern that arises.

\begin{table}[H]
  \center
  \caption{Modern German cases}
\begin{tabular}{cc}
  \toprule
\tsc{int} = \tsc{ext} & ✔  \\
\tsc{int} > \tsc{ext} & ✔  \\
\tsc{int} < \tsc{ext} & *  \\
  \bottomrule
\end{tabular}
\label{tbl:overview-mg}
\end{table}

I show that the surface pronoun in Modern German headless relatives is the relative pronoun.
The evidence comes from extraposition data. In Modern German, it is possible to extrapose a CP (a clause), but not a DP (a noun phrase). In this section I first show that Modern German CPs can be extraposed and DPs cannot. Then I illustrate how relative clauses including the relative pronoun in headless relatives pattern with CPs: they can be extraposed as well.

The sentences in \ref{ex:mg-extrapose-cp} show that it is possible to extrapose a CP. In \ref{ex:mg-extrapose-cp-base}, the clausal object \tit{wie es dir geht} `how you are doing', marked here in bold, appears in its base position. It can be extraposed to the right edge of the clause, shown in \ref{ex:mg-extrapose-cp-moved}.

\ex.\label{ex:mg-extrapose-cp}
\ag. Mir ist \tbf{wie} \tbf{es} \tbf{dir} \tbf{geht} egal.\\
1\tsc{sg}.\tsc{dat} is how it 2\tsc{sg}.\tsc{dat} goes {the same}\\
`I don't care how you are doing.'\label{ex:mg-extrapose-cp-base}
\bg. Mir is egal \tbf{wie} \tbf{es} \tbf{dir} \tbf{geht}.\\
1\tsc{sg}.\tsc{dat} is {the same} how it 2\tsc{sg}.\tsc{dat} goes\\
`I don't care how you are doing.' \label{ex:mg-extrapose-cp-moved}\flushfill{Modern German}

\ref{ex:mg-extrapose-dp} illustrates that it is impossible to extrapose a DP. The clausal object of \ref{ex:mg-extrapose-cp} is replaced by the simplex noun phrase \tit{die Sache} `that matter'.
In \ref{ex:mg-extrapose-dp-base} the object, marked in bold, appears in its base position. In \ref{ex:mg-extrapose-dp-moved} it is extraposed, and the sentence is no longer grammatical.

\ex.\label{ex:mg-extrapose-dp}
\ag. Mir ist \tbf{die} \tbf{Sache} egal.\\
1\tsc{sg}.\tsc{dat} is that matter {the same}\\
`I don't care about that matter.'\label{ex:mg-extrapose-dp-base}
\bg. *Mir ist egal \tbf{die} \tbf{Sache}.\\
1\tsc{sg}.\tsc{dat} is {the same} that matter\\
`I don't care about that matter.' \label{ex:mg-extrapose-dp-moved}\flushfill{Modern German}

The same asymmetry between CPs and DPs can be observed with relative clauses. A relative clause is a CP, and the head of a relative clause is a DP. The sentences in \ref{ex:extra-headed} contain the relative clause \tit{was er gekocht hat} `what he has stolen'. This is marked in bold in the examples. The (light) head of the relative clause is \tit{das}.
In \ref{ex:extra-headed-base}, the relative clause and its head appear in base position. In \ref{ex:extra-headed-only-clause}, the relative clause is extraposed. This is grammatical, because it is possible to extrapose CPs in Modern German. In \ref{ex:extra-headed-head-clause}, the relative clause and the head are extraposed. This is ungrammatical, because it is possible to extrapose DPs.

\ex.\label{ex:extra-headed}
\ag. Jan hat das, \tbf{was} \tbf{er} \tbf{gekocht} \tbf{hat}, aufgegessen.\\
Jan has that what he cooked has eaten\\
`Jan has eaten what he cooked.'\label{ex:extra-headed-base}
\bg. Jan hat das aufgegessen, \tbf{was} \tbf{er} \tbf{gekocht} \tbf{hat}.\\
Jan has that eaten what he cooked has\\
`Jan has eaten what he cooked.'\label{ex:extra-headed-only-clause}
\cg. *Jan hat aufgegessen, das, \tbf{was} \tbf{er} \tbf{gekocht} \tbf{hat}.\\
Jan has eaten that what he cooked has\\
`Jan has eaten what he cooked.'\label{ex:extra-headed-head-clause} \flushfill{Modern German}

The same can be observed in relative clauses without a head. \ref{ex:extra-headless} is the same sentence as in \ref{ex:extra-headed} only without the overt head. The relative clause is marked in bold again.
In \ref{ex:extra-headless-base}, the relative clause appears in base position. In \ref{ex:extra-headless-clause}, the relative clause is extraposed. This is grammatical, because it is possible to extrapose CPs in Modern German. In \ref{ex:extra-headless-no-rel}, the relative clause is extraposed without the relative pronouns. This is ungrammatical, because the relative pronoun is part of the CP.
This shows that the relative pronoun in headless relatives in Modern German are necessarily part of a CP, which is here a relative clause.

\ex.\label{ex:extra-headless}
\ag. Jan hat \tbf{was} \tbf{er} \tbf{gekocht} \tbf{hat} aufgegessen.\\
Jan has what he cooked has eaten\\
`Jan has eaten what he cooked.'\label{ex:extra-headless-base}
\bg. Jan hat aufgegessen \tbf{was} \tbf{er} \tbf{gekocht} \tbf{hat}.\\
Jan has eaten what he cooked has\\
`Jan has eaten what he cooked.'\label{ex:extra-headless-clause}
\bg. *Jan hat \tbf{was} aufgegessen \tbf{er} \tbf{gekocht} \tbf{hat}.\\
Jan has what eaten he cooked has\\
`Jan has eaten what he cooked.'\label{ex:extra-headless-no-rel}\flushfill{Modern German}

In conclusion, extraposition facts show that the surface pronoun in Modern German is the relative pronoun.


\begin{table}[H]
 \center
 \caption {Modern German relative pronouns}
  \begin{tabular}{ccc}
  \toprule
              & \ac{an}  & \tsc{inan}\\
    \cmidrule{2-3}
    \ac{nom}  & w-er    & w-as     \\
    \ac{acc}  & w-en    & w-as     \\
    \ac{dat}  & w-em    & (w-em)   \\
  \bottomrule
  \end{tabular}
\end{table}

\begin{table}[H]
 \center
 \caption {Modern German demonstrative pronouns}
  \begin{tabular}{cccc}
  \toprule
              & \ac{m}  & \tsc{n} & \tsc{f} \\
    \cmidrule{2-4}
    \ac{nom}  & d-er   & d-as   & d-ie    \\
    \ac{acc}  & d-en   & d-as   & d-ie    \\
    \ac{dat}  & d-em   & d-em   & d-er    \\
  \bottomrule
  \end{tabular}
\end{table}

Interesting that there is no feminine headless relative pronoun. This happens more often btw.

The wh is used as interrogative and as relative pronoun.

\exg. Wer ist da?\\
who is there\\
`Who is there?'

Hachem describes it as set of alternatives etc. etc. Lander and Baunaz have this containment between \tsc{rel}, \tsc{wh}, \tsc{dem} etc. I adopt the \tsc{rel}/\tsc{wh} part.

\ex. \begin{forest} boom
  [\tsc{rel}P
      [\tsc{rel}]
      [\tsc{wh}]
  ]
  {\draw (.east) node[right]{⇔ \tit{w}}; }
\end{forest}

This leaves the suffix. I split this up in two parts. I see the final consonant in different places.

\begin{table}[H]
 \center
 \caption {Modern German relative pronouns}
  \begin{tabular}{ccc}
  \toprule
              & \ac{an}  & \tsc{inan}\\
    \cmidrule{2-3}
    \ac{nom}  & w-e-r    & w-a-s     \\
    \ac{acc}  & w-e-n    & w-a-s     \\
    \ac{dat}  & w-e-m    & (w-e-m)   \\
  \bottomrule
  \end{tabular}
\end{table}

In strong adjectives.

\begin{table}[H]
 \center
 \caption {Modern German strong adjectives}
  \begin{tabular}{cccc}
  \toprule
              & \ac{m}    & \tsc{n}   & \tsc{f}  \\
    \cmidrule{2-4}
    \ac{nom}  & neu-ə-r   & neu-ə-s   & neu-ə    \\
    \ac{acc}  & neu-ə-n   & neu-ə-s   & neu-ə    \\
    \ac{dat}  & neu-ə-m   & neu-ə-m   & neu-ə-r  \\
  \bottomrule
  \end{tabular}
\end{table}

In pronouns, in the non-suppletives.

\begin{table}[H]
 \center
 \caption {Modern German personal pronouns}
  \begin{tabular}{cccc}
  \toprule
              & \ac{m} & \tsc{n} & \tsc{f}  \\
    \cmidrule{2-4}
    \ac{nom}  & er     & es      & sie      \\
    \ac{acc}  & ih-n   & es      & sie      \\
    \ac{dat}  & ih-m   & ih-m    & ih-r     \\
  \bottomrule
  \end{tabular}
\end{table}

Zooming in on \tit{r} and \tit{n}, one knows for sure that the consonants express case. In a second I show how they interact with gender.

I use the case features introduced by \citet{caha2009}, which I already discussed in Chapter \ref{ch:decomposition}. \tsc{f}1 refers to a nominative, and \tsc{f}1 and \tsc{f}2 refers to an accusative.

\ex. \begin{forest} boom
  [\tsc{acc}P
      [\tsc{f}2]
      [\tsc{nom}P
          [\tsc{f}1]
      ]
  ]
  {\draw (.east) node[right]{⇔ \tit{n}}; }
\end{forest}

\ex. \begin{forest} boom
  [\tsc{nom}P
      [\tsc{f}1]
  ]
  {\draw (.east) node[right]{⇔ \tit{r}}; }
\end{forest}

For now, this leaves the vowel to express features having to do with referentiality, gender, number and deixis.

I use pronominal features that are distinguished by \citet{harley2002}: \tsc{class}, \tsc{masc} and \tsc{ind}. \tsc{ref} refers to a referring expression, which all pronouns contain. The feature \tsc{class} refers to gender features, which is neuter if it is not combined with any other features. Combining \tsc{class} with the feature \tsc{masc} gives a masculine gender. \tsc{ind} refer to number, which is singular if it is not combined with any other features.

In addition, I use the Lander/Haegeman features for deixis. In \tsc{wh}-elements the distal is used. This can be shown in English.

\ex.
\ag. th-is\\
 \tsc{dem}-\tsc{prox}\\
\bg. th-at\\
 \tsc{dem}-\tsc{dist}\\
\bg. wh-at\\
 \tsc{wh}-\tsc{prox}\\

\footnote{
Conceptually, this can be made sense of if you see distal as something far away from you as a speaker, because it is unknown to you. Something that is known to you is expressed with a proximal.

\ex.
\a. Yesterday I talked to this woman, and she told me all I needed to know.
\b. Please tell me about that thing later.

cite Wiltschko.
}

I give the lexical entries for the morphemes.

\ex. \begin{forest} boom
[\tsc{ind}P
    [\tsc{ind}]
    [\tsc{masc}P
        [\tsc{masc}]
        [\tsc{class}P
            [\tsc{class}]
            [\tsc{dist}P
                [\tsc{deix}\scsub{3}]
                [\tsc{med}P
                    [\tsc{deix}\scsub{2}]
                    [\tsc{prox}P
                        [\tsc{deix}\scsub{1}]
                        [\tsc{ref} [\phantom{xxx}, roof]]
                    ]
                ]
            ]
        ]
    ]
]
  {\draw (.east) node[right]{⇔ \tit{e}}; }
\end{forest}

Now I show how the relative pronoun are built.

\ex.
\scriptsize{
\begin{forest} boompje
  [\tsc{rel}P, s sep=15mm
      [\tsc{rel}P,
      tikz={
      \node[label=below:\tit{w},
      draw,circle,
      scale=0.9,
      fit to=tree]{};
      }
          [\tsc{rel}]
          [\tsc{wh}]
      ]
      [\tsc{acc}P, s sep=50mm
          [\tsc{ind}P,
          tikz={
          \node[label=below:\tit{e},
          draw,circle,
          scale=0.95,
          fit to=tree]{};
          }
              [\tsc{ind}]
              [\tsc{masc}P
                  [\tsc{masc}]
                  [\tsc{class}P
                      [\tsc{class}]
                      [\tsc{dist}P
                          [\tsc{deix}\scsub{3}]
                          [\tsc{med}P
                              [\tsc{deix}\scsub{2}]
                              [\tsc{prox}P
                                  [\tsc{deix}\scsub{1}]
                                  [\tsc{ref} [\phantom{xxx}, roof]]
                              ]
                          ]
                      ]
                  ]
              ]
          ]
          [\tsc{acc}P,
          tikz={
          \node[label=below:\tit{n},
          draw,circle,
          scale=0.85,
          fit to=tree]{};
          }
              [\tsc{f}2]
              [\tsc{nom}P
                  [\tsc{f}1]
              ]
          ]
      ]
  ]
\end{forest}
}


\ex.
\scriptsize{
\begin{forest} boompje
  [\tsc{rel}P, s sep=15mm
      [\tsc{rel}P,
      tikz={
      \node[label=below:\tit{w},
      draw,circle,
      scale=0.9,
      fit to=tree]{};
      }
          [\tsc{rel}]
          [\tsc{wh}]
      ]
      [\tsc{nom}P, s sep=50mm
          [\tsc{ind}P,
          tikz={
          \node[label=below:\tit{e},
          draw,circle,
          scale=0.95,
          fit to=tree]{};
          }
              [\tsc{ind}]
              [\tsc{masc}P
                  [\tsc{masc}]
                  [\tsc{class}P
                      [\tsc{class}]
                      [\tsc{dist}P
                          [\tsc{deix}\scsub{3}]
                          [\tsc{med}P
                              [\tsc{deix}\scsub{2}]
                              [\tsc{prox}P
                                [\tsc{deix}\scsub{1}]
                                [\tsc{ref} [\phantom{xxx}, roof]
                                ]
                            ]
                        ]
                    ]
                ]
            ]
        ]
          [\tsc{nom}P,
          tikz={
          \node[label=below:\tit{r},
          draw,circle,
          scale=0.85,
          fit to=tree]{};
          }
              [\tsc{f}1]
          ]
      ]
  ]
\end{forest}
}

But how are these relative pronouns formed, how are the features packaged together the way they are? The basis of these relative pronouns are a functional sequence, and there is a spellout algorithm that drives the moving around of elements.

The functional sequence is this:

\ex. \begin{forest} boom
[\tsc{acc}P
    [\tsc{f}2]
    [\tsc{nom}P
        [\tsc{f}1]
        [\tsc{rel}P
            [\tsc{rel}]
            [\tsc{wh}P
                [\tsc{wh}]
                [\tsc{ind}P
                    [\tsc{ind}]
                    [\tsc{masc}P
                        [\tsc{masc}]
                        [\tsc{class}P
                            [\tsc{class}]
                            [\tsc{dist}P
                                [\tsc{deix}\scsub{3}]
                                [\tsc{med}P
                                    [\tsc{deix}\scsub{2}]
                                    [\tsc{prox}P
                                        [\tsc{deix}\scsub{1}]
                                        [\tsc{ref} [\phantom{xxx}, roof]]
                                    ]
                                ]
                            ]
                        ]
                    ]
                ]
            ]
        ]
    ]
]
\end{forest}

The spellout algorithm is this.

\ex. \tbf{Spellout Algorithm:}\\
Merge F and \label{ex:spellout}
 \a. Spell out FP.
 \b. If (a) fails, attempt movement of the spec of the complement of \tsc{f}, and retry (a).
 \b. If (b) fails, move the complement of \tsc{f}, and retry (a).

When a new match is found, it overrides previous spellouts.

\ex. \tbf{Cyclic Override} \citep{starke2018}:\\
Lexicalisation at a node XP overrides any previous match at a phrase contained in XP.

If the spellout procedure in \ref{ex:spellout} fails, backtracking takes place.

\ex. \tbf{Backtracking} \citep{starke2018}:\\
When spellout fails, go back to the previous cycle, and try the next option for that cycle.\label{ex:backtracking}

\ex. \tbf{Spec Formation} \citep{starke2018}:\\
If Merge F has failed to spell out (even after backtracking), try to spawn a new derivation providing the feature F and merge that with the current derivation, projecting the feature F at the top node.\label{ex:specformation}

For the full derivation, I refer the reader to Section X.

I start at the beginning, with the \tsc{ref}, merging it with \tsc{deix}\scsub{1}, giving a \tit{e}.

So if then \tsc{deix}\scsub{2} is merged, it is overwritten by \tit{e}.

I move forward a bit to when \tsc{wh} is merged. First the spellout driven movement happen, but this does not bring anything. Also backtracking does not help, so we build a spec.

Feature \tsc{rel} is merged. First try to merge it on the whole tree, then the spellout driven movements, nothing works. So, backtracking. The first step of backtracking is that the two trees are split, and the feature is merged on both parts. If the feature is spelled out on one of them, we are done. It can be phrasally spelled out with \tsc{wh}, so we move on.

Then feature \tsc{f}1 is merged. Whole tree, spellout driven movement: yes! it is spelled out as a suffix on the whole thing.









Now I come back to the interaction between gender and case. For the neuter relative pronoun the vowel is different: \tit{a}. So, this needs to be reflected in the features: I left out \tsc{masc}. However, there also needs to be a difference in lexical entries between the final consonant for the neuter and for the masculine. Therefore, I let the \tit{s} spell out the \tsc{ind} feature.

\ex. \begin{forest} boom
  [\tsc{class}P
      [\tsc{class}]
      [\tsc{dist}P
          [\tsc{deix}\scsub{3}]
          [\tsc{med}P
              [\tsc{deix}\scsub{2}]
              [\tsc{prox}P
                  [\tsc{deix}\scsub{1}]
                  [\tsc{ref} [\phantom{xxx}, roof]]
              ]
          ]
      ]
  ]
{\draw (.east) node[right]{⇔ \tit{a}}; }
\end{forest}

\ex. \begin{forest} boom
  [\tsc{nom}P
      [\tsc{f}1]
      [\tsc{ind}P
          [\tsc{ind}]
      ]
  ]
  {\draw (.east) node[right]{⇔ \tit{a}}; }
\end{forest}






Modern German has two types of demonstratives: the strong one and the weak one.

\ex. Ich habe n gesehen.

\ex. Ich habe m geholfen.

\ex. Hat r nen Motorrad?

The strong article is used when there is an anaphoric relation. Often there is a linguistic antecedent that is referred back to.

\exg. Hans hat heute \tbf{einen} \tbf{Freund} zum Essen mit nach Hause gebracht. Er hat uns vorher ein Foto \tbf{vom}/ \tbf{von} \tbf{dem} \tbf{Freund} gezeigt.\\
Hans has today a friend {to the} dinner with to home brought he has us beforehand a photo {of the\scsub{weak}} of the\scsub{strong} friend shown\\
`Hans brought a friend home for dinner today. He had shown us a photo of the friend beforehand.'

Weak articles are used when situational uniqueness is involved. Uniqueness can be global or within a restricted domain. The discourse participants mutually shared knowledge that uniqueness holds.

\ex.
\ag. Der Einbrecher ist {zum Glück} vom /von dem Hund verjagt worden.\\
the burglar is luckily {by the\scsub{weak}} by the\scsub{strong} dog {chased away} been\\
`Luckily, the burglar was chased away by the dog.'
\bg. Armstrong flog als erster zum Mond.\\
Armstrong flew as {first one} {to the\scsub{weak}} moon\\
`Armstrong was the first one to fly to the moon.' \flushfill{Modern German, \pgcitealt{schwarz2009}{40}}

In the headless relatives, there is uniqueness. Show?

The strong article cannot be used because it does not go together with the free choice interpretation of \tsc{wh}-relatives (say something about Hanink).

The deletion in Modern German is not optional, but obligatory. The reason for that is that the weak demonstrative is phonologically(?) not heavy enough to be the head of a relative clause. Maybe not only phonologically, because \tit{vom} also does not work..

are free relatives restrictive or non-restrictive? > restrictive, and restrictive and weak are incompatible :)  >> this is why we have deletion!

but why not in Polish? It then cannot be because of some feature.. or ə cannot realize a focus feature while the Polish counterpart can..

\ex. *Sie ist vom Mann, mit dem sie gestern ausgegangen ist, versetzt worden.


Lexical entry for the light head

schwa r is really the proximal and medial.
dieser and der
so the promximal and medial just never appears without the emphasizer



\ex. \begin{forest} boom
  [\tsc{med}P
      [\tsc{deix}\scsub{2}]
      [\tsc{prox}P
          [\tsc{deix}\scsub{1}]
          [\tsc{ind}P
              [\tsc{ind}]
              [\tsc{masc}P
                  [\tsc{masc}]
                  [\tsc{class}P
                      [\tsc{class}]
                      [\tsc{ref} [\phantom{xxx}, roof]]
                  ]
              ]
          ]
      ]
  ]
  {\draw (.east) node[right]{⇔ \tit{ə}}; }
\end{forest}

I show how the light heads are built.

\ex.
\scriptsize{
\begin{forest} boompje
  [\tsc{acc}P, s sep=40mm
      [\tsc{med}P,
      tikz={
      \node[label=below:\tit{ə},
      draw,circle,
      scale=0.95,
      fit to=tree]{};
      }
          [\tsc{deix}\scsub{2}]
          [\tsc{prox}P
              [\tsc{deix}\scsub{1}]
              [\tsc{ind}P
                  [\tsc{ind}]
                  [\tsc{masc}P
                      [\tsc{masc}]
                      [\tsc{class}P
                          [\tsc{class}]
                          [\tsc{ref} [\phantom{xxx}, roof]]
                      ]
                  ]
              ]
          ]
      ]
      [\tsc{acc}P,
      tikz={
      \node[label=below:\tit{n},
      draw,circle,
      scale=0.85,
      fit to=tree]{};
      }
          [\tsc{f}2]
          [\tsc{nom}P
              [\tsc{f}1]
          ]
      ]
  ]
\end{forest}
}

\ex.
\scriptsize{
\begin{forest} boompje
  [\tsc{nom}P, s sep=40mm
      [\tsc{med}P,
      tikz={
      \node[label=below:\tit{ə},
      draw,circle,
      scale=0.95,
      fit to=tree]{};
      }
          [\tsc{deix}\scsub{2}]
          [\tsc{prox}P
              [\tsc{deix}\scsub{1}]
              [\tsc{ind}P
                  [\tsc{ind}]
                  [\tsc{masc}P
                      [\tsc{masc}]
                      [\tsc{class}P
                          [\tsc{class}]
                          [\tsc{ref} [\phantom{xxx}, roof]]
                      ]
                  ]
              ]
          ]
      ]
      [\tsc{nom}P,
      tikz={
      \node[label=below:\tit{r},
      draw,circle,
      scale=0.85,
      fit to=tree]{};
      }
          [\tsc{f}1]
      ]
  ]
\end{forest}
}

Consider the example in \ref{ex:mg-nom-nom-rep}, in which the internal nominative case competes against the external nominative case. The relative clause is marked in bold, and the light head and the relative pronoun are underlined.
The internal case is nominative, as the predicate \tit{mögen} `to like' takes nominative subjects. The relative pronoun \tit{wer} `\ac{rel}.\ac{an}.\ac{nom}' appears in the nominative case. This is the element that surfaces.
The external case is nominative as well, as the predicate \tit{besuchen} `to visit' also takes nominative subjects. The light head \tit{ər} `\ac{dem}.\ac{an}.\ac{nom}' appears in the nominative case. It is placed between square brackets because it does not surface.

\exg. Uns besucht [\underline{ər}], \underline{\tbf{wer}} \tbf{Maria} \tbf{mag}.\\
 2\ac{pl}.\ac{acc} visit.\ac{pres}.3\ac{sg}\scsub{[nom]} \ac{dem}.\ac{an}.\ac{nom} \ac{rel}.\ac{an}.\ac{nom} Maria.\ac{acc} like.\ac{pres}.3\ac{sg}\scsub{[nom]}\\
 `Who visits us likes Maria.' \flushfill{Modern German, adapted from \pgcitealt{vogel2001}{343}}\label{ex:mg-nom-nom-rep}

In Figure \ref{fig:mg-int=ext}, I give the syntactic structure of the relative pronoun at the top and the syntactic structure of the light head at the bottom.
The relative pronoun consists of three morphemes: \tit{w}, \tit{e} and \tit{r}.
The light head consists of two morphemes: \tit{ə} and \tit{r}.
As usual, I circle the part of the structure that corresponds to a particular lexical entry, and I place the corresponding phonology under it.
I draw a dashed circle around each constituent that is a constituent in both the light head and the relative pronoun.
As each constituent of the light head is also a constituent within the relative pronoun, the light head can be absent. I illustrate this by marking the content of the dashed circles for the light head gray.

I explain this constituent by constituent.
I start with the right-most constituent of the light head that spells out as \tit{r} (\tsc{nom}P). This constituent is also a constituent in the relative pronoun.
I continue with the left-most constituent of the light head that spells out as \tit{ə} (\tsc{med}P). This constituent is also a constituent in the relative pronoun, contained in \tsc{dist}P.
Both constituent of the light head are also a constituent within the relative pronoun, and the light head can be absent.

\begin{figure}[H]
  \center
  \begin{tabular}[b]{c}
        \toprule
        \tsc{nom} light head \tit{ə-r}\\
        \cmidrule{1-1}
      \scriptsize{
      \begin{forest} boompje
        [{\tsc{nom}P}, s sep=45mm
            [{\tsc{med}P},
            tikz={
            \node[label=below:{\tit{ə}},
            draw,circle,
            scale=0.95,
            fit to=tree]{};
            \node[
            draw,circle,
            scale=0.98,
            fill=DG,fill opacity=0.2,
            dashed,
            fit to=tree]{};
            }
                [{\tsc{deix}\scsub{2}}]
                [{\tsc{prox}P}
                    [{\tsc{deix}\scsub{1}}]
                    [{\tsc{ind}P}
                        [{\tsc{ind}}]
                        [{\tsc{masc}P}
                            [{\tsc{masc}}]
                            [{\tsc{class}P}
                                [{\tsc{class}}]
                                [{\tsc{ref}}]
                            ]
                        ]
                    ]
                ]
            ]
            [{\tsc{nom}P},
            tikz={
            \node[label=below:{\tit{r}},
            draw,circle,
            scale=0.8,
            fit to=tree]{};
            \node[
            draw,circle,
            fill=DG,fill opacity=0.2,
            scale=0.9,
            dashed,
            fit to=tree]{};
            }
                [{\tsc{f}1}]
            ]
        ]
      \end{forest}
      }
      \\
      \toprule
      \tsc{nom} relative pronoun \tit{w-e-r}
      \\
      \cmidrule{1-1}
      \scriptsize{
          \begin{forest} boompje
          [\tsc{rel}P, s sep=15mm
              [\tsc{rel}P,
              tikz={
              \node[label=below:\tit{w},
              draw,circle,
              scale=0.9,
              fit to=tree]{};
              }
                  [\tsc{rel}]
                  [\tsc{wh}]
              ]
              [\tsc{nom}P, s sep=50mm
                  [\tsc{dist}P,
                  tikz={
                  \node[label=below:\tit{e},
                  draw,circle,
                  scale=0.95,
                  fit to=tree]{};
                  }
                      [\tsc{deix}\scsub{3}]
                      [\tsc{med}P,
                      tikz={
                      \node[draw,circle,
                      dashed,
                      scale=0.9,
                      fit to=tree]{};
                      }
                          [\tsc{deix}\scsub{2}]
                          [\tsc{prox}P
                              [\tsc{deix}\scsub{1}]
                              [\tsc{ind}P
                                  [\tsc{ind}]
                                  [\tsc{masc}P
                                      [\tsc{masc}]
                                      [\tsc{class}P
                                          [\tsc{class}]
                                          [\tsc{ref} [\phantom{xxx}, roof]]
                                      ]
                                  ]
                              ]
                          ]
                      ]
                  ]
                  [\tsc{nom}P,
                  tikz={
                  \node[label=below:\tit{r},
                  draw,circle,
                  scale=0.8,
                  fit to=tree]{};
                  \node[draw,circle,
                  dashed,
                  scale=0.9,
                  fit to=tree]{};
                  }
                      [\tsc{f}1]
                  ]
              ]
          ]
        \end{forest}
        }
        \\
      \bottomrule
  \end{tabular}
  \caption {Modern German \tsc{ext}\scsub{nom} vs. \tsc{int}\scsub{nom} = \tit{wer}}
  \label{fig:mg-int=ext}
\end{figure}

Consider the example in \ref{ex:mg-nom-acc-rep}, in which the internal accusative case competes against the external nominative case. The relative clause is marked in bold, and the light head and the relative pronoun are underlined.
The internal case is accusative, as the predicate \tit{mögen} `to like' takes accusative objects. The relative pronoun \tit{wen} `\ac{rel}.\ac{an}.\ac{acc}' appears in the accusative case. This is the element that surfaces.
The external case is nominative, as the predicate \tit{besuchen} `to visit' takes nominative subjects. The light head \tit{ər} `\ac{dem}.\ac{an}.\ac{nom}' appears in the nominative case. It is placed between square brackets because it does not surface.

\exg. Uns besucht [\underline{ər}] \underline{\tbf{wen}} \tbf{Maria} \tbf{mag}.\\
 we.\ac{acc} visit.3\ac{sg}\scsub{[nom]} \tsc{dem}.\ac{nom}.\tsc{an} \tsc{rel}.\ac{acc}.\tsc{an} Maria.\ac{nom} like.3\ac{sg}\scsub{[acc]}\\
 `Who visits us, Maria likes.' \flushfill{adapted from \pgcitealt{vogel2001}{343}}\label{ex:mg-nom-acc-rep}

In Figure \ref{fig:mg-int-wins}, I give the syntactic structure of the relative pronoun at the top and the syntactic structure of the light head at the bottom.
The relative pronoun consists of three morphemes: \tit{w}, \tit{e} and \tit{n}.
The light head consists of two morphemes: \tit{ə} and \tit{r}.
Again, I circle the part of the structure that corresponds to a particular lexical entry, and I place the corresponding phonology under it.
I draw a dashed circle around each constituent that is a constituent in both the light head and the relative pronoun.
As each constituent of the light head is also a constituent within the relative pronoun, the light head can be absent. I illustrate this by marking the content of the dashed circles for the light head gray.

I explain this constituent by constituent.
I start with the right-most constituent of the light head that spells out as \tit{r} (\tsc{nom}P). This constituent is also a constituent in the relative pronoun, contained in \tsc{acc}P.
I continue with the left-most constituent of the light head that spells out as \tit{ə} (\tsc{med}P). This constituent is also a constituent in the relative pronoun, contained in \tsc{dist}P.
Both constituent of the light head are also a constituent within the relative pronoun, and the light head can be absent.

\begin{figure}[H]
  \center
 \caption {Modern German \tsc{ext}\scsub{nom} vs. \tsc{int}\scsub{acc} = \tit{wen}}
  \begin{tabular}[b]{c}
      \toprule
      \tsc{nom} light head \tit{ə-r}
      \\
      \cmidrule{1-1}
      \scriptsize{
      \begin{forest} boompje
        [{\tsc{nom}P}, s sep=45mm
            [{\tsc{med}P},
            tikz={
            \node[label=below:{\tit{ə}},
            draw,circle,
            scale=0.95,
            fit to=tree]{};
            \node[
            draw,circle,
            scale=0.98,
            dashed,
            fill=DG,fill opacity=0.2,
            fit to=tree]{};
            }
                [{\tsc{deix}\scsub{2}}]
                [{\tsc{prox}P}
                    [{\tsc{deix}\scsub{1}}]
                    [{\tsc{ind}P}
                        [{\tsc{ind}}]
                        [{\tsc{masc}P}
                            [{\tsc{masc}}]
                            [{\tsc{class}P}
                                [{\tsc{class}}]
                                [{\tsc{ref}}]
                            ]
                        ]
                    ]
                ]
            ]
            [{\tsc{nom}P},
            tikz={
            \node[label=below:{\tit{r}},
            draw,circle,
            scale=0.8,
            fit to=tree]{};
            \node[
            draw,circle,
            scale=0.9,
            dashed,
            fill=DG,fill opacity=0.2,
            fit to=tree]{};
            }
                [{\tsc{f}1}]
            ]
        ]
      \end{forest}
      }
      \\
      \toprule
      \tsc{acc} relative pronoun \tit{w-e-n}
      \\
      \cmidrule{1-1}
      \scriptsize{
      \begin{forest} boompje
        [\tsc{rel}P, s sep=15mm
            [\tsc{rel}P,
            tikz={
            \node[label=below:\tit{w},
            draw,circle,
            scale=0.9,
            fit to=tree]{};
            }
                [\tsc{rel}]
                [\tsc{wh}]
            ]
            [\tsc{acc}P, s sep=50mm
                [\tsc{dist}P,
                tikz={
                \node[label=below:\tit{e},
                draw,circle,
                scale=0.95,
                fit to=tree]{};
                }
                    [\tsc{deix}\scsub{3}]
                    [\tsc{med}P,
                    tikz={
                    \node[draw,circle,
                    dashed,
                    scale=0.9,
                    fit to=tree]{};
                    }
                        [\tsc{deix}\scsub{2}]
                        [\tsc{prox}P
                            [\tsc{deix}\scsub{1}]
                            [\tsc{ind}P
                                [\tsc{ind}]
                                [\tsc{masc}P
                                    [\tsc{masc}]
                                    [\tsc{class}P
                                        [\tsc{class}]
                                        [\tsc{ref} [\phantom{xxx}, roof]]
                                    ]
                                ]
                            ]
                        ]
                    ]
                ]
                [\tsc{acc}P,
                tikz={
                \node[label=below:\tit{n},
                draw,circle,
                scale=0.9,
                fit to=tree]{};
                }
                    [\tsc{f}2]
                    [\tsc{nom}P,
                    tikz={
                    \node[draw,circle,
                    dashed,
                    scale=0.8,
                    fit to=tree]{};
                    }
                        [\tsc{f}1]
                    ]
                ]
            ]
        ]
      \end{forest}
      }
      \\
      \bottomrule
  \end{tabular}
  \label{fig:mg-int-wins}
\end{figure}

Consider the examples in \ref{ex:mg-acc-nom-rep}, in which the internal nominative case competes against the external accusative case. The relative clauses are marked in bold, and the light heads and the relative pronouns are underlined. It is not possible to make a grammatical headless relative in this situation.
The internal case is nominative, as the predicate \tit{sein} `to be' takes nominative subjects. The relative pronoun \tit{wer} `\ac{rel}.\ac{an}.\ac{nom}' appears in the nominative case.
The external case is accusative, as the predicate \tit{einladen} `to invite' takes accusative objects. The light head \tit{ən} `\ac{dem}.\ac{an}.\ac{acc}' appears in the accusative case.
\ref{ex:mg-acc-nom-rep-rel} is the variant of the sentence in which the light head is absent (indicated by the square brackets) and the relative pronoun surfaces, and it is ungrammatical.
\ref{ex:mg-acc-nom-rep-lh} is the variant of the sentence in which the relative pronoun is absent (indicated by the square brackets) and the light head surfaces, and it is ungrammatical too.

\ex.\label{ex:mg-acc-nom-rep}
\ag. *Ich {lade ein}, [\underline{ən}] \underline{\tbf{wer}} \tbf{mir} \tbf{sympathisch} \tbf{ist}.\\
1\ac{sg}.\ac{nom} invite.\ac{pres}.1\ac{sg}\scsub{[acc]} \ac{rel}.\ac{an}.\ac{nom} 1\ac{sg}.\ac{dat} nice be.\ac{pres}.3\ac{sg}\scsub{[nom]}\\
`I invite who I like.' \flushfill{Modern German, adapted from \pgcitealt{vogel2001}{344}}\label{ex:mg-acc-nom-rep-rel}
\bg. *Ich {lade ein}, \underline{ən} [\underline{\tbf{wer}}] \tbf{mir} \tbf{sympathisch} \tbf{ist}.\\
1\ac{sg}.\ac{nom} invite.\ac{pres}.1\ac{sg}\scsub{[acc]} \ac{rel}.\ac{an}.\ac{nom} 1\ac{sg}.\ac{dat} nice be.\ac{pres}.3\ac{sg}\scsub{[nom]}\\
`I invite who I like.' \flushfill{Modern German, adapted from \pgcitealt{vogel2001}{344}}\label{ex:mg-acc-nom-rep-lh}

In Figure \ref{fig:mg-ext-wins}, I give the syntactic structure of the relative pronoun at the top and the syntactic structure of the light head at the bottom.
The relative pronoun consists of three morphemes: \tit{w}, \tit{e} and \tit{r}.
The light head consists of two morphemes: \tit{ə} and \tit{n}.
Again, I circle the part of the structure that corresponds to a particular lexical entry, and I place the corresponding phonology under it.
I draw a dashed circle around each constituent that is a constituent in both the light head and the relative pronoun.
Neither of the elements contains all constituents that the other element contains. The relative pronoun does not contain all constituents that the light head contains, and the light head does not contain all constituents that the relative pronoun contains. As a result, none of the elements can be absent.\footnote{
Why do we not see this result surface? Very good question.
}

I explain this constituent by constituent.
I start by showing that the light head cannot be absent.
Consider the right-most constituent of the light head that spells out as \tit{n} (\tsc{acc}P). This constituent is not a constituent in the relative pronoun: the relative pronoun has a constituent \tsc{nom}P, but it does not contain \tsc{f}2 to make it an \tsc{acc}P.
The light head has a constituent that is not a constituent in the relative pronoun, so the light head cannot be absent.

The relative pronoun can also not be absent.
Consider the middle constituent of the relative pronoun that spells out as \tit{e} (\tsc{dist}P). This constituent is not a constituent in the light head: the light head has a constituent \tsc{med}P, but it does not contain \tsc{deix}\scsub{3} to make it an \tsc{dist}P.
The same hold for the left-most constituent of the relative pronoun that spells out as \tit{w} (\tsc{rel}P). The light head lacks the features \tsc{wh} and \tsc{rel} that form the \tsc{rel}P.
The relative pronoun has constituents that are not constituents in the light head, so the relative pronoun cannot be absent.
In sum, neither of the elements contains all constituents that the other element contains, and none of the elements can be absent, so none of them is marked gray.

\begin{figure}[H]
  \center
 \caption {Modern German \tsc{ext}\scsub{acc} vs. \tsc{int}\scsub{nom} ≠ \tit{wer}/\tit{ən}}
  \begin{tabular}[b]{c}
        \toprule
        \tsc{acc} light head \tit{ə-n} \\
        \cmidrule{1-1}
      \scriptsize{
      \begin{forest} boompje
        [{\tsc{acc}P}, s sep=45mm
            [{\tsc{med}P},
            tikz={
            \node[label=below:{\tit{ə}},
            draw,circle,
            scale=0.95,
            fit to=tree]{};
            \node[
            draw,circle,
            scale=0.98,
            dashed,
            fit to=tree]{};
            }
                [{\tsc{deix}\scsub{2}}]
                [{\tsc{prox}P}
                    [{\tsc{deix}\scsub{1}}]
                    [{\tsc{ind}P}
                        [{\tsc{ind}}]
                        [{\tsc{masc}P}
                            [{\tsc{masc}}]
                            [{\tsc{class}P}
                                [{\tsc{class}}]
                                [{\tsc{ref}}]
                            ]
                        ]
                    ]
                ]
            ]
            [\tsc{acc}P,
            tikz={
            \node[label=below:\tit{n},
            draw,circle,
            scale=0.9,
            fit to=tree]{};
            }
                [\tsc{f}2]
                [{\tsc{nom}P},
                tikz={
                \node[
                draw,circle,
                scale=0.8,
                dashed,
                fit to=tree]{};
                }
                    [{\tsc{f}1}]
                ]
            ]
        ]
      \end{forest}
      }
      \\
      \toprule
      \tsc{nom} relative pronoun \tit{w-e-r}
      \\
      \cmidrule{1-1}
      \scriptsize{
          \begin{forest} boompje
          [\tsc{rel}P, s sep=15mm
              [\tsc{rel}P,
              tikz={
              \node[label=below:\tit{w},
              draw,circle,
              scale=0.9,
              fit to=tree]{};
              }
                  [\tsc{rel}]
                  [\tsc{wh}]
              ]
              [\tsc{nom}P, s sep=50mm
                  [\tsc{dist}P,
                  tikz={
                  \node[label=below:\tit{e},
                  draw,circle,
                  scale=0.95,
                  fit to=tree]{};
                  }
                      [\tsc{deix}\scsub{3}]
                      [\tsc{med}P,
                      tikz={
                      \node[draw,circle,
                      dashed,
                      scale=0.9,
                      fit to=tree]{};
                      }
                          [\tsc{deix}\scsub{2}]
                          [\tsc{prox}P
                              [\tsc{deix}\scsub{1}]
                              [\tsc{ind}P
                                  [\tsc{ind}]
                                  [\tsc{masc}P
                                      [\tsc{masc}]
                                      [\tsc{class}P
                                          [\tsc{class}]
                                          [\tsc{ref} [\phantom{xxx}, roof]]
                                      ]
                                  ]
                              ]
                          ]
                      ]
                  ]
                  [\tsc{nom}P,
                  tikz={
                  \node[label=below:\tit{r},
                  draw,circle,
                  scale=0.8,
                  fit to=tree]{};
                  \node[draw,circle,
                  dashed,
                  scale=0.9,
                  fit to=tree]{};
                  }
                      [\tsc{f}1]
                  ]
              ]
          ]
        \end{forest}
        }
      \\
      \bottomrule
  \end{tabular}
  \label{fig:mg-ext-wins}
\end{figure}




\section{Deriving the matching type}\label{sec:deriving-matching}

% The idea that a different constituency leads to the absence of different syntactic structured is illustrated by Cinque with nouns and adjectives. > move this the polish part

\ex. Polish: \tsc{ext} \tsc{dat}\\
\scriptsize{
\begin{forest} boompje
  [\tsc{Med}P, s sep=10mm
      [\tsc{Med}P,
      tikz={
      \node[label=below:\tit{t},
      draw,circle,
      scale=0.9,
      fit to=tree]{};
      }
          [\tsc{deix\scsub{2}}]
          [\tsc{deix\scsub{1}}]
      ]
      [\tsc{dat}P, s sep=20mm
          [\tsc{masc}P,
          tikz={
          \node[label=below:\tit{e/o},
          draw,circle,
          scale=0.9,
          fit to=tree]{};
          }
              [\tsc{masc}]
              [\tsc{class}P
                  [\tsc{class}]
                  [\tsc{ref} [\phantom{xxx}, roof]]
              ]
          ]
          [\tsc{dat}P,
          tikz={
          \node[label=below:\tit{mu},
          draw,circle,
          scale=0.9,
          fit to=tree]{};
          }
              [\tsc{f}3]
              [\tsc{acc}P
                  [\tsc{f}2]
                  [\tsc{nom}P
                      [\tsc{f}1]
                      [\tsc{ind}P
                          [\tsc{ind}]
                      ]
                  ]
              ]
          ]
      ]
  ]
\end{forest}
}

\ex. Polish: \tsc{ext} \tsc{acc}\\
\scriptsize{
\begin{forest} boompje
  [\tsc{Med}P, s sep=10mm
      [\tsc{Med}P,
      tikz={
      \node[label=below:\tit{t},
      draw,circle,
      scale=0.9,
      fit to=tree]{};
      }
          [\tsc{deix\scsub{2}}]
          [\tsc{deix\scsub{1}}]
      ]
      [\tsc{dat}P, s sep=25mm
          [\tsc{ind}P,
          tikz={
          \node[label=below:\tit{e/o},
          draw,circle,
          scale=0.9,
          fit to=tree]{};
          }
              [\tsc{ind}]
              [\tsc{mascP}
                  [\tsc{masc}]
                  [\tsc{class}P
                      [\tsc{class}]
                      [\tsc{ref} [\phantom{xxx}, roof]]
                  ]
              ]
          ]
          [\tsc{acc}P,
          tikz={
          \node[label=below:\tit{go},
          draw,circle,
          scale=0.85,
          fit to=tree]{};
          }
              [\tsc{f}2]
              [\tsc{nom}P
                  [\tsc{f}1]
              ]
          ]
      ]
  ]
\end{forest}
}

\ex. Polish: \tsc{int} \tsc{dat}\\
\scriptsize{
\begin{forest} boompje
  [\tsc{rel}P, s sep=27mm
      [\tsc{rel}P,
      tikz={
      \node[label=below:\tit{k},
      draw,circle,
      scale=0.925,
      fit to=tree]{};
      }
          [\tsc{rel}]
          [\tsc{wh}P
              [\tsc{wh}]
              [\tsc{dist}P
                  [\tsc{deix\scsub{3}}]
                  [\tsc{med}P,
                      [\tsc{deix\scsub{2}}]
                      [\tsc{deix\scsub{1}}]
                  ]
              ]
          ]
      ]
      [\tsc{dat}P, s sep=20mm
          [\tsc{masc}P,
          tikz={
          \node[label=below:\tit{e/o},
          draw,circle,
          scale=0.9,
          fit to=tree]{};
          }
              [\tsc{masc}]
              [\tsc{class}P
                  [\tsc{class}]
                  [\tsc{ref} [\phantom{xxx}, roof]]
              ]
          ]
          [\tsc{dat}P,
          tikz={
          \node[label=below:\tit{mu},
          draw,circle,
          scale=0.9,
          fit to=tree]{};
          }
              [\tsc{f}3]
              [\tsc{acc}P
                  [\tsc{f}2]
                  [\tsc{nom}P
                      [\tsc{f}1]
                      [\tsc{ind}P
                          [\tsc{ind}]
                      ]
                  ]
              ]
          ]
      ]
  ]
\end{forest}
}

\ex. Polish: \tsc{int} \tsc{acc}\\
\scriptsize{
\begin{forest} boompje
  [\tsc{rel}P, s sep=27mm
      [\tsc{rel}P,
      tikz={
      \node[label=below:\tit{k},
      draw,circle,
      scale=0.925,
      fit to=tree]{};
      }
          [\tsc{rel}]
          [\tsc{wh}P
              [\tsc{wh}]
              [\tsc{dist}P
                  [\tsc{deix\scsub{3}}]
                  [\tsc{med}P,
                      [\tsc{deix\scsub{2}}]
                      [\tsc{deix\scsub{1}}]
                  ]
              ]
          ]
      ]
      [\tsc{acc}P, s sep=25mm
          [\tsc{ind}P,
          tikz={
          \node[label=below:\tit{e/o},
          draw,circle,
          scale=0.9,
          fit to=tree]{};
          }
              [\tsc{ind}]
              [\tsc{masc}P
                  [\tsc{masc}]
                  [\tsc{class}P
                      [\tsc{class}]
                      [\tsc{ref} [\phantom{xxx}, roof]]
                  ]
              ]
          ]
          [\tsc{acc}P,
          tikz={
          \node[label=below:\tit{go},
          draw,circle,
          scale=0.85,
          fit to=tree]{};
          }
              [\tsc{f}2]
              [\tsc{nom}P
                  [\tsc{f}1]
              ]
          ]
      ]
  ]
\end{forest}
}



Polish only allows the deletion of the light head in the matching situation. It is not obligatory there, you can just as well have a light-headed relative. The deletion is possible, because you have two elements that are pretty similar?

\exg. Jan czyta to, co Maria czyta.\\
 Jan read this what Maria reads\\
 `Jan reads what Maria reads.' \flushfill{Polish, \pgcitealt{citko2004}{96}}


Radek: Czech distinguishes between accidental uniqueness and inherent uniqueness. Accidental uniqueness: with \tsc{dem}, inherent uniqueness: without \tsc{dem}.

Radek's situation:

Two student assistants A and B are at their shared workdesk, which they share with other student assistants and where there’s a computer and a couple of other things, including a book (it doesn’t really matter to whom the book belongs). A is looking for a pencil, B says

\exg. Nějaká tužka je vedle {počítače /\#toho počítače}.\\
some pencil is {next to} computer \tsc{dem} computer\\
`There’s a pencil next to the computer.'

All situations like the topic situation – A and B’s shared office (desk)– have exactly one computer in it.

\exg. Nějaká tužka je vedle {té knížky /\#knížky}\\
some pencil is {next to} \tsc{dem} book book\\
`There’s a pencil next to the book.'

There is exactly one book in the topic situation – A and B’s shared office (desk) – and it does not hold that all situation like the topic situation have exactly one book in it

Florian showed that this is different for Modern German:

\begin{table}[H]
\begin{tabular}{c|ccc}
\toprule
       & anaphoric                & situational uniqueness              & inherent uniqueness                 \\
       \cmidrule{2-4}
Polish & \tsc{dem}  & \cellcolor{DG}\tsc{dem}             & ∅                                   \\
German & \tsc{dem}\scsub{strong}  & \cellcolor{LG}\tsc{dem}\scsub{weak} & \cellcolor{LG}\tsc{dem}\scsub{weak} \\
\bottomrule
\end{tabular}
\end{table}

\tit{to} is incompatible with \tit{ever}, because \tit{to} makes it accidentally uniqueness and \tit{ever} requires inherent uniqueness



\section{Deriving the non-matching type}\label{sec:deriving-nonmatching}

\exg. quham dher chisendit scolda uuerdhan\\
 come.\ac{pst}.3\ac{sg}\scsub{[nom]} \ac{rel}.\ac{sg}.\ac{m}.\ac{nom} send.\ac{pst}.\ac{ptcp}\scsub{[nom]} should.\ac{pst}.3\ac{sg} become.\ac{inf}\\
 `the one, who should have been sent, came' \flushfill{Old High German, \ac{isid} 35:5}\label{ex:ohg-nom-nom-rep-workout}

\exg. Thíz ist \tbf{then} \tbf{sie} \tbf{zéllent}\\
\ac{dem}.\ac{sg}.\ac{n}.\ac{nom} be.\ac{pres}.3\ac{sg}\scsub{[nom]} \ac{rel}.\ac{sg}.\ac{m}.\ac{acc} 3\ac{pl}.\ac{m}.\ac{nom} tell.\ac{pres}.3\ac{pl}\scsub{[acc]}\\
`this is the one whom they talk about' \flushfill{Old High German, \ac{otfrid} III 16:50}\label{ex:ohg-nom-acc-rep}

\exg. ih bibringu fona iacobes samin endi fona iuda dhen \tbf{mina} \tbf{berga} \tbf{chisitzit}\\
1\ac{sg}.\ac{nom} {create}.\ac{pres}.1\ac{sg}\scsub{[acc]} of Jakob.\ac{gen} seed.\ac{sg}.\ac{dat} and of Judah.\ac{dat} \ac{rel}.\ac{sg}.\ac{m}.\ac{acc} my.\ac{acc}.\ac{m}.\ac{pl} mountain.\ac{acc}.\ac{pl} possess.\ac{pres}.3\ac{sg}\scsub{[nom]}\\
`I create of the seed of Jacob and of Judah the one, who possess my mountains' \flushfill{Old High German, \ac{isid} 34:3}\label{ex:ohg-acc-nom-rep}

\ex. Old High German: \tsc{ext} \tsc{acc}\\
\scriptsize{
\begin{forest} boompje
  [\tsc{d}P, s sep=15mm
      [\tsc{d}P,
      tikz={
      \node[label=below:\tit{d},
      draw,circle,
      scale=0.9,
      fit to=tree]{};
      }
          [\tsc{d}, roof]
      ]
      [\tsc{acc}P, s sep=50mm
          [\tsc{dist}P,
          tikz={
          \node[label=below:\tit{e},
          draw,circle,
          scale=0.95,
          fit to=tree]{};
          }
              [\tsc{deix}\scsub{3}]
              [\tsc{med}P
                  [\tsc{deix}\scsub{2}]
                  [\tsc{prox}P
                      [\tsc{deix}\scsub{1}]
                      [\tsc{ind}P
                          [\tsc{ind}]
                          [\tsc{masc}P
                              [\tsc{masc}]
                              [\tsc{class}P
                                  [\tsc{class}]
                                  [\tsc{ref} [\phantom{xxx}, roof]]
                              ]
                          ]
                      ]
                  ]
              ]
          ]
          [\tsc{acc}P,
          tikz={
          \node[label=below:\tit{n},
          draw,circle,
          scale=0.85,
          fit to=tree]{};
          }
              [\tsc{f}2]
              [\tsc{nom}P
                  [\tsc{f}1]
              ]
          ]
      ]
  ]
\end{forest}
}

\ex. Old High German: \tsc{ext} \tsc{nom}\\
\scriptsize{
\begin{forest} boompje
  [\tsc{d}P, s sep=15mm
      [\tsc{d}P,
      tikz={
      \node[label=below:\tit{d},
      draw,circle,
      scale=0.9,
      fit to=tree]{};
      }
          [\tsc{d}, roof]
      ]
      [\tsc{acc}P, s sep=50mm
          [\tsc{dist}P,
          tikz={
          \node[label=below:\tit{e},
          draw,circle,
          scale=0.95,
          fit to=tree]{};
          }
              [\tsc{deix}\scsub{3}]
              [\tsc{med}P
                  [\tsc{deix}\scsub{2}]
                  [\tsc{prox}P
                      [\tsc{deix}\scsub{1}]
                      [\tsc{ind}P
                          [\tsc{ind}]
                          [\tsc{masc}P
                              [\tsc{masc}]
                              [\tsc{class}P
                                  [\tsc{class}]
                                  [\tsc{ref} [\phantom{xxx}, roof]]
                              ]
                          ]
                      ]
                  ]
              ]
          ]
          [\tsc{nom}P,
          tikz={
          \node[label=below:\tit{r},
          draw,circle,
          scale=0.85,
          fit to=tree]{};
          }
              [\tsc{f}1]
          ]
      ]
  ]
\end{forest}
}

\ex. Old High German: \tsc{int} \tsc{acc}\\
\scriptsize{
\begin{forest} boompje
  [\tsc{rel}P, s sep=15mm
      [\tsc{rel}P,
      tikz={
      \node[label=below:\tit{d},
      draw,circle,
      scale=0.9,
      fit to=tree]{};
      }
          [\tsc{rel}]
          [\tsc{d}]
      ]
      [\tsc{acc}P, s sep=50mm
          [\tsc{dist}P,
          tikz={
          \node[label=below:\tit{e},
          draw,circle,
          scale=0.95,
          fit to=tree]{};
          }
              [\tsc{deix}\scsub{3}]
              [\tsc{med}P
                  [\tsc{deix}\scsub{2}]
                  [\tsc{prox}P
                      [\tsc{deix}\scsub{1}]
                      [\tsc{ind}P
                          [\tsc{ind}]
                          [\tsc{masc}P
                              [\tsc{masc}]
                              [\tsc{class}P
                                  [\tsc{class}]
                                  [\tsc{ref} [\phantom{xxx}, roof]]
                              ]
                          ]
                      ]
                  ]
              ]
          ]
          [\tsc{acc}P,
          tikz={
          \node[label=below:\tit{n},
          draw,circle,
          scale=0.85,
          fit to=tree]{};
          }
              [\tsc{f}2]
              [\tsc{nom}P
                  [\tsc{f}1]
              ]
          ]
      ]
  ]
\end{forest}
}

\ex. Old High German: \tsc{int} \tsc{acc}\\
\scriptsize{
\begin{forest} boompje
  [\tsc{rel}P, s sep=15mm
      [\tsc{rel}P,
      tikz={
      \node[label=below:\tit{d},
      draw,circle,
      scale=0.9,
      fit to=tree]{};
      }
          [\tsc{rel}]
          [\tsc{d}]
      ]
      [\tsc{acc}P, s sep=50mm
          [\tsc{dist}P,
          tikz={
          \node[label=below:\tit{e},
          draw,circle,
          scale=0.95,
          fit to=tree]{};
          }
              [\tsc{deix}\scsub{3}]
              [\tsc{med}P
                  [\tsc{deix}\scsub{2}]
                  [\tsc{prox}P
                      [\tsc{deix}\scsub{1}]
                      [\tsc{ind}P
                          [\tsc{ind}]
                          [\tsc{masc}P
                              [\tsc{masc}]
                              [\tsc{class}P
                                  [\tsc{class}]
                                  [\tsc{ref} [\phantom{xxx}, roof]]
                              ]
                          ]
                      ]
                  ]
              ]
          ]
          [\tsc{nom}P,
          tikz={
          \node[label=below:\tit{r},
          draw,circle,
          scale=0.85,
          fit to=tree]{};
          }
              [\tsc{f}1]
          ]
      ]
  ]
\end{forest}
}


The non-matching type of language allows for matching cases, it allows the internal case to win, and it allows the external case to win. I have been describing Old High German as an example of this type. In this section, I show what it is about Old High German that causes the language to be of the non-matching type. I propose that the crucial factor is that Old High German has a syncretic internal and external base. Since they are syncretic, the features in the internal base contain the features in the external base, and the features in the external base contain just as well the features in the internal base. The internal base containing the external base causes the internal case to be allowed to surface when it wins the case competition. The external base containing the internal base causes the external case to be allowed to surface when it wins the case competition.

This section is structured as follows. First, I argue that Old High German headless relatives are derived from relative clauses headed by a light head, i.e. light-headed relatives. In this analysis, the internal element is what can descriptively be called the relative pronoun, and the external element is what can descriptively be called the light head. The internal element surfaces as the relative pronoun when the internal case is more complex, and the external element surfaces as the relative pronoun when the external case is more complex. In this section, I decompose the internal and external element, and I show which morpheme corresponds to which features. Both elements consist of two morphemes: a base part and a case part. I go through the examples in Table \ref{tbl:forms-ohg}, showing per situation how the base and case parts syntactically contain the other base and case parts. This containment is crucial. When the internal base contains the external base, the internal case is allowed to surface when it is more complex, and when the external base contains the internal base, the external case is allowed to surface when it is more complex.

\begin{table}[H]
  \center
  \caption{Base comparison in Old High German}
\begin{tabular}{ccccccc}
  \toprule
                      & \multicolumn{2}{c}{\tsc{int} element}  & \multicolumn{2}{c}{\tsc{ext} element}  & \multicolumn{2}{c}{\tsc{rel} pronoun} \\
                        \cmidrule(lr){2-3}                        \cmidrule(lr){4-5}                      \cmidrule(lr){6-7}
                      & base\scsub{int} & case\scsub{int}       & base\scsub{ext} & case\scsub{ext}     & base\scsub{rel} & case\scsub{rel} \\
                        \cmidrule(lr){2-2}    \cmidrule(lr){3-3}  \cmidrule(lr){4-4} \cmidrule(lr){5-5}   \cmidrule(lr){6-6} \cmidrule(lr){7-7}
\tsc{int} = \tsc{ext} & dhe & r                                 & dhe & r                               & dhe & r                           \\
\tsc{int} > \tsc{ext} & dhe & n                                 & dhe & r                               & dhe & n                           \\
\tsc{int} < \tsc{ext} & dhe & r                                 & dhe & n                               & dhe & n                           \\
\bottomrule
\end{tabular}
\label{tbl:forms-ohg}
\end{table}

I propose headless relatives are derived from light-headed relatives (\citealt{fuss2014,hanink2018} argue the same but for Modern German\footnote{
A difference with Modern German is that one of the elements can only be absent when the cases match. In Section \ref{ch:discussion} I return to the point why Modern German does not have non-matching headless relatives that look like Old High German, although it still has syncretic light heads and relative pronouns.
}).
In a light-headed relative, the head of a relative is not a full noun phrase, but it is a bit `lighter': it only consists of a demonstrative. Consider the light-headed relative in \ref{ex:ohg-double}. \tit{Thér} `\tsc{dem}.\tsc{sg}.\tsc{m}.\tsc{nom}' is the head of the relative clause, which is the external element. \tit{Then} `\tsc{rel}.\tsc{sg}.\tsc{m}.\tsc{acc}' is the relative pronoun of the relative clause, which is the internal element.

\exg. eno nist thiz thér then ir suochet zi arslahanne?\\
 now {not be.3\ac{sg}} \tsc{dem}.\tsc{sg}.\tsc{n}.\tsc{nom} \tsc{dem}.\tsc{sg}.\tsc{m}.\tsc{nom}
 \tsc{rel}.\tsc{sg}.\tsc{m}.\tsc{acc} 2\ac{pl}.\tsc{nom} seek.2\tsc{pl} to kill.\tsc{inf}.\ac{sg}.\tsc{dat}\\
 `Isn't this now the one, who you seek to kill?'\label{ex:ohg-double}

The difference between a light-headed relative and a headless relative is that in headless relatives, either the internal or the external is absent. The absent element is the one that has the least complex case. This shows the presence of two elements in Old High German is optional.\footnote{
This sharply contrasts with headless relatives in Modern German, which are always ungrammatical when both the internal and external elements surface. I come back to this in Section \ref{sec:deriving-only-internal}.
}
In Old High German, there are three possible constructions: the internal and external element can both surface, only the internal element can surface and only the external element can surface. If only one of the two elements surfaces, this is the element that bears the most complex case, which is either the internal or the external one, as I have shown in Chapter \ref{ch:typology}. I assume that whether both or only one of the elements surfaces is determined by information structure. In \ref{ex:ohg-double}, the external element \tit{thér} `\tsc{dem}.\tsc{sg}.\tsc{m}.\tsc{nom}' is the candidate to be absent. However, it seems plausible that this is emphasized in this sentence and that it, therefore, cannot be absent.

Support for the idea that Old High German headless relatives are derived from light-headed ones comes from their interpretation. Headless relatives in which the relative pronoun starts with a \tit{d}, such as in Old High German, seem to be linked to individuating or definite readings and not to generalizing or indefinite readings \citep[cf.][]{fuss2017}. I illustrate this with the two examples I repeat from Chapter  \ref{ch:typology}.

Consider the example in \ref{ex:ohg-nom-acc-interpretation}, repeated from Chapter \ref{ch:typology}.
In this example, the author refers to the specific person which was talked about, and not to any or every person that was talked about.

\exg. Thíz ist \tbf{then} \tbf{sie} \tbf{zéllent}\\
\ac{dem}.\ac{sg}.\ac{n}.\ac{nom} be.\ac{pres}.3\ac{sg}\scsub{[nom]} \ac{rel}.\ac{sg}.\ac{m}.\ac{acc}
3\ac{pl}.\ac{m}.\ac{nom} tell.\ac{pres}.3\ac{pl}\scsub{[acc]}\\
`this is the one whom they talk about'\\
not: `this is whoever they talk about' \flushfill{Old High German, \ac{otfrid} III 16:50}\label{ex:ohg-nom-acc-interpretation}

Consider also the example in \ref{ex:ohg-nom-acc-interpretation}, repeated from Chapter \ref{ch:typology}.
In this example, the author refers to the specific person who spoke to someone, and not to any or every person who spoke to someone.

\exg. enti aer {ant uurta} demo \tbf{zaimo} \tbf{sprah}\\
and 3\ac{sg}.\ac{m}.\ac{nom} reply.\ac{pst}.3\ac{sg}\scsub{[dat]} \ac{rel}.\ac{sg}.\ac{m}.\ac{dat} {to 3\ac{sg}.\ac{m}.\ac{dat}} speak.\ac{pst}.3\ac{sg}\scsub{[nom]}\\
`and he replied to the one who spoke to him'\\
not: `and he replied to whoever spoke to him'
 \flushfill{Old High German, \ac{mons} 7:24, adapted from \pgcitealt{pittner1995}{199}}\label{ex:ohg-dat-nom-rep}

I conclude that the internal element in Old High German is the descriptive relative pronoun, and the external element in Old High German is the descriptive light head. In what follows I closely examine the internal structure of the internal and external element. I illustrate how the internal base and the external base are identical, so they contain each other.

The light head in a light-headed relative is a demonstrative pronoun. Relative and demonstrative pronouns are syncretic in Old High German \pgcitep{braune2018}{338}. Table \ref{tbl:rel-dem-ohg} gives an overview of the forms in singular and plural, neuter, masculine and feminine and nominative, accusative and dative. The pronouns consist of two morphemes: a \tit{d} and suffix that differs per number, gender and case.\footnote{
\tit{d} can also be written as \tit{dh} and \tit{th}, \tit{ë} and \tit{ē} can also be \tit{e} and \tit{é} \pgcitep{braune2018}{339}.
}\footnote{
The suffix could also be further divided into a vowel and a suffix. As this is not relevant for the discussion here, I refrain from doing that.
}

\begin{table}[H]
 \center
 \caption {Relative/demonstrative pronouns in Old High German \pgcitep{braune2018}{339}}
  \begin{tabular}{cccc}
  \toprule
            & \ac{n}.\ac{sg}  & \ac{m}.\ac{sg}      & \ac{f}.\ac{sg}    \\
        \cmidrule{2-4}
  \ac{nom}  & d-aȥ            & d-ër                & d-iu               \\
  \ac{acc}  & d-aȥ            & d-ën                & d-ea/d-ia         \\
  \ac{dat}  & d-ëmu/d-ëmo     & d-ëmu/d-ëmo         & d-ëru/d-ëro       \\
  \bottomrule
            & \ac{n}.\ac{pl}  & \ac{m}.\ac{pl}      &  \ac{f}.\ac{pl}  \\
        \cmidrule{2-4}
  \ac{nom}  & d-iu            &  d-ē/d-ea/d-ia/d-ie & d-eo/-io         \\
  \ac{acc}  & d-iu            &  d-ē/d-ea/d-ia/d-ie & d-eo/-io         \\
  \ac{dat}  & d-ēm/d-ēn       &  d-ēm/d-ēn          & d-ēm/d-ēn        \\
    \bottomrule
  \end{tabular}
  \label{tbl:rel-dem-ohg}
\end{table}

The suffixes that combine with the \tit{d} in demonstrative and relative pronouns also appear on adjectives. This is illustrated in Table \ref{tbl:adj-ohg}.

\begin{table}[H]
 \center
 \caption {Adjectives on \tit{-a-/-ō-} in Old High German \pgcitealt{braune2018}{300}}
  \begin{tabular}{cccc}
  \toprule
            & \ac{n}.\ac{sg}    & \ac{m}.\ac{sg}      & \ac{f}.\ac{sg}    \\
    \cmidrule{2-4}
  \ac{nom}  & jung, jung-aȥ     & jung, jung-ēr       & jung, jung-iu     \\
  \ac{acc}  & jung, jung-aȥ     & jung-an             & jung-a            \\
  \ac{dat}  & jung-emu/jung-emo & jung-emu/jung-emo   & jung-eru/jung-ero \\
  \bottomrule
            & \ac{n}.\ac{pl}    & \ac{m}.\ac{pl}      &  \ac{f}.\ac{pl}   \\
      \cmidrule{2-4}
  \ac{nom}  & jung-iu           &  jung-e             & jung-o            \\
  \ac{acc}  & jung-iu           &  jung-e             & jung-o            \\
  \ac{dat}  & jung-ēm/jung-ēn   &  jung-ēm/jung-ēn    & jung-ēm/jung-ēn   \\
    \bottomrule
  \end{tabular}
  \label{tbl:adj-ohg}
\end{table}

I conclude from this that the suffix expresses features that are specific to being nominal, like number, gender and case. Not part of the suffix are features that are specific to being a demonstrative or relative pronoun, like anaphoricity and definiteness. I assume that these are expressed by the morpheme \tit{d}.

In this section, I only discuss two forms: the nominative and accusative masculine singular relative and demonstrative pronoun. The nominative is \tit{dër} and the accusative is \tit{dën}. In what follows, I discuss the feature content of the morphemes \tit{d}, \tit{ër} and \tit{ën}. I start with the features that are expressed by the suffixes \tit{ër} and \tit{ën}.

For the suffixes, I use pronominal features that are distinguished by \citet{harley2002}: \tsc{ref}, \tsc{class}, \tsc{masc} and \tsc{ind}. \tsc{ref} refers to a referring expression, which all pronouns contain. The feature \tsc{class} refers to gender features, which is neuter if it is not combined with any other features. Combining \tsc{class} with the feature \tsc{masc} gives a masculine gender. \tsc{ind} refer to number, which is singular if it is not combined with any other features.
I addition, I use the case features introduced by \citet{caha2009}, which I already discussed in Chapter \ref{ch:decomposition}. \tsc{f}1 refers to a nominative, and \tsc{f}1 and \tsc{f}2 refers to an accusative.

This allows me to propose the following lexical entries for the two suffixes.

\ex.
\begin{forest} boom
  [\tsc{nom}P
      [\tsc{f}1]
      [\tsc{ind}P
          [\tsc{ind}]
          [\tsc{masc}P
              [\tsc{masc}]
              [\tsc{class}P
                  [\tsc{class}]
                  [\tsc{ref} [\phantom{xxx}, roof]]
              ]
          ]
      ]
  ]
  {\draw (.east) node[right]{⇔ \tit{en}}; }
\end{forest}
\label{ex:ohg-er-lexicon}

\ex.
\begin{forest} boom
  [\tsc{acc}P
      [\tsc{f}2]
      [\tsc{nom}P
          [\tsc{f}1]
          [\tsc{ind}P
              [\tsc{ind}]
              [\tsc{masc}P
                  [\tsc{masc}]
                  [\tsc{class}P
                      [\tsc{class}]
                      [\tsc{ref} [\phantom{xxx}, roof]]
                  ]
              ]
          ]
      ]
  ]
  {\draw (.east) node[right]{⇔ \tit{en}}; }
\end{forest}
\label{ex:ohg-en-lexicon}

The \tit{d} morpheme corresponds to definiteness and anaphoricity. Anaphoricity establishes a relation with another element in the (linguistic) discourse. Definiteness encodes that the referent is specific.

\ex.
\begin{forest} boom
  [\tsc{d}P
      [\tsc{d}]
      [\tsc{ana}]
  ]
  {\draw (.east) node[right]{⇔ \tit{d}}; }
\end{forest}
\label{ex:ohg-d-lexicon}

So, the two relative pronouns look like this.\footnote{A question that arises here is how the case features can form a constituent to the exclusion of definiteness and anaphoricity. I come back to this issue in Chapter \ref{ch:discussion}.}

\ex.
\a.
\begin{forest} boom
  [\tsc{d}P
      [\tsc{d}P,
      tikz={
      \node[label=below:\tit{d},
      draw,circle,
      scale=0.80,
      fit to=tree]{};
      }
          [\tsc{d}]
          [\tsc{ana}]
      ]
      [\tsc{acc}P,
      tikz={
      \node[label=below:\tit{en},
      draw,circle,
      scale=0.85,
      fit to=tree]{};
      }
          [\tsc{f}2]
          [\tsc{nom}P
              [\tsc{f1}]
              [\tsc{ind}P
                  [\tsc{ind}]
                  [\tsc{masc}P
                      [\tsc{masc}]
                      [\tsc{class}P
                          [\tsc{class}]
                          [\tsc{ref} [\phantom{xxx}, roof]]
                      ]
                  ]
              ]
          ]
      ]
  ]
\end{forest}
\b.
\begin{forest} boom
  [\tsc{d}P
      [\tsc{d}P,
      tikz={
      \node[label=below:\tit{d},
      draw,circle,
      scale=0.80,
      fit to=tree]{};
      }
          [\tsc{d}]
          [\tsc{ana}]
      ]
      [\tsc{nom}P,
      tikz={
      \node[label=below:\tit{er},
      draw,circle,
      scale=0.85,
      fit to=tree]{};
      }
          [\tsc{f1}]
          [\tsc{ind}P
              [\tsc{ind}]
              [\tsc{masc}P
                  [\tsc{masc}]
                  [\tsc{class}P
                      [\tsc{class}]
                      [\tsc{ref} [\phantom{xxx}, roof]]
                  ]
              ]
          ]
      ]
  ]
\end{forest}

To sum up, Old High German allows the internal and the external case to surface when either of them wins the case competition. This is due to the fact that the bases of the internal and the external element are syncretic. Because of that, the internal base contains the external base, which allows the internal case to surface, and the external base contains the internal base, which allows the external case to surface.




\section{Technical details}





\section{Summary}

The linguistic counterpart of `allow \tsc{ext}?' is whether the internal base and the external base are syncretic (base\scsub{int} = base\scsub{ext}?).
The linguistic counterpart of `allow \tsc{int}?' is whether the external base is a clitic (base\scsub{ext} = clitic?).

\begin{figure}[H]
  \centering
    \footnotesize{
    \begin{tikzpicture}[node distance=1.5cm]
      \node (question2) [question]
      {base\scsub{int} = base\scsub{ext}?};
          \node (outcome2) [outcome, below of=question2, xshift=-1.5cm]
          {complex};
              \node (example2) [example, below of=outcome2, yshift=0.25cm]
              {\scriptsize{e.g. Gothic, Old High German, Classical Greek}};
          \node (question3) [question, below of=question2, xshift=2cm, yshift=-0.5cm]
          {base\scsub{ext} = clitic?};
              \node (outcome3) [outcome, below of=question3, xshift=-1.5cm]
              {\ac{int} + complex};
                  \node (example3) [example, below of=outcome3, yshift=0.25cm]
                  {\scriptsize{e.g. Modern German\\\phantom{x}}};
              \node (outcome4) [outcome, below of=question3, xshift=1.5cm]
              {matching};
                  \node (example4) [example, below of=outcome4, yshift=0.25cm]
                  {\scriptsize{e.g. Polish\\\phantom{x}}};

    \draw [arrow] (question2) -- node[anchor=east] {yes} (outcome2);
    \draw [arrow] (question2) -- node[anchor=west] {no} (question3);
    \draw [arrow] (question3) -- node[anchor=east] {no} (outcome3);
    \draw [arrow] (question3) -- node[anchor=west] {yes} (outcome4);
    \end{tikzpicture}
    }
    \caption{Two theoretical parameters generate three language types}
    \label{fig:formal-parameters}
\end{figure}

\section{Aside: a larger syntactic context}\label{sec:larger-syntax}

If you talk about different patterns, there can be different locations to put your parameters. Himmelreich put her parameters in the structure. I put my parameters in the elements themselves. I show what an analysis like Himmelreich looks like, and I show then that it is difficult to reduce that then to differences in the lexicon (because it has to do with agree?).

So what I do is keep the parameters that she was differing stable. I change the things that she kept constant, the internal and external element. Does her structure then work with what I want? Not entirely, because I have to do a c-command that is going in the wrong direction.
Then I show a syntactic structure that could be compatible with mine, and I show why a grafting one is not.



In this dissertation I focus on when languages allow the internal and external case to win the case competition. In my proposal, this depends on the comparison between the internal and external base. The larger syntactic context in which this takes place should be kept stable. For concreteness, I show a possible implementation in Cinque's double-headed analysis of relative clause. I do by no means claim that claim this is the only or even correct implementation.



% In the previous section I introduced the relative pronoun as the internal element. This means that the other element is the external element. This section starts with the observation that there actually are languages in which two elements surface in so-called double-headed relative clauses. In these languages, the external head is a subset of the internal head, and that some features like \tsc{d} and case are necessarily excluded in the external head. I adopt this insight, and I apply it to the headless relative situation. I propose that the external head in headless relatives is a copy of a specific part of the relative pronoun.
%
% As I said earlier, I need two elements to do case competition with. In headless relatives, I only see a single one surfacing. However, some languages actually show two elements surfacing. Here there are two copies of the element, one inside the relative clause, one outside of the relative clause.
%
% \exg. [\tbf{doü} adiyan-o-no] \tbf{doü} deyalukhe\\
%  sago give.3\tsc{pl}.\tsc{nonfut}-{tr}-\tsc{conn} sago finished.\tsc{adj}\\
%  `The sago that they gave is finished.' \flushfill{Kombai, \pgcitealt{vries1993}{78}}
%
% The external element is not always an exact copy of the element inside of the relative clause. An example from Kombai shows that the element outside of the relative clause can also be a subset of what the element inside of the relative clause is. Here I give two examples, there is an \tit{old man} and a \tit{person}, and there is \tit{pig} and a \tit{thing}.
%
% \ex.
% \ag. [\tbf{yare} gamo khereja bogi-n-o] \tbf{rumu} na-momof-a\\
%  {old man} join.\ac{ss} work do.\ac{dur}.3\ac{sg}.\ac{nf}-\ac{tr}-\ac{conn} person my-uncle-\ac{pred}\\
%  `The old man, who is joining the work, is my uncle.' 77
% \bg. [\tbf{ai} fali-khano] \tbf{ro} nagu-n-ay-a.\\
%  pig carry-go.3\tsc{pl}.\tsc{nf} thing our-\tsc{tr}-pig-\ac{pred}\\
%  `The pig they took away, is ours.' \flushfill{Kombai, \pgcitealt{vries1993}{77}}
%
% Let me now apply what we have seen so far to headless relatives. Headless relatives do not have an overt NP, so this cannot be copied. However, there is the relative pronoun which is specified for number, gender, case, etc. Are all of these features copied onto the external element? The copy is the portion of the nominal extended projection c-commanded by the relative clause. A headless relative is a restrictive relative clause. Therefore, there is no \tsc{d} and no case.
%
% Is it possible to add features onto the external head after it has been copied? Yes, for example D, as the example shows, but also case.
%
% \exg. Junya-wa [Ayaka-ga \tbf{ringo}-o mui-ta] sono \tbf{ringo}-o tabe-ta.\\
% Junya-\ac{top} Ayaka-\ac{nom} apple-\ac{acc} peel-\ac{pst} that apple-\ac{acc} eat-\ac{pst}\\
% ‘Junya ate the apples that Ayaka peeled.’ \flushfill{Japanese, \pgcitealt{erlewine2016}{2}}
%
% In sum, the external element is a copy of a subset of the features of the relative pronoun. Definiteness and case are not copied. New features can be merged onto the external element.


According to Cinque, every type of relative clause in every language is underlyingly double-headed. Evidence for this claim comes from languages that show this morphologically. An example from Kombai is given in \ref{ex:kombai}. The head of the relative clause is \tit{doü} `sago', and it appears inside the relative clause and outside.

\exg. [\tbf{doü} adiyan-o-no] \tbf{doü} deyalukhe\\
 sago give.3\tsc{pl}.\tsc{nonfut}-{tr}-\tsc{conn} sago finished.\tsc{adj}\\
 `The sago that they gave is finished.' \flushfill{Kombai, \pgcitealt{vries1993}{78}}\label{ex:kombai}

The internal and external instances of \tit{doü} correspond to the internal and external element I assume to be there in the headless relatives.

\ref{ex:double-syntax} shows the syntactic structure of the sentence in \ref{ex:kombai}.

\ex.
\begin{forest} boom
[CP
   [FP
      [CP
          [\tsc{int}
             [\tit{doü}, roof]
          ]
          [CP
              [\tit{adiyan-o-no}, roof]
          ]
      ]
      [\tsc{ext}
         [\tit{doü}, roof]
      ]
   ]
   [VP
      [\tit{deyalukhe}, roof]
   ]
]
\end{forest}\label{ex:double-syntax}

In most languages one of the two heads is deleted throughout the derivation.

According to \citealt{cinqueforthcoming}, the internal element can delete the external element, because the internal element c-commands the external element. This is c-command according to Kayne's definition of it: the internal element is in the specifier of the specifier of the FP.

\ex.
\begin{forest} boom
[
   [CP
       [\tsc{int}
          [\phantom{xxx}, roof]
       ]
       [CP
           [\phantom{xxx}, roof]
       ]
   ]
   [\tsc{ext}
      [\phantom{xxx}, roof]
   ]
]
\end{forest}\label{ex:cinque-int-wins}

In order for the internal element to be able to delete the external element, a movement needs to take place. The external element moves over the relative clause.\footnote{
What remains unclear is what the trigger is for the movement of the external element over relative clause is.
}
From this position, the external element can delete the internal one, because the external element c-commands the internal one.

\ex.
\begin{forest} boom
[
    [\tsc{ext}
       [\phantom{xxx}, roof]
    ]
    [FP
       [CP
           [\tsc{int}
              [\phantom{xxx}, roof]
           ]
           [CP
               [\phantom{xxx}, roof]
           ]
       ]
       [\tit{t\scsub{ext}}]
    ]
]
\end{forest}

Also talk about \tsc{d} here, and that maybe Old High German deletes a thing without a \tsc{d} when the internal thing wins. does that also have a not so definite interpretation?


What does not work:

For this pattern a single element analysis seems intuitive, if you assume that case is complex and that syntax works bottom-up. First you built the relative clause, with the big case in there. Then you build the main clause and you let the more complex case in the embedded clause license the main clause predicate.

Consider the example in \ref{ex:mg-nom-acc-grafting}. Here the internal case is accusative and the external one nominative.

\exg. Uns besucht \tbf{wen} \tbf{Maria} \tbf{mag}.\\
 we.\ac{acc} visit.3\ac{sg}\scsub{[nom]} \tsc{rel}.\ac{acc}.\tsc{an} Maria.\ac{nom} like.3\ac{sg}\scsub{[acc]}\\
 `Who visits us, Maria likes.' \flushfill{adapted from \pgcitealt{vogel2001}{343}}\label{ex:mg-nom-acc-grafting}

The relative clause is built, including the accusative relative pronoun. Now the main clause predicate can merge with the nominative that is contained within the accusative.

 \ex.
 \begin{forest} boom
  [,name=src, s sep=15mm
   [VP
      [\tit{besucht}, roof]
   ]
    [,no edge, s sep=20mm
        [\ac{acc}P,
     tikz={
     \node[label=below:\tit{wen},
     draw,circle,
     scale=0.85,
     fit to=tree]{};
     }
            [\tsc{f2}]
            [\tsc{nomP},name=tgt
                [\tsc{f1}]
                [XP
                    [\phantom{xxx}, roof]
                ]
            ]
        ]
     [VP
        [\tit{Maria mag}, roof]
     ]
   ]
  ]
  \draw (src) to[out=south east,in=north east] (tgt);
 \end{forest}\label{ex:acc-nom-grafting}

The other way around does not work. Consider \ref{ex:mg-acc-nom-grafting}. This is an example with nominative as internal case and accusative as external case.

\exg. *Ich {lade ein}, wen \tbf{mir} \tbf{sympathisch} \tbf{ist}.\\
I.\ac{nom} invite.1\ac{sg}\scsub{[acc]} \tsc{rel}.\ac{acc}.\tsc{an} I.\ac{dat} nice be.3\ac{sg}\scsub{[nom]}\\
`I invite who I like.' \flushfill{adapted from \pgcitealt{vogel2001}{344}}\label{ex:mg-acc-nom-grafting}

Now the relative clause is built first again, this time only including the nominative case. There is no accusative node to merge with for the external predicate. Instead, the relative pronoun would need to grow to accusative somehow and then the merge could take place. This is the desired result, because the sentence is ungrammatical.

\ex.
\begin{forest} boom
  [,name=src, s sep=15mm
     [VP
         [\tit{lade ein}, roof]
     ]
         [,no edge
       [\tsc{nomP},
       tikz={
       \node[label=below:\tit{wer},
       draw,circle,
       scale=0.85,
       fit to=tree]{};
       }
         [\tsc{f1}]
         [XP
           [\phantom{xxx}, roof]
         ]
       ]
       [VP
         [\tit{mir sympatisch ist}, roof]
       ]
      ]
    ]
\end{forest}\label{ex:nom-acc-grafting}

So, this seems to work fine. The assumptions you have to do in order to make this are the following. First, case is complex. Second, you can remerge an embedded node (grafting). For the first one I have argued in Chapter \ref{ch:decomposition}. The second one could use some additional argumentation. It is a mix between internal remerge (move) and external merge, namely external remerge. Other literature on multidominance and grafting, other phenomena. Problems: linearization, .. But even if fix all these theoretical problems, there is an empirical one.

That is, I want to connect this behavior of Modern German headless relatives to the shape of its relative pronouns. These pronouns are \tsc{wh}-elements. The OHG and Gothic ones are not \tsc{wh}, they are \tsc{d}. Their relative pronouns look different, and so their headless relatives can also behave differently.




Himmelreich

there are agree relations between
- V\scsub{ext} and \tsc{ext}
- V\scsub{int} and \tsc{int}
- \tsc{int} and \tsc{ext}

three parameters:
1 relation between V\scsub{ext} and \tsc{ext} + V\scsub{int} and \tsc{int} are symmetric or asymmetric
2 relation between \tsc{ext} and \tsc{int} are symmetric or asymmetric
3 if \tsc{ext} --- \tsc{int} is asymmetric, \tsc{ext} or \tsc{int} probes

I keep the parameters she has stable, the bigger syntactic context is the same everywhere. I vary the content of \tsc{ext}
