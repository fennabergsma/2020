% !TEX root = thesis.tex

\chapter{Base comparison}\label{ch:relativization}

In Chapter \ref{ch:case-competition-typology} I introduced two descriptive parameters that generate the attested languages, as shown in Figure \ref{fig:two-parameters}.
The first parameter concerns whether the external case is allowed to surface when it wins the case competition (allow \tsc{ext}?). This parameter distinguishes between non-matching languages (e.g. Old High German) on the one hand and internal-only languages (e.g. Modern German) and matching languages (e.g. Polish) on the other hand.
The second parameter concerns whether the internal case is allowed to surface when it wins the case competition (allow \tsc{int?}). This parameter distinguishes between internal-only languages (e.g. as Modern German) on the one hand and non-matching languages (e.g. Polish) on the other hand.

\begin{figure}[H]
  \centering
    \footnotesize{
    \begin{tikzpicture}[node distance=1.5cm]
      \node (question2) [question]
      {allow \tsc{ext}?};
          \node (outcome2) [outcome, below of=question2, xshift=-1.5cm]
          {non-matching};
              \node (example2) [example, below of=outcome2, yshift=0.25cm]
              {\scriptsize{e.g. Gothic, Old High German, Classical Greek}};
          \node (question3) [question, below of=question2, xshift=2cm, yshift=-0.5cm]
          {allow \tsc{int}?};
              \node (outcome3) [outcome, below of=question3, xshift=-1.5cm]
              {internal-only};
                  \node (example3) [example, below of=outcome3, yshift=0.25cm]
                  {\scriptsize{e.g. Modern German\\\phantom{x}}};
              \node (outcome4) [outcome, below of=question3, xshift=1.5cm]
              {matching};
                  \node (example4) [example, below of=outcome4, yshift=0.25cm]
                  {\scriptsize{e.g. Polish\\\phantom{x}}};

    \draw [arrow] (question2) -- node[anchor=east] {yes} (outcome2);
    \draw [arrow] (question2) -- node[anchor=west] {no} (question3);
    \draw [arrow] (question3) -- node[anchor=east] {yes} (outcome3);
    \draw [arrow] (question3) -- node[anchor=west] {no} (outcome4);
    \end{tikzpicture}
    }
    \caption{Two descriptive parameters generate three language types}
    \label{fig:two-parameters}
\end{figure}

The goal of this chapter is to give the theoretical counterparts of these descriptive parameters. Goal, not language-specific, but something that can be observed independently.

This chapter is structured as follows.


\section{The basic idea}

The goal of this chapter is to explain why languages differ in whether they allow the internal case and the external case to surface when either of them wins the case competition. In other words, the chapter gives the theoretical counterparts of the descriptive parameters given in Figure \ref{fig:two-parameters}. Before I can describe these theoretical counterparts, I need to introduce some concepts: the internal element, the external element and the base of these elements.

I start with the internal and external element. In the thesis so far, I stated that the relative pronoun appears in the case that wins the case competition. I have not been explicit about where the case competition takes place. In order to avoid introducing theoretical machinery just for case competition situations, I assume it takes place in syntax. I propose that at some point in the derivation headless relatives have an internal and an external element.\footnote{
I am far from the only one that assumes this. Himmelreich, Hanink, but also Bresnan/Grimshaw, Groos/Riemsdijk, Harbert..
} The internal element bears the internal case, and the external element bears the external case. At the end of the derivation, the element bearing the more complex case surfaces as the relative pronoun, if it is allowed to.

Now I turn to the so-called base of the internal and external element. The internal and the external element do not only consist of case features. They also contain other features. These other features have to do with referentiality, number, gender, uniqueness and definiteness. I call the part of the internal and external element that corresponds to these features the base part. I refer to the part that corresponds to the case features as the case part.

Table \ref{tbl:component-elements} summarizes what I just laid out. At some point in the derivation, headless relatives contain an internal and an external element. The internal element consists of an internal base part and an internal case part. The external element also consists of an external base part and an external case part. The internal and the external base are the main focus of this chapter.

\begin{table}[H]
  \center
  \caption{Components of the internal and external element}
\begin{tabular}{cccc}
  \toprule
\multicolumn{2}{c}{\tsc{int} element}  & \multicolumn{2}{c}{\tsc{ext} element} \\
\cmidrule(lr){1-2}                        \cmidrule(lr){3-4}
base\scsub{int} & case\scsub{int}       & base\scsub{ext} & case\scsub{ext}     \\
\bottomrule
\end{tabular}
\label{tbl:component-elements}
\end{table}

To make this concrete, consider the example in \ref{ex:ohg-nom-nom-rep}, repeated from Chapter \ref{ch:case-competition-typology}. In this example, the internal nominative case competes against the external nominative case. The relative pronoun surfaces in the nominative case.

\exg. quham dher chisendit scolda uuerdhan\\
 come.\ac{pst}.3\ac{sg}\scsub{[nom]} \ac{rel}.\ac{sg}.\ac{m}.\ac{nom} send.\ac{pst}.\ac{ptcp}\scsub{[nom]} should.\ac{pst}.3\ac{sg} become.\ac{inf}\\
 `the one, who should have been sent, came' \flushfill{Old High German, \ac{isid} 35:5}\label{ex:ohg-nom-nom-rep}

The relative pronoun in this sentence is \tit{dher} `\ac{rel}.\ac{sg}.\ac{m}.\ac{nom}'. In my proposal, the internal element for this sentence is \tit{dher} and the external element is \tit{dher} too. The base part of \tit{dher} is the morpheme \tit{dhe}, and the case part of \tit{dher} is the morpheme \tit{r}. This is schematically shown in Table \ref{tbl:component-dhen}.
In Section \ref{sec:deriving-non-matching} I motivate the featural content and phonological form of the internal and external element.\footnote{
This is a simplification of the reality. The morpheme \tit{r} realizes besides case features also non-case features, such as gender and number. This simplification can be made, because the non-case features that are present in the internal element are also present in the external element.
}

\begin{table}[H]
  \center
  \caption{The internal and external element of \ref{ex:ohg-nom-nom-rep}}
\begin{tabular}{cccc}
  \toprule
\multicolumn{2}{c}{\tsc{int} element}  & \multicolumn{2}{c}{\tsc{ext} element} \\
\cmidrule(lr){1-2}                        \cmidrule(lr){3-4}
base\scsub{int} & case\scsub{int}       & base\scsub{ext} & case\scsub{ext}     \\
\cmidrule(lr){1-1}  \cmidrule(lr){2-2}    \cmidrule(lr){3-3}  \cmidrule(lr){4-4}
dhe & r                                 & dhe & r                               \\
\bottomrule
\end{tabular}
\label{tbl:component-dhen}
\end{table}

Now I have introduced the concepts internal and external element and their bases, I turn to the basic idea behind my proposal. The goal is to give a theoretical explanation for why a language allows the internal or external case to surface when it wins the case competition or why it does not. I propose that this follows from the comparison between the internal and external base within a language.
In the comparison, I rely on containment, just as I did in Chapter \ref{ch:decomposition} when comparing cases. I went with the following reasoning. A more complex case wins over a less complex case because the former contains all features that the latter contains. Concretely, the dative wins over the accusative because the dative contains all features that the accusative contains, the dative wins over the nominative because the dative contains all features that the nominative contains, and the accusative wins over the nominative because the accusative contains all features that the nominative contains.
I apply the same reasoning in comparing the internal and external base. When the internal base contains the features that the external base contains, the internal element is allowed to surface. When the external base contains the features that the internal base contains, the external element is allowed to surface.

I illustrate this proposal by showing how this plays out in three different languages.
In Old High German, the internal base contains the external base, and the external base contains the internal base. Therefore, Old High German allows for matching cases, it allows the internal to win, and it allows the external to win.
In Modern German, the internal base contains the external base, but the external base does not contain the internal base. Therefore, Modern German allows for matching cases, and it allows the internal to win.
In Polish, the internal base does not contain the external base, and the external base does not contain the internal base. Therefore, Polish only allows for matching cases.

\begin{table}[H]
  \center
  \caption{Overview of languages}
\begin{tabular}{cccc}
  \toprule
                & \tsc{int} = \tsc{ext} & \tsc{int} > \tsc{ext} & \tsc{int} < \tsc{ext} \\
                      \cmidrule{1-4}
Old High German & ✔                     & ✔                     & ✔                   \\
Modern German   & ✔                     & ✔                     & *                   \\
Polish          & ✔                     & *                     & *                   \\
\bottomrule
\end{tabular}
\end{table}





Consider Old High German. First there is a tie, \tsc{int} = \tsc{ext}, and everything is grammatical.
If \tsc{int} > \tsc{ext}, we have a competition and a winner: internal \tsc{acc}. The bases are compared. (How to get from forms to features?) They are identical, so the internal base contains all features the external base contains.
If \tsc{int} > \tsc{ext}, we have a competition and a winner: external \tsc{acc}. The bases are compared. They are identical, so the external base contains all features the internal base contains.

\begin{table}[H]
  \center
  \caption{Base comparison in Old High German}
\begin{tabular}{ccccccc}
  \toprule
                      & \multicolumn{2}{c}{\tsc{int} element}  & \multicolumn{2}{c}{\tsc{ext} element}  & \multicolumn{2}{c}{\tsc{rel} pronoun} \\
                        \cmidrule(lr){2-3}                        \cmidrule(lr){4-5}                      \cmidrule(lr){6-7}
                      & base\scsub{int} & case\scsub{int}       & base\scsub{ext} & case\scsub{ext}     & base\scsub{rel} & case\scsub{rel} \\
                        \cmidrule(lr){2-2}    \cmidrule(lr){3-3}  \cmidrule(lr){4-4} \cmidrule(lr){5-5}   \cmidrule(lr){6-6} \cmidrule(lr){7-7}
\tsc{int} = \tsc{ext} & dhe & r                                 & dhe & r                               & dhe & r                           \\
\tsc{int} > \tsc{ext} & dhe & n                                 & dhe & r                               & dhe & n                           \\
\tsc{int} < \tsc{ext} & dhe & r                                 & dhe & n                               & dhe & n                           \\
\bottomrule
\end{tabular}
\label{tbl:component-dhen}
\end{table}

The first row shows the situation in which internal case matches the internal case (\tsc{int} = \tsc{ext}). The relative pronoun surfaces in the 

The second row shows the situation in which internal case is more complex than the external case (\tsc{int} > \tsc{ext}). The internal case part, which corresponds to the accusative \tit{n} morpheme, contains the external case part, which corresponds to the nominative \tit{r} morpheme. The internal base part, the features that corresponds to the morpheme \tit{we}, contains the external base part, because there are no external base features. The element that surfaces is the internal element with the internal base and the internal case.

The third row shows the situation in which external case is more complex than the internal case (\tsc{int} < \tsc{ext}). The external case part, which corresponds to the accusative \tit{n} morpheme, contains the internal case part, which corresponds to the nominative \tit{r} morpheme. The external base part, which does not contain any features, does not contain the internal case part, the features that corresponds to the morpheme \tit{we}. The winner of the case competition is the external case, but the external base does not contain the internal base, so there is no grammatical relative pronoun.

\begin{table}[H]
  \center
  \caption{Base comparison in Modern German}
\begin{tabular}{ccccccc}
  \toprule
                      & \multicolumn{2}{c}{\tsc{int} element}  & \multicolumn{2}{c}{\tsc{ext} element}  & \multicolumn{2}{c}{\tsc{rel} pronoun} \\
                        \cmidrule(lr){2-3}                        \cmidrule(lr){4-5}                      \cmidrule(lr){6-7}
                      & base\scsub{int} & case\scsub{int}       & base\scsub{ext} & case\scsub{ext}     & base\scsub{rel} & case\scsub{rel} \\
                        \cmidrule(lr){2-2}    \cmidrule(lr){3-3}  \cmidrule(lr){4-4} \cmidrule(lr){5-5}   \cmidrule(lr){6-6} \cmidrule(lr){7-7}
\tsc{int} = \tsc{ext} & we & r                                  &  & r                                  & we & r                           \\
\tsc{int} > \tsc{ext} & we & n                                  &  & r                                  & we & n                           \\
\tsc{int} < \tsc{ext} & we & r                                  &  & n                                  & \multicolumn{2}{c}{\tsc{*}}      \\
\bottomrule
\end{tabular}
\label{tbl:component-dhen}
\end{table}



\begin{table}[H]
  \center
  \caption{Base comparison in Polish}
\begin{tabular}{ccccccc}
  \toprule
                      & \multicolumn{2}{c}{\tsc{int} element}  & \multicolumn{2}{c}{\tsc{ext} element}  & \multicolumn{2}{c}{\tsc{rel} pronoun} \\
                        \cmidrule(lr){2-3}                        \cmidrule(lr){4-5}                      \cmidrule(lr){6-7}
                      & base\scsub{int} & case\scsub{int}       & base\scsub{ext} & case\scsub{ext}     & base\scsub{rel} & case\scsub{rel} \\
                        \cmidrule(lr){2-2}    \cmidrule(lr){3-3}  \cmidrule(lr){4-4} \cmidrule(lr){5-5}   \cmidrule(lr){6-6} \cmidrule(lr){7-7}
\tsc{int} = \tsc{ext} & ko & go                                  & te & go                              & ko & go                           \\
\tsc{int} > \tsc{ext} & ko & mu                                  & te & go                              & \multicolumn{2}{c}{\tsc{*}}       \\
\tsc{int} < \tsc{ext} & ko & go                                  & te & mu                              & \multicolumn{2}{c}{\tsc{*}}       \\
\bottomrule
\end{tabular}
\label{tbl:component-dhen}
\end{table}

Taking this all together, there are two competitions taking place in headless relatives: case competition and base competition. Case competition determines which case wins and base competition determines whether this case is allowed to surface.
I put this in the metaphor with the committee that I introduced in Section \ref{sec:possible-patterns}. The committee learns who wins the case competition, and it can either approve this case or not approve it. The information that the committee uses for their decision is the comparison between the internal and the external base. The committee approves the winning case if the base associated with the winning case contains the base associated with the losing case.

In sum, a relative pronoun can surface in the internal case if the internal case and base contains the external case and base.
It works the same the other way around: a relative pronoun can surface in the external case if the external case and base contain the internal case and base.
Notice that there is a crucial difference between comparing the cases and comparing the bases. The internal and external case can differ from sentence to sentence. The internal and external base remain stable throughout the language.



\section{Deriving `non-matching'}\label{deriving non-matching}

I am not going to give a whole sentence. I only talk about the internal and external element in there. I also do not talk about the nature of the deletion process.

Old High German allows both internal and external to win. I propose headless relatives are derived from light-headed relatives. Internal element is relative pronoun, external element is demonstrative.

\exg. eno nist thiz thér then ir suochet zi arslahanne?\\
 now {not be.3\ac{sg}} \tsc{dem}.\tsc{sg}.\tsc{n}.\tsc{nom} \tsc{dem}.\tsc{sg}.\tsc{m}.\tsc{nom}
 \tsc{dem}.\tsc{sg}.\tsc{m}.\tsc{acc} 2\ac{pl}.\tsc{nom} seek.2\tsc{pl} to kill.\tsc{inf}.\ac{sg}.\tsc{dat}\\
 `Isn't this now the one, who you seek to kill?'




\ex.
\ag. Thíz ist \tbf{then} \tbf{sie} \tbf{zéllent}\\
\ac{dem}.\ac{sg}.\ac{n}.\ac{nom} be.\ac{pres}.3\ac{sg}\scsub{[nom]} \ac{rel}.\ac{sg}.\ac{m}.\ac{acc} 3\ac{pl}.\ac{m}.\ac{nom} tell.\ac{pres}.3\ac{pl}\scsub{[acc]}\\
`this is the one whom they talk about' \flushfill{Old High German, \ac{otfrid} III 16:50}\label{ex:ohg-nom-acc-intro-rep}
\bg. ih bibringu fona iacobes samin endi fona iuda dhen \tbf{mina} \tbf{berga} \tbf{chisitzit}\\
1\ac{sg}.\ac{nom} {create}.\ac{pres}.1\ac{sg}\scsub{[acc]} of Jakob.\ac{gen} seed.\ac{sg}.\ac{dat} and of Judah.\ac{dat}
\ac{rel}.\ac{sg}.\ac{m}.\ac{acc} my.\ac{acc}.\ac{m}.\ac{pl} mountain.\ac{acc}.\ac{pl} possess.\ac{pres}.3\ac{sg}\scsub{[nom]}\\
`I create of the seed of Jacob and of Judah the one, who possess my mountains' \flushfill{Old High German, \ac{isid} 34:3}\label{ex:ohg-acc-nom-intro-rep}

\subsection{Relative pronoun}

I give an overview of Old High German relative pronouns. I only give the neuter and masculine gender, because I do not have any examples with the feminine. Relative pronouns consist of three morphemes: a \tit{d}, a vowel (\tit{a}, \tit{e} or \tit{i}) and suffix that differs per number, gender and case.\footnote{
\tit{d} can also be \tit{dh} and \tit{th}, \tit{ë} and \tit{ē} can also be \tit{e} and \tit{é}.
}


\begin{table}[H]\label{tbl:paradigmohg}
 \center
 \caption {Relative pronouns in headless relatives in Old High German \pgcitealt{braune2018}{339}}
  \begin{tabular}{ccc}
  \toprule
              & \ac{n}.\ac{sg}  & \ac{m}.\ac{sg} \\
        \cmidrule{2-3}
    \ac{nom}  & d-a-ȥ           & d-ë-r          \\
    \ac{acc}  & d-a-ȥ           & d-ë-n          \\
    \ac{dat}  & d-ë-mo          & d-ë-mo         \\
    \bottomrule
  \end{tabular}
\end{table}




The \tit{d} morpheme corresponds to \tsc{def}.

The ending corresponds to number, gender and case: \tsc{ref}, \tsc{class}, \tsc{masc}, \tsc{ind}, \tsc{group} and \tsc{f1}, \tsc{f2}, \tsc{f3}. Illustrate this with nouns and adjectives.

The vowel corresponds to anaphoricity.
Do they also not appear in adjectives? Is there any anaphoricity?

Now I can specify the lexical entries.

The case etc. morpheme has a binary bottom, because it does not always surface as a suffix (at least in Modern German).

\exg. Ich habe 's Fahrrad vergessen.\\
 I have the\scsub{weak} bike forgotten\\
 `I forgot the bike.'

give nominative singular masculine r
give nominative singular masculine n

give anaphoricity e

give definiteness D How does this relate to being a relative pronoun?

I illustrate how this relative pronoun is built, using the spellout algorithm.

\ex. \tbf{Spellout Algorithm:}\\
Merge F and \label{ex:spellout}
 \a. Spell out FP.
 \b. If (a) fails, attempt movement of the spec of the complement of \tsc{f}, and retry (a).
 \b. If (b) fails, move the complement of \tsc{f}, and retry (a).

When a new match is found, it overrides previous spellouts.

\ex. \tbf{Cyclic Override} \citep{starke2018}:\\
Lexicalisation at a node XP overrides any previous match at a phrase contained in XP.

build build, until we reach anaphoricity

A specifier is constructed.

\ex. \tbf{Spec Formation} \citep{starke2018}:\\
If Merge F has failed to spell out (even after backtracking), try to spawn a new derivation providing the feature F and merge that with the current derivation, projecting the feature F at the top node.\label{ex:specformation}

\ex. Merge F, Move XP, Merge XP

how do I get case features onto the portmanteau? backtracking, open the three things, and see where I can spell it out

If the spellout procedure in \ref{ex:spellout} fails, backtracking takes place.

\ex. \tbf{Backtracking} \citep{starke2018}:\\
When spellout fails, go back to the previous cycle, and try the next option for that cycle.\label{ex:backtracking}

give relative pronoun here

\begin{forest} boom
  [DP
      [DP
          [\tit{d}, roof]
      ]
      [\tsc{ana}P, s sep=20mm
          [\tsc{ana}P
              [\tit{e}, roof]
          ]
          [\tsc{acc}P,
          tikz={
          \node[label=below:\tit{n},
          draw,circle,
          scale=0.85,
          fit to=tree]{};
          }
              [\tsc{f}2]
              [\tsc{nom}P
                  [\tsc{f1}]
                  [\tsc{ind}P
                      [\phantom{xxx},
                      roof, baseline
                      ]
                  ]
              ]
          ]
      ]
  ]
\end{forest}

\begin{forest} boom
  [DP
      [DP
          [\tit{d}, roof]
      ]
      [\tsc{ana}P, s sep=20mm
          [\tsc{ana}P
              [\tit{e}, roof]
          ]
          [\tsc{nom}P,
          tikz={
          \node[label=below:\tit{r},
          draw,circle,
          scale=0.85,
          fit to=tree]{};
          }
              [\tsc{f1}]
              [\tsc{ind}P
                  [\phantom{xxx},
                  roof, baseline
                  ]
              ]
          ]
      ]
  ]
\end{forest}


\subsection{External element}

I copy the \tsc{ind}P, and then I merge the external \tsc{ana}, \tsc{d} and cases.

\subsubsection{Comparison}

\exg. Thíz ist \tbf{then} \tbf{sie} \tbf{zéllent}\\
\ac{dem}.\ac{sg}.\ac{n}.\ac{nom} be.\ac{pres}.3\ac{sg}\scsub{[nom]} \ac{rel}.\ac{sg}.\ac{m}.\ac{acc} 3\ac{pl}.\ac{m}.\ac{nom} tell.\ac{pres}.3\ac{pl}\scsub{[acc]}\\
`this is the one whom they talk about' \flushfill{Old High German, \ac{otfrid} III 16:50}\label{ex:ohg-nom-acc-rep}

There are three independent containment relations. Problem?

\begin{table}[H]
  \center
 \caption {Old High German: \ac{int} > \ac{ext}}
  \begin{tabular}[b]{cc}
      \toprule
      \ac{int}  &   \ac{ext} \\ \cmidrule{1-2}
      \footnotesize{
      \begin{forest} boom
        [DP
            [DP,
              tikz={
              \node[draw,circle,
              fill=DG,fill opacity=0.2,
              scale=0.75,
              DG,dashed,
              fit to=tree]{};
              }
                [\tit{d}, roof]
            ]
            [\tsc{ana}P, s sep=20mm
                [\tsc{ana}P,
                tikz={
                \node[draw,circle,
                fill=DG,fill opacity=0.2,
                scale=0.75,
                DG,dashed,
                fit to=tree]{};
                }
                    [\tit{e}, roof]
                ]
                [\tsc{acc}P,
                tikz={
                \node[label=below:\tit{n},
                draw,circle,
                scale=0.85,
                fit to=tree]{};
                }
                    [\tsc{f}2]
                    [\tsc{nom}P,
                      tikz={
                      \node[draw,circle,
                      fill=DG,fill opacity=0.2,
                      DG,dashed,
                      scale=0.8,
                      fit to=tree]{};
                      }
                        [\tsc{f1}]
                        [\tsc{ind}P
                            [\phantom{xxx},
                            roof, baseline
                            ]
                        ]
                    ]
                ]
            ]
        ]
      \end{forest}
      }
      &
      \footnotesize{
      \begin{forest} boom
        [\textcolor{DG}{DP}
            [\textcolor{DG}{DP},edge=DG,
            tikz={
            \node[draw,circle,
            scale=0.75,
            DG,dashed,
            fit to=tree]{};
            }
                [\textcolor{DG}{\tit{d}}, roof, edge=DG]
            ]
            [\textcolor{DG}{\tsc{ana}P},edge=DG, s sep=20mm
                [\textcolor{DG}{\tsc{ana}P},edge=DG,
                tikz={
                \node[draw,circle,
                scale=0.75,
                DG,dashed,
                fit to=tree]{};
                }
                    [\textcolor{DG}{\tit{e}}, roof, edge=DG]
                ]
                [\textcolor{DG}{\ac{nom}},edge=DG,
                tikz={
                \node[label=below:\textcolor{DG}{\tit{r}},
                draw,circle,
                scale=0.75,
                DG,
                fit to=tree]{};
                \node[
                draw,circle,
                scale=0.8,
                dashed,DG,
                fit to=tree]{};
                }
                    [\textcolor{DG}{\tsc{f1}},baseline,edge=DG]
                    [\textcolor{DG}{\tsc{ind}P},edge=DG
                        [\phantom{xxx},
                        roof, baseline, edge=DG
                        ]
                    ]
                ]
            ]
        ]
      \end{forest}
      }\\
      \bottomrule
  \end{tabular}
  \label{tbl:ohg-int-wins}
\end{table}

\exg. ih bibringu fona iacobes samin endi fona iuda dhen \tbf{mina} \tbf{berga} \tbf{chisitzit}\\
1\ac{sg}.\ac{nom} {create}.\ac{pres}.1\ac{sg}\scsub{[acc]} of Jakob.\ac{gen} seed.\ac{sg}.\ac{dat} and of Judah.\ac{dat} \ac{rel}.\ac{sg}.\ac{m}.\ac{acc} my.\ac{acc}.\ac{m}.\ac{pl} mountain.\ac{acc}.\ac{pl} possess.\ac{pres}.3\ac{sg}\scsub{[nom]}\\
`I create of the seed of Jacob and of Judah the one, who possess my mountains' \flushfill{Old High German, \ac{isid} 34:3}\label{ex:ohg-acc-nom-rep}

\begin{table}[H]
  \center
 \caption {Old High German: \ac{int} < \ac{ext}}
  \begin{tabular}[b]{cc}
      \toprule
      \ac{int}  &   \ac{ext} \\ \cmidrule{1-2}
      \footnotesize{
      \begin{forest} boom
        [\textcolor{DG}{DP}
            [\textcolor{DG}{DP},edge=DG,
            tikz={
            \node[draw,circle,
            scale=0.75,
            DG,dashed,
            fit to=tree]{};
            }
                [\textcolor{DG}{\tit{d}}, roof, edge=DG]
            ]
            [\textcolor{DG}{\tsc{ana}P},edge=DG
                [\textcolor{DG}{\tsc{ana}P},edge=DG,
                tikz={
                \node[draw,circle,
                scale=0.75,
                DG,dashed,
                fit to=tree]{};
                }
                    [\textcolor{DG}{\tit{e}}, roof, edge=DG]
                ]
                [\textcolor{DG}{\ac{nom}},edge=DG,
                tikz={
                \node[label=below:\textcolor{DG}{\tit{r}},
                draw,circle,
                scale=0.75,
                DG,
                fit to=tree]{};
                \node[
                draw,circle,
                scale=0.8,
                dashed,DG,
                fit to=tree]{};
                }
                    [\textcolor{DG}{\tsc{f1}},baseline,edge=DG]
                    [\textcolor{DG}{\tsc{ind}P},edge=DG
                        [\phantom{xxx},
                        roof, baseline, edge=DG
                        ]
                    ]
                ]
            ]
        ]
      \end{forest}
      }
      &
      \footnotesize{
      \begin{forest} boom
        [DP
            [DP,
              tikz={
              \node[draw,circle,
              fill=DG,fill opacity=0.2,
              scale=0.75,
              DG,dashed,
              fit to=tree]{};
              }
                [\tit{d}, roof]
            ]
            [\tsc{ana}P
                [\tsc{ana}P,
                tikz={
                \node[draw,circle,
                fill=DG,fill opacity=0.2,
                scale=0.75,
                DG,dashed,
                fit to=tree]{};
                }
                    [\tit{e}, roof]
                ]
                [\tsc{acc}P,
                tikz={
                \node[label=below:\tit{n},
                draw,circle,
                scale=0.85,
                fit to=tree]{};
                }
                    [\tsc{f}2]
                    [\tsc{nom}P,
                      tikz={
                      \node[draw,circle,
                      fill=DG,fill opacity=0.2,
                      DG,dashed,
                      scale=0.8,
                      fit to=tree]{};
                      }
                        [\tsc{f1}]
                        [\tsc{ind}P
                            [\phantom{xxx},
                            roof, baseline
                            ]
                        ]
                    ]
                ]
            ]
        ]
      \end{forest} }\\
      \bottomrule
  \end{tabular}
  \label{tbl:ohg-ext-wins}
\end{table}



\section{Deriving `non-matching --- only internal'}

Only internal wins, external cannot. I illustrate this with nominative and accusative.

\ex.
\ag. Uns besucht, \tbf{wen} \tbf{Maria} \tbf{mag}.\\
 2\ac{pl}.\ac{acc} visit.\ac{pres}.3\ac{sg}\scsub{[nom]} \ac{rel}.\ac{an}.\ac{acc} Maria.\ac{nom} like.\ac{pres}.3\ac{sg}\scsub{[acc]}\\
 `Who visits us, Maria likes.' \flushfill{Modern German, adapted from \pgcitealt{vogel2001}{343}}\label{ex:mg-nom-acc-rep-intro}
\bg. *Ich {lade ein}, wen \tbf{mir} \tbf{sympathisch} \tbf{ist}.\\
 1\ac{sg}.\ac{nom} invite.\ac{pres}.1\ac{sg}\scsub{[acc]} \ac{rel}.\ac{an}.\ac{acc} 1\ac{sg}.\ac{dat} nice be.\ac{pres}.3\ac{sg}\scsub{[nom]}\\
 `I invite who I like.' \flushfill{Modern German, adapted from \pgcitealt{vogel2001}{344}}\label{ex:mg-acc-nom-rep-intro}

 \subsection{The relative pronoun = internal}\label{sec:internal-element}

 In headless relative constructions, there is a single element that surfaces: the relative pronoun. In this section, I show that the relative pronoun is syntactically part of the relative clause. The evidence comes from extraposition data in Modern German. In Modern German, it is possible to extrapose a CP (a clause), but not a DP (a noun phrase). In this section I first show that Modern German CPs can be extraposed and DPs cannot. Then I illustrate how relative clauses including the relative pronoun in headless relatives pattern with CPs: they can be extraposed as well. I conclude that the relative pronoun is the internal element in the headless relative.

 The sentences in \ref{ex:mg-extrapose-cp} show that it is possible to extrapose a CP. In \ref{ex:mg-extrapose-cp-base}, the clausal object \tit{wie es dir geht} `how you are doing', marked here in bold, appears in its base position. It can be extraposed to the right edge of the clause, shown in \ref{ex:mg-extrapose-cp-moved}.

 \ex.\label{ex:mg-extrapose-cp}
 \ag. Mir ist \tbf{wie} \tbf{es} \tbf{dir} \tbf{geht} egal.\\
  1\tsc{sg}.\tsc{dat} is how it 2\tsc{sg}.\tsc{dat} goes {the same}\\
  `I don't care how you are doing.'\label{ex:mg-extrapose-cp-base}
 \bg. Mir is egal \tbf{wie} \tbf{es} \tbf{dir} \tbf{geht}.\\
  1\tsc{sg}.\tsc{dat} is {the same} how it 2\tsc{sg}.\tsc{dat} goes\\
  `I don't care how you are doing.' \label{ex:mg-extrapose-cp-moved}\flushfill{Modern German}

 \ref{ex:mg-extrapose-dp} illustrates that it is impossible to extrapose a DP. The clausal object of \ref{ex:mg-extrapose-cp} is replaced by the simplex noun phrase \tit{die Sache} `that matter'.
 In \ref{ex:mg-extrapose-dp-base} the object, marked in bold, appears in its base position. In \ref{ex:mg-extrapose-dp-moved} it is extraposed, and the sentence is no longer grammatical.

 \ex.\label{ex:mg-extrapose-dp}
 \ag. Mir ist \tbf{die} \tbf{Sache} egal.\\
  1\tsc{sg}.\tsc{dat} is that matter {the same}\\
  `I don't care about that matter.'\label{ex:mg-extrapose-dp-base}
 \bg. *Mir ist egal \tbf{die} \tbf{Sache}.\\
  1\tsc{sg}.\tsc{dat} is {the same} that matter\\
  `I don't care about that matter.' \label{ex:mg-extrapose-dp-moved}\flushfill{Modern German}

 The same asymmetry between CPs and DPs can be observed with relative clauses. A relative clause is a CP, and the head of a relative clause is a DP. The sentences in \ref{ex:extra-headed} contain the relative clause \tit{was er gekocht hat} `what he has stolen'. This is marked in bold in the examples. The (light) head of the relative clause is \tit{das}.
 In \ref{ex:extra-headed-base}, the relative clause and its head appear in base position. In \ref{ex:extra-headed-only-clause}, the relative clause is extraposed. This is grammatical, because it is possible to extrapose CPs in Modern German. In \ref{ex:extra-headed-head-clause}, the relative clause and the head are extraposed. This is ungrammatical, because it is possible to extrapose DPs.

 \ex.\label{ex:extra-headed}
 \ag. Jan hat das, \tbf{was} \tbf{er} \tbf{gekocht} \tbf{hat}, aufgegessen.\\
  Jan has that what he cooked has eaten\\
 `Jan has eaten what he cooked.'\label{ex:extra-headed-base}
 \bg. Jan hat das aufgegessen, \tbf{was} \tbf{er} \tbf{gekocht} \tbf{hat}.\\
  Jan has that eaten what he cooked has\\
 `Jan has eaten what he cooked.'\label{ex:extra-headed-only-clause}
 \cg. *Jan hat aufgegessen, das, \tbf{was} \tbf{er} \tbf{gekocht} \tbf{hat}.\\
  Jan has eaten that what he cooked has\\
 `Jan has eaten what he cooked.'\label{ex:extra-headed-head-clause} \flushfill{Modern German}

 The same can be observed in relative clauses without a head. \ref{ex:extra-headless} is the same sentence as in \ref{ex:extra-headed} only without the overt head. The relative clause is marked in bold again.
 In \ref{ex:extra-headless-base}, the relative clause appears in base position. In \ref{ex:extra-headless-clause}, the relative clause is extraposed. This is grammatical, because it is possible to extrapose CPs in Modern German. In \ref{ex:extra-headless-no-rel}, the relative clause is extraposed without the relative pronouns. This is ungrammatical, because the relative pronoun is part of the CP.
 This shows that the relative pronoun in headless relatives in Modern German are necessarily part of a CP, which is here a relative clause.

 \ex.\label{ex:extra-headless}
 \ag. Jan hat \tbf{was} \tbf{er} \tbf{gekocht} \tbf{hat} aufgegessen.\\
 Jan has what he cooked has eaten\\
 `Jan has eaten what he cooked.'\label{ex:extra-headless-base}
 \bg. Jan hat aufgegessen \tbf{was} \tbf{er} \tbf{gekocht} \tbf{hat}.\\
 Jan has eaten what he cooked has\\
 `Jan has eaten what he cooked.'\label{ex:extra-headless-clause}
 \bg. *Jan hat \tbf{was} aufgegessen \tbf{er} \tbf{gekocht} \tbf{hat}.\\
 Jan has what eaten he cooked has\\
 `Jan has eaten what he cooked.'\label{ex:extra-headless-no-rel}\flushfill{Modern German}

 In conclusion, extraposition facts show that the relative pronoun in Modern German is syntactically part of the relative clause. Therefore, the relative pronoun is the internal element in headless relative construction.


 \subsection{The other element = external}\label{sec:external-element}

 In the previous section I introduced the relative pronoun as the internal element. This means that the other element is the external element. This section starts with the observation that there actually are languages in which two elements surface in so-called double-headed relative clauses. In these languages, the external head is a subset of the internal head, and that some features like \tsc{d} and case are necessarily excluded in the external head. I adopt this insight, and I apply it to the headless relative situation. I propose that the external head in headless relatives is a copy of a specific part of the relative pronoun.%cinque: headless relatives, what is the subset?

 As I said earlier, I need two elements to do case competition with. In headless relatives, I only see a single one surfacing. However, some languages actually show two elements surfacing. Here there are two copies of the element, one inside the relative clause, one outside of the relative clause.

 \exg. [\tbf{doü} adiyan-o-no] \tbf{doü} deyalukhe\\
  sago give.3\tsc{pl}.\tsc{nonfut}-{tr}-\tsc{conn} sago finished.\tsc{adj}\\
  `The sago that they gave is finished.' \flushfill{Kombai, \pgcitealt{vries1993}{78}}

 The external element is not always an exact copy of the element inside of the relative clause. An example from Kombai shows that the element outside of the relative clause can also be a subset of what the element inside of the relative clause is. Here I give two examples, there is an \tit{old man} and a \tit{person}, and there is \tit{pig} and a \tit{thing}.

 \ex.
 \ag. [\tbf{yare} gamo khereja bogi-n-o] \tbf{rumu} na-momof-a\\
  {old man} join.\ac{ss} work do.\ac{dur}.3\ac{sg}.\ac{nf}-\ac{tr}-\ac{conn} person my-uncle-\ac{pred}\\
  `The old man, who is joining the work, is my uncle.' 77
 \bg. [\tbf{ai} fali-khano] \tbf{ro} nagu-n-ay-a.\\
  pig carry-go.3\tsc{pl}.\tsc{nf} thing our-\tsc{tr}-pig-\ac{pred}\\
  `The pig they took away, is ours.' \flushfill{Kombai, \pgcitealt{vries1993}{77}}

 Let me now apply what we have seen so far to headless relatives. Headless relatives do not have an overt NP, so this cannot be copied. However, there is the relative pronoun which is specified for number, gender, case, etc. Are all of these features copied onto the external element? The copy is the portion of the nominal extended projection c-commanded by the relative clause. A headless relative is a restrictive relative clause. Therefore, there is no \tsc{d} and no case.

 Is it possible to add features onto the external head after it has been copied? Yes, for example D, as the example shows, but also case.

 \exg. Junya-wa [Ayaka-ga \tbf{ringo}-o mui-ta] sono \tbf{ringo}-o tabe-ta.\\
 Junya-\ac{top} Ayaka-\ac{nom} apple-\ac{acc} peel-\ac{pst} that apple-\ac{acc} eat-\ac{pst}\\
 ‘Junya ate the apples that Ayaka peeled.’ \flushfill{Japanese, \pgcitealt{erlewine2016}{2}}

 In sum, the external element is a copy of a subset of the features of the relative pronoun. Definiteness and case are not copied. New features can be merged onto the external element.




\subsection{Relative pronoun}

\begin{table}[H]
 \center
 \caption {Relative pronouns in headless relatives in Modern German}
  \begin{tabular}{cc}
  \toprule
              & \ac{an} \\
    \cmidrule{2-2}
    \ac{nom}  & w-e-r  \\
    \ac{acc}  & w-e-n  \\
    \ac{dat}  & w-e-m  \\
  \bottomrule
  \end{tabular}
\end{table}

three morphemes: \tsc{wh}, \tsc{ana}, number+gender+case

accusative relative pronoun

\ex.
\begin{forest} boom
  [\tsc{wh}P
      [\tsc{wh}P
          [\tit{w}, roof]
      ]
      [\tsc{ana}P
          [\tsc{ana}P
              [\tit{e}, roof]
          ]
          [\tsc{acc}P,
          tikz={
          \node[label=below:\tit{n},
          draw,circle,
          scale=0.85,
          fit to=tree]{};
          }
              [\tsc{f}2]
              [\tsc{nom}P
                  [\tsc{f1}]
                  [\tsc{ind}P
                      [\phantom{xxx},
                      roof
                      ]
                  ]
              ]
          ]
      ]
  ]
\end{forest}

nominative relative pronoun

\ex.
\begin{forest} boom
  [\tsc{wh}P
      [\tsc{wh}P
          [\tit{w}, roof]
      ]
      [\tsc{ana}P
          [\tsc{ana}P
              [\tit{e}, roof]
          ]
          [\tsc{nom}P,
          tikz={
          \node[label=below:\tit{r},
          draw,circle,
          scale=0.85,
          fit to=tree]{};
          }
              [\tsc{f}1]
              [\tsc{ind}P
                  [\phantom{xxx},
                  roof
                  ]
              ]
          ]
      ]
  ]
\end{forest}


\subsection{External element}

I copy the \tsc{ind} and I only merge the cases.

Modern German has two types of demonstratives: the strong one and the weak one.

The strong article is used when there is an anaphoric relation. Often there is a linguistic antecedent that is referred back to.

\exg. Hans hat heute \tbf{einen} \tbf{Freund} zum Essen mit nach Hause gebracht. Er hat uns vorher ein Foto \tbf{vom}/ \tbf{von} \tbf{dem} \tbf{Freund} gezeigt.\\
Hans has today a friend {to the} dinner with to home brought he has us beforehand a photo {of the\scsub{weak}} of the\scsub{strong} friend shown\\
`Hans brought a friend home for dinner today. He had shown us a photo of the friend beforehand.'

Weak articles are used when situational uniqueness is involved. Uniqueness can be global or within a restricted domain. The discourse participants mutually shared knowledge that uniqueness holds.

\ex.
\ag. Der Einbrecher ist {zum Glück} vom /von dem Hund verjagt worden.\\
the burglar is luckily {by the\scsub{weak}} by the\scsub{strong} dog {chased away} been\\
`Luckily, the burglar was chased away by the dog.'
\bg. Armstrong flog als erster zum Mond.\\
Armstrong flew as {first one} {to the\scsub{weak}} moon\\
`Armstrong was the first one to fly to the moon.' \flushfill{Modern German, \pgcitealt{schwarz2009}{40}}

In the headless relatives, there is uniqueness. Show?

The strong article cannot be used because it does not go together with the free choice interpretation of \tsc{wh}-relatives (say something about Hanink).

The weak article is used. accusative:

\ex.
\begin{forest} boom
[\tsc{acc}P,
tikz={
\node[label=below:\tit{n},
draw,circle,
scale=0.85,
fit to=tree]{};
}
    [\tsc{f}2]
    [\tsc{nom}P
        [\tsc{f1}]
        [\tsc{ind}P
            [\phantom{xxx},
            roof
            ]
        ]
    ]
]
\end{forest}

nominative:

\ex.
\begin{forest} boom
[\tsc{nom}P,
tikz={
\node[label=below:\tit{r},
draw,circle,
scale=0.85,
fit to=tree]{};
}
    [\tsc{f1}]
    [\tsc{ind}P
        [\phantom{xxx},
        roof
        ]
    ]
]
\end{forest}

\subsection{Comparison}

\exg. Uns besucht \tbf{wen} \tbf{Maria} \tbf{mag}.\\
 we.\ac{acc} visit.3\ac{sg}\scsub{[nom]} \tsc{rel}.\ac{acc}.\tsc{an} Maria.\ac{nom} like.3\ac{sg}\scsub{[acc]}\\
 `Who visits us, Maria likes.' \flushfill{adapted from \pgcitealt{vogel2001}{343}}

the internal case is more complex than the external case, and the internal base part is more complex than the external non-cas part

\begin{table}[H]
  \center
 \caption {Modern German: \ac{int} > \ac{ext}}
  \begin{tabular}[b]{cc}
      \toprule
      \ac{int}  &   \ac{ext} \\ \cmidrule{1-2}
      \begin{forest} boom
        [\tsc{wh}P
            [\tsc{wh}P
                [\tit{w}, roof]
            ]
            [\tsc{ana}P
                [\tsc{ana}P
                    [\tit{e}, roof]
                ]
                [\tsc{acc}P,
                tikz={
                \node[label=below:\tit{n},
                draw,circle,
                scale=0.85,
                fit to=tree]{};
                }
                    [\tsc{f}2]
                    [\tsc{nom}P,
                    tikz={
                    \node[draw,circle,
                    fill=DG,fill opacity=0.2,
                    DG,dashed,
                    scale=0.8,
                    fit to=tree]{};
                    }
                        [\tsc{f1}]
                        [\tsc{ind}P
                            [\phantom{xxx},
                            roof, baseline
                            ]
                        ]
                    ]
                ]
            ]
        ]
      \end{forest}
      &
      \begin{forest} boom
        [\textcolor{DG}{\ac{nom}},edge=DG,
        tikz={
        \node[label=below:\textcolor{DG}{\tit{r}},
        draw,circle,
        scale=0.75,
        DG,
        fit to=tree]{};
        \node[
        draw,circle,
        scale=0.8,
        dashed,DG,
        fit to=tree]{};
        }
            [\textcolor{DG}{\tsc{f1}},edge=DG]
            [\textcolor{DG}{\tsc{ind}P},edge=DG
                [\phantom{xxx},
                roof, baseline,edge=DG
                ]
            ]
        ]
      \end{forest}\\
      \bottomrule
  \end{tabular}
  \label{tbl:mg-int-wins}
\end{table}

\exg. *Ich {lade ein}, wen \tbf{mir} \tbf{sympathisch} \tbf{ist}.\\
1\ac{sg}.\ac{nom} invite.\ac{pres}.1\ac{sg}\scsub{[acc]} \ac{rel}.\ac{an}.\ac{acc} 1\ac{sg}.\ac{dat} nice be.\ac{pres}.3\ac{sg}\scsub{[nom]}\\
`I invite who I like.' \flushfill{Modern German, adapted from \pgcitealt{vogel2001}{344}}\label{ex:mg-acc-nom-rep}

the external case is more complex than the internal case, but the external base part is not more complex than the internal base part

\begin{table}[H]
  \center
 \caption {Modern German: \ac{int} > \ac{ext}}
  \begin{tabular}[b]{cc}
      \toprule
      \ac{int}  &   \ac{ext} \\ \cmidrule{1-2}
      \begin{forest} boom
        [\tsc{wh}P
            [\tsc{wh}P
                [\tit{w}, roof]
            ]
            [\tsc{ana}P
                [\tsc{ana}P
                    [\tit{e}, roof]
                ]
                [\textcolor{DG}{\ac{nom}},edge=DG,
                tikz={
                \node[label=below:\textcolor{DG}{\tit{r}},
                draw,circle,
                scale=0.75,
                DG,
                fit to=tree]{};
                \node[
                draw,circle,
                scale=0.8,
                dashed,DG,
                fit to=tree]{};
                }
                    [\textcolor{DG}{\tsc{f1}},edge=DG]
                    [\textcolor{DG}{\tsc{ind}P},edge=DG
                        [\phantom{xxx},
                        roof, baseline,edge=DG
                        ]
                    ]
                ]
            ]
        ]
      \end{forest}
      &
      \begin{forest} boom
      [\tsc{acc}P,
      tikz={
      \node[label=below:\tit{n},
      draw,circle,
      scale=0.85,
      fit to=tree]{};
      }
          [\tsc{f}2]
          [\tsc{nom}P,
          tikz={
          \node[draw,circle,
          fill=DG,fill opacity=0.2,
          DG,dashed,
          scale=0.8,
          fit to=tree]{};
          }
              [\tsc{f1}]
              [\tsc{ind}P
                  [\phantom{xxx},
                  roof, baseline
                  ]
              ]
          ]
      ]
      \end{forest}\\
      \bottomrule
  \end{tabular}
  \label{tbl:mg-ext-wins}
\end{table}



\section{Deriving `matching'}

Radek: Czech distinguishes between accidental uniqueness and inherent uniqueness. Accidental uniqueness: with \tsc{dem}, inherent uniqueness: without \tsc{dem}.

Radek's situation:

Two student assistants A and B are at their shared workdesk, which they share with other student assistants and where there’s a computer and a couple of other things, including a book (it doesn’t really matter to whom the book belongs). A is looking for a pencil, B says

\exg. Nějaká tužka je vedle {počítače /\#toho počítače}.\\
some pencil is {next to} computer \tsc{dem} computer\\
`There’s a pencil next to the computer.'

All situations like the topic situation – A and B’s shared office (desk)– have exactly one computer in it.

\exg. Nějaká tužka je vedle {té knížky /\#knížky}\\
some pencil is {next to} \tsc{dem} book book\\
`There’s a pencil next to the book.'

There is exactly one book in the topic situation – A and B’s shared office (desk) – and it does not hold that all situation like the topic situation have exactly one book in it

Florian showed that this is different for Modern German:

\begin{table}[H]
\begin{tabular}{c|ccc}
\toprule
       & anaphoric                & situational uniqueness              & inherent uniqueness                 \\
       \cmidrule{2-4}
Polish & \cellcolor{DG}\tsc{dem}  & \cellcolor{DG}\tsc{dem}             & ∅                                   \\
German & \tsc{dem}\scsub{strong}  & \cellcolor{LG}\tsc{dem}\scsub{weak} & \cellcolor{LG}\tsc{dem}\scsub{weak} \\
\bottomrule
\end{tabular}
\end{table}

\tit{to} is incompatible with \tit{ever}, because \tit{to} makes it accidentally uniqueness and \tit{ever} requires inherent uniqueness


\section{Excluding `non-matching --- external-only'}




\section{Summary}

The linguistic counterpart of `allow \tsc{ext}?' is whether the internal base and the external base are syncretic (base\scsub{int} = base\scsub{ext}?).
The linguistic counterpart of `allow \tsc{int}?' is whether the external base is a clitic (base\scsub{ext} = clitic?).

\begin{figure}[H]
  \centering
    \footnotesize{
    \begin{tikzpicture}[node distance=1.5cm]
      \node (question2) [question]
      {base\scsub{int} = base\scsub{ext}?};
          \node (outcome2) [outcome, below of=question2, xshift=-1.5cm]
          {complex};
              \node (example2) [example, below of=outcome2, yshift=0.25cm]
              {\scriptsize{e.g. Gothic, Old High German, Classical Greek}};
          \node (question3) [question, below of=question2, xshift=2cm, yshift=-0.5cm]
          {base\scsub{ext} = clitic?};
              \node (outcome3) [outcome, below of=question3, xshift=-1.5cm]
              {\ac{int} + complex};
                  \node (example3) [example, below of=outcome3, yshift=0.25cm]
                  {\scriptsize{e.g. Modern German\\\phantom{x}}};
              \node (outcome4) [outcome, below of=question3, xshift=1.5cm]
              {matching};
                  \node (example4) [example, below of=outcome4, yshift=0.25cm]
                  {\scriptsize{e.g. Polish\\\phantom{x}}};

    \draw [arrow] (question2) -- node[anchor=east] {yes} (outcome2);
    \draw [arrow] (question2) -- node[anchor=west] {no} (question3);
    \draw [arrow] (question3) -- node[anchor=east] {no} (outcome3);
    \draw [arrow] (question3) -- node[anchor=west] {yes} (outcome4);
    \end{tikzpicture}
    }
    \caption{Two theoretical parameters generate three language types}
    \label{fig:formal-parameters}
\end{figure}



\section{Alternative analyses}\label{sec:alternative-analyses}

\subsection{Himmelreich}

She specific languages for having different types of agree (up, down) and different types of probes (active, non-active). Doing that, she successfully derived free relatives and parasitic gaps in different languages.

\subsection{Grosu}

Grosu 1994 linked richness of inflection to liberality. He actually talked about the richness of pro.

\subsection{Grafting story}

For this pattern a single element analysis seems intuitive, if you assume that case is complex and that syntax works bottom-up. First you built the relative clause, with the big case in there. Then you build the main clause and you let the more complex case in the embedded clause license the main clause predicate.

Consider the example in \ref{ex:mg-nom-acc-grafting}. Here the internal case is accusative and the external one nominative.

\exg. Uns besucht \tbf{wen} \tbf{Maria} \tbf{mag}.\\
 we.\ac{acc} visit.3\ac{sg}\scsub{[nom]} \tsc{rel}.\ac{acc}.\tsc{an} Maria.\ac{nom} like.3\ac{sg}\scsub{[acc]}\\
 `Who visits us, Maria likes.' \flushfill{adapted from \pgcitealt{vogel2001}{343}}\label{ex:mg-nom-acc-grafting}

The relative clause is built, including the accusative relative pronoun. Now the main clause predicate can merge with the nominative that is contained within the accusative.

 \ex.
 \begin{forest} boom
  [,name=src, s sep=15mm
   [VP
      [\tit{besucht}, roof]
   ]
    [,no edge, s sep=20mm
        [\ac{acc}P,
     tikz={
     \node[label=below:\tit{wen},
     draw,circle,
     scale=0.85,
     fit to=tree]{};
     }
            [\tsc{f2}]
            [\tsc{nomP},name=tgt
                [\tsc{f1}]
                [XP
                    [\phantom{xxx}, roof]
                ]
            ]
        ]
     [VP
        [\tit{Maria mag}, roof]
     ]
   ]
  ]
  \draw (src) to[out=south east,in=north east] (tgt);
 \end{forest}\label{ex:acc-nom-grafting}

The other way around does not work. Consider \ref{ex:mg-acc-nom-grafting}. This is an example with nominative as internal case and accusative as external case.

\exg. *Ich {lade ein}, wen \tbf{mir} \tbf{sympathisch} \tbf{ist}.\\
I.\ac{nom} invite.1\ac{sg}\scsub{[acc]} \tsc{rel}.\ac{acc}.\tsc{an} I.\ac{dat} nice be.3\ac{sg}\scsub{[nom]}\\
`I invite who I like.' \flushfill{adapted from \pgcitealt{vogel2001}{344}}\label{ex:mg-acc-nom-grafting}

Now the relative clause is built first again, this time only including the nominative case. There is no accusative node to merge with for the external predicate. Instead, the relative pronoun would need to grow to accusative somehow and then the merge could take place. This is the desired result, because the sentence is ungrammatical.

\ex.
\begin{forest} boom
  [,name=src, s sep=15mm
     [VP
         [\tit{lade ein}, roof]
     ]
         [,no edge
       [\tsc{nomP},
       tikz={
       \node[label=below:\tit{wer},
       draw,circle,
       scale=0.85,
       fit to=tree]{};
       }
         [\tsc{f1}]
         [XP
           [\phantom{xxx}, roof]
         ]
       ]
       [VP
         [\tit{mir sympatisch ist}, roof]
       ]
      ]
    ]
\end{forest}\label{ex:nom-acc-grafting}

So, this seems to work fine. The assumptions you have to do in order to make this are the following. First, case is complex. Second, you can remerge an embedded node (grafting). For the first one I have argued in Chapter \ref{ch:decomposition}. The second one could use some additional argumentation. It is a mix between internal remerge (move) and external merge, namely external remerge. Other literature on multidominance and grafting, other phenomena. Problems: linearization, .. But even if fix all these theoretical problems, there is an empirical one.

That is, I want to connect this behavior of Modern German headless relatives to the shape of its relative pronouns. These pronouns are \tsc{wh}-elements. The OHG and Gothic ones are not \tsc{wh}, they are \tsc{d}. Their relative pronouns look different, and so their headless relatives can also behave differently.

\section{Summary}

here
