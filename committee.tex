% !TEX root = thesis.tex

\chapter{The committee}\label{ch:relativization}

In Chapter \ref{ch:case-competition-typology} I introduced two parameters that generates the attested languages, as shown in Table \ref{fig:two-parameters}. The first parameter is `allow mismatches?', which distinguishes between `matching' languages, as Polish, on the one hand, and `non-matching --- internal only' and `non-matching' languages as Modern German and Old High German. The second parameter is `cases considered', which distinguishes between `non-matching --- internal only' languages, as Modern German, on the one hand, and `non-matching' languages, as Old High German. This chapter gives the linguistic counterparts of these parameters.

\begin{figure}[H]
  \centering
    \footnotesize{
    \begin{tikzpicture}[node distance=1.5cm]
      \node (question2) [question]
      {allow mismatches?};
          \node (outcome2) [outcome, below of=question2, xshift=-1.5cm]
          {matching};
              \node (example2) [example, below of=outcome2, yshift=0.25cm]
              {\scriptsize{e.g. Polish\\\phantom{x}}};
          \node (question3) [question, below of=question2, xshift=2cm, yshift=-0.5cm]
          {cases allowed:};
              \node (outcome3) [outcome, below of=question3, xshift=-1.5cm]
              {\ac{int} + complex};
                  \node (example3) [example, below of=outcome3, yshift=0.25cm]
                  {\scriptsize{e.g. Modern German\\\phantom{x}}};
              \node (outcome4) [outcome, below of=question3, xshift=1.5cm]
              {complex};
                  \node (example4) [example, below of=outcome4, yshift=0.25cm]
                  {\scriptsize{e.g. Gothic, Old High German, Classical Greek}};

    \draw [arrow] (question2) -- node[anchor=east] {no} (outcome2);
    \draw [arrow] (question2) -- node[anchor=west] {yes} (question3);
    \draw [arrow] (question3) -- node[anchor=east] {\ac{int}} (outcome3);
    \draw [arrow] (question3) -- node[anchor=west] {\ac{int} + \ac{ext}} (outcome4);

    % \node[fit=(question3)(outcome3)(outcome4)(example3)(example4),
    %       draw, circle, scale=0.7]
    %       {};
    \end{tikzpicture}
    }
    \caption{Attested patterns in headless relatives with case competition (repeated)}
    \label{fig:two-parameters}
\end{figure}

So far, I have discussed three different concepts: the internal case, the external case and the relative pronoun. I said things like ``the relative pronoun is allowed to surface in the internal or in the external case''. I have not been explicit about where and how the competition between the internal and the external case takes place. In order to avoid introducing machinery just for case competition situations, I assume it takes place in syntax. I propose that headless relatives actually have an internal element and an external element at some point of the derivation. At the end of the derivation, only one of them surfaces as the relative pronoun.

Section \ref{sec:int-ext-elements} discusses the internal and external case into detail. It shows syntactic evidence for the relative being the internal element. I argue that the external element is syntactically placed outside the relative clause. Independent evidence for the presence of this external element comes from double-headed languages. I also adopt another crucial property of the external heads in these double-headed relative clauses: the external head is always a subset of the internal head. This observation ultimately leads me to be able to exclude the `non-matching --- external-only' language type.

However, before I go into detail about the specifics of the internal and the external element, I discuss in Section \ref{sec:containment-noncase} how the introduction of an internal and an external element helps to account for the difference between languages. I propose that both elements consists of part that concerns case, a case part, and a part that concerns other features than case, a non-case part. The latter parts, the non-case parts, are crucial in determining how the language behaves with the respect to the two parameters. Specifically, the comparison between the non-case part of the internal element and the non-case part of the external element decides whether the internal and elements are allowed to surface. Just as I did for the competition between case features, I rely on containment. When the internal non-case part contains the features that the external non-case part contains, the internal element is allowed to surface. When the external non-case part contains the features that the internal non-case part contains, the external element is allowed to surface.

In Section \ref{sec:interim-summary-proposal} I combine the findings from the first two sections, and I formulate my proposal. The variation between language follows from the comparison between the internal and external element. In other words, the factor that decides on the parameter setting in Figure \ref{fig:two-parameters} can be observed within the language. Partly it surfaces as the relative pronoun, and partly it is derived from what the relative pronoun looks like. The consequence of that is that the variation between the languages boils down to something independently established in the language.

In Section \ref{sec:deriving-languages}, I show how the proposal derives the different types of languages.

In Section \ref{sec:alternative-analyses}, I compare my approach with previous accounts of \citet{himmelreich2017} and Grosu.

\section{Containment in non-case}\label{sec:containment-noncase}

The goal is to explain why there are languages that allow internal and external, languages that allow only internal, but no languages that only allow only external. I propose we do this by comparing the non-case parts of the internal and the external elements in headless relatives.

Headless relatives have an internal element and an external element. Both elements consist of two parts: the case part and a non-case part. Part \ref{part:case-facts} was about the case part of the elements. A more complex case wins over a less complex case because the former contains all features the latter contains. Concretely, the dative wins over the accusative because the dative contains all features the accusative contains, the dative wins over the nominative because the dative contains all features the nominative contains, and the accusative wins over the nominative because the accusative contains all features the nominative contains.

I propose that a containment relation is also the crucial factor that determines in whether a particular language allows the internal and external case to surface. It is not the containment relationship having to do with case, but the one that has to do with the non-case part. I propose that a language allows the internal case to surface when it wins the case competition, if the non-case part of the internal element contains all features that the non-case part of the external element contains. The other way around works the same way: a language allows the external case to surface when it wins the case competition, if the non-case part of the external element contains all features that the non-case part of the internal element contains.

Table \ref{tbl:non-case} gives a table that resembles the case tables I discussed in Chapter \ref{ch:case-competition-typology}. The left column shows the features of the internal non-case part. The top row shows the features of the internal non-case part. The other cells indicate the features of the winning non-case part.
If the internal non-case part contains the features \tsc{a} or \tsc{a} and \tsc{b}, and the external non-case part contains only the features \tsc{a} or \tsc{a} and \tsc{b}, the internal and the external are allowed to surface in the language. This is the pattern that corresponds to the `non-matching' type of language like Old High German.
If the internal non-case part contains the feature \tsc{a}, and the external non-case part contains only the features \tsc{a} and \tsc{b}, the external is allowed to surface in the language and the internal is not. This is the pattern that is crosslinguistically not attested. In Section \ref{sec:int-ext-elements} I show why this is the case.
If the internal non-case part contains the features \tsc{a} and \tsc{b}, and the external non-case part contains only the feature \tsc{a}, the internal is allowed to surface in the language and the external is not. This is the pattern that corresponds to the `non-matching --- internal only' type of language like Modern German.

\begin{table}[H]
  \center
  \caption{Relative pronoun in external case}
  \begin{tabular}{c|c|c}
    \toprule
   \textsubscript{\ac{int}} \textsuperscript{\ac{ext}}
          & \tsc{a}
          & \tsc{a},\tsc{b}
          \\ \cmidrule{1-3}
      \tsc{a}
          & \tsc{a}
          & \tsc{a},\tsc{b}
          \\ \cmidrule{1-3}
      \tsc{a},\tsc{b}
          & \tsc{a}
          & \tsc{a},\tsc{b}
          \\
    \bottomrule
  \end{tabular}
  \label{tbl:non-case}
\end{table}

There is a crucial difference between the Table \ref{tbl:non-case} and the case tables in Chapter \ref{ch:case-competition-typology}. The case tables described different situations that are attested within a single language, as I illustrated by going through  examples that correspond to different cells in the tables. Case differs from sentence to sentence: in one sentence the internal case is the dative, in the other it is the accusative or the nominative. The non-case table is not like that: only a single cell in the table corresponds to a particular language. The non-case parts of the internal and external elements are namely stable per language. The relative pronoun has namely a single form, and so does the external element.

Taking this all together, there are two competitions taking place in headless relatives: case competition and non-case competition. Case competition decides which case wins the sentence and non-case competition decides whether the result can go through. Let me put this in the metaphor with the committee I introduced in Section \ref{sec:possible-patterns}. The committee learns who wins the case competition. It can either approve this case or not approve it. The information that the committee uses for their decision is the comparison between the internal and the external element. More specifically, it compares the non-case parts of the internal and external element. The committee approves the winning case if the non-case part of its element contains the non-case part of the element of the losing case.

In sum, a relative pronoun can surface in the internal case if both the non-case and the case part of the internal element contain non-case and case part of the external element. It works the same way the other way around: a relative pronoun can surface in the external case if both the non-case and the case part of the external element contain the non-case and case part of the internal element.

In this chapter, I introduce what the non-case part the internal and external element looks like, and I argue per language what they look like.


\section{The internal and external element}\label{sec:int-ext-elements}

The comparison between the internal and external element is crucial in determining what the type of a language is. This section discusses the featural content and the syntactic positions of both elements. In Section \ref{sec:internal-element}, I show that the relative pronoun is the internal element, and that this is part of the relative clause. Section \ref{sec:external-element} discusses the external element. This element is syntactically places outside of the relative clause, and its features form a subset of the features of the relative pronoun.

\subsection{The relative pronoun = internal}\label{sec:internal-element}

In headless relative constructions, there is a single element that surfaces: the relative pronoun. In this section, I show that the relative pronoun is syntactically part of the relative clause. The evidence comes from extraposition data in Modern German. In Modern German, it is possible to extrapose a CP (a clause), but not a DP (a noun phrase). In this section I first show that Modern German CPs can be extraposed and DPs cannot. Then I illustrate how relative clauses including the relative pronoun in headless relatives pattern with CPs: they can be extraposed as well. I conclude that the relative pronoun is the internal element in the headless relative.

The sentences in \ref{ex:mg-extrapose-cp} show that it is possible to extrapose a CP. In \ref{ex:mg-extrapose-cp-base}, the clausal object \tit{wie es dir geht} `how you are doing', marked here in bold, appears in its base position. It can be extraposed to the right edge of the clause, shown in \ref{ex:mg-extrapose-cp-moved}.

\ex.\label{ex:mg-extrapose-cp}
\ag. Mir ist \tbf{wie} \tbf{es} \tbf{dir} \tbf{geht} egal.\\
 1\tsc{sg}.\tsc{dat} is how it 2\tsc{sg}.\tsc{dat} goes {the same}\\
 `I don't care how you are doing.'\label{ex:mg-extrapose-cp-base}
\bg. Mir is egal \tbf{wie} \tbf{es} \tbf{dir} \tbf{geht}.\\
 1\tsc{sg}.\tsc{dat} is {the same} how it 2\tsc{sg}.\tsc{dat} goes\\
 `I don't care how you are doing.' \label{ex:mg-extrapose-cp-moved}\flushfill{Modern German}

\ref{ex:mg-extrapose-dp} illustrates that it is impossible to extrapose a DP. The clausal object of \ref{ex:mg-extrapose-cp} is replaced by the simplex noun phrase \tit{die Sache} `that matter'.
In \ref{ex:mg-extrapose-dp-base} the object, marked in bold, appears in its base position. In \ref{ex:mg-extrapose-dp-moved} it is extraposed, and the sentence is no longer grammatical.

\ex.\label{ex:mg-extrapose-dp}
\ag. Mir ist \tbf{die} \tbf{Sache} egal.\\
 1\tsc{sg}.\tsc{dat} is that matter {the same}\\
 `I don't care about that matter.'\label{ex:mg-extrapose-dp-base}
\bg. *Mir ist egal \tbf{die} \tbf{Sache}.\\
 1\tsc{sg}.\tsc{dat} is {the same} that matter\\
 `I don't care about that matter.' \label{ex:mg-extrapose-dp-moved}\flushfill{Modern German}

The same asymmetry between CPs and DPs can be observed with relative clauses. A relative clause is a CP, and the head of a relative clause is a DP. The sentences in \ref{ex:extra-headed} contain the relative clause \tit{was er gekocht hat} `what he has stolen'. This is marked in bold in the examples. The (light) head of the relative clause is \tit{das}.
In \ref{ex:extra-headed-base}, the relative clause and its head appear in base position. In \ref{ex:extra-headed-only-clause}, the relative clause is extraposed. This is grammatical, because it is possible to extrapose CPs in Modern German. In \ref{ex:extra-headed-head-clause}, the relative clause and the head are extraposed. This is ungrammatical, because it is possible to extrapose DPs.

\ex.\label{ex:extra-headed}
\ag. Jan hat das, \tbf{was} \tbf{er} \tbf{gekocht} \tbf{hat}, aufgegessen.\\
 Jan has that what he cooked has eaten\\
`Jan has eaten what he cooked.'\label{ex:extra-headed-base}
\bg. Jan hat das aufgegessen, \tbf{was} \tbf{er} \tbf{gekocht} \tbf{hat}.\\
 Jan has that eaten what he cooked has\\
`Jan has eaten what he cooked.'\label{ex:extra-headed-only-clause}
\cg. *Jan hat aufgegessen, das, \tbf{was} \tbf{er} \tbf{gekocht} \tbf{hat}.\\
 Jan has eaten that what he cooked has\\
`Jan has eaten what he cooked.'\label{ex:extra-headed-head-clause} \flushfill{Modern German}

The same can be observed in relative clauses without a head. \ref{ex:extra-headless} is the same sentence as in \ref{ex:extra-headed} only without the overt head. The relative clause is marked in bold again.
In \ref{ex:extra-headless-base}, the relative clause appears in base position. In \ref{ex:extra-headless-clause}, the relative clause is extraposed. This is grammatical, because it is possible to extrapose CPs in Modern German. In \ref{ex:extra-headless-no-rel}, the relative clause is extraposed without the relative pronouns. This is ungrammatical, because the relative pronoun is part of the CP.
This shows that the relative pronoun in headless relatives in Modern German are necessarily part of a CP, which is here a relative clause.

\ex.\label{ex:extra-headless}
\ag. Jan hat \tbf{was} \tbf{er} \tbf{gekocht} \tbf{hat} aufgegessen.\\
Jan has what he cooked has eaten\\
`Jan has eaten what he cooked.'\label{ex:extra-headless-base}
\bg. Jan hat aufgegessen \tbf{was} \tbf{er} \tbf{gekocht} \tbf{hat}.\\
Jan has eaten what he cooked has\\
`Jan has eaten what he cooked.'\label{ex:extra-headless-clause}
\bg. *Jan hat \tbf{was} aufgegessen \tbf{er} \tbf{gekocht} \tbf{hat}.\\
Jan has what eaten he cooked has\\
`Jan has eaten what he cooked.'\label{ex:extra-headless-no-rel}\flushfill{Modern German}

In conclusion, extraposition facts show that the relative pronoun in Modern German is syntactically part of the relative clause. Therefore, the relative pronoun is the internal element in headless relative construction.


\subsection{The other element = external}\label{sec:external-element}

In the previous section I introduced the relative pronoun as the internal element. This means that the other element is the external element. This section starts with the observation that there actually are languages in which two elements surface in so-called double-headed relative clauses. In these languages, the external head is always a subset of the internal head, and that some features like \tsc{d} are necessarily excluded in the external head. I adopt this insight, and I apply it to the headless relative situation. I propose that the external head in headless relatives is a copy of the ϕ-features of the relative pronoun.

There is independent evidence for this head, namely from languages that actually let the head surface. Here there are two identical copies of the head, one inside the relative clause, one outside of the relative clause.

\exg. [\tbf{doü} adiyan-o-no] \tbf{doü} deyalukhe\\
 sago give.3\tsc{pl}.\tsc{nonfut}-{tr}-\tsc{conn} sago finished.\tsc{adj}\\
 `The sago that they gave is finished.' \flushfill{Kombai, \pgcitealt[78]{vries1993}}

I give an example of a language in which the external head follows the relative clause. There are also languages in which the head precedes the relative clause, e.g. xx

The external head is not always an exact copy of the head inside of the relative clause. An example from xx here shows that the head outside of the relative clause can also be a subset of what the element inside of the relative clause is. In this case, there is an \tit{old man} and a \tit{person}. In that case, there is \tit{pig} and \tit{thing}.

\ex.
\ag. [\tbf{yare} gamo khereja bogi-n-o] \tbf{rumu} na-momof-a\\
 {old man} join.\tsc{ss} work \tsc{dur}.do.\tsc{3sg}.\tsc{nf}-\tsc{tr}-\tsc{conn} person my-uncle-\tsc{pred}\\
 `The old man, who is joining the work, is my uncle.' 77
\bg. [\tbf{ai} fali-khano] \tbf{ro} nagu-n-ay-a.\\
 pig carry-go.3\tsc{pl}.\tsc{nf} thing our-\tsc{tr}-pig-\tsc{pred}\\
 `The pig they took away, is ours.' \flushfill{Kombai, \pgcitealt[77]{vries1993}}

So, we have the head. Translating this to headless relatives, there is no overt NP, so we only see the relative pronoun. The

But, are there also some features that we can never copy, any restrictions? This copy is the portion of the nominal extended projection c-commanded by the RC. Therefore, it does not have D and no wh?







And what about the external head, does it need to remain the copy that I made it, or can other things be added to it?

Erlewine \& Gould (2016: 2)
Junya-wa [[ Ayaka-ga ringo-o mui-ta] sono ringo-o] tabe-ta.
J.-TOP A.-NOM apple-ACC peel-PAST that apple-ACC eat-PAST
‘J. ate the apples that A. peeled.’

So, the external head in the headless relative can also get a \tsc{d} merged to it. This is what I will do for the Old High German example in which the external case wins.




\section{Interim summary}\label{sec:interim-summary-proposal}

The internal element is the relative pronoun.

The external element is a subset of the relative pronoun, it does not do not include D or wh, which leaves phi-features.

and then you can merge some stuff on top of it later if you want

can internal win?
  not merge anything later
    ab a and ab ab are still possibilities, but a ab and ab ac are excluded. it's good that ab a is excluded, because this is a pattern we never saw to begin with, but ab ac we somehow want.. I have an different way of excluding that
  merge something later
    ab ab can still happen, if you merge b later. ab a cannot happen, because there was nothing to merge later. a ab cannot delete, ab ad can

can ext win?
  not merge anything later

  merge something later





\section{Deriving the patterns}\label{sec:deriving-languages}

\subsection{Deriving internal-and-external}

Old High German relative pronouns consist of a morpheme \tit{d-} and a phi and case portmanteau.

\begin{table}[H]\label{tbl:paradigmohg}
 \center
 \caption {Relative pronouns in headless relatives in Old High German}
  \begin{tabular}{cccc}
  \toprule
       & \ac{n}.\ac{sg} & \ac{m}.\ac{sg}  & \ac{f}.\ac{sg} \\
        \cmidrule{2-4}
  \ac{nom} & d-aȥ           & d-ër          & d-iu      \\
  \ac{acc} & d-aȥ        & d-ën      & d-ea/-ia/(-ie) \\
  \ac{dat} & d-ëmu/-ëmo     & d-ëmu/-ëmo   & d-ëru/-ëro   \\
  \bottomrule
         & \ac{n}.\ac{pl} & \ac{m}.\ac{pl}   & \ac{f}.\ac{pl} \\
          \cmidrule{2-4}
    \ac{nom}  & d-iu/-ei      &  d-ē/-ea/-ia/-ie & d-eo/-io        \\
    \ac{acc}  & d-iu/-ei      &  d-ē/-ea/-ia/-ie & d-eo/-io        \\
    \ac{dat}  & d-ēm/-ēn      &  d-ēm/-ēn        & d-ēm/-ēn        \\
    \bottomrule
  \end{tabular}
\end{table}

The external head is the phi features bundle.

\ex.
\begin{forest} boom
[, s sep=20mm
    [\tbf{ϕP}
        [\phantom{xxx},
        roof, baseline
        ]
    ]
    [CP
        [\tsc{rel}P
            [DP
                [\tit{d-}, roof]
            ]
            [\tsc{accP},
            tikz={
            \node[label=below:\tit{-en},
            draw,circle,
            scale=0.8,
            fit to=tree]{};
            }
                [\tsc{f2}]
                [\tsc{nomP}
                    [\tsc{f1}]
                    [\tbf{ϕP}
                        [\phantom{xxx}, roof]
                    ]
                ]
            ]
        ]
        [TP
            [\phantom{xxx}, roof]
        ]
    ]
]
\end{forest}

merge \tsc{k} (and \tsc{d})? Here it looks like you could do it, or you could not do it, so I will not do it.

\ex.
\begin{forest} boom
[, s sep=20mm
    [\tsc{nom}P,
    tikz={
    \node[label=below:\tit{-er},
    draw,circle,
    scale=0.8,
    fit to=tree]{};
    }
        [\tsc{f1}]
        [ϕP
            [\phantom{xxx},
            roof, baseline
            ]
        ]
    ]
    [CP
        [\tsc{rel}P
            [DP
                [\tit{d-}, roof]
            ]
            [\tsc{accP},
                [\tit{-en}, roof]
            ]
        ]
        [TP
            [\phantom{xxx}, roof]
        ]
    ]
]
\end{forest}

Now if we compare the internal element to the external element, we see that the internal element contains the external element

\begin{table}[H]
  \center
	\caption {Old High German: \ac{int} > \ac{ext}}
		\begin{tabular}[b]{cc}
      \toprule
      \ac{int}  &   \ac{ext} \\ \cmidrule{1-2}
      \begin{forest} boom
        [\tsc{rel}P
            [DP
                [\tit{d-}, roof]
            ]
            [\tsc{acc}P,
            tikz={
            \node[label=below:\tit{-en},
            draw,circle,
            scale=0.85,
            fit to=tree]{};
            }
                [\tsc{f}2]
                [\tsc{nom}P
                    [\tsc{f1}]
                    [ϕP
                        [\phantom{xxx},
                        roof, baseline
                        ]
                    ]
                ]
            ]
        ]
      \end{forest}
      &
      \begin{forest} boom
        [\tsc{nom}P,
        tikz={
        \node[label=below:\tit{-er},
        draw,circle,
        scale=0.8,
        fit to=tree]{};
        }
            [\tsc{f1}]
            [ϕP
                [\phantom{xxx},
                roof, baseline
                ]
            ]
        ]
      \end{forest}\\
      \bottomrule
  \end{tabular}
  \label{tbl:ohg-int-wins}
\end{table}

There is structure with case and phi which contains



\begin{table}[H]
  \center
	\caption {Old High German: \ac{int} < \ac{ext}}
		\begin{tabular}[b]{cc}
      \toprule
      \ac{int}  &   \ac{ext} \\ \cmidrule{1-2}
      \begin{forest} boom
        [\tsc{rel}P
            [DP
                [\tit{d-}, roof]
            ]
            [\tsc{acc}P,
            tikz={
            \node[label=below:\tit{-en},
            draw,circle,
            scale=0.85,
            fit to=tree]{};
            }
                [\tsc{f}2]
                [\tsc{nom}P
                    [\tsc{f1}]
                    [ϕP
                        [\phantom{xxx},
                        roof, baseline
                        ]
                    ]
                ]
            ]
        ]
      \end{forest}
      &
      \begin{forest} boom
        [DP
            [DP
                [\tit{d-}, roof]
            ]
            [\tsc{dat}P,
            tikz={
            \node[label=below:\tit{-em},
            draw,circle,
            scale=0.85,
            fit to=tree]{};
            }
                [\tsc{f}3]
                [\tsc{acc}P
                    [\tsc{f}2]
                    [\tsc{nom}P
                        [\tsc{f1}]
                        [ϕP
                            [\phantom{xxx},
                            roof, baseline
                            ]
                        ]
                    ]
                ]
            ]
        ]
      \end{forest}\\
      \bottomrule
  \end{tabular}
  \label{tbl:ohg-ext-wins}
\end{table}


\ex. \tbf{Spellout Algorithm:}\\
Merge F and \label{ex:spellout}
 \a. Spell out FP.
 \b. If (a) fails, attempt movement of the spec of the complement of \tsc{f}, and retry (a).
 \b. If (b) fails, move the complement of \tsc{f}, and retry (a).

When a new match is found, it overrides previous spellouts.

\ex. \tbf{Cyclic Override} \citep{starke2018}:\\
Lexicalisation at a node XP overrides any previous match at a phrase contained in XP.

If the spellout procedure in \ref{ex:spellout} fails, backtracking takes place.

\ex. \tbf{Backtracking} \citep{starke2018}:\\
When spellout fails, go back to the previous cycle, and try the next option for that cycle.\label{ex:backtracking}

If backtracking also does not help, a specifier is constructed.

\ex. \tbf{Spec Formation} \citep{starke2018}:\\
If Merge F has failed to spell out (even after backtracking), try to spawn a new derivation providing the feature F and merge that with the current derivation, projecting the feature F at the top node.\label{ex:specformation}

\ex. Merge F, Move XP, Merge XP

show how internal-wins works
show how external-wins works

\subsection{Deriving internal-only}

Modern German

featural content of relative pronoun

\begin{table}[H]
 \center
 \caption {Relative pronouns in headless relatives in Modern German}
  \begin{tabular}{ccc}
  \toprule
       & \ac{inan} & \ac{an} \\
        \cmidrule{2-3}
    \ac{nom}  & w-as     & w-er    \\
    \ac{acc}  & w-as     & w-en   \\
    \ac{dat}  & -      & w-em    \\
  \bottomrule
  \end{tabular}
\end{table}

\exg. Uns besucht \tbf{wen} \tbf{Maria} \tbf{mag}.\\
 we.\ac{acc} visit.3\ac{sg}\scsub{[nom]} \tsc{rel}.\ac{acc}.\tsc{an} Maria.\ac{nom} like.3\ac{sg}\scsub{[acc]}\\
 `Who visits us, Maria likes.' \flushfill{adapted from \pgcitealt{vogel2001}{343}}

\begin{table}[H]
  \center
	\caption {Modern German: \ac{int} > \ac{ext}}
		\begin{tabular}[b]{cc}
      \toprule
      \ac{int}  &   \ac{ext} \\ \cmidrule{1-2}
      \begin{forest} boom
        [\tsc{rel}P
            [\tsc{wh}P
                [\tit{w-}, roof]
            ]
            [\tsc{acc}P,
            tikz={
            \node[label=below:\tit{-en},
            draw,circle,
            scale=0.85,
            fit to=tree]{};
            }
                [\tsc{f}2]
                [\tsc{nom}P
                    [\tsc{f1}]
                    [ϕP
                        [\phantom{xxx},
                        roof, baseline
                        ]
                    ]
                ]
            ]
        ]
      \end{forest}
      &
      \begin{forest} boom
        [\tsc{nom}P,
        tikz={
        \node[label=below:\tit{-er},
        draw,circle,
        scale=0.8,
        fit to=tree]{};
        }
            [\tsc{f1}]
                [ϕP
                [\phantom{xxx},
                roof, baseline
                ]
            ]
        ]
      \end{forest}\\
      \bottomrule
  \end{tabular}
  \label{tbl:mg-int-wins}
\end{table}


\begin{table}[H]
  \center
	\caption {Modern German: \ac{int} > \ac{ext}}
		\begin{tabular}[b]{cc}
      \toprule
      \ac{int}  &   \ac{ext} \\ \cmidrule{1-2}
      \begin{forest} boom
        [\tsc{rel}P
            [DP
                [\tit{w-}, roof]
            ]
            [\tsc{acc}P,
            tikz={
            \node[label=below:\tit{-en},
            draw,circle,
            scale=0.85,
            fit to=tree]{};
            }
                [\tsc{f}2]
                [\tsc{nom}P
                    [\tsc{f1}]
                    [\tsc{dP}
                        [\tsc{d}]
                        [ϕP
                            [\phantom{xxx},
                            roof, baseline
                            ]
                        ]
                    ]
                ]
            ]
        ]
      \end{forest}
      &
      \begin{forest} boom
        [\tsc{dat}P,
        tikz={
        \node[label=below:\tit{-em},
        draw,circle,
        scale=0.85,
        fit to=tree]{};
        }
            [\tsc{f}3]
            [\tsc{acc}P
                [\tsc{f}2]
                [\tsc{nom}P
                    [\tsc{f1}]
                    [\tsc{dP}
                        [\tsc{d}]
                        [ϕP
                            [\phantom{xxx},
                            roof, baseline
                            ]
                        ]
                    ]
                ]
            ]
        ]
      \end{forest}\\
      \bottomrule
  \end{tabular}
  \label{tbl:mg-ext-wins}
\end{table}


Florian with his \tit{am Main}




\subsection{Deriving `matching'}



\begin{table}[H]
  \center
	\caption {Polish: \ac{int} > \ac{ext}}
		\begin{tabular}[b]{cc}
      \toprule
      \ac{int}  &   \ac{ext} \\ \cmidrule{1-2}
      \begin{forest} boom
        [\tsc{rel}P
            [\tsc{wh}P
                [\tit{k-}, roof]
            ]
            [\tsc{dat}P,
            tikz={
            \node[label=below:\tit{-omu},
            draw,circle,
            scale=0.85,
            fit to=tree]{};
            }
                [\tsc{f}3]
                [\tsc{acc}P
                    [\tsc{f}2]
                    [\tsc{nom}P
                        [\tsc{f1}]
                        [\tsc{dP}
                            [\tsc{d}]
                            [ϕP
                                [\phantom{xxx},
                                roof, baseline
                                ]
                            ]
                        ]
                    ]
                ]
            ]
        ]
      \end{forest}
      &
      \begin{forest} boom
        [\tsc{acc}P,
        tikz={
        \node[label=below:\tit{togo},
        draw,circle,
        scale=0.8,
        fit to=tree]{};
        }
            [\tsc{f2}]
            [\tsc{nom}P
                [\tsc{f1}]
                    [ϕP
                    [\phantom{xxx},
                    roof, baseline
                    ]
                ]
            ]
        ]
      \end{forest}\\
      \bottomrule
  \end{tabular}
  \label{tbl:polish-int-wins}
\end{table}


\begin{table}[H]
  \center
	\caption {Polish: \ac{int} < \ac{ext}}
		\begin{tabular}[b]{cc}
      \toprule
      \ac{int}  &   \ac{ext} \\ \cmidrule{1-2}
      \begin{forest} boom
        [\tsc{rel}P
            [\tsc{wh}P
                [\tit{k-}, roof]
            ]
            [\tsc{acc}P,
            tikz={
            \node[label=below:\tit{tego},
            draw,circle,
            scale=0.8,
            fit to=tree]{};
            }
                [\tsc{f2}]
                [\tsc{nom}P
                    [\tsc{f1}]
                        [ϕP
                        [\phantom{xxx},
                        roof, baseline
                        ]
                    ]
                ]
            ]
        ]
      \end{forest}
      &
      \begin{forest} boom
        [\tsc{acc}P,
        tikz={
        \node[label=below:\tit{tego},
        draw,circle,
        scale=0.85,
        fit to=tree]{};
        }
            [\tsc{f}2]
            [\tsc{nom}P
                [\tsc{f1}]
                [\tsc{dP}
                    [\tsc{d}]
                    [ϕP
                        [\phantom{xxx},
                        roof, baseline
                        ]
                    ]
                ]
            ]
        ]
      \end{forest}\\
      \bottomrule
  \end{tabular}
  \label{tbl:polish-ext-wins}
\end{table}

Radek with his definitnessless of Czech demonstratives


\subsection{Excluding external-only}


x







\section{Alternative analyses}\label{sec:alternative-analyses}

\subsection{Himmelreich}

She specific languages for having different types of agree (up, down) and different types of probes (active, non-active). Doing that, she successfully derived free relatives and parasitic gaps in different languages.

\subsection{Grosu}

Grosu 1994 linked richness of inflection to liberality. He actually talked about the richness of pro.

\subsection{Grafting story}

For this pattern a single element analysis seems intuitive, if you assume that case is complex and that syntax works bottom-up. First you built the relative clause, with the big case in there. Then you build the main clause and you let the more complex case in the embedded clause license the main clause predicate.

Consider the example in \ref{ex:mg-nom-acc-grafting}. Here the internal case is accusative and the external one nominative.

\exg. Uns besucht \tbf{wen} \tbf{Maria} \tbf{mag}.\\
 we.\ac{acc} visit.3\ac{sg}\scsub{[nom]} \tsc{rel}.\ac{acc}.\tsc{an} Maria.\ac{nom} like.3\ac{sg}\scsub{[acc]}\\
 `Who visits us, Maria likes.' \flushfill{adapted from \pgcitealt{vogel2001}{343}}\label{ex:mg-nom-acc-grafting}

The relative clause is built, including the accusative relative pronoun. Now the main clause predicate can merge with the nominative that is contained within the accusative.

 \ex.
 \begin{forest} boom
	 [,name=src, s sep=15mm
			[VP
			 		[\tit{besucht}, roof]
			]
		 	[,no edge, s sep=20mm
	       [\ac{acc}P,
				 tikz={
				 \node[label=below:\tit{wen},
				 draw,circle,
				 scale=0.85,
				 fit to=tree]{};
				 }
	           [\tsc{f2}]
	           [\tsc{nomP},name=tgt
	               [\tsc{f1}]
	               [XP
	                   [\phantom{xxx}, roof]
	               ]
	           ]
	       ]
				 [VP
				 		 [\tit{Maria mag}, roof]
				 ]
			]
	 ]
	 \draw (src) to[out=south east,in=north east] (tgt);
 \end{forest}\label{ex:acc-nom-grafting}

The other way around does not work. Consider \ref{ex:mg-acc-nom-grafting}. This is an example with nominative as internal case and accusative as external case.

\exg. *Ich {lade ein}, wen \tbf{mir} \tbf{sympathisch} \tbf{ist}.\\
I.\ac{nom} invite.1\ac{sg}\scsub{[acc]} \tsc{rel}.\ac{acc}.\tsc{an} I.\ac{dat} nice be.3\ac{sg}\scsub{[nom]}\\
`I invite who I like.' \flushfill{adapted from \pgcitealt{vogel2001}{344}}\label{ex:mg-acc-nom-grafting}

Now the relative clause is built first again, this time only including the nominative case. There is no accusative node to merge with for the external predicate. Instead, the relative pronoun would need to grow to accusative somehow and then the merge could take place. This is the desired result, because the sentence is ungrammatical.

\ex.
\begin{forest} boom
  [,name=src, s sep=15mm
     [VP
         [\tit{lade ein}, roof]
     ]
         [,no edge
    			[\tsc{nomP},
    			tikz={
    			\node[label=below:\tit{wer},
    			draw,circle,
    			scale=0.85,
    			fit to=tree]{};
    			}
    					[\tsc{f1}]
    					[XP
    							[\phantom{xxx}, roof]
    					]
    			]
    			[VP
    					[\tit{mir sympatisch ist}, roof]
    			]
    	 ]
    ]
\end{forest}\label{ex:nom-acc-grafting}

So, this seems to work fine. The assumptions you have to do in order to make this are the following. First, case is complex. Second, you can remerge an embedded node (grafting). For the first one I have argued in Chapter \ref{ch:decomposition}. The second one could use some additional argumentation. It is a mix between internal remerge (move) and external merge, namely external remerge. Other literature on multidominance and grafting, other phenomena. Problems: linearization, .. But even if fix all these theoretical problems, there is an empirical one.

That is, I want to connect this behavior of Modern German headless relatives to the shape of its relative pronouns. These pronouns are \tsc{wh}-elements. The OHG and Gothic ones are not \tsc{wh}, they are \tsc{d}. Their relative pronouns look different, and so their headless relatives can also behave differently.

\section{Summary}

here
