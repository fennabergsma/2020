% !TEX root = thesis.tex

\chapter{Constituent containment}\label{ch:relativization}

In Chapter \ref{ch:typology} I introduced two descriptive parameters that generate the attested languages, as shown in Figure \ref{fig:two-parameters}.
The first parameter concerns whether the external case is allowed to surface when it wins the case competition (allow \tsc{ext}?). This parameter distinguishes between unrestricted languages (e.g. Old High German) on the one hand and internal-only languages (e.g. Modern German) and matching languages (e.g. Polish) on the other hand.
The second parameter concerns whether the internal case is allowed to surface when it wins the case competition (allow \tsc{int?}). This parameter distinguishes between internal-only languages (e.g. as Modern German) on the one hand and unrestricted languages (e.g. Old High German) on the other hand.

\begin{figure}[htbp]
  \centering
    \footnotesize{
    \begin{tikzpicture}[node distance=1.5cm]
      \node (question2) [question]
      {allow \tsc{int}?};
          \node (outcome2) [outcome, below of=question2, xshift=-1.5cm]
          {matching};
              \node (example2) [example, below of=outcome2, yshift=0.25cm]
              {\scriptsize{e.g. Polish\\\phantom{x}}};
          \node (question3) [question, below of=question2, xshift=2cm, yshift=-0.5cm]
          {allow \tsc{ext}?};
              \node (outcome3) [outcome, below of=question3, xshift=-1.5cm]
              {internal-only};
                  \node (example3) [example, below of=outcome3, yshift=0.25cm]
                  {\scriptsize{e.g. Modern German\\\phantom{x}}};
              \node (outcome4) [outcome, below of=question3, xshift=1.5cm]
              {un-restricted};
                  \node (example4) [example, below of=outcome4, yshift=0.25cm]
                  {\scriptsize{e.g. Gothic, Old High German, Classical Greek}};

    \draw [arrow] (question2) -- node[anchor=east] {no} (outcome2);
    \draw [arrow] (question2) -- node[anchor=west] {yes} (question3);
    \draw [arrow] (question3) -- node[anchor=east] {no} (outcome3);
    \draw [arrow] (question3) -- node[anchor=west] {yes} (outcome4);
    \end{tikzpicture}
    }
    \caption{Two descriptive parameters generate three language types}
    \label{fig:two-parameters}
\end{figure}

``A natural question at this point is whether this typology needs to be fully stipulative, or is to some extent derivable from independent properties of individual languages'' \citet{grosu1994}{147}

In this chapter I show how the typology can be derived from the morphology of the languages.

This chapter is structured as follows.


\section{The basic idea}\label{sec:basic-idea}

This section gives the basic idea behind my proposal. Throughout the rest of the chapter I motivate the proposal, and I illustrate it with examples.

\subsection{Underlying assumptions}

I start with my assumption that headless relatives are derived from light-headed relatives.\footnote{
The same is argued for headless relatives with \tsc{d}-pronouns in Modern German by \citealt{fuss2014,hanink2018} and for Polish by \citealt{citko2004}.
A difference with Modern German and Polish is that one of the elements can only be absent when the cases match. In Section \ref{ch:discussion} I return to the point why Modern German does not have unrestricted headless relatives that look like Old High German, although it still has syncretic light heads and relative pronouns.

Several others claim that headless relatives have a head, but that it is phonologically empty, cf. \citealt{bresnan1978,groos1981,himmelreich2017}.
}
The light head bears the external case, and the relative pronoun bears the internal case, as illustrated in \ref{ex:light+rel}.

\ex. light head\scsub{ext} [relative pronoun\scsub{int} ... ]\label{ex:light+rel}

In a headless relative, either the light head or the relative pronoun is absent.
This happens under the following condition: a light head or a relative pronoun is absent when each of its constituents is contained in a constituent of the other element (i.e. the light head or the relative pronoun).

To see what a light-headed relative looks like, consider the light-headed relative in \ref{ex:ohg-light-headed}.
\tit{Thér} `\tsc{dem}.\tsc{sg}.\tsc{m}.\tsc{nom}' is the light head of the relative clause. This is the element that appears in the external case, the case that reflects the grammatical role in the main clause.
\tit{Then} `\tsc{rel}.\tsc{sg}.\tsc{m}.\tsc{acc}' is the relative pronoun in the relative clause. This is the element that appears in the internal case, the case that reflects the grammatical role within the relative clause.

\exg. eno nist thiz thér \tbf{then} \tbf{ir} \tbf{suochet} \tbf{zi} \tbf{arslahanne}?\\
 now {not be.3\ac{sg}} \tsc{dem}.\tsc{sg}.\tsc{n}.\tsc{nom} \tsc{dem}.\tsc{sg}.\tsc{m}.\tsc{nom}
 \tsc{rel}.\tsc{sg}.\tsc{m}.\tsc{acc} 2\ac{pl}.\tsc{nom} seek.2\tsc{pl} to kill.\tsc{inf}.\ac{sg}.\tsc{dat}\\
 `Isn't this now the one, who you seek to kill?'\label{ex:ohg-light-headed}

The difference between a light-headed relative and a headless relative is that in a headless relative either the light head or the relative pronoun does not surface.
The surfacing element is the one that bears the winning case, and the absent element is the one that bears the losing case. This means that what I have so far been glossing as and calling the relative pronoun is actually sometimes the light head and sometimes the relative pronoun. To reflect that, I call the surfacing element from now on the surface pronoun.

Table \ref{tbl:options-surface-pronoun} lists the two options that I just laid out plus an additional one.
The first option is that the relative pronoun, which bears the internal case, can appear as the surface pronoun. The second option is that the light head, which bears the external case, can appear as the surface pronoun. The third option is that there is no grammatical form for the surface pronoun.

\begin{table}[htbp]
  \center
  \caption{Options for the surface pronoun}
\begin{tabular}{ccc}
  \toprule
surface pronoun             \\
\cmidrule(lr){1-1}
light head\scsub{ext}       \\
relative pronoun\scsub{int} \\
{*}                         \\
\bottomrule
\end{tabular}
\label{tbl:options-surface-pronoun}
\end{table}

I propose that whether or the surface pronoun is the light head, the relative pronoun or none of them depends on whether one of the elements (i.e. the light head or the relative pronoun) can delete the other.
The light head appears as the surface pronoun when the light head can delete the relative pronoun. The relative pronoun appears as the surface pronoun when the relative pronoun can delete the light head. There is no grammatical surface pronoun possible when neither of them can delete the other one.

Whether or not one element can delete the other depends on the comparison between the two. Specifically, I compare the constituents within light heads and relative pronouns to each other. Light heads and relative pronouns do not only correspond to case features, but also to other features (having to do with number, gender, etc.). It differs per language how language organize these features into constituents. In this chapter, I illustrate how these different constituents within light heads and relative pronouns lead to the differences in whether or not the light head and the relative pronoun can be deleted, and therefore to different language types.

In order to be able to compare the light head and the relative pronoun, I zoom in on their syntactic structures. In Section \ref{sec:deriving-only-internal} to \ref{sec:deriving-nonmatching} I give arguments to support the structures I am assuming here. Figure \ref{fig:rel-lh-structure} gives a simplified representation of them.\footnote{
The structures in Figure \ref{fig:rel-lh-structure} are not base structures but derived ones. I assume the base structure of the light head to be as in \ref{ex:base-light-head} and the base structure of the relative pronoun to be as in \ref{ex:base-relative-pronoun}.

\ex.
\a.\label{ex:base-light-head}
  \begin{forest} boom
    [\tsc{k}P,
        [\tsc{k}]
        [ϕP
            [\phantom{x}ϕ\phantom{x}, roof]
        ]
    ]
  \end{forest}
\b.\label{ex:base-relative-pronoun}
  \begin{forest} boom
    [\tsc{k}P
        [\tsc{k}]
        [\tsc{rel}P
            [\tsc{rel}]
            [ϕP
                [\phantom{x}ϕ\phantom{x}, roof]
            ]
        ]
    ]
  \end{forest}

The structure for the relative pronoun in Figure \ref{fig:rel-lh-structure} cannot be derived from the base structures in \ref{ex:base-relative-pronoun}. It is a simplification of a more complex situation for which I only give the intuition here.

In Section \ref{sec:deriving-only-internal} I show the actual decomposition of the light head and the relative pronoun and how I reach the derived structure. I work with the derived structure in the main text because this is the configuration in which the containment relations under discussion hold.
}
The light head and the relative pronoun partly contain the same syntactic features. The features they have in common are case (\tsc{k}) and what I here simplify as phi-features (ϕ). The light head and the relative pronoun differ from each other in that the relative pronoun in addition has a relative feature (\tsc{rel}).

\begin{figure}[htbp]
  \center
  \begin{tabular}[b]{ccc}
      \toprule
      light head & & relative pronoun \\
      \cmidrule(lr){1-1} \cmidrule(lr){3-3}
      \begin{forest} boom
      [\tsc{k}P
          [ϕP
              [\phantom{x}ϕ\phantom{x}, roof]
          ]
          [\tsc{k}P,
              [\tsc{k}]
          ]
      ]
      \end{forest}
      & \phantom{x} &
    \begin{forest} boom
      [\tsc{rel}P
          [\tsc{rel}]
          [\tsc{k}P
              [ϕP
                  [\phantom{x}ϕ\phantom{x}, roof]
              ]
              [\tsc{k}P,
                  [\tsc{k}]
              ]
          ]
      ]
    \end{forest}\\
      \bottomrule
  \end{tabular}
   \caption {Light head and relative pronoun}
  \label{fig:rel-lh-structure}
\end{figure}

I compare the light head and the relative pronoun in terms of containment. The relative pronoun can delete the light head because the relative pronoun contains all constituents the light head contains.
I illustrate this in Figure \ref{fig:rel-lh-structure-containment}. I draw a dashed circle around the constituent that is a constituent in both the light head and the relative pronoun.
The \tsc{k}P is contained in the \tsc{rel}P, so the relative pronoun can delete the light head. I illustrate this by marking the content of the dashed circle for the \tsc{k}P gray.

\begin{figure}[htbp]
  \center
  \begin{tabular}[b]{ccc}
      \toprule
      light head & & relative pronoun \\
      \cmidrule(lr){1-1} \cmidrule(lr){3-3}
      \begin{forest} boom
        [\tsc{k}P,
        tikz={
        \node[draw,circle,
        dashed,
        scale=0.85,
        fill=DG,fill opacity=0.2,
        fit to=tree]{};
        }
            [ϕP
                [\phantom{x}ϕ\phantom{x}, roof]
            ]
            [\tsc{k}P,
                [\tsc{k}]
            ]
        ]
      \end{forest}
      & \phantom{x} &
      \begin{forest} boom
        [\tsc{rel}P, s sep=15mm
            [\tsc{rel}]
            [\tsc{k}P,
            tikz={
            \node[draw,circle,
            dashed,
            scale=0.85,
            fit to=tree]{};
            }
                [ϕP
                    [\phantom{x}ϕ\phantom{x}, roof]
                ]
                [\tsc{k}P
                    [\tsc{k}]
                ]
            ]
        ]
      \end{forest}\\
      \bottomrule
  \end{tabular}
   \caption {Light head and relative pronoun}
  \label{fig:rel-lh-structure-containment}
\end{figure}

The light head cannot delete the relative pronoun, because it does not contain all constituents of the relative pronoun.
The light head has a constituent \tsc{k}P, but it does not contain the feature \tsc{rel} to make it an \tsc{rel}P.

With the set of assumptions I introduced in this section, I can account for the internal-only type of language. Moreover, the system I set up excludes the external-only type of language. An external-only type of language would be one in which the light head can delete the relative pronoun, but the relative pronoun cannot delete the light head. In my proposal, an element can the delete the other one if it contains all of the other's constituents. Relative pronouns always contain one more feature than light heads: \tsc{rel}. From that it follows that the light head does not contain all features that the relative pronoun contains. Therefore, it is impossible for a light head to contain all constituents of the relative pronoun.

However, not all languages are of the internal-only type. I argue that the other two attested languages differ from the internal-only type in how light heads and relative pronouns are spelled out. Before I come back to how the different spellout leads to different language types, I show how the internal-only type fares with differing internal and external cases.


\subsection{The internal-only type}

I start with the example in Figure \ref{fig:rel-acc-lh-nom-structure}, in which the relative pronoun bears a more complex case than the light head.

\begin{figure}[htbp]
  \center
  \begin{tabular}[b]{ccc}
      \toprule
      light head & & relative pronoun \\
      \cmidrule(lr){1-1} \cmidrule(lr){3-3}
      \begin{forest} boom
        [\tsc{nom}P, s sep=15mm
            [ϕP,
            tikz={
            \node[draw,circle,
            dashed,
            scale=0.8,
            fill=DG,fill opacity=0.2,
            fit to=tree]{};
            }
                [\phantom{x}ϕ\phantom{x}, roof]
            ]
            [\tsc{nom}P,
            tikz={
            \node[draw,circle,
            dashed,
            scale=0.8,
            fill=DG,fill opacity=0.2,
            fit to=tree]{};
            }
                [\tsc{f}1]
            ]
        ]
      \end{forest}
      & \phantom{x} &
      \begin{forest} boom
        [\tsc{rel}P
            [\tsc{rel}]
            [\tsc{acc}P
                [ϕP,
                tikz={
                \node[draw,circle,
                dashed,
                scale=0.8,
                fit to=tree]{};
                }
                    [\phantom{x}ϕ\phantom{x}, roof]
                ]
                [\tsc{acc}P
                    [\tsc{f}2]
                    [\tsc{nom}P,
                    tikz={
                    \node[draw,circle,
                    dashed,
                    scale=0.8,
                    fit to=tree]{};
                    }
                        [\tsc{f}1]
                    ]
                ]
            ]
        ]
      \end{forest}\\
      \bottomrule
  \end{tabular}
   \caption {\tsc{nom} light head and \tsc{acc} relative pronoun}
  \label{fig:rel-acc-lh-nom-structure}
\end{figure}

I draw a dashed circle around each constituent that is a constituent in both the light head and the relative pronoun. There are two separate constituents.
I start with the right-most constituent of the light head: \tsc{nom}P. This constituent is also a constituent in the relative pronoun, contained in the lower \tsc{acc}P.
I continue with the left-most constituent of the light head: the ϕP. This constituent is also a constituent in the relative pronoun, contained in the higher \tsc{acc}P.
As each constituent of the light head is also a constituent within the relative pronoun, the light head can be absent. I illustrate this by marking the content of the dashed circles for the light head gray.

I continue with the example in Figure \ref{fig:rel-nom-lh-acc-structure}, in which the light head bears a more complex case than the relative pronoun.

\begin{figure}[htbp]
  \center
  \begin{tabular}[b]{ccc}
      \toprule
      light head & & relative pronoun \\
      \cmidrule(lr){1-1} \cmidrule(lr){3-3}
      \begin{forest} boom
        [\tsc{acc}P
            [ϕP,
            tikz={
            \node[draw,circle,
            dashed,
            scale=0.8,
            fit to=tree]{};
            }
                [\phantom{x}ϕ\phantom{x}, roof]
            ]
            [\tsc{acc}P
                [\tsc{f}2]
                [\tsc{nom}P,
                tikz={
                \node[draw,circle,
                dashed,
                scale=0.8,
                fit to=tree]{};
                }
                    [\tsc{f}1]
                ]
            ]
        ]
      \end{forest}
      & \phantom{x} &
      \begin{forest} boom
        [\tsc{rel}P
            [\tsc{rel}]
            [\tsc{nom}P, s sep=15mm
                [ϕP,
                tikz={
                \node[draw,circle,
                dashed,
                scale=0.8,
                fit to=tree]{};
                }
                    [\phantom{x}ϕ\phantom{x}, roof]
                ]
                [\tsc{nom}P,
                tikz={
                \node[draw,circle,
                dashed,
                scale=0.8,
                fit to=tree]{};
                }
                    [\tsc{f}1]
                ]
            ]
        ]
      \end{forest}\\
      \bottomrule
  \end{tabular}
   \caption { \tsc{nom} relative pronoun and \tsc{acc} light head}
  \label{fig:rel-nom-lh-acc-structure}
\end{figure}

I draw a dashed circle around each constituent that is a constituent in both the light head and the relative pronoun. Different from the example in Figure \ref{fig:rel-acc-lh-nom-structure}, neither of the elements contains all of the other's constituents.
The relative pronoun has a constituent \tsc{nom}P, but it lacks the \tsc{f}2 to make it an \tsc{acc}P. The light head has a constituent that is not a constituent in the relative pronoun, so the light head cannot be absent.
The light head has a constituent \tsc{nom}P, but it does not contain \tsc{rel} to make it a \tsc{rel}P. The relative pronoun has a constituent that is not a constituent in the light head, so the relative pronoun cannot be absent.
As a result, none of the elements can be absent.

Now I return to the other two attested language types.
The differences between the languages do not arise from changing the feature content of the light head and relative pronoun per language.\footnote{
The feature content of the unrestricted languages differs slightly from that of the internal-only and matching languages. This is due to the fact that this language type uses a different type of relative pronoun. The basic idea of the relative pronoun having at least one more feature than the light head remains the same.
}
Instead, the differences come from how the light heads and relative pronouns are spelled out.

\subsection{The matching type}

In matching languages like Polish, the light head cannot delete the relative pronoun and the relative pronoun cannot delete the light head. The intuition for this type of language is that they package their features together differently from internal-only languages like Modern German. The packaging happens in such a way that the constituents of the relative pronoun do not contain the constituents of the light head. As a result, the relative pronoun cannot delete the light head anymore. This account crucially relies on constituent containment being the containment requirement that needs to be fulfilled. Feature containment is too weak of a requirement.

I illustrate the difference between feature and constituent containment with two structures. In Figure \ref{fig:acc-nom-structure}, I repeated the light head and relative pronoun from Figure \ref{fig:rel-acc-lh-nom-structure}.

\begin{figure}[htbp]
  \center
  \begin{tabular}[b]{ccc}
      \toprule
      light head & & relative pronoun \\
      \cmidrule(lr){1-1} \cmidrule(lr){3-3}
      \begin{forest} boom
        [\tsc{k}P,
        tikz={
        \node[draw,circle,
        dashed,
        scale=0.8,
        fill=DG,fill opacity=0.2,
        fit to=tree]{};
        }
            [\tsc{k}]
            [ϕP
                [\phantom{x}ϕ\phantom{x}, roof]
            ]
        ]
      \end{forest}
      & \phantom{x} &
      \begin{forest} boom
        [\tsc{acc}P
            [\tsc{f}2]
            [\tsc{nom}P,
            tikz={
            \node[draw,circle,
            dashed,
            scale=0.8,
            fit to=tree]{};
            }
                [\tsc{f}1]
                [XP
                    [\phantom{xxx}, roof]
                ]
            ]
        ]
      \end{forest}\\
      \bottomrule
  \end{tabular}
   \caption {\tsc{lh} vs. \tsc{rel} → \tsc{rel} (repeated)}
  \label{fig:acc-nom-structure}
\end{figure}

In Figure \ref{fig:acc-nom-structure}, two different types of containment hold: feature containment and constituent containment.
I start with feature containment. Each feature of the \tsc{k}P (i.e. ϕ and \tsc{k}) is also a feature within the \tsc{rel}P, so the \tsc{rel}P contains the \tsc{k}P.
Constituent containment works as follows. Each constituent of the \tsc{k}P (i.e. ϕP and \tsc{k}P that contains \tsc{k} and ϕP) is also a constituent of the \tsc{k}P. Therefore, \tsc{rel}P contains contains the \tsc{k}P.

Constituent containment is a stronger requirement than feature containment. In Figure \ref{fig:acc-nom-structure-moved-out} I show a situation in which the feature containment requirement holds but the constituent containment requirement does not. It is the same picture as in Figure \ref{fig:acc-nom-structure} except for that the ϕP has moved out of the \tsc{rel}P.

\begin{figure}[htbp]
  \center
  \begin{tabular}[b]{ccc}
      \toprule
      light head & & relative pronoun \\
      \cmidrule(lr){1-1} \cmidrule(lr){3-3}
      \begin{forest} boom
        [\tsc{k}P
            [\tsc{k}]
            [ϕP,
            tikz={
            \node[draw,circle,
            dashed,
            scale=0.8,
            fit to=tree]{};
            }
                [\phantom{x}ϕ\phantom{x}, roof]
            ]
        ]
      \end{forest}
      & \phantom{x} &
      \begin{forest} boom
        [\tsc{rel}P
            [ϕP,
            tikz={
            \node[draw,circle,
            dashed,
            scale=0.8,
            fit to=tree]{};
            }
                [\phantom{x}ϕ\phantom{x}, roof]
            ]
            [\tsc{rel}P
                [\tsc{rel}]
                [\tsc{k}P
                    [\tsc{k}]
                ]
            ]
        ]
      \end{forest}\\
      \bottomrule
  \end{tabular}
   \caption {\tsc{lh} vs. \tsc{rel} after extraction ↛ \tsc{rel}}
  \label{fig:acc-nom-structure-moved-out}
\end{figure}

There is still feature containment: the \tsc{k}P contains ϕ and \tsc{k} and so does the \tsc{rel}P.
However, there is no longer constituent containment: the \tsc{k}P constituent containing ϕP and \tsc{k}P that contains \tsc{k} and ϕP is no longer a constituent within the \tsc{rel}P.

In Section \ref{sec:deriving-matching} I show that only the stronger requirement of constituent containment is able to distinguish the internal-only from the matching type of language, and that the weaker requirement of feature containment is not.

Constituent containment is also what seems to be crucial in the deletion of nominal modifiers. Cinque argues that nominal modifiers can only be absent if they form a constituent with the NP \citep{cinqueforthcoming}. If they are not, they can also not be interpreted.

In \ref{ex:dutch-houses}, I give an example of a conjunction with two noun phrases in Dutch. The first conjunct consists of a demonstrative, an adjective and a noun, and the second one only of a demonstrative.

\exg. deze witte huizen en die\\
 these white houses and those\\
 `these white houses and those white houses' \flushfill{Dutch}\label{ex:dutch-houses}

The adjective \tit{witte} `white' forms a constituent with \tit{huizen} `houses'. I showed this in Figure \ref{fig:dutch-houses} under first conjunct. In the second conjunct, the constituent with the adjective and the noun in it is deleted. The adjective can still be interpreted in \ref{ex:dutch-houses}, because it forms a constituent with the noun.

 \begin{figure}[htbp]
   \center
   \begin{tabular}[b]{ccc}
       \toprule
       first conjunct & & second conjunct \\
       \cmidrule(lr){1-1} \cmidrule(lr){3-3}
       \begin{forest} boom
         [
             [Dem]
             [
                 [A]
                 [N]
             ]
         ]
       \end{forest}
       & \phantom{x} &
       \begin{forest} boom
         [
             [Dem]
             [
                 [\sout{A}]
                 [\sout{N}]
             ]
         ]
       \end{forest}\\
       \bottomrule
   \end{tabular}
    \caption {Nominal ellipsis in Dutch}
   \label{fig:dutch-houses}
 \end{figure}

The situation is different in Kipsigis, a Nilotic Kalenjin language spoken in Kenya. In \ref{ex:kipsigis-houses}, I give an example of a conjunction of two noun phrases in Kipsigis. The first conjunct consists of a noun, a demonstrative and an adjective, and the second one only of a demonstrative \citep{cinqueforthcoming}.

\exg. kaarii-chuun leel-ach ak chu\\
houses-those white-\tsc{pl} and these\\
`those white houses and these houses'\\
not: `those white houses and these white houses'\label{ex:kipsigis-houses} \flushfill{Kipsigis, \pgcitealt{cinqueforthcoming}{24}}

The adjective \tit{leel} `white' does not forms a constituent with \tit{kaarii} `houses'. I showed this in Figure \ref{fig:kipsigis-houses} under first conjunct. In the second conjunct, the adjective and the noun are deleted. Different from the Dutch example in \ref{fig:dutch-houses}, this is not a single constituent. The adjective cannot be interpreted in \ref{ex:kipsigis-houses}, because it does not form a constituent with the noun.

\begin{figure}[htbp]
  \center
  \begin{tabular}[b]{ccc}
      \toprule
      first conjunct & & second conjunct \\
      \cmidrule(lr){1-1} \cmidrule(lr){3-3}
      \begin{forest} boom
        [
            [
                [NP]
                [Dem]
            ]
            [AP]
        ]
      \end{forest}
      & \phantom{x} &
      \begin{forest} boom
        [
            [
                [\sout{N}]
                [Dem]
            ]
            [\sout{A}]
        ]
      \end{forest}\\
      \bottomrule
  \end{tabular}
   \caption {Nominal ellipsis in Kipsigis}
   \label{fig:kipsigis-houses}
\end{figure}

To sum up, the comparison between light heads requires constituent containment. Feature containment is not enough.


\subsection{The unrestricted type}

In unrestricted languages like Old High German, the light head can delete the relative pronoun and the relative pronoun can delete the light head. The property of unrestricted languages that I connect to this behavior is that their light heads and relative pronoun are syncretic. I suggest that if there is no constituent containment, but the two forms are spelled out by the same morpheme, one element can still delete the other.
Consider Figure \ref{fig:rel-lh-syncretism1}, in which the relative pronoun deletes the light head.

\begin{figure}[htbp]
  \center
  \begin{tabular}[b]{ccc}
      \toprule
      light head & & relative pronoun \\
      \cmidrule(lr){1-1} \cmidrule(lr){3-3}
      \begin{forest} boom
        [\tsc{k}P, s sep=20mm
            [ϕP,
            tikz={
            \node[label=below:\tit{α},
            draw,circle,
            scale=0.85,
            fit to=tree]{};
            \node[draw,circle,
            dashed,
            scale=0.9,
            fill=DG,fill opacity=0.2,
            fit to=tree]{};
            }
                [\phantom{x}ϕ\phantom{x}, roof]
            ]
            [\tsc{k}P,
            tikz={
            \node[draw,circle,
            dashed,
            scale=0.8,
            fill=DG,fill opacity=0.2,
            fit to=tree]{};
            }
                [\tsc{k}]
            ]
        ]
      \end{forest}
      & \phantom{x} &
      \begin{forest} boom
        [\tsc{k}P, s sep=20mm
            [\tsc{rel}P,
            tikz={
            \node[label=below:\tit{α},
            draw,circle,
            scale=0.95,
            fit to=tree]{};
            }
                [\tsc{rel}]
                [ϕP,
                tikz={
                \node[draw,circle,
                dashed,
                scale=0.8,
                fit to=tree]{};
                }
                    [\phantom{x}ϕ\phantom{x}, roof]
                ]
            ]
            [\tsc{k}P,
            tikz={
            \node[draw,circle,
            dashed,
            scale=0.8,
            fit to=tree]{};
            }
                [\tsc{k}]
            ]
        ]
      \end{forest}\\
      \bottomrule
  \end{tabular}
   \caption {Syncretism: relative pronoun deletes light head}
  \label{fig:rel-lh-syncretism1}
\end{figure}

The ϕP in the light head is spelled out as \tit{α}, illustrated by the circle around the ϕP and the \tit{α} under it. The \tsc{rel}P in the relative pronoun is spelled out as \tit{α} too, illustrated in the same way. I draw a dashed circle around each constituent that is a constituent in both the light head and the relative pronoun.

I start with the right-most constituent of the light head: \tsc{k}P. This constituent is also a constituent in the relative pronoun.
I continue with the left-most constituent of the light head: the ϕP. This constituent is also a constituent in the relative pronoun, contained in the \tsc{rel}P.
As each constituent of the light head is also a constituent within the relative pronoun, the light head can be absent. I illustrate this by marking the content of the dashed circles for the light head gray.

Consider Figure \ref{fig:rel-lh-syncretism2}, in which the light head deletes the relative pronoun.

\begin{figure}[htbp]
  \center
  \begin{tabular}[b]{ccc}
      \toprule
      light head & & relative pronoun \\
      \cmidrule(lr){1-1} \cmidrule(lr){3-3}
      \begin{forest} boom
        [\tsc{k}P, s sep=20mm
            [ϕP,
            tikz={
            \node[label=below:\tit{α},
            draw,circle,
            scale=0.85,
            fit to=tree]{};
            \node[draw,circle,
            dashed,
            scale=0.9,
            fit to=tree]{};
            }
                [\phantom{x}ϕ\phantom{x}, roof]
            ]
            [\tsc{k}P,
            tikz={
            \node[draw,circle,
            dashed,
            scale=0.8,
            fit to=tree]{};
            }
                [\tsc{k}]
            ]
        ]
      \end{forest}
      & \phantom{x} &
      \begin{forest} boom
        [\tsc{k}P, s sep=20mm
            [\tsc{rel}P,
            tikz={
            \node[label=below:\tit{α},
            draw,circle,
            scale=0.95,
            fill=LG,fill opacity=0.2,
            fit to=tree]{};
            }
                [\tsc{rel}]
                [ϕP,
                tikz={
                \node[draw,circle,
                dashed,
                scale=0.8,
                fill=DG,fill opacity=0.2,
                fit to=tree]{};
                }
                    [\phantom{x}ϕ\phantom{x}, roof]
                ]
            ]
            [\tsc{k}P,
            tikz={
            \node[draw,circle,
            dashed,
            scale=0.8,
            fill=DG,fill opacity=0.2,
            fit to=tree]{};
            }
                [\tsc{k}]
            ]
        ]
      \end{forest}\\
      \bottomrule
  \end{tabular}
   \caption {Syncretism: light head deletes relative pronoun}
  \label{fig:rel-lh-syncretism2}
\end{figure}

Just as in Figure \ref{fig:rel-lh-syncretism1}, the ϕP in the light head is spelled out as \tit{α} and the \tsc{rel}P in the relative pronoun is spelled out as \tit{α} too. I draw a dashed circle around each constituent that is a constituent in both the light head and the relative pronoun.

I start with the right-most constituent of the relative pronoun: \tsc{k}P. This constituent is also a constituent in the relative pronoun.
I continue with the left-most constituent of the relative pronoun: the \tsc{rel}P. This constituent is not contained in the light head. The ϕP lacks the \tsc{rel} to make it a \tsc{rel}P. However, the two constituents are syncretic: the ϕP is also spelled out as \tit{α}. I suggest that this syncretism is also enough to license the deletion. I illustrate this by marking the content of the dashed circles for the relative pronoun gray and the portion that is deleted by syncretism in a lighter shade of gray.

To sum up, each constituent of the relative pronoun is either also a constituent within the light head or it is syncretic with a constituent within the light head. Therefore, the relative pronoun can be absent. The fact that syncretism licenses deletion is not specific to the portion of the structure that corresponds to ϕ and \tsc{rel}. Syncretic cases can have the same effect, the inanimate nominative and accusative in Modern German being an instance of it. I give examples of this in Section \ref{sec:deriving-nonmatching}.

\subsection{Everything is constituent containment}

In summing up this section, I return to the metaphor with the committee that I introduced in Chapter \ref{ch:typology}. I wrote that first case competition takes place, in which a more complex case wins over a less complex case. This case competition can now be reformulated into a more general mechanism, namely constituent comparison. A more complex case corresponds to a constituent that contains the constituent of a less complex case.

Subsequently, I noted that there is a committee that can either approve the winning case or not approve it. In Chapter \ref{ch:typology} I wrote that the approval happens based on where the winning case comes from: from inside of the relative clause (internal) or from outside of the relative clause (external). I argued in this section that headless relatives are derived from light-headed relatives. The light head bears that external case and the relative pronoun bears the internal case. The `approval' of an internal or external case relies on the same mechanism as case competition, namely constituent comparison. If each constituent of the light head is contained in a constituent of the relative pronoun, the relative pronoun can delete the light head. The light head with its external case is absent, and the relative pronoun with its internal case surfaces. This is what corresponds to the the internal case `being allowed to surface'. If each constituent of the relative pronoun is contained in a constituent of the light head, the light head can delete the relative pronoun. The relative pronoun with its internal case is absent, and the light head with its external case surfaces. This is what corresponds to the the external case `being allowed to surface'.

In other words, the grammaticality of a headless relative depends on several instances of constituent comparison. The constituents that are compared are those of the light head and the relative pronoun, which both bear their own case. Case is special in that it can differ from sentence to sentence within a language. Therefore, the grammaticality of a sentence can differ within a language depending on the internal and external case. The part of the light head and relative pronoun that does not involve case features is stable within a language. Therefore, whether the internal or external case is `allowed to surface' does not differ within a language.

In this dissertation I describe different language types in case competition in headless relatives. In my account, the different language types are a result of a comparison of the light head and the relative pronoun in the language.
The larger syntactic context in which this takes place should be kept stable. The operation that deletes the light head or the relative pronoun is the same for all language types. In this work, I do not specify on which larger syntactic structure and which deletion operation should be used. In Section \ref{sec:larger-syntax} I discuss existing proposals on these topics and to what extend they are compatible with my account.

To conclude, in this section I introduced the assumptions that headless relatives are derived from light-headed relatives and that relative pronouns contain at least one more feature than light heads. A headless relative is grammatical when either the light head or the relative pronoun contains all constituents of the other element. This set of assumptions derives that only the most complex case can surface and that there is no language of the external-only type.

%at first sight this seems very much related to what Hanink proposes for Modern German. Something is non-pronounced if it contains the features. A crucial difference here is that she formulates it in terms of context sensitive rules, but she does not motivate where these rules come from. I do not have language-specific rules.


\section{Deriving the internal-only type}\label{sec:deriving-only-internal}

Internal-only languages can be summarizes as in Table \ref{tbl:overview-rel-light-mg}.

\begin{table}[htbp]
  \center
  \caption{The surface pronoun with differing cases in Modern German}
\begin{tabular}{cccc}
  \toprule
                & \tsc{k}\scsub{int} > \tsc{k}\scsub{ext} & \tsc{k}\scsub{ext} > \tsc{k}\scsub{int} &   \\
                \cmidrule{2-3}
internal-only   & relative pronoun\scsub{int} & *  & Modern German    \\
\bottomrule
\end{tabular}
\label{tbl:overview-rel-light-mg}
\end{table}

A language of the internal-only type (like Modern German) allows only the internal case to surface when it wins the case competition, and it does not allow the external case to do so. The relative pronoun with its internal case can be the surface pronoun and the light head with its external case cannot. The goal of this section is to derive this from the way light heads and relative pronouns are spelled out in Modern German.

The section is structured as follows. According to my assumptions in Section \ref{sec:basic-idea}, the relative pronouns are part of the relative clause. I confirm this independently for Modern German with data from extraposition.
Then I argue that Modern German headless relatives are derived from light-headed relatives with as a light head the weak demonstrative in the sense of \citealt{schwarz2009}.
I decompose the light heads and relative pronouns intro smaller morphemes, and I show which features each of the morphemes corresponds to.
Finally, I compare the constituents of the light head and the relative pronoun.
When the internal and the external case match, the relative pronoun can delete the light head, because it contains all its constituents.
When the internal case is more complex than the external case, the relative pronoun can still delete the light head, for the same reason: the relative pronoun contains all constituents of the light head.
This is no longer the case when the external case is more complex than the internal case. The light head does not contain all constituents of the relative pronoun, and the relative pronoun does not contain all constituents of the light head. As a result, there is no grammatical form to surface when the external case is more complex.

\subsection{The relative pronoun is the surface pronoun}

In this section I show that the relative pronoun in Modern German headless relatives is the surface pronoun. The evidence comes from extraposition data.

The sentences in \ref{ex:mg-extrapose-cp} show that it is possible to extrapose a CP. In \ref{ex:mg-extrapose-cp-base}, the clausal object \tit{wie es dir geht} `how you are doing', marked here in bold, appears in its base position. It can be extraposed to the right edge of the clause, shown in \ref{ex:mg-extrapose-cp-moved}.

\ex.\label{ex:mg-extrapose-cp}
\ag. Mir ist \tbf{wie} \tbf{es} \tbf{dir} \tbf{geht} egal.\\
1\tsc{sg}.\tsc{dat} is how it 2\tsc{sg}.\tsc{dat} goes {the same}\\
`I don't care how you are doing.'\label{ex:mg-extrapose-cp-base}
\bg. Mir is egal \tbf{wie} \tbf{es} \tbf{dir} \tbf{geht}.\\
1\tsc{sg}.\tsc{dat} is {the same} how it 2\tsc{sg}.\tsc{dat} goes\\
`I don't care how you are doing.' \label{ex:mg-extrapose-cp-moved}\flushfill{Modern German}

\ref{ex:mg-extrapose-dp} illustrates that it is impossible to extrapose a DP. The clausal object of \ref{ex:mg-extrapose-cp} is replaced by the simplex noun phrase \tit{die Sache} `that matter'.
In \ref{ex:mg-extrapose-dp-base} the object, marked in bold, appears in its base position. In \ref{ex:mg-extrapose-dp-moved} it is extraposed, and the sentence is no longer grammatical.

\ex.\label{ex:mg-extrapose-dp}
\ag. Mir ist \tbf{die} \tbf{Sache} egal.\\
1\tsc{sg}.\tsc{dat} is that matter {the same}\\
`I don't care about that matter.'\label{ex:mg-extrapose-dp-base}
\bg. *Mir ist egal \tbf{die} \tbf{Sache}.\\
1\tsc{sg}.\tsc{dat} is {the same} that matter\\
`I don't care about that matter.' \label{ex:mg-extrapose-dp-moved}\flushfill{Modern German}

The same asymmetry between CPs and DPs can be observed with relative clauses. A relative clause is a CP, and the head of a relative clause is a DP. The sentences in \ref{ex:extra-headed} contain the relative clause \tit{was er gekocht hat} `what he has stolen'. This is marked in bold in the examples. The (light) head of the relative clause is \tit{das}.\footnote{
Not all speakers of Modern German accept the combination of \tit{das} as a light head and \tit{was} as a relative pronoun and prefer \tit{das} as a relative pronoun instead. I use the combination of \tit{das} and \tit{was} to have a more minimal pair with the headless relatives (that uses the relative pronoun \tit{was}).
\label{ftn:das-was}
}
In \ref{ex:extra-headed-base}, the relative clause and its head appear in base position. In \ref{ex:extra-headed-only-clause}, the relative clause is extraposed. This is grammatical, because it is possible to extrapose CPs in Modern German. In \ref{ex:extra-headed-head-clause}, the relative clause and the head are extraposed. This is ungrammatical, because it is possible to extrapose DPs.

\ex.\label{ex:extra-headed}
\ag. Jan hat das, \tbf{was} \tbf{er} \tbf{gekocht} \tbf{hat}, aufgegessen.\\
Jan has that what he cooked has eaten\\
`Jan has eaten what he cooked.'\label{ex:extra-headed-base}
\bg. Jan hat das aufgegessen, \tbf{was} \tbf{er} \tbf{gekocht} \tbf{hat}.\\
Jan has that eaten what he cooked has\\
`Jan has eaten what he cooked.'\label{ex:extra-headed-only-clause}
\cg. *Jan hat aufgegessen, das, \tbf{was} \tbf{er} \tbf{gekocht} \tbf{hat}.\\
Jan has eaten that what he cooked has\\
`Jan has eaten what he cooked.'\label{ex:extra-headed-head-clause} \flushfill{Modern German}

The same can be observed in relative clauses without a head. \ref{ex:extra-headless} is the same sentence as in \ref{ex:extra-headed} only without the overt head. The relative clause is marked in bold again.
In \ref{ex:extra-headless-base}, the relative clause appears in base position. In \ref{ex:extra-headless-clause}, the relative clause is extraposed. This is grammatical, because it is possible to extrapose CPs in Modern German. In \ref{ex:extra-headless-no-rel}, the relative clause is extraposed without the relative pronouns. This is ungrammatical, because the relative pronoun is part of the CP.
This shows that the relative pronoun in headless relatives in Modern German are necessarily part of a CP, which is here a relative clause.

\ex.\label{ex:extra-headless}
\ag. Jan hat \tbf{was} \tbf{er} \tbf{gekocht} \tbf{hat} aufgegessen.\\
Jan has what he cooked has eaten\\
`Jan has eaten what he cooked.'\label{ex:extra-headless-base}
\bg. Jan hat aufgegessen \tbf{was} \tbf{er} \tbf{gekocht} \tbf{hat}.\\
Jan has eaten what he cooked has\\
`Jan has eaten what he cooked.'\label{ex:extra-headless-clause}
\bg. *Jan hat \tbf{was} aufgegessen \tbf{er} \tbf{gekocht} \tbf{hat}.\\
Jan has what eaten he cooked has\\
`Jan has eaten what he cooked.'\label{ex:extra-headless-no-rel}\flushfill{Modern German}

In conclusion, extraposition facts show that the surface pronoun in Modern German headless relatives corresponds to the relative pronoun.

\subsection{The light-headed relative clause}

In this section, I discuss from which light-headed relatives Modern German headless relatives are derived. This is not easy to determine, because the light head in headless relative is always obligatorily deleted, so it never surfaces.

In \ref{ex:mg-light-headed}, I give the two types of light-headed relatives that Modern German has. They differ in their choice of relative pronoun.

\ex.\label{ex:mg-light-headed}
\ag. Jan umarmt den \tbf{den} \tbf{er} \tbf{mag}.\\
Jan hugs \tsc{d}.\tsc{m}.\tsc{sg}.\tsc{acc} \tsc{rel}.\tsc{m}.\tsc{sg}.\tsc{acc} he likes\\
`Jan hugs the man that he likes.'\label{ex:mg-d-for-light-headed}
\bg. Jan umarmt den \tbf{wen} \tbf{er} \tbf{mag}.\\
Jan hugs \tsc{d}.\tsc{m}.\tsc{sg}.\tsc{acc} \tsc{rel}.\tsc{m}.\tsc{sg}.\tsc{acc} he likes\\
`Jan hugs the man that he likes.'\label{ex:mg-wh-for-light-headed}

In \ref{ex:mg-d-for-light-headed}, the relative pronoun is the \tsc{d}-pronoun which generally appears in headed relatives. An example of a headed relative is given in \ref{ex:mg-d-for-headed}.

\exg. Jan umarmt den Mann \tbf{den} \tbf{er} \tbf{mag}.\\
Jan hugs \tsc{d}.\tsc{m}.\tsc{sg}.\tsc{acc} man \tsc{rel}.\tsc{m}.\tsc{sg}.\tsc{acc} he likes\\
`Jan hugs the man that he likes.'\label{ex:mg-d-for-headed}

I exclude the possibility that Modern German headless relatives are derived from these light-headed relatives, because they appear with the incorrect relative pronoun.

In \ref{ex:mg-wh-for-light-headed}, the relative pronoun is the \tsc{wh}-pronoun which generally appears in headed relatives. An example of a headless relative is given in \ref{ex:mg-wh-for-headless}.

\exg. Jan umarmt \tbf{wen} \tbf{er} \tbf{mag}.\\
Jan hugs \tsc{rel}.\tsc{an}.\tsc{acc} he likes\\
`Jan hugs who he likes.'\label{ex:mg-wh-for-headless}

At first sight, it looks like headless relatives can be derived from this light-headed relative.\footnote{
This is exactly what \citet{hanink2018} does. She argues that the feature content of the light head matches the feature content of the relative pronoun. Therefore, the light head is by default deleted. Only if the light head carries an extra focus feature it surfaces.
}
I argue that this is not the case. I already briefly noted in footnote \ref{ftn:das-was} that not all speakers of Modern German allow for the combination of a \tsc{d}-pronoun as light head and a \tsc{wh}-pronoun as relative pronoun. Some of my informants reported that the meaning of these two elements are incompatible for them. They prefer the \tsc{d}-pronoun as relative pronoun instead.
I assume that the demonstrative \tit{den} spells out features having to do with definiteness. The relative pronoun \tit{wen} in \ref{ex:mg-wh-for-light-headed} spells out features having to do with sets of alternatives. The combination of this two does not match. A better match for the demonstrative is the relative pronoun \tit{den} in \ref{ex:mg-d-for-light-headed}, which also spells out features having to do with definiteness.

The definite aspect of the demonstrative as a light head is even more problematic in the context of a headless relative. The main interpretation of headless relatives is namely not a definite one. Consider the headless relative in \ref{ex:mg-wh-for-headless-ever} that is combined with the the particle \tit{auch immer} `ever'. Light-headed relatives do not allow for this particle to be inserted.

\exg. Jan unarmt \tbf{wen} {\tbf{auch} \tbf{immer}} \tbf{er} \tbf{mag}.\\
Jan hugs \tsc{rel}.\tsc{an}.\tsc{acc} ever he likes\\
`Jan hugs whoever he likes.'\label{ex:mg-wh-for-headless-ever}

The sentence in \ref{ex:mg-wh-for-headless-ever} has two interpretations (see \citealt{s̆imík2020} for a recent elaborate overview on the semantics of free relatives). The first interpretation is described as universal-like and the second as definite-like. The universal-like interpretation corresponds to a universal quantifier: Jan hugs everybody that he likes. The definite-like interpretation corresponds to a definite description: Jan hugs the person that he likes. The latter interpretation is compatible with the light-headed relative in \ref{ex:mg-wh-for-light-headed}.

\pgcitet{s̆imík2020}{4} notes that some languages do not easily allow for the definite-like interpretation of headless relatives with an \tit{ever}-morpheme. There is no language documented that does not allow for the universal-like interpretation, but does allow the definite-like interpretation.
In the same spirit, informants have reported to me that headless relatives with case mismatches become more acceptable in the universal-like interpretation compared to the definite-like interpretation.
What I take away from that, is that the universal-like interpretation of headless relatives is the main interpretation that should be accounted for. This is not the interpretation that the light-headed relative in \ref{ex:mg-d-for-light-headed} provides.

I suggest that the light head is a form that shows resemblance to the weak definite of \citet{schwarz2009}. \citet{schwarz2009} distinguishes strong and weak definites in Modern German. The strong definite is anaphoric in nature, and the weak definite encodes uniqueness. First I give an example of a strong definite in \ref{ex:mg-florian-strong}. The strong definite is \tit{dem} that precedes \tit{Freund} `friend'. It refers back to the linguistic antecedent \tit{einen Freund} `a friend'.

\exg. Hans hat heute einen Freund zum Essen mit nach Hause gebracht. Er hat uns vorher ein Foto von dem Freund gezeigt.\\
Hans has today a friend {to the} dinner with to home brought he has us beforehand a photo of the\scsub{strong} friend shown\\
`Hans brought a friend home for dinner today. He had shown us a photo of the friend beforehand.'\label{ex:mg-florian-strong}

In a light-headed relative, this is appropriate when the light head refers back to something or someone introduced earlier in the discourse. An example of such a context is given \ref{ex:mg-context-strong}.

\ex. Yesterday Jan met with two friends. He likes one of them. The other one he does not like so much.\label{ex:mg-context-strong}

Following this context, the light head \tit{den}, so the strong definite, is appropriate. This, however, is not the universal-like interpretation I discussed is the main interpretation that should be accounted for in headless relatives.

Weak definites are used when situational uniqueness is involved. This uniqueness can be global or within a restricted domain. I give two examples in \ref{ex:mg-florian-weak}. In \ref{ex:mg-florian-weak-hund}, the dog is unique in this specific situation of the break-in. In \ref{ex:mg-florian-weak-mond}, the moon is unique for us people on the planet.

\ex.\label{ex:mg-florian-weak}
\ag. Der Einbrecher ist {zum Glück} vom Hund verjagt worden.\\
the burglar is luckily {by the\scsub{weak}} dog {chased away} been\\
`Luckily, the burglar was chased away by the dog.'\label{ex:mg-florian-weak-hund}
\bg. Armstrong flog als erster zum Mond.\\
Armstrong flew as {first one} {to the\scsub{weak}} moon\\
`Armstrong was the first one to fly to the moon.' \flushfill{Modern German, \pgcitealt{schwarz2009}{40}}\label{ex:mg-florian-weak-mond}

% question:

In line with this, I propose that headless relatives in Modern German are derived from the light-headed relatives as in \ref{ex:mg-base-headless-rels}. The brackets around the light head indicate that it is obligatorily deleted.

\exg. Jan umarmt [ən] \tbf{wen} \tbf{er} \tbf{mag}.\\
Jan hugs \tsc{lh}.\tsc{an}.\tsc{acc} \tsc{rel}.\tsc{an}.\tsc{acc} he likes\\
`Jan hugs who he likes.'\label{ex:mg-base-headless-rels}

% question:

\ex.
\ag. Hat ər einen Motorrad?\\
 has \tsc{dem}.\tsc{nom}\scsub{weak} a motor bike\\
 `Does he have a motorbike?'
\bg. Ich habe ən gesehen.\\
 I have \tsc{dem}.\tsc{acc}\scsub{weak} seen\\
 `I have been him.'
\bg. Ich folge əm nach Hause.\\
 I follow \tsc{dem}.\tsc{dat}\scsub{weak} to house\\
 `I follow him home.'

%

\subsection{Decomposing light heads and relative pronouns}

%

In this section I discuss two light heads and two relative pronouns. These are the two element that I compare the constituents of at the end of this section. The forms are the animate singulars in nominative and accusative case, given in \ref{ex:mg-lhs-rels}.

\ex.\label{ex:mg-lhs-rels}
\ag. w-e-r\\
 `\tsc{rel}.\tsc{an}.\tsc{sg}.\tsc{nom}'\\
\bg. w-e-n\\
 `\tsc{rel}.\tsc{an}.\tsc{sg}.\tsc{acc}'\\
\bg. ə-r\\
 `\tsc{lh}.\tsc{an}.\tsc{sg}.\tsc{nom}'\\
\bg. ə-n\\
 `\tsc{lh}.\tsc{an}.\tsc{sg}.\tsc{acc}'\\

I decompose the relative pronouns in three morphemes: the \tit{w}, the \tit{e} and the final consonant. I splits the light heads up into two morphemes: \tit{ə} and the final consonant. For each morpheme, I discuss which features they spell out. Then I give the feature content of light heads and relative pronouns as a whole. Finally, I give the lexical entries for each of the morphemes.

I start with the morpheme \tit{w} of the relative pronoun. Table \ref{tbl:mg-paradigm-wh-rels} gives the animate and inanimate forms of the relative pronouns in three cases: nominative, accusative and dative.

\begin{table}[htbp]
 \center
 \caption {Modern German relative pronouns}
  \begin{tabular}{ccc}
  \toprule
              & \ac{an}  & \tsc{inan}\\
    \cmidrule{2-3}
    \ac{nom}  & w-er    & w-as     \\
    \ac{acc}  & w-en    & w-as     \\
    \ac{dat}  & w-em    & (w-em)   \\
  \bottomrule
  \end{tabular}
  \label{tbl:mg-paradigm-wh-rels}
\end{table}

The \tit{w} combines with the same endings as the \tit{d} does in demonstratives (or relative pronouns in headed relatives). Table \ref{tbl:mg-paradigm-dem} gives the singular forms in three gender and three cases.\footnote{
Note here that the relative pronouns, unlike the demonstratives, do not have a feminine form for the relative pronouns in Table \ref{tbl:mg-paradigm-wh-rels}. Demonstratives also have plural forms (which are not given here), and relative pronouns do not. As far as I know, this holds for all relative pronouns in languages of the internal-only type (cf. also for Finnish, even though it makes a lot of morphological distinctions) and of the matching type. Relative pronouns in languages of the unrestricted type do inflect for feminine, as well as always-external languages. In Chapter \ref{ch:discussion} I return to this observation in relation with the always-external languages.
}

\begin{table}[htbp]
 \center
 \caption {Modern German demonstrative pronouns} %dieser source
  \begin{tabular}{cccc}
  \toprule
              & \ac{m}  & \tsc{n} & \tsc{f} \\
    \cmidrule{2-4}
    \ac{nom}  & d-er   & d-as   & d-ie    \\
    \ac{acc}  & d-en   & d-as   & d-ie    \\
    \ac{dat}  & d-em   & d-em   & d-er    \\
  \bottomrule
  \end{tabular}
  \label{tbl:mg-paradigm-dem}
\end{table}

This identifies the \tit{d} and, more importantly for the discussion here, the \tit{w} as a separate morpheme. The morpheme \tit{w} appears besides in headless relatives clauses also in different contexts, such as interrogative pronouns. I give an example in \ref{ex:mg-interrogative}.

\exg. Wer ist da?\\
who is there\\
`Who is there?'\label{ex:mg-interrogative}

Two features that \tit{w} spells out are important for the discussion here. The first one I refer to as \tsc{wh}. This is the meaning part that relative pronouns and interrogatives share: the \tsc{wh}-element triggers the construction of a set of alternatives in the sense of Rooth (1985; 1992).\footnote{
This contrasts with the \tsc{d}, which is responsible for establishing a definite reference. I return to the \tsc{d} in Section \ref{sec:deriving-nonmatching}
}
The second relevant feature is \tsc{rel}, which establishes a relation.
In sum, the \tit{w} spells out the features \tsc{wh} and \tsc{rel}, shown in \ref{ex:mg-w-features}. This is not a lexical entry, but only a notation that sums of which features the \tit{w} spells out. I give lexical entries at the end of this section.

\ex. [\tsc{wh} + \tsc{rel}] = \tit{w}\label{ex:mg-w-features}

I continue with the final consonants (\tit{r} and \tit{n} ), which is present in the relative pronoun and in the light head. The final consonant can be observed in several contexts besides light heads and relative pronouns. Table \ref{tbl:mg-str-adj} gives an overview of strong adjective \tit{neu} `new' in Modern German in two numbers, three genders and three cases.\footnote{
The vowel preceding the final consonant is written as \tit{e}. I write it as \tit{ə}, because this is how it is pronounced. I make this distinction to emphasize that this differs from the vowel used in the relative pronouns. It might be the same as the \tit{ə} in the light head.
}

\begin{table}[htbp]
 \center
 \caption {Modern German strong adjective \tit{neu} `new'}
  \begin{tabular}{ccccc}
  \toprule
              & \ac{m}.\tsc{sg}    & \tsc{n}.\tsc{sg}   & \tsc{f}.\tsc{sg}  & \tsc{pl} \\
    \cmidrule{2-5}
    \ac{nom}  & neu-ə-r   & neu-ə-s   & neu-ə    & neu-ə    \\
    \ac{acc}  & neu-ə-n   & neu-ə-s   & neu-ə    & neu-ə    \\
    \ac{dat}  & neu-ə-m   & neu-ə-m   & neu-ə-r  & neu-ə-n  \\
  \bottomrule
  \end{tabular}
  \label{tbl:mg-str-adj}
\end{table}

Table \ref{tbl:mg-pers-pron} gives an overview of the third person pronouns in Modern German in two numbers, three genders and three cases. I divided the non-suppletive forms in separate morphemes.

\begin{table}[htbp]
 \center
 \caption {Modern German third person pronouns}
  \begin{tabular}{ccccc}
  \toprule
              & \ac{m}.\tsc{sg} & \tsc{n}.\tsc{sg} & \tsc{f}.\tsc{sg} & \tsc{pl} \\
    \cmidrule{2-5}
    \ac{nom}  & er     & es      & sie     & sie      \\
    \ac{acc}  & ih-n   & es      & sie     & sie      \\
    \ac{dat}  & ih-m   & ih-m    & ih-r    & ih-nen   \\
  \bottomrule
  \end{tabular}
  \label{tbl:mg-pers-pron}
\end{table}

Table \ref{tbl:mg-str-adj} and \ref{tbl:mg-pers-pron} show that the final consonants take different shapes depending on gender, number and case. I conclude from that that the consonant realizes features having to do with these three aspects.

Since I only discuss singular masculine forms in nominative and accusative case, I only introduce features that are realized by these morphemes.
For case, I adopt the features of \citet{caha2009}, already introduced in Chapter \ref{ch:decomposition}. \tsc{f}1 refers to a nominative, and \tsc{f}1 and \tsc{f}2 refers to an accusative.
For number and gender, I adopt the features that are distinguished by \citet{harley2002} for pronouns. The feature \tsc{class} refers to gender features, which is neuter if it is not combined with any other features. Combining \tsc{class} with the feature \tsc{masc} gives a masculine gender. The feature \tsc{ind} refers to number, which is singular if it is not combined with any other features.

The \tit{r} is the nominative masculine singular, so it spells out the features \tsc{class}, \tsc{masc}, \tsc{ind} and \tsc{f}1. The \tit{n} is the accusative masculine singular, so it spells out the features that the \tit{r} spells out plus \tsc{f}2. I summarizes this in \ref{ex:r/n-features}.

\ex.\label{ex:r/n-features}
\a. [\tsc{class} + \tsc{masc} + \tsc{ind} + \tsc{f}1] = \tit{r}
\b. [\tsc{class} + \tsc{masc} + \tsc{ind} + \tsc{f}1 + \tsc{f}2] = \tit{n}

At this point, only the \tit{e} in the relative pronoun and the \tit{ə} in the light head are left. About the \tit{ə} I can be brief. I assume that the vowel corresponds to the feature \tsc{ref}, which \citet{harley2002} claim all pronouns contain.

\ex. [\tsc{ref}] = \tit{ə}

%

This leaves the \tit{e} in the relative pronoun. I assume that, just like \tit{ə}, it contains \tsc{ref}. Besides that, it expresses features involving deixis. Modern German has a syncretic form for the proximal, the medial and the distal \pgcitep{lander2018}{387}. Therefore, it is impossible to tell which of the three a \tsc{wh}-pronoun expresses. Demonstratives in English are informative here, since they distinguishes between proximal on the one hand and medial and distal on the other hand, shown in \ref{ex:english-deixis}.

\ex.\label{ex:english-deixis}
\ag. th-is\\
`close to speaker'\\
\bg. th-at\\
`not close to speaker'\\

The English \tsc{wh}-pronoun combines with the medial/distal marker, and not with the proximal, shown in \ref{ex:english-wh}.

\ex.\label{ex:english-wh}
\a. *wh-is
\b. wh-at

% Ewondo
%
% class 2 pl
% mī `near S'
% mīlí `near H'
% mīlíí `away from S + H'

I conclude from this that the \tsc{wh}-pronoun in Modern German combines with at least the medial.

\footnote{
Conceptually, this can be made sense of if you see distal as something far away from you as a speaker, because it is unknown to you. Something that is known to you is expressed with a proximal.

\ex.
\a. Yesterday I talked to this woman, and she told me all I needed to know.
\b. Please tell me about that thing later.

cite Wiltschko.
}

\citet{lander2018} propose that a \tsc{dx}\scsub{1} corresponds to a proximal, \tsc{dx}\scsub{1} and \tsc{dx}\scsub{2} corresponds to a medial.

In sum, the \tit{e} spells out the features \tsc{ref}, \tsc{dx}\scsub{1} and \tsc{dx}\scsub{2}.

\ex. [\tsc{ref} + \tsc{dx}\scsub{1} + \tsc{dx}\scsub{2}] = \tit{e}

Taking this all together, I propose that the structure in \ref{ex:fseq-wh-rel} is the functional sequence for the relative pronoun. It consists of referential features at the bottom (\tsc{ref},) features having to do with deixis (\tsc{dx}\scsub{1} and \tsc{dx}\tsc{2}), with gender (\tsc{class} and \tsc{masc}) and number (\tsc{ind}), with the relative pronoun (\tsc{wh} and \tsc{rel}) and with case (\tsc{f}1 and \tsc{f}2).

\ex. \begin{forest} for tree={s sep=13mm, inner sep=0, l=0}
[\tsc{acc}P
    [\tsc{f}2]
    [\tsc{nom}P
        [\tsc{f}1]
        [\tsc{rel}P
            [\tsc{rel}]
            [\tsc{wh}P
                [\tsc{wh}]
                [\tsc{ind}P
                    [\tsc{ind}]
                    [\tsc{masc}P
                        [\tsc{masc}]
                        [\tsc{class}P
                            [\tsc{class}]
                            [\tsc{med}P
                                [\tsc{dx}\scsub{2}]
                                [\tsc{prox}P
                                    [\tsc{dx}\scsub{1}]
                                    [\tsc{ref} [\phantom{xxx}, roof]]
                                ]
                            ]
                        ]
                    ]
                ]
            ]
        ]
    ]
]
\end{forest}
\label{ex:fseq-wh-rel}

This order is independently supported by work in the literature. .. %The ordering: picallo, kramer, bittner hale, bayer


The functional sequence for the light heads:

\ex. \begin{forest} for tree={s sep=13mm, inner sep=0, l=0}
[(\tsc{acc}P)
    [(\tsc{f}2)]
    [\tsc{nom}P
        [\tsc{f}1]
        [\tsc{ind}P
            [\tsc{ind}]
            [\tsc{masc}P
                [\tsc{masc}]
                [\tsc{class}P
                    [\tsc{class}]
                    [\tsc{prox}P
                        [\tsc{dx}\scsub{1}]
                        [\tsc{ref} [\phantom{xxx}, roof]]
                    ]
                ]
            ]
        ]
    ]
]
\end{forest}
\label{ex:fseq-wh-lh}

Now I give the lexical entries.

\ex.
\begin{forest} boom
  [\tsc{ref} [\phantom{xxx}, roof]]
  {\draw (.east) node[right]{⇔ \tit{ə}}; }
\end{forest}
\label{ex:mg-entry-schwa}


\ex.
\begin{forest} boom
  [\tsc{dist}P
      [\tsc{dx}\scsub{3}]
      [\tsc{med}P
          [\tsc{dx}\scsub{2}]
          [\tsc{prox}P
              [\tsc{dx}\scsub{1}]
              [\tsc{ref} [\phantom{xxx}, roof]]
          ]
      ]
  ]
  {\draw (.east) node[right]{⇔ \tit{e}}; }
  \label{ex:mg-entry-e}
\end{forest}


suffixes are the ones that have a unary bottom. In the next section..

\ex. \begin{forest} boom
  [\tsc{nom}P
      [\tsc{f}1]
      [\tsc{ind}P
          [\tsc{ind}]
          [\tsc{masc}P
              [\tsc{masc}]
              [\tsc{class}P
                  [\tsc{class}]
              ]
          ]
      ]
  ]
  {\draw (.east) node[right]{⇔ \tit{r}}; }
\end{forest}
\label{ex:mg-entry-r}

\ex. \begin{forest} boom
  [\tsc{acc}P
      [\tsc{f}2]
      [\tsc{nom}P
          [\tsc{f}1]
          [\tsc{ind}P
              [\tsc{ind}]
              [\tsc{masc}P
                  [\tsc{masc}]
                  [\tsc{class}P
                      [\tsc{class}]
                  ]
              ]
          ]
      ]
  ]
  {\draw (.east) node[right]{⇔ \tit{n}}; }
\end{forest}
\label{ex:mg-entry-n}


Prefix has a binary bottom

\ex. \begin{forest} boom
  [\tsc{rel}P
      [\tsc{rel}]
      [\tsc{wh}]
  ]
  {\draw (.east) node[right]{⇔ \tit{w}}; }
\end{forest}\label{ex:mg-entry-w}


% At first sight it seems like \citet{fuss2014} discuss a exception to this claim, namely headless relatives with \tsc{d}-pronouns. However, they claim that these headless relatives are actually light-headed relatives in which one of two syncretic elements is deleted by haplology.

% I argue that there is a different form for the proximal that appears in combination with a reinforcer (Lander). ``As a determiner and a pronoun \tit{dieser} typically refers to something near at hand'' \pgcitep{durrell2011}{5.1.2(a)}
% The \tit{e} `\tsc{med}/\tsc{dist}' can be used for distal, medial and proximal. The \tit{ə} is typically only compatible with proximal references.
%
% \ex.
% \ag. d-e-r dort/ hier\\
%  \tsc{d}-\tsc{med}/\tsc{dist}-\tsc{m.sg.nom} there/ here\\
%  `that one there'
% \bg. d-ies-ə-r hier/ \#dort\\
%  \tsc{d}-\tsc{reinf}-\tsc{prox}-\tsc{m.sg.nom} here/ there\\
%  `this one here'





% Now I come back to the interaction between gender and case. For the neuter relative pronoun the vowel is different: \tit{a}. So, this needs to be reflected in the features: I left out \tsc{masc}. However, there also needs to be a difference in lexical entries between the final consonant for the neuter and for the masculine. Therefore, I let the \tit{s} spell out the \tsc{ind} feature.
%
% \ex. \begin{forest} boom
%   [\tsc{class}P
%       [\tsc{class}]
%       [\tsc{med}P
%           [\tsc{dx}\scsub{2}]
%           [\tsc{prox}P
%               [\tsc{dx}\scsub{1}]
%               [\tsc{ref} [\phantom{xxx}, roof]]
%           ]
%       ]
%   ]
% {\draw (.east) node[right]{⇔ \tit{a}}; }
% \end{forest}
%
% \ex. \begin{forest} boom
%   [\tsc{nom}P
%       [\tsc{f}1]
%       [\tsc{ind}P
%           [\tsc{ind}]
%       ]
%   ]
%   {\draw (.east) node[right]{⇔ \tit{s}}; }
% \end{forest}

\subsection{Recomposing light heads and relative pronouns}

In Nanosyntax, lexical entries are not stuck together directly from the lexicon. Instead, spellout happens in a cyclic derivation.

The features are merged one by one, and the constituency is the consequence of the spellout algorithm.

\ex. Cyclic phrasal spellout. Caha:declension\\
Spellout must successfully apply to the output of every Merge F operation. After successfull spellout, the derivation may terminate, or proceed to another round of Merge F.

\ex. \tbf{Cyclic Override} \citep{starke2018}:\\
Lexicalisation at a node XP overrides any previous match at a phrase contained in XP.

Compare \tit{ə} and \tit{e}, show cyclic override.

Now merge next feature \tsc{class}, which cannot be spelled out anymore. How do we reach the different constituents? Spellout algorithm.

\ex. \tbf{Spellout Algorithm:}\label{ex:spellout-algorithm}
 \a. Merge F and spell out.\label{ex:spellout-algorithm-phrasal}
 \b. If \ref{ex:spellout-algorithm-phrasal} fails, move the Spec of the complement and spell out.\label{ex:spellout-algorithm-spec}
 \b. If \ref{ex:spellout-algorithm-spec} fails, move the complement of F and spell out.\label{ex:spellout-algorithm-comp}

I informally reformulate what is in \ref{ex:spellout-algorithm}. I start with the first line in \ref{ex:spellout-algorithm-phrasal}. This says that a feature F is merged, and the newly created phrase FP is attempted to spell out.
The next two lines, \ref{ex:spellout-algorithm-spec} and \ref{ex:spellout-algorithm-comp}, describe two types of rescue movements that take place when the spellout in \ref{ex:spellout-algorithm-phrasal} fails (i.e. when there is no match in the lexicon).

So show this movement, it's a result of the lexical entries, nothing else.

Next problem arises at \tsc{wh}. No suffix, because no unary bottom.

\ex. \tbf{Spec Formation} \citep{starke2018}:\\
If Merge F has failed to spell out, try to spawn a new derivation providing the feature F and merge that with the current derivation, projecting the feature F at the top node.\label{ex:specformation}

The last problem is the case feature. What happens then is backtracking + elements are split up, merged onto both of them, case can be spelled out with suffix.



Note here that the functional sequence is identical for all languages. The same holds for the spellout algorithm. It is only the lexical entries that differ per language. It is these lexical entries that cause the different constituency within the light head and the relative pronoun. If a feature is merged, it follows the procedure in the spellout algorithm to be realized. In some cases that means that it is spelled out on top of the other features, and in other cases it means that it is spelled out after one of the movements has taken place.

The morphemes for gender, number and case (\tit{r}/\tit{n}) appear as a suffix because the feature \tsc{class} cannot be spelled out together with the morpheme for referentiality and deixis (\tit{e}/ə). Therefore, a movement operation takes place, and the feature \tsc{class} can be spelled out on its own. The constituent for \tit{w} in Modern German arises because a complex spec was created.

%

The nominative masculine singular relative pronoun is built as follows.
The \tsc{ref}P is spelled out as \tit{ə}.
The \tsc{ref}P is merged with \tsc{dx}\scsub{1}, and the whole phrase (\tsc{prox}P) is spelled out as \tit{e}.
The \tsc{prox}P is merged with \tsc{dx}\scsub{2}, and the whole phrase (\tsc{med}P) is spelled out as \tit{e}.

The \tsc{med}P is merged with \tsc{class}. There is no lexical entry that matches the whole phrase. There is no specifier to move, so this movement is irrelevant. The complement of \tsc{class}, the \tsc{med}P, is moved to the specifier of \tsc{class}P, and the \tsc{class}P is spelled out as \tit{r}.
The \tsc{class}P is merged with \tsc{masc}. There is no lexical entry that matches the whole phrase. The specifier of \tsc{class}P, the \tsc{med}P, is moved to the specifier of \tsc{masc}P, and the \tsc{masc}P is spelled out as \tit{r}.
The \tsc{masc}P is merged with \tsc{ind}. There is no lexical entry that matched the whole phrase. The specifier of \tsc{masc}P, the \tsc{med}P, is moved to the specifier of \tsc{ind}P, and the \tsc{ind}P is spelled out as \tit{r}.

The \tsc{ind}P is merged with \tit{wh}. There is no lexical entry that matches the whole phrase, there is no match after spec-to-spec movement, and there is no match after complement movement. Backtracking also does not lead to a matching lexical entry. A complex specifier is created.

The \tsc{wh}P is merged with \tsc{rel}. There is no lexical entry that matches the whole phrase, there is no match after spec-to-spec movement, and there is no match after complement movement. The first step of backtracking is that the two branches, the \tsc{wh}P and the \tsc{ind}P are separated. The \tsc{rel} is merged with the \tsc{wh}P (that only contains the \tsc{wh}) in the left branch and with the \tsc{ind}P in the right branch. In the left branch, the whole phrase (\tsc{rel}P) is spelled out as \tit{w}. In the right branch, there is no spellout. %

The \tsc{rel}P is merged with \tsc{f}1. There is no lexical entry that matches the whole phrase, there is no match after spec-to-spec movement, and there is no match after complement movement. The first step of backtracking is that the two branches, the \tsc{rel}P and the \tsc{ind}P are separated. The \tsc{f}1 is merged with the \tsc{rel}P in the left branch and with the \tsc{ind}P in the right branch. In both branches, there is no lexical entry that matches the whole phrase. In the left branch, there is no specifier to move, so this movement is irrelevant. In the right branch, the specifier of \tsc{ind}P, the \tsc{med}P, is moved to the specifier of \tsc{nom}P, and the \tsc{nom}P is spelled out as \tit{r}.

The final result is given in \ref{ex:mg-spellout-rel-nom}.

\ex.
\scriptsize{
\begin{forest} boompje
  [\tsc{rel}P, s sep=12mm
      [\tsc{rel}P,
      tikz={
      \node[label=below:\tit{w},
      draw,circle,
      scale=0.9,
      fit to=tree]{};
      }
          [\tsc{rel}]
          [\tsc{wh}]
      ]
      [\tsc{nom}P, s sep=21mm
          [\tsc{med}P,
          tikz={
          \node[label=below:\tit{e},
          draw,circle,
          scale=0.85,
          fit to=tree]{};
          }
              [\tsc{dx}\scsub{2}]
              [\tsc{prox}P
                [\tsc{dx}\scsub{1}]
                [\tsc{ref} [\phantom{xxx}, roof]
                ]
            ]
        ]
          [\tsc{nom}P,
          tikz={
          \node[label=below:\tit{r},
          draw,circle,
          scale=0.95,
          fit to=tree]{};
          }
              [\tsc{f}1]
              [\tsc{ind}P
                  [\tsc{ind}]
                  [\tsc{masc}P
                      [\tsc{masc}]
                      [\tsc{class}P
                          [\tsc{class}]
                      ]
                  ]
              ]
          ]
      ]
  ]
\end{forest}
}
\label{ex:mg-spellout-rel-nom}

The accusative masculine singular relative pronoun is built as the nominative singular relative pronoun, except for that the feature \tsc{f}2 is added to make it an accusative.

The \tsc{nom}P is merged with \tsc{f}2. There is no lexical entry that matches the whole phrase, there is no match after spec-to-spec movement, and there is no match after complement movement. The first step of backtracking is that the two branches, the \tsc{rel}P and the \tsc{nom}P are separated. The \tsc{f}2 is merged with the \tsc{rel}P in the left branch and with the \tsc{nom}P in the right branch. In both branches, there is no lexical entry that matches the whole phrase. In the left branch, there is no specifier to move, so this movement is irrelevant. In the right branch, the specifier of \tsc{nom}P, the \tsc{med}P, is moved to the specifier of \tsc{acc}P, and the \tsc{acc}P is spelled out as \tit{n}.

The final result is given in \ref{ex:mg-spellout-rel-acc}.

\ex.
\scriptsize{
\begin{forest} boompje
  [\tsc{rel}P, s sep=12mm
      [\tsc{rel}P,
      tikz={
      \node[label=below:\tit{w},
      draw,circle,
      scale=0.9,
      fit to=tree]{};
      }
          [\tsc{rel}]
          [\tsc{wh}]
      ]
      [\tsc{acc}P, s sep=23mm
          [\tsc{med}P,
          tikz={
          \node[label=below:\tit{e},
          draw,circle,
          scale=0.85,
          fit to=tree]{};
          }
              [\tsc{dx}\scsub{2}]
              [\tsc{prox}P
                  [\tsc{dx}\scsub{1}]
                  [\tsc{ref} [\phantom{xxx}, roof]]
              ]
          ]
          [\tsc{acc}P,
          tikz={
          \node[label=below:\tit{n},
          draw,circle,
          scale=0.95,
          fit to=tree]{};
          }
              [\tsc{f}2]
              [\tsc{nom}P
                  [\tsc{f}1]
                  [\tsc{ind}P
                      [\tsc{ind}]
                      [\tsc{masc}P
                          [\tsc{masc}]
                          [\tsc{class}P
                              [\tsc{class}]
                          ]
                      ]
                  ]
              ]
          ]
      ]
  ]
\end{forest}
}
\label{ex:mg-spellout-rel-acc}

The nominative masculine singular light head is built as follows.
The \tsc{ref}P is spelled out as \tit{ə}.

The \tsc{ref}P is merged with \tsc{class}. There is no lexical entry that matches the whole phrase. There is no specifier to move, so this movement is irrelevant. The complement of \tsc{class}, the \tsc{ref}P, is moved to the specifier of \tsc{class}P, and the \tsc{class}P is spelled out as \tit{r}.
The \tsc{class}P is merged with \tsc{masc}. There is no lexical entry that matches the whole phrase. The specifier of \tsc{class}P, the \tsc{ref}P, is moved to the specifier of \tsc{masc}P, and the \tsc{masc}P is spelled out as \tit{r}.
The \tsc{masc}P is merged with \tsc{ind}. There is no lexical entry that matched the whole phrase. The specifier of \tsc{masc}P, the \tsc{ref}P, is moved to the specifier of \tsc{ind}P, and the \tsc{ind}P is spelled out as \tit{r}.
The \tsc{ind}P is merged with \tsc{f}1. There is no lexical entry that matched the whole phrase. The specifier of \tsc{ind}P, the \tsc{ref}P, is moved to the specifier of \tsc{nom}P, and the \tsc{nom}P is spelled out as \tit{r}.

The final result is given in \ref{ex:mg-spellout-lh-nom}.

\ex.
\scriptsize{
\begin{forest} boompje
  [\tsc{acc}P, s sep=15mm
      [\tsc{prox}P,
      tikz={
      \node[label=below:\tit{ə},
      draw,circle,
      scale=0.8,
      fit to=tree]{};
      }
          [\tsc{dx}\scsub{2}]
          [\tsc{ref} [\phantom{xxx}, roof]]
      ]
      [\tsc{nom}P,
      tikz={
      \node[label=below:\tit{r},
      draw,circle,
      scale=0.95,
      fit to=tree]{};
      }
          [\tsc{f}1]
          [\tsc{ind}P
              [\tsc{ind}]
              [\tsc{masc}P
                  [\tsc{masc}]
                  [\tsc{class}P
                      [\tsc{class}]
                  ]
              ]
          ]
      ]
  ]
\end{forest}
}
\label{ex:mg-spellout-lh-nom}

The accusative masculine singular light head is built as the nominative singular light head, except for that the feature \tsc{f}2 is added to make it an accusative.

The \tsc{nom}P is merged with \tsc{f}2. There is no lexical entry that matched the whole phrase. The specifier of \tsc{nom}P, the \tsc{ref}P, is moved to the specifier of \tsc{acc}P, and the \tsc{acc}P is spelled out as \tit{n}.

The final result is given in \ref{ex:mg-spellout-lh-acc}.

\ex.
\scriptsize{
\begin{forest} boompje
  [\tsc{acc}P, s sep=15mm
      [\tsc{prox}P,
      tikz={
      \node[label=below:\tit{ə},
      draw,circle,
      scale=0.8,
      fit to=tree]{};
      }
          [\tsc{dx}\scsub{1}]
          [\tsc{ref} [\phantom{xxx}, roof]]
      ]
      [\tsc{acc}P,
      tikz={
      \node[label=below:\tit{n},
      draw,circle,
      scale=0.95,
      fit to=tree]{};
      }
          [\tsc{f}2]
          [\tsc{nom}P
              [\tsc{f}1]
              [\tsc{ind}P
                  [\tsc{ind}]
                  [\tsc{masc}P
                      [\tsc{masc}]
                      [\tsc{class}P
                          [\tsc{class}]
                      ]
                  ]
              ]
          ]
      ]
  ]
\end{forest}
}
\label{ex:mg-spellout-lh-acc}




\subsection{Comparing constituents}

Consider the example in \ref{ex:mg-nom-nom-rep}, in which the internal nominative case competes against the external nominative case. The relative clause is marked in bold, and the light head and the relative pronoun are underlined.
The internal case is nominative, as the predicate \tit{mögen} `to like' takes nominative subjects. The relative pronoun \tit{wer} `\ac{rel}.\ac{an}.\ac{nom}' appears in the nominative case. This is the element that surfaces.
The external case is nominative as well, as the predicate \tit{besuchen} `to visit' also takes nominative subjects. The light head \tit{ər} `\ac{dem}.\ac{an}.\ac{nom}' appears in the nominative case. It is placed between square brackets because it does not surface.

\exg. Uns besucht [\underline{ər}], \underline{\tbf{wer}} \tbf{Maria} \tbf{mag}.\\
 2\ac{pl}.\ac{acc} visit.\ac{pres}.3\ac{sg}\scsub{[nom]} \ac{dem}.\ac{an}.\ac{nom} \ac{rel}.\ac{an}.\ac{nom} Maria.\ac{acc} like.\ac{pres}.3\ac{sg}\scsub{[nom]}\\
 `Who visits us likes Maria.' \flushfill{Modern German, adapted from \pgcitealt{vogel2001}{343}}\label{ex:mg-nom-nom-rep}

In Figure \ref{fig:mg-int=ext}, I give the syntactic structure of the light head at the top and the syntactic structure of the relative pronoun at the bottom.

\begin{figure}[htbp]
  \center
  \begin{tabular}[b]{c}
        \toprule
        \tsc{nom} light head \tit{ə-r}\\
        \cmidrule{1-1}
      \scriptsize{
      \begin{forest} boompje
        [{\tsc{nom}P}, s sep=15mm
            [{\tsc{prox}P},
            tikz={
            \node[label=below:{\tit{ə}},
            draw,circle,
            scale=0.8,
            fit to=tree]{};
            \node[
            draw,circle,
            scale=0.85,
            fill=DG,fill opacity=0.2,
            dashed,
            fit to=tree]{};
            }
                [{\tsc{dx}\scsub{1}}]
                [\tsc{ref} [\phantom{xxx}, roof]]
            ]
            [{\tsc{nom}P},
            tikz={
            \node[label=below:{\tit{r}},
            draw,circle,
            scale=0.95,
            fit to=tree]{};
            \node[
            draw,circle,
            fill=DG,fill opacity=0.2,
            scale=1,
            dashed,
            fit to=tree]{};
            }
                [{\tsc{f}1}]
                [{\tsc{ind}P}
                    [{\tsc{ind}}]
                    [{\tsc{masc}P}
                        [{\tsc{masc}}]
                        [{\tsc{class}P}
                            [{\tsc{class}}]
                        ]
                    ]
                ]
            ]
        ]
      \end{forest}
      }
      \\
      \toprule
      \tsc{nom} relative pronoun \tit{w-e-r}
      \\
      \cmidrule{1-1}
      \scriptsize{
          \begin{forest} boompje
          [\tsc{rel}P, s sep=15mm
              [\tsc{rel}P,
              tikz={
              \node[label=below:\tit{w},
              draw,circle,
              scale=0.9,
              fit to=tree]{};
              }
                  [\tsc{rel}]
                  [\tsc{wh}]
              ]
              [\tsc{nom}P, s sep=25mm
                  [\tsc{med}P,
                  tikz={
                  \node[label=below:\tit{e},
                  draw,circle,
                  scale=0.85,
                  fit to=tree]{};
                  }
                      [\tsc{dx}\scsub{2}]
                      [\tsc{prox}P,
                      tikz={
                      \node[draw,circle,
                      dashed,
                      scale=0.8,
                      fit to=tree]{};
                      }
                          [\tsc{dx}\scsub{1}]
                          [\tsc{ref} [\phantom{xxx}, roof]]
                      ]
                  ]
                  [\tsc{nom}P,
                  tikz={
                  \node[label=below:\tit{r},
                  draw,circle,
                  scale=0.95,
                  fit to=tree]{};
                  \node[draw,circle,
                  dashed,
                  scale=1,
                  fit to=tree]{};
                  }
                      [\tsc{f}1]
                      [\tsc{ind}P
                          [\tsc{ind}]
                          [\tsc{masc}P
                              [\tsc{masc}]
                              [\tsc{class}P
                                  [\tsc{class}]
                              ]
                          ]
                      ]
                  ]
              ]
          ]
        \end{forest}
        }
        \\
      \bottomrule
  \end{tabular}
  \caption {Modern German \tsc{ext}\scsub{nom} vs. \tsc{int}\scsub{nom} → \tit{wer}}
  \label{fig:mg-int=ext}
\end{figure}

The relative pronoun consists of three morphemes: \tit{w}, \tit{e} and \tit{r}.
The light head consists of two morphemes: \tit{ə} and \tit{r}.
As usual, I circle the part of the structure that corresponds to a particular lexical entry, and I place the corresponding phonology under it.
I draw a dashed circle around each constituent that is a constituent in both the light head and the relative pronoun.
As each constituent of the light head is also a constituent within the relative pronoun, the light head can be absent. I illustrate this by marking the content of the dashed circles for the light head gray.

I explain this constituent by constituent.
I start with the right-most constituent of the light head that spells out as \tit{r} (\tsc{nom}P). This constituent is also a constituent in the relative pronoun.
I continue with the left-most constituent of the light head that spells out as \tit{ə} (\tsc{prox}P). This constituent is also a constituent in the relative pronoun, contained in \tsc{med}P.
Both constituent of the light head are also a constituent within the relative pronoun, and the light head can be absent.

Consider the example in \ref{ex:mg-nom-acc-rep}, in which the internal accusative case competes against the external nominative case. The relative clause is marked in bold, and the light head and the relative pronoun are underlined.
The internal case is accusative, as the predicate \tit{mögen} `to like' takes accusative objects. The relative pronoun \tit{wen} `\ac{rel}.\ac{an}.\ac{acc}' appears in the accusative case. This is the element that surfaces.
The external case is nominative, as the predicate \tit{besuchen} `to visit' takes nominative subjects. The light head \tit{ər} `\ac{dem}.\ac{an}.\ac{nom}' appears in the nominative case. It is placed between square brackets because it does not surface.

\exg. Uns besucht [\underline{ər}] \underline{\tbf{wen}} \tbf{Maria} \tbf{mag}.\\
 we.\ac{acc} visit.3\ac{sg}\scsub{[nom]} \tsc{dem}.\ac{nom}.\tsc{an} \tsc{rel}.\ac{acc}.\tsc{an} Maria.\ac{nom} like.3\ac{sg}\scsub{[acc]}\\
 `Who visits us, Maria likes.' \flushfill{adapted from \pgcitealt{vogel2001}{343}}\label{ex:mg-nom-acc-rep}

In Figure \ref{fig:mg-int-wins}, I give the syntactic structure of the light head at the top and the syntactic structure of the relative pronoun at the bottom.

\begin{figure}[htbp]
  \center
  \begin{tabular}[b]{c}
      \toprule
      \tsc{nom} light head \tit{ə-r}
      \\
      \cmidrule{1-1}
      \scriptsize{
      \begin{forest} boompje
        [{\tsc{nom}P}, s sep=20mm
            [{\tsc{prox}P},
            tikz={
            \node[label=below:{\tit{ə}},
            draw,circle,
            scale=0.8,
            fit to=tree]{};
            \node[
            draw,circle,
            scale=0.85,
            fill=DG,fill opacity=0.2,
            dashed,
            fit to=tree]{};
            }
                [{\tsc{dx}\scsub{1}}]
                [\tsc{ref} [\phantom{xxx}, roof]]
            ]
            [{\tsc{nom}P},
            tikz={
            \node[label=below:{\tit{r}},
            draw,circle,
            scale=0.95,
            fit to=tree]{};
            \node[
            draw,circle,
            fill=DG,fill opacity=0.2,
            scale=1,
            dashed,
            fit to=tree]{};
            }
                [{\tsc{f}1}]
                [{\tsc{ind}P}
                    [{\tsc{ind}}]
                    [{\tsc{masc}P}
                        [{\tsc{masc}}]
                        [{\tsc{class}P}
                            [{\tsc{class}}]
                        ]
                    ]
                ]
            ]
        ]
      \end{forest}
      }
      \\
      \toprule
      \tsc{acc} relative pronoun \tit{w-e-n}
      \\
      \cmidrule{1-1}
      \scriptsize{
          \begin{forest} boompje
          [\tsc{rel}P, s sep=15mm
              [\tsc{rel}P,
              tikz={
              \node[label=below:\tit{w},
              draw,circle,
              scale=0.9,
              fit to=tree]{};
              }
                  [\tsc{rel}]
                  [\tsc{wh}]
              ]
              [\tsc{nom}P, s sep=25mm
                  [\tsc{med}P,
                  tikz={
                  \node[label=below:\tit{e},
                  draw,circle,
                  scale=0.85,
                  fit to=tree]{};
                  }
                      [\tsc{dx}\scsub{2}]
                      [\tsc{prox}P,
                      tikz={
                      \node[draw,circle,
                      dashed,
                      scale=0.8,
                      fit to=tree]{};
                      }
                          [\tsc{dx}\scsub{1}]
                          [\tsc{ref} [\phantom{xxx}, roof]]
                      ]
                  ]
                  [\tsc{acc}P,
                  tikz={
                  \node[label=below:\tit{n},
                  draw,circle,
                  scale=0.95,
                  fit to=tree]{};
                  }
                      [\tsc{f}2]
                      [\tsc{nom}P,
                      tikz={
                      \node[draw,circle,
                      dashed,
                      scale=0.9,
                      fit to=tree]{};
                      }
                          [\tsc{f}1]
                          [\tsc{ind}P
                              [\tsc{ind}]
                              [\tsc{masc}P
                                  [\tsc{masc}]
                                  [\tsc{class}P
                                      [\tsc{class}]
                                  ]
                              ]
                          ]
                      ]
                  ]
              ]
          ]
        \end{forest}
        }
        \\
      \bottomrule
  \end{tabular}
   \caption {Modern German \tsc{ext}\scsub{nom} vs. \tsc{int}\scsub{acc} → \tit{wen}}
  \label{fig:mg-int-wins}
\end{figure}

The relative pronoun consists of three morphemes: \tit{w}, \tit{e} and \tit{n}.
The light head consists of two morphemes: \tit{ə} and \tit{r}.
Again, I circle the part of the structure that corresponds to a particular lexical entry, and I place the corresponding phonology under it.
I draw a dashed circle around each constituent that is a constituent in both the light head and the relative pronoun.
As each constituent of the light head is also a constituent within the relative pronoun, the light head can be absent. I illustrate this by marking the content of the dashed circles for the light head gray.

I explain this constituent by constituent.
I start with the right-most constituent of the light head that spells out as \tit{r} (\tsc{nom}P). This constituent is also a constituent in the relative pronoun, contained in \tsc{acc}P.
I continue with the left-most constituent of the light head that spells out as \tit{ə} (\tsc{prox}P). This constituent is also a constituent in the relative pronoun, contained in \tsc{med}P.
Both constituent of the light head are also a constituent within the relative pronoun, and the light head can be absent.

Consider the examples in \ref{ex:mg-acc-nom-rep}, in which the internal nominative case competes against the external accusative case. The relative clauses are marked in bold, and the light heads and the relative pronouns are underlined. It is not possible to make a grammatical headless relative in this situation.
The internal case is nominative, as the predicate \tit{sein} `to be' takes nominative subjects. The relative pronoun \tit{wer} `\ac{rel}.\ac{an}.\ac{nom}' appears in the nominative case.
The external case is accusative, as the predicate \tit{einladen} `to invite' takes accusative objects. The light head \tit{ən} `\ac{dem}.\ac{an}.\ac{acc}' appears in the accusative case.
\ref{ex:mg-acc-nom-rep-rel} is the variant of the sentence in which the light head is absent (indicated by the square brackets) and the relative pronoun surfaces, and it is ungrammatical.
\ref{ex:mg-acc-nom-rep-lh} is the variant of the sentence in which the relative pronoun is absent (indicated by the square brackets) and the light head surfaces, and it is ungrammatical too.

\ex.\label{ex:mg-acc-nom-rep}
\ag. *Ich {lade ein}, [\underline{ən}] \underline{\tbf{wer}} \tbf{mir} \tbf{sympathisch} \tbf{ist}.\\
1\ac{sg}.\ac{nom} invite.\ac{pres}.1\ac{sg}\scsub{[acc]} \ac{rel}.\ac{an}.\ac{nom} 1\ac{sg}.\ac{dat} nice be.\ac{pres}.3\ac{sg}\scsub{[nom]}\\
`I invite who I like.' \flushfill{Modern German, adapted from \pgcitealt{vogel2001}{344}}\label{ex:mg-acc-nom-rep-rel}
\bg. *Ich {lade ein}, \underline{ən} [\underline{\tbf{wer}}] \tbf{mir} \tbf{sympathisch} \tbf{ist}.\\
1\ac{sg}.\ac{nom} invite.\ac{pres}.1\ac{sg}\scsub{[acc]} \ac{rel}.\ac{an}.\ac{nom} 1\ac{sg}.\ac{dat} nice be.\ac{pres}.3\ac{sg}\scsub{[nom]}\\
`I invite who I like.' \flushfill{Modern German, adapted from \pgcitealt{vogel2001}{344}}\label{ex:mg-acc-nom-rep-lh}

In Figure \ref{fig:mg-ext-wins}, I give the syntactic structure of the light head at the top and the syntactic structure of the relative pronoun at the bottom.

\begin{figure}[htbp]
  \center
  \begin{tabular}[b]{c}
        \toprule
        \tsc{acc} light head \tit{ə-n} \\
        \cmidrule{1-1}
      \scriptsize{
      \begin{forest} boompje
        [{\tsc{acc}P}, s sep=20mm
            [{\tsc{prox}P},
            tikz={
            \node[label=below:{\tit{ə}},
            draw,circle,
            scale=0.8,
            fit to=tree]{};
            \node[
            draw,circle,
            scale=0.85,
            dashed,
            fit to=tree]{};
            }
                [{\tsc{dx}\scsub{1}}]
                [\tsc{ref} [\phantom{xxx}, roof]]
            ]
            [{\tsc{acc}P},
            tikz={
            \node[label=below:{\tit{n}},
            draw,circle,
            scale=0.95,
            fit to=tree]{};
            }
                [\tsc{f}2]
                [\tsc{nom}P,
                tikz={
                \node[
                draw,circle,
                scale=0.9,
                dashed,
                fit to=tree]{};
                }
                    [{\tsc{f}1}]
                    [{\tsc{ind}P}
                        [{\tsc{ind}}]
                        [{\tsc{masc}P}
                            [{\tsc{masc}}]
                            [{\tsc{class}P}
                                [{\tsc{class}}]
                            ]
                        ]
                    ]
                ]
            ]
        ]
      \end{forest}
      }
      \\
      \toprule
      \tsc{nom} relative pronoun \tit{w-e-r}
      \\
      \cmidrule{1-1}
      \scriptsize{
      \begin{forest} boompje
      [\tsc{rel}P, s sep=15mm
          [\tsc{rel}P,
          tikz={
          \node[label=below:\tit{w},
          draw,circle,
          scale=0.9,
          fit to=tree]{};
          }
              [\tsc{rel}]
              [\tsc{wh}]
          ]
          [\tsc{nom}P, s sep=25mm
              [\tsc{med}P,
              tikz={
              \node[label=below:\tit{e},
              draw,circle,
              scale=0.85,
              fit to=tree]{};
              }
                  [\tsc{dx}\scsub{2}]
                  [\tsc{prox}P,
                  tikz={
                  \node[draw,circle,
                  dashed,
                  scale=0.8,
                  fit to=tree]{};
                  }
                      [\tsc{dx}\scsub{1}]
                      [\tsc{ref} [\phantom{xxx}, roof]]
                  ]
              ]
              [\tsc{nom}P,
              tikz={
              \node[label=below:\tit{r},
              draw,circle,
              scale=0.95,
              fit to=tree]{};
              \node[draw,circle,
              dashed,
              scale=1,
              fit to=tree]{};
              }
                  [\tsc{f}1]
                  [\tsc{ind}P
                      [\tsc{ind}]
                      [\tsc{masc}P
                          [\tsc{masc}]
                          [\tsc{class}P
                              [\tsc{class}]
                          ]
                      ]
                  ]
              ]
          ]
      ]
    \end{forest}
        }
      \\
      \bottomrule
  \end{tabular}
   \caption {Modern German \tsc{ext}\scsub{acc} vs. \tsc{int}\scsub{nom} ↛ \tit{wer}/\tit{ən}}
  \label{fig:mg-ext-wins}
\end{figure}

The relative pronoun consists of three morphemes: \tit{w}, \tit{e} and \tit{r}.
The light head consists of two morphemes: \tit{ə} and \tit{n}.
Again, I circle the part of the structure that corresponds to a particular lexical entry, and I place the corresponding phonology under it.
I draw a dashed circle around each constituent that is a constituent in both the light head and the relative pronoun.
Neither of the elements contains all constituents that the other element contains. The relative pronoun does not contain all constituents that the light head contains, and the light head does not contain all constituents that the relative pronoun contains. As a result, none of the elements can be absent.\footnote{
Why do we not see this result surface? Very good question.
}

I explain this constituent by constituent.
I start by showing that the light head cannot be absent.
Consider the right-most constituent of the light head that spells out as \tit{n} (\tsc{acc}P). This constituent is not a constituent in the relative pronoun: the relative pronoun has a constituent \tsc{nom}P, but it does not contain \tsc{f}2 to make it an \tsc{acc}P.
The light head has a constituent that is not a constituent in the relative pronoun, so the light head cannot be absent.

The relative pronoun can also not be absent.
Consider the middle constituent of the relative pronoun that spells out as \tit{e} (\tsc{med}P). This constituent is not a constituent in the light head: the light head has a constituent \tsc{med}P, but it does not contain \tsc{dx}\scsub{3} to make it an \tsc{med}P.
The same hold for the left-most constituent of the relative pronoun that spells out as \tit{w} (\tsc{rel}P). The light head lacks the features \tsc{wh} and \tsc{rel} that form the \tsc{rel}P.
The relative pronoun has constituents that are not constituents in the light head, so the relative pronoun cannot be absent.
In sum, neither of the elements contains all constituents that the other element contains, and none of the elements can be absent, so none of them is marked gray.


\section{Deriving the matching type}\label{sec:deriving-matching}

Matching languages can be summarizes as in Table \ref{tbl:overview-rel-light-polish}.

\begin{table}[htbp]
  \center
  \caption{The surface pronoun with differing cases in Polish}
\begin{tabular}{cccc}
  \toprule
                & \tsc{k}\scsub{int} > \tsc{k}\scsub{ext} & \tsc{k}\scsub{ext} > \tsc{k}\scsub{int} &   \\
                \cmidrule{2-3}
matching        & *                            & *                     & Polish           \\
\bottomrule
\end{tabular}
\label{tbl:overview-rel-light-polish}
\end{table}

A language of the matching type (like Polish) allows neither the internal nor the external case to surface when either of them wins the case competition. Neither the relative pronoun with its internal case nor the light head with its external case can be the surface pronoun. The goal of this section is to derive this from the way light heads and relative pronouns are spelled out in Polish.

I argue that Polish headless relatives are derived from regular light-headed relatives.
I decompose the light heads and relative pronouns intro smaller morphemes, and I show which features each of the morphemes corresponds to.
Finally, I compare the constituents of the light head and the relative pronoun.
When the internal and the external case match, the relative pronoun can delete the light head, because it contains all its constituents.
This does not work when the internal case is more complex than the external case. The relative pronoun does not contain all constituents of the light head, and the light head does not contain all constituents of the relative pronoun. As a result, there is no grammatical form to surface when the external case is more complex.
The same holds when when the external case is more complex than the internal case. The light head does not contain all constituents of the relative pronoun, and the relative pronoun does not contain all constituents of the light head. As a result, there is no grammatical form to surface when the external case is more complex.



\subsection{The light-headed relative clause}

% The idea that a different constituency leads to the absence of different syntactic structured is illustrated by Cinque with nouns and adjectives. > move this the polish part

\ex.
\ag. Jan lubi \underline{tego} \underline{\tbf{kogo}} \tbf{Maria} \tbf{lubi}.\\
 Jan like.\tsc{3sg}\scsub{[acc]} \tsc{dem}.\tsc{acc}.\tsc{an}.\tsc{sg}  \tsc{rel}.\tsc{acc}.\tsc{an}.\tsc{sg} Maria like.\tsc{3sg}\scsub{[acc]}\\
 `Jan likes whoever Maria likes.' \flushfill{Polish, adapted from \citealt{citko2013} after \pgcitealt{himmelreich2017}{17}}\label{ex:polish-acc-acc-rep}
\bg. *Jan lubi \underline{tego} \underline{\tbf{kogo}} \tbf{-kolkwiek} \tbf{Maria} \tbf{lubi}.\\
 Jan like.\tsc{3sg}\scsub{[acc]} \tsc{dem}.\tsc{acc}.\tsc{an}.\tsc{sg}  \tsc{rel}.\tsc{acc}.\tsc{an}.\tsc{sg} ever Maria like.\tsc{3sg}\scsub{[acc]}\\
 `Jan likes whoever Maria likes.' \flushfill{Polish, adapted from \citealt{citko2013} after \pgcitealt{himmelreich2017}{17}}\label{ex:polish-acc-acc-rep}
\bg. Jan lubi [\underline{ego}] \underline{\tbf{kogo}} \tbf{-kolkwiek} \tbf{Maria} \tbf{lubi}.\\
Jan like.\tsc{3sg}\scsub{[acc]} \tsc{lh}.\tsc{acc}.\tsc{an}.\tsc{sg}  \tsc{rel}.\tsc{acc}.\tsc{an}.\tsc{sg} ever Maria like.\tsc{3sg}\scsub{[acc]}\\
`Jan likes whoever Maria likes.' \flushfill{Polish, adapted from \citealt{citko2013} after \pgcitealt{himmelreich2017}{17}}\label{ex:polish-acc-acc-rep}

here:
Alternative account for why the demonstrative and \tsc{ever} are incompatible: it is the same syntactic position

\subsection{Decomposing light heads and relative pronouns}

\subsection{Recomposing light heads and relative pronouns}

\ex. \tbf{Backtracking} \citep{starke2018}:\\
When spellout fails, go back to the previous cycle, and try the next option for that cycle.\label{ex:backtracking}

\ex. Polish: \tsc{ext} \tsc{dat}\\
\tiny{
\begin{forest} boompje
  [\tsc{prox}P, s sep=15mm
      [\tsc{prox}P,
      tikz={
      \node[label=below:\tit{t},
      draw,circle,
      scale=0.9,
      fit to=tree]{};
      }
          [\tsc{deix\scsub{1}}, roof]
      ]
      [\tsc{dat}P, s sep=20mm
          [\tsc{masc}P,
          tikz={
          \node[label=below:\tit{e/o},
          draw,circle,
          scale=0.85,
          fit to=tree]{};
          }
              [\tsc{masc}]
              [\tsc{class}P
                  [\tsc{class}]
                  [\tsc{ref} [\phantom{xxx}, roof]]
              ]
          ]
          [\tsc{dat}P,
          tikz={
          \node[label=below:\tit{mu},
          draw,circle,
          scale=0.9,
          fit to=tree]{};
          }
              [\tsc{f}3]
              [\tsc{acc}P
                  [\tsc{f}2]
                  [\tsc{nom}P
                      [\tsc{f}1]
                      [\tsc{ind}P
                          [\tsc{ind}]
                      ]
                  ]
              ]
          ]
      ]
  ]
\end{forest}
}

\ex. Polish: \tsc{ext} \tsc{acc}\\
\tiny{
\begin{forest} boompje
  [\tsc{prox}P, s sep=15mm
      [\tsc{prox}P,
      tikz={
      \node[label=below:\tit{t},
      draw,circle,
      scale=0.9,
      fit to=tree]{};
      }
          [\tsc{deix\scsub{1}}, roof]
      ]
      [\tsc{dat}P, s sep=25mm
          [\tsc{ind}P,
          tikz={
          \node[label=below:\tit{e/o},
          draw,circle,
          scale=0.9,
          fit to=tree]{};
          }
              [\tsc{ind}]
              [\tsc{mascP}
                  [\tsc{masc}]
                  [\tsc{class}P
                      [\tsc{class}]
                      [\tsc{ref} [\phantom{xxx}, roof]]
                  ]
              ]
          ]
          [\tsc{acc}P,
          tikz={
          \node[label=below:\tit{go},
          draw,circle,
          scale=0.85,
          fit to=tree]{};
          }
              [\tsc{f}2]
              [\tsc{nom}P
                  [\tsc{f}1]
              ]
          ]
      ]
  ]
\end{forest}
}

\ex. Polish: \tsc{int} \tsc{dat}\\
\tiny{
\begin{forest} boompje
  [\tsc{rel}P, s sep=17mm
      [\tsc{rel}P,
      tikz={
      \node[label=below:\tit{k},
      draw,circle,
      scale=0.925,
      fit to=tree]{};
      }
          [\tsc{rel}]
          [\tsc{wh}P
              [\tsc{wh}]
              [\tsc{med}P,
                  [\tsc{deix\scsub{2}}]
                  [\tsc{deix\scsub{1}}]
              ]
          ]
      ]
      [\tsc{dat}P, s sep=17.5mm
          [\tsc{masc}P,
          tikz={
          \node[label=below:\tit{e/o},
          draw,circle,
          scale=0.9,
          fit to=tree]{};
          }
              [\tsc{masc}]
              [\tsc{class}P
                  [\tsc{class}]
                  [\tsc{ref} [\phantom{xxx}, roof]]
              ]
          ]
          [\tsc{dat}P,
          tikz={
          \node[label=below:\tit{mu},
          draw,circle,
          scale=0.9,
          fit to=tree]{};
          }
              [\tsc{f}3]
              [\tsc{acc}P
                  [\tsc{f}2]
                  [\tsc{nom}P
                      [\tsc{f}1]
                      [\tsc{ind}P
                          [\tsc{ind}]
                      ]
                  ]
              ]
          ]
      ]
  ]
\end{forest}
}

\ex. Polish: \tsc{int} \tsc{acc}\\
\tiny{
\begin{forest} boompje
  [\tsc{rel}P, s sep=17mm
      [\tsc{rel}P,
      tikz={
      \node[label=below:\tit{k},
      draw,circle,
      scale=0.925,
      fit to=tree]{};
      }
          [\tsc{rel}]
          [\tsc{wh}P
              [\tsc{wh}]
              [\tsc{med}P,
                  [\tsc{deix\scsub{2}}]
                  [\tsc{deix\scsub{1}}]
              ]
          ]
      ]
      [\tsc{acc}P, s sep=24mm
          [\tsc{ind}P,
          tikz={
          \node[label=below:\tit{e/o},
          draw,circle,
          scale=0.9,
          fit to=tree]{};
          }
              [\tsc{ind}]
              [\tsc{masc}P
                  [\tsc{masc}]
                  [\tsc{class}P
                      [\tsc{class}]
                      [\tsc{ref} [\phantom{xxx}, roof]]
                  ]
              ]
          ]
          [\tsc{acc}P,
          tikz={
          \node[label=below:\tit{go},
          draw,circle,
          scale=0.85,
          fit to=tree]{};
          }
              [\tsc{f}2]
              [\tsc{nom}P
                  [\tsc{f}1]
              ]
          ]
      ]
  ]
\end{forest}
}

\subsection{Comparing consituents}

Consider the example in \ref{ex:polish-acc-acc-rep}, in which the internal accusative case competes against the external accusative case. The relative clause is marked in bold, and the light head and the relative pronoun are underlined.
The internal case is accusative, as the predicate \tit{lubić} `to like' takes accusative objects. The relative pronoun \tit{kogo} `\ac{rel}.\ac{an}.\ac{acc}' appears in the accusative case. This is the element that surfaces.
The external case is accusative as well, as the predicate \tit{lubić} `to like' also takes accusative objects. The light head \tit{tego} `\ac{dem}.\ac{an}.\ac{acc}' appears in the accusative case. It is placed between square brackets because it does not surface.

\exg. Jan lubi \underline{[tego]} \underline{\tbf{kogo}} \tbf{-kolkwiek} \tbf{Maria} \tbf{lubi}.\\
 Jan like.\tsc{3sg}\scsub{[acc]} \tsc{dem}.\tsc{acc}.\tsc{an}.\tsc{sg}  \tsc{rel}.\tsc{acc}.\tsc{an}.\tsc{sg} ever Maria like.\tsc{3sg}\scsub{[acc]}\\
 `Jan likes whoever Maria likes.' \flushfill{Polish, adapted from \citealt{citko2013} after \pgcitealt{himmelreich2017}{17}}\label{ex:polish-acc-acc-rep}

In Figure \ref{fig:polish-int=ext}, I give the syntactic structure of the light head at the top and the syntactic structure of the relative pronoun at the bottom.

\begin{figure}[htbp]
  \center
  \begin{tabular}[b]{c}
        \toprule
        \tsc{acc} light head \tit{t-e-go} \\
        \cmidrule{1-1}
        \tiny{
        \begin{forest} boompje
          [\tsc{prox}P, s sep=15mm
              [\tsc{prox}P,
              tikz={
              \node[label=below:\tit{t},
              draw,circle,
              scale=0.9,
              fit to=tree]{};
              \node[
              draw,circle,
              scale=1,
              dashed,
              fill=DG,fill opacity=0.2,
              fit to=tree]{};
              }
                  [\tsc{deix\scsub{1}}, roof]
              ]
              [\tsc{acc}P, s sep=25mm
                  [\tsc{ind}P,
                  tikz={
                  \node[label=below:\tit{e/o},
                  draw,circle,
                  scale=0.9,
                  fit to=tree]{};
                  \node[
                  draw,circle,
                  scale=0.95,
                  dashed,
                  fill=DG,fill opacity=0.2,
                  fit to=tree]{};
                  }
                      [\tsc{ind}]
                      [\tsc{mascP}
                          [\tsc{masc}]
                          [\tsc{class}P
                              [\tsc{class}]
                              [\tsc{ref} [\phantom{xxx}, roof]]
                          ]
                      ]
                  ]
                  [\tsc{acc}P,
                  tikz={
                  \node[label=below:\tit{go},
                  draw,circle,
                  scale=0.85,
                  fit to=tree]{};
                  \node[
                  draw,circle,
                  scale=0.9,
                  dashed,
                  fill=DG,fill opacity=0.2,
                  fit to=tree]{};
                  }
                      [\tsc{f}2]
                      [\tsc{nom}P
                          [\tsc{f}1]
                      ]
                  ]
              ]
          ]
        \end{forest}
        }
      \\
      \toprule
      \tsc{acc} relative pronoun \tit{k-o-go}
      \\
      \cmidrule{1-1}
      \tiny{
      \begin{forest} boompje
        [\tsc{rel}P, s sep=22mm
            [\tsc{rel}P,
            tikz={
            \node[label=below:\tit{k},
            draw,circle,
            scale=0.95,
            fit to=tree]{};
            }
                [\tsc{rel}]
                [\tsc{wh}P
                    [\tsc{wh}]
                    [\tsc{med}P
                        [\tsc{deix\scsub{2}}]
                        [\tsc{prox}P,
                        tikz={
                        \node[
                        draw,circle,
                        scale=0.8,
                        dashed,
                        fit to=tree]{};
                        }
                            [\tsc{deix\scsub{1}}, roof]
                        ]
                    ]
                ]
            ]
            [\tsc{acc}P, s sep=25mm
                [\tsc{ind}P,
                tikz={
                \node[label=below:\tit{e/o},
                draw,circle,
                scale=0.9,
                fit to=tree]{};
                \node[
                draw,circle,
                scale=0.95,
                dashed,
                fit to=tree]{};
                }
                    [\tsc{ind}]
                    [\tsc{masc}P
                        [\tsc{masc}]
                        [\tsc{class}P
                            [\tsc{class}]
                            [\tsc{ref} [\phantom{xxx}, roof]]
                        ]
                    ]
                ]
                [\tsc{acc}P,
                tikz={
                \node[label=below:\tit{go},
                draw,circle,
                scale=0.85,
                fit to=tree]{};
                \node[
                draw,circle,
                scale=0.9,
                dashed,
                fit to=tree]{};
                }
                    [\tsc{f}2]
                    [\tsc{nom}P
                        [\tsc{f}1]
                    ]
                ]
            ]
        ]
      \end{forest}
      }
      \\
      \bottomrule
  \end{tabular}
   \caption {Polish \tsc{ext}\scsub{acc} vs. \tsc{int}\scsub{acc} → \tit{kogo}}
  \label{fig:polish-int=ext}
\end{figure}

The relative pronoun consists of three morphemes: \tit{k}, \tit{o} and \tit{go}.
The light head consists of three morphemes: \tit{t}, \tit{e} and \tit{go}.
As usual, I circle the part of the structure that corresponds to a particular lexical entry, and I place the corresponding phonology under it.
I draw a dashed circle around each constituent that is a constituent in both the light head and the relative pronoun.
As each constituent of the light head is also a constituent within the relative pronoun, the light head can be absent. I illustrate this by marking the content of the dashed circles for the light head gray.

I explain this constituent by constituent.
I start with the right-most constituent of the light head that spells out as \tit{go} (\tsc{acc}P). This constituent is also a constituent in the relative pronoun.
I continue with the middle constituent of the light head that spells out as \tit{e} (\tsc{ind}P). This constituent is also a constituent in the relative pronoun.
I continue with the left-most constituent of the light head that spells out as \tit{t} (\tsc{med}P). This constituent is also a constituent in the relative pronoun, contained in \tsc{rel}P.
All three constituent of the light head are also a constituent within the relative pronoun, and the light head can be absent.

Consider the examples in \ref{ex:polish-dat-acc-rep}, in which the internal dative case competes against the external accusative case. The relative clauses are marked in bold, and the light heads and the relative pronouns are underlined. It is not possible to make a grammatical headless relative in this situation.
The internal case is dative, as the predicate \tit{dokuczać} `to tease' takes dative objects. The relative pronoun \tit{komu} `\ac{rel}.\ac{an}.\ac{dat}' appears in the nominative case.
The external case is accusative, as the predicate \tit{lubić} `to like' takes accusative objects. The light head \tit{tego} `\ac{dem}.\ac{an}.\ac{acc}' appears in the accusative case.
\ref{ex:polish-dat-acc-rep-rel} is the variant of the sentence in which the light head is absent (indicated by the square brackets) and the relative pronoun surfaces, and it is ungrammatical.
\ref{ex:polish-dat-acc-rep-lh} is the variant of the sentence in which the relative pronoun is absent (indicated by the square brackets) and the light head surfaces, and it is ungrammatical too.

\ex.
\ag. *Jan lubi [\underline{tego}] \underline{\tbf{komu}} \tbf{-kolkwiek} \tbf{dokucza}.\\
Jan like.\tsc{3sg}\scsub{[acc]} \tsc{dem}.\tsc{acc}.\tsc{an}.\tsc{sg} \tsc{rel}.\tsc{dat}.\tsc{an}.\tsc{sg} ever tease.\tsc{3sg}\scsub{[dat]}\\
`Jan likes whoever he teases.' \flushfill{Polish, adapted from \citealt{citko2013} after \pgcitealt{himmelreich2017}{17}}\label{ex:polish-acc-dat-rep-rel}
\bg. *Jan lubi \underline{tego} [\underline{\tbf{komu}}] \tbf{-kolkwiek} \tbf{dokucza}.\\
Jan like.\tsc{3sg}\scsub{[acc]} \tsc{dem}.\tsc{acc}.\tsc{an}.\tsc{sg} \tsc{rel}.\tsc{dat}.\tsc{an}.\tsc{sg} ever tease.\tsc{3sg}\scsub{[dat]}\\
`Jan likes whoever he teases.' \flushfill{Polish, adapted from \citealt{citko2013} after \pgcitealt{himmelreich2017}{17}}\label{ex:polish-acc-dat-rep-lh}

In Figure \ref{fig:polish-int-wins}, I give the syntactic structure of the light head at the top and the syntactic structure of the relative pronoun at the bottom.

\begin{figure}[htbp]
  \center
  \begin{tabular}[b]{c}
        \toprule
        \tsc{acc} light head \tit{t-e-go} \\
        \cmidrule{1-1}
        \tiny{
        \begin{forest} boompje
          [\tsc{prox}P, s sep=13mm
              [\tsc{prox}P,
              tikz={
              \node[label=below:\tit{t},
              draw,circle,
              scale=0.9,
              fit to=tree]{};
              \node[
              draw,circle,
              scale=1,
              dashed,
              fit to=tree]{};
              }
                  [\tsc{deix\scsub{1}}, roof]
              ]
              [\tsc{acc}P, s sep=21mm
                  [\tsc{ind}P,
                  tikz={
                  \node[label=below:\tit{e/o},
                  draw,circle,
                  scale=0.85,
                  fit to=tree]{};
                  }
                      [\tsc{ind}]
                      [\tsc{mascP},
                      tikz={
                      \node[
                      draw,circle,
                      scale=0.8,
                      dashed,
                      fit to=tree]{};
                      }
                          [\tsc{masc}]
                          [\tsc{class}P
                              [\tsc{class}]
                              [\tsc{ref} [\phantom{xxx}, roof]]
                          ]
                      ]
                  ]
                  [\tsc{acc}P,
                  tikz={
                  \node[label=below:\tit{go},
                  draw,circle,
                  scale=0.85,
                  fit to=tree]{};
                  }
                      [\tsc{f}2]
                      [\tsc{nom}P
                          [\tsc{f}1]
                      ]
                  ]
              ]
          ]
        \end{forest}
        }
      \\
      \toprule
      \tsc{acc} relative pronoun \tit{k-o-mu}
      \\
      \cmidrule{1-1}
      \tiny{
      \begin{forest} boompje
        [\tsc{rel}P, s sep=20mm
            [\tsc{rel}P,
            tikz={
            \node[label=below:\tit{k},
            draw,circle,
            scale=0.95,
            fit to=tree]{};
            }
                [\tsc{rel}]
                [\tsc{wh}P
                    [\tsc{wh}]
                    [\tsc{med}P
                        [\tsc{deix\scsub{2}}]
                        [\tsc{prox}P,
                        tikz={
                        \node[
                        draw,circle,
                        scale=0.8,
                        dashed,
                        fit to=tree]{};
                        }
                            [\tsc{deix\scsub{1}}, roof]
                        ]
                    ]
                ]
            ]
            [\tsc{dat}P, s sep=18mm
                [\tsc{masc}P,
                tikz={
                \node[label=below:\tit{e/o},
                draw,circle,
                scale=0.85,
                fit to=tree]{};
                \node[
                draw,circle,
                scale=0.9,
                dashed,
                fit to=tree]{};
                }
                    [\tsc{masc}]
                    [\tsc{class}P
                        [\tsc{class}]
                        [\tsc{ref} [\phantom{xxx}, roof]]
                    ]
                ]
                [\tsc{dat}P,
                tikz={
                \node[label=below:\tit{mu},
                draw,circle,
                scale=0.9,
                fit to=tree]{};
                }
                    [\tsc{f}3]
                    [\tsc{acc}P
                        [\tsc{f}2]
                        [\tsc{nom}P
                            [\tsc{f}1]
                            [\tsc{ind}P
                                [\tsc{ind}]
                            ]
                        ]
                    ]
                ]
            ]
        ]
      \end{forest}
      }
      \\
      \bottomrule
  \end{tabular}
   \caption {Polish \tsc{ext}\scsub{acc} vs. \tsc{int}\scsub{dat} ↛ \tit{tego}/\tit{komu}}
  \label{fig:polish-int-wins}
\end{figure}

The relative pronoun consists of three morphemes: \tit{k}, \tit{o} and \tit{mu}.
The light head consists of three morphemes: \tit{t}, \tit{e} and \tit{go}.
Again, I circle the part of the structure that corresponds to a particular lexical entry, and I place the corresponding phonology under it.
I draw a dashed circle around each constituent that is a constituent in both the light head and the relative pronoun.
Neither of the elements contains all constituents that the other element contains. The relative pronoun does not contain all constituents that the light head contains, and the light head does not contain all constituents that the relative pronoun contains. As a result, none of the elements can be absent.

I explain this constituent by constituent.
I start by showing that the light head cannot be absent.
Consider the right-most constituent of the light head that spells out as \tit{go} (\tsc{acc}P). This constituent is not a constituent in the relative pronoun: the relative pronoun has a constituent \tsc{acc}P, contained in \tsc{dat}P, but that constituent also contains \tit{ind}P.
Consider the middle constituent of the light head that spells out as \tit{e} (\tsc{ind}P). This constituent is not a constituent in the relative pronoun: the relative pronoun has a constituent \tsc{ind}P, contained in \tsc{dat}P, but that constituent does not contain \tsc{ref}, \tsc{class} and \tsc{masc}.
Note here that there is feature containment: the relative pronoun contains all features that the light head contains. It is here crucial to use the stronger constituent containment requirement.
The light head has a constituent that is not a constituent in the relative pronoun, so the light head cannot be absent.

The relative pronoun can also not be absent.
Consider the right-most constituent of the relative pronoun that spells out as \tit{mu} (\tsc{dat}P). This constituent is not a constituent in the light head: the light head has a constituent \tsc{acc}P, but it does not contain \tsc{ind}P and also not \tsc{f}3 to make it a \tsc{dat}P.
The same hold for the left-most constituent of the relative pronoun that spells out as \tit{k} (\tsc{rel}P). The light head lacks the features \tsc{med}, \tsc{wh} and \tsc{rel} that form the \tsc{rel}P.
The relative pronoun has constituents that are not constituents in the light head, so the relative pronoun cannot be absent.
In sum, neither of the elements contains all constituents that the other element contains, and none of the elements can be absent, so none of them is marked gray.

Consider the examples in \ref{ex:polish-dat-acc-rep}, in which the internal dative case competes against the external accusative case. The relative clauses are marked in bold, and the light heads and the relative pronouns are underlined. It is not possible to make a grammatical headless relative in this situation.
The internal case is accusative, as the predicate \tit{wpuścić} `to let' takes accusative objects. The relative pronoun \tit{kogo} `\ac{rel}.\ac{an}.\ac{acc}' appears in the nominative case.
The external case is dative, as the predicate \tit{ufać} `to trust' takes dative objects. The light head \tit{temu} `\ac{dem}.\ac{an}.\ac{dat}' appears in the accusative case.
\ref{ex:polish-dat-acc-rep-rel} is the variant of the sentence in which the light head is absent (indicated by the square brackets) and the relative pronoun surfaces, and it is ungrammatical.
\ref{ex:polish-dat-acc-rep-lh} is the variant of the sentence in which the relative pronoun is absent (indicated by the square brackets) and the light head surfaces, and it is ungrammatical too.

\ex.
\ag. *Jan ufa [\underline{temu}] \underline{\tbf{kogo}} \tbf{-kolkwiek} \tbf{wpuścil} \tbf{do} \tbf{domu}.\\
Jan trust.\tsc{3sg}\scsub{[dat]} \tsc{rel}.\tsc{dat}.\tsc{an}.\tsc{sg} ever let.\tsc{3sg}\scsub{[acc]} to home\\
`Jan trusts whoever he let into the house.' \flushfill{Polish, adapted from \citealt{citko2013} after \pgcitealt{himmelreich2017}{17}}\label{ex:polish-dat-acc-rel}
\bg. Jan ufa \underline{temu} [\underline{\tbf{kogo}}] \tbf{-kolkwiek} \tbf{wpuścil} \tbf{do} \tbf{domu}.\\
Jan trust.\tsc{3sg}\scsub{[dat]} \tsc{rel}.\tsc{dat}.\tsc{an}.\tsc{sg} ever let.\tsc{3sg}\scsub{[acc]} to home\\
`Jan trusts whoever he let into the house.' \flushfill{Polish, adapted from \citealt{citko2013} after \pgcitealt{himmelreich2017}{17}}\label{ex:polish-dat-acc-lh}

In Figure \ref{fig:polish-ext-wins}, I give the syntactic structure of the light head at the top and the syntactic structure of the relative pronoun at the bottom.

\begin{figure}[htbp]
  \center
  \begin{tabular}[b]{c}
        \toprule
        \tsc{dat} light head \tit{t-e-mu} \\
        \cmidrule{1-1}
        \tiny{
        \begin{forest} boompje
          [\tsc{prox}P, s sep=15mm
              [\tsc{prox}P,
              tikz={
              \node[label=below:\tit{t},
              draw,circle,
              scale=0.9,
              fit to=tree]{};
              \node[
              draw,circle,
              scale=1,
              dashed,
              fit to=tree]{};
              }
                  [\tsc{deix\scsub{1}}, roof]
              ]
              [\tsc{dat}P, s sep=20mm
                  [\tsc{masc}P,
                  tikz={
                  \node[label=below:\tit{e/o},
                  draw,circle,
                  scale=0.9,
                  fit to=tree]{};
                  \node[
                  draw,circle,
                  scale=0.95,
                  dashed,
                  fit to=tree]{};
                  }
                      [\tsc{masc}]
                      [\tsc{class}P
                          [\tsc{class}]
                          [\tsc{ref} [\phantom{xxx}, roof]]
                      ]
                  ]
                  [\tsc{dat}P,
                  tikz={
                  \node[label=below:\tit{mu},
                  draw,circle,
                  scale=0.9,
                  fit to=tree]{};
                  }
                      [\tsc{f}3]
                      [\tsc{acc}P
                          [\tsc{f}2]
                          [\tsc{nom}P
                              [\tsc{f}1]
                              [\tsc{ind}P
                                  [\tsc{ind}]
                              ]
                          ]
                      ]
                  ]
              ]
          ]
        \end{forest}
        }
      \\
      \toprule
      \tsc{acc} relative pronoun \tit{k-o-go}
      \\
      \cmidrule{1-1}
      \tiny{
      \begin{forest} boompje
        [\tsc{rel}P, s sep=22mm
            [\tsc{rel}P,
            tikz={
            \node[label=below:\tit{k},
            draw,circle,
            scale=0.95,
            fit to=tree]{};
            }
                [\tsc{rel}]
                [\tsc{wh}P
                    [\tsc{wh}]
                    [\tsc{med}P
                        [\tsc{deix\scsub{2}}]
                        [\tsc{prox}P,
                        tikz={
                        \node[
                        draw,circle,
                        scale=0.8,
                        dashed,
                        fit to=tree]{};
                        }
                            [\tsc{deix\scsub{1}}, roof]
                        ]
                    ]
                ]
            ]
            [\tsc{acc}P, s sep=25mm
                [\tsc{ind}P,
                tikz={
                \node[label=below:\tit{e/o},
                draw,circle,
                scale=0.9,
                fit to=tree]{};
                \node[
                draw,circle,
                scale=0.95,
                dashed,
                fit to=tree]{};
                }
                    [\tsc{ind}]
                    [\tsc{masc}P
                        [\tsc{masc}]
                        [\tsc{class}P
                            [\tsc{class}]
                            [\tsc{ref} [\phantom{xxx}, roof]]
                        ]
                    ]
                ]
                [\tsc{acc}P,
                tikz={
                \node[label=below:\tit{go},
                draw,circle,
                scale=0.85,
                fit to=tree]{};
                \node[
                draw,circle,
                scale=0.9,
                dashed,
                fit to=tree]{};
                }
                    [\tsc{f}2]
                    [\tsc{nom}P
                        [\tsc{f}1]
                    ]
                ]
            ]
        ]
      \end{forest}
      }
      \\
      \bottomrule
  \end{tabular}
   \caption {Polish \tsc{ext}\scsub{dat} vs. \tsc{int}\scsub{acc} ↛ \tit{temu}/\tit{kogo}}
  \label{fig:polish-ext-wins}
\end{figure}

The relative pronoun consists of three morphemes: \tit{k}, \tit{o} and \tit{go}.
The light head consists of three morphemes: \tit{t}, \tit{e} and \tit{mu}.
Again, I circle the part of the structure that corresponds to a particular lexical entry, and I place the corresponding phonology under it.
I draw a dashed circle around each constituent that is a constituent in both the light head and the relative pronoun.
Neither of the elements contains all constituents that the other element contains. The relative pronoun does not contain all constituents that the light head contains, and the light head does not contain all constituents that the relative pronoun contains. As a result, none of the elements can be absent.

I explain this constituent by constituent.
I start by showing that the light head cannot be absent.
Consider the right-most constituent of the light head that spells out as \tit{mu} (\tsc{dat}P). This constituent is not a constituent in the relative pronoun: the relative pronoun has a constituent \tsc{acc}P, but it does not contain \tsc{ind}P and also not \tsc{f}3 to make it a \tsc{dat}P.
Consider the middle constituent of the light head that spells out as \tit{e} (\tsc{ind}P). This constituent is not a constituent in the relative pronoun: the relative pronoun has a constituent \tsc{ind}P, contained in \tsc{dat}P, but that constituent does not contain \tsc{ref}, \tsc{class} and \tsc{masc}.
The light head has a constituent that is not a constituent in the relative pronoun, so the light head cannot be absent.

The relative pronoun can also not be absent.
Consider the right-most constituent of the relative pronoun that spells out as \tit{go} (\tsc{acc}P). This constituent is not a constituent in the light head: the light head has a constituent \tsc{acc}P, contained in \tsc{dat}P, but that constituent also contains \tit{ind}P.
Consider left-most constituent of the relative pronoun that spells out as \tit{k} (\tsc{rel}P). The light head lacks the features \tsc{med}, \tsc{wh} and \tsc{rel} that form the \tsc{rel}P.
The relative pronoun has constituents that are not constituents in the light head, so the relative pronoun cannot be absent.
In sum, neither of the elements contains all constituents that the other element contains, and none of the elements can be absent, so none of them is marked gray.


Polish only allows the deletion of the light head in the matching situation. It is not obligatory there, you can just as well have a light-headed relative. The deletion is possible, because you have two elements that are pretty similar?

\exg. Jan czyta to, co Maria czyta.\\
 Jan read this what Maria reads\\
 `Jan reads what Maria reads.' \flushfill{Polish, \pgcitealt{citko2004}{96}}


Radek: Czech distinguishes between accidental uniqueness and inherent uniqueness. Accidental uniqueness: with \tsc{dem}, inherent uniqueness: without \tsc{dem}.

Radek's situation:

Two student assistants A and B are at their shared workdesk, which they share with other student assistants and where there’s a computer and a couple of other things, including a book (it doesn’t really matter to whom the book belongs). A is looking for a pencil, B says

\exg. Nějaká tužka je vedle {počítače /\#toho počítače}.\\
some pencil is {next to} computer \tsc{dem} computer\\
`There’s a pencil next to the computer.'

All situations like the topic situation – A and B’s shared office (desk)– have exactly one computer in it.

\exg. Nějaká tužka je vedle {té knížky /\#knížky}\\
some pencil is {next to} \tsc{dem} book book\\
`There’s a pencil next to the book.'

There is exactly one book in the topic situation – A and B’s shared office (desk) – and it does not hold that all situation like the topic situation have exactly one book in it

Florian showed that this is different for Modern German:

\begin{table}[htbp]
\begin{tabular}{c|ccc}
\toprule
       & anaphoric                & situational uniqueness              & inherent uniqueness                 \\
       \cmidrule{2-4}
Polish & \tsc{dem}  & \cellcolor{DG}\tsc{dem}             & ∅                                   \\
German & \tsc{dem}\scsub{strong}  & \cellcolor{LG}\tsc{dem}\scsub{weak} & \cellcolor{LG}\tsc{dem}\scsub{weak} \\
\bottomrule
\end{tabular}
\end{table}

\tit{to} is incompatible with \tit{ever}, because \tit{to} makes it accidentally uniqueness and \tit{ever} requires inherent uniqueness



\section{Deriving the unrestricted type}\label{sec:deriving-nonmatching}

Internal-only languages can be summarizes as in Table \ref{tbl:overview-rel-light-ohg}.

\begin{table}[htbp]
  \center
  \caption{The surface pronoun with differing cases in Polish}
\begin{tabular}{cccc}
  \toprule
                & \tsc{k}\scsub{int} > \tsc{k}\scsub{ext} & \tsc{k}\scsub{ext} > \tsc{k}\scsub{int} &   \\
                \cmidrule{2-3}
unrestricted    & relative pronoun\scsub{int}  & light head\scsub{ext} & Old High German  \\
\bottomrule
\end{tabular}
\label{tbl:overview-rel-light-ohg}
\end{table}

A language of the unrestricted type (like Old High German) allows both the internal case and the external case to surface when either of them wins the case competition. Either the light head with its external case or the relative pronoun with its internal case can be the surface pronoun. The goal of this section is to derive this from the way light heads and relative pronouns are spelled out in Old High German.

The section is structured as follows.
Old High German differs from the other two languages I discussed in that its headless relatives have a different interpretation: they have a individuating or definite reading. This leads me to argue for slightly different functional sequences in Old High German.
I argue that Old High German headless relatives are derived from regular light-headed relatives.
I decompose the light heads and relative pronouns intro smaller morphemes, and I show which features each of the morphemes corresponds to.
Then I compare the constituents of the light head and the relative pronoun.
When the internal and the external case match, the relative pronoun can delete the light head, because it contains all its constituents.
When the internal case is more complex than the external case, the relative pronoun can still delete the light head, for the same reason: the relative pronoun contains all constituents of the light head.
The situation becomes a bit more complicated when the external case is more complex than the internal case. The light head does not contain all constituents of the relative pronoun. However, the constituent that is not contained in a constituent of the light head is syncretic with a constituent of the light head. I suggest that this syncretism is also enough to license the deletion of the relative pronoun.
Finally, I show that the effect of syncretism is not limited to Old High German and the part of the light head and relative pronoun that does not involve case. I give examples from Modern German that show that syncretism can also license the deletion of a more complex case by a less complex case.



\subsection{The light-headed relative clause}

Headless relatives in which the relative pronoun starts with a \tit{d}, such as in Old High German, seem to be linked to individuating or definite readings and not to generalizing or indefinite readings \citep[cf.][]{fuss2017}. I illustrate this with the two examples I repeat from Chapter  \ref{ch:typology}.

Consider the example in \ref{ex:ohg-nom-acc-interpretation}, repeated from Chapter \ref{ch:typology}.
In this example, the author refers to the specific person which was talked about, and not to any or every person that was talked about.

\exg. Thíz ist \tbf{then} \tbf{sie} \tbf{zéllent}\\
\ac{dem}.\ac{sg}.\ac{n}.\ac{nom} be.\ac{pres}.3\ac{sg}\scsub{[nom]} \ac{rel}.\ac{sg}.\ac{m}.\ac{acc}
3\ac{pl}.\ac{m}.\ac{nom} tell.\ac{pres}.3\ac{pl}\scsub{[acc]}\\
`this is the one whom they talk about'\\
not: `this is whoever they talk about' \flushfill{Old High German, \ac{otfrid} III 16:50}\label{ex:ohg-nom-acc-interpretation}

Consider also the example in \ref{ex:ohg-nom-acc-interpretation}, repeated from Chapter \ref{ch:typology}.
In this example, the author refers to the specific person who spoke to someone, and not to any or every person who spoke to someone.

\exg. enti aer {ant uurta} demo \tbf{zaimo} \tbf{sprah}\\
and 3\ac{sg}.\ac{m}.\ac{nom} reply.\ac{pst}.3\ac{sg}\scsub{[dat]} \ac{rel}.\ac{sg}.\ac{m}.\ac{dat} {to 3\ac{sg}.\ac{m}.\ac{dat}} speak.\ac{pst}.3\ac{sg}\scsub{[nom]}\\
`and he replied to the one who spoke to him'\\
not: `and he replied to whoever spoke to him'
 \flushfill{Old High German, \ac{mons} 7:24, adapted from \pgcitealt{pittner1995}{199}}\label{ex:ohg-dat-nom-rep}

 Consider the light-headed relative in \ref{ex:ohg-double}. \tit{Thér} `\tsc{dem}.\tsc{sg}.\tsc{m}.\tsc{nom}' is the head of the relative clause, which is the external element. \tit{Then} `\tsc{rel}.\tsc{sg}.\tsc{m}.\tsc{acc}' is the relative pronoun in the relative clause, which is the internal element.

 \exg. eno nist thiz thér then ir suochet zi arslahanne?\\
  now {not be.3\ac{sg}} \tsc{dem}.\tsc{sg}.\tsc{n}.\tsc{nom} \tsc{dem}.\tsc{sg}.\tsc{m}.\tsc{nom}
  \tsc{rel}.\tsc{sg}.\tsc{m}.\tsc{acc} 2\ac{pl}.\tsc{nom} seek.2\tsc{pl} to kill.\tsc{inf}.\ac{sg}.\tsc{dat}\\
  `Isn't this now the one, who you seek to kill?'\label{ex:ohg-double}

 The difference between a light-headed relative and a headless relative is that in headless relatives, either the internal or the external is absent. The absent element is the one that has the least complex case. This shows the presence of two elements in Old High German is optional.\footnote{
 This sharply contrasts with headless relatives in Modern German, which are always ungrammatical when both the internal and external elements surface. I come back to this in Section \ref{sec:deriving-only-internal}.
 }
 In Old High German, there are three possible constructions: the internal and external element can both surface, only the internal element can surface and only the external element can surface. If only one of the two elements surfaces, this is the element that bears the most complex case, which is either the internal or the external one, as I have shown in Chapter \ref{ch:typology}. I assume that whether both or only one of the elements surfaces is determined by information structure. In \ref{ex:ohg-double}, the external element \tit{thér} `\tsc{dem}.\tsc{sg}.\tsc{m}.\tsc{nom}' is the candidate to be absent. However, it seems plausible that this is emphasized in this sentence and that it, therefore, cannot be absent.


\subsection{Decomposing light heads and relative pronouns}

 The light head in a light-headed relative is a demonstrative pronoun. Relative and demonstrative pronouns are syncretic in Old High German \pgcitep{braune2018}{338}. Table \ref{tbl:rel-dem-ohg} gives an overview of the forms in singular and plural, neuter, masculine and feminine and nominative, accusative and dative. The pronouns consist of two morphemes: a \tit{d} and suffix that differs per number, gender and case.\footnote{
 \tit{d} can also be written as \tit{dh} and \tit{th}, \tit{ë} and \tit{ē} can also be \tit{e} and \tit{é} \pgcitep{braune2018}{339}.
 }\footnote{
 The suffix could also be further divided into a vowel and a suffix. As this is not relevant for the discussion here, I refrain from doing that.
 }

 \begin{table}[htbp]
  \center
  \caption {Relative/demonstrative pronouns in Old High German \pgcitep{braune2018}{339}}
   \begin{tabular}{cccc}
   \toprule
             & \ac{n}.\ac{sg}  & \ac{m}.\ac{sg}      & \ac{f}.\ac{sg}    \\
         \cmidrule{2-4}
   \ac{nom}  & d-aȥ            & d-ër                & d-iu               \\
   \ac{acc}  & d-aȥ            & d-ën                & d-ea/d-ia         \\
   \ac{dat}  & d-ëmu/d-ëmo     & d-ëmu/d-ëmo         & d-ëru/d-ëro       \\
   \bottomrule
             & \ac{n}.\ac{pl}  & \ac{m}.\ac{pl}      &  \ac{f}.\ac{pl}  \\
         \cmidrule{2-4}
   \ac{nom}  & d-iu            &  d-ē/d-ea/d-ia/d-ie & d-eo/-io         \\
   \ac{acc}  & d-iu            &  d-ē/d-ea/d-ia/d-ie & d-eo/-io         \\
   \ac{dat}  & d-ēm/d-ēn       &  d-ēm/d-ēn          & d-ēm/d-ēn        \\
     \bottomrule
   \end{tabular}
   \label{tbl:rel-dem-ohg}
 \end{table}


 The suffixes that combine with the \tit{d} in demonstrative and relative pronouns also appear on adjectives. This is illustrated in Table \ref{tbl:adj-ohg}.

 \begin{table}[htbp]
  \center
  \caption {Adjectives on \tit{-a-/-ō-} in Old High German \pgcitealt{braune2018}{300}}
   \begin{tabular}{cccc}
   \toprule
             & \ac{n}.\ac{sg}    & \ac{m}.\ac{sg}      & \ac{f}.\ac{sg}    \\
     \cmidrule{2-4}
   \ac{nom}  & jung, jung-aȥ     & jung, jung-ēr       & jung, jung-iu     \\
   \ac{acc}  & jung, jung-aȥ     & jung-an             & jung-a            \\
   \ac{dat}  & jung-emu/jung-emo & jung-emu/jung-emo   & jung-eru/jung-ero \\
   \bottomrule
             & \ac{n}.\ac{pl}    & \ac{m}.\ac{pl}      &  \ac{f}.\ac{pl}   \\
       \cmidrule{2-4}
   \ac{nom}  & jung-iu           &  jung-e             & jung-o            \\
   \ac{acc}  & jung-iu           &  jung-e             & jung-o            \\
   \ac{dat}  & jung-ēm/jung-ēn   &  jung-ēm/jung-ēn    & jung-ēm/jung-ēn   \\
     \bottomrule
   \end{tabular}
   \label{tbl:adj-ohg}
 \end{table}

 I conclude from this that the suffix expresses features that are specific to being nominal, like number, gender and case. Not part of the suffix are features that are specific to being a demonstrative or relative pronoun, like anaphoricity and definiteness. I assume that these are expressed by the morpheme \tit{d}.

split the suffix up in two morphemes


In this section, I only discuss two forms: the nominative and accusative masculine singular relative and demonstrative pronoun. The nominative is \tit{dër} and the accusative is \tit{dën}. In what follows, I discuss the feature content of the morphemes \tit{d}, \tit{ër} and \tit{ën}. I start with the features that are expressed by the suffixes \tit{ër} and \tit{ën}.

This allows me to propose the following lexical entries for the two suffixes.



The \tit{d} morpheme corresponds to definiteness and anaphoricity. Anaphoricity establishes a relation with another element in the (linguistic) discourse. Definiteness encodes that the referent is specific.

\ex.
\begin{forest} boom
  [\tsc{d}P
      [\tsc{d}]
      [\tsc{ana}]
  ]
  {\draw (.east) node[right]{⇔ \tit{d}}; }
\end{forest}
\label{ex:ohg-d-lexicon}

So, the two relative pronouns look like this.\footnote{A question that arises here is how the case features can form a constituent to the exclusion of definiteness and anaphoricity. I come back to this issue in Chapter \ref{ch:discussion}.}







\subsection{Recomposing light heads and relative pronouns}

\ex. Old High German: \tsc{ext} \tsc{acc}\\
\tiny{
\begin{forest} boompje
  [\tsc{d}P, s sep=15mm
      [\tsc{d}P,
      tikz={
      \node[label=below:\tit{d},
      draw,circle,
      scale=0.8,
      fit to=tree]{};
      }
          [\tsc{d}, roof]
      ]
      [\tsc{acc}P, s sep=20mm
          [\tsc{med}P,
          tikz={
          \node[label=below:\tit{e},
          draw,circle,
          scale=0.85,
          fit to=tree]{};
          }
              [\tsc{dx}\scsub{2}]
              [\tsc{prox}P
                  [\tsc{dx}\scsub{1}]
                  [\tsc{ref} [\phantom{xxx}, roof]]
              ]
          ]
          [\tsc{acc}P,
          tikz={
          \node[label=below:\tit{n},
          draw,circle,
          scale=0.95,
          fit to=tree]{};
          }
              [\tsc{f}2]
              [\tsc{nom}P
                  [\tsc{f}1]
                  [\tsc{ind}P
                      [\tsc{ind}]
                      [\tsc{masc}P
                          [\tsc{masc}]
                          [\tsc{class}P
                              [\tsc{class}]
                          ]
                      ]
                  ]
              ]
          ]
      ]
  ]
\end{forest}
}

\ex. Old High German: \tsc{ext} \tsc{nom}\\
\tiny{
\begin{forest} boompje
  [\tsc{d}P, s sep=15mm
      [\tsc{d}P,
      tikz={
      \node[label=below:\tit{d},
      draw,circle,
      scale=0.8,
      fit to=tree]{};
      }
          [\tsc{d}, roof]
      ]
      [\tsc{nom}P, s sep=20mm
          [\tsc{med}P,
          tikz={
          \node[label=below:\tit{e},
          draw,circle,
          scale=0.85,
          fit to=tree]{};
          }
              [\tsc{dx}\scsub{2}]
              [\tsc{prox}P
                  [\tsc{dx}\scsub{1}]
                  [\tsc{ref} [\phantom{xxx}, roof]]
              ]
          ]
          [\tsc{nom}P,
          tikz={
          \node[label=below:\tit{r},
          draw,circle,
          scale=0.95,
          fit to=tree]{};
          }
              [\tsc{f}1]
              [\tsc{ind}P
                  [\tsc{ind}]
                  [\tsc{masc}P
                      [\tsc{masc}]
                      [\tsc{class}P
                          [\tsc{class}]
                      ]
                  ]
              ]
          ]
      ]
  ]
\end{forest}
}

\ex. Old High German: \tsc{int} \tsc{acc}\\
\tiny{
\begin{forest} boompje
  [\tsc{rel}P, s sep=10mm
      [\tsc{rel}P,
      tikz={
      \node[label=below:\tit{d},
      draw,circle,
      scale=0.95,
      fit to=tree]{};
      }
          [\tsc{rel}]
          [\tsc{d}P
              [\tsc{d}, roof]
          ]
      ]
      [\tsc{acc}P, s sep=20mm
          [\tsc{med}P,
          tikz={
          \node[label=below:\tit{e},
          draw,circle,
          scale=0.85,
          fit to=tree]{};
          }
              [\tsc{dx}\scsub{2}]
              [\tsc{prox}P
                  [\tsc{dx}\scsub{1}]
                  [\tsc{ref} [\phantom{xxx}, roof]]
              ]
          ]
          [\tsc{acc}P,
          tikz={
          \node[label=below:\tit{n},
          draw,circle,
          scale=0.95,
          fit to=tree]{};
          }
              [\tsc{f}2]
              [\tsc{nom}P
                  [\tsc{f}1]
                  [\tsc{ind}P
                      [\tsc{ind}]
                      [\tsc{masc}P
                          [\tsc{masc}]
                          [\tsc{class}P
                              [\tsc{class}]
                          ]
                      ]
                  ]
              ]
          ]
      ]
  ]
\end{forest}
}

\ex. Old High German: \tsc{int} \tsc{nom}\\
\tiny{
\begin{forest} boompje
  [\tsc{rel}P, s sep=10mm
      [\tsc{rel}P,
      tikz={
      \node[label=below:\tit{d},
      draw,circle,
      scale=0.95,
      fit to=tree]{};
      }
          [\tsc{rel}]
          [\tsc{d}P
              [\tsc{d}, roof]
          ]
      ]
      [\tsc{nom}P, s sep=20mm
          [\tsc{med}P,
          tikz={
          \node[label=below:\tit{e},
          draw,circle,
          scale=0.85,
          fit to=tree]{};
          }
              [\tsc{dx}\scsub{2}]
              [\tsc{prox}P
                  [\tsc{dx}\scsub{1}]
                  [\tsc{ref} [\phantom{xxx}, roof]]
              ]
          ]
          [\tsc{nom}P,
          tikz={
          \node[label=below:\tit{r},
          draw,circle,
          scale=0.95,
          fit to=tree]{};
          }
              [\tsc{f}1]
              [\tsc{ind}P
                  [\tsc{ind}]
                  [\tsc{masc}P
                      [\tsc{masc}]
                      [\tsc{class}P
                          [\tsc{class}]
                      ]
                  ]
              ]
          ]
      ]
  ]
\end{forest}
}






\subsection{Comparing constituents}

Consider the examples in \ref{ex:ohg-nom-nom-rep}, in which the internal nominative case competes against the external nominative case. The relative clauses are marked in bold, and the light heads and the relative pronouns are underlined. As the light head and the relative pronoun are identical it is impossible to see which of them surfaces.
The internal case is nominative, as the predicate \tit{senten} `to send' takes nominative subjects. The relative pronoun \tit{dher} `\ac{rel}.\ac{sg}.\ac{m}.\ac{nom}' appears in the nominative case.
The external case is nominative as well, as the predicate \tit{queman} `to come' also takes nominative subjects. The light head \tit{dher} `\ac{dem}.\ac{sg}.\ac{m}.\ac{nom}' appears in the nominative case.
\ref{ex:ohg-nom-nom-rep-rel} is the variant of the sentence in which the light head is absent (indicated by the square brackets) and the relative pronoun surfaces.
\ref{ex:ohg-nom-nom-rep-lh} is the variant of the sentence in which the relative pronoun is absent (indicated by the square brackets) and the light head surfaces.

\ex.\label{ex:ohg-nom-nom-rep}
\ag. quham \underline{[dher]} \tbf{dher} \underline{\tbf{chisendit}} \tbf{scolda} \tbf{uuerdhan}\\
 come.\ac{pst}.3\ac{sg}\scsub{[nom]} \ac{dem}.\ac{sg}.\ac{m}.\ac{nom} \ac{rel}.\ac{sg}.\ac{m}.\ac{nom} send.\ac{pst}.\ac{ptcp}\scsub{[nom]} should.\ac{pst}.3\ac{sg} become.\ac{inf}\\
 `the one, who should have been sent, came' \flushfill{Old High German, \ac{isid} 35:5}\label{ex:ohg-nom-nom-rep-rel}
\bg. quham \underline{dher} [\tbf{dher}] \underline{\tbf{chisendit}} \tbf{scolda} \tbf{uuerdhan}\\
 come.\ac{pst}.3\ac{sg}\scsub{[nom]} \ac{dem}.\ac{sg}.\ac{m}.\ac{nom} \ac{rel}.\ac{sg}.\ac{m}.\ac{nom} send.\ac{pst}.\ac{ptcp}\scsub{[nom]} should.\ac{pst}.3\ac{sg} become.\ac{inf}\\
 `the one, who should have been sent, came' \flushfill{Old High German, \ac{isid} 35:5}\label{ex:ohg-nom-nom-rep-lh}

In Figure \ref{fig:ohg-int=ext}, I give the syntactic structure of the light head at the top and the syntactic structure of the relative pronoun at the bottom.

\begin{figure}[htbp]
  \center
  \begin{tabular}[b]{c}
        \toprule
        \tsc{nom} light head \tit{dh-e-r}\\
        \cmidrule{1-1}
        \tiny{
        \begin{forest} boompje
          [\tsc{d}P, s sep=20mm
              [\tsc{d}P,
              tikz={
              \node[label=below:\tit{dh},
              draw,circle,
              scale=0.8,
              fit to=tree]{};
              \node[draw,circle,
              dashed,
              fill=DG,fill opacity=0.2,
              scale=0.9,
              fit to=tree]{};
              }
                  [\tsc{d}, roof]
              ]
              [\tsc{nom}P, s sep=25mm
                  [\tsc{med}P,
                  tikz={
                  \node[label=below:\tit{e},
                  draw,circle,
                  scale=0.85,
                  fit to=tree]{};
                  \node[draw,circle,
                  dashed,
                  fill=DG,fill opacity=0.2,
                  scale=0.9,
                  fit to=tree]{};
                  }
                      [\tsc{dx}\scsub{2}]
                      [\tsc{prox}P
                          [\tsc{dx}\scsub{1}]
                          [\tsc{ref} [\phantom{xxx}, roof]]
                      ]
                  ]
                  [\tsc{nom}P,
                  tikz={
                  \node[label=below:\tit{r},
                  draw,circle,
                  scale=0.95,
                  fit to=tree]{};
                  \node[draw,circle,
                  dashed,
                  scale=1,
                  fill=DG,fill opacity=0.2,
                  fit to=tree]{};
                  }
                      [\tsc{f}1]
                      [\tsc{ind}P
                          [\tsc{ind}]
                          [\tsc{masc}P
                              [\tsc{masc}]
                              [\tsc{class}P
                                  [\tsc{class}]
                              ]
                          ]
                      ]
                  ]
              ]
          ]
        \end{forest}
        }
      \\
      \toprule
      \tsc{nom} relative pronoun \tit{dh-e-r}
      \\
      \cmidrule{1-1}
      \tiny{
      \begin{forest} boompje
        [\tsc{rel}P, s sep=15mm
            [\tsc{rel}P,
            tikz={
            \node[label=below:\tit{dh},
            draw,circle,
            scale=0.95,
            fit to=tree]{};
            }
                [\tsc{rel}]
                [\tsc{d}P,
                tikz={
                \node[draw,circle,
                dashed,
                scale=0.8,
                fit to=tree]{};
                }
                    [\tsc{d}, roof]
                ]
            ]
            [\tsc{nom}P, s sep=25mm
                [\tsc{med}P,
                tikz={
                \node[label=below:\tit{e},
                draw,circle,
                scale=0.85,
                fit to=tree]{};
                \node[draw,circle,
                dashed,
                scale=0.9,
                fit to=tree]{};
                }
                    [\tsc{dx}\scsub{2}]
                    [\tsc{prox}P
                        [\tsc{dx}\scsub{1}]
                        [\tsc{ref} [\phantom{xxx}, roof]]
                    ]
                ]
                [\tsc{nom}P,
                tikz={
                \node[label=below:\tit{r},
                draw,circle,
                scale=0.95,
                fit to=tree]{};
                \node[draw,circle,
                dashed,
                scale=1,
                fit to=tree]{};
                }
                    [\tsc{f}1]
                    [\tsc{ind}P
                        [\tsc{ind}]
                        [\tsc{masc}P
                            [\tsc{masc}]
                            [\tsc{class}P
                                [\tsc{class}]
                            ]
                        ]
                    ]
                ]
            ]
        ]
      \end{forest}
      }
        \\
      \bottomrule
  \end{tabular}
  \caption {Old High German \tsc{ext}\scsub{nom} vs. \tsc{int}\scsub{nom} → \tit{dher}}
  \label{fig:ohg-int=ext}
\end{figure}

The relative pronoun consists of three morphemes: \tit{dh}, \tit{e} and \tit{r}.
The light head consists of three morphemes: \tit{dh}, \tit{e} and \tit{r}.
As usual, I circle the part of the structure that corresponds to a particular lexical entry, and I place the corresponding phonology under it.
I draw a dashed circle around each constituent that is a constituent in both the light head and the relative pronoun.
As each constituent of the light head is also a constituent within the relative pronoun, the light head can be absent. I illustrate this by marking the content of the dashed circles for the light head gray.

I explain this constituent by constituent.
I start with the right-most constituent of the light head that spells out as \tit{r} (\tsc{nom}P). This constituent is also a constituent in the relative pronoun.
I continue with the middle constituent of the light head that spells out as \tit{e} (\tsc{med}P). This constituent is also a constituent in the relative pronoun.
I end with the left-most constituent of the light head that spells out as \tit{d} {\tsc{d}P}. This constituent is also a constituent in the relative pronoun, contained in \tsc{rel}P.
All three constituent of the light head are also a constituent within the relative pronoun, and the light head can be absent.

Consider the example in \ref{ex:ohg-nom-acc-rep}, in which the internal accusative case competes against the external nominative case. The relative clause is marked in bold, and the light head and the relative pronoun are underlined.
The internal case is accusative, as the predicate \tit{zellen} `to tell' takes accusative objects. The relative pronoun \tit{then} `\ac{rel}.\ac{sg}.\ac{m}.\ac{acc}' appears in the accusative case. This is the element that surfaces.
The external case is nominative, as the predicate \tit{sin} `to be' takes nominative objects. The light head \tit{ther} `\ac{dem}.\ac{sg}.\ac{m}.\ac{nom}' appears in the nominative case. It is placed between square brackets because it does not surface.

\exg. Thíz ist [\underline{ther}] \underline[\tbf{then}] \tbf{sie} \tbf{zéllent}\\
\ac{dem}.\ac{sg}.\ac{n}.\ac{nom} be.\ac{pres}.3\ac{sg}\scsub{[nom]} \ac{dem}.\ac{sg}.\ac{m}.\ac{nom} \ac{rel}.\ac{sg}.\ac{m}.\ac{acc} 3\ac{pl}.\ac{m}.\ac{nom} tell.\ac{pres}.3\ac{pl}\scsub{[acc]}\\
`this is the one whom they talk about' \flushfill{Old High German, \ac{otfrid} III 16:50}\label{ex:ohg-nom-acc-rep}

In Figure \ref{fig:ohg-int-wins}, I give the syntactic structure of the light head at the top and the syntactic structure of the relative pronoun at the bottom.

\begin{figure}[htbp]
  \center
  \begin{tabular}[b]{c}
      \toprule
      \tsc{nom} light head \tit{th-e-r}
      \\
      \cmidrule{1-1}
      \tiny{
      \begin{forest} boompje
        [\tsc{d}P, s sep=20mm
            [\tsc{d}P,
            tikz={
            \node[label=below:\tit{th},
            draw,circle,
            scale=0.8,
            fit to=tree]{};
            \node[draw,circle,
            dashed,
            fill=DG,fill opacity=0.2,
            scale=0.9,
            fit to=tree]{};
            }
                [\tsc{d}, roof]
            ]
            [\tsc{nom}P, s sep=25mm
                [\tsc{med}P,
                tikz={
                \node[label=below:\tit{e},
                draw,circle,
                scale=0.85,
                fit to=tree]{};
                \node[draw,circle,
                dashed,
                fill=DG,fill opacity=0.2,
                scale=0.9,
                fit to=tree]{};
                }
                    [\tsc{dx}\scsub{2}]
                    [\tsc{prox}P
                        [\tsc{dx}\scsub{1}]
                        [\tsc{ref} [\phantom{xxx}, roof]]
                    ]
                ]
                [\tsc{nom}P,
                tikz={
                \node[label=below:\tit{r},
                draw,circle,
                scale=0.95,
                fit to=tree]{};
                \node[draw,circle,
                dashed,
                fill=DG,fill opacity=0.2,
                scale=1,
                fit to=tree]{};
                }
                    [\tsc{f}1]
                    [\tsc{ind}P
                        [\tsc{ind}]
                        [\tsc{masc}P
                            [\tsc{masc}]
                            [\tsc{class}P
                                [\tsc{class}]
                            ]
                        ]
                    ]
                ]
            ]
        ]
      \end{forest}
      }
      \\
      \toprule
      \tsc{acc} relative pronoun \tit{th-e-n}
      \\
      \cmidrule{1-1}
      \tiny{
          \begin{forest} boompje
            [\tsc{rel}P, s sep=15mm
                [\tsc{rel}P,
                tikz={
                \node[label=below:\tit{th},
                draw,circle,
                scale=0.95,
                fit to=tree]{};
                }
                    [\tsc{rel}]
                    [\tsc{d}P,
                    tikz={
                    \node[draw,circle,
                    dashed,
                    scale=0.8,
                    fit to=tree]{};
                    }
                        [\tsc{d}, roof]
                    ]
                ]
              [\tsc{acc}P, s sep=25mm
                  [\tsc{med}P,
                  tikz={
                  \node[label=below:\tit{e},
                  draw,circle,
                  scale=0.85,
                  fit to=tree]{};
                  \node[draw,circle,
                  dashed,
                  scale=0.9,
                  fit to=tree]{};
                  }
                      [\tsc{dx}\scsub{2}]
                      [\tsc{prox}P
                          [\tsc{dx}\scsub{1}]
                          [\tsc{ref} [\phantom{xxx}, roof]]
                      ]
                  ]
                  [\tsc{acc}P,
                  tikz={
                  \node[label=below:\tit{n},
                  draw,circle,
                  scale=0.95,
                  fit to=tree]{};
                  }
                      [\tsc{f}2]
                      [\tsc{nom}P,
                      tikz={
                      \node[draw,circle,
                      dashed,
                      scale=0.9,
                      fit to=tree]{};
                      }
                          [\tsc{f}1]
                          [\tsc{ind}P
                              [\tsc{ind}]
                              [\tsc{masc}P
                                  [\tsc{masc}]
                                  [\tsc{class}P
                                      [\tsc{class}]
                                  ]
                              ]
                          ]
                      ]
                  ]
              ]
          ]
        \end{forest}
        }
        \\
      \bottomrule
  \end{tabular}
 \caption {Old High German \tsc{ext}\scsub{nom} vs. \tsc{int}\scsub{acc} → \tit{then}}
  \label{fig:ohg-int-wins}
\end{figure}

The relative pronoun consists of three morphemes: \tit{th}, \tit{e} and \tit{n}.
The light head consists of three morphemes: \tit{th}, \tit{e} and \tit{r}.
Again, I circle the part of the structure that corresponds to a particular lexical entry, and I place the corresponding phonology under it.
I draw a dashed circle around each constituent that is a constituent in both the light head and the relative pronoun.
As each constituent of the light head is also a constituent within the relative pronoun, the light head can be absent. I illustrate this by marking the content of the dashed circles for the light head gray.

I explain this constituent by constituent.
I start with the right-most constituent of the light head that spells out as \tit{r} (\tsc{nom}P). This constituent is also a constituent in the relative pronoun, contained in \tsc{acc}P.
I continue with the middle constituent of the light head that spells out as \tit{e} (\tsc{med}P). This constituent is also a constituent in the relative pronoun.
I end with the left-most constituent of the light head that spells out as \tit{d} {\tsc{d}P}. This constituent is also a constituent in the relative pronoun, contained in \tsc{rel}P.
All three constituent of the light head are also a constituent within the relative pronoun, and the light head can be absent.

Consider the examples in \ref{ex:ohg-acc-nom-rep}, in which the internal nominative case competes against the external accusative case. The relative clauses are marked in bold, and the light heads and the relative pronouns are underlined.
The internal case is nominative, as the predicate \tit{gisizzen} `to possess' takes nominative subjects. The relative pronoun \tit{dher} `\ac{rel}.\ac{sg}.\ac{m}.\ac{nom}' appears in the nominative case. It is placed between square brackets because it does not surface.
The external case is accusative, as the predicate \tit{bibringan} `to create' takes accusative objects. The light head \tit{dhen} `\ac{dem}.\ac{sg}.\ac{m}.\ac{acc}' appears in the accusative case. This is the element that surfaces.

\exg. ih bibringu fona iacobes samin endi fona iuda \underline{dhen} [\underline{\tbf{dher}}] \tbf{mina} \tbf{berga} \tbf{chisitzit}\\
1\ac{sg}.\ac{nom} {create}.\ac{pres}.1\ac{sg}\scsub{[acc]} of Jakob.\ac{gen} seed.\ac{sg}.\ac{dat} and of Judah.\ac{dat} \ac{rel}.\ac{sg}.\ac{m}.\ac{acc} my.\ac{acc}.\ac{m}.\ac{pl} mountain.\ac{acc}.\ac{pl} possess.\ac{pres}.3\ac{sg}\scsub{[nom]}\\
`I create of the seed of Jacob and of Judah the one, who possess my mountains' \flushfill{Old High German, \ac{isid} 34:3}\label{ex:ohg-acc-nom-rep}

In Figure \ref{fig:ohg-ext-wins}, I give the syntactic structure of the light head at the top and the syntactic structure of the relative pronoun at the bottom.

\begin{figure}[htbp]
  \center
  \begin{tabular}[b]{c}
        \toprule
        \tsc{acc} light head \tit{dh-e-n} \\
        \cmidrule{1-1}
        \tiny{
        \begin{forest} boompje
          [\tsc{d}P, s sep=20mm
              [\tsc{d}P,
              tikz={
              \node[label=below:\tit{dh},
              draw,circle,
              scale=0.8,
              fit to=tree]{};
              \node[draw,circle,
              dashed,
              scale=0.9,
              fit to=tree]{};
              }
                  [\tsc{d}, roof]
              ]
              [\tsc{acc}P, s sep=25mm
                  [\tsc{med}P,
                  tikz={
                  \node[label=below:\tit{e},
                  draw,circle,
                  scale=0.85,
                  fit to=tree]{};
                  \node[draw,circle,
                  dashed,
                  scale=0.9,
                  fit to=tree]{};
                  }
                      [\tsc{dx}\scsub{2}]
                      [\tsc{prox}P
                          [\tsc{dx}\scsub{1}]
                          [\tsc{ref} [\phantom{xxx}, roof]]
                      ]
                  ]
                  [\tsc{acc}P,
                  tikz={
                  \node[label=below:\tit{n},
                  draw,circle,
                  scale=0.95,
                  fit to=tree]{};
                  }
                      [\tsc{f}2]
                      [\tsc{nom}P,
                      tikz={
                      \node[draw,circle,
                      dashed,
                      scale=0.9,
                      fit to=tree]{};
                      }
                          [\tsc{f}1]
                          [\tsc{ind}P
                              [\tsc{ind}]
                              [\tsc{masc}P
                                  [\tsc{masc}]
                                  [\tsc{class}P
                                      [\tsc{class}]
                                  ]
                              ]
                          ]
                      ]
                  ]
              ]
          ]
        \end{forest}
        }
        \\
        \toprule
        \tsc{nom} relative pronoun \tit{dh-e-r}
        \\
        \cmidrule{1-1}
        \tiny{
            \begin{forest} boompje
              [\tsc{rel}P, s sep=15mm
                  [\tsc{rel}P,
                  tikz={
                  \node[label=below:\tit{dh},
                  draw,circle,
                  scale=0.95,
                  fill=DG,fill opacity=0.1,
                  fit to=tree]{};
                  }
                      [\tsc{rel}]
                      [\tsc{d}P,
                      tikz={
                      \node[draw,circle,
                      dashed,
                      fill=DG,fill opacity=0.2,
                      scale=0.8,
                      fit to=tree]{};
                      }
                          [\tsc{d}, roof]
                      ]
                  ]
                  [\tsc{nom}P, s sep=25mm
                      [\tsc{med}P,
                      tikz={
                      \node[label=below:\tit{e},
                      draw,circle,
                      scale=0.85,
                      fit to=tree]{};
                      \node[draw,circle,
                      dashed,
                      scale=0.9,
                      fill=DG,fill opacity=0.2,
                      fit to=tree]{};
                      }
                          [\tsc{dx}\scsub{2}]
                          [\tsc{prox}P
                              [\tsc{dx}\scsub{1}]
                              [\tsc{ref} [\phantom{xxx}, roof]]
                          ]
                      ]
                      [\tsc{nom}P,
                      tikz={
                      \node[label=below:\tit{r},
                      draw,circle,
                      scale=0.95,
                      fit to=tree]{};
                      \node[draw,circle,
                      dashed,
                      fill=DG,fill opacity=0.2,
                      scale=1,
                      fit to=tree]{};
                      }
                          [\tsc{f}1]
                          [\tsc{ind}P
                              [\tsc{ind}]
                              [\tsc{masc}P
                                  [\tsc{masc}]
                                  [\tsc{class}P
                                      [\tsc{class}]
                                  ]
                              ]
                          ]
                      ]
                  ]
              ]
          \end{forest}
          }
          \\
      \bottomrule
  \end{tabular}
   \caption {Old High German \tsc{ext}\scsub{acc} vs. \tsc{int}\scsub{nom} → \tit{dhen}}
  \label{fig:ohg-ext-wins}
\end{figure}

The relative pronoun consists of three morphemes: \tit{dh}, \tit{e} and \tit{r}.
The light head consists of three morphemes: \tit{dh}, \tit{e} and \tit{n}.
Again, I circle the part of the structure that corresponds to a particular lexical entry, and I place the corresponding phonology under it.
I draw a dashed circle around each constituent that is a constituent in both the light head and the relative pronoun.
As each constituent of the light head is also a constituent within the relative pronoun or is syncretic with one, the relative pronoun can be absent. I illustrate this by marking the content of the dashed circles for the relative pronoun gray.

I explain this constituent by constituent.
I start with the right-most constituent of the relative pronoun head that spells out as \tit{r} (\tsc{nom}P). This constituent is also a constituent in the light head, contained in \tsc{acc}P.
I continue with the middle constituent of the relative pronoun that spells out as \tit{e} (\tsc{med}P). This constituent is also a constituent in the light head.
I end with the left-most constituent of the relative pronoun that spells out as \tit{d} {\tsc{rel}P}. This consituent is not contained in the light head, but it is syncretic with it. The \tsc{d}P is also spelled out as \tit{d}.
All three constituent of the light head are also a constituent within the relative pronoun or are syncretic with them, and the relative pronoun can be absent.







\section{Technical details}

Modern German:

I start at the beginning, with the \tsc{ref}, merging it with \tsc{dx}\scsub{1}, giving a \tit{e}.

So if then \tsc{dx}\scsub{2} is merged, it is overwritten by \tit{e}.

I move forward a bit to when \tsc{wh} is merged. First the spellout driven movement happen, but this does not bring anything. Also backtracking does not help, so we build a spec.

Feature \tsc{rel} is merged. First try to merge it on the whole tree, then the spellout driven movements, nothing works. So, backtracking. The first step of backtracking is that the two trees are split, and the feature is merged on both parts. If the feature is spelled out on one of them, we are done. It can be phrasally spelled out with \tsc{wh}, so we move on.

Then feature \tsc{f}1 is merged. Whole tree, spellout driven movement: yes! it is spelled out as a suffix on the whole thing.



\section{Summary}

Table \ref{tbl:overview-rel-light} shows per language type which of the three options in Table \ref{tbl:options-surface-pronoun} is chosen when the internal and external case differ.

\begin{table}[htbp]
  \center
  \caption{The surface pronoun with differing cases per language}
\begin{tabular}{cccc}
  \toprule
                & \tsc{k}\scsub{int} > \tsc{k}\scsub{ext} & \tsc{k}\scsub{ext} > \tsc{k}\scsub{int} &                  \\
                \cmidrule{2-3}
unrestricted    & relative pronoun\scsub{int}  & light head\scsub{ext} & Old High German  \\
internal-only   & relative pronoun\scsub{int}  & *                     & Modern German    \\
matching        & *                            & *                     & Polish           \\
external-only   & *                            & light head\scsub{ext} & not attested     \\
\bottomrule
\end{tabular}
\label{tbl:overview-rel-light}
\end{table}

The first column list the types of languages.
The second column shows the situation in which the internal case is the most complex. The relative pronoun that bears the internal case is the potential surface pronoun.
The third column shows the situation in which the external case is the most complex. The light head that bears the external case is the potential surface pronoun.
The asterix (*) indicates that there is no grammatical form for the surface pronoun.
The fourth column gives the example of the language type that I discuss in this chapter.
A language of the unrestricted type (like Old High German) allows both the internal case and the external case to surface when either of them wins the case competition. Either the light head with its external case or the relative pronoun with its internal case can be the surface pronoun.
A language of the internal-only type (like Modern German) allows only the internal case to surface when it wins the case competition, and it does not allow the external case to do so. The relative pronoun with its internal case can be the surface pronoun and the light head with its external case cannot.
A language of the matching type (like Polish) allows neither the internal nor the external case to surface when either of them wins the case competition. Neither the relative pronoun with its internal case nor the light head with its external case can be the surface pronoun.\footnote{
This holds for the situation in which the internal and external case differ. In Section \ref{sec:deriving-matching}, I show that the relative pronoun surfaces in matching contexts.
}
The language type that is not attested is the external-only type. That means that there is no language that allows only the external case to surface when it wins the case competition, and it does not allow the internal case to do so. In other words, there exist no language, in which the surface pronoun can only be the light head and not the relative pronoun.

What I have done in this section so far is reformulate the two descriptive parameters from Figure \ref{fig:two-parameters} into two other descriptive parameters.
Whether the internal case is allowed to surface corresponds to whether the relative pronoun surfaces. That implicates that the light head has been deleted and is therefore absent.
Similarly, whether the external case is allowed to surface corresponds to whether the light head surfaces. That implicates that the relative pronoun has been deleted and is therefore absent.
I show this in Figure \ref{fig:two-parameters-different}.

\begin{figure}[htbp]
  \centering
    \footnotesize{
    \begin{tikzpicture}[node distance=1.5cm]
      \node (question2) [question]
      {{delete relative pronoun?}};
          \node (outcome2) [outcome, below of=question2, xshift=-1.5cm]
          {matching};
              \node (example2) [example, below of=outcome2, yshift=0.25cm]
              {\scriptsize{e.g. Polish\\\phantom{x}}};
          \node (question3) [question, below of=question2, xshift=2cm, yshift=-0.5cm]
          {{delete light head?}};
              \node (outcome3) [outcome, below of=question3, xshift=-1.5cm]
              {internal-only};
                  \node (example3) [example, below of=outcome3, yshift=0.25cm]
                  {\scriptsize{e.g. Modern German\\\phantom{x}}};
              \node (outcome4) [outcome, below of=question3, xshift=1.5cm]
              {un-restricted};
                  \node (example4) [example, below of=outcome4, yshift=0.25cm]
                  {\scriptsize{e.g. Gothic, Old High German, Classical Greek}};

    \draw [arrow] (question2) -- node[anchor=east] {no} (outcome2);
    \draw [arrow] (question2) -- node[anchor=west] {yes} (question3);
    \draw [arrow] (question3) -- node[anchor=east] {no} (outcome3);
    \draw [arrow] (question3) -- node[anchor=west] {yes} (outcome4);
    \end{tikzpicture}
    }
    \caption{Delete relative pronoun/light head as parameters}
    \label{fig:two-parameters-different}
\end{figure}

Reformulating these parameters is not just restating the generalization in different terms. With this new formulation, I am able to identify the elements (i.e. the light head and the relative pronoun) that bear the internal and external cases. The difference between languages lies in whether or not it is possible to delete the light head (and with it the external case) and the relative pronoun (and with it the internal case).





\section{Aside: a larger syntactic context}\label{sec:larger-syntax}

If you talk about different patterns, there can be different locations to put your parameters. Himmelreich put her parameters in the structure. I put my parameters in the elements themselves. I show what an analysis like Himmelreich looks like, and I show then that it is difficult to reduce that then to differences in the lexicon (because it has to do with agree?).

So what I do is keep the parameters that she was differing stable. I change the things that she kept constant, the internal and external element. Does her structure then work with what I want? Not entirely, because I have to do a c-command that is going in the wrong direction.
Then I show a syntactic structure that could be compatible with mine, and I show why a grafting one is not.



In this dissertation I focus on when languages allow the internal and external case to win the case competition. In my proposal, this depends on the comparison between the internal and external base. The larger syntactic context in which this takes place should be kept stable. For concreteness, I show a possible implementation in Cinque's double-headed analysis of relative clause. I do by no means claim that claim this is the only or even correct implementation.



% In the previous section I introduced the relative pronoun as the internal element. This means that the other element is the external element. This section starts with the observation that there actually are languages in which two elements surface in so-called double-headed relative clauses. In these languages, the external head is a subset of the internal head, and that some features like \tsc{d} and case are necessarily excluded in the external head. I adopt this insight, and I apply it to the headless relative situation. I propose that the external head in headless relatives is a copy of a specific part of the relative pronoun.
%
% As I said earlier, I need two elements to do case competition with. In headless relatives, I only see a single one surfacing. However, some languages actually show two elements surfacing. Here there are two copies of the element, one inside the relative clause, one outside of the relative clause.
%
% \exg. [\tbf{doü} adiyan-o-no] \tbf{doü} deyalukhe\\
%  sago give.3\tsc{pl}.\tsc{nonfut}-{tr}-\tsc{conn} sago finished.\tsc{adj}\\
%  `The sago that they gave is finished.' \flushfill{Kombai, \pgcitealt{vries1993}{78}}
%
% The external element is not always an exact copy of the element inside of the relative clause. An example from Kombai shows that the element outside of the relative clause can also be a subset of what the element inside of the relative clause is. Here I give two examples, there is an \tit{old man} and a \tit{person}, and there is \tit{pig} and a \tit{thing}.
%
% \ex.
% \ag. [\tbf{yare} gamo khereja bogi-n-o] \tbf{rumu} na-momof-a\\
%  {old man} join.\ac{ss} work do.\ac{dur}.3\ac{sg}.\ac{nf}-\ac{tr}-\ac{conn} person my-uncle-\ac{pred}\\
%  `The old man, who is joining the work, is my uncle.' 77
% \bg. [\tbf{ai} fali-khano] \tbf{ro} nagu-n-ay-a.\\
%  pig carry-go.3\tsc{pl}.\tsc{nf} thing our-\tsc{tr}-pig-\ac{pred}\\
%  `The pig they took away, is ours.' \flushfill{Kombai, \pgcitealt{vries1993}{77}}
%
% Let me now apply what we have seen so far to headless relatives. Headless relatives do not have an overt NP, so this cannot be copied. However, there is the relative pronoun which is specified for number, gender, case, etc. Are all of these features copied onto the external element? The copy is the portion of the nominal extended projection c-commanded by the relative clause. A headless relative is a restrictive relative clause. Therefore, there is no \tsc{d} and no case.
%
% Is it possible to add features onto the external head after it has been copied? Yes, for example D, as the example shows, but also case.
%
% \exg. Junya-wa [Ayaka-ga \tbf{ringo}-o mui-ta] sono \tbf{ringo}-o tabe-ta.\\
% Junya-\ac{top} Ayaka-\ac{nom} apple-\ac{acc} peel-\ac{pst} that apple-\ac{acc} eat-\ac{pst}\\
% ‘Junya ate the apples that Ayaka peeled.’ \flushfill{Japanese, \pgcitealt{erlewine2016}{2}}
%
% In sum, the external element is a copy of a subset of the features of the relative pronoun. Definiteness and case are not copied. New features can be merged onto the external element.


According to Cinque, every type of relative clause in every language is underlyingly double-headed. Evidence for this claim comes from languages that show this morphologically. An example from Kombai is given in \ref{ex:kombai}. The head of the relative clause is \tit{doü} `sago', and it appears inside the relative clause and outside.

\exg. [\tbf{doü} adiyan-o-no] \tbf{doü} deyalukhe\\
 sago give.3\tsc{pl}.\tsc{nonfut}-{tr}-\tsc{conn} sago finished.\tsc{adj}\\
 `The sago that they gave is finished.' \flushfill{Kombai, \pgcitealt{vries1993}{78}}\label{ex:kombai}

The internal and external instances of \tit{doü} correspond to the internal and external element I assume to be there in the headless relatives.

\ref{ex:double-syntax} shows the syntactic structure of the sentence in \ref{ex:kombai}.

\ex.
\begin{forest} boom
[CP
   [FP
      [CP
          [\tsc{int}
             [\tit{doü}, roof]
          ]
          [CP
              [\tit{adiyan-o-no}, roof]
          ]
      ]
      [\tsc{ext}
         [\tit{doü}, roof]
      ]
   ]
   [VP
      [\tit{deyalukhe}, roof]
   ]
]
\end{forest}\label{ex:double-syntax}

In most languages one of the two heads is deleted throughout the derivation.

According to \citealt{cinqueforthcoming}, the internal element can delete the external element, because the internal element c-commands the external element. This is c-command according to Kayne's definition of it: the internal element is in the specifier of the specifier of the FP.

\ex.
\begin{forest} boom
[
   [CP
       [\tsc{int}
          [\phantom{xxx}, roof]
       ]
       [CP
           [\phantom{xxx}, roof]
       ]
   ]
   [\tsc{ext}
      [\phantom{xxx}, roof]
   ]
]
\end{forest}\label{ex:cinque-int-wins}

In order for the internal element to be able to delete the external element, a movement needs to take place. The external element moves over the relative clause.\footnote{
What remains unclear is what the trigger is for the movement of the external element over relative clause is.
}
From this position, the external element can delete the internal one, because the external element c-commands the internal one.

\ex.
\begin{forest} boom
[
    [\tsc{ext}
       [\phantom{xxx}, roof]
    ]
    [FP
       [CP
           [\tsc{int}
              [\phantom{xxx}, roof]
           ]
           [CP
               [\phantom{xxx}, roof]
           ]
       ]
       [\tit{t\scsub{ext}}]
    ]
]
\end{forest}

Also talk about \tsc{d} here, and that maybe Old High German deletes a thing without a \tsc{d} when the internal thing wins. does that also have a not so definite interpretation?


What does not work:

For this pattern a single element analysis seems intuitive, if you assume that case is complex and that syntax works bottom-up. First you built the relative clause, with the big case in there. Then you build the main clause and you let the more complex case in the embedded clause license the main clause predicate.

Consider the example in \ref{ex:mg-nom-acc-grafting}. Here the internal case is accusative and the external one nominative.

\exg. Uns besucht \tbf{wen} \tbf{Maria} \tbf{mag}.\\
 we.\ac{acc} visit.3\ac{sg}\scsub{[nom]} \tsc{rel}.\ac{acc}.\tsc{an} Maria.\ac{nom} like.3\ac{sg}\scsub{[acc]}\\
 `Who visits us, Maria likes.' \flushfill{adapted from \pgcitealt{vogel2001}{343}}\label{ex:mg-nom-acc-grafting}

The relative clause is built, including the accusative relative pronoun. Now the main clause predicate can merge with the nominative that is contained within the accusative.

 \ex.
 \begin{forest} boom
  [,name=src, s sep=15mm
   [VP
      [\tit{besucht}, roof]
   ]
    [,no edge, s sep=20mm
        [\ac{acc}P,
     tikz={
     \node[label=below:\tit{wen},
     draw,circle,
     scale=0.85,
     fit to=tree]{};
     }
            [\tsc{f2}]
            [\tsc{nomP},name=tgt
                [\tsc{f1}]
                [XP
                    [\phantom{xxx}, roof]
                ]
            ]
        ]
     [VP
        [\tit{Maria mag}, roof]
     ]
   ]
  ]
  \draw (src) to[out=south east,in=north east] (tgt);
 \end{forest}\label{ex:acc-nom-grafting}

The other way around does not work. Consider \ref{ex:mg-acc-nom-grafting}. This is an example with nominative as internal case and accusative as external case.

\exg. *Ich {lade ein}, wen \tbf{mir} \tbf{sympathisch} \tbf{ist}.\\
I.\ac{nom} invite.1\ac{sg}\scsub{[acc]} \tsc{rel}.\ac{acc}.\tsc{an} I.\ac{dat} nice be.3\ac{sg}\scsub{[nom]}\\
`I invite who I like.' \flushfill{adapted from \pgcitealt{vogel2001}{344}}\label{ex:mg-acc-nom-grafting}

Now the relative clause is built first again, this time only including the nominative case. There is no accusative node to merge with for the external predicate. Instead, the relative pronoun would need to grow to accusative somehow and then the merge could take place. This is the desired result, because the sentence is ungrammatical.

\ex.
\begin{forest} boom
  [,name=src, s sep=15mm
     [VP
         [\tit{lade ein}, roof]
     ]
         [,no edge
       [\tsc{nomP},
       tikz={
       \node[label=below:\tit{wer},
       draw,circle,
       scale=0.85,
       fit to=tree]{};
       }
         [\tsc{f1}]
         [XP
           [\phantom{xxx}, roof]
         ]
       ]
       [VP
         [\tit{mir sympatisch ist}, roof]
       ]
      ]
    ]
\end{forest}\label{ex:nom-acc-grafting}

So, this seems to work fine. The assumptions you have to do in order to make this are the following. First, case is complex. Second, you can remerge an embedded node (grafting). For the first one I have argued in Chapter \ref{ch:decomposition}. The second one could use some additional argumentation. It is a mix between internal remerge (move) and external merge, namely external remerge. Other literature on multidominance and grafting, other phenomena. Problems: linearization, .. But even if fix all these theoretical problems, there is an empirical one.

That is, I want to connect this behavior of Modern German headless relatives to the shape of its relative pronouns. These pronouns are \tsc{wh}-elements. The OHG and Gothic ones are not \tsc{wh}, they are \tsc{d}. Their relative pronouns look different, and so their headless relatives can also behave differently.




Himmelreich

there are agree relations between
- V\scsub{ext} and \tsc{ext}
- V\scsub{int} and \tsc{int}
- \tsc{int} and \tsc{ext}

three parameters:
1 relation between V\scsub{ext} and \tsc{ext} + V\scsub{int} and \tsc{int} are symmetric or asymmetric
2 relation between \tsc{ext} and \tsc{int} are symmetric or asymmetric
3 if \tsc{ext} --- \tsc{int} is asymmetric, \tsc{ext} or \tsc{int} probes

I keep the parameters she has stable, the bigger syntactic context is the same everywhere. I vary the content of \tsc{ext}
