% !TEX root = thesis.tex

\chapter{The committee}\label{ch:relativization}

In the previous chapter I showed that languages with case competition come in two variants. First, there are languages that allow the internal and external case to surface when they win the competition, such as Gothic and Old High German. Second, there are languages that only allow the internal case to surface when it wins the competition, such as Modern German. Crucially, there is no language that only allows the external case to surface when it wins the competition.

The aim of this chapter is twofold. Third, I discuss how the difference between internal-and-external languages and internal-only languages can be derived. I introduce both matters a bit more in this introduction. Second, I discuss how the non-existence of the external-only pattern can be explained.

Let me now turn to the matter of crosslinguistic differences. Every speaker of a language needs to learn what the pattern for its language is. Headless relatives are infrequent, is what can be said about at least Modern German. Even though not everybody likes the construction to begin with (they prefer (light-)headed relatives), people seem to have the clear intuition that \tsc{int}>\tsc{ext} is much better than the other way around. It seems implausible that learners of German learn this pattern from the few examples they got (there are just too few to make a generalization). Still, the intuition exist. And it is very particular: more complex case wins over less complex case, but only if the internal case is more complex than the external case. This already sounds hard to learn from the input as a generalization. People have also been describing it like this: formulation from Cinque in his book. If it does not come from the input, where does it come from? I claim that it comes from other properties of the language. In Grosu's terminology: is it derived or basic? Ideally, we would want it to be derived.

A similar avenue was pursued by \citealt{himmelreich2017}. She specific languages for having different types of agree (up, down) and different types of probes (active, non-active). Doing that, she successfully derived free relatives and parasitic gaps in different languages. Grosu 1994 linked richness of inflection to liberality. He actually talked about the richness of pro.

The crucial difference with I'm doing is that I'm not relying on an arbitrary value I assigned to a language (say null head is active probe, probing only happens upwards). Like I briefly mentioned in Chapter \ref{ch:decomposition}, Nanosyntax models crosslinguistic variation as differences in the lexicon, how the features are packaged together differently. That means that I look for patterns within the languages themselves, and let the facts of the headless relatives follow from those. Specifically, I derive the different behaviors from relative pronouns and the external head that I introduce in this chapter.

In Section \ref{sec:internal-wins} I discuss the situation in which the internal case wins the competition and the relative pronoun surfaces in the internal case. This situation is attested in all languages with case competition: the internal-only ones, such as Modern German, and the internal-and-external ones, such as Gothic and Old High German.

In Section \ref{sec:external-wins} I discuss the situation in which the external case wins the case competition and the relative pronoun surfaces in the external case. This situation is not attested in all of the case competition languages. This situation arises if the external head contains all case features of the internal head. In addition, and this is what distinguishes the internal-and-external languages from the internal-only languages, the features of the relative pronoun are a subset of the features of the external head. So, \tit{em} cannot delete \tit{wen}, but \tit{dem} can delete \tit{den}.

From these two proposals follows that it is impossible to have third pattern, which is indeed also not attested. When the external case wins over the internal one, there is a situation in which the external case could delete the internal one. So, it is impossible to have the second option but not the first one.


\section{The idea}

what criteria does the committee use? ideally, something independently observable within the language.

if int contains ext, int is approved
if ext contains int, ext is approved


\section{The internal and external element}

what are they? we see only one: the relative pronoun. however, we really need to elements! because we need to compare internal and external

you can see this very very well in modern german: it does not suffice to let the relative pronoun surfaces in the internal case. we always need a way to refer to the more complex case.



The relative pronoun in Modern German headless relatives is sensitive to both the internal and the external case. Consider the examples in \ref{ex:mg-internal}. In both sentences, the internal case is accusative, because the predicate in the relative clause \tit{mögen} `to like' takes accusative objects. The external case differs between the two sentences. In \ref{ex:mg-dat-acc-wen} the external case is dative, because the predicate \tit{vertrauen} `to trust' takes dative objects.  In \ref{ex:mg-int}, the external case is nominative, because \tit{besuchen} `to visit' takes nominative subjects.

\ex.\label{ex:mg-internal}
\ag. *Ich vertraue wen \tbf{auch} \tbf{Maria} \tbf{mag}. \\
I.\ac{nom} trust.1\ac{sg}\scsub{[dat]} \tsc{rel}.\ac{acc}.\tsc{an} also Maria.\ac{nom} like.3\ac{sg}\scsub{[acc]}.\\
`I trust whoever Maria also likes.' \flushfill{adapted from \pgcitealt{vogel2001}{345}}\label{ex:mg-dat-acc-wen}
\bg. Uns besucht \tbf{wen} \tbf{Maria} \tbf{mag}.\\
 we.\ac{acc} visit.3\ac{sg}\scsub{[nom]} \tsc{rel}.\ac{acc}.\tsc{an} Maria.\ac{nom} like.3\ac{sg}\scsub{[acc]}\\
 `Who visits us, Maria likes.' \flushfill{adapted from \pgcitealt{vogel2001}{343}}\label{ex:mg-int-rep}

The sentence in \ref{ex:mg-dat-acc-wen} is ungrammatical, and the one in \ref{ex:mg-int} is not. The internal case cannot be the source of ungrammaticality, because the relative clauses are identical regarding case, i.e. they both take accusative. The external case differs, however. In Chapter X I showed that headless relatives in Modern German are (just like e.g. Gothic) sensitive to the case scale: \tsc{nom} < \tsc{acc} < \tsc{dat}.

\ref{ex:mg-dat-acc-wen} is grammatical, because the internal accusative case wins over the external nominative. \ref{ex:mg-int} is ungrammatical, because the internal accusative case cannot win the case competition over the external dative. It can be concluded that the relative pronoun in Modern German headless relatives cares about both the internal and the external case.

In sum, even though the relative pronoun in Modern German headless relatives is always part of the relative clause, the relative pronoun also takes the external case into account. That means that the relative pronoun needs to have access to the main clause case. I propose that this can be achieved by introducing an external head to the relative clause. In Section X I show how this solves the issue.

I introduce two elements: the relative pronoun, which is always the internal element, and the external element, which contains a subset of the features of the internal element



\subsection{The relative pronoun = internal}

The sentences in \ref{ex:mg-extrapose-cp} show that it is possible to extrapose a CP. In \ref{ex:mg-extrapose-cp-base}, the clausal object \tit{wie es dir geht} `how you are doing', marked here in bold, appears in its base position. It can be extraposed to the right edge of the clause, shown in \ref{ex:mg-extrapose-cp-moved}.

\ex.\label{ex:mg-extrapose-cp}
\ag. Mir ist \tbf{wie} \tbf{es} \tbf{dir} \tbf{geht} egal.\\
 1\tsc{sg}.\tsc{dat} is how it 2\tsc{sg}.\tsc{dat} goes {the same}\\
 `I don't care how you are doing.'\label{ex:mg-extrapose-cp-base}
\bg. Mir is egal \tbf{wie} \tbf{es} \tbf{dir} \tbf{geht}.\\
 1\tsc{sg}.\tsc{dat} is {the same} how it 2\tsc{sg}.\tsc{dat} goes\\
 `I don't care how you are doing.' \flushfill{Modern German}\label{ex:mg-extrapose-cp-moved}

\ref{ex:mg-extrapose-dp} illustrates that it is impossible to extrapose a DP. The clausal object of \ref{ex:mg-extrapose-cp} is replaced by the simplex noun phrase \tit{die Sache} `that matter'.
In \ref{ex:mg-extrapose-dp-base} the object, marked in bold, appears in its base position. In \ref{ex:mg-extrapose-dp-moved} it is extraposed, and the sentence is no longer grammatical.

\ex.\label{ex:mg-extrapose-dp}
\ag. Mir ist \tbf{die} \tbf{Sache} egal.\\
 1\tsc{sg}.\tsc{dat} is that matter {the same}\\
 `I don't care about that matter.'\label{ex:mg-extrapose-dp-base}
\bg. *Mir ist egal \tbf{die} \tbf{Sache}.\\
 1\tsc{sg}.\tsc{dat} is {the same} that matter\\
 `I don't care about that matter.' \flushfill{Modern German}\label{ex:mg-extrapose-dp-moved}

The same asymmetry between CPs and DPs can be observed with relative clauses. A relative clause is a CP, and the head of a relative clause is a DP. The sentences in \ref{ex:extra-headed} contain the relative clause \tit{was er gekocht hat} `what he has stolen'. This is marked in bold in the examples. The (light) head of the relative clause is \tit{das}.
In \ref{ex:extra-headed-base}, the relative clause and its head appear in base position. In \ref{ex:extra-headed-only-clause}, the relative clause is extraposed. This is grammatical, because it is possible to extrapose CPs in Modern German. In \ref{ex:extra-headed-head-clause}, the relative clause and the head are extraposed. This is ungrammatical, because it is possible to extrapose DPs.

\ex.\label{ex:extra-headed}
\ag. Jan hat das, \tbf{was} \tbf{er} \tbf{gekocht} \tbf{hat}, aufgegessen.\\
 Jan has that what he cooked has eaten\\
`Jan has eaten what he cooked.'\label{ex:extra-headed-base}
\bg. Jan hat das aufgegessen, \tbf{was} \tbf{er} \tbf{gekocht} \tbf{hat}.\\
 Jan has that eaten what he cooked has\\
`Jan has eaten what he cooked.'\label{ex:extra-headed-only-clause}
\cg. *Jan hat aufgegessen, das, \tbf{was} \tbf{er} \tbf{gekocht} \tbf{hat}.\\
 Jan has eaten that what he cooked has\\
`Jan has eaten what he cooked.'\label{ex:extra-headed-head-clause}

The same can be observed in relative clauses without a head. \ref{ex:extra-headless} is the same sentence as in \ref{ex:extra-headed} only without the overt head. The relative clause is marked in bold again.
In \ref{ex:extra-headless-base}, the relative clause appears in base position. In \ref{ex:extra-headless-clause}, the relative clause is extraposed. This is grammatical, because it is possible to extrapose CPs in Modern German. In \ref{ex:extra-headless-no-rel}, the relative clause is extraposed without the relative pronouns. This is ungrammatical, because the relative pronoun is part of the CP.
This shows that the relative pronoun in headless relatives in Modern German are necessarily part of a CP, which is here a relative clause.

\ex.\label{ex:extra-headless}
\ag. Jan hat \tbf{was} \tbf{er} \tbf{gekocht} \tbf{hat} aufgegessen.\\
Jan has what he cooked has eaten\\
`Jan has eaten what he cooked.'\label{ex:extra-headless-base}
\bg. Jan hat aufgegessen \tbf{was} \tbf{er} \tbf{gekocht} \tbf{hat}.\\
Jan has eaten what he cooked has\\
`Jan has eaten what he cooked.'\label{ex:extra-headless-clause}
\bg. *Jan hat \tbf{was} aufgegessen \tbf{er} \tbf{gekocht} \tbf{hat}.\\
Jan has what eaten he cooked has\\
`Jan has eaten what he cooked.'\label{ex:extra-headless-no-rel}

In conclusion, extraposition facts show, that the relative pronoun in Modern German is syntactically part of the relative clause.


\subsection{The other element = external}

Like I said, we need an element here. But where and how? Well, there are actually languages that show we have it!
The existence of this element is independently motivated by languages that overtly show it.
I show that this element contains a subset of the features that the relative pronoun contains.
I placee the external head in a syntactic position from which it is c-commanded by the relative pronoun and it can receive case from the main clause predicate.


There is independent evidence for this head, namely from languages that actually let the head surface. Here there are two identical copies of the head, one inside the relative clause, one outside of the relative clause.

\exg. [\tbf{doü} adiyano-no] \tbf{doü} deyalukhe\\
 sago give.3\tsc{pl}.\tsc{nonfut}-\tsc{conn} sago finished.\tsc{ajd}\\
 `The sago that they gave is finished.' \flushfill{Kombai, Dryer 2005}

I give an example of a language in which the external head follows the relative clause. There are also languages in which the head precedes the relative clause, e.g. xx

The external head is not always an exact copy of the head inside of the relative clause. An example from xx here shows that the head outside of the relative clause can also be a subset of what the element inside of the relative clause is. In this case, there is an \tit{old man} and a \tit{person}.

\exg. [\tbf{yare} gamo khereja bogi-n-o] \tbf{rumu} na-momof-a\\
 {old man} join.\tsc{ss} work \tsc{dur}.do.\tsc{3sg}.\tsc{nf}-\tsc{tr}-\tsc{conn} person my-uncle-\tsc{pred}\\
 `The old man who is joining the work is my uncle.'

So, we have the head. Translating this to relative pronouns, there is the relative pronoun, and something identical or smaller than a relative pronoun outside of the relative clause. In Chapter X I show what the feature content of the head exactly is.

Let me now show how this solves the external case problems and how it helps exclude some languages.

indefinite noun, as cinque and the content of the external head visible in some languages

Where is this head in the syntactic structure?

  \begin{itemize}
    \item Somewhere where the relative pronoun can delete it: where it is c-commanded by the relative pronoun
    \item Somewhere where it can receive case from the main clause
    \item Where it normally is in SOV languages (does the thing in Polish move because it is a svo language?)
  \end{itemize}

x


So this works.

\ex.
\begin{forest} boom
[, s sep=20mm
    [CP, s sep=20mm
        [\ac{acc}P,
        tikz={
        \node[label=below:\tit{wen},
        draw,circle,
        scale=0.85,
        fit to=tree]{};
        }
            [\tsc{f2}]
            [\tsc{nomP},
            tikz={
            \node[draw,circle,transparent,
            fill=DG,fill opacity=0.2,
            scale=0.8,
            fit to=tree]{};
            }
                [\tsc{f1}]
                [XP
                    [\phantom{xxx}, roof]
                ]
            ]
        ]
        [VP
            [\tit{Maria mag}, roof]
        ]
    ]
    [\textcolor{LG}{\tsc{nomP}},
    tikz={
    \node[draw,circle,
    scale=0.8,
    fit to=tree]{};
    }
        [\textcolor{LG}{\ac{nom}},edge=LG]
        [\textcolor{LG}{XP},baseline,edge=LG
            [\textcolor{LG}{\phantom{xxx}},
            roof, baseline, edge=LG
            ]
        ]
    ]
]
\end{forest}

But here it does not.

\ex.
\begin{forest} boom
[, s sep=20mm
    [CP, s sep=20mm
        [\ac{nom}P,
        tikz={
        \node[label=below:\tit{wer},
        draw,circle,
        fill=DG,fill opacity=0.2,
        scale=0.85,
        fit to=tree]{};
        }
            [\tsc{f1}]
            [XP
                [\phantom{xxx}, roof]
            ]
        ]
        [VP
            [\tit{mir sympatisch ist}, roof]
        ]
    ]
    [\ac{acc}P
        [\tsc{f2}]
        [\textcolor{LG}{\tsc{nomP}},
        tikz={
        \node[draw,circle,
        scale=0.8,
        fit to=tree]{};
        }
            [\textcolor{LG}{\ac{nom}},edge=LG]
            [\textcolor{LG}{XP},baseline,edge=LG
                [\textcolor{LG}{\phantom{xxx}},
                roof, baseline, edge=LG
                ]
            ]
        ]
    ]
]
\end{forest}

x

\section{Deriving the patterns}

\subsection{Deriving internal-and-external}

Old High German

featural content of relative pronoun

\begin{table}[H]\label{tbl:paradigmohg}
 \center
 \caption {Relative pronouns in headless relatives in Old High German}
  \begin{tabular}{cccc}
  \toprule
       & \ac{n}.\ac{sg} & \ac{m}.\ac{sg}  & \ac{f}.\ac{sg} \\
        \cmidrule{2-4}
  \ac{nom} & d-aȥ           & d-ër          & d-iu      \\
  \ac{acc} & d-aȥ        & d-ën      & d-ea/-ia/(-ie) \\
  \ac{dat} & d-ëmu/-ëmo     & d-ëmu/-ëmo   & d-ëru/-ëro   \\
  \bottomrule
         & \ac{n}.\ac{pl} & \ac{m}.\ac{pl}   & \ac{f}.\ac{pl} \\
          \cmidrule{2-4}
    \ac{nom}  & d-iu/-ei      &  d-ē/-ea/-ia/-ie & d-eo/-io        \\
    \ac{acc}  & d-iu/-ei      &  d-ē/-ea/-ia/-ie & d-eo/-io        \\
    \ac{dat}  & d-ēm/-ēn      &  d-ēm/-ēn        & d-ēm/-ēn        \\
    \bottomrule
  \end{tabular}
\end{table}

featural content of external head

\ex. \tbf{Spellout Algorithm:}\\
Merge F and \label{ex:spellout}
 \a. Spell out FP.
 \b. If (a) fails, attempt movement of the spec of the complement of \tsc{f}, and retry (a).
 \b. If (b) fails, move the complement of \tsc{f}, and retry (a).

When a new match is found, it overrides previous spellouts.

\ex. \tbf{Cyclic Override} \citep{starke2018}:\\
Lexicalisation at a node XP overrides any previous match at a phrase contained in XP.

If the spellout procedure in \ref{ex:spellout} fails, backtracking takes place.

\ex. \tbf{Backtracking} \citep{starke2018}:\\
When spellout fails, go back to the previous cycle, and try the next option for that cycle.\label{ex:backtracking}

If backtracking also does not help, a specifier is constructed.

\ex. \tbf{Spec Formation} \citep{starke2018}:\\
If Merge F has failed to spell out (even after backtracking), try to spawn a new derivation providing the feature F and merge that with the current derivation, projecting the feature F at the top node.\label{ex:specformation}

\ex. Merge F, Move XP, Merge XP

show how internal-wins works
show how external-wins works

\subsection{Deriving internal-only}

Modern German

featural content of relative pronoun

\begin{table}[H]
 \center
 \caption {Relative pronouns in headless relatives in Modern German}
  \begin{tabular}{ccc}
  \toprule
       & \ac{inan} & \ac{an} \\
        \cmidrule{2-3}
    \ac{nom}  & w-as     & w-er    \\
    \ac{acc}  & w-as     & w-en   \\
    \ac{dat}  & -      & w-em    \\
  \bottomrule
  \end{tabular}
\end{table}


featural content of external head


So German relative pronoun:

\begin{forest} boom
[\tsc{relP}, s sep=20mm
    [\tit{w}, roof]
    [, s sep=30mm
        [\tsc{deix}P,
        tikz={
        \node[label=below:\tit{e},
        draw,circle,
        scale=0.875,
        fit to=tree]{};
        }
            [\tsc{deix}]
            [\tsc{refP}
                [\tsc{ref2}]
                [\tsc{ref1}]
            ]
        ]
        [\tsc{acc}P,
        tikz={
        \node[label=below:\tit{n},
        draw,circle,
        scale=0.925,
        fit to=tree]{};
        }
            [\tsc{f2}]
            [\tsc{nom}P
                [\tsc{f1}]
                [\tsc{num}P
                    [\tsc{num}]
                    [\tsc{m}P
                        [\tsc{m}]
                        [\tsc{n}P
                            [\tsc{n}]
                            [\tsc{persP}
                                [\tsc{pers}]
                            ]
                        ]
                    ]
                ]
            ]
        ]
    ]
]
\end{forest}

and German head:

\begin{forest} boom
[, s sep=30mm
    [\tsc{deix}P,
    tikz={
    \node[label=below:\tit{e},
    draw,circle,
    scale=0.875,
    fit to=tree]{};
    }
        [\tsc{deix}]
        [\tsc{refP}
            [\tsc{ref2}]
            [\tsc{ref1}]
        ]
    ]
    [\tsc{nom}P,
    tikz={
    \node[label=below:\tit{r},
    draw,circle,
    scale=0.9,
    fit to=tree]{};
    }
        [\tsc{f1}]
        [\tsc{num}P
            [\tsc{num}]
            [\tsc{m}P
                [\tsc{m}]
                [\tsc{n}P
                    [\tsc{n}]
                    [\tsc{persP}
                        [\tsc{pers}]
                    ]
                ]
            ]
        ]
    ]
]
\end{forest}



show how internal-wins works
show how external-wins does not work

Florian with his am Main



\exg. Uns besucht \tbf{wen} \tbf{Maria} \tbf{mag}.\\
 we.\ac{acc} visit.3\ac{sg}\scsub{[nom]} \tsc{rel}.\ac{acc}.\tsc{an} Maria.\ac{nom} like.3\ac{sg}\scsub{[acc]}\\
 `Who visits us, Maria likes.' \flushfill{adapted from \pgcitealt{vogel2001}{343}}

Internal structure of the relative clause.

\tit{w} got merged as a complex spec. \tsc{f1} and \tsc{f2} ended up there via backtracking: taking \tit{w} off, spec to spec movement, and spelling it out with the suffix.

\ex.
\begin{forest} boom
[, s sep=50mm
    [\tsc{relP}, s sep=20mm
        [\tit{w}, roof]
        [, s sep=30mm
            [\tsc{deix}P,
            tikz={
            \node[label=below:\tit{e},
            draw,circle,
            scale=0.875,
            fit to=tree]{};
            }
                [\tsc{deix}]
                [\tsc{refP}
                    [\tsc{ref2}]
                    [\tsc{ref1}]
                ]
            ]
            [\tsc{acc}P,
            tikz={
            \node[label=below:\tit{n},
            draw,circle,
            scale=0.925,
            fit to=tree]{};
            }
                [\tsc{f2}]
                [\tsc{nom}P
                    [\tsc{f1}]
                    [\tsc{num}P
                        [\tsc{num}]
                        [\tsc{m}P
                            [\tsc{m}]
                            [\tsc{n}P
                                [\tsc{n}]
                                [\tsc{persP}
                                    [\tsc{pers}]
                                ]
                            ]
                        ]
                    ]
                ]
            ]
        ]
    ]
    [VP
       [\tit{Maria mag}, roof]
    ]
]
\end{forest}

Structure of the relative clause + the external head that is going to be deleted.

Case is merged above the relative clause. Backtracking takes place, meaning that the relative clause and the head are going to be split up again. Then it can be spelled out with the suffix of the head after spec-to-spec movement.

\ex.
\begin{forest} boom
[\tsc{nom}P
    [\tsc{f1}]
        [, s sep=15mm
        [CP
            [\tsc{relP}
                [\tit{w}, roof]
                [
                    [\tit{e}, roof]
                    [\tit{n}, roof]
                ]
            ]
            [VP
               [\tit{Maria mag}, roof]
            ]
        ]
        [, s sep=30mm
            [\tsc{deix}P,
        	  tikz={
        	  \node[label=below:\tit{e},
        	  draw,circle,
        	  scale=0.875,
        	  fit to=tree]{};
            }
                [\tsc{deix}]
                [\tsc{refP}
                    [\tsc{ref2}]
                    [\tsc{ref1}]
                ]
            ]
            [\tsc{num}P,
        	  tikz={
        	  \node[label=below:\tit{r},
        	  draw,circle,
        	  scale=0.9,
        	  fit to=tree]{};
            }
                [\tsc{num}]
                [\tsc{m}P
                    [\tsc{m}]
                    [\tsc{n}P
                        [\tsc{n}]
                        [\tsc{persP}
                            [\tsc{pers}]
                        ]
                    ]
                ]
            ]
        ]
    ]
]
\end{forest}

\phantom{x}




\subsection{Deriving neither}

Polish

featural content of relative pronoun
featural content of external head


Polish relative pronoun

\begin{forest} boom
    [, s sep=40mm
        [XP,
        tikz={
        \node[label=below:\tit{k},
        draw,circle,
        scale=0.875,
        fit to=tree]{};
        }
            [XP]
            [\tsc{persP}
                [\tsc{pers}]
                [\tsc{deix}P
                    [\tsc{deix}]
                    [\tsc{refP}
                        [\tsc{ref2}]
                        [\tsc{ref1}]
                    ]
                ]
            ]
        ]
        [, s sep=30mm
            [\tsc{m}P,
            tikz={
            \node[label=below:\tit{o},
            draw,circle,
            scale=0.925,
            fit to=tree]{};
            }
                [\tsc{m}]
                [\tsc{n}P
                    [\tsc{n}]
                ]
            ]
            [\tsc{dat}P,
            tikz={
            \node[label=below:\tit{mu},
            draw,circle,
            scale=0.925,
            fit to=tree]{};
            }
                [\tsc{f3}]
                    [\tsc{acc}P
                    [\tsc{f2}]
                    [\tsc{nom}P
                        [\tsc{f1}]
                        [\tsc{num}P
                            [\tsc{num}]
                        ]
                    ]
                ]
            ]
        ]
    ]
\end{forest}

Polish head

\begin{forest} boom
    [, s sep=30mm
        [\tsc{pers}P,
        tikz={
        \node[label=below:\tit{k},
        draw,circle,
        scale=0.875,
        fit to=tree]{};
        }
            [\tsc{pers}]
            [\tsc{deix}P
                [\tsc{deix}]
                [\tsc{refP}
                    [\tsc{ref2}]
                    [\tsc{ref1}]
                ]
            ]
        ]
        [, s sep=20mm
            [\tsc{m}P,
            tikz={
            \node[label=below:\tit{o},
            draw,circle,
            scale=0.925,
            fit to=tree]{};
            }
                [\tsc{m}]
                [\tsc{n}P
                    [\tsc{n}]
                ]
            ]
            [\tsc{acc}P,
            tikz={
            \node[label=below:\tit{go},
            draw,circle,
            scale=0.925,
            fit to=tree]{};
            }
                [\tsc{f2}]
                [\tsc{nom}P
                    [\tsc{f1}]
                    [\tsc{num}P
                        [\tsc{num}]
                    ]
                ]
            ]
        ]
    ]
\end{forest}

show how internal-wins does not work
show how external-wins does not work

Radek with his definitnessless of Czech demonstratives


\subsection{Excluding external-only}


x







\section{Alternative analyses}

\subsection{Himmelreich}



\subsection{Grafting story}

For this pattern a single element analysis seems intuitive, if you assume that case is complex and that syntax works bottom-up. First you built the relative clause, with the big case in there. Then you build the main clause and you let the more complex case in the embedded clause license the main clause predicate.

Consider the example in \ref{ex:mg-nom-acc-grafting}. Here the internal case is accusative and the external one nominative.

\exg. Uns besucht \tbf{wen} \tbf{Maria} \tbf{mag}.\\
 we.\ac{acc} visit.3\ac{sg}\scsub{[nom]} \tsc{rel}.\ac{acc}.\tsc{an} Maria.\ac{nom} like.3\ac{sg}\scsub{[acc]}\\
 `Who visits us, Maria likes.' \flushfill{adapted from \pgcitealt{vogel2001}{343}}\label{ex:mg-nom-acc-grafting}

The relative clause is built, including the accusative relative pronoun. Now the main clause predicate can merge with the nominative that is contained within the accusative.

 \ex.
 \begin{forest} boom
	 [,name=src, s sep=15mm
			[VP
			 		[\tit{besucht}, roof]
			]
		 	[,no edge, s sep=20mm
	       [\ac{acc}P,
				 tikz={
				 \node[label=below:\tit{wen},
				 draw,circle,
				 scale=0.85,
				 fit to=tree]{};
				 }
	           [\tsc{f2}]
	           [\tsc{nomP},name=tgt
	               [\tsc{f1}]
	               [XP
	                   [\phantom{xxx}, roof]
	               ]
	           ]
	       ]
				 [VP
				 		 [\tit{Maria mag}, roof]
				 ]
			]
	 ]
	 \draw (src) to[out=south east,in=north east] (tgt);
 \end{forest}\label{ex:acc-nom-grafting}

The other way around does not work. Consider \ref{ex:mg-acc-nom-grafting}. This is an example with nominative as internal case and accusative as external case.

\exg. *Ich {lade ein}, wen \tbf{mir} \tbf{sympathisch} \tbf{ist}.\\
I.\ac{nom} invite.1\ac{sg}\scsub{[acc]} \tsc{rel}.\ac{acc}.\tsc{an} I.\ac{dat} nice be.3\ac{sg}\scsub{[nom]}\\
`I invite who I like.' \flushfill{adapted from \pgcitealt{vogel2001}{344}}\label{ex:mg-acc-nom-grafting}

Now the relative clause is built first again, this time only including the nominative case. There is no accusative node to merge with for the external predicate. Instead, the relative pronoun would need to grow to accusative somehow and then the merge could take place. This is the desired result, because the sentence is ungrammatical.

\ex.
\begin{forest} boom
  [,name=src, s sep=15mm
     [VP
         [\tit{lade ein}, roof]
     ]
         [,no edge
    			[\tsc{nomP},
    			tikz={
    			\node[label=below:\tit{wer},
    			draw,circle,
    			scale=0.85,
    			fit to=tree]{};
    			}
    					[\tsc{f1}]
    					[XP
    							[\phantom{xxx}, roof]
    					]
    			]
    			[VP
    					[\tit{mir sympatisch ist}, roof]
    			]
    	 ]
    ]
\end{forest}\label{ex:nom-acc-grafting}

So, this seems to work fine. The assumptions you have to do in order to make this are the following. First, case is complex. Second, you can remerge an embedded node (grafting). For the first one I have argued in Chapter \ref{ch:decomposition}. The second one could use some additional argumentation. It is a mix between internal remerge (move) and external merge, namely external remerge. Other literature on multidominance and grafting, other phenomena. Problems: linearization, .. But even if fix all these theoretical problems, there is an empirical one.

That is, I want to connect this behavior of Modern German headless relatives to the shape of its relative pronouns. These pronouns are \tsc{wh}-elements. The OHG and Gothic ones are not \tsc{wh}, they are \tsc{d}. Their relative pronouns look different, and so their headless relatives can also behave differently.

\section{Summary}

here
